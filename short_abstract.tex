\documentclass{easychair}
\bibliographystyle{plain}

\title{A Strongly Normalizing Computation Rule for the Univalence Axiom in Higher-Order Propositional Logic}
\author{Robin Adams\inst{1} \and Marc Bezem\inst{1} \and Thierry Coquard\inst{2}}
\institute{Universitetet i Bergen,
Bergen, Norway \\
\email{\{robin.adams,marc\}@uib.no}
\and
University of Gothenburg,
Gothenburg, Sweden \\
\email{coquand@chalmers.se}}
\titlerunning{Strongly Normalizang Computation Rule for Univalence}
\authorrunning{R. Adams, M. Bezem, T. Coquand}

\usepackage{proof}
\usepackage{amssymb}
\usepackage{amsthm}

\newcommand{\vald}{\ \mathrm{valid}}
\newcommand{\univ}[4]{\mathsf{univ}_{{#1},{#2}} \left( {#3} , {#4} \right)}
\newcommand{\triplelambda}{\lambda \!\! \lambda \!\! \lambda}
\newcommand{\reff}[1]{\mathsf{ref} \left( {#1} \right)}
\newcommand{\eqdef}{\stackrel{\mathrm{def}}{=}}
\newcommand{\SN}{\mathbf{SN}}
\newcommand{\WN}{\mathbf{WN}}

\newtheorem{lemma}{Lemma}
\newtheorem{theorem}[lemma]{Theorem}
\newtheorem{proposition}[lemma]{Proposition}

\begin{document}

\maketitle

Homotopy type theory offers the promise of a formal system for the univalent foundations of mathematics.  However, if
we simply add the univalence axiom to type theory, then we lose the property of canonicity --- that every term computes to
a normal form.  A computation becomes `stuck' when it reaches the point that it needs to evaluate a proof term
that is an application of the univalence axiom.  We wish to find a way to compute with the univalence axiom.

As a first step towards such a system, we present here a system of higher-order propositional logic,  with a universe $\Omega$ of propositions
closed under implication and quantification over any simple type over $\Omega$.  We add a type $a =_A b$ for any terms $a$, $b$ of type $A$
(this type is not a proposition in $\Omega$), and two ways to prove an equality: reflexivity, and the univalence axiom.  We present
reduction relations for this system, and prove the reduction confluent and strongly normalizing.

\paragraph{Predicative higher-order propositional logic with equality.}

We call the following type theory predicative higher-order propositional logic.  It contains a universe $\Omega$ of propositions that contains $\bot$ and
is closed under $\rightarrow$.  The system also includes the higher-order types that can be built from $\Omega$ by $\rightarrow$.  Its rules of deduction are

\begin{gather*}
\infer{\langle \rangle \vald}{} \qquad
\infer{\Gamma, x : A \vald}{\Gamma \vald} \qquad 
\infer{\Gamma, p : \phi \vald}{\Gamma \vdash \phi : \Omega} \qquad
\infer[(x : A \in \Gamma)]{\Gamma \vdash x : A}{\Gamma \vald} \qquad
\infer[(p : \phi \in \Gamma)]{\Gamma \vdash p : \phi}{\Gamma \vald} \\
\infer{\Gamma \vdash \bot : \Omega}{\Gamma \vald} \qquad
\infer{\Gamma \vdash \phi \rightarrow \psi : \Omega}{\Gamma \vdash \phi : \Omega \quad \Gamma \vdash \psi : \Omega} \\
\infer{\Gamma \vdash M N : B} {\Gamma \vdash M : A \rightarrow B \quad \Gamma \vdash N : A} \qquad
\infer{\Gamma \vdash \delta \epsilon : \psi} {\Gamma \vdash \delta : \phi \rightarrow \psi \quad \Gamma \vdash \epsilon : \phi} \\
\infer{\Gamma \vdash \lambda x:A.M : A \rightarrow B}{\Gamma, x : A \vdash M : B} \qquad
\infer{\Gamma \vdash \lambda p : \phi . \delta : \phi \rightarrow \psi}{\Gamma, p : \phi \vdash \delta : \psi} \qquad
\infer[(\phi \simeq \phi)]{\Gamma \vdash \delta : \psi}{\Gamma \vdash \delta : \phi \quad \Gamma \vdash \psi : \Omega}
\end{gather*}

\paragraph{Extensional equality.}

On top of this system, we add an equality relation that satisfies univalence.  We add a new judgement form,
$\Gamma \vdash P : M =_A M$, to denote that $P$ is a proof of that $M$ and $N$ are equal terms of type $A$.  We also add the following constructions:
\begin{itemize}
\item
For any $M : A$, a proof $\reff{M} : M =_A M$.
\item
\textbf{Univalence.}  Given proofs $\delta : \phi \rightarrow \psi$ and $\epsilon : \psi \rightarrow \phi$, a proof
$\univ{\phi}{\psi}{\delta}{\epsilon} : \phi =_\Omega \psi$.
\item
Given a proof $P : \phi =_\Omega \psi$, proofs $P^+ : \phi \rightarrow \psi$ and $P^- : \psi \rightarrow \phi$.
\item
Given a proof $\Gamma, x : A, y : A, e : x =_A y \vdash P : Mx =_B Ny$, a proof \\ $\triplelambda e : x=_Ay . P : M =_{A \rightarrow B} N$.  (Here, $e$, $x$ and $y$ are bound within $P$.)
\item
Rules to ensure that the equality is a congruence for $\rightarrow$ and application.
\end{itemize}

\paragraph{The reduction relation.}

We define the following reduction relation on proofs and equality proofs.

\begin{gather*}
(\reff{\phi})^+ \rightsquigarrow \lambda x : \phi . x
\qquad
(\reff{\phi})^- \rightsquigarrow \lambda x : \phi . x
\\
\univ{\phi}{\psi}{\delta}{\epsilon}^+ \rightsquigarrow \delta
\qquad
\univ{\phi}{\psi}{\delta}{\epsilon}^- \rightsquigarrow \epsilon
\\ \\
(\reff \phi \rightarrow \univ{\psi}{\chi}{\delta}{\epsilon}) \rightsquigarrow \univ{\phi \rightarrow \psi}{\phi \rightarrow \chi}{\lambda f : \phi \rightarrow \psi . \lambda x : \phi . \delta (f x)}{\lambda g : \phi \rightarrow \chi . \lambda x : \phi . \epsilon (g x)}
\\
(\univ{\phi}{\psi}{\delta}{\epsilon} \rightarrow \reff{\chi}) \rightsquigarrow \univ{\phi \rightarrow \chi}{\psi \rightarrow \chi}{\lambda f : \phi \rightarrow \chi. \lambda x : \psi . f (\epsilon x)}{\lambda g : \psi \rightarrow \chi . \lambda x : \phi . g (\delta x)}
\\
(\univ{\phi}{\psi}{\delta}{\epsilon} \rightarrow \univ{\phi'}{\psi'}{\delta'}{\epsilon'} \hspace{8cm} \\
\qquad \rightsquigarrow \univ{\phi \rightarrow \phi'}{\psi \rightarrow \psi'}
{\lambda f : \phi \rightarrow \phi' . \lambda x : \psi . \delta' (f (\epsilon x))}{\lambda g : \psi \rightarrow \psi' . \lambda y : \phi . \epsilon' (g (\delta y))}
\\ \\
(\reff{\phi} \rightarrow \reff{\psi}) \rightsquigarrow \reff{\phi \rightarrow \psi}
\qquad
\reff{M} \reff{N} \rightsquigarrow \reff{MN}
\\
(\reff{\lambda x:A.M})P \rightsquigarrow \{ P / x \} M \qquad (P \text{ a normal form not of the form } \reff{\_})
\\
(\triplelambda e : x =_A y.P)Q \rightsquigarrow [M/x, N/y, Q/e]P \qquad (Q : M =_A N)
\end{gather*}

Here, $\{ P / x \}M$ is an operation called \emph{path substitution} defined such that, if $P : N =_A N'$ then $\{ P / x \} M : [N/x]M = [N'/x]M$.

We can prove the following result about the canonical forms in this system:

\begin{proposition}
Every closed normal form of type $\phi =_\Omega \psi$ either has the form $\reff{\_}$ or $\mathsf{univ}(\_,\_)$.  Every closed normal form of the type $M =_{A \rightarrow B} N$ either has the form $\reff{\_}$ or is a $\triplelambda$-term.
\end{proposition}

Thus, once we have proved strong normalization, we know that a well-typed computation never gets `stuck' at an application of the univalence axiom.

\paragraph{Proof of strong normalization.}

The proof of strong normalization follows the method of Tait \cite{Tait1967}
We define the set of \emph{computable} terms $E_\Gamma(A)$ for each type $A$,
and computable proofs $E_\Gamma(M =_A N)$ for any terms $\Gamma \vdash M,N : A$

Tait's proof relies on confluence, which does not hold for this reduction relation in general.  But we do have that the following, which turns out to be sufficient.

\begin{proposition}
Reduction is locally confluent.  All computable terms are strongly normalizing and confluent.  The computability predicates are closed under reduction and well-tiped expansion.
\end{proposition}

\begin{theorem}
If $\Gamma \vdash M : A$ then $M \in E_\Gamma(A)$.  If $\Gamma \vdash P : M =_A N$ then $P \in E_\Gamma(M =_A N)$.
\end{theorem}
It follows that this system is strongly normalizing.

In the proof, we prove confluence `on-the-fly'.  That is, whenever we require a term to be confluent, the induction hypothesis provides us with the fact that that term is computable, and hence strongly normalizing and confluent.

% In the future, we wish to extend this work in several directions.  We wish to add constructions $\mathsf{sym}(P)$ and $\mathsf{trans}(P,Q)$ that make our equality into an equivalence
% relation, place the equations $M =_A N$ inside $
% and provide reduction rules for each of these.  We wish to extend our logic by placing the equations $M =_A N$ inside $\Omega$, and allow quantification over any type in $A$ (including $\Omega$) in our logic.  This would move our system ever closer to homotopy type theory.  Our ultimate goal is to provide a strongly normalizing reduction relation for homotopy type theory, including the terms involving the univalence axiom, and we believe we have taken the first step.

\bibliography{type}

\end{document}