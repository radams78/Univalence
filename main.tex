\title{A Strongly Normalizing Computation Rule for Univalence in Higher-Order Propositional Logic}
\author{\alert{Robin Adams} \and Marc Bezem \and Thierry Coquand}
\date{May 26 2016}

\usepackage{amsmath}
\usepackage{amssymb}
\usepackage{bbm}
\usepackage[greek,english]{babel}
\usepackage{ucs}
\usepackage[utf8x]{inputenc}
\usepackage{autofe}
\usepackage{fancyvrb}
\usepackage{proof}
\usepackage{stmaryrd}

\DeclareUnicodeCharacter{8608}{\ensuremath{\twoheadrightarrow}}
\DeclareUnicodeCharacter{8667}{\ensuremath{\Rrightarrow}}
\DeclareUnicodeCharacter{8718}{\ensuremath{\qed}}
\DeclareUnicodeCharacter{8759}{\ensuremath{::}}
\DeclareUnicodeCharacter{8988}{\ensuremath{\ulcorner}}
\DeclareUnicodeCharacter{8989}{\ensuremath{\urcorner}}
\DeclareUnicodeCharacter{8803}{\ensuremath{\overline{\equiv}}}
\DeclareUnicodeCharacter{9001}{\ensuremath{\langle}}
\DeclareUnicodeCharacter{9002}{\ensuremath{\rangle}}
\DeclareUnicodeCharacter{9655}{\ensuremath{\rhd}}
\DeclareUnicodeCharacter{10214}{\ensuremath{[}}
\DeclareUnicodeCharacter{10215}{\ensuremath{]}}
\DeclareUnicodeCharacter{10219}{\ensuremath{\rangle\rangle}}


\usepackage{textalpha}

\DefineVerbatimEnvironment{code}{Verbatim}{fontsize=\small}

% \newtheorem{lemma}{Lemma}[section]
% \newtheorem{corollary}[lemma]{Corollary}
\newtheorem{prop}{Proposition}[section]
% \newtheorem{theorem}{Theorem}[section]
% \theoremstyle{definition}
% \newtheorem{definition}[lemma]{Definition}

\newcommand{\Set}{\mathbf{Set}}
\newcommand{\eqdef}{\mathrel{\smash{\stackrel{\text{def}}{=}}}}
\newcommand{\AgdaHide}[1]{}
\newcommand{\isotoid}{\ensuremath{isotoid}}
\newcommand{\vald}{\ensuremath{\ \mathrm{valid}}}
\newcommand{\reff}[1]{\ensuremath{\mathsf{ref} \left( {#1} \right)}}
\newcommand{\univ}[4]{\ensuremath{\mathsf{univ}_{{#1} , {#2}} \left( {#3} , {#4} \right)}}
\newcommand{\triplelambda}{\lambda \!\! \lambda \!\! \lambda}

\begin{document}

\begin{frame}
\maketitle

\mode<beamer>{
\begin{small}
\begin{center}
This talk is a literate Agda file
\texttt{https://github.com/radams78/Univalence}
\end{center}
\end{small}
}

%TODO Full author and conference information

\end{frame}

\section{Introduction}

\begin{frame}
\frametitle{The problem}
\begin{itemize}
\item
A type theory should enjoy three properties:
\begin{description}
\item[Canonicity] Every well-typed term of type $A$ reduces to a canonical form of $A$.
\item[Confluence] Reduction is confluent.  (Therefore, the canonical form is unique.)
\item[Strong Normalization] Every reduction strategy terminates.
\end{description}
\item 
The \emph{univalence axiom} postulates a function
\[ \isotoid : A \simeq B \rightarrow A = B \]
that is an inverse to the obvious function $A = B \rightarrow A \simeq B$.
\item
This breaks canonicity.
\end{itemize}
\end{frame}
%TODO Define canonical form

\begin{frame}
\frametitle{Possible Solutions}
\begin{itemize}
\item
Give up.
\item
Redefine equality. %TODO Cite Polonsky
\item
Use a type theory in which $\isotoid$ is definable (e.g. Cubical Type Theory) %TODO Cite
\item
Introduce a reduction rule for $\isotoid$.
\end{itemize}
\end{frame}
%TODO Look up other possible solutions.

\begin{frame}
\frametitle{Our Approach}
We begin with a small type theory, and work our way up to the full HoTT.
\begin{enumerate}
\item \emph{Predicative Higher-Order Propositional Logic} A type theory with:
  \begin{itemize}
  \item a universe $\Omega$ of \emph{propositions} with $\bot$ and $\supset$
  \item a universe $U$ of \emph{small types} with $\Omega$ and $\rightarrow$
  \item for any two terms $M, N : A$, a (large) type $M =_A N$.
  \end{itemize}
\item \emph{PHOPL with Equality}
Make $\delta =_\phi \epsilon$ a proposition.  (So we can form propositions like $M =_A N \epsilon \supset N =_A M$.)
\item \emph{Predicative Higher-Order Predicate Logic} Close $\Omega$ under $\forall$.  (So we can form propositions like $\forall x : \phi. x =_A x$.)
\end{enumerate}
For the future: natural numbers, inductive types, path elimination, \ldots
\end{frame}

\AgdaHide{
\begin{code}%
\>\AgdaKeyword{module} \AgdaModule{Prelims} \AgdaKeyword{where}\<%
\end{code}
}

\begin{code}%
\>\AgdaKeyword{open} \AgdaKeyword{import} \AgdaModule{Relation.Binary} \AgdaKeyword{public}\<%
\\
%
\\
\>\AgdaKeyword{module} \AgdaModule{Prelims.Bifunction}\<%
\\
\>[0]\AgdaIndent{2}{}\<[2]%
\>[2]\AgdaSymbol{\{}\AgdaBound{r₁} \AgdaBound{r₂} \AgdaBound{s₁} \AgdaBound{s₂} \AgdaBound{t₁} \AgdaBound{t₂}\AgdaSymbol{\}} \AgdaSymbol{\{}\AgdaBound{A} \AgdaSymbol{:} \AgdaRecord{Setoid} \AgdaBound{r₁} \AgdaBound{r₂}\AgdaSymbol{\}} \AgdaSymbol{\{}\AgdaBound{B} \AgdaSymbol{:} \AgdaRecord{Setoid} \AgdaBound{s₁} \AgdaBound{s₂}\AgdaSymbol{\}} \AgdaSymbol{\{}\AgdaBound{C} \AgdaSymbol{:} \AgdaRecord{Setoid} \AgdaBound{t₁} \AgdaBound{t₂}\AgdaSymbol{\}} \<[79]%
\>[79]\<%
\\
\>[0]\AgdaIndent{2}{}\<[2]%
\>[2]\AgdaSymbol{(}\AgdaBound{f} \AgdaSymbol{:} \AgdaField{Setoid.Carrier} \AgdaBound{A} \AgdaSymbol{→} \AgdaField{Setoid.Carrier} \AgdaBound{B} \AgdaSymbol{→} \AgdaField{Setoid.Carrier} \AgdaBound{C}\AgdaSymbol{)} \AgdaKeyword{where}\<%
\\
%
\\
\>[0]\AgdaIndent{2}{}\<[2]%
\>[2]\AgdaFunction{wdl} \AgdaSymbol{:} \AgdaPrimitiveType{Set} \AgdaSymbol{\_}\<%
\\
\>[0]\AgdaIndent{2}{}\<[2]%
\>[2]\AgdaFunction{wdl} \AgdaSymbol{=} \AgdaSymbol{∀} \AgdaSymbol{\{}\AgdaBound{a} \AgdaBound{a'}\AgdaSymbol{\}} \AgdaSymbol{→} \AgdaField{Setoid.\_≈\_} \AgdaBound{A} \AgdaBound{a} \AgdaBound{a'} \AgdaSymbol{→} \AgdaSymbol{∀} \AgdaBound{b} \AgdaSymbol{→} \AgdaField{Setoid.\_≈\_} \AgdaBound{C} \AgdaSymbol{(}\AgdaBound{f} \AgdaBound{a} \AgdaBound{b}\AgdaSymbol{)} \AgdaSymbol{(}\AgdaBound{f} \AgdaBound{a'} \AgdaBound{b}\AgdaSymbol{)}\<%
\\
%
\\
\>[0]\AgdaIndent{2}{}\<[2]%
\>[2]\AgdaFunction{wdr} \AgdaSymbol{:} \AgdaPrimitiveType{Set} \AgdaSymbol{\_}\<%
\\
\>[0]\AgdaIndent{2}{}\<[2]%
\>[2]\AgdaFunction{wdr} \AgdaSymbol{=} \AgdaSymbol{∀} \AgdaBound{a} \AgdaSymbol{\{}\AgdaBound{b} \AgdaBound{b'}\AgdaSymbol{\}} \AgdaSymbol{→} \AgdaField{Setoid.\_≈\_} \AgdaBound{B} \AgdaBound{b} \AgdaBound{b'} \AgdaSymbol{→} \AgdaField{Setoid.\_≈\_} \AgdaBound{C} \AgdaSymbol{(}\AgdaBound{f} \AgdaBound{a} \AgdaBound{b}\AgdaSymbol{)} \AgdaSymbol{(}\AgdaBound{f} \AgdaBound{a} \AgdaBound{b'}\AgdaSymbol{)}\<%
\\
%
\\
\>[0]\AgdaIndent{2}{}\<[2]%
\>[2]\AgdaFunction{wd2} \AgdaSymbol{:} \AgdaPrimitiveType{Set} \AgdaSymbol{\_}\<%
\\
\>[0]\AgdaIndent{2}{}\<[2]%
\>[2]\AgdaFunction{wd2} \AgdaSymbol{=} \AgdaSymbol{∀} \AgdaSymbol{\{}\AgdaBound{a} \AgdaBound{a'} \AgdaBound{b} \AgdaBound{b'}\AgdaSymbol{\}} \AgdaSymbol{→} \AgdaField{Setoid.\_≈\_} \AgdaBound{A} \AgdaBound{a} \AgdaBound{a'} \AgdaSymbol{→} \AgdaField{Setoid.\_≈\_} \AgdaBound{B} \AgdaBound{b} \AgdaBound{b'} \AgdaSymbol{→} \AgdaField{Setoid.\_≈\_} \AgdaBound{C} \AgdaSymbol{(}\AgdaBound{f} \AgdaBound{a} \AgdaBound{b}\AgdaSymbol{)} \AgdaSymbol{(}\AgdaBound{f} \AgdaBound{a'} \AgdaBound{b'}\AgdaSymbol{)}\<%
\\
%
\\
\>[0]\AgdaIndent{2}{}\<[2]%
\>[2]\AgdaFunction{congl} \AgdaSymbol{:} \AgdaFunction{wd2} \AgdaSymbol{→} \AgdaFunction{wdl}\<%
\\
\>[0]\AgdaIndent{2}{}\<[2]%
\>[2]\AgdaFunction{congl} \AgdaBound{wd} \AgdaBound{a≈a'} \AgdaSymbol{\_} \AgdaSymbol{=} \AgdaBound{wd} \AgdaBound{a≈a'} \AgdaSymbol{(}\AgdaFunction{Setoid.refl} \AgdaBound{B}\AgdaSymbol{)}\<%
\\
%
\\
\>[0]\AgdaIndent{2}{}\<[2]%
\>[2]\AgdaFunction{congr} \AgdaSymbol{:} \AgdaFunction{wd2} \AgdaSymbol{→} \AgdaFunction{wdr}\<%
\\
\>[0]\AgdaIndent{2}{}\<[2]%
\>[2]\AgdaFunction{congr} \AgdaBound{wd} \AgdaSymbol{\_} \AgdaBound{b≈b'} \AgdaSymbol{=} \AgdaBound{wd} \AgdaSymbol{(}\AgdaFunction{Setoid.refl} \AgdaBound{A}\AgdaSymbol{)} \AgdaBound{b≈b'}\<%
\\
%
\\
\>[0]\AgdaIndent{2}{}\<[2]%
\>[2]\AgdaFunction{cong2} \AgdaSymbol{:} \AgdaFunction{wdl} \AgdaSymbol{→} \AgdaFunction{wdr} \AgdaSymbol{→} \AgdaFunction{wd2}\<%
\\
\>[0]\AgdaIndent{2}{}\<[2]%
\>[2]\AgdaFunction{cong2} \AgdaBound{wdl} \AgdaBound{wdr} \AgdaSymbol{\{\_\}} \AgdaSymbol{\{}\AgdaBound{a'}\AgdaSymbol{\}} \AgdaSymbol{\{}\AgdaBound{b}\AgdaSymbol{\}} \AgdaBound{a≈a'} \AgdaBound{b≈b'} \AgdaSymbol{=} \AgdaFunction{Setoid.trans} \AgdaBound{C} \AgdaSymbol{(}\AgdaBound{wdl} \AgdaBound{a≈a'} \AgdaBound{b}\AgdaSymbol{)} \AgdaSymbol{(}\AgdaBound{wdr} \AgdaBound{a'} \AgdaBound{b≈b'}\AgdaSymbol{)}\<%
\end{code}

\AgdaHide{
\begin{code}%
\>\AgdaKeyword{open} \AgdaKeyword{import} \AgdaModule{Prelims.Bifunction} \AgdaKeyword{public}\<%
\end{code}
}

\begin{code}%
\>\AgdaKeyword{module} \AgdaModule{Prelims.EqReasoning} \AgdaKeyword{where}\<%
\\
\>\AgdaKeyword{open} \AgdaKeyword{import} \AgdaModule{Relation.Binary} \AgdaKeyword{public} \AgdaKeyword{hiding} \AgdaSymbol{(}\AgdaFunction{\_⇒\_}\AgdaSymbol{)}\<%
\\
\>\AgdaKeyword{import} \AgdaModule{Relation.Binary.PreorderReasoning}\<%
\\
\>\AgdaKeyword{import} \AgdaModule{Relation.Binary.EqReasoning}\<%
\\
\>\AgdaKeyword{open} \AgdaKeyword{import} \AgdaModule{Relation.Binary.PropositionalEquality} \AgdaKeyword{public} \AgdaKeyword{using} \AgdaSymbol{(}\AgdaDatatype{\_≡\_}\AgdaSymbol{;}\AgdaInductiveConstructor{refl}\AgdaSymbol{;}\AgdaFunction{sym}\AgdaSymbol{;}\AgdaFunction{trans}\AgdaSymbol{;}\AgdaFunction{cong}\AgdaSymbol{;}\AgdaFunction{cong₂}\AgdaSymbol{;}\AgdaFunction{subst}\AgdaSymbol{;}\AgdaFunction{subst₂}\AgdaSymbol{;}\AgdaFunction{setoid}\AgdaSymbol{)}\<%
\\
%
\\
\>\AgdaKeyword{module} \AgdaModule{PreorderReasoning} \AgdaSymbol{\{}\AgdaBound{p₁} \AgdaBound{p₂} \AgdaBound{p₃}\AgdaSymbol{\}} \AgdaSymbol{(}\AgdaBound{P} \AgdaSymbol{:} \AgdaRecord{Preorder} \AgdaBound{p₁} \AgdaBound{p₂} \AgdaBound{p₃}\AgdaSymbol{)} \AgdaKeyword{where}\<%
\\
\>[0]\AgdaIndent{2}{}\<[2]%
\>[2]\AgdaKeyword{open} \AgdaModule{Relation.Binary.PreorderReasoning} \AgdaBound{P} \AgdaKeyword{public}\<%
\\
%
\\
\>[0]\AgdaIndent{2}{}\<[2]%
\>[2]\AgdaKeyword{infixr} \AgdaNumber{2} \AgdaFixityOp{\_≈⟨⟨\_⟩⟩\_}\<%
\\
\>[0]\AgdaIndent{2}{}\<[2]%
\>[2]\AgdaFunction{\_≈⟨⟨\_⟩⟩\_} \AgdaSymbol{:} \AgdaSymbol{∀} \AgdaBound{x} \AgdaSymbol{\{}\AgdaBound{y} \AgdaBound{z}\AgdaSymbol{\}} \AgdaSymbol{→} \AgdaField{Preorder.\_≈\_} \AgdaBound{P} \AgdaBound{y} \AgdaBound{x} \AgdaSymbol{→} \AgdaBound{y} \AgdaDatatype{IsRelatedTo} \AgdaBound{z} \AgdaSymbol{→} \AgdaBound{x} \AgdaDatatype{IsRelatedTo} \AgdaBound{z}\<%
\\
\>[0]\AgdaIndent{2}{}\<[2]%
\>[2]\AgdaBound{x} \AgdaFunction{≈⟨⟨} \AgdaBound{y≈x} \AgdaFunction{⟩⟩} \AgdaBound{y∼z} \AgdaSymbol{=} \AgdaBound{x} \AgdaFunction{≈⟨} \AgdaFunction{Preorder.Eq.sym} \AgdaBound{P} \AgdaBound{y≈x} \AgdaFunction{⟩} \AgdaBound{y∼z}\<%
\\
%
\\
\>\AgdaKeyword{module} \AgdaModule{EqReasoning} \AgdaSymbol{\{}\AgdaBound{s₁} \AgdaBound{s₂}\AgdaSymbol{\}} \AgdaSymbol{(}\AgdaBound{S} \AgdaSymbol{:} \AgdaRecord{Setoid} \AgdaBound{s₁} \AgdaBound{s₂}\AgdaSymbol{)} \AgdaKeyword{where}\<%
\\
\>[0]\AgdaIndent{2}{}\<[2]%
\>[2]\AgdaKeyword{open} \AgdaModule{Setoid} \AgdaBound{S} \AgdaKeyword{using} \AgdaSymbol{(}\_≈\_\AgdaSymbol{)}\<%
\\
\>[0]\AgdaIndent{2}{}\<[2]%
\>[2]\AgdaKeyword{open} \AgdaModule{Relation.Binary.EqReasoning} \AgdaBound{S} \AgdaKeyword{public}\<%
\\
%
\\
\>[0]\AgdaIndent{2}{}\<[2]%
\>[2]\AgdaKeyword{infixr} \AgdaNumber{2} \AgdaFixityOp{\_≡⟨⟨\_⟩⟩\_}\<%
\\
\>[0]\AgdaIndent{2}{}\<[2]%
\>[2]\AgdaFunction{\_≡⟨⟨\_⟩⟩\_} \AgdaSymbol{:} \AgdaSymbol{∀} \AgdaBound{x} \AgdaSymbol{\{}\AgdaBound{y} \AgdaBound{z}\AgdaSymbol{\}} \AgdaSymbol{→} \AgdaBound{y} \AgdaDatatype{≡} \AgdaBound{x} \AgdaSymbol{→} \AgdaBound{y} \AgdaDatatype{IsRelatedTo} \AgdaBound{z} \AgdaSymbol{→} \AgdaBound{x} \AgdaDatatype{IsRelatedTo} \AgdaBound{z}\<%
\\
\>[0]\AgdaIndent{2}{}\<[2]%
\>[2]\AgdaBound{x} \AgdaFunction{≡⟨⟨} \AgdaBound{y≡x} \AgdaFunction{⟩⟩} \AgdaBound{y≈z} \AgdaSymbol{=} \AgdaBound{x} \AgdaFunction{≡⟨} \AgdaFunction{sym} \AgdaBound{y≡x} \AgdaFunction{⟩} \AgdaBound{y≈z}\<%
\\
%
\\
\>[0]\AgdaIndent{2}{}\<[2]%
\>[2]\AgdaKeyword{infixr} \AgdaNumber{2} \AgdaFixityOp{\_≈⟨⟨\_⟩⟩\_}\<%
\\
\>[0]\AgdaIndent{2}{}\<[2]%
\>[2]\AgdaFunction{\_≈⟨⟨\_⟩⟩\_} \AgdaSymbol{:} \AgdaSymbol{∀} \AgdaBound{x} \AgdaSymbol{\{}\AgdaBound{y} \AgdaBound{z}\AgdaSymbol{\}} \AgdaSymbol{→} \AgdaBound{y} \AgdaField{≈} \AgdaBound{x} \AgdaSymbol{→} \AgdaBound{y} \AgdaDatatype{IsRelatedTo} \AgdaBound{z} \AgdaSymbol{→} \AgdaBound{x} \AgdaDatatype{IsRelatedTo} \AgdaBound{z}\<%
\\
\>[0]\AgdaIndent{2}{}\<[2]%
\>[2]\AgdaBound{x} \AgdaFunction{≈⟨⟨} \AgdaBound{y≈x} \AgdaFunction{⟩⟩} \AgdaBound{y≈z} \AgdaSymbol{=} \AgdaBound{x} \AgdaFunction{≈⟨} \AgdaFunction{Setoid.sym} \AgdaBound{S} \<[40]%
\>[40]\AgdaBound{y≈x} \AgdaFunction{⟩} \AgdaBound{y≈z}\<%
\\
%
\\
\>\AgdaKeyword{module} \AgdaModule{≡-Reasoning} \AgdaSymbol{\{}\AgdaBound{a}\AgdaSymbol{\}} \AgdaSymbol{\{}\AgdaBound{A} \AgdaSymbol{:} \AgdaPrimitiveType{Set} \AgdaBound{a}\AgdaSymbol{\}} \AgdaKeyword{where}\<%
\\
\>[0]\AgdaIndent{2}{}\<[2]%
\>[2]\AgdaKeyword{open} \AgdaModule{Relation.Binary.PropositionalEquality}\<%
\\
\>[0]\AgdaIndent{2}{}\<[2]%
\>[2]\AgdaKeyword{open} \AgdaModule{≡-Reasoning} \AgdaSymbol{\{}\AgdaBound{a}\AgdaSymbol{\}} \AgdaSymbol{\{}\AgdaBound{A}\AgdaSymbol{\}} \AgdaKeyword{public}\<%
\\
%
\\
\>[0]\AgdaIndent{2}{}\<[2]%
\>[2]\AgdaKeyword{infixr} \AgdaNumber{2} \AgdaFixityOp{\_≡⟨⟨\_⟩⟩\_}\<%
\\
\>[0]\AgdaIndent{2}{}\<[2]%
\>[2]\AgdaFunction{\_≡⟨⟨\_⟩⟩\_} \AgdaSymbol{:} \AgdaSymbol{∀} \AgdaSymbol{(}\AgdaBound{x} \AgdaSymbol{:} \AgdaBound{A}\AgdaSymbol{)} \AgdaSymbol{\{}\AgdaBound{y} \AgdaBound{z}\AgdaSymbol{\}} \AgdaSymbol{→} \AgdaBound{y} \AgdaDatatype{≡} \AgdaBound{x} \AgdaSymbol{→} \AgdaBound{y} \AgdaDatatype{≡} \AgdaBound{z} \AgdaSymbol{→} \AgdaBound{x} \AgdaDatatype{≡} \AgdaBound{z}\<%
\\
\>[0]\AgdaIndent{2}{}\<[2]%
\>[2]\AgdaSymbol{\_} \AgdaFunction{≡⟨⟨} \AgdaBound{y≡x} \AgdaFunction{⟩⟩} \AgdaBound{y≡z} \AgdaSymbol{=} \AgdaFunction{trans} \AgdaSymbol{(}\AgdaFunction{sym} \AgdaBound{y≡x}\AgdaSymbol{)} \AgdaBound{y≡z}\<%
\\
%
\\
\>\AgdaFunction{cong₃} \AgdaSymbol{:} \AgdaSymbol{∀} \AgdaSymbol{\{}\AgdaBound{A} \AgdaBound{B} \AgdaBound{C} \AgdaBound{D} \AgdaSymbol{:} \AgdaPrimitiveType{Set}\AgdaSymbol{\}} \AgdaSymbol{(}\AgdaBound{f} \AgdaSymbol{:} \AgdaBound{A} \AgdaSymbol{→} \AgdaBound{B} \AgdaSymbol{→} \AgdaBound{C} \AgdaSymbol{→} \AgdaBound{D}\AgdaSymbol{)} \AgdaSymbol{\{}\AgdaBound{a} \AgdaBound{a'} \AgdaBound{b} \AgdaBound{b'} \AgdaBound{c} \AgdaBound{c'}\AgdaSymbol{\}} \AgdaSymbol{→}\<%
\\
\>[2]\AgdaIndent{10}{}\<[10]%
\>[10]\AgdaBound{a} \AgdaDatatype{≡} \AgdaBound{a'} \AgdaSymbol{→} \AgdaBound{b} \AgdaDatatype{≡} \AgdaBound{b'} \AgdaSymbol{→} \AgdaBound{c} \AgdaDatatype{≡} \AgdaBound{c'} \AgdaSymbol{→} \AgdaBound{f} \AgdaBound{a} \AgdaBound{b} \AgdaBound{c} \AgdaDatatype{≡} \AgdaBound{f} \AgdaBound{a'} \AgdaBound{b'} \AgdaBound{c'}\<%
\\
\>\AgdaFunction{cong₃} \AgdaSymbol{\_} \AgdaInductiveConstructor{refl} \AgdaInductiveConstructor{refl} \AgdaInductiveConstructor{refl} \AgdaSymbol{=} \AgdaInductiveConstructor{refl}\<%
\\
%
\\
\>\AgdaFunction{cong₄} \AgdaSymbol{:} \AgdaSymbol{∀} \AgdaSymbol{\{}\AgdaBound{A} \AgdaBound{B} \AgdaBound{C} \AgdaBound{D} \AgdaBound{E} \AgdaSymbol{:} \AgdaPrimitiveType{Set}\AgdaSymbol{\}} \AgdaSymbol{(}\AgdaBound{f} \AgdaSymbol{:} \AgdaBound{A} \AgdaSymbol{→} \AgdaBound{B} \AgdaSymbol{→} \AgdaBound{C} \AgdaSymbol{→} \AgdaBound{D} \AgdaSymbol{→} \AgdaBound{E}\AgdaSymbol{)} \AgdaSymbol{\{}\AgdaBound{a} \AgdaBound{a'} \AgdaBound{b} \AgdaBound{b'} \AgdaBound{c} \AgdaBound{c'} \AgdaBound{d} \AgdaBound{d'}\AgdaSymbol{\}} \AgdaSymbol{→}\<%
\\
\>[2]\AgdaIndent{10}{}\<[10]%
\>[10]\AgdaBound{a} \AgdaDatatype{≡} \AgdaBound{a'} \AgdaSymbol{→} \AgdaBound{b} \AgdaDatatype{≡} \AgdaBound{b'} \AgdaSymbol{→} \AgdaBound{c} \AgdaDatatype{≡} \AgdaBound{c'} \AgdaSymbol{→} \AgdaBound{d} \AgdaDatatype{≡} \AgdaBound{d'} \AgdaSymbol{→} \AgdaBound{f} \AgdaBound{a} \AgdaBound{b} \AgdaBound{c} \AgdaBound{d} \AgdaDatatype{≡} \AgdaBound{f} \AgdaBound{a'} \AgdaBound{b'} \AgdaBound{c'} \AgdaBound{d'}\<%
\\
\>\AgdaFunction{cong₄} \AgdaSymbol{\_} \AgdaInductiveConstructor{refl} \AgdaInductiveConstructor{refl} \AgdaInductiveConstructor{refl} \AgdaInductiveConstructor{refl} \AgdaSymbol{=} \AgdaInductiveConstructor{refl}\<%
\\
%
\\
\>\AgdaFunction{subst₃} \AgdaSymbol{:} \AgdaSymbol{∀} \AgdaSymbol{\{}\AgdaBound{A} \AgdaBound{B} \AgdaBound{C} \AgdaSymbol{:} \AgdaPrimitiveType{Set}\AgdaSymbol{\}} \AgdaSymbol{(}\AgdaBound{P} \AgdaSymbol{:} \AgdaBound{A} \AgdaSymbol{→} \AgdaBound{B} \AgdaSymbol{→} \AgdaBound{C} \AgdaSymbol{→} \AgdaPrimitiveType{Set}\AgdaSymbol{)} \AgdaSymbol{\{}\AgdaBound{a} \AgdaBound{a'} \AgdaBound{b} \AgdaBound{b'} \AgdaBound{c} \AgdaBound{c'}\AgdaSymbol{\}} \AgdaSymbol{→}\<%
\\
\>[10]\AgdaIndent{11}{}\<[11]%
\>[11]\AgdaBound{a} \AgdaDatatype{≡} \AgdaBound{a'} \AgdaSymbol{→} \AgdaBound{b} \AgdaDatatype{≡} \AgdaBound{b'} \AgdaSymbol{→} \AgdaBound{c} \AgdaDatatype{≡} \AgdaBound{c'} \AgdaSymbol{→} \AgdaBound{P} \AgdaBound{a} \AgdaBound{b} \AgdaBound{c} \AgdaSymbol{→} \AgdaBound{P} \AgdaBound{a'} \AgdaBound{b'} \AgdaBound{c'}\<%
\\
\>\AgdaFunction{subst₃} \AgdaSymbol{\_} \AgdaInductiveConstructor{refl} \AgdaInductiveConstructor{refl} \AgdaInductiveConstructor{refl} \AgdaBound{Pabc} \AgdaSymbol{=} \AgdaBound{Pabc}\<%
\\
%
\\
\>\AgdaFunction{subst₄} \AgdaSymbol{:} \AgdaSymbol{∀} \AgdaSymbol{\{}\AgdaBound{A1} \AgdaBound{A2} \AgdaBound{A3} \AgdaBound{A4} \AgdaSymbol{:} \AgdaPrimitiveType{Set}\AgdaSymbol{\}} \AgdaSymbol{(}\AgdaBound{P} \AgdaSymbol{:} \AgdaBound{A1} \AgdaSymbol{→} \AgdaBound{A2} \AgdaSymbol{->} \AgdaBound{A3} \AgdaSymbol{->} \AgdaBound{A4} \<[57]%
\>[57]\AgdaSymbol{→} \AgdaPrimitiveType{Set}\AgdaSymbol{)} \<[64]%
\>[64]\<%
\\
\>[11]\AgdaIndent{13}{}\<[13]%
\>[13]\AgdaSymbol{\{}\AgdaBound{a1} \AgdaBound{a1'} \AgdaBound{a2} \AgdaBound{a2'} \AgdaBound{a3} \AgdaBound{a3'} \AgdaBound{a4} \AgdaBound{a4'}\AgdaSymbol{\}} \AgdaSymbol{→}\<%
\\
\>[0]\AgdaIndent{11}{}\<[11]%
\>[11]\AgdaBound{a1} \AgdaDatatype{≡} \AgdaBound{a1'} \AgdaSymbol{->} \AgdaBound{a2} \AgdaDatatype{≡} \AgdaBound{a2'} \AgdaSymbol{->} \AgdaBound{a3} \AgdaDatatype{≡} \AgdaBound{a3'} \AgdaSymbol{->} \AgdaBound{a4} \AgdaDatatype{≡} \AgdaBound{a4'} \AgdaSymbol{->}\<%
\\
\>[0]\AgdaIndent{11}{}\<[11]%
\>[11]\AgdaBound{P} \AgdaBound{a1} \AgdaBound{a2} \AgdaBound{a3} \AgdaBound{a4} \AgdaSymbol{->} \AgdaBound{P} \AgdaBound{a1'} \AgdaBound{a2'} \AgdaBound{a3'} \AgdaBound{a4'}\<%
\\
\>\AgdaFunction{subst₄} \AgdaSymbol{\_} \AgdaInductiveConstructor{refl} \AgdaInductiveConstructor{refl} \AgdaInductiveConstructor{refl} \AgdaInductiveConstructor{refl} \AgdaBound{Paaaa} \AgdaSymbol{=} \AgdaBound{Paaaa}\<%
\end{code}

\AgdaHide{
\begin{code}%
\>\AgdaKeyword{open} \AgdaKeyword{import} \AgdaModule{Prelims.EqReasoning} \AgdaKeyword{public}\<%
\end{code}
}

\begin{code}%
\>\AgdaKeyword{module} \AgdaModule{Prelims.Snoclist} \AgdaKeyword{where}\<%
\\
\>\AgdaKeyword{open} \AgdaKeyword{import} \AgdaModule{Data.Nat}\<%
\\
\>\AgdaKeyword{open} \AgdaKeyword{import} \AgdaModule{Data.Fin}\<%
\\
%
\\
\>\AgdaKeyword{infixl} \AgdaNumber{20} \AgdaFixityOp{\_snoc\_}\<%
\\
\>\AgdaKeyword{data} \AgdaDatatype{snocList} \AgdaSymbol{(}\AgdaBound{A} \AgdaSymbol{:} \AgdaPrimitiveType{Set}\AgdaSymbol{)} \AgdaSymbol{:} \AgdaPrimitiveType{Set} \AgdaKeyword{where}\<%
\\
\>[0]\AgdaIndent{2}{}\<[2]%
\>[2]\AgdaInductiveConstructor{[]} \AgdaSymbol{:} \AgdaDatatype{snocList} \AgdaBound{A}\<%
\\
\>[0]\AgdaIndent{2}{}\<[2]%
\>[2]\AgdaInductiveConstructor{\_snoc\_} \AgdaSymbol{:} \AgdaDatatype{snocList} \AgdaBound{A} \AgdaSymbol{→} \AgdaBound{A} \AgdaSymbol{→} \AgdaDatatype{snocList} \AgdaBound{A}\<%
\\
%
\\
\>\AgdaFunction{replicate} \AgdaSymbol{:} \AgdaSymbol{∀} \AgdaSymbol{\{}\AgdaBound{A}\AgdaSymbol{\}} \AgdaSymbol{→} \AgdaDatatype{ℕ} \AgdaSymbol{→} \AgdaBound{A} \AgdaSymbol{→} \AgdaDatatype{snocList} \AgdaBound{A}\<%
\\
\>\AgdaFunction{replicate} \AgdaInductiveConstructor{zero} \AgdaSymbol{\_} \AgdaSymbol{=} \AgdaInductiveConstructor{[]}\<%
\\
\>\AgdaFunction{replicate} \AgdaSymbol{(}\AgdaInductiveConstructor{suc} \AgdaBound{n}\AgdaSymbol{)} \AgdaBound{a} \AgdaSymbol{=} \AgdaFunction{replicate} \AgdaBound{n} \AgdaBound{a} \AgdaInductiveConstructor{snoc} \AgdaBound{a}\<%
\\
%
\\
\>\AgdaKeyword{data} \AgdaDatatype{snocVec} \AgdaSymbol{(}\AgdaBound{A} \AgdaSymbol{:} \AgdaPrimitiveType{Set}\AgdaSymbol{)} \AgdaSymbol{:} \AgdaDatatype{ℕ} \AgdaSymbol{→} \AgdaPrimitiveType{Set} \AgdaKeyword{where}\<%
\\
\>[0]\AgdaIndent{2}{}\<[2]%
\>[2]\AgdaInductiveConstructor{[]} \AgdaSymbol{:} \AgdaDatatype{snocVec} \AgdaBound{A} \AgdaInductiveConstructor{zero}\<%
\\
\>[0]\AgdaIndent{2}{}\<[2]%
\>[2]\AgdaInductiveConstructor{\_snoc\_} \AgdaSymbol{:} \AgdaSymbol{∀} \AgdaSymbol{\{}\AgdaBound{n}\AgdaSymbol{\}} \AgdaSymbol{→} \AgdaDatatype{snocVec} \AgdaBound{A} \AgdaBound{n} \AgdaSymbol{→} \AgdaBound{A} \AgdaSymbol{→} \AgdaDatatype{snocVec} \AgdaBound{A} \AgdaSymbol{(}\AgdaInductiveConstructor{suc} \AgdaBound{n}\AgdaSymbol{)}\<%
\\
%
\\
\>\AgdaFunction{lookup} \AgdaSymbol{:} \AgdaSymbol{∀} \AgdaSymbol{\{}\AgdaBound{A} \AgdaSymbol{:} \AgdaPrimitiveType{Set}\AgdaSymbol{\}} \AgdaSymbol{\{}\AgdaBound{n}\AgdaSymbol{\}} \AgdaSymbol{→} \AgdaDatatype{Fin} \AgdaBound{n} \AgdaSymbol{→} \AgdaDatatype{snocVec} \AgdaBound{A} \AgdaBound{n} \AgdaSymbol{→} \AgdaBound{A}\<%
\\
\>\AgdaFunction{lookup} \AgdaInductiveConstructor{zero} \AgdaSymbol{(\_} \AgdaInductiveConstructor{snoc} \AgdaBound{x}\AgdaSymbol{)} \AgdaSymbol{=} \AgdaBound{x}\<%
\\
\>\AgdaFunction{lookup} \AgdaSymbol{(}\AgdaInductiveConstructor{suc} \AgdaBound{i}\AgdaSymbol{)} \AgdaSymbol{(}\AgdaBound{v} \AgdaInductiveConstructor{snoc} \AgdaSymbol{\_)} \AgdaSymbol{=} \AgdaFunction{lookup} \AgdaBound{i} \AgdaBound{v}\<%
\end{code}

\AgdaHide{
\begin{code}%
\>\AgdaKeyword{open} \AgdaKeyword{import} \AgdaModule{Prelims.Snoclist} \AgdaKeyword{public}\<%
\end{code}
}

\newcommand{\id}[1]{\mathsf{id}_{#1}}

\section{Grammars}

\subsection{Taxonomy}

\AgdaHide{
\begin{code}%
\>\AgdaKeyword{module} \AgdaModule{Grammar.Taxonomy} \AgdaKeyword{where}\<%
\\
\>\AgdaKeyword{open} \AgdaKeyword{import} \AgdaModule{Data.List}\<%
\\
\>\AgdaKeyword{open} \AgdaKeyword{import} \AgdaModule{Prelims}\<%
\end{code}
}

Before we begin investigating the several theories we wish to consider, we present a general theory of syntax and
capture-avoiding substitution.

A \emph{taxononmy} consists of:
\begin{itemize}
\item a set of \emph{expression kinds};
\item a subset of expression kinds, called the \emph{variable kinds}.  We refer to the other expession kinds as \emph{non-variable kinds}.
\end{itemize}

%<*Taxonomy>
\begin{code}%
\>\AgdaKeyword{record} \AgdaRecord{Taxonomy} \AgdaSymbol{:} \AgdaPrimitiveType{Set₁} \AgdaKeyword{where}\<%
\\
\>[0]\AgdaIndent{2}{}\<[2]%
\>[2]\AgdaKeyword{field}\<%
\\
\>[2]\AgdaIndent{4}{}\<[4]%
\>[4]\AgdaField{VarKind} \AgdaSymbol{:} \AgdaPrimitiveType{Set}\<%
\\
\>[2]\AgdaIndent{4}{}\<[4]%
\>[4]\AgdaField{NonVarKind} \AgdaSymbol{:} \AgdaPrimitiveType{Set}\<%
\\
%
\\
\>[0]\AgdaIndent{2}{}\<[2]%
\>[2]\AgdaKeyword{data} \AgdaDatatype{ExpressionKind} \AgdaSymbol{:} \AgdaPrimitiveType{Set} \AgdaKeyword{where}\<%
\\
\>[2]\AgdaIndent{4}{}\<[4]%
\>[4]\AgdaInductiveConstructor{varKind} \AgdaSymbol{:} \AgdaField{VarKind} \AgdaSymbol{→} \AgdaDatatype{ExpressionKind}\<%
\\
\>[2]\AgdaIndent{4}{}\<[4]%
\>[4]\AgdaInductiveConstructor{nonVarKind} \AgdaSymbol{:} \AgdaField{NonVarKind} \AgdaSymbol{→} \AgdaDatatype{ExpressionKind}\<%
\end{code}
%</Taxonomy>

\begin{frame}[fragile]
\frametitle{Alphabets}
An \emph{alphabet} $A$ consists of a finite set of \emph{variables},
\mode<article>{to each of which is assigned a variable kind $K$.
Let $\emptyset$ be the empty alphabet, and $(A , K)$ be the result of extending the alphabet $A$ with one
fresh variable $x₀$ of kind $K$.  We write $\mathsf{Var}\ A\ K$ for the set of all variables in $A$ of kind $K$.}
\mode<beamer>{each with a variable kind.}

\begin{code}%
\>[0]\AgdaIndent{2}{}\<[2]%
\>[2]\AgdaKeyword{infixl} \AgdaNumber{55} \AgdaFixityOp{\_,\_}\<%
\\
\>[0]\AgdaIndent{2}{}\<[2]%
\>[2]\AgdaKeyword{data} \AgdaDatatype{Alphabet} \AgdaSymbol{:} \AgdaPrimitiveType{Set} \AgdaKeyword{where}\<%
\\
\>[2]\AgdaIndent{4}{}\<[4]%
\>[4]\AgdaInductiveConstructor{∅} \AgdaSymbol{:} \AgdaDatatype{Alphabet}\<%
\\
\>[2]\AgdaIndent{4}{}\<[4]%
\>[4]\AgdaInductiveConstructor{\_,\_} \AgdaSymbol{:} \AgdaDatatype{Alphabet} \AgdaSymbol{→} \AgdaField{VarKind} \AgdaSymbol{→} \AgdaDatatype{Alphabet}\<%
\end{code}

\AgdaHide{
\begin{code}%
\>[0]\AgdaIndent{2}{}\<[2]%
\>[2]\AgdaFunction{extend} \AgdaSymbol{:} \AgdaDatatype{Alphabet} \AgdaSymbol{→} \AgdaDatatype{List} \AgdaField{VarKind} \AgdaSymbol{→} \AgdaDatatype{Alphabet}\<%
\\
\>[0]\AgdaIndent{2}{}\<[2]%
\>[2]\AgdaFunction{extend} \AgdaBound{A} \AgdaInductiveConstructor{[]} \AgdaSymbol{=} \AgdaBound{A}\<%
\\
\>[0]\AgdaIndent{2}{}\<[2]%
\>[2]\AgdaFunction{extend} \AgdaBound{A} \AgdaSymbol{(}\AgdaBound{K} \AgdaInductiveConstructor{∷} \AgdaBound{KK}\AgdaSymbol{)} \AgdaSymbol{=} \AgdaFunction{extend} \AgdaSymbol{(}\AgdaBound{A} \AgdaInductiveConstructor{,} \AgdaBound{K}\AgdaSymbol{)} \AgdaBound{KK}\<%
\\
%
\\
\>[0]\AgdaIndent{2}{}\<[2]%
\>[2]\AgdaFunction{snoc-extend} \AgdaSymbol{:} \AgdaDatatype{Alphabet} \AgdaSymbol{→} \AgdaDatatype{snocList} \AgdaField{VarKind} \AgdaSymbol{→} \AgdaDatatype{Alphabet}\<%
\\
\>[0]\AgdaIndent{2}{}\<[2]%
\>[2]\AgdaFunction{snoc-extend} \AgdaBound{A} \AgdaInductiveConstructor{[]} \AgdaSymbol{=} \AgdaBound{A}\<%
\\
\>[0]\AgdaIndent{2}{}\<[2]%
\>[2]\AgdaFunction{snoc-extend} \AgdaBound{A} \AgdaSymbol{(}\AgdaBound{KK} \AgdaInductiveConstructor{snoc} \AgdaBound{K}\AgdaSymbol{)} \AgdaSymbol{=} \AgdaFunction{snoc-extend} \AgdaBound{A} \AgdaBound{KK} \AgdaInductiveConstructor{,} \AgdaBound{K}\<%
\end{code}
}

\begin{code}%
\>[0]\AgdaIndent{2}{}\<[2]%
\>[2]\AgdaKeyword{data} \AgdaDatatype{Var} \AgdaSymbol{:} \AgdaDatatype{Alphabet} \AgdaSymbol{→} \AgdaField{VarKind} \AgdaSymbol{→} \AgdaPrimitiveType{Set} \AgdaKeyword{where}\<%
\\
\>[2]\AgdaIndent{4}{}\<[4]%
\>[4]\AgdaInductiveConstructor{x₀} \AgdaSymbol{:} \AgdaSymbol{∀} \AgdaSymbol{\{}\AgdaBound{V}\AgdaSymbol{\}} \AgdaSymbol{\{}\AgdaBound{K}\AgdaSymbol{\}} \AgdaSymbol{→} \AgdaDatatype{Var} \AgdaSymbol{(}\AgdaBound{V} \AgdaInductiveConstructor{,} \AgdaBound{K}\AgdaSymbol{)} \AgdaBound{K}\<%
\\
\>[2]\AgdaIndent{4}{}\<[4]%
\>[4]\AgdaInductiveConstructor{↑} \AgdaSymbol{:} \AgdaSymbol{∀} \AgdaSymbol{\{}\AgdaBound{V}\AgdaSymbol{\}} \AgdaSymbol{\{}\AgdaBound{K}\AgdaSymbol{\}} \AgdaSymbol{\{}\AgdaBound{L}\AgdaSymbol{\}} \AgdaSymbol{→} \AgdaDatatype{Var} \AgdaBound{V} \AgdaBound{L} \AgdaSymbol{→} \AgdaDatatype{Var} \AgdaSymbol{(}\AgdaBound{V} \AgdaInductiveConstructor{,} \AgdaBound{K}\AgdaSymbol{)} \AgdaBound{L}\<%
\end{code}

\AgdaHide{
\begin{code}%
\>[0]\AgdaIndent{2}{}\<[2]%
\>[2]\AgdaFunction{x₁} \AgdaSymbol{:} \AgdaSymbol{∀} \AgdaSymbol{\{}\AgdaBound{V}\AgdaSymbol{\}} \AgdaSymbol{\{}\AgdaBound{K}\AgdaSymbol{\}} \AgdaSymbol{\{}\AgdaBound{L}\AgdaSymbol{\}} \AgdaSymbol{→} \AgdaDatatype{Var} \AgdaSymbol{(}\AgdaBound{V} \AgdaInductiveConstructor{,} \AgdaBound{K} \AgdaInductiveConstructor{,} \AgdaBound{L}\AgdaSymbol{)} \AgdaBound{K}\<%
\end{code}
}

\begin{code}%
\>[0]\AgdaIndent{2}{}\<[2]%
\>[2]\AgdaFunction{x₁} \AgdaSymbol{=} \AgdaInductiveConstructor{↑} \AgdaInductiveConstructor{x₀}\<%
\end{code}

\AgdaHide{
\begin{code}%
\>[0]\AgdaIndent{2}{}\<[2]%
\>[2]\AgdaFunction{x₂} \AgdaSymbol{:} \AgdaSymbol{∀} \AgdaSymbol{\{}\AgdaBound{V}\AgdaSymbol{\}} \AgdaSymbol{\{}\AgdaBound{K}\AgdaSymbol{\}} \AgdaSymbol{\{}\AgdaBound{L}\AgdaSymbol{\}} \AgdaSymbol{\{}\AgdaBound{L'}\AgdaSymbol{\}} \AgdaSymbol{→} \AgdaDatatype{Var} \AgdaSymbol{(}\AgdaBound{V} \AgdaInductiveConstructor{,} \AgdaBound{K} \AgdaInductiveConstructor{,} \AgdaBound{L} \AgdaInductiveConstructor{,} \AgdaBound{L'}\AgdaSymbol{)} \AgdaBound{K}\<%
\end{code}
}

\begin{code}%
\>[0]\AgdaIndent{2}{}\<[2]%
\>[2]\AgdaFunction{x₂} \AgdaSymbol{=} \AgdaInductiveConstructor{↑} \AgdaFunction{x₁}\<%
\end{code}
\end{frame}

A constructor $c$ of kind (\ref{eq:conkind}) is a constructor that takes $m$ arguments of kind $B_1$, \ldots, $B_m$, and binds $r_i$ variables in its $i$th argument of kind $A_{ij}$,
producing an expression of kind $C$.  We write this expression as

\begin{equation}
\label{eq:expression}
c([x_{11}, \ldots, x_{1r_1}]E_1, \ldots, [x_{m1}, \ldots, x_{mr_m}]E_m) \enspace .
\end{equation}

The subexpressions of the form $[x_{i1}, \ldots, x_{ir_i}]E_i$ shall be called \emph{abstractions}.

When giving a specific grammar, we shall feel free to use BNF notation.  

We formalise this as follows.  First, we construct the sets of expression kinds and constructor kinds over a taxonomy:

\begin{frame}[fragile]
There are two \emph{classes} of kinds: expression kinds and constructor kinds.

\begin{code}%
\>[0]\AgdaIndent{2}{}\<[2]%
\>[2]\AgdaKeyword{data} \AgdaDatatype{KindClass} \AgdaSymbol{:} \AgdaPrimitiveType{Set} \AgdaKeyword{where}\<%
\\
\>[2]\AgdaIndent{4}{}\<[4]%
\>[4]\AgdaInductiveConstructor{-Expression} \AgdaSymbol{:} \AgdaDatatype{KindClass}\<%
\\
\>[2]\AgdaIndent{4}{}\<[4]%
\>[4]\AgdaInductiveConstructor{-Constructor} \AgdaSymbol{:} \AgdaDatatype{ExpressionKind} \AgdaSymbol{→} \AgdaDatatype{KindClass}\<%
\\
%
\\
\>[0]\AgdaIndent{2}{}\<[2]%
\>[2]\AgdaKeyword{data} \AgdaDatatype{Kind} \AgdaSymbol{:} \AgdaDatatype{KindClass} \AgdaSymbol{→} \AgdaPrimitiveType{Set} \AgdaKeyword{where}\<%
\\
\>[2]\AgdaIndent{4}{}\<[4]%
\>[4]\AgdaInductiveConstructor{base} \AgdaSymbol{:} \AgdaDatatype{ExpressionKind} \AgdaSymbol{→} \AgdaDatatype{Kind} \AgdaInductiveConstructor{-Expression}\<%
\\
%
\\
\>[2]\AgdaIndent{4}{}\<[4]%
\>[4]\AgdaInductiveConstructor{out} \<[9]%
\>[9]\AgdaSymbol{:} \AgdaSymbol{∀} \AgdaBound{K} \AgdaSymbol{→} \AgdaDatatype{Kind} \AgdaSymbol{(}\AgdaInductiveConstructor{-Constructor} \AgdaBound{K}\AgdaSymbol{)}\<%
\\
\>[2]\AgdaIndent{4}{}\<[4]%
\>[4]\AgdaInductiveConstructor{Π} \<[9]%
\>[9]\AgdaSymbol{:} \AgdaSymbol{∀} \AgdaSymbol{\{}\AgdaBound{K}\AgdaSymbol{\}} \AgdaSymbol{→} \AgdaDatatype{List} \AgdaField{VarKind} \AgdaSymbol{→} \AgdaDatatype{ExpressionKind} \AgdaSymbol{→} \<[51]%
\>[51]\<%
\\
\>[4]\AgdaIndent{11}{}\<[11]%
\>[11]\AgdaDatatype{Kind} \AgdaSymbol{(}\AgdaInductiveConstructor{-Constructor} \AgdaBound{K}\AgdaSymbol{)} \AgdaSymbol{→} \AgdaDatatype{Kind} \AgdaSymbol{(}\AgdaInductiveConstructor{-Constructor} \AgdaBound{K}\AgdaSymbol{)}\<%
\end{code}
\end{frame}
%TODO Colours in Agda code?

\AgdaHide{
\begin{code}%
\>\AgdaComment{\{- Metavariable conventions:\<\\
\>  A, B    range over abstraction kinds\<\\
\>  C       range over kind classes\<\\
\>  AA, BB  range over lists of abstraction kinds\<\\
\>  E, F, G range over subexpressions\<\\
\>  K, L    range over expression kinds including variable kinds\<\\
\>  M, N, P range over expressions\<\\
\>  U, V, W range over alphabets -\}}\<%
\\
\>\AgdaKeyword{open} \AgdaKeyword{import} \AgdaModule{Function}\<%
\\
\>\AgdaKeyword{open} \AgdaKeyword{import} \AgdaModule{Data.List}\<%
\\
\>\AgdaKeyword{open} \AgdaKeyword{import} \AgdaModule{Prelims}\<%
\\
\>\AgdaKeyword{open} \AgdaKeyword{import} \AgdaModule{Grammar.Taxonomy}\<%
\\
%
\\
\>\AgdaKeyword{module} \AgdaModule{Grammar.Base} \AgdaKeyword{where}\<%
\\
%
\\
\>\AgdaKeyword{record} \AgdaRecord{IsGrammar} \AgdaSymbol{(}\AgdaBound{T} \AgdaSymbol{:} \AgdaRecord{Taxonomy}\AgdaSymbol{)} \AgdaSymbol{:} \AgdaPrimitiveType{Set₁} \AgdaKeyword{where}\<%
\\
\>[0]\AgdaIndent{2}{}\<[2]%
\>[2]\AgdaKeyword{open} \AgdaModule{Taxonomy} \AgdaBound{T}\<%
\\
\>[0]\AgdaIndent{2}{}\<[2]%
\>[2]\AgdaKeyword{field}\<%
\\
\>[2]\AgdaIndent{4}{}\<[4]%
\>[4]\AgdaField{Constructor} \<[19]%
\>[19]\AgdaSymbol{:} \AgdaFunction{ConKind} \AgdaSymbol{→} \AgdaPrimitiveType{Set}\<%
\\
\>[2]\AgdaIndent{4}{}\<[4]%
\>[4]\AgdaField{parent} \<[19]%
\>[19]\AgdaSymbol{:} \AgdaFunction{VarKind} \AgdaSymbol{→} \AgdaDatatype{ExpKind}\<%
\\
%
\\
\>\AgdaKeyword{record} \AgdaRecord{Grammar} \AgdaSymbol{:} \AgdaPrimitiveType{Set₁} \AgdaKeyword{where}\<%
\\
\>[0]\AgdaIndent{2}{}\<[2]%
\>[2]\AgdaKeyword{field}\<%
\\
\>[2]\AgdaIndent{4}{}\<[4]%
\>[4]\AgdaField{taxonomy} \AgdaSymbol{:} \AgdaRecord{Taxonomy}\<%
\\
\>[2]\AgdaIndent{4}{}\<[4]%
\>[4]\AgdaField{isGrammar} \AgdaSymbol{:} \AgdaRecord{IsGrammar} \AgdaField{taxonomy}\<%
\\
\>[0]\AgdaIndent{2}{}\<[2]%
\>[2]\AgdaKeyword{open} \AgdaModule{Taxonomy} \AgdaField{taxonomy} \AgdaKeyword{public}\<%
\\
\>[0]\AgdaIndent{2}{}\<[2]%
\>[2]\AgdaKeyword{open} \AgdaModule{IsGrammar} \AgdaField{isGrammar} \AgdaKeyword{public}\<%
\end{code}
}

%<*Expression>
\begin{code}%
\>[0]\AgdaIndent{2}{}\<[2]%
\>[2]\AgdaKeyword{data} \AgdaDatatype{Subexpression} \AgdaSymbol{(}\AgdaBound{V} \AgdaSymbol{:} \AgdaDatatype{Alphabet}\AgdaSymbol{)} \AgdaSymbol{:} \AgdaSymbol{∀} \AgdaBound{C} \AgdaSymbol{→} \AgdaFunction{Kind} \AgdaBound{C} \AgdaSymbol{→} \AgdaPrimitiveType{Set}\<%
\\
\>[0]\AgdaIndent{2}{}\<[2]%
\>[2]\AgdaFunction{Expression} \AgdaSymbol{:} \AgdaDatatype{Alphabet} \AgdaSymbol{→} \AgdaDatatype{ExpKind} \AgdaSymbol{→} \AgdaPrimitiveType{Set}\<%
\\
\>[0]\AgdaIndent{2}{}\<[2]%
\>[2]\AgdaFunction{VExpression} \AgdaSymbol{:} \AgdaDatatype{Alphabet} \AgdaSymbol{→} \AgdaFunction{VarKind} \AgdaSymbol{→} \AgdaPrimitiveType{Set}\<%
\\
\>[0]\AgdaIndent{2}{}\<[2]%
\>[2]\AgdaFunction{Abstraction} \AgdaSymbol{:} \AgdaDatatype{Alphabet} \AgdaSymbol{→} \AgdaFunction{AbsKind} \AgdaSymbol{→} \AgdaPrimitiveType{Set}\<%
\\
\>[0]\AgdaIndent{2}{}\<[2]%
\>[2]\AgdaFunction{ListAbs} \AgdaSymbol{:} \AgdaDatatype{Alphabet} \AgdaSymbol{→} \AgdaDatatype{List} \AgdaFunction{AbsKind} \AgdaSymbol{→} \AgdaPrimitiveType{Set}\<%
\\
%
\\
\>[0]\AgdaIndent{2}{}\<[2]%
\>[2]\AgdaFunction{Expression} \AgdaBound{V} \AgdaBound{K} \AgdaSymbol{=} \AgdaDatatype{Subexpression} \AgdaBound{V} \AgdaInductiveConstructor{-Expression} \AgdaBound{K}\<%
\\
\>[0]\AgdaIndent{2}{}\<[2]%
\>[2]\AgdaFunction{VExpression} \AgdaBound{V} \AgdaBound{K} \AgdaSymbol{=} \AgdaFunction{Expression} \AgdaBound{V} \AgdaSymbol{(}\AgdaInductiveConstructor{varKind} \AgdaBound{K}\AgdaSymbol{)}\<%
\\
\>[0]\AgdaIndent{2}{}\<[2]%
\>[2]\AgdaFunction{Abstraction} \AgdaBound{V} \AgdaSymbol{(}\AgdaInductiveConstructor{SK} \AgdaBound{AA} \AgdaBound{K}\AgdaSymbol{)} \AgdaSymbol{=} \AgdaFunction{Expression} \AgdaSymbol{(}\AgdaFunction{extend} \AgdaBound{V} \AgdaBound{AA}\AgdaSymbol{)} \AgdaBound{K}\<%
\\
\>[0]\AgdaIndent{2}{}\<[2]%
\>[2]\AgdaFunction{ListAbs} \AgdaBound{V} \AgdaBound{AA} \AgdaSymbol{=} \AgdaDatatype{Subexpression} \AgdaBound{V} \AgdaInductiveConstructor{-ListAbs} \AgdaBound{AA}\<%
\\
%
\\
\>[0]\AgdaIndent{2}{}\<[2]%
\>[2]\AgdaKeyword{infixr} \AgdaNumber{5} \AgdaFixityOp{\_∷\_}\<%
\\
\>[0]\AgdaIndent{2}{}\<[2]%
\>[2]\AgdaKeyword{data} \AgdaDatatype{Subexpression} \AgdaBound{V} \AgdaKeyword{where}\<%
\\
\>[2]\AgdaIndent{4}{}\<[4]%
\>[4]\AgdaInductiveConstructor{var} \AgdaSymbol{:} \AgdaSymbol{∀} \AgdaSymbol{\{}\AgdaBound{K}\AgdaSymbol{\}} \AgdaSymbol{→} \AgdaDatatype{Var} \AgdaBound{V} \AgdaBound{K} \AgdaSymbol{→} \AgdaFunction{VExpression} \AgdaBound{V} \AgdaBound{K}\<%
\\
\>[2]\AgdaIndent{4}{}\<[4]%
\>[4]\AgdaInductiveConstructor{app} \AgdaSymbol{:} \AgdaSymbol{∀} \AgdaSymbol{\{}\AgdaBound{AA}\AgdaSymbol{\}} \AgdaSymbol{\{}\AgdaBound{K}\AgdaSymbol{\}} \AgdaSymbol{→} \AgdaFunction{Constructor} \AgdaSymbol{(}\AgdaInductiveConstructor{SK} \AgdaBound{AA} \AgdaBound{K}\AgdaSymbol{)} \AgdaSymbol{→} \AgdaFunction{ListAbs} \AgdaBound{V} \AgdaBound{AA} \AgdaSymbol{→} \AgdaFunction{Expression} \AgdaBound{V} \AgdaBound{K}\<%
\\
\>[2]\AgdaIndent{4}{}\<[4]%
\>[4]\AgdaInductiveConstructor{[]} \AgdaSymbol{:} \AgdaFunction{ListAbs} \AgdaBound{V} \AgdaInductiveConstructor{[]}\<%
\\
\>[2]\AgdaIndent{4}{}\<[4]%
\>[4]\AgdaInductiveConstructor{\_∷\_} \AgdaSymbol{:} \AgdaSymbol{∀} \AgdaSymbol{\{}\AgdaBound{A}\AgdaSymbol{\}} \AgdaSymbol{\{}\AgdaBound{AA}\AgdaSymbol{\}} \AgdaSymbol{→} \AgdaFunction{Abstraction} \AgdaBound{V} \AgdaBound{A} \AgdaSymbol{→} \AgdaFunction{ListAbs} \AgdaBound{V} \AgdaBound{AA} \AgdaSymbol{→} \AgdaFunction{ListAbs} \AgdaBound{V} \AgdaSymbol{(}\AgdaBound{A} \AgdaInductiveConstructor{∷} \AgdaBound{AA}\AgdaSymbol{)}\<%
\end{code}
%</Expression>

We prove that the constructor \AgdaRef{var} is injective.

\begin{code}%
\>[0]\AgdaIndent{2}{}\<[2]%
\>[2]\AgdaFunction{var-inj} \AgdaSymbol{:} \AgdaSymbol{∀} \AgdaSymbol{\{}\AgdaBound{V}\AgdaSymbol{\}} \AgdaSymbol{\{}\AgdaBound{K}\AgdaSymbol{\}} \AgdaSymbol{\{}\AgdaBound{x} \AgdaBound{y} \AgdaSymbol{:} \AgdaDatatype{Var} \AgdaBound{V} \AgdaBound{K}\AgdaSymbol{\}} \AgdaSymbol{→} \AgdaInductiveConstructor{var} \AgdaBound{x} \AgdaDatatype{≡} \AgdaInductiveConstructor{var} \AgdaBound{y} \AgdaSymbol{→} \AgdaBound{x} \AgdaDatatype{≡} \AgdaBound{y}\<%
\\
\>[0]\AgdaIndent{2}{}\<[2]%
\>[2]\AgdaFunction{var-inj} \AgdaInductiveConstructor{refl} \AgdaSymbol{=} \AgdaInductiveConstructor{refl}\<%
\end{code}

For the future, we also define the type of all snoc-lists of expressions $(M_1, \ldots, M_n)$
such that $M_i$ is of type $K_i$, given a snoc-list of variable kinds $(K_1, \ldots, K_n)$.

\begin{code}%
\>[0]\AgdaIndent{2}{}\<[2]%
\>[2]\AgdaKeyword{infixl} \AgdaNumber{20} \AgdaFixityOp{\_snoc\_}\<%
\\
\>[0]\AgdaIndent{2}{}\<[2]%
\>[2]\AgdaKeyword{data} \AgdaDatatype{snocListExp} \AgdaBound{V} \AgdaSymbol{:} \AgdaDatatype{snocList} \AgdaFunction{VarKind} \AgdaSymbol{→} \AgdaPrimitiveType{Set} \AgdaKeyword{where}\<%
\\
\>[2]\AgdaIndent{4}{}\<[4]%
\>[4]\AgdaInductiveConstructor{[]} \AgdaSymbol{:} \AgdaDatatype{snocListExp} \AgdaBound{V} \AgdaInductiveConstructor{[]}\<%
\\
\>[2]\AgdaIndent{4}{}\<[4]%
\>[4]\AgdaInductiveConstructor{\_snoc\_} \AgdaSymbol{:} \AgdaSymbol{∀} \AgdaSymbol{\{}\AgdaBound{A}\AgdaSymbol{\}} \AgdaSymbol{\{}\AgdaBound{K}\AgdaSymbol{\}} \AgdaSymbol{→} \AgdaDatatype{snocListExp} \AgdaBound{V} \AgdaBound{A} \AgdaSymbol{→} \AgdaFunction{Expression} \AgdaBound{V} \AgdaSymbol{(}\AgdaInductiveConstructor{varKind} \AgdaBound{K}\AgdaSymbol{)} \AgdaSymbol{→} \AgdaDatatype{snocListExp} \AgdaBound{V} \AgdaSymbol{(}\AgdaBound{A} \AgdaInductiveConstructor{snoc} \AgdaBound{K}\AgdaSymbol{)}\<%
\end{code}

A \emph{reduction} is a relation $\rhd$ between expressions such that, if $E \rhd F$,
then $E$ is not a variable.  It is given by a term $R : \AgdaRef{Reduction}$ such that
$R\, c\, MM\, N$ iff $c[MM] \rhd N$.

\begin{code}%
\>[0]\AgdaIndent{2}{}\<[2]%
\>[2]\AgdaFunction{Reduction} \AgdaSymbol{:} \AgdaPrimitiveType{Set₁}\<%
\\
\>[0]\AgdaIndent{2}{}\<[2]%
\>[2]\AgdaFunction{Reduction} \AgdaSymbol{=} \AgdaSymbol{∀} \AgdaSymbol{\{}\AgdaBound{V}\AgdaSymbol{\}} \AgdaSymbol{\{}\AgdaBound{AA}\AgdaSymbol{\}} \AgdaSymbol{\{}\AgdaBound{K}\AgdaSymbol{\}} \AgdaSymbol{→} \AgdaFunction{Constructor} \AgdaSymbol{(}\AgdaInductiveConstructor{SK} \AgdaBound{AA} \AgdaBound{K}\AgdaSymbol{)} \AgdaSymbol{→} \AgdaFunction{ListAbs} \AgdaBound{V} \AgdaBound{AA} \AgdaSymbol{→} \AgdaFunction{Expression} \AgdaBound{V} \AgdaBound{K} \AgdaSymbol{→} \AgdaPrimitiveType{Set}\<%
\end{code}
}


We define the operations of replacement and substitution on
expressions.  The details are given in Appendix \ref{appendix:repsub}.

\AgdaHide{
\begin{code}%
\>\AgdaKeyword{open} \AgdaKeyword{import} \AgdaModule{Grammar.Base}\<%
\\
%
\\
\>\AgdaKeyword{module} \AgdaModule{Grammar.Context} \AgdaSymbol{(}\AgdaBound{G} \AgdaSymbol{:} \AgdaRecord{Grammar}\AgdaSymbol{)} \AgdaKeyword{where}\<%
\\
%
\\
\>\AgdaKeyword{open} \AgdaKeyword{import} \AgdaModule{Data.Nat}\<%
\\
\>\AgdaKeyword{open} \AgdaKeyword{import} \AgdaModule{Data.Fin}\<%
\\
\>\AgdaKeyword{open} \AgdaKeyword{import} \AgdaModule{Relation.Binary.PropositionalEquality}\<%
\\
\>\AgdaKeyword{open} \AgdaModule{Grammar} \AgdaBound{G}\<%
\\
\>\AgdaKeyword{open} \AgdaKeyword{import} \AgdaModule{Grammar.Replacement} \AgdaBound{G}\<%
\end{code}
}

\subsection{Contexts}

A \emph{context} has the form $x_1 : A_1, \ldots, x_n : A_n$ where, for each $i$:
\begin{itemize}
\item $x_i$ is a variable of kind $K_i$ distinct from $x_1$, \ldots, $x_{i-1}$;
\item $A_i$ is an expression whose kind is the parent of $K_i$.
\end{itemize}
The \emph{domain} of this context is the alphabet $\{ x_1, \ldots, x_n \}$.

\begin{code}%
\>\AgdaKeyword{infixl} \AgdaNumber{55} \AgdaFixityOp{\_,\_}\<%
\\
\>\AgdaKeyword{data} \AgdaDatatype{Context} \AgdaSymbol{:} \AgdaDatatype{Alphabet} \AgdaSymbol{→} \AgdaPrimitiveType{Set} \AgdaKeyword{where}\<%
\\
\>[0]\AgdaIndent{2}{}\<[2]%
\>[2]\AgdaInductiveConstructor{〈〉} \AgdaSymbol{:} \AgdaDatatype{Context} \AgdaInductiveConstructor{∅}\<%
\\
\>[0]\AgdaIndent{2}{}\<[2]%
\>[2]\AgdaInductiveConstructor{\_,\_} \AgdaSymbol{:} \AgdaSymbol{∀} \AgdaSymbol{\{}\AgdaBound{V}\AgdaSymbol{\}} \AgdaSymbol{\{}\AgdaBound{K}\AgdaSymbol{\}} \AgdaSymbol{→} \AgdaDatatype{Context} \AgdaBound{V} \AgdaSymbol{→} \AgdaFunction{Expression} \AgdaBound{V} \AgdaSymbol{(}\AgdaFunction{parent} \AgdaBound{K}\AgdaSymbol{)} \AgdaSymbol{→} \<[58]%
\>[58]\<%
\\
\>[2]\AgdaIndent{4}{}\<[4]%
\>[4]\AgdaDatatype{Context} \AgdaSymbol{(}\AgdaBound{V} \AgdaInductiveConstructor{,} \AgdaBound{K}\AgdaSymbol{)}\<%
\\
%
\\
\>\AgdaComment{-- Define typeof such that, if x : A ∈ Γ, then typeof x Γ ≡ A}\<%
\\
\>\AgdaComment{-- We define it the following way so that typeof x Γ computes to an expression of the form}\<%
\\
\>\AgdaComment{-- M 〈 upRep 〉, even if x is not in canonical form}\<%
\\
\>\AgdaFunction{pretypeof} \AgdaSymbol{:} \AgdaSymbol{∀} \AgdaSymbol{\{}\AgdaBound{V}\AgdaSymbol{\}} \AgdaSymbol{\{}\AgdaBound{K}\AgdaSymbol{\}} \AgdaSymbol{\{}\AgdaBound{L}\AgdaSymbol{\}} \AgdaSymbol{(}\AgdaBound{x} \AgdaSymbol{:} \AgdaDatatype{Var} \AgdaSymbol{(}\AgdaBound{V} \AgdaInductiveConstructor{,} \AgdaBound{K}\AgdaSymbol{)} \AgdaBound{L}\AgdaSymbol{)} \AgdaSymbol{(}\AgdaBound{Γ} \AgdaSymbol{:} \AgdaDatatype{Context} \AgdaSymbol{(}\AgdaBound{V} \AgdaInductiveConstructor{,} \AgdaBound{K}\AgdaSymbol{))} \AgdaSymbol{→} \AgdaFunction{Expression} \AgdaBound{V} \AgdaSymbol{(}\AgdaFunction{parent} \AgdaBound{L}\AgdaSymbol{)}\<%
\\
\>\AgdaFunction{typeof} \AgdaSymbol{:} \AgdaSymbol{∀} \AgdaSymbol{\{}\AgdaBound{V}\AgdaSymbol{\}} \AgdaSymbol{\{}\AgdaBound{K}\AgdaSymbol{\}} \AgdaSymbol{(}\AgdaBound{x} \AgdaSymbol{:} \AgdaDatatype{Var} \AgdaBound{V} \AgdaBound{K}\AgdaSymbol{)} \AgdaSymbol{(}\AgdaBound{Γ} \AgdaSymbol{:} \AgdaDatatype{Context} \AgdaBound{V}\AgdaSymbol{)} \AgdaSymbol{→} \AgdaFunction{Expression} \AgdaBound{V} \AgdaSymbol{(}\AgdaFunction{parent} \AgdaBound{K}\AgdaSymbol{)}\<%
\\
%
\\
\>\AgdaFunction{pretypeof} \AgdaInductiveConstructor{x₀} \AgdaSymbol{(}\AgdaBound{Γ} \AgdaInductiveConstructor{,} \AgdaBound{A}\AgdaSymbol{)} \AgdaSymbol{=} \AgdaBound{A}\<%
\\
\>\AgdaFunction{pretypeof} \AgdaSymbol{(}\AgdaInductiveConstructor{↑} \AgdaBound{x}\AgdaSymbol{)} \AgdaSymbol{(}\AgdaBound{Γ} \AgdaInductiveConstructor{,} \AgdaBound{A}\AgdaSymbol{)} \AgdaSymbol{=} \AgdaFunction{typeof} \AgdaBound{x} \AgdaBound{Γ}\<%
\\
%
\\
\>\AgdaFunction{typeof} \AgdaSymbol{\{}\AgdaInductiveConstructor{∅}\AgdaSymbol{\}} \AgdaSymbol{()}\<%
\\
\>\AgdaFunction{typeof} \AgdaSymbol{\{\_} \AgdaInductiveConstructor{,} \AgdaSymbol{\_\}} \AgdaBound{x} \AgdaBound{Γ} \AgdaSymbol{=} \AgdaFunction{pretypeof} \AgdaBound{x} \AgdaBound{Γ} \AgdaFunction{⇑}\<%
\end{code}

We say that a replacement $\rho$ is a \emph{(well-typed) replacement from $\Gamma$ to $\Delta$},
$\rho : \Gamma \rightarrow \Delta$, iff, for each entry $x : A$ in $\Gamma$, we have that
$\rho(x) : A \langle ρ \rangle$ is an entry in $\Delta$.

\begin{code}%
\>\AgdaFunction{\_∶\_⇒R\_} \AgdaSymbol{:} \AgdaSymbol{∀} \AgdaSymbol{\{}\AgdaBound{U}\AgdaSymbol{\}} \AgdaSymbol{\{}\AgdaBound{V}\AgdaSymbol{\}} \AgdaSymbol{→} \AgdaFunction{Rep} \AgdaBound{U} \AgdaBound{V} \AgdaSymbol{→} \AgdaDatatype{Context} \AgdaBound{U} \AgdaSymbol{→} \AgdaDatatype{Context} \AgdaBound{V} \AgdaSymbol{→} \AgdaPrimitiveType{Set}\<%
\\
\>\AgdaBound{ρ} \AgdaFunction{∶} \AgdaBound{Γ} \AgdaFunction{⇒R} \AgdaBound{Δ} \AgdaSymbol{=} \AgdaSymbol{∀} \AgdaSymbol{\{}\AgdaBound{K}\AgdaSymbol{\}} \AgdaBound{x} \AgdaSymbol{→} \AgdaFunction{typeof} \AgdaSymbol{(}\AgdaBound{ρ} \AgdaBound{K} \AgdaBound{x}\AgdaSymbol{)} \AgdaBound{Δ} \AgdaDatatype{≡} \AgdaFunction{typeof} \AgdaBound{x} \AgdaBound{Γ} \AgdaFunction{〈} \AgdaBound{ρ} \AgdaFunction{〉}\<%
\end{code}

\AgdaHide{
\begin{code}%
\>\AgdaComment{\{- Metavariable conventions:\<\\
\>  A, B    range over abstraction kinds\<\\
\>  C       range over kind classes\<\\
\>  AA, BB  range over lists of abstraction kinds\<\\
\>  E, F, G range over subexpressions\<\\
\>  K, L    range over expression kinds including variable kinds\<\\
\>  M, N, P range over expressions\<\\
\>  U, V, W range over alphabets -\}}\<%
\\
\>\AgdaKeyword{open} \AgdaKeyword{import} \AgdaModule{Function}\<%
\\
\>\AgdaKeyword{open} \AgdaKeyword{import} \AgdaModule{Data.List}\<%
\\
\>\AgdaKeyword{open} \AgdaKeyword{import} \AgdaModule{Prelims}\<%
\\
\>\AgdaKeyword{open} \AgdaKeyword{import} \AgdaModule{Grammar.Taxonomy}\<%
\\
%
\\
\>\AgdaKeyword{module} \AgdaModule{Grammar.Base} \AgdaKeyword{where}\<%
\\
%
\\
\>\AgdaKeyword{record} \AgdaRecord{IsGrammar} \AgdaSymbol{(}\AgdaBound{T} \AgdaSymbol{:} \AgdaRecord{Taxonomy}\AgdaSymbol{)} \AgdaSymbol{:} \AgdaPrimitiveType{Set₁} \AgdaKeyword{where}\<%
\\
\>[0]\AgdaIndent{2}{}\<[2]%
\>[2]\AgdaKeyword{open} \AgdaModule{Taxonomy} \AgdaBound{T}\<%
\\
\>[0]\AgdaIndent{2}{}\<[2]%
\>[2]\AgdaKeyword{field}\<%
\\
\>[2]\AgdaIndent{4}{}\<[4]%
\>[4]\AgdaField{Constructor} \<[19]%
\>[19]\AgdaSymbol{:} \AgdaFunction{ConKind} \AgdaSymbol{→} \AgdaPrimitiveType{Set}\<%
\\
\>[2]\AgdaIndent{4}{}\<[4]%
\>[4]\AgdaField{parent} \<[19]%
\>[19]\AgdaSymbol{:} \AgdaFunction{VarKind} \AgdaSymbol{→} \AgdaDatatype{ExpKind}\<%
\\
%
\\
\>\AgdaKeyword{record} \AgdaRecord{Grammar} \AgdaSymbol{:} \AgdaPrimitiveType{Set₁} \AgdaKeyword{where}\<%
\\
\>[0]\AgdaIndent{2}{}\<[2]%
\>[2]\AgdaKeyword{field}\<%
\\
\>[2]\AgdaIndent{4}{}\<[4]%
\>[4]\AgdaField{taxonomy} \AgdaSymbol{:} \AgdaRecord{Taxonomy}\<%
\\
\>[2]\AgdaIndent{4}{}\<[4]%
\>[4]\AgdaField{isGrammar} \AgdaSymbol{:} \AgdaRecord{IsGrammar} \AgdaField{taxonomy}\<%
\\
\>[0]\AgdaIndent{2}{}\<[2]%
\>[2]\AgdaKeyword{open} \AgdaModule{Taxonomy} \AgdaField{taxonomy} \AgdaKeyword{public}\<%
\\
\>[0]\AgdaIndent{2}{}\<[2]%
\>[2]\AgdaKeyword{open} \AgdaModule{IsGrammar} \AgdaField{isGrammar} \AgdaKeyword{public}\<%
\end{code}
}

%<*Expression>
\begin{code}%
\>[0]\AgdaIndent{2}{}\<[2]%
\>[2]\AgdaKeyword{data} \AgdaDatatype{Subexpression} \AgdaSymbol{(}\AgdaBound{V} \AgdaSymbol{:} \AgdaDatatype{Alphabet}\AgdaSymbol{)} \AgdaSymbol{:} \AgdaSymbol{∀} \AgdaBound{C} \AgdaSymbol{→} \AgdaFunction{Kind} \AgdaBound{C} \AgdaSymbol{→} \AgdaPrimitiveType{Set}\<%
\\
\>[0]\AgdaIndent{2}{}\<[2]%
\>[2]\AgdaFunction{Expression} \AgdaSymbol{:} \AgdaDatatype{Alphabet} \AgdaSymbol{→} \AgdaDatatype{ExpKind} \AgdaSymbol{→} \AgdaPrimitiveType{Set}\<%
\\
\>[0]\AgdaIndent{2}{}\<[2]%
\>[2]\AgdaFunction{VExpression} \AgdaSymbol{:} \AgdaDatatype{Alphabet} \AgdaSymbol{→} \AgdaFunction{VarKind} \AgdaSymbol{→} \AgdaPrimitiveType{Set}\<%
\\
\>[0]\AgdaIndent{2}{}\<[2]%
\>[2]\AgdaFunction{Abstraction} \AgdaSymbol{:} \AgdaDatatype{Alphabet} \AgdaSymbol{→} \AgdaFunction{AbsKind} \AgdaSymbol{→} \AgdaPrimitiveType{Set}\<%
\\
\>[0]\AgdaIndent{2}{}\<[2]%
\>[2]\AgdaFunction{ListAbs} \AgdaSymbol{:} \AgdaDatatype{Alphabet} \AgdaSymbol{→} \AgdaDatatype{List} \AgdaFunction{AbsKind} \AgdaSymbol{→} \AgdaPrimitiveType{Set}\<%
\\
%
\\
\>[0]\AgdaIndent{2}{}\<[2]%
\>[2]\AgdaFunction{Expression} \AgdaBound{V} \AgdaBound{K} \AgdaSymbol{=} \AgdaDatatype{Subexpression} \AgdaBound{V} \AgdaInductiveConstructor{-Expression} \AgdaBound{K}\<%
\\
\>[0]\AgdaIndent{2}{}\<[2]%
\>[2]\AgdaFunction{VExpression} \AgdaBound{V} \AgdaBound{K} \AgdaSymbol{=} \AgdaFunction{Expression} \AgdaBound{V} \AgdaSymbol{(}\AgdaInductiveConstructor{varKind} \AgdaBound{K}\AgdaSymbol{)}\<%
\\
\>[0]\AgdaIndent{2}{}\<[2]%
\>[2]\AgdaFunction{Abstraction} \AgdaBound{V} \AgdaSymbol{(}\AgdaInductiveConstructor{SK} \AgdaBound{AA} \AgdaBound{K}\AgdaSymbol{)} \AgdaSymbol{=} \AgdaFunction{Expression} \AgdaSymbol{(}\AgdaFunction{extend} \AgdaBound{V} \AgdaBound{AA}\AgdaSymbol{)} \AgdaBound{K}\<%
\\
\>[0]\AgdaIndent{2}{}\<[2]%
\>[2]\AgdaFunction{ListAbs} \AgdaBound{V} \AgdaBound{AA} \AgdaSymbol{=} \AgdaDatatype{Subexpression} \AgdaBound{V} \AgdaInductiveConstructor{-ListAbs} \AgdaBound{AA}\<%
\\
%
\\
\>[0]\AgdaIndent{2}{}\<[2]%
\>[2]\AgdaKeyword{infixr} \AgdaNumber{5} \AgdaFixityOp{\_∷\_}\<%
\\
\>[0]\AgdaIndent{2}{}\<[2]%
\>[2]\AgdaKeyword{data} \AgdaDatatype{Subexpression} \AgdaBound{V} \AgdaKeyword{where}\<%
\\
\>[2]\AgdaIndent{4}{}\<[4]%
\>[4]\AgdaInductiveConstructor{var} \AgdaSymbol{:} \AgdaSymbol{∀} \AgdaSymbol{\{}\AgdaBound{K}\AgdaSymbol{\}} \AgdaSymbol{→} \AgdaDatatype{Var} \AgdaBound{V} \AgdaBound{K} \AgdaSymbol{→} \AgdaFunction{VExpression} \AgdaBound{V} \AgdaBound{K}\<%
\\
\>[2]\AgdaIndent{4}{}\<[4]%
\>[4]\AgdaInductiveConstructor{app} \AgdaSymbol{:} \AgdaSymbol{∀} \AgdaSymbol{\{}\AgdaBound{AA}\AgdaSymbol{\}} \AgdaSymbol{\{}\AgdaBound{K}\AgdaSymbol{\}} \AgdaSymbol{→} \AgdaFunction{Constructor} \AgdaSymbol{(}\AgdaInductiveConstructor{SK} \AgdaBound{AA} \AgdaBound{K}\AgdaSymbol{)} \AgdaSymbol{→} \AgdaFunction{ListAbs} \AgdaBound{V} \AgdaBound{AA} \AgdaSymbol{→} \AgdaFunction{Expression} \AgdaBound{V} \AgdaBound{K}\<%
\\
\>[2]\AgdaIndent{4}{}\<[4]%
\>[4]\AgdaInductiveConstructor{[]} \AgdaSymbol{:} \AgdaFunction{ListAbs} \AgdaBound{V} \AgdaInductiveConstructor{[]}\<%
\\
\>[2]\AgdaIndent{4}{}\<[4]%
\>[4]\AgdaInductiveConstructor{\_∷\_} \AgdaSymbol{:} \AgdaSymbol{∀} \AgdaSymbol{\{}\AgdaBound{A}\AgdaSymbol{\}} \AgdaSymbol{\{}\AgdaBound{AA}\AgdaSymbol{\}} \AgdaSymbol{→} \AgdaFunction{Abstraction} \AgdaBound{V} \AgdaBound{A} \AgdaSymbol{→} \AgdaFunction{ListAbs} \AgdaBound{V} \AgdaBound{AA} \AgdaSymbol{→} \AgdaFunction{ListAbs} \AgdaBound{V} \AgdaSymbol{(}\AgdaBound{A} \AgdaInductiveConstructor{∷} \AgdaBound{AA}\AgdaSymbol{)}\<%
\end{code}
%</Expression>

We prove that the constructor \AgdaRef{var} is injective.

\begin{code}%
\>[0]\AgdaIndent{2}{}\<[2]%
\>[2]\AgdaFunction{var-inj} \AgdaSymbol{:} \AgdaSymbol{∀} \AgdaSymbol{\{}\AgdaBound{V}\AgdaSymbol{\}} \AgdaSymbol{\{}\AgdaBound{K}\AgdaSymbol{\}} \AgdaSymbol{\{}\AgdaBound{x} \AgdaBound{y} \AgdaSymbol{:} \AgdaDatatype{Var} \AgdaBound{V} \AgdaBound{K}\AgdaSymbol{\}} \AgdaSymbol{→} \AgdaInductiveConstructor{var} \AgdaBound{x} \AgdaDatatype{≡} \AgdaInductiveConstructor{var} \AgdaBound{y} \AgdaSymbol{→} \AgdaBound{x} \AgdaDatatype{≡} \AgdaBound{y}\<%
\\
\>[0]\AgdaIndent{2}{}\<[2]%
\>[2]\AgdaFunction{var-inj} \AgdaInductiveConstructor{refl} \AgdaSymbol{=} \AgdaInductiveConstructor{refl}\<%
\end{code}

For the future, we also define the type of all snoc-lists of expressions $(M_1, \ldots, M_n)$
such that $M_i$ is of type $K_i$, given a snoc-list of variable kinds $(K_1, \ldots, K_n)$.

\begin{code}%
\>[0]\AgdaIndent{2}{}\<[2]%
\>[2]\AgdaKeyword{infixl} \AgdaNumber{20} \AgdaFixityOp{\_snoc\_}\<%
\\
\>[0]\AgdaIndent{2}{}\<[2]%
\>[2]\AgdaKeyword{data} \AgdaDatatype{snocListExp} \AgdaBound{V} \AgdaSymbol{:} \AgdaDatatype{snocList} \AgdaFunction{VarKind} \AgdaSymbol{→} \AgdaPrimitiveType{Set} \AgdaKeyword{where}\<%
\\
\>[2]\AgdaIndent{4}{}\<[4]%
\>[4]\AgdaInductiveConstructor{[]} \AgdaSymbol{:} \AgdaDatatype{snocListExp} \AgdaBound{V} \AgdaInductiveConstructor{[]}\<%
\\
\>[2]\AgdaIndent{4}{}\<[4]%
\>[4]\AgdaInductiveConstructor{\_snoc\_} \AgdaSymbol{:} \AgdaSymbol{∀} \AgdaSymbol{\{}\AgdaBound{A}\AgdaSymbol{\}} \AgdaSymbol{\{}\AgdaBound{K}\AgdaSymbol{\}} \AgdaSymbol{→} \AgdaDatatype{snocListExp} \AgdaBound{V} \AgdaBound{A} \AgdaSymbol{→} \AgdaFunction{Expression} \AgdaBound{V} \AgdaSymbol{(}\AgdaInductiveConstructor{varKind} \AgdaBound{K}\AgdaSymbol{)} \AgdaSymbol{→} \AgdaDatatype{snocListExp} \AgdaBound{V} \AgdaSymbol{(}\AgdaBound{A} \AgdaInductiveConstructor{snoc} \AgdaBound{K}\AgdaSymbol{)}\<%
\end{code}

A \emph{reduction} is a relation $\rhd$ between expressions such that, if $E \rhd F$,
then $E$ is not a variable.  It is given by a term $R : \AgdaRef{Reduction}$ such that
$R\, c\, MM\, N$ iff $c[MM] \rhd N$.

\begin{code}%
\>[0]\AgdaIndent{2}{}\<[2]%
\>[2]\AgdaFunction{Reduction} \AgdaSymbol{:} \AgdaPrimitiveType{Set₁}\<%
\\
\>[0]\AgdaIndent{2}{}\<[2]%
\>[2]\AgdaFunction{Reduction} \AgdaSymbol{=} \AgdaSymbol{∀} \AgdaSymbol{\{}\AgdaBound{V}\AgdaSymbol{\}} \AgdaSymbol{\{}\AgdaBound{AA}\AgdaSymbol{\}} \AgdaSymbol{\{}\AgdaBound{K}\AgdaSymbol{\}} \AgdaSymbol{→} \AgdaFunction{Constructor} \AgdaSymbol{(}\AgdaInductiveConstructor{SK} \AgdaBound{AA} \AgdaBound{K}\AgdaSymbol{)} \AgdaSymbol{→} \AgdaFunction{ListAbs} \AgdaBound{V} \AgdaBound{AA} \AgdaSymbol{→} \AgdaFunction{Expression} \AgdaBound{V} \AgdaBound{K} \AgdaSymbol{→} \AgdaPrimitiveType{Set}\<%
\end{code}
}

\AgdaHide{
\begin{code}%
\>\AgdaKeyword{open} \AgdaKeyword{import} \AgdaModule{Grammar.Base}\<%
\\
%
\\
\>\AgdaKeyword{module} \AgdaModule{Reduction.SN} \AgdaSymbol{(}\AgdaBound{G} \AgdaSymbol{:} \AgdaRecord{Grammar}\AgdaSymbol{)} \AgdaSymbol{(}\AgdaBound{R} \AgdaSymbol{:} \AgdaFunction{Grammar.Reduction} \AgdaBound{G}\AgdaSymbol{)} \AgdaKeyword{where}\<%
\\
%
\\
\>\AgdaKeyword{open} \AgdaKeyword{import} \AgdaModule{Prelims}\<%
\\
\>\AgdaKeyword{open} \AgdaKeyword{import} \AgdaModule{Prelims.RTClosure}\<%
\\
\>\AgdaKeyword{open} \AgdaKeyword{import} \AgdaModule{Grammar} \AgdaBound{G}\<%
\\
\>\AgdaKeyword{open} \AgdaKeyword{import} \AgdaModule{Reduction.Base} \AgdaBound{G} \AgdaBound{R}\<%
\end{code}
}

\subsection{Strong Normalization}

The \emph{strongly normalizable} expressions are defined inductively as follows.

\begin{code}%
\>\AgdaKeyword{data} \AgdaDatatype{SN} \AgdaSymbol{\{}\AgdaBound{V} \AgdaBound{C} \AgdaBound{K}\AgdaSymbol{\}} \AgdaSymbol{:} \AgdaDatatype{Subexp} \AgdaBound{V} \AgdaBound{C} \AgdaBound{K} \AgdaSymbol{→} \AgdaPrimitiveType{Set} \AgdaKeyword{where}\<%
\\
\>[0]\AgdaIndent{2}{}\<[2]%
\>[2]\AgdaInductiveConstructor{SNI} \AgdaSymbol{:} \AgdaSymbol{∀} \AgdaBound{E} \AgdaSymbol{→} \AgdaSymbol{(∀} \AgdaBound{F} \AgdaSymbol{→} \AgdaBound{E} \AgdaDatatype{⇒} \AgdaBound{F} \AgdaSymbol{→} \AgdaDatatype{SN} \AgdaBound{F}\AgdaSymbol{)} \AgdaSymbol{→} \AgdaDatatype{SN} \AgdaBound{E}\<%
\end{code}

\begin{lemma}$ $
\begin{enumerate}
\item
If $c([\vec{x_1}]E_1, \ldots, [\vec{x_n}]E_n)$ is strongly normalizable, then each $E_i$ is strongly normalizable.

\begin{code}%
\>\AgdaFunction{SNapp'} \AgdaSymbol{:} \AgdaSymbol{∀} \AgdaSymbol{\{}\AgdaBound{V} \AgdaBound{K} \AgdaBound{AA}\AgdaSymbol{\}} \AgdaSymbol{\{}\AgdaBound{c} \AgdaSymbol{:} \AgdaFunction{Con} \AgdaSymbol{(}\AgdaInductiveConstructor{SK} \AgdaBound{AA} \AgdaBound{K}\AgdaSymbol{)\}} \AgdaSymbol{\{}\AgdaBound{E} \AgdaSymbol{:} \AgdaFunction{ListAbs} \AgdaBound{V} \AgdaBound{AA}\AgdaSymbol{\}} \AgdaSymbol{→} \AgdaDatatype{SN} \AgdaSymbol{(}\AgdaInductiveConstructor{app} \AgdaBound{c} \AgdaBound{E}\AgdaSymbol{)} \AgdaSymbol{→} \AgdaDatatype{SN} \AgdaBound{E}\<%
\end{code}

\AgdaHide{
\begin{code}%
\>\AgdaFunction{SNapp'} \AgdaSymbol{\{}\AgdaArgument{c} \AgdaSymbol{=} \AgdaBound{c}\AgdaSymbol{\}} \AgdaSymbol{\{}\AgdaArgument{E} \AgdaSymbol{=} \AgdaBound{E}\AgdaSymbol{\}} \AgdaSymbol{(}\AgdaInductiveConstructor{SNI} \AgdaSymbol{\_} \AgdaBound{SNcE}\AgdaSymbol{)} \AgdaSymbol{=} \AgdaInductiveConstructor{SNI} \AgdaBound{E} \AgdaSymbol{(λ} \AgdaBound{F} \AgdaBound{E→F} \AgdaSymbol{→} \AgdaFunction{SNapp'} \AgdaSymbol{(}\AgdaBound{SNcE} \AgdaSymbol{(}\AgdaInductiveConstructor{app} \AgdaBound{c} \AgdaBound{F}\AgdaSymbol{)} \AgdaSymbol{(}\AgdaInductiveConstructor{app} \AgdaBound{E→F}\AgdaSymbol{)))}\<%
\end{code}
}

\begin{code}%
\>\AgdaFunction{SNappl'} \AgdaSymbol{:} \AgdaSymbol{∀} \AgdaSymbol{\{}\AgdaBound{V} \AgdaBound{A} \AgdaBound{L} \AgdaBound{M} \AgdaBound{N}\AgdaSymbol{\}} \AgdaSymbol{→} \AgdaDatatype{SN} \AgdaSymbol{(}\AgdaInductiveConstructor{\_∷\_} \AgdaSymbol{\{}\AgdaBound{V}\AgdaSymbol{\}} \AgdaSymbol{\{}\AgdaBound{A}\AgdaSymbol{\}} \AgdaSymbol{\{}\AgdaBound{L}\AgdaSymbol{\}} \AgdaBound{M} \AgdaBound{N}\AgdaSymbol{)} \AgdaSymbol{→} \AgdaDatatype{SN} \AgdaBound{M}\<%
\end{code}

\AgdaHide{
\begin{code}%
\>\AgdaFunction{SNappl'} \AgdaSymbol{\{}\AgdaBound{V}\AgdaSymbol{\}} \AgdaSymbol{\{}\AgdaBound{A}\AgdaSymbol{\}} \AgdaSymbol{\{}\AgdaBound{L}\AgdaSymbol{\}} \AgdaSymbol{\{}\AgdaBound{M}\AgdaSymbol{\}} \AgdaSymbol{\{}\AgdaBound{N}\AgdaSymbol{\}} \AgdaSymbol{(}\AgdaInductiveConstructor{SNI} \AgdaSymbol{\_} \AgdaBound{SNM}\AgdaSymbol{)} \AgdaSymbol{=} \<[42]%
\>[42]\<%
\\
\>[0]\AgdaIndent{2}{}\<[2]%
\>[2]\AgdaInductiveConstructor{SNI} \AgdaBound{M} \AgdaSymbol{(λ} \AgdaBound{M'} \AgdaBound{M⇒M'} \AgdaSymbol{→} \AgdaFunction{SNappl'} \AgdaSymbol{(}\AgdaBound{SNM} \AgdaSymbol{(}\AgdaBound{M'} \AgdaInductiveConstructor{∷} \AgdaBound{N}\AgdaSymbol{)} \AgdaSymbol{(}\AgdaInductiveConstructor{appl} \AgdaBound{M⇒M'}\AgdaSymbol{)))}\<%
\end{code}
}

\begin{code}%
\>\AgdaFunction{SNappr'} \AgdaSymbol{:} \AgdaSymbol{∀} \AgdaSymbol{\{}\AgdaBound{V} \AgdaBound{A} \AgdaBound{L} \AgdaBound{M} \AgdaBound{N}\AgdaSymbol{\}} \AgdaSymbol{→} \AgdaDatatype{SN} \AgdaSymbol{(}\AgdaInductiveConstructor{\_∷\_} \AgdaSymbol{\{}\AgdaBound{V}\AgdaSymbol{\}} \AgdaSymbol{\{}\AgdaBound{A}\AgdaSymbol{\}} \AgdaSymbol{\{}\AgdaBound{L}\AgdaSymbol{\}} \AgdaBound{M} \AgdaBound{N}\AgdaSymbol{)} \AgdaSymbol{→} \AgdaDatatype{SN} \AgdaBound{N}\<%
\end{code}

\AgdaHide{
\begin{code}%
\>\AgdaFunction{SNappr'} \AgdaSymbol{\{}\AgdaBound{V}\AgdaSymbol{\}} \AgdaSymbol{\{}\AgdaBound{A}\AgdaSymbol{\}} \AgdaSymbol{\{}\AgdaBound{L}\AgdaSymbol{\}} \AgdaSymbol{\{}\AgdaBound{M}\AgdaSymbol{\}} \AgdaSymbol{\{}\AgdaBound{N}\AgdaSymbol{\}} \AgdaSymbol{(}\AgdaInductiveConstructor{SNI} \AgdaSymbol{\_} \AgdaBound{SNN}\AgdaSymbol{)} \AgdaSymbol{=} \<[42]%
\>[42]\<%
\\
\>[0]\AgdaIndent{2}{}\<[2]%
\>[2]\AgdaInductiveConstructor{SNI} \AgdaBound{N} \AgdaSymbol{(λ} \AgdaBound{N'} \AgdaBound{N⇒N'} \AgdaSymbol{→} \AgdaFunction{SNappr'} \AgdaSymbol{(}\AgdaBound{SNN} \AgdaSymbol{(}\AgdaBound{M} \AgdaInductiveConstructor{∷} \AgdaBound{N'}\AgdaSymbol{)} \AgdaSymbol{(}\AgdaInductiveConstructor{appr} \AgdaBound{N⇒N'}\AgdaSymbol{)))}\<%
\end{code}
}

\item
Let $F$ be a family of operations and $\sigma \in F$.
If $E[\sigma]$ is strongly normalizable and $R$ respects $F$ then $E$ is strongly normalizable.

\begin{code}%
\>\AgdaFunction{SNap'} \AgdaSymbol{:} \AgdaSymbol{∀} \AgdaSymbol{\{}\AgdaBound{Ops} \AgdaBound{U} \AgdaBound{V} \AgdaBound{C} \AgdaBound{K}\AgdaSymbol{\}} \AgdaSymbol{\{}\AgdaBound{E} \AgdaSymbol{:} \AgdaDatatype{Subexp} \AgdaBound{U} \AgdaBound{C} \AgdaBound{K}\AgdaSymbol{\}} \AgdaSymbol{\{}\AgdaBound{σ} \AgdaSymbol{:} \AgdaFunction{OpFamily.Op} \AgdaBound{Ops} \AgdaBound{U} \AgdaBound{V}\AgdaSymbol{\}} \AgdaSymbol{→}\<%
\\
\>[0]\AgdaIndent{2}{}\<[2]%
\>[2]\AgdaFunction{respects'} \AgdaBound{Ops} \AgdaSymbol{→} \AgdaDatatype{SN} \AgdaSymbol{(}\AgdaFunction{OpFamily.ap} \AgdaBound{Ops} \AgdaBound{σ} \AgdaBound{E}\AgdaSymbol{)} \AgdaSymbol{→} \AgdaDatatype{SN} \AgdaBound{E}\<%
\end{code}

\AgdaHide{
\begin{code}%
\>\AgdaFunction{SNap'} \AgdaSymbol{\{}\AgdaBound{Ops}\AgdaSymbol{\}} \AgdaSymbol{\{}\AgdaArgument{E} \AgdaSymbol{=} \AgdaBound{E}\AgdaSymbol{\}} \AgdaSymbol{\{}\AgdaArgument{σ} \AgdaSymbol{=} \AgdaBound{σ}\AgdaSymbol{\}} \AgdaBound{hyp} \AgdaSymbol{(}\AgdaInductiveConstructor{SNI} \AgdaSymbol{\_} \AgdaBound{SNσE}\AgdaSymbol{)} \AgdaSymbol{=} \<[47]%
\>[47]\<%
\\
\>[0]\AgdaIndent{2}{}\<[2]%
\>[2]\AgdaInductiveConstructor{SNI} \AgdaBound{E} \AgdaSymbol{(λ} \AgdaBound{F} \AgdaBound{E→F} \AgdaSymbol{→} \AgdaFunction{SNap'} \AgdaSymbol{\{}\AgdaBound{Ops}\AgdaSymbol{\}} \AgdaBound{hyp} \AgdaSymbol{(}\AgdaBound{SNσE} \AgdaSymbol{\_} \AgdaSymbol{(}\AgdaFunction{respects-osr} \AgdaBound{Ops} \AgdaBound{hyp} \AgdaBound{E→F}\AgdaSymbol{)))}\<%
\end{code}
}

\item
If $\rho$ is a replacement, $E$ is strongly normalizable and $R$ creates replacements then $E \langle \rho \rangle$ is strongly normalizable.

\begin{code}%
\>\AgdaFunction{SNrep} \AgdaSymbol{:} \AgdaSymbol{∀} \AgdaSymbol{\{}\AgdaBound{U} \AgdaBound{V} \AgdaBound{C} \AgdaBound{K}\AgdaSymbol{\}} \AgdaSymbol{\{}\AgdaBound{E} \AgdaSymbol{:} \AgdaDatatype{Subexp} \AgdaBound{U} \AgdaBound{C} \AgdaBound{K}\AgdaSymbol{\}} \AgdaSymbol{\{}\AgdaBound{σ} \AgdaSymbol{:} \AgdaFunction{Rep} \AgdaBound{U} \AgdaBound{V}\AgdaSymbol{\}} \AgdaSymbol{→}\<%
\\
\>[0]\AgdaIndent{2}{}\<[2]%
\>[2]\AgdaFunction{creates'} \AgdaFunction{REP} \AgdaSymbol{→} \AgdaDatatype{SN} \AgdaBound{E} \AgdaSymbol{→} \AgdaDatatype{SN} \AgdaSymbol{(}\AgdaBound{E} \AgdaFunction{〈} \AgdaBound{σ} \AgdaFunction{〉}\AgdaSymbol{)}\<%
\end{code}

\AgdaHide{
\begin{code}%
\>\AgdaFunction{SNrep} \AgdaSymbol{\{}\AgdaBound{U}\AgdaSymbol{\}} \AgdaSymbol{\{}\AgdaBound{V}\AgdaSymbol{\}} \AgdaSymbol{\{}\AgdaBound{C}\AgdaSymbol{\}} \AgdaSymbol{\{}\AgdaBound{K}\AgdaSymbol{\}} \AgdaSymbol{\{}\AgdaBound{E}\AgdaSymbol{\}} \AgdaSymbol{\{}\AgdaBound{σ}\AgdaSymbol{\}} \AgdaBound{hyp} \AgdaSymbol{(}\AgdaInductiveConstructor{SNI} \AgdaSymbol{.}\AgdaBound{E} \AgdaBound{SNE}\AgdaSymbol{)} \AgdaSymbol{=} \AgdaInductiveConstructor{SNI} \AgdaSymbol{(}\AgdaBound{E} \AgdaFunction{〈} \AgdaBound{σ} \AgdaFunction{〉}\AgdaSymbol{)} \AgdaSymbol{(λ} \AgdaBound{F} \AgdaBound{σE⇒F} \AgdaSymbol{→} \<[75]%
\>[75]\<%
\\
\>[0]\AgdaIndent{2}{}\<[2]%
\>[2]\AgdaKeyword{let} \AgdaKeyword{open} \AgdaModule{creation} \AgdaSymbol{\{}\AgdaArgument{Ops} \AgdaSymbol{=} \AgdaFunction{REP}\AgdaSymbol{\}} \AgdaSymbol{(}\AgdaFunction{create-osr} \AgdaBound{hyp} \AgdaBound{E} \AgdaBound{σE⇒F}\AgdaSymbol{)} \AgdaKeyword{in}\<%
\\
\>[0]\AgdaIndent{2}{}\<[2]%
\>[2]\AgdaFunction{subst} \AgdaDatatype{SN} \AgdaFunction{ap-created} \AgdaSymbol{(}\AgdaFunction{SNrep} \AgdaBound{hyp} \AgdaSymbol{(}\AgdaBound{SNE} \AgdaFunction{created} \AgdaFunction{red-created}\AgdaSymbol{)))}\<%
\end{code}
}

\item
If $E$ is strongly normalizable and $E \twoheadrightarrow_R F$ then $F$ is strongly normalizable.
\begin{code}%
\>\AgdaFunction{SNred} \AgdaSymbol{:} \AgdaSymbol{∀} \AgdaSymbol{\{}\AgdaBound{V} \AgdaBound{K}\AgdaSymbol{\}} \AgdaSymbol{\{}\AgdaBound{E} \AgdaBound{F} \AgdaSymbol{:} \AgdaFunction{Expression} \AgdaBound{V} \AgdaBound{K}\AgdaSymbol{\}} \AgdaSymbol{→} \AgdaDatatype{SN} \AgdaBound{E} \AgdaSymbol{→} \AgdaBound{E} \AgdaFunction{↠} \AgdaBound{F} \AgdaSymbol{→} \AgdaDatatype{SN} \AgdaBound{F}\<%
\end{code}

\AgdaHide{
\begin{code}%
\>\AgdaFunction{SNred} \AgdaSymbol{\{}\AgdaBound{V}\AgdaSymbol{\}} \AgdaSymbol{\{}\AgdaBound{K}\AgdaSymbol{\}} \AgdaSymbol{\{}\AgdaBound{E}\AgdaSymbol{\}} \AgdaSymbol{\{}\AgdaBound{F}\AgdaSymbol{\}} \AgdaSymbol{(}\AgdaInductiveConstructor{SNI} \AgdaSymbol{.}\AgdaBound{E} \AgdaBound{SNE}\AgdaSymbol{)} \AgdaSymbol{(}\AgdaInductiveConstructor{inc} \AgdaBound{E→F}\AgdaSymbol{)} \AgdaSymbol{=} \AgdaBound{SNE} \AgdaBound{F} \AgdaBound{E→F}\<%
\\
\>\AgdaFunction{SNred} \AgdaBound{SNE} \AgdaInductiveConstructor{ref} \AgdaSymbol{=} \AgdaBound{SNE}\<%
\\
\>\AgdaFunction{SNred} \AgdaBound{SNE} \AgdaSymbol{(}\AgdaInductiveConstructor{trans} \AgdaBound{E↠F} \AgdaBound{F↠G}\AgdaSymbol{)} \AgdaSymbol{=} \AgdaFunction{SNred} \AgdaSymbol{(}\AgdaFunction{SNred} \AgdaBound{SNE} \AgdaBound{E↠F}\AgdaSymbol{)} \AgdaBound{F↠G}\<%
\end{code}
}
\item
Every variable is strongly normalizing.
\begin{code}%
\>\AgdaFunction{SNvar} \AgdaSymbol{:} \AgdaSymbol{∀} \AgdaSymbol{\{}\AgdaBound{V}\AgdaSymbol{\}} \AgdaSymbol{\{}\AgdaBound{K}\AgdaSymbol{\}} \AgdaSymbol{(}\AgdaBound{x} \AgdaSymbol{:} \AgdaDatatype{Var} \AgdaBound{V} \AgdaBound{K}\AgdaSymbol{)} \AgdaSymbol{→} \AgdaDatatype{SN} \AgdaSymbol{(}\AgdaInductiveConstructor{var} \AgdaBound{x}\AgdaSymbol{)}\<%
\end{code}

\AgdaHide{
\begin{code}%
\>\AgdaFunction{SNvar} \AgdaBound{x} \AgdaSymbol{=} \AgdaInductiveConstructor{SNI} \AgdaSymbol{(}\AgdaInductiveConstructor{var} \AgdaBound{x}\AgdaSymbol{)} \AgdaSymbol{(λ} \AgdaBound{\_} \AgdaSymbol{())}\<%
\end{code}
}
\end{enumerate}
\end{lemma}


\newcommand{\id}[1]{\mathsf{id}_{#1}}

\section{Grammars}

\subsection{Taxonomy}

\AgdaHide{
\begin{code}%
\>\AgdaKeyword{module} \AgdaModule{Grammar.Taxonomy} \AgdaKeyword{where}\<%
\\
\>\AgdaKeyword{open} \AgdaKeyword{import} \AgdaModule{Data.List}\<%
\\
\>\AgdaKeyword{open} \AgdaKeyword{import} \AgdaModule{Prelims}\<%
\end{code}
}

Before we begin investigating the several theories we wish to consider, we present a general theory of syntax and
capture-avoiding substitution.

A \emph{taxononmy} consists of:
\begin{itemize}
\item a set of \emph{expression kinds};
\item a subset of expression kinds, called the \emph{variable kinds}.  We refer to the other expession kinds as \emph{non-variable kinds}.
\end{itemize}

%<*Taxonomy>
\begin{code}%
\>\AgdaKeyword{record} \AgdaRecord{Taxonomy} \AgdaSymbol{:} \AgdaPrimitiveType{Set₁} \AgdaKeyword{where}\<%
\\
\>[0]\AgdaIndent{2}{}\<[2]%
\>[2]\AgdaKeyword{field}\<%
\\
\>[2]\AgdaIndent{4}{}\<[4]%
\>[4]\AgdaField{VarKind} \AgdaSymbol{:} \AgdaPrimitiveType{Set}\<%
\\
\>[2]\AgdaIndent{4}{}\<[4]%
\>[4]\AgdaField{NonVarKind} \AgdaSymbol{:} \AgdaPrimitiveType{Set}\<%
\\
%
\\
\>[0]\AgdaIndent{2}{}\<[2]%
\>[2]\AgdaKeyword{data} \AgdaDatatype{ExpressionKind} \AgdaSymbol{:} \AgdaPrimitiveType{Set} \AgdaKeyword{where}\<%
\\
\>[2]\AgdaIndent{4}{}\<[4]%
\>[4]\AgdaInductiveConstructor{varKind} \AgdaSymbol{:} \AgdaField{VarKind} \AgdaSymbol{→} \AgdaDatatype{ExpressionKind}\<%
\\
\>[2]\AgdaIndent{4}{}\<[4]%
\>[4]\AgdaInductiveConstructor{nonVarKind} \AgdaSymbol{:} \AgdaField{NonVarKind} \AgdaSymbol{→} \AgdaDatatype{ExpressionKind}\<%
\end{code}
%</Taxonomy>

\begin{frame}[fragile]
\frametitle{Alphabets}
An \emph{alphabet} $A$ consists of a finite set of \emph{variables},
\mode<article>{to each of which is assigned a variable kind $K$.
Let $\emptyset$ be the empty alphabet, and $(A , K)$ be the result of extending the alphabet $A$ with one
fresh variable $x₀$ of kind $K$.  We write $\mathsf{Var}\ A\ K$ for the set of all variables in $A$ of kind $K$.}
\mode<beamer>{each with a variable kind.}

\begin{code}%
\>[0]\AgdaIndent{2}{}\<[2]%
\>[2]\AgdaKeyword{infixl} \AgdaNumber{55} \AgdaFixityOp{\_,\_}\<%
\\
\>[0]\AgdaIndent{2}{}\<[2]%
\>[2]\AgdaKeyword{data} \AgdaDatatype{Alphabet} \AgdaSymbol{:} \AgdaPrimitiveType{Set} \AgdaKeyword{where}\<%
\\
\>[2]\AgdaIndent{4}{}\<[4]%
\>[4]\AgdaInductiveConstructor{∅} \AgdaSymbol{:} \AgdaDatatype{Alphabet}\<%
\\
\>[2]\AgdaIndent{4}{}\<[4]%
\>[4]\AgdaInductiveConstructor{\_,\_} \AgdaSymbol{:} \AgdaDatatype{Alphabet} \AgdaSymbol{→} \AgdaField{VarKind} \AgdaSymbol{→} \AgdaDatatype{Alphabet}\<%
\end{code}

\AgdaHide{
\begin{code}%
\>[0]\AgdaIndent{2}{}\<[2]%
\>[2]\AgdaFunction{extend} \AgdaSymbol{:} \AgdaDatatype{Alphabet} \AgdaSymbol{→} \AgdaDatatype{List} \AgdaField{VarKind} \AgdaSymbol{→} \AgdaDatatype{Alphabet}\<%
\\
\>[0]\AgdaIndent{2}{}\<[2]%
\>[2]\AgdaFunction{extend} \AgdaBound{A} \AgdaInductiveConstructor{[]} \AgdaSymbol{=} \AgdaBound{A}\<%
\\
\>[0]\AgdaIndent{2}{}\<[2]%
\>[2]\AgdaFunction{extend} \AgdaBound{A} \AgdaSymbol{(}\AgdaBound{K} \AgdaInductiveConstructor{∷} \AgdaBound{KK}\AgdaSymbol{)} \AgdaSymbol{=} \AgdaFunction{extend} \AgdaSymbol{(}\AgdaBound{A} \AgdaInductiveConstructor{,} \AgdaBound{K}\AgdaSymbol{)} \AgdaBound{KK}\<%
\\
%
\\
\>[0]\AgdaIndent{2}{}\<[2]%
\>[2]\AgdaFunction{snoc-extend} \AgdaSymbol{:} \AgdaDatatype{Alphabet} \AgdaSymbol{→} \AgdaDatatype{snocList} \AgdaField{VarKind} \AgdaSymbol{→} \AgdaDatatype{Alphabet}\<%
\\
\>[0]\AgdaIndent{2}{}\<[2]%
\>[2]\AgdaFunction{snoc-extend} \AgdaBound{A} \AgdaInductiveConstructor{[]} \AgdaSymbol{=} \AgdaBound{A}\<%
\\
\>[0]\AgdaIndent{2}{}\<[2]%
\>[2]\AgdaFunction{snoc-extend} \AgdaBound{A} \AgdaSymbol{(}\AgdaBound{KK} \AgdaInductiveConstructor{snoc} \AgdaBound{K}\AgdaSymbol{)} \AgdaSymbol{=} \AgdaFunction{snoc-extend} \AgdaBound{A} \AgdaBound{KK} \AgdaInductiveConstructor{,} \AgdaBound{K}\<%
\end{code}
}

\begin{code}%
\>[0]\AgdaIndent{2}{}\<[2]%
\>[2]\AgdaKeyword{data} \AgdaDatatype{Var} \AgdaSymbol{:} \AgdaDatatype{Alphabet} \AgdaSymbol{→} \AgdaField{VarKind} \AgdaSymbol{→} \AgdaPrimitiveType{Set} \AgdaKeyword{where}\<%
\\
\>[2]\AgdaIndent{4}{}\<[4]%
\>[4]\AgdaInductiveConstructor{x₀} \AgdaSymbol{:} \AgdaSymbol{∀} \AgdaSymbol{\{}\AgdaBound{V}\AgdaSymbol{\}} \AgdaSymbol{\{}\AgdaBound{K}\AgdaSymbol{\}} \AgdaSymbol{→} \AgdaDatatype{Var} \AgdaSymbol{(}\AgdaBound{V} \AgdaInductiveConstructor{,} \AgdaBound{K}\AgdaSymbol{)} \AgdaBound{K}\<%
\\
\>[2]\AgdaIndent{4}{}\<[4]%
\>[4]\AgdaInductiveConstructor{↑} \AgdaSymbol{:} \AgdaSymbol{∀} \AgdaSymbol{\{}\AgdaBound{V}\AgdaSymbol{\}} \AgdaSymbol{\{}\AgdaBound{K}\AgdaSymbol{\}} \AgdaSymbol{\{}\AgdaBound{L}\AgdaSymbol{\}} \AgdaSymbol{→} \AgdaDatatype{Var} \AgdaBound{V} \AgdaBound{L} \AgdaSymbol{→} \AgdaDatatype{Var} \AgdaSymbol{(}\AgdaBound{V} \AgdaInductiveConstructor{,} \AgdaBound{K}\AgdaSymbol{)} \AgdaBound{L}\<%
\end{code}

\AgdaHide{
\begin{code}%
\>[0]\AgdaIndent{2}{}\<[2]%
\>[2]\AgdaFunction{x₁} \AgdaSymbol{:} \AgdaSymbol{∀} \AgdaSymbol{\{}\AgdaBound{V}\AgdaSymbol{\}} \AgdaSymbol{\{}\AgdaBound{K}\AgdaSymbol{\}} \AgdaSymbol{\{}\AgdaBound{L}\AgdaSymbol{\}} \AgdaSymbol{→} \AgdaDatatype{Var} \AgdaSymbol{(}\AgdaBound{V} \AgdaInductiveConstructor{,} \AgdaBound{K} \AgdaInductiveConstructor{,} \AgdaBound{L}\AgdaSymbol{)} \AgdaBound{K}\<%
\end{code}
}

\begin{code}%
\>[0]\AgdaIndent{2}{}\<[2]%
\>[2]\AgdaFunction{x₁} \AgdaSymbol{=} \AgdaInductiveConstructor{↑} \AgdaInductiveConstructor{x₀}\<%
\end{code}

\AgdaHide{
\begin{code}%
\>[0]\AgdaIndent{2}{}\<[2]%
\>[2]\AgdaFunction{x₂} \AgdaSymbol{:} \AgdaSymbol{∀} \AgdaSymbol{\{}\AgdaBound{V}\AgdaSymbol{\}} \AgdaSymbol{\{}\AgdaBound{K}\AgdaSymbol{\}} \AgdaSymbol{\{}\AgdaBound{L}\AgdaSymbol{\}} \AgdaSymbol{\{}\AgdaBound{L'}\AgdaSymbol{\}} \AgdaSymbol{→} \AgdaDatatype{Var} \AgdaSymbol{(}\AgdaBound{V} \AgdaInductiveConstructor{,} \AgdaBound{K} \AgdaInductiveConstructor{,} \AgdaBound{L} \AgdaInductiveConstructor{,} \AgdaBound{L'}\AgdaSymbol{)} \AgdaBound{K}\<%
\end{code}
}

\begin{code}%
\>[0]\AgdaIndent{2}{}\<[2]%
\>[2]\AgdaFunction{x₂} \AgdaSymbol{=} \AgdaInductiveConstructor{↑} \AgdaFunction{x₁}\<%
\end{code}
\end{frame}

A constructor $c$ of kind (\ref{eq:conkind}) is a constructor that takes $m$ arguments of kind $B_1$, \ldots, $B_m$, and binds $r_i$ variables in its $i$th argument of kind $A_{ij}$,
producing an expression of kind $C$.  We write this expression as

\begin{equation}
\label{eq:expression}
c([x_{11}, \ldots, x_{1r_1}]E_1, \ldots, [x_{m1}, \ldots, x_{mr_m}]E_m) \enspace .
\end{equation}

The subexpressions of the form $[x_{i1}, \ldots, x_{ir_i}]E_i$ shall be called \emph{abstractions}.

When giving a specific grammar, we shall feel free to use BNF notation.  

We formalise this as follows.  First, we construct the sets of expression kinds and constructor kinds over a taxonomy:

\begin{frame}[fragile]
There are two \emph{classes} of kinds: expression kinds and constructor kinds.

\begin{code}%
\>[0]\AgdaIndent{2}{}\<[2]%
\>[2]\AgdaKeyword{data} \AgdaDatatype{KindClass} \AgdaSymbol{:} \AgdaPrimitiveType{Set} \AgdaKeyword{where}\<%
\\
\>[2]\AgdaIndent{4}{}\<[4]%
\>[4]\AgdaInductiveConstructor{-Expression} \AgdaSymbol{:} \AgdaDatatype{KindClass}\<%
\\
\>[2]\AgdaIndent{4}{}\<[4]%
\>[4]\AgdaInductiveConstructor{-Constructor} \AgdaSymbol{:} \AgdaDatatype{ExpressionKind} \AgdaSymbol{→} \AgdaDatatype{KindClass}\<%
\\
%
\\
\>[0]\AgdaIndent{2}{}\<[2]%
\>[2]\AgdaKeyword{data} \AgdaDatatype{Kind} \AgdaSymbol{:} \AgdaDatatype{KindClass} \AgdaSymbol{→} \AgdaPrimitiveType{Set} \AgdaKeyword{where}\<%
\\
\>[2]\AgdaIndent{4}{}\<[4]%
\>[4]\AgdaInductiveConstructor{base} \AgdaSymbol{:} \AgdaDatatype{ExpressionKind} \AgdaSymbol{→} \AgdaDatatype{Kind} \AgdaInductiveConstructor{-Expression}\<%
\\
%
\\
\>[2]\AgdaIndent{4}{}\<[4]%
\>[4]\AgdaInductiveConstructor{out} \<[9]%
\>[9]\AgdaSymbol{:} \AgdaSymbol{∀} \AgdaBound{K} \AgdaSymbol{→} \AgdaDatatype{Kind} \AgdaSymbol{(}\AgdaInductiveConstructor{-Constructor} \AgdaBound{K}\AgdaSymbol{)}\<%
\\
\>[2]\AgdaIndent{4}{}\<[4]%
\>[4]\AgdaInductiveConstructor{Π} \<[9]%
\>[9]\AgdaSymbol{:} \AgdaSymbol{∀} \AgdaSymbol{\{}\AgdaBound{K}\AgdaSymbol{\}} \AgdaSymbol{→} \AgdaDatatype{List} \AgdaField{VarKind} \AgdaSymbol{→} \AgdaDatatype{ExpressionKind} \AgdaSymbol{→} \<[51]%
\>[51]\<%
\\
\>[4]\AgdaIndent{11}{}\<[11]%
\>[11]\AgdaDatatype{Kind} \AgdaSymbol{(}\AgdaInductiveConstructor{-Constructor} \AgdaBound{K}\AgdaSymbol{)} \AgdaSymbol{→} \AgdaDatatype{Kind} \AgdaSymbol{(}\AgdaInductiveConstructor{-Constructor} \AgdaBound{K}\AgdaSymbol{)}\<%
\end{code}
\end{frame}
%TODO Colours in Agda code?

\AgdaHide{
\begin{code}%
\>\AgdaKeyword{open} \AgdaKeyword{import} \AgdaModule{Grammar.Taxonomy} \AgdaKeyword{public}\<%
\end{code}
}

\AgdaHide{
\begin{code}%
\>\AgdaComment{\{- Metavariable conventions:\<\\
\>  A, B    range over abstraction kinds\<\\
\>  C       range over kind classes\<\\
\>  AA, BB  range over lists of abstraction kinds\<\\
\>  E, F, G range over subexpressions\<\\
\>  K, L    range over expression kinds including variable kinds\<\\
\>  M, N, P range over expressions\<\\
\>  U, V, W range over alphabets -\}}\<%
\\
\>\AgdaKeyword{open} \AgdaKeyword{import} \AgdaModule{Function}\<%
\\
\>\AgdaKeyword{open} \AgdaKeyword{import} \AgdaModule{Data.List}\<%
\\
\>\AgdaKeyword{open} \AgdaKeyword{import} \AgdaModule{Prelims}\<%
\\
\>\AgdaKeyword{open} \AgdaKeyword{import} \AgdaModule{Grammar.Taxonomy}\<%
\\
%
\\
\>\AgdaKeyword{module} \AgdaModule{Grammar.Base} \AgdaKeyword{where}\<%
\\
%
\\
\>\AgdaKeyword{record} \AgdaRecord{IsGrammar} \AgdaSymbol{(}\AgdaBound{T} \AgdaSymbol{:} \AgdaRecord{Taxonomy}\AgdaSymbol{)} \AgdaSymbol{:} \AgdaPrimitiveType{Set₁} \AgdaKeyword{where}\<%
\\
\>[0]\AgdaIndent{2}{}\<[2]%
\>[2]\AgdaKeyword{open} \AgdaModule{Taxonomy} \AgdaBound{T}\<%
\\
\>[0]\AgdaIndent{2}{}\<[2]%
\>[2]\AgdaKeyword{field}\<%
\\
\>[2]\AgdaIndent{4}{}\<[4]%
\>[4]\AgdaField{Constructor} \<[19]%
\>[19]\AgdaSymbol{:} \AgdaFunction{ConKind} \AgdaSymbol{→} \AgdaPrimitiveType{Set}\<%
\\
\>[2]\AgdaIndent{4}{}\<[4]%
\>[4]\AgdaField{parent} \<[19]%
\>[19]\AgdaSymbol{:} \AgdaFunction{VarKind} \AgdaSymbol{→} \AgdaDatatype{ExpKind}\<%
\\
%
\\
\>\AgdaKeyword{record} \AgdaRecord{Grammar} \AgdaSymbol{:} \AgdaPrimitiveType{Set₁} \AgdaKeyword{where}\<%
\\
\>[0]\AgdaIndent{2}{}\<[2]%
\>[2]\AgdaKeyword{field}\<%
\\
\>[2]\AgdaIndent{4}{}\<[4]%
\>[4]\AgdaField{taxonomy} \AgdaSymbol{:} \AgdaRecord{Taxonomy}\<%
\\
\>[2]\AgdaIndent{4}{}\<[4]%
\>[4]\AgdaField{isGrammar} \AgdaSymbol{:} \AgdaRecord{IsGrammar} \AgdaField{taxonomy}\<%
\\
\>[0]\AgdaIndent{2}{}\<[2]%
\>[2]\AgdaKeyword{open} \AgdaModule{Taxonomy} \AgdaField{taxonomy} \AgdaKeyword{public}\<%
\\
\>[0]\AgdaIndent{2}{}\<[2]%
\>[2]\AgdaKeyword{open} \AgdaModule{IsGrammar} \AgdaField{isGrammar} \AgdaKeyword{public}\<%
\end{code}
}

%<*Expression>
\begin{code}%
\>[0]\AgdaIndent{2}{}\<[2]%
\>[2]\AgdaKeyword{data} \AgdaDatatype{Subexpression} \AgdaSymbol{(}\AgdaBound{V} \AgdaSymbol{:} \AgdaDatatype{Alphabet}\AgdaSymbol{)} \AgdaSymbol{:} \AgdaSymbol{∀} \AgdaBound{C} \AgdaSymbol{→} \AgdaFunction{Kind} \AgdaBound{C} \AgdaSymbol{→} \AgdaPrimitiveType{Set}\<%
\\
\>[0]\AgdaIndent{2}{}\<[2]%
\>[2]\AgdaFunction{Expression} \AgdaSymbol{:} \AgdaDatatype{Alphabet} \AgdaSymbol{→} \AgdaDatatype{ExpKind} \AgdaSymbol{→} \AgdaPrimitiveType{Set}\<%
\\
\>[0]\AgdaIndent{2}{}\<[2]%
\>[2]\AgdaFunction{VExpression} \AgdaSymbol{:} \AgdaDatatype{Alphabet} \AgdaSymbol{→} \AgdaFunction{VarKind} \AgdaSymbol{→} \AgdaPrimitiveType{Set}\<%
\\
\>[0]\AgdaIndent{2}{}\<[2]%
\>[2]\AgdaFunction{Abstraction} \AgdaSymbol{:} \AgdaDatatype{Alphabet} \AgdaSymbol{→} \AgdaFunction{AbsKind} \AgdaSymbol{→} \AgdaPrimitiveType{Set}\<%
\\
\>[0]\AgdaIndent{2}{}\<[2]%
\>[2]\AgdaFunction{ListAbs} \AgdaSymbol{:} \AgdaDatatype{Alphabet} \AgdaSymbol{→} \AgdaDatatype{List} \AgdaFunction{AbsKind} \AgdaSymbol{→} \AgdaPrimitiveType{Set}\<%
\\
%
\\
\>[0]\AgdaIndent{2}{}\<[2]%
\>[2]\AgdaFunction{Expression} \AgdaBound{V} \AgdaBound{K} \AgdaSymbol{=} \AgdaDatatype{Subexpression} \AgdaBound{V} \AgdaInductiveConstructor{-Expression} \AgdaBound{K}\<%
\\
\>[0]\AgdaIndent{2}{}\<[2]%
\>[2]\AgdaFunction{VExpression} \AgdaBound{V} \AgdaBound{K} \AgdaSymbol{=} \AgdaFunction{Expression} \AgdaBound{V} \AgdaSymbol{(}\AgdaInductiveConstructor{varKind} \AgdaBound{K}\AgdaSymbol{)}\<%
\\
\>[0]\AgdaIndent{2}{}\<[2]%
\>[2]\AgdaFunction{Abstraction} \AgdaBound{V} \AgdaSymbol{(}\AgdaInductiveConstructor{SK} \AgdaBound{AA} \AgdaBound{K}\AgdaSymbol{)} \AgdaSymbol{=} \AgdaFunction{Expression} \AgdaSymbol{(}\AgdaFunction{extend} \AgdaBound{V} \AgdaBound{AA}\AgdaSymbol{)} \AgdaBound{K}\<%
\\
\>[0]\AgdaIndent{2}{}\<[2]%
\>[2]\AgdaFunction{ListAbs} \AgdaBound{V} \AgdaBound{AA} \AgdaSymbol{=} \AgdaDatatype{Subexpression} \AgdaBound{V} \AgdaInductiveConstructor{-ListAbs} \AgdaBound{AA}\<%
\\
%
\\
\>[0]\AgdaIndent{2}{}\<[2]%
\>[2]\AgdaKeyword{infixr} \AgdaNumber{5} \AgdaFixityOp{\_∷\_}\<%
\\
\>[0]\AgdaIndent{2}{}\<[2]%
\>[2]\AgdaKeyword{data} \AgdaDatatype{Subexpression} \AgdaBound{V} \AgdaKeyword{where}\<%
\\
\>[2]\AgdaIndent{4}{}\<[4]%
\>[4]\AgdaInductiveConstructor{var} \AgdaSymbol{:} \AgdaSymbol{∀} \AgdaSymbol{\{}\AgdaBound{K}\AgdaSymbol{\}} \AgdaSymbol{→} \AgdaDatatype{Var} \AgdaBound{V} \AgdaBound{K} \AgdaSymbol{→} \AgdaFunction{VExpression} \AgdaBound{V} \AgdaBound{K}\<%
\\
\>[2]\AgdaIndent{4}{}\<[4]%
\>[4]\AgdaInductiveConstructor{app} \AgdaSymbol{:} \AgdaSymbol{∀} \AgdaSymbol{\{}\AgdaBound{AA}\AgdaSymbol{\}} \AgdaSymbol{\{}\AgdaBound{K}\AgdaSymbol{\}} \AgdaSymbol{→} \AgdaFunction{Constructor} \AgdaSymbol{(}\AgdaInductiveConstructor{SK} \AgdaBound{AA} \AgdaBound{K}\AgdaSymbol{)} \AgdaSymbol{→} \AgdaFunction{ListAbs} \AgdaBound{V} \AgdaBound{AA} \AgdaSymbol{→} \AgdaFunction{Expression} \AgdaBound{V} \AgdaBound{K}\<%
\\
\>[2]\AgdaIndent{4}{}\<[4]%
\>[4]\AgdaInductiveConstructor{[]} \AgdaSymbol{:} \AgdaFunction{ListAbs} \AgdaBound{V} \AgdaInductiveConstructor{[]}\<%
\\
\>[2]\AgdaIndent{4}{}\<[4]%
\>[4]\AgdaInductiveConstructor{\_∷\_} \AgdaSymbol{:} \AgdaSymbol{∀} \AgdaSymbol{\{}\AgdaBound{A}\AgdaSymbol{\}} \AgdaSymbol{\{}\AgdaBound{AA}\AgdaSymbol{\}} \AgdaSymbol{→} \AgdaFunction{Abstraction} \AgdaBound{V} \AgdaBound{A} \AgdaSymbol{→} \AgdaFunction{ListAbs} \AgdaBound{V} \AgdaBound{AA} \AgdaSymbol{→} \AgdaFunction{ListAbs} \AgdaBound{V} \AgdaSymbol{(}\AgdaBound{A} \AgdaInductiveConstructor{∷} \AgdaBound{AA}\AgdaSymbol{)}\<%
\end{code}
%</Expression>

We prove that the constructor \AgdaRef{var} is injective.

\begin{code}%
\>[0]\AgdaIndent{2}{}\<[2]%
\>[2]\AgdaFunction{var-inj} \AgdaSymbol{:} \AgdaSymbol{∀} \AgdaSymbol{\{}\AgdaBound{V}\AgdaSymbol{\}} \AgdaSymbol{\{}\AgdaBound{K}\AgdaSymbol{\}} \AgdaSymbol{\{}\AgdaBound{x} \AgdaBound{y} \AgdaSymbol{:} \AgdaDatatype{Var} \AgdaBound{V} \AgdaBound{K}\AgdaSymbol{\}} \AgdaSymbol{→} \AgdaInductiveConstructor{var} \AgdaBound{x} \AgdaDatatype{≡} \AgdaInductiveConstructor{var} \AgdaBound{y} \AgdaSymbol{→} \AgdaBound{x} \AgdaDatatype{≡} \AgdaBound{y}\<%
\\
\>[0]\AgdaIndent{2}{}\<[2]%
\>[2]\AgdaFunction{var-inj} \AgdaInductiveConstructor{refl} \AgdaSymbol{=} \AgdaInductiveConstructor{refl}\<%
\end{code}

For the future, we also define the type of all snoc-lists of expressions $(M_1, \ldots, M_n)$
such that $M_i$ is of type $K_i$, given a snoc-list of variable kinds $(K_1, \ldots, K_n)$.

\begin{code}%
\>[0]\AgdaIndent{2}{}\<[2]%
\>[2]\AgdaKeyword{infixl} \AgdaNumber{20} \AgdaFixityOp{\_snoc\_}\<%
\\
\>[0]\AgdaIndent{2}{}\<[2]%
\>[2]\AgdaKeyword{data} \AgdaDatatype{snocListExp} \AgdaBound{V} \AgdaSymbol{:} \AgdaDatatype{snocList} \AgdaFunction{VarKind} \AgdaSymbol{→} \AgdaPrimitiveType{Set} \AgdaKeyword{where}\<%
\\
\>[2]\AgdaIndent{4}{}\<[4]%
\>[4]\AgdaInductiveConstructor{[]} \AgdaSymbol{:} \AgdaDatatype{snocListExp} \AgdaBound{V} \AgdaInductiveConstructor{[]}\<%
\\
\>[2]\AgdaIndent{4}{}\<[4]%
\>[4]\AgdaInductiveConstructor{\_snoc\_} \AgdaSymbol{:} \AgdaSymbol{∀} \AgdaSymbol{\{}\AgdaBound{A}\AgdaSymbol{\}} \AgdaSymbol{\{}\AgdaBound{K}\AgdaSymbol{\}} \AgdaSymbol{→} \AgdaDatatype{snocListExp} \AgdaBound{V} \AgdaBound{A} \AgdaSymbol{→} \AgdaFunction{Expression} \AgdaBound{V} \AgdaSymbol{(}\AgdaInductiveConstructor{varKind} \AgdaBound{K}\AgdaSymbol{)} \AgdaSymbol{→} \AgdaDatatype{snocListExp} \AgdaBound{V} \AgdaSymbol{(}\AgdaBound{A} \AgdaInductiveConstructor{snoc} \AgdaBound{K}\AgdaSymbol{)}\<%
\end{code}

A \emph{reduction} is a relation $\rhd$ between expressions such that, if $E \rhd F$,
then $E$ is not a variable.  It is given by a term $R : \AgdaRef{Reduction}$ such that
$R\, c\, MM\, N$ iff $c[MM] \rhd N$.

\begin{code}%
\>[0]\AgdaIndent{2}{}\<[2]%
\>[2]\AgdaFunction{Reduction} \AgdaSymbol{:} \AgdaPrimitiveType{Set₁}\<%
\\
\>[0]\AgdaIndent{2}{}\<[2]%
\>[2]\AgdaFunction{Reduction} \AgdaSymbol{=} \AgdaSymbol{∀} \AgdaSymbol{\{}\AgdaBound{V}\AgdaSymbol{\}} \AgdaSymbol{\{}\AgdaBound{AA}\AgdaSymbol{\}} \AgdaSymbol{\{}\AgdaBound{K}\AgdaSymbol{\}} \AgdaSymbol{→} \AgdaFunction{Constructor} \AgdaSymbol{(}\AgdaInductiveConstructor{SK} \AgdaBound{AA} \AgdaBound{K}\AgdaSymbol{)} \AgdaSymbol{→} \AgdaFunction{ListAbs} \AgdaBound{V} \AgdaBound{AA} \AgdaSymbol{→} \AgdaFunction{Expression} \AgdaBound{V} \AgdaBound{K} \AgdaSymbol{→} \AgdaPrimitiveType{Set}\<%
\end{code}
}

\AgdaHide{
\begin{code}%
\>\AgdaKeyword{open} \AgdaKeyword{import} \AgdaModule{Grammar.Base} \AgdaKeyword{public}\<%
\\
%
\\
\>\AgdaKeyword{module} \AgdaModule{Grammar} \AgdaSymbol{(}\AgdaBound{G} \AgdaSymbol{:} \AgdaRecord{Grammar}\AgdaSymbol{)} \AgdaKeyword{where}\<%
\\
\>\AgdaKeyword{open} \AgdaModule{Grammar} \AgdaBound{G} \AgdaKeyword{public}\<%
\end{code}
}

\AgdaHide{
\begin{code}%
\>\AgdaKeyword{open} \AgdaKeyword{import} \AgdaModule{Grammar.Base}\<%
\\
%
\\
\>\AgdaKeyword{module} \AgdaModule{Grammar.Substitution.OpFamily} \AgdaSymbol{(}\AgdaBound{G} \AgdaSymbol{:} \AgdaRecord{Grammar}\AgdaSymbol{)} \AgdaKeyword{where}\<%
\\
\>\AgdaKeyword{open} \AgdaKeyword{import} \AgdaModule{Prelims}\<%
\\
\>\AgdaKeyword{open} \AgdaModule{Grammar} \AgdaBound{G}\<%
\\
\>\AgdaKeyword{open} \AgdaKeyword{import} \AgdaModule{Grammar.OpFamily} \AgdaBound{G}\<%
\\
\>\AgdaKeyword{open} \AgdaKeyword{import} \AgdaModule{Grammar.Replacement} \AgdaBound{G}\<%
\\
\>\AgdaKeyword{open} \AgdaKeyword{import} \AgdaModule{Grammar.Substitution.PreOpFamily} \AgdaBound{G}\<%
\\
\>\AgdaKeyword{open} \AgdaKeyword{import} \AgdaModule{Grammar.Substitution.Lifting} \AgdaBound{G}\<%
\\
\>\AgdaKeyword{open} \AgdaKeyword{import} \AgdaModule{Grammar.Substitution.LiftFamily} \AgdaBound{G}\<%
\\
\>\AgdaKeyword{open} \AgdaKeyword{import} \AgdaModule{Grammar.Substitution.RepSub} \AgdaBound{G}\<%
\end{code}
}

We now define two compositions $\bullet_1 : \mathrm{replacement} ; \mathrm{substitution} \rightarrow \mathrm{substitution}$ and $\bullet_2 : \mathrm{substitution};\mathrm{replacement} \rightarrow \mathrm{substitution}$.

\begin{code}%
\>\AgdaKeyword{infixl} \AgdaNumber{60} \AgdaFixityOp{\_•RS\_}\<%
\\
\>\AgdaFunction{\_•RS\_} \AgdaSymbol{:} \AgdaSymbol{∀} \AgdaSymbol{\{}\AgdaBound{U}\AgdaSymbol{\}} \AgdaSymbol{\{}\AgdaBound{V}\AgdaSymbol{\}} \AgdaSymbol{\{}\AgdaBound{W}\AgdaSymbol{\}} \AgdaSymbol{→} \AgdaFunction{Rep} \AgdaBound{V} \AgdaBound{W} \AgdaSymbol{→} \AgdaFunction{Sub} \AgdaBound{U} \AgdaBound{V} \AgdaSymbol{→} \AgdaFunction{Sub} \AgdaBound{U} \AgdaBound{W}\<%
\\
\>\AgdaSymbol{(}\AgdaBound{ρ} \AgdaFunction{•RS} \AgdaBound{σ}\AgdaSymbol{)} \AgdaBound{K} \AgdaBound{x} \AgdaSymbol{=} \AgdaSymbol{(}\AgdaBound{σ} \AgdaBound{K} \AgdaBound{x}\AgdaSymbol{)} \AgdaFunction{〈} \AgdaBound{ρ} \AgdaFunction{〉}\<%
\\
%
\\
\>\AgdaFunction{Sub↑-compRS} \AgdaSymbol{:} \AgdaSymbol{∀} \AgdaSymbol{\{}\AgdaBound{U}\AgdaSymbol{\}} \AgdaSymbol{\{}\AgdaBound{V}\AgdaSymbol{\}} \AgdaSymbol{\{}\AgdaBound{W}\AgdaSymbol{\}} \AgdaSymbol{\{}\AgdaBound{K}\AgdaSymbol{\}} \AgdaSymbol{\{}\AgdaBound{ρ} \AgdaSymbol{:} \AgdaFunction{Rep} \AgdaBound{V} \AgdaBound{W}\AgdaSymbol{\}} \AgdaSymbol{\{}\AgdaBound{σ} \AgdaSymbol{:} \AgdaFunction{Sub} \AgdaBound{U} \AgdaBound{V}\AgdaSymbol{\}} \AgdaSymbol{→} \AgdaFunction{Sub↑} \AgdaBound{K} \AgdaSymbol{(}\AgdaBound{ρ} \AgdaFunction{•RS} \AgdaBound{σ}\AgdaSymbol{)} \AgdaFunction{∼} \AgdaFunction{Rep↑} \AgdaBound{K} \AgdaBound{ρ} \AgdaFunction{•RS} \AgdaFunction{Sub↑} \AgdaBound{K} \AgdaBound{σ}\<%
\end{code}

\AgdaHide{
\begin{code}%
\>\AgdaFunction{Sub↑-compRS} \AgdaSymbol{\{}\AgdaArgument{K} \AgdaSymbol{=} \AgdaBound{K}\AgdaSymbol{\}} \AgdaInductiveConstructor{x₀} \AgdaSymbol{=} \AgdaInductiveConstructor{refl}\<%
\\
\>\AgdaFunction{Sub↑-compRS} \AgdaSymbol{\{}\AgdaBound{U}\AgdaSymbol{\}} \AgdaSymbol{\{}\AgdaBound{V}\AgdaSymbol{\}} \AgdaSymbol{\{}\AgdaBound{W}\AgdaSymbol{\}} \AgdaSymbol{\{}\AgdaBound{K}\AgdaSymbol{\}} \AgdaSymbol{\{}\AgdaBound{ρ}\AgdaSymbol{\}} \AgdaSymbol{\{}\AgdaBound{σ}\AgdaSymbol{\}} \AgdaSymbol{\{}\AgdaBound{L}\AgdaSymbol{\}} \AgdaSymbol{(}\AgdaInductiveConstructor{↑} \AgdaBound{x}\AgdaSymbol{)} \AgdaSymbol{=} \AgdaKeyword{let} \AgdaKeyword{open} \AgdaModule{≡-Reasoning} \AgdaSymbol{\{}\AgdaArgument{A} \AgdaSymbol{=} \AgdaFunction{Expression} \AgdaSymbol{(}\AgdaBound{W} \AgdaInductiveConstructor{,} \AgdaBound{K}\AgdaSymbol{)} \AgdaSymbol{(}\AgdaInductiveConstructor{varKind} \AgdaBound{L}\AgdaSymbol{)\}} \AgdaKeyword{in} \<[109]%
\>[109]\<%
\\
\>[0]\AgdaIndent{2}{}\<[2]%
\>[2]\AgdaFunction{begin} \<[8]%
\>[8]\<%
\\
\>[2]\AgdaIndent{4}{}\<[4]%
\>[4]\AgdaSymbol{(}\AgdaBound{σ} \AgdaBound{L} \AgdaBound{x}\AgdaSymbol{)} \AgdaFunction{〈} \AgdaBound{ρ} \AgdaFunction{〉} \AgdaFunction{〈} \AgdaFunction{upRep} \AgdaFunction{〉}\<%
\\
\>[0]\AgdaIndent{2}{}\<[2]%
\>[2]\AgdaFunction{≡⟨⟨} \AgdaFunction{rep-comp} \AgdaSymbol{(}\AgdaBound{σ} \AgdaBound{L} \AgdaBound{x}\AgdaSymbol{)} \AgdaFunction{⟩⟩}\<%
\\
\>[2]\AgdaIndent{4}{}\<[4]%
\>[4]\AgdaSymbol{(}\AgdaBound{σ} \AgdaBound{L} \AgdaBound{x}\AgdaSymbol{)} \AgdaFunction{〈} \AgdaFunction{upRep} \AgdaFunction{•R} \AgdaBound{ρ} \AgdaFunction{〉}\<%
\\
\>[0]\AgdaIndent{2}{}\<[2]%
\>[2]\AgdaFunction{≡⟨⟩}\<%
\\
\>[2]\AgdaIndent{4}{}\<[4]%
\>[4]\AgdaSymbol{(}\AgdaBound{σ} \AgdaBound{L} \AgdaBound{x}\AgdaSymbol{)} \AgdaFunction{〈} \AgdaFunction{Rep↑} \AgdaBound{K} \AgdaBound{ρ} \AgdaFunction{•R} \AgdaFunction{upRep} \AgdaFunction{〉}\<%
\\
\>[0]\AgdaIndent{2}{}\<[2]%
\>[2]\AgdaFunction{≡⟨} \AgdaFunction{rep-comp} \AgdaSymbol{(}\AgdaBound{σ} \AgdaBound{L} \AgdaBound{x}\AgdaSymbol{)} \AgdaFunction{⟩}\<%
\\
\>[2]\AgdaIndent{4}{}\<[4]%
\>[4]\AgdaSymbol{(}\AgdaBound{σ} \AgdaBound{L} \AgdaBound{x}\AgdaSymbol{)} \AgdaFunction{〈} \AgdaFunction{upRep} \AgdaFunction{〉} \AgdaFunction{〈} \AgdaFunction{Rep↑} \AgdaBound{K} \AgdaBound{ρ} \AgdaFunction{〉}\<%
\\
\>[0]\AgdaIndent{2}{}\<[2]%
\>[2]\AgdaFunction{∎}\<%
\end{code}
}

\begin{code}%
\>\AgdaFunction{COMPRS} \AgdaSymbol{:} \AgdaRecord{Composition} \AgdaFunction{proto-replacement} \AgdaFunction{proto-substitution} \AgdaFunction{proto-substitution}\<%
\\
\>\AgdaFunction{COMPRS} \AgdaSymbol{=} \AgdaKeyword{record} \AgdaSymbol{\{} \<[18]%
\>[18]\<%
\\
\>[0]\AgdaIndent{2}{}\<[2]%
\>[2]\AgdaField{circ} \AgdaSymbol{=} \AgdaFunction{\_•RS\_} \AgdaSymbol{;} \<[17]%
\>[17]\<%
\\
\>[0]\AgdaIndent{2}{}\<[2]%
\>[2]\AgdaField{liftOp-circ} \AgdaSymbol{=} \AgdaFunction{Sub↑-compRS} \AgdaSymbol{;} \<[30]%
\>[30]\<%
\\
\>[0]\AgdaIndent{2}{}\<[2]%
\>[2]\AgdaField{apV-circ} \AgdaSymbol{=} \AgdaInductiveConstructor{refl} \AgdaSymbol{\}}\<%
\\
%
\\
\>\AgdaFunction{sub-compRS} \AgdaSymbol{:} \AgdaSymbol{∀} \AgdaSymbol{\{}\AgdaBound{U}\AgdaSymbol{\}} \AgdaSymbol{\{}\AgdaBound{V}\AgdaSymbol{\}} \AgdaSymbol{\{}\AgdaBound{W}\AgdaSymbol{\}} \AgdaSymbol{\{}\AgdaBound{C}\AgdaSymbol{\}} \AgdaSymbol{\{}\AgdaBound{K}\AgdaSymbol{\}} \<[35]%
\>[35]\<%
\\
\>[0]\AgdaIndent{2}{}\<[2]%
\>[2]\AgdaSymbol{(}\AgdaBound{E} \AgdaSymbol{:} \AgdaDatatype{Subexpression} \AgdaBound{U} \AgdaBound{C} \AgdaBound{K}\AgdaSymbol{)} \AgdaSymbol{\{}\AgdaBound{ρ} \AgdaSymbol{:} \AgdaFunction{Rep} \AgdaBound{V} \AgdaBound{W}\AgdaSymbol{\}} \AgdaSymbol{\{}\AgdaBound{σ} \AgdaSymbol{:} \AgdaFunction{Sub} \AgdaBound{U} \AgdaBound{V}\AgdaSymbol{\}} \AgdaSymbol{→}\<%
\\
\>[0]\AgdaIndent{2}{}\<[2]%
\>[2]\AgdaBound{E} \AgdaFunction{⟦} \AgdaBound{ρ} \AgdaFunction{•RS} \AgdaBound{σ} \AgdaFunction{⟧} \AgdaDatatype{≡} \AgdaBound{E} \AgdaFunction{⟦} \AgdaBound{σ} \AgdaFunction{⟧} \AgdaFunction{〈} \AgdaBound{ρ} \AgdaFunction{〉}\<%
\\
\>\AgdaFunction{sub-compRS} \AgdaBound{E} \AgdaSymbol{=} \AgdaFunction{Composition.ap-circ} \AgdaFunction{COMPRS} \AgdaBound{E}\<%
\\
%
\\
\>\AgdaKeyword{infixl} \AgdaNumber{60} \AgdaFixityOp{\_•SR\_}\<%
\\
\>\AgdaFunction{\_•SR\_} \AgdaSymbol{:} \AgdaSymbol{∀} \AgdaSymbol{\{}\AgdaBound{U}\AgdaSymbol{\}} \AgdaSymbol{\{}\AgdaBound{V}\AgdaSymbol{\}} \AgdaSymbol{\{}\AgdaBound{W}\AgdaSymbol{\}} \AgdaSymbol{→} \AgdaFunction{Sub} \AgdaBound{V} \AgdaBound{W} \AgdaSymbol{→} \AgdaFunction{Rep} \AgdaBound{U} \AgdaBound{V} \AgdaSymbol{→} \AgdaFunction{Sub} \AgdaBound{U} \AgdaBound{W}\<%
\\
\>\AgdaSymbol{(}\AgdaBound{σ} \AgdaFunction{•SR} \AgdaBound{ρ}\AgdaSymbol{)} \AgdaBound{K} \AgdaBound{x} \AgdaSymbol{=} \AgdaBound{σ} \AgdaBound{K} \AgdaSymbol{(}\AgdaBound{ρ} \AgdaBound{K} \AgdaBound{x}\AgdaSymbol{)}\<%
\\
%
\\
\>\AgdaFunction{Sub↑-compSR} \AgdaSymbol{:} \AgdaSymbol{∀} \AgdaSymbol{\{}\AgdaBound{U}\AgdaSymbol{\}} \AgdaSymbol{\{}\AgdaBound{V}\AgdaSymbol{\}} \AgdaSymbol{\{}\AgdaBound{W}\AgdaSymbol{\}} \AgdaSymbol{\{}\AgdaBound{K}\AgdaSymbol{\}} \AgdaSymbol{\{}\AgdaBound{σ} \AgdaSymbol{:} \AgdaFunction{Sub} \AgdaBound{V} \AgdaBound{W}\AgdaSymbol{\}} \AgdaSymbol{\{}\AgdaBound{ρ} \AgdaSymbol{:} \AgdaFunction{Rep} \AgdaBound{U} \AgdaBound{V}\AgdaSymbol{\}} \AgdaSymbol{→} \<[62]%
\>[62]\<%
\\
\>[0]\AgdaIndent{2}{}\<[2]%
\>[2]\AgdaFunction{Sub↑} \AgdaBound{K} \AgdaSymbol{(}\AgdaBound{σ} \AgdaFunction{•SR} \AgdaBound{ρ}\AgdaSymbol{)} \AgdaFunction{∼} \AgdaFunction{Sub↑} \AgdaBound{K} \AgdaBound{σ} \AgdaFunction{•SR} \AgdaFunction{Rep↑} \AgdaBound{K} \AgdaBound{ρ}\<%
\end{code}

\AgdaHide{
\begin{code}%
\>\AgdaFunction{Sub↑-compSR} \AgdaSymbol{\{}\AgdaArgument{K} \AgdaSymbol{=} \AgdaBound{K}\AgdaSymbol{\}} \AgdaInductiveConstructor{x₀} \AgdaSymbol{=} \AgdaInductiveConstructor{refl}\<%
\\
\>\AgdaFunction{Sub↑-compSR} \AgdaSymbol{(}\AgdaInductiveConstructor{↑} \AgdaBound{x}\AgdaSymbol{)} \AgdaSymbol{=} \AgdaInductiveConstructor{refl}\<%
\end{code}
}

\begin{code}%
\>\AgdaFunction{COMPSR} \AgdaSymbol{:} \AgdaRecord{Composition} \AgdaFunction{proto-substitution} \AgdaFunction{proto-replacement} \AgdaFunction{proto-substitution}\<%
\\
\>\AgdaFunction{COMPSR} \AgdaSymbol{=} \AgdaKeyword{record} \AgdaSymbol{\{} \<[18]%
\>[18]\<%
\\
\>[0]\AgdaIndent{2}{}\<[2]%
\>[2]\AgdaField{circ} \AgdaSymbol{=} \AgdaFunction{\_•SR\_} \AgdaSymbol{;} \<[17]%
\>[17]\<%
\\
\>[0]\AgdaIndent{2}{}\<[2]%
\>[2]\AgdaField{liftOp-circ} \AgdaSymbol{=} \AgdaFunction{Sub↑-compSR} \AgdaSymbol{;} \<[30]%
\>[30]\<%
\\
\>[0]\AgdaIndent{2}{}\<[2]%
\>[2]\AgdaField{apV-circ} \AgdaSymbol{=} \AgdaInductiveConstructor{refl} \AgdaSymbol{\}}\<%
\\
%
\\
\>\AgdaFunction{sub-compSR} \AgdaSymbol{:} \AgdaSymbol{∀} \AgdaSymbol{\{}\AgdaBound{U}\AgdaSymbol{\}} \AgdaSymbol{\{}\AgdaBound{V}\AgdaSymbol{\}} \AgdaSymbol{\{}\AgdaBound{W}\AgdaSymbol{\}} \AgdaSymbol{\{}\AgdaBound{C}\AgdaSymbol{\}} \AgdaSymbol{\{}\AgdaBound{K}\AgdaSymbol{\}} \<[35]%
\>[35]\<%
\\
\>[0]\AgdaIndent{2}{}\<[2]%
\>[2]\AgdaSymbol{(}\AgdaBound{E} \AgdaSymbol{:} \AgdaDatatype{Subexpression} \AgdaBound{U} \AgdaBound{C} \AgdaBound{K}\AgdaSymbol{)} \AgdaSymbol{\{}\AgdaBound{σ} \AgdaSymbol{:} \AgdaFunction{Sub} \AgdaBound{V} \AgdaBound{W}\AgdaSymbol{\}} \AgdaSymbol{\{}\AgdaBound{ρ} \AgdaSymbol{:} \AgdaFunction{Rep} \AgdaBound{U} \AgdaBound{V}\AgdaSymbol{\}} \AgdaSymbol{→} \<[58]%
\>[58]\<%
\\
\>[0]\AgdaIndent{2}{}\<[2]%
\>[2]\AgdaBound{E} \AgdaFunction{⟦} \AgdaBound{σ} \AgdaFunction{•SR} \AgdaBound{ρ} \AgdaFunction{⟧} \AgdaDatatype{≡} \AgdaBound{E} \AgdaFunction{〈} \AgdaBound{ρ} \AgdaFunction{〉} \AgdaFunction{⟦} \AgdaBound{σ} \AgdaFunction{⟧}\<%
\end{code}

\AgdaHide{
\begin{code}%
\>\AgdaFunction{sub-compSR} \AgdaBound{E} \AgdaSymbol{=} \AgdaFunction{Composition.ap-circ} \AgdaFunction{COMPSR} \AgdaBound{E}\<%
\end{code}
}

\begin{code}%
\>\AgdaFunction{Sub↑-upRep} \AgdaSymbol{:} \AgdaSymbol{∀} \AgdaSymbol{\{}\AgdaBound{U}\AgdaSymbol{\}} \AgdaSymbol{\{}\AgdaBound{V}\AgdaSymbol{\}} \AgdaSymbol{\{}\AgdaBound{C}\AgdaSymbol{\}} \AgdaSymbol{\{}\AgdaBound{K}\AgdaSymbol{\}} \AgdaSymbol{\{}\AgdaBound{L}\AgdaSymbol{\}} \AgdaSymbol{(}\AgdaBound{E} \AgdaSymbol{:} \AgdaDatatype{Subexpression} \AgdaBound{U} \AgdaBound{C} \AgdaBound{K}\AgdaSymbol{)} \AgdaSymbol{\{}\AgdaBound{σ} \AgdaSymbol{:} \AgdaFunction{Sub} \AgdaBound{U} \AgdaBound{V}\AgdaSymbol{\}} \AgdaSymbol{→}\<%
\\
\>[0]\AgdaIndent{2}{}\<[2]%
\>[2]\AgdaBound{E} \AgdaFunction{〈} \AgdaFunction{upRep} \AgdaFunction{〉} \AgdaFunction{⟦} \AgdaFunction{Sub↑} \AgdaBound{L} \AgdaBound{σ} \AgdaFunction{⟧} \AgdaDatatype{≡} \AgdaBound{E} \AgdaFunction{⟦} \AgdaBound{σ} \AgdaFunction{⟧} \AgdaFunction{〈} \AgdaFunction{upRep} \AgdaFunction{〉}\<%
\end{code}

\AgdaHide{
\begin{code}%
\>\AgdaFunction{Sub↑-upRep} \AgdaBound{E} \AgdaSymbol{=} \AgdaFunction{liftOp-up-mixed} \AgdaFunction{COMPSR} \AgdaFunction{COMPRS} \AgdaSymbol{(λ} \AgdaSymbol{\{}\AgdaBound{\_}\AgdaSymbol{\}} \AgdaSymbol{\{}\AgdaBound{\_}\AgdaSymbol{\}} \AgdaSymbol{\{}\AgdaBound{\_}\AgdaSymbol{\}} \AgdaSymbol{\{}\AgdaBound{\_}\AgdaSymbol{\}} \AgdaSymbol{\{}\AgdaBound{E}\AgdaSymbol{\}} \AgdaSymbol{→} \AgdaFunction{sym} \AgdaSymbol{(}\AgdaFunction{up-is-up'} \AgdaSymbol{\{}\AgdaArgument{E} \AgdaSymbol{=} \AgdaBound{E}\AgdaSymbol{\}))} \AgdaSymbol{\{}\AgdaBound{E}\AgdaSymbol{\}}\<%
\end{code}
}

Composition is defined by $(\sigma \circ \rho)(x) \equiv \rho(x) [ \sigma ]$.

\begin{code}%
\>\AgdaKeyword{infixl} \AgdaNumber{60} \AgdaFixityOp{\_•\_}\<%
\\
\>\AgdaFunction{\_•\_} \AgdaSymbol{:} \AgdaSymbol{∀} \AgdaSymbol{\{}\AgdaBound{U}\AgdaSymbol{\}} \AgdaSymbol{\{}\AgdaBound{V}\AgdaSymbol{\}} \AgdaSymbol{\{}\AgdaBound{W}\AgdaSymbol{\}} \AgdaSymbol{→} \AgdaFunction{Sub} \AgdaBound{V} \AgdaBound{W} \AgdaSymbol{→} \AgdaFunction{Sub} \AgdaBound{U} \AgdaBound{V} \AgdaSymbol{→} \AgdaFunction{Sub} \AgdaBound{U} \AgdaBound{W}\<%
\\
\>\AgdaSymbol{(}\AgdaBound{σ} \AgdaFunction{•} \AgdaBound{ρ}\AgdaSymbol{)} \AgdaBound{K} \AgdaBound{x} \AgdaSymbol{=} \AgdaBound{ρ} \AgdaBound{K} \AgdaBound{x} \AgdaFunction{⟦} \AgdaBound{σ} \AgdaFunction{⟧}\<%
\end{code}

Using the fact that $\bullet_1$ and $\bullet_2$ are compositions, we are
able to prove that this is a composition $\mathrm{substitution} ; \mathrm{substitution} \rightarrow \mathrm{substitution}$, and hence that substitution is a family of operations.

\begin{code}%
\>\AgdaFunction{Sub↑-comp} \AgdaSymbol{:} \AgdaSymbol{∀} \AgdaSymbol{\{}\AgdaBound{U}\AgdaSymbol{\}} \AgdaSymbol{\{}\AgdaBound{V}\AgdaSymbol{\}} \AgdaSymbol{\{}\AgdaBound{W}\AgdaSymbol{\}} \AgdaSymbol{\{}\AgdaBound{ρ} \AgdaSymbol{:} \AgdaFunction{Sub} \AgdaBound{U} \AgdaBound{V}\AgdaSymbol{\}} \AgdaSymbol{\{}\AgdaBound{σ} \AgdaSymbol{:} \AgdaFunction{Sub} \AgdaBound{V} \AgdaBound{W}\AgdaSymbol{\}} \AgdaSymbol{\{}\AgdaBound{K}\AgdaSymbol{\}} \AgdaSymbol{→} \<[60]%
\>[60]\<%
\\
\>[0]\AgdaIndent{2}{}\<[2]%
\>[2]\AgdaFunction{Sub↑} \AgdaBound{K} \AgdaSymbol{(}\AgdaBound{σ} \AgdaFunction{•} \AgdaBound{ρ}\AgdaSymbol{)} \AgdaFunction{∼} \AgdaFunction{Sub↑} \AgdaBound{K} \AgdaBound{σ} \AgdaFunction{•} \AgdaFunction{Sub↑} \AgdaBound{K} \AgdaBound{ρ}\<%
\end{code}

\AgdaHide{
\begin{code}%
\>\AgdaFunction{Sub↑-comp} \AgdaInductiveConstructor{x₀} \AgdaSymbol{=} \AgdaInductiveConstructor{refl}\<%
\\
\>\AgdaFunction{Sub↑-comp} \AgdaSymbol{\{}\AgdaArgument{W} \AgdaSymbol{=} \AgdaBound{W}\AgdaSymbol{\}} \AgdaSymbol{\{}\AgdaArgument{ρ} \AgdaSymbol{=} \AgdaBound{ρ}\AgdaSymbol{\}} \AgdaSymbol{\{}\AgdaArgument{σ} \AgdaSymbol{=} \AgdaBound{σ}\AgdaSymbol{\}} \AgdaSymbol{\{}\AgdaArgument{K} \AgdaSymbol{=} \AgdaBound{K}\AgdaSymbol{\}} \AgdaSymbol{\{}\AgdaBound{L}\AgdaSymbol{\}} \AgdaSymbol{(}\AgdaInductiveConstructor{↑} \AgdaBound{x}\AgdaSymbol{)} \AgdaSymbol{=} \AgdaFunction{sym} \AgdaSymbol{(}\AgdaFunction{Sub↑-upRep} \AgdaSymbol{(}\AgdaBound{ρ} \AgdaBound{L} \AgdaBound{x}\AgdaSymbol{))}\<%
\\
%
\\
\>\AgdaFunction{Sub↑-upRep₂} \AgdaSymbol{:} \AgdaSymbol{∀} \AgdaSymbol{\{}\AgdaBound{U}\AgdaSymbol{\}} \AgdaSymbol{\{}\AgdaBound{V}\AgdaSymbol{\}} \AgdaSymbol{\{}\AgdaBound{C}\AgdaSymbol{\}} \AgdaSymbol{\{}\AgdaBound{K}\AgdaSymbol{\}} \AgdaSymbol{\{}\AgdaBound{L}\AgdaSymbol{\}} \AgdaSymbol{\{}\AgdaBound{M}\AgdaSymbol{\}} \AgdaSymbol{(}\AgdaBound{E} \AgdaSymbol{:} \AgdaDatatype{Subexpression} \AgdaBound{U} \AgdaBound{C} \AgdaBound{M}\AgdaSymbol{)} \AgdaSymbol{\{}\AgdaBound{σ} \AgdaSymbol{:} \AgdaFunction{Sub} \AgdaBound{U} \AgdaBound{V}\AgdaSymbol{\}} \AgdaSymbol{→} \AgdaBound{E} \AgdaFunction{⇑} \AgdaFunction{⇑} \AgdaFunction{⟦} \AgdaFunction{Sub↑} \AgdaBound{K} \AgdaSymbol{(}\AgdaFunction{Sub↑} \AgdaBound{L} \AgdaBound{σ}\AgdaSymbol{)} \AgdaFunction{⟧} \AgdaDatatype{≡} \AgdaBound{E} \AgdaFunction{⟦} \AgdaBound{σ} \AgdaFunction{⟧} \AgdaFunction{⇑} \AgdaFunction{⇑}\<%
\\
\>\AgdaFunction{Sub↑-upRep₂} \AgdaSymbol{\{}\AgdaBound{U}\AgdaSymbol{\}} \AgdaSymbol{\{}\AgdaBound{V}\AgdaSymbol{\}} \AgdaSymbol{\{}\AgdaBound{C}\AgdaSymbol{\}} \AgdaSymbol{\{}\AgdaBound{K}\AgdaSymbol{\}} \AgdaSymbol{\{}\AgdaBound{L}\AgdaSymbol{\}} \AgdaSymbol{\{}\AgdaBound{M}\AgdaSymbol{\}} \AgdaBound{E} \AgdaSymbol{\{}\AgdaBound{σ}\AgdaSymbol{\}} \AgdaSymbol{=} \AgdaKeyword{let} \AgdaKeyword{open} \AgdaModule{≡-Reasoning} \AgdaKeyword{in} \<[68]%
\>[68]\<%
\\
\>[0]\AgdaIndent{2}{}\<[2]%
\>[2]\AgdaFunction{begin}\<%
\\
\>[2]\AgdaIndent{4}{}\<[4]%
\>[4]\AgdaBound{E} \AgdaFunction{⇑} \AgdaFunction{⇑} \AgdaFunction{⟦} \AgdaFunction{Sub↑} \AgdaBound{K} \AgdaSymbol{(}\AgdaFunction{Sub↑} \AgdaBound{L} \AgdaBound{σ}\AgdaSymbol{)} \AgdaFunction{⟧}\<%
\\
\>[0]\AgdaIndent{2}{}\<[2]%
\>[2]\AgdaFunction{≡⟨} \AgdaFunction{Sub↑-upRep} \AgdaSymbol{(}\AgdaBound{E} \AgdaFunction{⇑}\AgdaSymbol{)} \AgdaFunction{⟩}\<%
\\
\>[2]\AgdaIndent{4}{}\<[4]%
\>[4]\AgdaBound{E} \AgdaFunction{⇑} \AgdaFunction{⟦} \AgdaFunction{Sub↑} \AgdaBound{L} \AgdaBound{σ} \AgdaFunction{⟧} \AgdaFunction{⇑}\<%
\\
\>[0]\AgdaIndent{2}{}\<[2]%
\>[2]\AgdaFunction{≡⟨} \AgdaFunction{rep-congl} \AgdaSymbol{(}\AgdaFunction{Sub↑-upRep} \AgdaBound{E}\AgdaSymbol{)} \AgdaFunction{⟩}\<%
\\
\>[2]\AgdaIndent{4}{}\<[4]%
\>[4]\AgdaBound{E} \AgdaFunction{⟦} \AgdaBound{σ} \AgdaFunction{⟧} \AgdaFunction{⇑} \AgdaFunction{⇑}\<%
\\
\>[0]\AgdaIndent{2}{}\<[2]%
\>[2]\AgdaFunction{∎}\<%
\\
%
\\
\>\AgdaFunction{Sub↑-upRep₃} \AgdaSymbol{:} \AgdaSymbol{∀} \AgdaSymbol{\{}\AgdaBound{U}\AgdaSymbol{\}} \AgdaSymbol{\{}\AgdaBound{V}\AgdaSymbol{\}} \AgdaSymbol{\{}\AgdaBound{C}\AgdaSymbol{\}} \AgdaSymbol{\{}\AgdaBound{K}\AgdaSymbol{\}} \AgdaSymbol{\{}\AgdaBound{L}\AgdaSymbol{\}} \AgdaSymbol{\{}\AgdaBound{M}\AgdaSymbol{\}} \AgdaSymbol{\{}\AgdaBound{N}\AgdaSymbol{\}} \AgdaSymbol{(}\AgdaBound{E} \AgdaSymbol{:} \AgdaDatatype{Subexpression} \AgdaBound{U} \AgdaBound{C} \AgdaBound{N}\AgdaSymbol{)} \AgdaSymbol{\{}\AgdaBound{σ} \AgdaSymbol{:} \AgdaFunction{Sub} \AgdaBound{U} \AgdaBound{V}\AgdaSymbol{\}} \AgdaSymbol{→} \AgdaBound{E} \AgdaFunction{⇑} \AgdaFunction{⇑} \AgdaFunction{⇑} \AgdaFunction{⟦} \AgdaFunction{Sub↑} \AgdaBound{K} \AgdaSymbol{(}\AgdaFunction{Sub↑} \AgdaBound{L} \AgdaSymbol{(}\AgdaFunction{Sub↑} \AgdaBound{M} \AgdaBound{σ}\AgdaSymbol{))} \AgdaFunction{⟧} \AgdaDatatype{≡} \AgdaBound{E} \AgdaFunction{⟦} \AgdaBound{σ} \AgdaFunction{⟧} \AgdaFunction{⇑} \AgdaFunction{⇑} \AgdaFunction{⇑}\<%
\\
\>\AgdaFunction{Sub↑-upRep₃} \AgdaSymbol{\{}\AgdaBound{U}\AgdaSymbol{\}} \AgdaSymbol{\{}\AgdaBound{V}\AgdaSymbol{\}} \AgdaSymbol{\{}\AgdaBound{C}\AgdaSymbol{\}} \AgdaSymbol{\{}\AgdaBound{K}\AgdaSymbol{\}} \AgdaSymbol{\{}\AgdaBound{L}\AgdaSymbol{\}} \AgdaSymbol{\{}\AgdaBound{M}\AgdaSymbol{\}} \AgdaSymbol{\{}\AgdaBound{N}\AgdaSymbol{\}} \AgdaBound{E} \AgdaSymbol{\{}\AgdaBound{σ}\AgdaSymbol{\}} \AgdaSymbol{=} \AgdaKeyword{let} \AgdaKeyword{open} \AgdaModule{≡-Reasoning} \AgdaKeyword{in} \<[72]%
\>[72]\<%
\\
\>[0]\AgdaIndent{2}{}\<[2]%
\>[2]\AgdaFunction{begin}\<%
\\
\>[2]\AgdaIndent{4}{}\<[4]%
\>[4]\AgdaBound{E} \AgdaFunction{⇑} \AgdaFunction{⇑} \AgdaFunction{⇑} \AgdaFunction{⟦} \AgdaFunction{Sub↑} \AgdaBound{K} \AgdaSymbol{(}\AgdaFunction{Sub↑} \AgdaBound{L} \AgdaSymbol{(}\AgdaFunction{Sub↑} \AgdaBound{M} \AgdaBound{σ}\AgdaSymbol{))} \AgdaFunction{⟧}\<%
\\
\>[0]\AgdaIndent{2}{}\<[2]%
\>[2]\AgdaFunction{≡⟨} \AgdaFunction{Sub↑-upRep₂} \AgdaSymbol{(}\AgdaBound{E} \AgdaFunction{⇑}\AgdaSymbol{)} \AgdaFunction{⟩}\<%
\\
\>[2]\AgdaIndent{4}{}\<[4]%
\>[4]\AgdaBound{E} \AgdaFunction{⇑} \AgdaFunction{⟦} \AgdaFunction{Sub↑} \AgdaBound{M} \AgdaBound{σ} \AgdaFunction{⟧} \AgdaFunction{⇑} \AgdaFunction{⇑}\<%
\\
\>[0]\AgdaIndent{2}{}\<[2]%
\>[2]\AgdaFunction{≡⟨} \AgdaFunction{rep-congl} \AgdaSymbol{(}\AgdaFunction{rep-congl} \AgdaSymbol{(}\AgdaFunction{Sub↑-upRep} \AgdaBound{E}\AgdaSymbol{))} \AgdaFunction{⟩}\<%
\\
\>[2]\AgdaIndent{4}{}\<[4]%
\>[4]\AgdaBound{E} \AgdaFunction{⟦} \AgdaBound{σ} \AgdaFunction{⟧} \AgdaFunction{⇑} \AgdaFunction{⇑} \AgdaFunction{⇑}\<%
\\
\>[0]\AgdaIndent{2}{}\<[2]%
\>[2]\AgdaFunction{∎}\<%
\\
%
\\
\>\AgdaFunction{Rep↑-Sub↑-upRep₃} \AgdaSymbol{:} \AgdaSymbol{∀} \AgdaSymbol{\{}\AgdaBound{U}\AgdaSymbol{\}} \AgdaSymbol{\{}\AgdaBound{V}\AgdaSymbol{\}} \AgdaSymbol{\{}\AgdaBound{W}\AgdaSymbol{\}} \AgdaSymbol{\{}\AgdaBound{K1}\AgdaSymbol{\}} \AgdaSymbol{\{}\AgdaBound{K2}\AgdaSymbol{\}} \AgdaSymbol{\{}\AgdaBound{K3}\AgdaSymbol{\}} \AgdaSymbol{\{}\AgdaBound{C}\AgdaSymbol{\}} \AgdaSymbol{\{}\AgdaBound{K4}\AgdaSymbol{\}} \<[57]%
\>[57]\<%
\\
\>[2]\AgdaIndent{19}{}\<[19]%
\>[19]\AgdaSymbol{(}\AgdaBound{M} \AgdaSymbol{:} \AgdaDatatype{Subexpression} \AgdaBound{U} \AgdaBound{C} \AgdaBound{K4}\AgdaSymbol{)}\<%
\\
\>[2]\AgdaIndent{19}{}\<[19]%
\>[19]\AgdaSymbol{(}\AgdaBound{σ} \AgdaSymbol{:} \AgdaFunction{Sub} \AgdaBound{U} \AgdaBound{V}\AgdaSymbol{)} \AgdaSymbol{(}\AgdaBound{ρ} \AgdaSymbol{:} \AgdaFunction{Rep} \AgdaBound{V} \AgdaBound{W}\AgdaSymbol{)} \AgdaSymbol{→}\<%
\\
\>[19]\AgdaIndent{20}{}\<[20]%
\>[20]\AgdaBound{M} \AgdaFunction{⇑} \AgdaFunction{⇑} \AgdaFunction{⇑} \AgdaFunction{⟦} \AgdaFunction{Sub↑} \AgdaBound{K1} \AgdaSymbol{(}\AgdaFunction{Sub↑} \AgdaBound{K2} \AgdaSymbol{(}\AgdaFunction{Sub↑} \AgdaBound{K3} \AgdaBound{σ}\AgdaSymbol{))} \AgdaFunction{⟧} \AgdaFunction{〈} \AgdaFunction{Rep↑} \AgdaBound{K1} \AgdaSymbol{(}\AgdaFunction{Rep↑} \AgdaBound{K2} \AgdaSymbol{(}\AgdaFunction{Rep↑} \AgdaBound{K3} \AgdaBound{ρ}\AgdaSymbol{))} \AgdaFunction{〉}\<%
\\
\>[19]\AgdaIndent{20}{}\<[20]%
\>[20]\AgdaDatatype{≡} \AgdaBound{M} \AgdaFunction{⟦} \AgdaBound{σ} \AgdaFunction{⟧} \AgdaFunction{〈} \AgdaBound{ρ} \AgdaFunction{〉} \AgdaFunction{⇑} \AgdaFunction{⇑} \AgdaFunction{⇑}\<%
\\
\>\AgdaFunction{Rep↑-Sub↑-upRep₃} \AgdaBound{M} \AgdaBound{σ} \AgdaBound{ρ} \AgdaSymbol{=} \AgdaFunction{trans} \AgdaSymbol{(}\AgdaFunction{rep-congl} \AgdaSymbol{(}\AgdaFunction{Sub↑-upRep₃} \AgdaBound{M} \AgdaSymbol{\{}\AgdaBound{σ}\AgdaSymbol{\}))} \AgdaSymbol{(}\AgdaFunction{Rep↑-upRep₃} \AgdaSymbol{(}\AgdaBound{M} \AgdaFunction{⟦} \AgdaBound{σ} \AgdaFunction{⟧}\AgdaSymbol{))}\<%
\\
%
\\
\>\AgdaFunction{assocRSSR} \AgdaSymbol{:} \AgdaSymbol{∀} \AgdaSymbol{\{}\AgdaBound{U}\AgdaSymbol{\}} \AgdaSymbol{\{}\AgdaBound{V}\AgdaSymbol{\}} \AgdaSymbol{\{}\AgdaBound{W}\AgdaSymbol{\}} \AgdaSymbol{\{}\AgdaBound{X}\AgdaSymbol{\}} \AgdaSymbol{\{}\AgdaBound{ρ} \AgdaSymbol{:} \AgdaFunction{Sub} \AgdaBound{W} \AgdaBound{X}\AgdaSymbol{\}} \AgdaSymbol{\{}\AgdaBound{σ} \AgdaSymbol{:} \AgdaFunction{Rep} \AgdaBound{V} \AgdaBound{W}\AgdaSymbol{\}} \AgdaSymbol{\{}\AgdaBound{τ} \AgdaSymbol{:} \AgdaFunction{Sub} \AgdaBound{U} \AgdaBound{V}\AgdaSymbol{\}} \AgdaSymbol{→}\<%
\\
\>[0]\AgdaIndent{12}{}\<[12]%
\>[12]\AgdaBound{ρ} \AgdaFunction{•} \AgdaSymbol{(}\AgdaBound{σ} \AgdaFunction{•RS} \AgdaBound{τ}\AgdaSymbol{)} \AgdaFunction{∼} \AgdaSymbol{(}\AgdaBound{ρ} \AgdaFunction{•SR} \AgdaBound{σ}\AgdaSymbol{)} \AgdaFunction{•} \AgdaBound{τ}\<%
\\
\>\AgdaFunction{assocRSSR} \AgdaSymbol{\{}\AgdaArgument{ρ} \AgdaSymbol{=} \AgdaBound{ρ}\AgdaSymbol{\}} \AgdaSymbol{\{}\AgdaBound{σ}\AgdaSymbol{\}} \AgdaSymbol{\{}\AgdaBound{τ}\AgdaSymbol{\}} \AgdaBound{x} \AgdaSymbol{=} \AgdaFunction{sym} \AgdaSymbol{(}\AgdaFunction{sub-compSR} \AgdaSymbol{(}\AgdaBound{τ} \AgdaSymbol{\_} \AgdaBound{x}\AgdaSymbol{))}\<%
\end{code}
}

\begin{code}%
\>\AgdaFunction{substitution} \AgdaSymbol{:} \AgdaRecord{OpFamily}\<%
\\
\>\AgdaFunction{substitution} \AgdaSymbol{=} \AgdaKeyword{record} \AgdaSymbol{\{} \<[24]%
\>[24]\<%
\\
\>[0]\AgdaIndent{2}{}\<[2]%
\>[2]\AgdaField{liftFamily} \AgdaSymbol{=} \AgdaFunction{proto-substitution} \AgdaSymbol{;} \<[36]%
\>[36]\<%
\\
\>[0]\AgdaIndent{2}{}\<[2]%
\>[2]\AgdaField{isOpFamily} \AgdaSymbol{=} \AgdaKeyword{record} \AgdaSymbol{\{} \<[24]%
\>[24]\<%
\\
\>[2]\AgdaIndent{4}{}\<[4]%
\>[4]\AgdaField{\_∘\_} \AgdaSymbol{=} \AgdaFunction{\_•\_} \AgdaSymbol{;} \<[16]%
\>[16]\<%
\\
\>[2]\AgdaIndent{4}{}\<[4]%
\>[4]\AgdaField{liftOp-comp} \AgdaSymbol{=} \AgdaFunction{Sub↑-comp} \AgdaSymbol{;} \<[30]%
\>[30]\<%
\\
\>[2]\AgdaIndent{4}{}\<[4]%
\>[4]\AgdaField{apV-comp} \AgdaSymbol{=} \AgdaInductiveConstructor{refl} \AgdaSymbol{\}} \<[22]%
\>[22]\<%
\\
\>[0]\AgdaIndent{2}{}\<[2]%
\>[2]\AgdaSymbol{\}}\<%
\end{code}

\AgdaHide{
\begin{code}%
\>\AgdaKeyword{open} \AgdaModule{OpFamily} \AgdaFunction{substitution} \AgdaKeyword{using} \AgdaSymbol{(}comp-congl\AgdaSymbol{;}comp-congr\AgdaSymbol{)}\<%
\\
\>[0]\AgdaIndent{2}{}\<[2]%
\>[2]\AgdaKeyword{renaming} \AgdaSymbol{(}liftOp-idOp \AgdaSymbol{to} Sub↑-idOp\AgdaSymbol{;}\<\\
\>           ap-idOp \AgdaSymbol{to} sub-idOp\AgdaSymbol{;}\<\\
\>           ap-congl \AgdaSymbol{to} sub-congr\AgdaSymbol{;}\<\\
\>           ap-congr \AgdaSymbol{to} sub-congl\AgdaSymbol{;}\<\\
\>           unitl \AgdaSymbol{to} sub-unitl\AgdaSymbol{;}\<\\
\>           unitr \AgdaSymbol{to} sub-unitr\AgdaSymbol{;}\<\\
\>           ∼-sym \AgdaSymbol{to} sub-sym\AgdaSymbol{;}\<\\
\>           ∼-trans \AgdaSymbol{to} sub-trans\AgdaSymbol{;}\<\\
\>           assoc \AgdaSymbol{to} sub-assoc\AgdaSymbol{)}\<%
\\
\>[0]\AgdaIndent{2}{}\<[2]%
\>[2]\AgdaKeyword{public}\<%
\end{code}
}

\begin{frame}[fragile]
\frametitle{Metatheorems}
We can now prove general results such as:

\begin{code}%
\>\AgdaFunction{sub-comp} \AgdaSymbol{:} \AgdaSymbol{∀} \AgdaSymbol{\{}\AgdaBound{U}\AgdaSymbol{\}} \AgdaSymbol{\{}\AgdaBound{V}\AgdaSymbol{\}} \AgdaSymbol{\{}\AgdaBound{W}\AgdaSymbol{\}} \AgdaSymbol{\{}\AgdaBound{C}\AgdaSymbol{\}} \AgdaSymbol{\{}\AgdaBound{K}\AgdaSymbol{\}}\<%
\\
\>[0]\AgdaIndent{2}{}\<[2]%
\>[2]\AgdaSymbol{(}\AgdaBound{E} \AgdaSymbol{:} \AgdaDatatype{Subexpression} \AgdaBound{U} \AgdaBound{C} \AgdaBound{K}\AgdaSymbol{)} \AgdaSymbol{\{}\AgdaBound{σ} \AgdaSymbol{:} \AgdaFunction{Sub} \AgdaBound{V} \AgdaBound{W}\AgdaSymbol{\}} \AgdaSymbol{\{}\AgdaBound{ρ} \AgdaSymbol{:} \AgdaFunction{Sub} \AgdaBound{U} \AgdaBound{V}\AgdaSymbol{\}} \AgdaSymbol{→}\<%
\\
\>[0]\AgdaIndent{2}{}\<[2]%
\>[2]\AgdaBound{E} \AgdaFunction{⟦} \AgdaBound{σ} \AgdaFunction{•} \AgdaBound{ρ} \AgdaFunction{⟧} \AgdaDatatype{≡} \AgdaBound{E} \AgdaFunction{⟦} \AgdaBound{ρ} \AgdaFunction{⟧} \AgdaFunction{⟦} \AgdaBound{σ} \AgdaFunction{⟧}\<%
\end{code}
\end{frame}

\AgdaHide{
\begin{code}%
\>\AgdaFunction{sub-comp} \AgdaSymbol{=} \AgdaFunction{OpFamily.ap-circ} \AgdaFunction{substitution}\<%
\end{code}
}

\AgdaHide{
\begin{code}%
\>\AgdaKeyword{open} \AgdaKeyword{import} \AgdaModule{Grammar.OpFamily} \AgdaBound{G} \AgdaKeyword{public}\<%
\end{code}
}

\AgdaHide{
\begin{code}%
\>\AgdaComment{--Variable convention: ρ ranges over replacements}\<%
\\
\>\AgdaKeyword{open} \AgdaKeyword{import} \AgdaModule{Grammar.Base}\<%
\\
%
\\
\>\AgdaKeyword{module} \AgdaModule{Grammar.Replacement} \AgdaSymbol{(}\AgdaBound{G} \AgdaSymbol{:} \AgdaRecord{Grammar}\AgdaSymbol{)} \AgdaKeyword{where}\<%
\\
%
\\
\>\AgdaKeyword{open} \AgdaKeyword{import} \AgdaModule{Function}\<%
\\
\>\AgdaKeyword{open} \AgdaKeyword{import} \AgdaModule{Prelims}\<%
\\
\>\AgdaKeyword{open} \AgdaModule{Grammar} \AgdaBound{G}\<%
\\
\>\AgdaKeyword{open} \AgdaKeyword{import} \AgdaModule{Grammar.OpFamily.PreOpFamily} \AgdaBound{G}\<%
\\
\>\AgdaKeyword{open} \AgdaKeyword{import} \AgdaModule{Grammar.OpFamily.LiftFamily} \AgdaBound{G}\<%
\\
\>\AgdaKeyword{open} \AgdaKeyword{import} \AgdaModule{Grammar.OpFamily.OpFamily} \AgdaBound{G}\<%
\end{code}
}

\subsection{Replacement}

The operation family of \emph{replacement} is defined as follows.  A replacement $\rho : U \rightarrow V$ is a function
that maps every variable in $U$ to a variable in $V$ of the same kind.  Application, identity and composition are simply
function application, the identity function and function composition.  The successor is the canonical injection $V \rightarrow (V, K)$,
and $(\sigma , K)$ is the extension of $\sigma$ that maps $x_0$ to $x_0$.

\begin{code}%
\>\AgdaFunction{Rep} \AgdaSymbol{:} \AgdaDatatype{Alphabet} \AgdaSymbol{→} \AgdaDatatype{Alphabet} \AgdaSymbol{→} \AgdaPrimitiveType{Set}\<%
\\
\>\AgdaFunction{Rep} \AgdaBound{U} \AgdaBound{V} \AgdaSymbol{=} \AgdaSymbol{∀} \AgdaBound{K} \AgdaSymbol{→} \AgdaDatatype{Var} \AgdaBound{U} \AgdaBound{K} \AgdaSymbol{→} \AgdaDatatype{Var} \AgdaBound{V} \AgdaBound{K}\<%
\\
%
\\
\>\AgdaFunction{upRep} \AgdaSymbol{:} \AgdaSymbol{∀} \AgdaSymbol{\{}\AgdaBound{V}\AgdaSymbol{\}} \AgdaSymbol{\{}\AgdaBound{K}\AgdaSymbol{\}} \AgdaSymbol{→} \AgdaFunction{Rep} \AgdaBound{V} \AgdaSymbol{(}\AgdaBound{V} \AgdaInductiveConstructor{,} \AgdaBound{K}\AgdaSymbol{)}\<%
\\
\>\AgdaFunction{upRep} \AgdaSymbol{\_} \AgdaSymbol{=} \AgdaInductiveConstructor{↑}\<%
\\
%
\\
\>\AgdaFunction{idRep} \AgdaSymbol{:} \AgdaSymbol{∀} \AgdaBound{V} \AgdaSymbol{→} \AgdaFunction{Rep} \AgdaBound{V} \AgdaBound{V}\<%
\\
\>\AgdaFunction{idRep} \AgdaSymbol{\_} \AgdaSymbol{\_} \AgdaBound{x} \AgdaSymbol{=} \AgdaBound{x}\<%
\\
%
\\
\>\AgdaFunction{Rep∶POF} \AgdaSymbol{:} \AgdaRecord{PreOpFamily}\<%
\\
\>\AgdaFunction{Rep∶POF} \AgdaSymbol{=} \AgdaKeyword{record} \AgdaSymbol{\{} \<[19]%
\>[19]\<%
\\
\>[0]\AgdaIndent{2}{}\<[2]%
\>[2]\AgdaField{Op} \AgdaSymbol{=} \AgdaFunction{Rep}\AgdaSymbol{;} \<[12]%
\>[12]\<%
\\
\>[0]\AgdaIndent{2}{}\<[2]%
\>[2]\AgdaField{apV} \AgdaSymbol{=} \AgdaSymbol{λ} \AgdaBound{ρ} \AgdaBound{x} \AgdaSymbol{→} \AgdaInductiveConstructor{var} \AgdaSymbol{(}\AgdaBound{ρ} \AgdaSymbol{\_} \AgdaBound{x}\AgdaSymbol{);} \<[29]%
\>[29]\<%
\\
\>[0]\AgdaIndent{2}{}\<[2]%
\>[2]\AgdaField{up} \AgdaSymbol{=} \AgdaFunction{upRep}\AgdaSymbol{;} \<[14]%
\>[14]\<%
\\
\>[0]\AgdaIndent{2}{}\<[2]%
\>[2]\AgdaField{apV-up} \AgdaSymbol{=} \AgdaInductiveConstructor{refl}\AgdaSymbol{;} \<[17]%
\>[17]\<%
\\
\>[0]\AgdaIndent{2}{}\<[2]%
\>[2]\AgdaField{idOp} \AgdaSymbol{=} \AgdaFunction{idRep}\AgdaSymbol{;} \<[16]%
\>[16]\<%
\\
\>[0]\AgdaIndent{2}{}\<[2]%
\>[2]\AgdaField{apV-idOp} \AgdaSymbol{=} \AgdaSymbol{λ} \AgdaBound{\_} \AgdaSymbol{→} \AgdaInductiveConstructor{refl} \AgdaSymbol{\}}\<%
\\
%
\\
\>\AgdaFunction{\_∼R\_} \AgdaSymbol{:} \AgdaSymbol{∀} \AgdaSymbol{\{}\AgdaBound{U}\AgdaSymbol{\}} \AgdaSymbol{\{}\AgdaBound{V}\AgdaSymbol{\}} \AgdaSymbol{→} \AgdaFunction{Rep} \AgdaBound{U} \AgdaBound{V} \AgdaSymbol{→} \AgdaFunction{Rep} \AgdaBound{U} \AgdaBound{V} \AgdaSymbol{→} \AgdaPrimitiveType{Set}\<%
\\
\>\AgdaFunction{\_∼R\_} \AgdaSymbol{=} \AgdaFunction{PreOpFamily.\_∼op\_} \AgdaFunction{Rep∶POF}\<%
\\
%
\\
\>\AgdaFunction{liftRep} \AgdaSymbol{:} \AgdaSymbol{∀} \AgdaSymbol{\{}\AgdaBound{U}\AgdaSymbol{\}} \AgdaSymbol{\{}\AgdaBound{V}\AgdaSymbol{\}} \AgdaBound{K} \AgdaSymbol{→} \AgdaFunction{Rep} \AgdaBound{U} \AgdaBound{V} \AgdaSymbol{→} \AgdaFunction{Rep} \AgdaSymbol{(}\AgdaBound{U} \AgdaInductiveConstructor{,} \AgdaBound{K}\AgdaSymbol{)} \AgdaSymbol{(}\AgdaBound{V} \AgdaInductiveConstructor{,} \AgdaBound{K}\AgdaSymbol{)}\<%
\\
\>\AgdaFunction{liftRep} \AgdaSymbol{\_} \AgdaSymbol{\_} \AgdaSymbol{\_} \AgdaInductiveConstructor{x₀} \AgdaSymbol{=} \AgdaInductiveConstructor{x₀}\<%
\\
\>\AgdaFunction{liftRep} \AgdaSymbol{\_} \AgdaBound{ρ} \AgdaBound{K} \AgdaSymbol{(}\AgdaInductiveConstructor{↑} \AgdaBound{x}\AgdaSymbol{)} \AgdaSymbol{=} \AgdaInductiveConstructor{↑} \AgdaSymbol{(}\AgdaBound{ρ} \AgdaBound{K} \AgdaBound{x}\AgdaSymbol{)}\<%
\\
%
\\
\>\AgdaFunction{liftRep-cong} \AgdaSymbol{:} \AgdaSymbol{∀} \AgdaSymbol{\{}\AgdaBound{U}\AgdaSymbol{\}} \AgdaSymbol{\{}\AgdaBound{V}\AgdaSymbol{\}} \AgdaSymbol{\{}\AgdaBound{K}\AgdaSymbol{\}} \AgdaSymbol{\{}\AgdaBound{ρ} \AgdaBound{ρ'} \AgdaSymbol{:} \AgdaFunction{Rep} \AgdaBound{U} \AgdaBound{V}\AgdaSymbol{\}} \AgdaSymbol{→} \<[48]%
\>[48]\<%
\\
\>[0]\AgdaIndent{2}{}\<[2]%
\>[2]\AgdaBound{ρ} \AgdaFunction{∼R} \AgdaBound{ρ'} \AgdaSymbol{→} \AgdaFunction{liftRep} \AgdaBound{K} \AgdaBound{ρ} \AgdaFunction{∼R} \AgdaFunction{liftRep} \AgdaBound{K} \AgdaBound{ρ'}\<%
\end{code}

\AgdaHide{
\begin{code}%
\>\AgdaFunction{liftRep-cong} \AgdaBound{ρ-is-ρ'} \AgdaInductiveConstructor{x₀} \AgdaSymbol{=} \AgdaInductiveConstructor{refl}\<%
\\
\>\AgdaFunction{liftRep-cong} \AgdaBound{ρ-is-ρ'} \AgdaSymbol{(}\AgdaInductiveConstructor{↑} \AgdaBound{x}\AgdaSymbol{)} \AgdaSymbol{=} \AgdaFunction{cong} \AgdaSymbol{(}\AgdaInductiveConstructor{var} \AgdaFunction{∘} \AgdaInductiveConstructor{↑}\AgdaSymbol{)} \AgdaSymbol{(}\AgdaFunction{var-inj} \AgdaSymbol{(}\AgdaBound{ρ-is-ρ'} \AgdaBound{x}\AgdaSymbol{))}\<%
\end{code}
}

\begin{code}%
\>\AgdaFunction{Rep∶LF} \AgdaSymbol{:} \AgdaRecord{LiftFamily}\<%
\\
\>\AgdaFunction{Rep∶LF} \AgdaSymbol{=} \AgdaKeyword{record} \AgdaSymbol{\{} \<[18]%
\>[18]\<%
\\
\>[0]\AgdaIndent{2}{}\<[2]%
\>[2]\AgdaField{preOpFamily} \AgdaSymbol{=} \AgdaFunction{Rep∶POF} \AgdaSymbol{;} \<[26]%
\>[26]\<%
\\
\>[0]\AgdaIndent{2}{}\<[2]%
\>[2]\AgdaField{lifting} \AgdaSymbol{=} \AgdaKeyword{record} \AgdaSymbol{\{} \<[21]%
\>[21]\<%
\\
\>[2]\AgdaIndent{4}{}\<[4]%
\>[4]\AgdaField{liftOp} \AgdaSymbol{=} \AgdaFunction{liftRep} \AgdaSymbol{;} \<[23]%
\>[23]\<%
\\
\>[2]\AgdaIndent{4}{}\<[4]%
\>[4]\AgdaField{liftOp-cong} \AgdaSymbol{=} \AgdaFunction{liftRep-cong} \AgdaSymbol{\}} \AgdaSymbol{;} \<[35]%
\>[35]\<%
\\
\>[0]\AgdaIndent{2}{}\<[2]%
\>[2]\AgdaField{isLiftFamily} \AgdaSymbol{=} \AgdaKeyword{record} \AgdaSymbol{\{} \<[26]%
\>[26]\<%
\\
\>[2]\AgdaIndent{4}{}\<[4]%
\>[4]\AgdaField{liftOp-x₀} \AgdaSymbol{=} \AgdaInductiveConstructor{refl} \AgdaSymbol{;} \<[23]%
\>[23]\<%
\\
\>[2]\AgdaIndent{4}{}\<[4]%
\>[4]\AgdaField{liftOp-↑} \AgdaSymbol{=} \AgdaSymbol{λ} \AgdaBound{\_} \AgdaSymbol{→} \AgdaInductiveConstructor{refl} \AgdaSymbol{\}} \AgdaSymbol{\}}\<%
\\
%
\\
\>\AgdaKeyword{infix} \AgdaNumber{70} \AgdaFixityOp{\_〈\_〉}\<%
\\
\>\AgdaFunction{\_〈\_〉} \AgdaSymbol{:} \AgdaSymbol{∀} \AgdaSymbol{\{}\AgdaBound{U}\AgdaSymbol{\}} \AgdaSymbol{\{}\AgdaBound{V}\AgdaSymbol{\}} \AgdaSymbol{\{}\AgdaBound{C}\AgdaSymbol{\}} \AgdaSymbol{\{}\AgdaBound{K}\AgdaSymbol{\}} \AgdaSymbol{→} \AgdaDatatype{Subexpression} \AgdaBound{U} \AgdaBound{C} \AgdaBound{K} \AgdaSymbol{→} \AgdaFunction{Rep} \AgdaBound{U} \AgdaBound{V} \AgdaSymbol{→} \AgdaDatatype{Subexpression} \AgdaBound{V} \AgdaBound{C} \AgdaBound{K}\<%
\\
\>\AgdaBound{E} \AgdaFunction{〈} \AgdaBound{ρ} \AgdaFunction{〉} \AgdaSymbol{=} \AgdaFunction{LiftFamily.ap} \AgdaFunction{Rep∶LF} \AgdaBound{ρ} \AgdaBound{E}\<%
\\
%
\\
\>\AgdaKeyword{infixl} \AgdaNumber{75} \AgdaFixityOp{\_•R\_}\<%
\\
\>\AgdaFunction{\_•R\_} \AgdaSymbol{:} \AgdaSymbol{∀} \AgdaSymbol{\{}\AgdaBound{U}\AgdaSymbol{\}} \AgdaSymbol{\{}\AgdaBound{V}\AgdaSymbol{\}} \AgdaSymbol{\{}\AgdaBound{W}\AgdaSymbol{\}} \AgdaSymbol{→} \AgdaFunction{Rep} \AgdaBound{V} \AgdaBound{W} \AgdaSymbol{→} \AgdaFunction{Rep} \AgdaBound{U} \AgdaBound{V} \AgdaSymbol{→} \AgdaFunction{Rep} \AgdaBound{U} \AgdaBound{W}\<%
\\
\>\AgdaSymbol{(}\AgdaBound{ρ'} \AgdaFunction{•R} \AgdaBound{ρ}\AgdaSymbol{)} \AgdaBound{K} \AgdaBound{x} \AgdaSymbol{=} \AgdaBound{ρ'} \AgdaBound{K} \AgdaSymbol{(}\AgdaBound{ρ} \AgdaBound{K} \AgdaBound{x}\AgdaSymbol{)}\<%
\\
%
\\
\>\AgdaFunction{liftRep-comp} \AgdaSymbol{:} \AgdaSymbol{∀} \AgdaSymbol{\{}\AgdaBound{U}\AgdaSymbol{\}} \AgdaSymbol{\{}\AgdaBound{V}\AgdaSymbol{\}} \AgdaSymbol{\{}\AgdaBound{W}\AgdaSymbol{\}} \AgdaSymbol{\{}\AgdaBound{K}\AgdaSymbol{\}} \AgdaSymbol{\{}\AgdaBound{ρ'} \AgdaSymbol{:} \AgdaFunction{Rep} \AgdaBound{V} \AgdaBound{W}\AgdaSymbol{\}} \AgdaSymbol{\{}\AgdaBound{ρ} \AgdaSymbol{:} \AgdaFunction{Rep} \AgdaBound{U} \AgdaBound{V}\AgdaSymbol{\}} \AgdaSymbol{→} \<[64]%
\>[64]\<%
\\
\>[0]\AgdaIndent{2}{}\<[2]%
\>[2]\AgdaFunction{liftRep} \AgdaBound{K} \AgdaSymbol{(}\AgdaBound{ρ'} \AgdaFunction{•R} \AgdaBound{ρ}\AgdaSymbol{)} \AgdaFunction{∼R} \AgdaFunction{liftRep} \AgdaBound{K} \AgdaBound{ρ'} \AgdaFunction{•R} \AgdaFunction{liftRep} \AgdaBound{K} \AgdaBound{ρ}\<%
\end{code}

\AgdaHide{
\begin{code}%
\>\AgdaFunction{liftRep-comp} \AgdaInductiveConstructor{x₀} \AgdaSymbol{=} \AgdaInductiveConstructor{refl}\<%
\\
\>\AgdaFunction{liftRep-comp} \AgdaSymbol{(}\AgdaInductiveConstructor{↑} \AgdaSymbol{\_)} \AgdaSymbol{=} \AgdaInductiveConstructor{refl}\<%
\end{code}
}

\begin{code}%
\>\AgdaFunction{REP} \AgdaSymbol{:} \AgdaRecord{OpFamily}\<%
\\
\>\AgdaFunction{REP} \AgdaSymbol{=} \AgdaKeyword{record} \AgdaSymbol{\{} \<[15]%
\>[15]\<%
\\
\>[0]\AgdaIndent{2}{}\<[2]%
\>[2]\AgdaField{liftFamily} \AgdaSymbol{=} \AgdaFunction{Rep∶LF} \AgdaSymbol{;} \<[24]%
\>[24]\<%
\\
\>[0]\AgdaIndent{2}{}\<[2]%
\>[2]\AgdaField{comp} \AgdaSymbol{=} \AgdaKeyword{record} \AgdaSymbol{\{} \<[18]%
\>[18]\<%
\\
\>[2]\AgdaIndent{4}{}\<[4]%
\>[4]\AgdaField{\_∘\_} \AgdaSymbol{=} \AgdaFunction{\_•R\_} \AgdaSymbol{;} \<[17]%
\>[17]\<%
\\
\>[2]\AgdaIndent{4}{}\<[4]%
\>[4]\AgdaField{apV-comp} \AgdaSymbol{=} \AgdaInductiveConstructor{refl} \AgdaSymbol{;} \<[22]%
\>[22]\<%
\\
\>[2]\AgdaIndent{4}{}\<[4]%
\>[4]\AgdaField{liftOp-comp} \AgdaSymbol{=} \AgdaFunction{liftRep-comp} \AgdaSymbol{\}} \AgdaSymbol{\}}\<%
\end{code}

\AgdaHide{
\begin{code}%
\>\AgdaFunction{•R-congl} \AgdaSymbol{:} \AgdaSymbol{∀} \AgdaSymbol{\{}\AgdaBound{U} \AgdaBound{V} \AgdaBound{W}\AgdaSymbol{\}} \AgdaSymbol{\{}\AgdaBound{ρ₁} \AgdaBound{ρ₂} \AgdaSymbol{:} \AgdaFunction{Rep} \AgdaBound{V} \AgdaBound{W}\AgdaSymbol{\}} \AgdaSymbol{→} \AgdaBound{ρ₁} \AgdaFunction{∼R} \AgdaBound{ρ₂} \AgdaSymbol{→} \AgdaSymbol{∀} \AgdaSymbol{(}\AgdaBound{ρ₃} \AgdaSymbol{:} \AgdaFunction{Rep} \AgdaBound{U} \AgdaBound{V}\AgdaSymbol{)} \AgdaSymbol{→} \AgdaBound{ρ₁} \AgdaFunction{•R} \AgdaBound{ρ₃} \AgdaFunction{∼R} \AgdaBound{ρ₂} \AgdaFunction{•R} \AgdaBound{ρ₃}\<%
\\
\>\AgdaFunction{•R-congl} \AgdaSymbol{=} \AgdaFunction{OpFamily.comp-congl} \AgdaFunction{REP}\<%
\\
%
\\
\>\AgdaFunction{•R-congr} \AgdaSymbol{:} \AgdaSymbol{∀} \AgdaSymbol{\{}\AgdaBound{U} \AgdaBound{V} \AgdaBound{W}\AgdaSymbol{\}} \AgdaSymbol{\{}\AgdaBound{ρ₁} \AgdaSymbol{:} \AgdaFunction{Rep} \AgdaBound{V} \AgdaBound{W}\AgdaSymbol{\}} \AgdaSymbol{\{}\AgdaBound{ρ₂} \AgdaBound{ρ₃} \AgdaSymbol{:} \AgdaFunction{Rep} \AgdaBound{U} \AgdaBound{V}\AgdaSymbol{\}} \AgdaSymbol{→} \AgdaBound{ρ₂} \AgdaFunction{∼R} \AgdaBound{ρ₃} \AgdaSymbol{→} \AgdaBound{ρ₁} \AgdaFunction{•R} \AgdaBound{ρ₂} \AgdaFunction{∼R} \AgdaBound{ρ₁} \AgdaFunction{•R} \AgdaBound{ρ₃}\<%
\\
\>\AgdaFunction{•R-congr} \AgdaSymbol{\{}\AgdaArgument{ρ₁} \AgdaSymbol{=} \AgdaBound{ρ₁}\AgdaSymbol{\}} \AgdaSymbol{=} \AgdaFunction{OpFamily.comp-congr} \AgdaFunction{REP} \AgdaBound{ρ₁}\<%
\\
%
\\
\>\AgdaFunction{rep-congr} \AgdaSymbol{:} \AgdaSymbol{∀} \AgdaSymbol{\{}\AgdaBound{U} \AgdaBound{V} \AgdaBound{C} \AgdaBound{K}\AgdaSymbol{\}} \AgdaSymbol{\{}\AgdaBound{ρ} \AgdaBound{ρ'} \AgdaSymbol{:} \AgdaFunction{Rep} \AgdaBound{U} \AgdaBound{V}\AgdaSymbol{\}} \AgdaSymbol{→} \AgdaBound{ρ} \AgdaFunction{∼R} \AgdaBound{ρ'} \AgdaSymbol{→} \AgdaSymbol{∀} \AgdaSymbol{(}\AgdaBound{E} \AgdaSymbol{:} \AgdaDatatype{Subexpression} \AgdaBound{U} \AgdaBound{C} \AgdaBound{K}\AgdaSymbol{)} \AgdaSymbol{→} \AgdaBound{E} \AgdaFunction{〈} \AgdaBound{ρ} \AgdaFunction{〉} \AgdaDatatype{≡} \AgdaBound{E} \AgdaFunction{〈} \AgdaBound{ρ'} \AgdaFunction{〉}\<%
\\
\>\AgdaFunction{rep-congr} \AgdaSymbol{=} \AgdaFunction{OpFamily.ap-congl} \AgdaFunction{REP}\<%
\\
%
\\
\>\AgdaFunction{rep-congl} \AgdaSymbol{:} \AgdaSymbol{∀} \AgdaSymbol{\{}\AgdaBound{U} \AgdaBound{V} \AgdaBound{C} \AgdaBound{K}\AgdaSymbol{\}} \AgdaSymbol{\{}\AgdaBound{ρ} \AgdaSymbol{:} \AgdaFunction{Rep} \AgdaBound{U} \AgdaBound{V}\AgdaSymbol{\}} \AgdaSymbol{\{}\AgdaBound{E} \AgdaBound{F} \AgdaSymbol{:} \AgdaDatatype{Subexpression} \AgdaBound{U} \AgdaBound{C} \AgdaBound{K}\AgdaSymbol{\}} \AgdaSymbol{→} \AgdaBound{E} \AgdaDatatype{≡} \AgdaBound{F} \AgdaSymbol{→} \AgdaBound{E} \AgdaFunction{〈} \AgdaBound{ρ} \AgdaFunction{〉} \AgdaDatatype{≡} \AgdaBound{F} \AgdaFunction{〈} \AgdaBound{ρ} \AgdaFunction{〉}\<%
\\
\>\AgdaFunction{rep-congl} \AgdaSymbol{=} \AgdaFunction{OpFamily.ap-congr} \AgdaFunction{REP}\<%
\\
%
\\
\>\AgdaFunction{rep-idOp} \AgdaSymbol{:} \AgdaSymbol{∀} \AgdaSymbol{\{}\AgdaBound{V} \AgdaBound{C} \AgdaBound{K}\AgdaSymbol{\}} \AgdaSymbol{\{}\AgdaBound{E} \AgdaSymbol{:} \AgdaDatatype{Subexpression} \AgdaBound{V} \AgdaBound{C} \AgdaBound{K}\AgdaSymbol{\}} \AgdaSymbol{→} \AgdaBound{E} \AgdaFunction{〈} \AgdaFunction{idRep} \AgdaBound{V} \AgdaFunction{〉} \AgdaDatatype{≡} \AgdaBound{E}\<%
\\
\>\AgdaFunction{rep-idOp} \AgdaSymbol{=} \AgdaFunction{OpFamily.ap-idOp} \AgdaFunction{REP}\<%
\\
%
\\
\>\AgdaFunction{rep-comp} \AgdaSymbol{:} \AgdaSymbol{∀} \AgdaSymbol{\{}\AgdaBound{U} \AgdaBound{V} \AgdaBound{W} \AgdaBound{C} \AgdaBound{K}\AgdaSymbol{\}} \AgdaSymbol{(}\AgdaBound{E} \AgdaSymbol{:} \AgdaDatatype{Subexpression} \AgdaBound{U} \AgdaBound{C} \AgdaBound{K}\AgdaSymbol{)} \AgdaSymbol{\{}\AgdaBound{σ} \AgdaSymbol{:} \AgdaFunction{Rep} \AgdaBound{V} \AgdaBound{W}\AgdaSymbol{\}} \AgdaSymbol{\{}\AgdaBound{ρ}\AgdaSymbol{\}} \AgdaSymbol{→} \AgdaBound{E} \AgdaFunction{〈} \AgdaBound{σ} \AgdaFunction{•R} \AgdaBound{ρ} \AgdaFunction{〉} \AgdaDatatype{≡} \AgdaBound{E} \AgdaFunction{〈} \AgdaBound{ρ} \AgdaFunction{〉} \AgdaFunction{〈} \AgdaBound{σ} \AgdaFunction{〉}\<%
\\
\>\AgdaFunction{rep-comp} \AgdaSymbol{=} \AgdaFunction{OpFamily.ap-comp} \AgdaFunction{REP}\<%
\\
%
\\
\>\AgdaFunction{liftRep-idOp} \AgdaSymbol{:} \AgdaSymbol{∀} \AgdaSymbol{\{}\AgdaBound{V} \AgdaBound{K}\AgdaSymbol{\}} \AgdaSymbol{→} \AgdaFunction{liftRep} \AgdaBound{K} \AgdaSymbol{(}\AgdaFunction{idRep} \AgdaBound{V}\AgdaSymbol{)} \AgdaFunction{∼R} \AgdaFunction{idRep} \AgdaSymbol{(}\AgdaBound{V} \AgdaInductiveConstructor{,} \AgdaBound{K}\AgdaSymbol{)}\<%
\\
\>\AgdaFunction{liftRep-idOp} \AgdaSymbol{=} \AgdaFunction{OpFamily.liftOp-idOp} \AgdaFunction{REP}\<%
\\
%
\\
\>\AgdaFunction{liftRep-upRep} \AgdaSymbol{:} \AgdaSymbol{∀} \AgdaSymbol{\{}\AgdaBound{U} \AgdaBound{V} \AgdaBound{C} \AgdaBound{K} \AgdaBound{L}\AgdaSymbol{\}} \AgdaSymbol{\{}\AgdaBound{σ} \AgdaSymbol{:} \AgdaFunction{Rep} \AgdaBound{U} \AgdaBound{V}\AgdaSymbol{\}} \AgdaSymbol{(}\AgdaBound{E} \AgdaSymbol{:} \AgdaDatatype{Subexpression} \AgdaBound{U} \AgdaBound{C} \AgdaBound{K}\AgdaSymbol{)} \AgdaSymbol{→} \AgdaBound{E} \AgdaFunction{〈} \AgdaFunction{upRep} \AgdaFunction{〉} \AgdaFunction{〈} \AgdaFunction{liftRep} \AgdaBound{L} \AgdaBound{σ} \AgdaFunction{〉} \AgdaDatatype{≡} \AgdaBound{E} \AgdaFunction{〈} \AgdaBound{σ} \AgdaFunction{〉} \AgdaFunction{〈} \AgdaFunction{upRep} \AgdaFunction{〉}\<%
\\
\>\AgdaFunction{liftRep-upRep} \AgdaSymbol{=} \AgdaFunction{OpFamily.liftOp-up} \AgdaFunction{REP}\<%
\\
%
\\
\>\AgdaFunction{liftRep-comp₄} \AgdaSymbol{:} \AgdaSymbol{∀} \AgdaSymbol{\{}\AgdaBound{U}\AgdaSymbol{\}} \AgdaSymbol{\{}\AgdaBound{V1}\AgdaSymbol{\}} \AgdaSymbol{\{}\AgdaBound{V2}\AgdaSymbol{\}} \AgdaSymbol{\{}\AgdaBound{V3}\AgdaSymbol{\}} \AgdaSymbol{\{}\AgdaBound{V4}\AgdaSymbol{\}} \AgdaSymbol{\{}\AgdaBound{K}\AgdaSymbol{\}} \AgdaSymbol{\{}\AgdaBound{ρ1} \AgdaSymbol{:} \AgdaFunction{Rep} \AgdaBound{U} \AgdaBound{V1}\AgdaSymbol{\}} \AgdaSymbol{\{}\AgdaBound{ρ2} \AgdaSymbol{:} \AgdaFunction{Rep} \AgdaBound{V1} \AgdaBound{V2}\AgdaSymbol{\}} \AgdaSymbol{\{}\AgdaBound{ρ3} \AgdaSymbol{:} \AgdaFunction{Rep} \AgdaBound{V2} \AgdaBound{V3}\AgdaSymbol{\}} \AgdaSymbol{\{}\AgdaBound{ρ4} \AgdaSymbol{:} \AgdaFunction{Rep} \AgdaBound{V3} \AgdaBound{V4}\AgdaSymbol{\}} \AgdaSymbol{→}\<%
\\
\>[4]\AgdaIndent{16}{}\<[16]%
\>[16]\AgdaFunction{liftRep} \AgdaBound{K} \AgdaSymbol{(}\AgdaBound{ρ4} \AgdaFunction{•R} \AgdaBound{ρ3} \AgdaFunction{•R} \AgdaBound{ρ2} \AgdaFunction{•R} \AgdaBound{ρ1}\AgdaSymbol{)} \AgdaFunction{∼R} \AgdaFunction{liftRep} \AgdaBound{K} \AgdaBound{ρ4} \AgdaFunction{•R} \AgdaFunction{liftRep} \AgdaBound{K} \AgdaBound{ρ3} \AgdaFunction{•R} \AgdaFunction{liftRep} \AgdaBound{K} \AgdaBound{ρ2} \AgdaFunction{•R} \AgdaFunction{liftRep} \AgdaBound{K} \AgdaBound{ρ1}\<%
\\
\>\AgdaFunction{liftRep-comp₄} \AgdaSymbol{\{}\AgdaBound{U}\AgdaSymbol{\}} \AgdaSymbol{\{}\AgdaBound{V1}\AgdaSymbol{\}} \AgdaSymbol{\{}\AgdaBound{V2}\AgdaSymbol{\}} \AgdaSymbol{\{}\AgdaBound{V3}\AgdaSymbol{\}} \AgdaSymbol{\{}\AgdaBound{V4}\AgdaSymbol{\}} \AgdaSymbol{\{}\AgdaBound{K}\AgdaSymbol{\}} \AgdaSymbol{\{}\AgdaBound{ρ1}\AgdaSymbol{\}} \AgdaSymbol{\{}\AgdaBound{ρ2}\AgdaSymbol{\}} \AgdaSymbol{\{}\AgdaBound{ρ3}\AgdaSymbol{\}} \AgdaSymbol{\{}\AgdaBound{ρ4}\AgdaSymbol{\}} \AgdaSymbol{=}\<%
\\
\>[0]\AgdaIndent{2}{}\<[2]%
\>[2]\AgdaKeyword{let} \AgdaKeyword{open} \AgdaModule{Prelims.}\AgdaModule{EqReasoning} \AgdaSymbol{(}\AgdaFunction{PreOpFamily.OP} \AgdaFunction{Rep∶POF} \AgdaSymbol{(}\AgdaBound{U} \AgdaInductiveConstructor{,} \AgdaBound{K}\AgdaSymbol{)} \AgdaSymbol{(}\AgdaBound{V4} \AgdaInductiveConstructor{,} \AgdaBound{K}\AgdaSymbol{))} \AgdaKeyword{in} \<[76]%
\>[76]\<%
\\
\>[0]\AgdaIndent{2}{}\<[2]%
\>[2]\AgdaFunction{begin}\<%
\\
\>[2]\AgdaIndent{4}{}\<[4]%
\>[4]\AgdaFunction{liftRep} \AgdaBound{K} \AgdaSymbol{(}\AgdaBound{ρ4} \AgdaFunction{•R} \AgdaBound{ρ3} \AgdaFunction{•R} \AgdaBound{ρ2} \AgdaFunction{•R} \AgdaBound{ρ1}\AgdaSymbol{)}\<%
\\
\>[0]\AgdaIndent{2}{}\<[2]%
\>[2]\AgdaFunction{≈⟨} \AgdaFunction{liftRep-comp} \AgdaFunction{⟩}\<%
\\
\>[2]\AgdaIndent{4}{}\<[4]%
\>[4]\AgdaFunction{liftRep} \AgdaBound{K} \AgdaSymbol{(}\AgdaBound{ρ4} \AgdaFunction{•R} \AgdaBound{ρ3} \AgdaFunction{•R} \AgdaBound{ρ2}\AgdaSymbol{)} \AgdaFunction{•R} \AgdaFunction{liftRep} \AgdaBound{K} \AgdaBound{ρ1}\<%
\\
\>[0]\AgdaIndent{2}{}\<[2]%
\>[2]\AgdaFunction{≈⟨} \AgdaFunction{•R-congl} \AgdaFunction{liftRep-comp} \AgdaSymbol{(}\AgdaFunction{liftRep} \AgdaBound{K} \AgdaBound{ρ1}\AgdaSymbol{)}\AgdaFunction{⟩}\<%
\\
\>[2]\AgdaIndent{4}{}\<[4]%
\>[4]\AgdaFunction{liftRep} \AgdaBound{K} \AgdaSymbol{(}\AgdaBound{ρ4} \AgdaFunction{•R} \AgdaBound{ρ3}\AgdaSymbol{)} \AgdaFunction{•R} \AgdaFunction{liftRep} \AgdaBound{K} \AgdaBound{ρ2} \AgdaFunction{•R} \AgdaFunction{liftRep} \AgdaBound{K} \AgdaBound{ρ1}\<%
\\
\>[0]\AgdaIndent{2}{}\<[2]%
\>[2]\AgdaFunction{≈⟨} \AgdaFunction{•R-congl} \AgdaSymbol{(}\AgdaFunction{•R-congl} \AgdaFunction{liftRep-comp} \AgdaSymbol{(}\AgdaFunction{liftRep} \AgdaBound{K} \AgdaBound{ρ2}\AgdaSymbol{))} \AgdaSymbol{(}\AgdaFunction{liftRep} \AgdaBound{K} \AgdaBound{ρ1}\AgdaSymbol{)}\AgdaFunction{⟩}\<%
\\
\>[2]\AgdaIndent{4}{}\<[4]%
\>[4]\AgdaFunction{liftRep} \AgdaBound{K} \AgdaBound{ρ4} \AgdaFunction{•R} \AgdaFunction{liftRep} \AgdaBound{K} \AgdaBound{ρ3} \AgdaFunction{•R} \AgdaFunction{liftRep} \AgdaBound{K} \AgdaBound{ρ2} \AgdaFunction{•R} \AgdaFunction{liftRep} \AgdaBound{K} \AgdaBound{ρ1}\<%
\\
\>[0]\AgdaIndent{2}{}\<[2]%
\>[2]\AgdaFunction{∎}\<%
\\
%
\\
\>\AgdaFunction{rep-comp₄} \AgdaSymbol{:} \AgdaSymbol{∀} \AgdaSymbol{\{}\AgdaBound{U}\AgdaSymbol{\}} \AgdaSymbol{\{}\AgdaBound{V1}\AgdaSymbol{\}} \AgdaSymbol{\{}\AgdaBound{V2}\AgdaSymbol{\}} \AgdaSymbol{\{}\AgdaBound{V3}\AgdaSymbol{\}} \AgdaSymbol{\{}\AgdaBound{V4}\AgdaSymbol{\}} \<[38]%
\>[38]\<%
\\
\>[2]\AgdaIndent{12}{}\<[12]%
\>[12]\AgdaSymbol{\{}\AgdaBound{ρ1} \AgdaSymbol{:} \AgdaFunction{Rep} \AgdaBound{U} \AgdaBound{V1}\AgdaSymbol{\}} \AgdaSymbol{\{}\AgdaBound{ρ2} \AgdaSymbol{:} \AgdaFunction{Rep} \AgdaBound{V1} \AgdaBound{V2}\AgdaSymbol{\}} \AgdaSymbol{\{}\AgdaBound{ρ3} \AgdaSymbol{:} \AgdaFunction{Rep} \AgdaBound{V2} \AgdaBound{V3}\AgdaSymbol{\}} \AgdaSymbol{\{}\AgdaBound{ρ4} \AgdaSymbol{:} \AgdaFunction{Rep} \AgdaBound{V3} \AgdaBound{V4}\AgdaSymbol{\}} \<[79]%
\>[79]\<%
\\
\>[2]\AgdaIndent{12}{}\<[12]%
\>[12]\AgdaSymbol{\{}\AgdaBound{C}\AgdaSymbol{\}} \AgdaSymbol{\{}\AgdaBound{K}\AgdaSymbol{\}} \AgdaSymbol{(}\AgdaBound{E} \AgdaSymbol{:} \AgdaDatatype{Subexpression} \AgdaBound{U} \AgdaBound{C} \AgdaBound{K}\AgdaSymbol{)} \AgdaSymbol{→}\<%
\\
\>[2]\AgdaIndent{12}{}\<[12]%
\>[12]\AgdaBound{E} \AgdaFunction{〈} \AgdaBound{ρ4} \AgdaFunction{•R} \AgdaBound{ρ3} \AgdaFunction{•R} \AgdaBound{ρ2} \AgdaFunction{•R} \AgdaBound{ρ1} \AgdaFunction{〉} \AgdaDatatype{≡} \AgdaBound{E} \AgdaFunction{〈} \AgdaBound{ρ1} \AgdaFunction{〉} \AgdaFunction{〈} \AgdaBound{ρ2} \AgdaFunction{〉} \AgdaFunction{〈} \AgdaBound{ρ3} \AgdaFunction{〉} \AgdaFunction{〈} \AgdaBound{ρ4} \AgdaFunction{〉}\<%
\\
\>\AgdaFunction{rep-comp₄} \AgdaSymbol{\{}\AgdaBound{U}\AgdaSymbol{\}} \AgdaSymbol{\{}\AgdaBound{V1}\AgdaSymbol{\}} \AgdaSymbol{\{}\AgdaBound{V2}\AgdaSymbol{\}} \AgdaSymbol{\{}\AgdaBound{V3}\AgdaSymbol{\}} \AgdaSymbol{\{}\AgdaBound{V4}\AgdaSymbol{\}} \AgdaSymbol{\{}\AgdaBound{ρ1}\AgdaSymbol{\}} \AgdaSymbol{\{}\AgdaBound{ρ2}\AgdaSymbol{\}} \AgdaSymbol{\{}\AgdaBound{ρ3}\AgdaSymbol{\}} \AgdaSymbol{\{}\AgdaBound{ρ4}\AgdaSymbol{\}} \AgdaSymbol{\{}\AgdaBound{C}\AgdaSymbol{\}} \AgdaSymbol{\{}\AgdaBound{K}\AgdaSymbol{\}} \AgdaBound{E} \AgdaSymbol{=} \<[66]%
\>[66]\<%
\\
\>[0]\AgdaIndent{2}{}\<[2]%
\>[2]\AgdaKeyword{let} \AgdaKeyword{open} \AgdaModule{≡-Reasoning} \AgdaKeyword{in} \<[26]%
\>[26]\<%
\\
\>[0]\AgdaIndent{2}{}\<[2]%
\>[2]\AgdaFunction{begin}\<%
\\
\>[2]\AgdaIndent{4}{}\<[4]%
\>[4]\AgdaBound{E} \AgdaFunction{〈} \AgdaBound{ρ4} \AgdaFunction{•R} \AgdaBound{ρ3} \AgdaFunction{•R} \AgdaBound{ρ2} \AgdaFunction{•R} \AgdaBound{ρ1} \AgdaFunction{〉}\<%
\\
\>[4]\AgdaIndent{6}{}\<[6]%
\>[6]\AgdaFunction{≡⟨} \AgdaFunction{rep-comp} \AgdaBound{E} \AgdaFunction{⟩}\<%
\\
\>[0]\AgdaIndent{4}{}\<[4]%
\>[4]\AgdaBound{E} \AgdaFunction{〈} \AgdaBound{ρ1} \AgdaFunction{〉} \AgdaFunction{〈} \AgdaBound{ρ4} \AgdaFunction{•R} \AgdaBound{ρ3} \AgdaFunction{•R} \AgdaBound{ρ2} \AgdaFunction{〉}\<%
\\
\>[4]\AgdaIndent{6}{}\<[6]%
\>[6]\AgdaFunction{≡⟨} \AgdaFunction{rep-comp} \AgdaSymbol{(}\AgdaBound{E} \AgdaFunction{〈} \AgdaBound{ρ1} \AgdaFunction{〉}\AgdaSymbol{)} \AgdaFunction{⟩}\<%
\\
\>[0]\AgdaIndent{4}{}\<[4]%
\>[4]\AgdaBound{E} \AgdaFunction{〈} \AgdaBound{ρ1} \AgdaFunction{〉} \AgdaFunction{〈} \AgdaBound{ρ2} \AgdaFunction{〉} \AgdaFunction{〈} \AgdaBound{ρ4} \AgdaFunction{•R} \AgdaBound{ρ3} \AgdaFunction{〉}\<%
\\
\>[4]\AgdaIndent{6}{}\<[6]%
\>[6]\AgdaFunction{≡⟨} \AgdaFunction{rep-comp} \AgdaSymbol{(}\AgdaBound{E} \AgdaFunction{〈} \AgdaBound{ρ1} \AgdaFunction{〉} \AgdaFunction{〈} \AgdaBound{ρ2} \AgdaFunction{〉}\AgdaSymbol{)} \AgdaFunction{⟩}\<%
\\
\>[0]\AgdaIndent{4}{}\<[4]%
\>[4]\AgdaBound{E} \AgdaFunction{〈} \AgdaBound{ρ1} \AgdaFunction{〉} \AgdaFunction{〈} \AgdaBound{ρ2} \AgdaFunction{〉} \AgdaFunction{〈} \AgdaBound{ρ3} \AgdaFunction{〉} \AgdaFunction{〈} \AgdaBound{ρ4} \AgdaFunction{〉}\<%
\\
\>[0]\AgdaIndent{2}{}\<[2]%
\>[2]\AgdaFunction{∎}\<%
\end{code}
}

We write $E \uparrow$ for $E \langle \uparrow \rangle$.

\begin{code}%
\>\AgdaKeyword{infixl} \AgdaNumber{70} \AgdaFixityOp{\_⇑}\<%
\\
\>\AgdaFunction{\_⇑} \AgdaSymbol{:} \AgdaSymbol{∀} \AgdaSymbol{\{}\AgdaBound{V}\AgdaSymbol{\}} \AgdaSymbol{\{}\AgdaBound{K}\AgdaSymbol{\}} \AgdaSymbol{\{}\AgdaBound{C}\AgdaSymbol{\}} \AgdaSymbol{\{}\AgdaBound{L}\AgdaSymbol{\}} \AgdaSymbol{→} \AgdaDatatype{Subexpression} \AgdaBound{V} \AgdaBound{C} \AgdaBound{L} \AgdaSymbol{→} \AgdaDatatype{Subexpression} \AgdaSymbol{(}\AgdaBound{V} \AgdaInductiveConstructor{,} \AgdaBound{K}\AgdaSymbol{)} \AgdaBound{C} \AgdaBound{L}\<%
\\
\>\AgdaBound{E} \AgdaFunction{⇑} \AgdaSymbol{=} \AgdaBound{E} \AgdaFunction{〈} \AgdaFunction{upRep} \AgdaFunction{〉}\<%
\end{code}

We define the unique replacement $\emptyset \rightarrow V$ for any V, and prove it unique:

\begin{code}%
\>\AgdaFunction{magic} \AgdaSymbol{:} \AgdaSymbol{∀} \AgdaSymbol{\{}\AgdaBound{V}\AgdaSymbol{\}} \AgdaSymbol{→} \AgdaFunction{Rep} \AgdaInductiveConstructor{∅} \AgdaBound{V}\<%
\\
\>\AgdaFunction{magic} \AgdaSymbol{\_} \AgdaSymbol{()}\<%
\\
%
\\
\>\AgdaFunction{magic-unique} \AgdaSymbol{:} \AgdaSymbol{∀} \AgdaSymbol{\{}\AgdaBound{V}\AgdaSymbol{\}} \AgdaSymbol{\{}\AgdaBound{ρ} \AgdaSymbol{:} \AgdaFunction{Rep} \AgdaInductiveConstructor{∅} \AgdaBound{V}\AgdaSymbol{\}} \AgdaSymbol{→} \AgdaBound{ρ} \AgdaFunction{∼R} \AgdaFunction{magic}\<%
\end{code}

\AgdaHide{
\begin{code}%
\>\AgdaFunction{magic-unique} \AgdaSymbol{\{}\AgdaBound{V}\AgdaSymbol{\}} \AgdaSymbol{\{}\AgdaBound{ρ}\AgdaSymbol{\}} \AgdaSymbol{()}\<%
\end{code}
}

\begin{code}%
\>\AgdaFunction{magic-unique'} \AgdaSymbol{:} \AgdaSymbol{∀} \AgdaSymbol{\{}\AgdaBound{U}\AgdaSymbol{\}} \AgdaSymbol{\{}\AgdaBound{V}\AgdaSymbol{\}} \AgdaSymbol{\{}\AgdaBound{C}\AgdaSymbol{\}} \AgdaSymbol{\{}\AgdaBound{K}\AgdaSymbol{\}}\<%
\\
\>[0]\AgdaIndent{2}{}\<[2]%
\>[2]\AgdaSymbol{(}\AgdaBound{E} \AgdaSymbol{:} \AgdaDatatype{Subexpression} \AgdaInductiveConstructor{∅} \AgdaBound{C} \AgdaBound{K}\AgdaSymbol{)} \AgdaSymbol{\{}\AgdaBound{ρ} \AgdaSymbol{:} \AgdaFunction{Rep} \AgdaBound{U} \AgdaBound{V}\AgdaSymbol{\}} \AgdaSymbol{→} \<[44]%
\>[44]\<%
\\
\>[0]\AgdaIndent{2}{}\<[2]%
\>[2]\AgdaBound{E} \AgdaFunction{〈} \AgdaFunction{magic} \AgdaFunction{〉} \AgdaFunction{〈} \AgdaBound{ρ} \AgdaFunction{〉} \AgdaDatatype{≡} \AgdaBound{E} \AgdaFunction{〈} \AgdaFunction{magic} \AgdaFunction{〉}\<%
\end{code}

\AgdaHide{
\begin{code}%
\>\AgdaFunction{magic-unique'} \AgdaBound{E} \AgdaSymbol{\{}\AgdaBound{ρ}\AgdaSymbol{\}} \AgdaSymbol{=} \AgdaKeyword{let} \AgdaKeyword{open} \AgdaModule{≡-Reasoning} \AgdaKeyword{in}\<%
\\
\>[0]\AgdaIndent{2}{}\<[2]%
\>[2]\AgdaFunction{begin}\<%
\\
\>[2]\AgdaIndent{4}{}\<[4]%
\>[4]\AgdaBound{E} \AgdaFunction{〈} \AgdaFunction{magic} \AgdaFunction{〉} \AgdaFunction{〈} \AgdaBound{ρ} \AgdaFunction{〉}\<%
\\
\>[0]\AgdaIndent{2}{}\<[2]%
\>[2]\AgdaFunction{≡⟨⟨} \AgdaFunction{rep-comp} \AgdaBound{E} \AgdaFunction{⟩⟩}\<%
\\
\>[2]\AgdaIndent{4}{}\<[4]%
\>[4]\AgdaBound{E} \AgdaFunction{〈} \AgdaBound{ρ} \AgdaFunction{•R} \AgdaFunction{magic} \AgdaFunction{〉}\<%
\\
\>[0]\AgdaIndent{2}{}\<[2]%
\>[2]\AgdaFunction{≡⟨} \AgdaFunction{rep-congr} \AgdaSymbol{(}\AgdaFunction{magic-unique} \AgdaSymbol{\{}\AgdaArgument{ρ} \AgdaSymbol{=} \AgdaBound{ρ} \AgdaFunction{•R} \AgdaFunction{magic}\AgdaSymbol{\})} \AgdaBound{E} \AgdaFunction{⟩}\<%
\\
\>[2]\AgdaIndent{4}{}\<[4]%
\>[4]\AgdaBound{E} \AgdaFunction{〈} \AgdaFunction{magic} \AgdaFunction{〉}\<%
\\
\>[0]\AgdaIndent{2}{}\<[2]%
\>[2]\AgdaFunction{∎}\<%
\\
%
\\
\>\AgdaFunction{liftRep-upRep₂} \AgdaSymbol{:} \AgdaSymbol{∀} \AgdaSymbol{\{}\AgdaBound{U}\AgdaSymbol{\}} \AgdaSymbol{\{}\AgdaBound{V}\AgdaSymbol{\}} \AgdaSymbol{\{}\AgdaBound{C}\AgdaSymbol{\}} \AgdaSymbol{\{}\AgdaBound{K}\AgdaSymbol{\}} \AgdaSymbol{\{}\AgdaBound{L}\AgdaSymbol{\}} \AgdaSymbol{\{}\AgdaBound{M}\AgdaSymbol{\}} \AgdaSymbol{(}\AgdaBound{E} \AgdaSymbol{:} \AgdaDatatype{Subexpression} \AgdaBound{U} \AgdaBound{C} \AgdaBound{M}\AgdaSymbol{)} \AgdaSymbol{\{}\AgdaBound{ρ} \AgdaSymbol{:} \AgdaFunction{Rep} \AgdaBound{U} \AgdaBound{V}\AgdaSymbol{\}} \AgdaSymbol{→} \AgdaBound{E} \AgdaFunction{⇑} \AgdaFunction{⇑} \AgdaFunction{〈} \AgdaFunction{liftRep} \AgdaBound{K} \AgdaSymbol{(}\AgdaFunction{liftRep} \AgdaBound{L} \AgdaBound{ρ}\AgdaSymbol{)} \AgdaFunction{〉} \AgdaDatatype{≡} \AgdaBound{E} \AgdaFunction{〈} \AgdaBound{ρ} \AgdaFunction{〉} \AgdaFunction{⇑} \AgdaFunction{⇑}\<%
\\
\>\AgdaFunction{liftRep-upRep₂} \AgdaSymbol{\{}\AgdaBound{U}\AgdaSymbol{\}} \AgdaSymbol{\{}\AgdaBound{V}\AgdaSymbol{\}} \AgdaSymbol{\{}\AgdaBound{C}\AgdaSymbol{\}} \AgdaSymbol{\{}\AgdaBound{K}\AgdaSymbol{\}} \AgdaSymbol{\{}\AgdaBound{L}\AgdaSymbol{\}} \AgdaSymbol{\{}\AgdaBound{M}\AgdaSymbol{\}} \AgdaBound{E} \AgdaSymbol{\{}\AgdaBound{ρ}\AgdaSymbol{\}} \AgdaSymbol{=} \AgdaKeyword{let} \AgdaKeyword{open} \AgdaModule{≡-Reasoning} \AgdaKeyword{in} \<[71]%
\>[71]\<%
\\
\>[0]\AgdaIndent{2}{}\<[2]%
\>[2]\AgdaFunction{begin}\<%
\\
\>[2]\AgdaIndent{4}{}\<[4]%
\>[4]\AgdaBound{E} \AgdaFunction{⇑} \AgdaFunction{⇑} \AgdaFunction{〈} \AgdaFunction{liftRep} \AgdaBound{K} \AgdaSymbol{(}\AgdaFunction{liftRep} \AgdaBound{L} \AgdaBound{ρ}\AgdaSymbol{)} \AgdaFunction{〉}\<%
\\
\>[0]\AgdaIndent{2}{}\<[2]%
\>[2]\AgdaFunction{≡⟨} \AgdaFunction{liftRep-upRep} \AgdaSymbol{(}\AgdaBound{E} \AgdaFunction{⇑}\AgdaSymbol{)} \AgdaFunction{⟩}\<%
\\
\>[2]\AgdaIndent{4}{}\<[4]%
\>[4]\AgdaBound{E} \AgdaFunction{⇑} \AgdaFunction{〈} \AgdaFunction{liftRep} \AgdaBound{L} \AgdaBound{ρ} \AgdaFunction{〉} \AgdaFunction{⇑}\<%
\\
\>[0]\AgdaIndent{2}{}\<[2]%
\>[2]\AgdaFunction{≡⟨} \AgdaFunction{rep-congl} \AgdaSymbol{(}\AgdaFunction{liftRep-upRep} \AgdaBound{E}\AgdaSymbol{)} \AgdaFunction{⟩}\<%
\\
\>[2]\AgdaIndent{4}{}\<[4]%
\>[4]\AgdaBound{E} \AgdaFunction{〈} \AgdaBound{ρ} \AgdaFunction{〉} \AgdaFunction{⇑} \AgdaFunction{⇑}\<%
\\
\>[0]\AgdaIndent{2}{}\<[2]%
\>[2]\AgdaFunction{∎}\<%
\\
%
\\
\>\AgdaFunction{liftRep-upRep₃} \AgdaSymbol{:} \AgdaSymbol{∀} \AgdaSymbol{\{}\AgdaBound{U}\AgdaSymbol{\}} \AgdaSymbol{\{}\AgdaBound{V}\AgdaSymbol{\}} \AgdaSymbol{\{}\AgdaBound{C}\AgdaSymbol{\}} \AgdaSymbol{\{}\AgdaBound{K}\AgdaSymbol{\}} \AgdaSymbol{\{}\AgdaBound{L}\AgdaSymbol{\}} \AgdaSymbol{\{}\AgdaBound{M}\AgdaSymbol{\}} \AgdaSymbol{\{}\AgdaBound{N}\AgdaSymbol{\}} \AgdaSymbol{(}\AgdaBound{E} \AgdaSymbol{:} \AgdaDatatype{Subexpression} \AgdaBound{U} \AgdaBound{C} \AgdaBound{N}\AgdaSymbol{)} \AgdaSymbol{\{}\AgdaBound{ρ} \AgdaSymbol{:} \AgdaFunction{Rep} \AgdaBound{U} \AgdaBound{V}\AgdaSymbol{\}} \AgdaSymbol{→} \<[89]%
\>[89]\<%
\\
\>[0]\AgdaIndent{2}{}\<[2]%
\>[2]\AgdaBound{E} \AgdaFunction{⇑} \AgdaFunction{⇑} \AgdaFunction{⇑} \AgdaFunction{〈} \AgdaFunction{liftRep} \AgdaBound{K} \AgdaSymbol{(}\AgdaFunction{liftRep} \AgdaBound{L} \AgdaSymbol{(}\AgdaFunction{liftRep} \AgdaBound{M} \AgdaBound{ρ}\AgdaSymbol{))} \AgdaFunction{〉} \AgdaDatatype{≡} \AgdaBound{E} \AgdaFunction{〈} \AgdaBound{ρ} \AgdaFunction{〉} \AgdaFunction{⇑} \AgdaFunction{⇑} \AgdaFunction{⇑}\<%
\\
\>\AgdaFunction{liftRep-upRep₃} \AgdaSymbol{\{}\AgdaBound{U}\AgdaSymbol{\}} \AgdaSymbol{\{}\AgdaBound{V}\AgdaSymbol{\}} \AgdaSymbol{\{}\AgdaBound{C}\AgdaSymbol{\}} \AgdaSymbol{\{}\AgdaBound{K}\AgdaSymbol{\}} \AgdaSymbol{\{}\AgdaBound{L}\AgdaSymbol{\}} \AgdaSymbol{\{}\AgdaBound{M}\AgdaSymbol{\}} \AgdaBound{E} \AgdaSymbol{\{}\AgdaBound{ρ}\AgdaSymbol{\}} \AgdaSymbol{=} \AgdaKeyword{let} \AgdaKeyword{open} \AgdaModule{≡-Reasoning} \AgdaKeyword{in} \<[71]%
\>[71]\<%
\\
\>[0]\AgdaIndent{2}{}\<[2]%
\>[2]\AgdaFunction{begin}\<%
\\
\>[2]\AgdaIndent{4}{}\<[4]%
\>[4]\AgdaBound{E} \AgdaFunction{⇑} \AgdaFunction{⇑} \AgdaFunction{⇑} \AgdaFunction{〈} \AgdaFunction{liftRep} \AgdaBound{K} \AgdaSymbol{(}\AgdaFunction{liftRep} \AgdaBound{L} \AgdaSymbol{(}\AgdaFunction{liftRep} \AgdaBound{M} \AgdaBound{ρ}\AgdaSymbol{))} \AgdaFunction{〉}\<%
\\
\>[0]\AgdaIndent{2}{}\<[2]%
\>[2]\AgdaFunction{≡⟨} \AgdaFunction{liftRep-upRep₂} \AgdaSymbol{(}\AgdaBound{E} \AgdaFunction{⇑}\AgdaSymbol{)} \AgdaFunction{⟩}\<%
\\
\>[2]\AgdaIndent{4}{}\<[4]%
\>[4]\AgdaBound{E} \AgdaFunction{⇑} \AgdaFunction{〈} \AgdaFunction{liftRep} \AgdaBound{M} \AgdaBound{ρ} \AgdaFunction{〉} \AgdaFunction{⇑} \AgdaFunction{⇑}\<%
\\
\>[0]\AgdaIndent{2}{}\<[2]%
\>[2]\AgdaFunction{≡⟨} \AgdaFunction{rep-congl} \AgdaSymbol{(}\AgdaFunction{rep-congl} \AgdaSymbol{(}\AgdaFunction{liftRep-upRep} \AgdaBound{E}\AgdaSymbol{))} \AgdaFunction{⟩}\<%
\\
\>[2]\AgdaIndent{4}{}\<[4]%
\>[4]\AgdaBound{E} \AgdaFunction{〈} \AgdaBound{ρ} \AgdaFunction{〉} \AgdaFunction{⇑} \AgdaFunction{⇑} \AgdaFunction{⇑}\<%
\\
\>[0]\AgdaIndent{2}{}\<[2]%
\>[2]\AgdaFunction{∎}\<%
\\
%
\\
\>\AgdaKeyword{postulate} \AgdaPostulate{liftRep-upRep₄'} \AgdaSymbol{:} \AgdaSymbol{∀} \AgdaSymbol{\{}\AgdaBound{U}\AgdaSymbol{\}} \AgdaSymbol{\{}\AgdaBound{V}\AgdaSymbol{\}} \AgdaSymbol{(}\AgdaBound{ρ} \AgdaSymbol{:} \AgdaFunction{Rep} \AgdaBound{U} \AgdaBound{V}\AgdaSymbol{)} \AgdaSymbol{\{}\AgdaBound{K1}\AgdaSymbol{\}} \AgdaSymbol{\{}\AgdaBound{K2}\AgdaSymbol{\}} \AgdaSymbol{\{}\AgdaBound{K3}\AgdaSymbol{\}} \AgdaSymbol{→} \AgdaFunction{upRep} \AgdaFunction{•R} \AgdaFunction{upRep} \AgdaFunction{•R} \AgdaFunction{upRep} \AgdaFunction{•R} \AgdaBound{ρ} \AgdaFunction{∼R} \AgdaFunction{liftRep} \AgdaBound{K1} \AgdaSymbol{(}\AgdaFunction{liftRep} \AgdaBound{K2} \AgdaSymbol{(}\AgdaFunction{liftRep} \AgdaBound{K3} \AgdaBound{ρ}\AgdaSymbol{))} \AgdaFunction{•R} \AgdaFunction{upRep} \AgdaFunction{•R} \AgdaFunction{upRep} \AgdaFunction{•R} \AgdaFunction{upRep}\<%
\end{code}
}

\AgdaHide{
\begin{code}%
\>\AgdaKeyword{open} \AgdaKeyword{import} \AgdaModule{Grammar.Replacement} \AgdaBound{G} \AgdaKeyword{public}\<%
\end{code}
}

\AgdaHide{
\begin{code}%
\>\AgdaKeyword{open} \AgdaKeyword{import} \AgdaModule{Prelims}\<%
\\
\>\AgdaKeyword{open} \AgdaKeyword{import} \AgdaModule{Grammar.Base}\<%
\\
%
\\
\>\AgdaKeyword{module} \AgdaModule{Grammar.Replacement.SetFunctor} \AgdaSymbol{(}\AgdaBound{G} \AgdaSymbol{:} \AgdaRecord{Grammar}\AgdaSymbol{)} \AgdaKeyword{where}\<%
\\
\>\AgdaKeyword{open} \AgdaModule{Grammar} \AgdaBound{G}\<%
\\
\>\AgdaKeyword{open} \AgdaKeyword{import} \AgdaModule{Grammar.Replacement} \AgdaBound{G}\<%
\end{code}
}

Let $\mathbb{A}$ be the category of alphabets and replacements.  We introduce the type of
all functors $\mathbb{A} \rightarrow \mathbf{Set}$:

\begin{code}%
\>\AgdaKeyword{record} \AgdaRecord{SetFunctor} \AgdaSymbol{:} \AgdaPrimitiveType{Set₁} \AgdaKeyword{where}\<%
\\
\>[0]\AgdaIndent{2}{}\<[2]%
\>[2]\AgdaKeyword{field}\<%
\\
\>[2]\AgdaIndent{4}{}\<[4]%
\>[4]\AgdaField{Fib} \AgdaSymbol{:} \AgdaDatatype{Alphabet} \AgdaSymbol{→} \AgdaPrimitiveType{Set}\<%
\\
\>[2]\AgdaIndent{4}{}\<[4]%
\>[4]\AgdaField{\_〈〈\_〉〉} \AgdaSymbol{:} \AgdaSymbol{∀} \AgdaSymbol{\{}\AgdaBound{U} \AgdaBound{V}\AgdaSymbol{\}} \AgdaSymbol{→} \AgdaField{Fib} \AgdaBound{U} \AgdaSymbol{→} \AgdaFunction{Rep} \AgdaBound{U} \AgdaBound{V} \AgdaSymbol{→} \AgdaField{Fib} \AgdaBound{V}\<%
\\
\>[2]\AgdaIndent{4}{}\<[4]%
\>[4]\AgdaField{〈〈〉〉-id} \AgdaSymbol{:} \AgdaSymbol{∀} \AgdaSymbol{\{}\AgdaBound{V}\AgdaSymbol{\}} \AgdaSymbol{\{}\AgdaBound{a} \AgdaSymbol{:} \AgdaField{Fib} \AgdaBound{V}\AgdaSymbol{\}} \AgdaSymbol{→} \AgdaBound{a} \AgdaField{〈〈} \AgdaFunction{idRep} \AgdaBound{V} \AgdaField{〉〉} \AgdaDatatype{≡} \AgdaBound{a}\<%
\\
\>[2]\AgdaIndent{4}{}\<[4]%
\>[4]\AgdaField{〈〈〉〉-comp} \AgdaSymbol{:} \AgdaSymbol{∀} \AgdaSymbol{\{}\AgdaBound{U} \AgdaBound{V} \AgdaBound{W}\AgdaSymbol{\}} \AgdaSymbol{\{}\AgdaBound{ρ} \AgdaSymbol{:} \AgdaFunction{Rep} \AgdaBound{V} \AgdaBound{W}\AgdaSymbol{\}} \AgdaSymbol{\{}\AgdaBound{σ} \AgdaSymbol{:} \AgdaFunction{Rep} \AgdaBound{U} \AgdaBound{V}\AgdaSymbol{\}} \AgdaSymbol{\{}\AgdaBound{a} \AgdaSymbol{:} \AgdaField{Fib} \AgdaBound{U}\AgdaSymbol{\}} \AgdaSymbol{→} \<[68]%
\>[68]\<%
\\
\>[4]\AgdaIndent{6}{}\<[6]%
\>[6]\AgdaBound{a} \AgdaField{〈〈} \AgdaBound{ρ} \AgdaFunction{•R} \AgdaBound{σ} \AgdaField{〉〉} \AgdaDatatype{≡} \AgdaBound{a} \AgdaField{〈〈} \AgdaBound{σ} \AgdaField{〉〉} \AgdaField{〈〈} \AgdaBound{ρ} \AgdaField{〉〉}\<%
\end{code}

\begin{lemma}
For any kind $K$, the operation $\AgdaKeyword{Var} \, - \, K$ is a functor $\mathbb{A} \rightarrow \mathbf{Set}$.
\end{lemma}

\begin{code}%
\>\AgdaFunction{VAR} \AgdaSymbol{:} \AgdaFunction{VarKind} \AgdaSymbol{→} \AgdaRecord{SetFunctor}\<%
\\
\>\AgdaFunction{VAR} \AgdaBound{K} \AgdaSymbol{=} \AgdaKeyword{record} \AgdaSymbol{\{} \<[17]%
\>[17]\<%
\\
\>[0]\AgdaIndent{2}{}\<[2]%
\>[2]\AgdaField{Fib} \AgdaSymbol{=} \AgdaSymbol{λ} \AgdaBound{V} \AgdaSymbol{→} \AgdaDatatype{Var} \AgdaBound{V} \AgdaBound{K} \AgdaSymbol{;} \<[24]%
\>[24]\<%
\\
\>[0]\AgdaIndent{2}{}\<[2]%
\>[2]\AgdaField{\_〈〈\_〉〉} \AgdaSymbol{=} \AgdaSymbol{λ} \AgdaBound{x} \AgdaBound{ρ} \AgdaSymbol{→} \AgdaBound{ρ} \AgdaBound{K} \AgdaBound{x} \AgdaSymbol{;} \<[27]%
\>[27]\<%
\\
\>[0]\AgdaIndent{2}{}\<[2]%
\>[2]\AgdaField{〈〈〉〉-id} \AgdaSymbol{=} \AgdaInductiveConstructor{refl} \AgdaSymbol{;} \<[19]%
\>[19]\<%
\\
\>[0]\AgdaIndent{2}{}\<[2]%
\>[2]\AgdaField{〈〈〉〉-comp} \AgdaSymbol{=} \AgdaInductiveConstructor{refl} \AgdaSymbol{\}}\<%
\end{code}

\AgdaHide{
\begin{code}%
\>\AgdaKeyword{open} \AgdaKeyword{import} \AgdaModule{Grammar.Replacement.SetFunctor} \AgdaBound{G} \AgdaKeyword{public}\<%
\end{code}
}

\AgdaHide{
\begin{code}%
\>\AgdaKeyword{open} \AgdaKeyword{import} \AgdaModule{Grammar.Base}\<%
\\
%
\\
\>\AgdaKeyword{module} \AgdaModule{Grammar.Substitution} \AgdaSymbol{(}\AgdaBound{G} \AgdaSymbol{:} \AgdaRecord{Grammar}\AgdaSymbol{)} \AgdaKeyword{where}\<%
\end{code}
}

\AgdaHide{
\begin{code}%
\>\AgdaKeyword{open} \AgdaKeyword{import} \AgdaModule{Grammar.Base}\<%
\\
%
\\
\>\AgdaKeyword{module} \AgdaModule{Grammar.OpFamily.PreOpFamily} \AgdaSymbol{(}\AgdaBound{G} \AgdaSymbol{:} \AgdaRecord{Grammar}\AgdaSymbol{)} \AgdaKeyword{where}\<%
\\
\>\AgdaKeyword{open} \AgdaKeyword{import} \AgdaModule{Prelims}\<%
\\
\>\AgdaKeyword{open} \AgdaModule{Grammar} \AgdaBound{G}\<%
\end{code}
}

\subsection{Families of Operations}

Our aim here is to define the operations of \emph{replacement} and \emph{substitution}.  In order to organise this work, we introduce the following definitions.

A \emph{family of operations} over a grammar $G$ consists of:
\begin{enumerate}
\item
for any alphabets $U$ and $V$, a set $F[U,V]$ of \emph{operations} $\sigma$ from $U$ to $V$, $\sigma : U \rightarrow V$;
\item
for any operation $\sigma : U \rightarrow V$ and variable $x \in U$ of kind $K$, an expression $\sigma(x)$ over $V$ of kind $K$;
\item
for any alphabet $V$ and variable kind $K$, an operation $\uparrow : V \rightarrow (V , K)$, the \emph{lifting} operation;
\item
for any alphabet $V$, an operation $\id{V} : V \rightarrow V$, the \emph{identity} operation;
\item
for any operation $\sigma : U \rightarrow V$ and variable kind $K$, an operation $(\sigma , K) : (U , K) \rightarrow (V , K)$, the result of \emph{lifting} $\sigma$;
\item
for any operations $\rho : U \rightarrow V$ and $\sigma : V \rightarrow W$, an operation
$\sigma \circ \rho : U \rightarrow W$, the \emph{composition} of $\sigma$ and $\rho$;
\end{enumerate}
such that:
\begin{itemize}
\item
$\uparrow (x) \equiv x$
\item
$\id{V}(x) \equiv x$
\item
If $\rho \sim \sigma$ then $(\rho , K) \sim (\sigma , K)$
\item
$(\rho , K)(x_0) \equiv x_0$
\item
Given $\sigma : U \rightarrow V$ and $x \in U$, we have $(\sigma , K)(x) \equiv x$
\item
$(\sigma \circ \rho , K) \sim (\sigma , K) \circ (\rho , K)$
\item
$(\sigma \circ \rho)(x) \equiv \rho(x) [ \sigma ]$
\end{itemize}
where for $\sigma, \rho : U \rightarrow V$ we write $\sigma \sim \rho$ iff $\sigma(x) \equiv \rho(x)$ for all $x \in U$; and, given $\sigma : U \rightarrow V$ and $E$ an expression over $U$, we define $E[\sigma]$, the result of \emph{applying} the operation $\sigma$ to $E$, as follows:

\begin{align*}
x[\sigma] & \eqdef \sigma(x) \\
\lefteqn{c([\vec{x_1}] E_1, \ldots, [\vec{x_n}] E_n) [\sigma]} \\
 & \eqdef
c([\vec{x_1}] E_1 [(\sigma , K_{11}, \ldots, K_{1r_1})], \ldots,
[\vec{x_n}] E_n [(\sigma, K_{n1}, \ldots, K_{nr_n})])
\end{align*}
for $c$ a constructor of type (\ref{eq:conkind}).

\subsubsection{Pre-Families}
We formalize this definition in stages.  First, we define a \emph{pre-family of operations} to be a family with items of data 1--4 above that satisfies the first two axioms:

\begin{code}%
\>\AgdaKeyword{record} \AgdaRecord{PreOpFamily} \AgdaSymbol{:} \AgdaPrimitiveType{Set₂} \AgdaKeyword{where}\<%
\\
\>[0]\AgdaIndent{2}{}\<[2]%
\>[2]\AgdaKeyword{field}\<%
\\
\>[2]\AgdaIndent{4}{}\<[4]%
\>[4]\AgdaField{Op} \AgdaSymbol{:} \AgdaDatatype{Alphabet} \AgdaSymbol{→} \AgdaDatatype{Alphabet} \AgdaSymbol{→} \AgdaPrimitiveType{Set}\<%
\\
\>[2]\AgdaIndent{4}{}\<[4]%
\>[4]\AgdaField{apV} \AgdaSymbol{:} \AgdaSymbol{∀} \AgdaSymbol{\{}\AgdaBound{U}\AgdaSymbol{\}} \AgdaSymbol{\{}\AgdaBound{V}\AgdaSymbol{\}} \AgdaSymbol{\{}\AgdaBound{K}\AgdaSymbol{\}} \AgdaSymbol{→} \AgdaField{Op} \AgdaBound{U} \AgdaBound{V} \AgdaSymbol{→} \AgdaDatatype{Var} \AgdaBound{U} \AgdaBound{K} \AgdaSymbol{→} \AgdaFunction{VExpression} \AgdaBound{V} \AgdaBound{K}\<%
\\
\>[2]\AgdaIndent{4}{}\<[4]%
\>[4]\AgdaField{up} \AgdaSymbol{:} \AgdaSymbol{∀} \AgdaSymbol{\{}\AgdaBound{V}\AgdaSymbol{\}} \AgdaSymbol{\{}\AgdaBound{K}\AgdaSymbol{\}} \AgdaSymbol{→} \AgdaField{Op} \AgdaBound{V} \AgdaSymbol{(}\AgdaBound{V} \AgdaInductiveConstructor{,} \AgdaBound{K}\AgdaSymbol{)}\<%
\\
\>[2]\AgdaIndent{4}{}\<[4]%
\>[4]\AgdaField{apV-up} \AgdaSymbol{:} \AgdaSymbol{∀} \AgdaSymbol{\{}\AgdaBound{V}\AgdaSymbol{\}} \AgdaSymbol{\{}\AgdaBound{K}\AgdaSymbol{\}} \AgdaSymbol{\{}\AgdaBound{L}\AgdaSymbol{\}} \AgdaSymbol{\{}\AgdaBound{x} \AgdaSymbol{:} \AgdaDatatype{Var} \AgdaBound{V} \AgdaBound{K}\AgdaSymbol{\}} \AgdaSymbol{→} \AgdaField{apV} \AgdaSymbol{(}\AgdaField{up} \AgdaSymbol{\{}\AgdaArgument{K} \AgdaSymbol{=} \AgdaBound{L}\AgdaSymbol{\})} \AgdaBound{x} \AgdaDatatype{≡} \AgdaInductiveConstructor{var} \AgdaSymbol{(}\AgdaInductiveConstructor{↑} \AgdaBound{x}\AgdaSymbol{)}\<%
\\
\>[2]\AgdaIndent{4}{}\<[4]%
\>[4]\AgdaField{idOp} \AgdaSymbol{:} \AgdaSymbol{∀} \AgdaBound{V} \AgdaSymbol{→} \AgdaField{Op} \AgdaBound{V} \AgdaBound{V}\<%
\\
\>[2]\AgdaIndent{4}{}\<[4]%
\>[4]\AgdaField{apV-idOp} \AgdaSymbol{:} \AgdaSymbol{∀} \AgdaSymbol{\{}\AgdaBound{V}\AgdaSymbol{\}} \AgdaSymbol{\{}\AgdaBound{K}\AgdaSymbol{\}} \AgdaSymbol{(}\AgdaBound{x} \AgdaSymbol{:} \AgdaDatatype{Var} \AgdaBound{V} \AgdaBound{K}\AgdaSymbol{)} \AgdaSymbol{→} \AgdaField{apV} \AgdaSymbol{(}\AgdaField{idOp} \AgdaBound{V}\AgdaSymbol{)} \AgdaBound{x} \AgdaDatatype{≡} \AgdaInductiveConstructor{var} \AgdaBound{x}\<%
\end{code}

This allows us to define the relation $\sim$, and prove it is an equivalence relation:

\begin{code}%
\>[0]\AgdaIndent{2}{}\<[2]%
\>[2]\AgdaFunction{\_∼op\_} \AgdaSymbol{:} \AgdaSymbol{∀} \AgdaSymbol{\{}\AgdaBound{U}\AgdaSymbol{\}} \AgdaSymbol{\{}\AgdaBound{V}\AgdaSymbol{\}} \AgdaSymbol{→} \AgdaField{Op} \AgdaBound{U} \AgdaBound{V} \AgdaSymbol{→} \AgdaField{Op} \AgdaBound{U} \AgdaBound{V} \AgdaSymbol{→} \AgdaPrimitiveType{Set}\<%
\\
\>[0]\AgdaIndent{2}{}\<[2]%
\>[2]\AgdaFunction{\_∼op\_} \AgdaSymbol{\{}\AgdaBound{U}\AgdaSymbol{\}} \AgdaSymbol{\{}\AgdaBound{V}\AgdaSymbol{\}} \AgdaBound{ρ} \AgdaBound{σ} \AgdaSymbol{=} \AgdaSymbol{∀} \AgdaSymbol{\{}\AgdaBound{K}\AgdaSymbol{\}} \AgdaSymbol{(}\AgdaBound{x} \AgdaSymbol{:} \AgdaDatatype{Var} \AgdaBound{U} \AgdaBound{K}\AgdaSymbol{)} \AgdaSymbol{→} \AgdaField{apV} \AgdaBound{ρ} \AgdaBound{x} \AgdaDatatype{≡} \AgdaField{apV} \AgdaBound{σ} \AgdaBound{x}\<%
\\
\>[2]\AgdaIndent{4}{}\<[4]%
\>[4]\<%
\\
\>[0]\AgdaIndent{2}{}\<[2]%
\>[2]\AgdaFunction{∼-refl} \AgdaSymbol{:} \AgdaSymbol{∀} \AgdaSymbol{\{}\AgdaBound{U}\AgdaSymbol{\}} \AgdaSymbol{\{}\AgdaBound{V}\AgdaSymbol{\}} \AgdaSymbol{\{}\AgdaBound{σ} \AgdaSymbol{:} \AgdaField{Op} \AgdaBound{U} \AgdaBound{V}\AgdaSymbol{\}} \AgdaSymbol{→} \AgdaBound{σ} \AgdaFunction{∼op} \AgdaBound{σ}\<%
\\
\>[0]\AgdaIndent{2}{}\<[2]%
\>[2]\AgdaFunction{∼-refl} \AgdaSymbol{\_} \AgdaSymbol{=} \AgdaInductiveConstructor{refl}\<%
\\
\>[2]\AgdaIndent{4}{}\<[4]%
\>[4]\<%
\\
\>[0]\AgdaIndent{2}{}\<[2]%
\>[2]\AgdaFunction{∼-sym} \AgdaSymbol{:} \AgdaSymbol{∀} \AgdaSymbol{\{}\AgdaBound{U}\AgdaSymbol{\}} \AgdaSymbol{\{}\AgdaBound{V}\AgdaSymbol{\}} \AgdaSymbol{\{}\AgdaBound{σ} \AgdaBound{τ} \AgdaSymbol{:} \AgdaField{Op} \AgdaBound{U} \AgdaBound{V}\AgdaSymbol{\}} \AgdaSymbol{→} \AgdaBound{σ} \AgdaFunction{∼op} \AgdaBound{τ} \AgdaSymbol{→} \AgdaBound{τ} \AgdaFunction{∼op} \AgdaBound{σ}\<%
\\
\>[0]\AgdaIndent{2}{}\<[2]%
\>[2]\AgdaFunction{∼-sym} \AgdaBound{σ-is-τ} \AgdaBound{x} \AgdaSymbol{=} \AgdaFunction{sym} \AgdaSymbol{(}\AgdaBound{σ-is-τ} \AgdaBound{x}\AgdaSymbol{)}\<%
\\
%
\\
\>[0]\AgdaIndent{2}{}\<[2]%
\>[2]\AgdaFunction{∼-trans} \AgdaSymbol{:} \AgdaSymbol{∀} \AgdaSymbol{\{}\AgdaBound{U}\AgdaSymbol{\}} \AgdaSymbol{\{}\AgdaBound{V}\AgdaSymbol{\}} \AgdaSymbol{\{}\AgdaBound{ρ} \AgdaBound{σ} \AgdaBound{τ} \AgdaSymbol{:} \AgdaField{Op} \AgdaBound{U} \AgdaBound{V}\AgdaSymbol{\}} \AgdaSymbol{→} \AgdaBound{ρ} \AgdaFunction{∼op} \AgdaBound{σ} \AgdaSymbol{→} \AgdaBound{σ} \AgdaFunction{∼op} \AgdaBound{τ} \AgdaSymbol{→} \AgdaBound{ρ} \AgdaFunction{∼op} \AgdaBound{τ}\<%
\\
\>[0]\AgdaIndent{2}{}\<[2]%
\>[2]\AgdaFunction{∼-trans} \AgdaBound{ρ-is-σ} \AgdaBound{σ-is-τ} \AgdaBound{x} \AgdaSymbol{=} \AgdaFunction{trans} \AgdaSymbol{(}\AgdaBound{ρ-is-σ} \AgdaBound{x}\AgdaSymbol{)} \AgdaSymbol{(}\AgdaBound{σ-is-τ} \AgdaBound{x}\AgdaSymbol{)}\<%
\\
%
\\
\>[0]\AgdaIndent{2}{}\<[2]%
\>[2]\AgdaFunction{OP} \AgdaSymbol{:} \AgdaDatatype{Alphabet} \AgdaSymbol{→} \AgdaDatatype{Alphabet} \AgdaSymbol{→} \<[30]%
\>[30]\AgdaRecord{Setoid} \AgdaSymbol{\_} \AgdaSymbol{\_}\<%
\\
\>[0]\AgdaIndent{2}{}\<[2]%
\>[2]\AgdaFunction{OP} \AgdaBound{U} \AgdaBound{V} \AgdaSymbol{=} \AgdaKeyword{record} \AgdaSymbol{\{} \<[20]%
\>[20]\<%
\\
\>[2]\AgdaIndent{5}{}\<[5]%
\>[5]\AgdaField{Carrier} \AgdaSymbol{=} \AgdaField{Op} \AgdaBound{U} \AgdaBound{V} \AgdaSymbol{;} \<[24]%
\>[24]\<%
\\
\>[2]\AgdaIndent{5}{}\<[5]%
\>[5]\AgdaField{\_≈\_} \AgdaSymbol{=} \AgdaFunction{\_∼op\_} \AgdaSymbol{;} \<[19]%
\>[19]\<%
\\
\>[2]\AgdaIndent{5}{}\<[5]%
\>[5]\AgdaField{isEquivalence} \AgdaSymbol{=} \AgdaKeyword{record} \AgdaSymbol{\{} \<[30]%
\>[30]\<%
\\
\>[5]\AgdaIndent{7}{}\<[7]%
\>[7]\AgdaField{refl} \AgdaSymbol{=} \AgdaFunction{∼-refl} \AgdaSymbol{;} \<[23]%
\>[23]\<%
\\
\>[5]\AgdaIndent{7}{}\<[7]%
\>[7]\AgdaField{sym} \AgdaSymbol{=} \AgdaFunction{∼-sym} \AgdaSymbol{;} \<[21]%
\>[21]\<%
\\
\>[5]\AgdaIndent{7}{}\<[7]%
\>[7]\AgdaField{trans} \AgdaSymbol{=} \AgdaFunction{∼-trans} \AgdaSymbol{\}} \AgdaSymbol{\}}\<%
\end{code}


\AgdaHide{
\begin{code}%
\>\AgdaKeyword{open} \AgdaKeyword{import} \AgdaModule{Grammar.Substitution.PreOpFamily} \AgdaBound{G} \AgdaKeyword{public}\<%
\end{code}
}

\AgdaHide{
\begin{code}%
\>\AgdaKeyword{open} \AgdaKeyword{import} \AgdaModule{Grammar.Base}\<%
\\
%
\\
\>\AgdaKeyword{module} \AgdaModule{Grammar.OpFamily.Lifting} \AgdaSymbol{(}\AgdaBound{G} \AgdaSymbol{:} \AgdaRecord{Grammar}\AgdaSymbol{)} \AgdaKeyword{where}\<%
\\
\>\AgdaKeyword{open} \AgdaKeyword{import} \AgdaModule{Data.List}\<%
\\
\>\AgdaKeyword{open} \AgdaKeyword{import} \AgdaModule{Prelims}\<%
\\
\>\AgdaKeyword{open} \AgdaModule{Grammar} \AgdaBound{G}\<%
\\
\>\AgdaKeyword{open} \AgdaKeyword{import} \AgdaModule{Grammar.OpFamily.PreOpFamily} \AgdaBound{G}\<%
\end{code}
}

\subsubsection{Liftings}

Define a \emph{lifting} on a pre-family to be an function $(- , K)$ that respects $\sim$:

\begin{code}%
\>\AgdaKeyword{record} \AgdaRecord{Lifting} \AgdaSymbol{(}\AgdaBound{F} \AgdaSymbol{:} \AgdaRecord{PreOpFamily}\AgdaSymbol{)} \AgdaSymbol{:} \AgdaPrimitiveType{Set₁} \AgdaKeyword{where}\<%
\\
\>[0]\AgdaIndent{2}{}\<[2]%
\>[2]\AgdaKeyword{open} \AgdaModule{PreOpFamily} \AgdaBound{F}\<%
\\
\>[0]\AgdaIndent{2}{}\<[2]%
\>[2]\AgdaKeyword{field}\<%
\\
\>[2]\AgdaIndent{4}{}\<[4]%
\>[4]\AgdaField{liftOp} \AgdaSymbol{:} \AgdaSymbol{∀} \AgdaSymbol{\{}\AgdaBound{U}\AgdaSymbol{\}} \AgdaSymbol{\{}\AgdaBound{V}\AgdaSymbol{\}} \AgdaBound{K} \AgdaSymbol{→} \AgdaFunction{Op} \AgdaBound{U} \AgdaBound{V} \AgdaSymbol{→} \AgdaFunction{Op} \AgdaSymbol{(}\AgdaBound{U} \AgdaInductiveConstructor{,} \AgdaBound{K}\AgdaSymbol{)} \AgdaSymbol{(}\AgdaBound{V} \AgdaInductiveConstructor{,} \AgdaBound{K}\AgdaSymbol{)}\<%
\\
\>[2]\AgdaIndent{4}{}\<[4]%
\>[4]\AgdaField{liftOp-cong} \AgdaSymbol{:} \AgdaSymbol{∀} \AgdaSymbol{\{}\AgdaBound{V}\AgdaSymbol{\}} \AgdaSymbol{\{}\AgdaBound{W}\AgdaSymbol{\}} \AgdaSymbol{\{}\AgdaBound{K}\AgdaSymbol{\}} \AgdaSymbol{\{}\AgdaBound{ρ} \AgdaBound{σ} \AgdaSymbol{:} \AgdaFunction{Op} \AgdaBound{V} \AgdaBound{W}\AgdaSymbol{\}} \AgdaSymbol{→} \<[49]%
\>[49]\<%
\\
\>[4]\AgdaIndent{6}{}\<[6]%
\>[6]\AgdaBound{ρ} \AgdaFunction{∼op} \AgdaBound{σ} \AgdaSymbol{→} \AgdaField{liftOp} \AgdaBound{K} \AgdaBound{ρ} \AgdaFunction{∼op} \AgdaField{liftOp} \AgdaBound{K} \AgdaBound{σ}\<%
\end{code}

Given an operation $\sigma : U \rightarrow V$ and a list of variable kinds $A \equiv (A_1 , \ldots , A_n)$, define
the \emph{repeated lifting} $\sigma^A$ to be $((\cdots(\sigma , A_1) , A_2) , \cdots ) , A_n)$.

\begin{code}%
\>\AgdaComment{\{-  liftOp' : ∀ \{U\} \{V\} A → Op U V → Op (extend U A) (extend V A)\<\\
\>  liftOp' [] σ = σ\<\\
\>  liftOp' (K ∷ A) σ = liftOp' A (liftOp K σ) -\}}\<%
\\
%
\\
\>[0]\AgdaIndent{2}{}\<[2]%
\>[2]\AgdaFunction{liftOp''} \AgdaSymbol{:} \AgdaSymbol{∀} \AgdaSymbol{\{}\AgdaBound{U}\AgdaSymbol{\}} \AgdaSymbol{\{}\AgdaBound{V}\AgdaSymbol{\}} \AgdaSymbol{\{}\AgdaBound{K}\AgdaSymbol{\}} \AgdaBound{A} \AgdaSymbol{→} \AgdaFunction{Op} \AgdaBound{U} \AgdaBound{V} \AgdaSymbol{→} \AgdaFunction{Op} \AgdaSymbol{(}\AgdaFunction{dom} \AgdaBound{U} \AgdaSymbol{\{}\AgdaBound{K}\AgdaSymbol{\}} \AgdaBound{A}\AgdaSymbol{)} \AgdaSymbol{(}\AgdaFunction{dom} \AgdaBound{V} \AgdaBound{A}\AgdaSymbol{)}\<%
\\
\>[0]\AgdaIndent{2}{}\<[2]%
\>[2]\AgdaFunction{liftOp''} \AgdaSymbol{(\_} \AgdaInductiveConstructor{●}\AgdaSymbol{)} \AgdaBound{σ} \AgdaSymbol{=} \AgdaBound{σ}\<%
\\
\>[0]\AgdaIndent{2}{}\<[2]%
\>[2]\AgdaFunction{liftOp''} \AgdaSymbol{(}\AgdaBound{K} \AgdaInductiveConstructor{⟶} \AgdaBound{A}\AgdaSymbol{)} \AgdaBound{σ} \AgdaSymbol{=} \AgdaFunction{liftOp''} \AgdaBound{A} \AgdaSymbol{(}\AgdaField{liftOp} \AgdaBound{K} \AgdaBound{σ}\AgdaSymbol{)}\<%
\\
%
\\
\>\AgdaComment{\{-  liftOp'-cong : ∀ \{U\} \{V\} A \{ρ σ : Op U V\} → \<\\
\>    ρ ∼op σ → liftOp' A ρ ∼op liftOp' A σ\<\\
\>}\<%
\end{code}

\AgdaHide{
\begin{code}%
\>\AgdaComment{\<\\
\>  liftOp'-cong [] ρ-is-σ = ρ-is-σ\<\\
\>  liftOp'-cong (\_ ∷ A) ρ-is-σ = liftOp'-cong A (liftOp-cong ρ-is-σ) -\}}\<%
\\
%
\\
\>[0]\AgdaIndent{2}{}\<[2]%
\>[2]\AgdaKeyword{postulate} \AgdaPostulate{liftOp''-cong} \AgdaSymbol{:} \AgdaSymbol{∀} \AgdaSymbol{\{}\AgdaBound{U}\AgdaSymbol{\}} \AgdaSymbol{\{}\AgdaBound{V}\AgdaSymbol{\}} \AgdaSymbol{\{}\AgdaBound{K}\AgdaSymbol{\}} \AgdaBound{A} \AgdaSymbol{\{}\AgdaBound{ρ} \AgdaBound{σ} \AgdaSymbol{:} \AgdaFunction{Op} \AgdaBound{U} \AgdaBound{V}\AgdaSymbol{\}} \AgdaSymbol{→} \<[61]%
\>[61]\<%
\\
\>[2]\AgdaIndent{26}{}\<[26]%
\>[26]\AgdaBound{ρ} \AgdaFunction{∼op} \AgdaBound{σ} \AgdaSymbol{→} \AgdaFunction{liftOp''} \AgdaSymbol{\{}\AgdaArgument{K} \AgdaSymbol{=} \AgdaBound{K}\AgdaSymbol{\}} \AgdaBound{A} \AgdaBound{ρ} \AgdaFunction{∼op} \AgdaFunction{liftOp''} \AgdaBound{A} \AgdaBound{σ}\<%
\end{code}
}

This allows us to define the action of \emph{application} $E[\sigma]$:

\begin{code}%
\>[0]\AgdaIndent{2}{}\<[2]%
\>[2]\AgdaFunction{ap} \AgdaSymbol{:} \AgdaSymbol{∀} \AgdaSymbol{\{}\AgdaBound{U}\AgdaSymbol{\}} \AgdaSymbol{\{}\AgdaBound{V}\AgdaSymbol{\}} \AgdaSymbol{\{}\AgdaBound{C}\AgdaSymbol{\}} \AgdaSymbol{\{}\AgdaBound{K}\AgdaSymbol{\}} \AgdaSymbol{→} \<[27]%
\>[27]\<%
\\
\>[2]\AgdaIndent{4}{}\<[4]%
\>[4]\AgdaFunction{Op} \AgdaBound{U} \AgdaBound{V} \AgdaSymbol{→} \AgdaDatatype{Subexpression} \AgdaBound{U} \AgdaBound{C} \AgdaBound{K} \AgdaSymbol{→} \AgdaDatatype{Subexpression} \AgdaBound{V} \AgdaBound{C} \AgdaBound{K}\<%
\\
\>[0]\AgdaIndent{2}{}\<[2]%
\>[2]\AgdaFunction{ap} \AgdaBound{ρ} \AgdaSymbol{(}\AgdaInductiveConstructor{var} \AgdaBound{x}\AgdaSymbol{)} \AgdaSymbol{=} \AgdaFunction{apV} \AgdaBound{ρ} \AgdaBound{x}\<%
\\
\>[0]\AgdaIndent{2}{}\<[2]%
\>[2]\AgdaFunction{ap} \AgdaBound{ρ} \AgdaSymbol{(}\AgdaInductiveConstructor{app} \AgdaBound{c} \AgdaBound{EE}\AgdaSymbol{)} \AgdaSymbol{=} \AgdaInductiveConstructor{app} \AgdaBound{c} \AgdaSymbol{(}\AgdaFunction{ap} \AgdaBound{ρ} \AgdaBound{EE}\AgdaSymbol{)}\<%
\\
\>[0]\AgdaIndent{2}{}\<[2]%
\>[2]\AgdaFunction{ap} \AgdaSymbol{\_} \AgdaInductiveConstructor{out} \AgdaSymbol{=} \AgdaInductiveConstructor{out}\<%
\\
\>[0]\AgdaIndent{2}{}\<[2]%
\>[2]\AgdaFunction{ap} \AgdaBound{ρ} \AgdaSymbol{(}\AgdaInductiveConstructor{\_,,\_} \AgdaSymbol{\{}\AgdaArgument{A} \AgdaSymbol{=} \AgdaBound{A}\AgdaSymbol{\}} \AgdaBound{E} \AgdaBound{EE}\AgdaSymbol{)} \AgdaSymbol{=} \AgdaFunction{ap} \AgdaSymbol{(}\AgdaFunction{liftOp''} \AgdaBound{A} \AgdaBound{ρ}\AgdaSymbol{)} \AgdaBound{E} \AgdaInductiveConstructor{,,} \AgdaFunction{ap} \AgdaBound{ρ} \AgdaBound{EE}\<%
\end{code}

We prove that application respects $\sim$.

\begin{code}%
\>[0]\AgdaIndent{2}{}\<[2]%
\>[2]\AgdaFunction{ap-congl} \AgdaSymbol{:} \AgdaSymbol{∀} \AgdaSymbol{\{}\AgdaBound{U}\AgdaSymbol{\}} \AgdaSymbol{\{}\AgdaBound{V}\AgdaSymbol{\}} \AgdaSymbol{\{}\AgdaBound{C}\AgdaSymbol{\}} \AgdaSymbol{\{}\AgdaBound{K}\AgdaSymbol{\}} \<[31]%
\>[31]\<%
\\
\>[2]\AgdaIndent{4}{}\<[4]%
\>[4]\AgdaSymbol{\{}\AgdaBound{ρ} \AgdaBound{σ} \AgdaSymbol{:} \AgdaFunction{Op} \AgdaBound{U} \AgdaBound{V}\AgdaSymbol{\}} \AgdaSymbol{(}\AgdaBound{E} \AgdaSymbol{:} \AgdaDatatype{Subexpression} \AgdaBound{U} \AgdaBound{C} \AgdaBound{K}\AgdaSymbol{)} \AgdaSymbol{→}\<%
\\
\>[2]\AgdaIndent{4}{}\<[4]%
\>[4]\AgdaBound{ρ} \AgdaFunction{∼op} \AgdaBound{σ} \AgdaSymbol{→} \AgdaFunction{ap} \AgdaBound{ρ} \AgdaBound{E} \AgdaDatatype{≡} \AgdaFunction{ap} \AgdaBound{σ} \AgdaBound{E}\<%
\end{code}

\AgdaHide{
\begin{code}%
\>[0]\AgdaIndent{2}{}\<[2]%
\>[2]\AgdaFunction{ap-congl} \AgdaSymbol{(}\AgdaInductiveConstructor{var} \AgdaBound{x}\AgdaSymbol{)} \AgdaBound{ρ-is-σ} \AgdaSymbol{=} \AgdaBound{ρ-is-σ} \AgdaBound{x}\<%
\\
\>[0]\AgdaIndent{2}{}\<[2]%
\>[2]\AgdaFunction{ap-congl} \AgdaSymbol{(}\AgdaInductiveConstructor{app} \AgdaBound{c} \AgdaBound{E}\AgdaSymbol{)} \AgdaBound{ρ-is-σ} \AgdaSymbol{=} \AgdaFunction{cong} \AgdaSymbol{(}\AgdaInductiveConstructor{app} \AgdaBound{c}\AgdaSymbol{)} \AgdaSymbol{(}\AgdaFunction{ap-congl} \AgdaBound{E} \AgdaBound{ρ-is-σ}\AgdaSymbol{)}\<%
\\
\>[0]\AgdaIndent{2}{}\<[2]%
\>[2]\AgdaFunction{ap-congl} \AgdaInductiveConstructor{out} \AgdaSymbol{\_} \AgdaSymbol{=} \AgdaInductiveConstructor{refl}\<%
\\
\>[0]\AgdaIndent{2}{}\<[2]%
\>[2]\AgdaFunction{ap-congl} \AgdaSymbol{(}\AgdaInductiveConstructor{\_,,\_} \AgdaSymbol{\{}\AgdaArgument{L} \AgdaSymbol{=} \AgdaBound{L}\AgdaSymbol{\}} \AgdaSymbol{\{}\AgdaArgument{A} \AgdaSymbol{=} \AgdaBound{A}\AgdaSymbol{\}} \AgdaBound{E} \AgdaBound{F}\AgdaSymbol{)} \AgdaBound{ρ-is-σ} \AgdaSymbol{=} \<[47]%
\>[47]\<%
\\
\>[2]\AgdaIndent{4}{}\<[4]%
\>[4]\AgdaFunction{cong₂} \AgdaInductiveConstructor{\_,,\_} \AgdaSymbol{(}\AgdaFunction{ap-congl} \AgdaBound{E} \AgdaSymbol{(}\AgdaPostulate{liftOp''-cong} \AgdaBound{A} \AgdaBound{ρ-is-σ}\AgdaSymbol{))} \AgdaSymbol{(}\AgdaFunction{ap-congl} \AgdaBound{F} \AgdaBound{ρ-is-σ}\AgdaSymbol{)}\<%
\\
%
\\
\>[0]\AgdaIndent{2}{}\<[2]%
\>[2]\AgdaFunction{ap-congr} \AgdaSymbol{:} \AgdaSymbol{∀} \AgdaSymbol{\{}\AgdaBound{U}\AgdaSymbol{\}} \AgdaSymbol{\{}\AgdaBound{V}\AgdaSymbol{\}} \AgdaSymbol{\{}\AgdaBound{C}\AgdaSymbol{\}} \AgdaSymbol{\{}\AgdaBound{K}\AgdaSymbol{\}}\<%
\\
\>[2]\AgdaIndent{4}{}\<[4]%
\>[4]\AgdaSymbol{\{}\AgdaBound{σ} \AgdaSymbol{:} \AgdaFunction{Op} \AgdaBound{U} \AgdaBound{V}\AgdaSymbol{\}} \AgdaSymbol{\{}\AgdaBound{E} \AgdaBound{F} \AgdaSymbol{:} \AgdaDatatype{Subexpression} \AgdaBound{U} \AgdaBound{C} \AgdaBound{K}\AgdaSymbol{\}} \AgdaSymbol{→}\<%
\\
\>[2]\AgdaIndent{4}{}\<[4]%
\>[4]\AgdaBound{E} \AgdaDatatype{≡} \AgdaBound{F} \AgdaSymbol{→} \AgdaFunction{ap} \AgdaBound{σ} \AgdaBound{E} \AgdaDatatype{≡} \AgdaFunction{ap} \AgdaBound{σ} \AgdaBound{F}\<%
\\
\>[0]\AgdaIndent{2}{}\<[2]%
\>[2]\AgdaFunction{ap-congr} \AgdaSymbol{\{}\AgdaArgument{σ} \AgdaSymbol{=} \AgdaBound{σ}\AgdaSymbol{\}} \AgdaSymbol{=} \AgdaFunction{cong} \AgdaSymbol{(}\AgdaFunction{ap} \AgdaBound{σ}\AgdaSymbol{)}\<%
\\
%
\\
\>[0]\AgdaIndent{2}{}\<[2]%
\>[2]\AgdaFunction{ap-cong} \AgdaSymbol{:} \AgdaSymbol{∀} \AgdaSymbol{\{}\AgdaBound{U}\AgdaSymbol{\}} \AgdaSymbol{\{}\AgdaBound{V}\AgdaSymbol{\}} \AgdaSymbol{\{}\AgdaBound{C}\AgdaSymbol{\}} \AgdaSymbol{\{}\AgdaBound{K}\AgdaSymbol{\}}\<%
\\
\>[2]\AgdaIndent{4}{}\<[4]%
\>[4]\AgdaSymbol{\{}\AgdaBound{ρ} \AgdaBound{σ} \AgdaSymbol{:} \AgdaFunction{Op} \AgdaBound{U} \AgdaBound{V}\AgdaSymbol{\}} \AgdaSymbol{\{}\AgdaBound{M} \AgdaBound{N} \AgdaSymbol{:} \AgdaDatatype{Subexpression} \AgdaBound{U} \AgdaBound{C} \AgdaBound{K}\AgdaSymbol{\}} \AgdaSymbol{→}\<%
\\
\>[2]\AgdaIndent{4}{}\<[4]%
\>[4]\AgdaBound{ρ} \AgdaFunction{∼op} \AgdaBound{σ} \AgdaSymbol{→} \AgdaBound{M} \AgdaDatatype{≡} \AgdaBound{N} \AgdaSymbol{→} \AgdaFunction{ap} \AgdaBound{ρ} \AgdaBound{M} \AgdaDatatype{≡} \AgdaFunction{ap} \AgdaBound{σ} \AgdaBound{N}\<%
\\
\>[0]\AgdaIndent{2}{}\<[2]%
\>[2]\AgdaFunction{ap-cong} \AgdaSymbol{\{}\AgdaArgument{ρ} \AgdaSymbol{=} \AgdaBound{ρ}\AgdaSymbol{\}} \AgdaSymbol{\{}\AgdaBound{σ}\AgdaSymbol{\}} \AgdaSymbol{\{}\AgdaBound{M}\AgdaSymbol{\}} \AgdaSymbol{\{}\AgdaBound{N}\AgdaSymbol{\}} \AgdaBound{ρ∼σ} \AgdaBound{M≡N} \AgdaSymbol{=} \AgdaKeyword{let} \AgdaKeyword{open} \AgdaModule{≡-Reasoning} \AgdaKeyword{in} \<[64]%
\>[64]\<%
\\
\>[2]\AgdaIndent{4}{}\<[4]%
\>[4]\AgdaFunction{begin}\<%
\\
\>[4]\AgdaIndent{6}{}\<[6]%
\>[6]\AgdaFunction{ap} \AgdaBound{ρ} \AgdaBound{M}\<%
\\
\>[0]\AgdaIndent{4}{}\<[4]%
\>[4]\AgdaFunction{≡⟨} \AgdaFunction{ap-congl} \AgdaBound{M} \AgdaBound{ρ∼σ} \AgdaFunction{⟩}\<%
\\
\>[4]\AgdaIndent{6}{}\<[6]%
\>[6]\AgdaFunction{ap} \AgdaBound{σ} \AgdaBound{M}\<%
\\
\>[0]\AgdaIndent{4}{}\<[4]%
\>[4]\AgdaFunction{≡⟨} \AgdaFunction{ap-congr} \AgdaBound{M≡N} \AgdaFunction{⟩}\<%
\\
\>[4]\AgdaIndent{6}{}\<[6]%
\>[6]\AgdaFunction{ap} \AgdaBound{σ} \AgdaBound{N}\<%
\\
\>[0]\AgdaIndent{4}{}\<[4]%
\>[4]\AgdaFunction{∎}\<%
\end{code}
}

\AgdaHide{
\begin{code}%
\>\AgdaKeyword{open} \AgdaKeyword{import} \AgdaModule{Grammar.Substitution.Lifting} \AgdaBound{G} \AgdaKeyword{public}\<%
\end{code}
}

\begin{code}%
\>\AgdaKeyword{open} \AgdaKeyword{import} \AgdaModule{Grammar.Base}\<%
\\
%
\\
\>\AgdaKeyword{module} \AgdaModule{Grammar.Substitution.RepSub} \AgdaSymbol{(}\AgdaBound{G} \AgdaSymbol{:} \AgdaRecord{Grammar}\AgdaSymbol{)} \AgdaKeyword{where}\<%
\\
\>\AgdaKeyword{open} \AgdaKeyword{import} \AgdaModule{Data.List}\<%
\\
\>\AgdaKeyword{open} \AgdaKeyword{import} \AgdaModule{Prelims}\<%
\\
\>\AgdaKeyword{open} \AgdaModule{Grammar} \AgdaBound{G}\<%
\\
\>\AgdaKeyword{open} \AgdaKeyword{import} \AgdaModule{Grammar.OpFamily} \AgdaBound{G}\<%
\\
\>\AgdaKeyword{open} \AgdaKeyword{import} \AgdaModule{Grammar.Replacement} \AgdaBound{G}\<%
\\
\>\AgdaKeyword{open} \AgdaKeyword{import} \AgdaModule{Grammar.Substitution.PreOpFamily} \AgdaBound{G}\<%
\\
\>\AgdaKeyword{open} \AgdaKeyword{import} \AgdaModule{Grammar.Substitution.Lifting} \AgdaBound{G}\<%
\\
%
\\
\>\AgdaKeyword{open} \AgdaModule{OpFamily} \AgdaFunction{REP} \AgdaKeyword{using} \AgdaSymbol{()} \AgdaKeyword{renaming} \AgdaSymbol{(}liftsOp \AgdaSymbol{to} liftsOpR\AgdaSymbol{)}\<%
\\
\>\AgdaKeyword{open} \AgdaModule{PreOpFamily} \AgdaFunction{pre-substitution}\<%
\\
\>\AgdaKeyword{open} \AgdaModule{Lifting} \AgdaFunction{LIFTSUB}\<%
\end{code}

We can consider replacement to be a special case of substitution.  That is,
we can identify every replacement $\rho : U \rightarrow V$ with the substitution
that maps $x$ to $\rho(x)$.  
\begin{lemma}
Let $\rho$ be a replacement $U \rightarrow V$.
\begin{enumerate}
\item
The replacement $(\rho , K)$ and the substitution $(\rho , K)$ are equal.
\item
The replacement $\uparrow$ and the substitution $\uparrow$ are equal.
\item
The replacement $\rho^A$ and the substitution $\rho^A$ are equal.
\item
$ E \langle \rho \rangle \equiv E [ \rho ] $
\item
Hence $ E \langle \uparrow \rangle \equiv E [ \uparrow ]$.
\item
Substitution is a pre-family with lifting.
\end{enumerate}
\end{lemma}

\begin{code}%
\>\AgdaFunction{rep2sub} \AgdaSymbol{:} \AgdaSymbol{∀} \AgdaSymbol{\{}\AgdaBound{U}\AgdaSymbol{\}} \AgdaSymbol{\{}\AgdaBound{V}\AgdaSymbol{\}} \AgdaSymbol{→} \AgdaFunction{Rep} \AgdaBound{U} \AgdaBound{V} \AgdaSymbol{→} \AgdaFunction{Sub} \AgdaBound{U} \AgdaBound{V}\<%
\\
\>\AgdaFunction{rep2sub} \AgdaBound{ρ} \AgdaBound{K} \AgdaBound{x} \AgdaSymbol{=} \AgdaInductiveConstructor{var} \AgdaSymbol{(}\AgdaBound{ρ} \AgdaBound{K} \AgdaBound{x}\AgdaSymbol{)}\<%
\\
%
\\
\>\AgdaFunction{liftRep-is-liftSub} \AgdaSymbol{:} \AgdaSymbol{∀} \AgdaSymbol{\{}\AgdaBound{U}\AgdaSymbol{\}} \AgdaSymbol{\{}\AgdaBound{V}\AgdaSymbol{\}} \AgdaSymbol{\{}\AgdaBound{ρ} \AgdaSymbol{:} \AgdaFunction{Rep} \AgdaBound{U} \AgdaBound{V}\AgdaSymbol{\}} \AgdaSymbol{\{}\AgdaBound{K}\AgdaSymbol{\}} \AgdaSymbol{→} \<[51]%
\>[51]\<%
\\
\>[0]\AgdaIndent{2}{}\<[2]%
\>[2]\AgdaFunction{rep2sub} \AgdaSymbol{(}\AgdaFunction{liftRep} \AgdaBound{K} \AgdaBound{ρ}\AgdaSymbol{)} \AgdaFunction{∼} \AgdaFunction{liftSub} \AgdaBound{K} \AgdaSymbol{(}\AgdaFunction{rep2sub} \AgdaBound{ρ}\AgdaSymbol{)}\<%
\end{code}

\AgdaHide{
\begin{code}%
\>\AgdaFunction{liftRep-is-liftSub} \AgdaInductiveConstructor{x₀} \AgdaSymbol{=} \AgdaInductiveConstructor{refl}\<%
\\
\>\AgdaFunction{liftRep-is-liftSub} \AgdaSymbol{(}\AgdaInductiveConstructor{↑} \AgdaSymbol{\_)} \AgdaSymbol{=} \AgdaInductiveConstructor{refl}\<%
\end{code}
}

\begin{code}%
\>\AgdaFunction{up-is-up} \AgdaSymbol{:} \AgdaSymbol{∀} \AgdaSymbol{\{}\AgdaBound{V}\AgdaSymbol{\}} \AgdaSymbol{\{}\AgdaBound{K}\AgdaSymbol{\}} \AgdaSymbol{→} \AgdaFunction{rep2sub} \AgdaSymbol{(}\AgdaFunction{upRep} \AgdaSymbol{\{}\AgdaBound{V}\AgdaSymbol{\}} \AgdaSymbol{\{}\AgdaBound{K}\AgdaSymbol{\})} \AgdaFunction{∼} \AgdaFunction{upSub}\<%
\end{code}

\AgdaHide{
\begin{code}%
\>\AgdaFunction{up-is-up} \AgdaSymbol{\_} \AgdaSymbol{=} \AgdaInductiveConstructor{refl}\<%
\end{code}
}

\begin{code}%
\>\AgdaFunction{liftsOp-is-liftsOp} \AgdaSymbol{:} \AgdaSymbol{∀} \AgdaSymbol{\{}\AgdaBound{U}\AgdaSymbol{\}} \AgdaSymbol{\{}\AgdaBound{V}\AgdaSymbol{\}} \AgdaSymbol{\{}\AgdaBound{ρ} \AgdaSymbol{:} \AgdaFunction{Rep} \AgdaBound{U} \AgdaBound{V}\AgdaSymbol{\}} \AgdaSymbol{\{}\AgdaBound{A}\AgdaSymbol{\}} \AgdaSymbol{→} \<[51]%
\>[51]\<%
\\
\>[0]\AgdaIndent{2}{}\<[2]%
\>[2]\AgdaFunction{rep2sub} \AgdaSymbol{(}\AgdaFunction{liftsOpR} \<[21]%
\>[21]\AgdaBound{A} \AgdaBound{ρ}\AgdaSymbol{)} \AgdaFunction{∼} \AgdaFunction{liftsOp} \AgdaBound{A} \AgdaSymbol{(}\AgdaFunction{rep2sub} \AgdaBound{ρ}\AgdaSymbol{)}\<%
\end{code}

\AgdaHide{
\begin{code}%
\>\AgdaFunction{liftsOp-is-liftsOp} \AgdaSymbol{\{}\AgdaArgument{ρ} \AgdaSymbol{=} \AgdaBound{ρ}\AgdaSymbol{\}} \AgdaSymbol{\{}\AgdaArgument{A} \AgdaSymbol{=} \AgdaInductiveConstructor{[]}\AgdaSymbol{\}} \AgdaSymbol{=} \AgdaFunction{∼-refl} \AgdaSymbol{\{}\AgdaArgument{σ} \AgdaSymbol{=} \AgdaFunction{rep2sub} \AgdaBound{ρ}\AgdaSymbol{\}}\<%
\\
\>\AgdaFunction{liftsOp-is-liftsOp} \AgdaSymbol{\{}\AgdaBound{U}\AgdaSymbol{\}} \AgdaSymbol{\{}\AgdaBound{V}\AgdaSymbol{\}} \AgdaSymbol{\{}\AgdaBound{ρ}\AgdaSymbol{\}} \AgdaSymbol{\{}\AgdaBound{K} \AgdaInductiveConstructor{∷} \AgdaBound{A}\AgdaSymbol{\}} \AgdaSymbol{=} \AgdaKeyword{let} \AgdaKeyword{open} \AgdaModule{EqReasoning} \AgdaSymbol{(}\AgdaFunction{OP} \AgdaSymbol{\_} \AgdaSymbol{\_)} \AgdaKeyword{in} \<[74]%
\>[74]\<%
\\
\>[0]\AgdaIndent{2}{}\<[2]%
\>[2]\AgdaFunction{begin}\<%
\\
\>[2]\AgdaIndent{4}{}\<[4]%
\>[4]\AgdaFunction{rep2sub} \AgdaSymbol{(}\AgdaFunction{liftsOpR} \AgdaBound{A} \AgdaSymbol{(}\AgdaFunction{liftRep} \AgdaBound{K} \AgdaBound{ρ}\AgdaSymbol{))}\<%
\\
\>[0]\AgdaIndent{2}{}\<[2]%
\>[2]\AgdaFunction{≈⟨} \AgdaFunction{liftsOp-is-liftsOp} \AgdaSymbol{\{}\AgdaArgument{A} \AgdaSymbol{=} \AgdaBound{A}\AgdaSymbol{\}} \AgdaFunction{⟩}\<%
\\
\>[2]\AgdaIndent{4}{}\<[4]%
\>[4]\AgdaFunction{liftsOp} \AgdaBound{A} \AgdaSymbol{(}\AgdaFunction{rep2sub} \AgdaSymbol{(}\AgdaFunction{liftRep} \AgdaBound{K} \AgdaBound{ρ}\AgdaSymbol{))}\<%
\\
\>[0]\AgdaIndent{2}{}\<[2]%
\>[2]\AgdaFunction{≈⟨} \AgdaFunction{liftsOp-cong} \AgdaBound{A} \AgdaFunction{liftRep-is-liftSub} \AgdaFunction{⟩}\<%
\\
\>[2]\AgdaIndent{4}{}\<[4]%
\>[4]\AgdaFunction{liftsOp} \AgdaBound{A} \AgdaSymbol{(}\AgdaFunction{liftSub} \AgdaBound{K} \AgdaSymbol{(}\AgdaFunction{rep2sub} \AgdaBound{ρ}\AgdaSymbol{))}\<%
\\
\>[0]\AgdaIndent{2}{}\<[2]%
\>[2]\AgdaFunction{∎}\<%
\end{code}
}

\begin{code}%
\>\AgdaFunction{rep-is-sub} \AgdaSymbol{:} \AgdaSymbol{∀} \AgdaSymbol{\{}\AgdaBound{U}\AgdaSymbol{\}} \AgdaSymbol{\{}\AgdaBound{V}\AgdaSymbol{\}} \AgdaSymbol{\{}\AgdaBound{K}\AgdaSymbol{\}} \AgdaSymbol{\{}\AgdaBound{C}\AgdaSymbol{\}} \AgdaSymbol{(}\AgdaBound{E} \AgdaSymbol{:} \AgdaDatatype{Subexp} \AgdaBound{U} \AgdaBound{K} \AgdaBound{C}\AgdaSymbol{)} \AgdaSymbol{\{}\AgdaBound{ρ} \AgdaSymbol{:} \AgdaFunction{Rep} \AgdaBound{U} \AgdaBound{V}\AgdaSymbol{\}} \AgdaSymbol{→} \<[66]%
\>[66]\<%
\\
\>[0]\AgdaIndent{2}{}\<[2]%
\>[2]\AgdaBound{E} \AgdaFunction{〈} \AgdaBound{ρ} \AgdaFunction{〉} \AgdaDatatype{≡} \AgdaBound{E} \AgdaFunction{⟦} \AgdaFunction{rep2sub} \AgdaBound{ρ} \AgdaFunction{⟧}\<%
\end{code}

\AgdaHide{
\begin{code}%
\>\AgdaFunction{rep-is-sub} \AgdaSymbol{(}\AgdaInductiveConstructor{var} \AgdaSymbol{\_)} \AgdaSymbol{=} \AgdaInductiveConstructor{refl}\<%
\\
\>\AgdaFunction{rep-is-sub} \AgdaSymbol{(}\AgdaInductiveConstructor{app} \AgdaBound{c} \AgdaBound{E}\AgdaSymbol{)} \AgdaSymbol{=} \AgdaFunction{cong} \AgdaSymbol{(}\AgdaInductiveConstructor{app} \AgdaBound{c}\AgdaSymbol{)} \AgdaSymbol{(}\AgdaFunction{rep-is-sub} \AgdaBound{E}\AgdaSymbol{)}\<%
\\
\>\AgdaFunction{rep-is-sub} \AgdaInductiveConstructor{[]} \AgdaSymbol{=} \AgdaInductiveConstructor{refl}\<%
\\
\>\AgdaFunction{rep-is-sub} \AgdaSymbol{\{}\AgdaBound{U}\AgdaSymbol{\}} \AgdaSymbol{\{}\AgdaBound{V}\AgdaSymbol{\}} \AgdaSymbol{(}\AgdaInductiveConstructor{\_∷\_} \AgdaSymbol{\{}\AgdaArgument{A} \AgdaSymbol{=} \AgdaInductiveConstructor{SK} \AgdaBound{A} \AgdaSymbol{\_\}} \AgdaBound{E} \AgdaBound{F}\AgdaSymbol{)} \AgdaSymbol{\{}\AgdaBound{ρ}\AgdaSymbol{\}} \AgdaSymbol{=} \AgdaFunction{cong₂} \AgdaInductiveConstructor{\_∷\_} \<[58]%
\>[58]\<%
\\
\>[0]\AgdaIndent{2}{}\<[2]%
\>[2]\AgdaSymbol{(}\AgdaKeyword{let} \AgdaKeyword{open} \AgdaModule{≡-Reasoning} \AgdaSymbol{\{}\AgdaArgument{A} \AgdaSymbol{=} \AgdaDatatype{Subexp} \AgdaSymbol{(}\AgdaFunction{extend} \AgdaBound{V} \AgdaBound{A}\AgdaSymbol{)} \AgdaSymbol{\_} \AgdaSymbol{\_\}} \AgdaKeyword{in}\<%
\\
\>[0]\AgdaIndent{2}{}\<[2]%
\>[2]\AgdaFunction{begin} \<[8]%
\>[8]\<%
\\
\>[2]\AgdaIndent{4}{}\<[4]%
\>[4]\AgdaBound{E} \AgdaFunction{〈} \AgdaFunction{liftsOpR} \AgdaBound{A} \AgdaBound{ρ} \AgdaFunction{〉}\<%
\\
\>[0]\AgdaIndent{2}{}\<[2]%
\>[2]\AgdaFunction{≡⟨} \AgdaFunction{rep-is-sub} \AgdaBound{E} \AgdaFunction{⟩}\<%
\\
\>[2]\AgdaIndent{4}{}\<[4]%
\>[4]\AgdaBound{E} \AgdaFunction{⟦} \AgdaSymbol{(λ} \AgdaBound{K} \AgdaBound{x} \AgdaSymbol{→} \AgdaInductiveConstructor{var} \AgdaSymbol{(}\AgdaFunction{liftsOpR} \AgdaBound{A} \AgdaBound{ρ} \AgdaBound{K} \AgdaBound{x}\AgdaSymbol{))} \AgdaFunction{⟧} \<[43]%
\>[43]\<%
\\
\>[0]\AgdaIndent{2}{}\<[2]%
\>[2]\AgdaFunction{≡⟨} \AgdaFunction{ap-congl} \AgdaSymbol{(}\AgdaFunction{liftsOp-is-liftsOp} \AgdaSymbol{\{}\AgdaArgument{A} \AgdaSymbol{=} \AgdaBound{A}\AgdaSymbol{\})} \AgdaBound{E} \AgdaFunction{⟩}\<%
\\
\>[2]\AgdaIndent{4}{}\<[4]%
\>[4]\AgdaBound{E} \AgdaFunction{⟦} \AgdaFunction{liftsOp} \AgdaBound{A} \AgdaSymbol{(λ} \AgdaBound{K} \AgdaBound{x} \AgdaSymbol{→} \AgdaInductiveConstructor{var} \AgdaSymbol{(}\AgdaBound{ρ} \AgdaBound{K} \AgdaBound{x}\AgdaSymbol{))} \AgdaFunction{⟧}\<%
\\
\>[0]\AgdaIndent{2}{}\<[2]%
\>[2]\AgdaFunction{∎}\AgdaSymbol{)}\<%
\\
\>[0]\AgdaIndent{2}{}\<[2]%
\>[2]\AgdaSymbol{(}\AgdaFunction{rep-is-sub} \AgdaBound{F}\AgdaSymbol{)}\<%
\end{code}
}

\begin{code}%
\>\AgdaFunction{up-is-up'} \AgdaSymbol{:} \AgdaSymbol{∀} \AgdaSymbol{\{}\AgdaBound{V}\AgdaSymbol{\}} \AgdaSymbol{\{}\AgdaBound{C}\AgdaSymbol{\}} \AgdaSymbol{\{}\AgdaBound{K}\AgdaSymbol{\}} \AgdaSymbol{\{}\AgdaBound{L}\AgdaSymbol{\}} \AgdaSymbol{\{}\AgdaBound{E} \AgdaSymbol{:} \AgdaDatatype{Subexp} \AgdaBound{V} \AgdaBound{C} \AgdaBound{K}\AgdaSymbol{\}} \AgdaSymbol{→} \<[51]%
\>[51]\<%
\\
\>[0]\AgdaIndent{2}{}\<[2]%
\>[2]\AgdaBound{E} \AgdaFunction{〈} \AgdaFunction{upRep} \AgdaSymbol{\{}\AgdaArgument{K} \AgdaSymbol{=} \AgdaBound{L}\AgdaSymbol{\}} \AgdaFunction{〉} \AgdaDatatype{≡} \AgdaBound{E} \AgdaFunction{⟦} \AgdaFunction{upSub} \AgdaFunction{⟧}\<%
\end{code}

\AgdaHide{
\begin{code}%
\>\AgdaFunction{up-is-up'} \AgdaSymbol{\{}\AgdaArgument{E} \AgdaSymbol{=} \AgdaBound{E}\AgdaSymbol{\}} \AgdaSymbol{=} \AgdaFunction{rep-is-sub} \AgdaBound{E}\<%
\end{code}
}

\AgdaHide{
\begin{code}%
\>\AgdaKeyword{open} \AgdaKeyword{import} \AgdaModule{Grammar.Substitution.RepSub} \AgdaBound{G} \AgdaKeyword{public}\<%
\end{code}
}

\AgdaHide{
\begin{code}%
\>\AgdaKeyword{open} \AgdaKeyword{import} \AgdaModule{Grammar.Base}\<%
\\
%
\\
\>\AgdaKeyword{module} \AgdaModule{Grammar.Substitution.LiftFamily} \AgdaSymbol{(}\AgdaBound{G} \AgdaSymbol{:} \AgdaRecord{Grammar}\AgdaSymbol{)} \AgdaKeyword{where}\<%
\\
\>\AgdaKeyword{open} \AgdaKeyword{import} \AgdaModule{Prelims}\<%
\\
\>\AgdaKeyword{open} \AgdaModule{Grammar} \AgdaBound{G}\<%
\\
\>\AgdaKeyword{open} \AgdaKeyword{import} \AgdaModule{Grammar.OpFamily.LiftFamily} \AgdaBound{G}\<%
\\
\>\AgdaKeyword{open} \AgdaKeyword{import} \AgdaModule{Grammar.Substitution.PreOpFamily} \AgdaBound{G}\<%
\\
\>\AgdaKeyword{open} \AgdaKeyword{import} \AgdaModule{Grammar.Substitution.Lifting} \AgdaBound{G}\<%
\\
\>\AgdaKeyword{open} \AgdaKeyword{import} \AgdaModule{Grammar.Substitution.RepSub} \AgdaBound{G}\<%
\end{code}
}

It is now easy to show that substitution forms a pre-family with lifting.  If $\sigma : U \rightarrow V$ and $x \in U$ then $(\sigma , K)(\uparrow x) \equiv
\sigma(x) \langle \uparrow \rangle \equiv (\sigma , K)(x) [ \uparrow ]$.

\begin{code}%
\>\AgdaFunction{SubLF} \AgdaSymbol{:} \AgdaRecord{LiftFamily}\<%
\\
\>\AgdaFunction{SubLF} \AgdaSymbol{=} \AgdaKeyword{record} \AgdaSymbol{\{} \<[17]%
\>[17]\<%
\\
\>[0]\AgdaIndent{2}{}\<[2]%
\>[2]\AgdaField{preOpFamily} \AgdaSymbol{=} \AgdaFunction{pre-substitution} \AgdaSymbol{;} \<[35]%
\>[35]\<%
\\
\>[0]\AgdaIndent{2}{}\<[2]%
\>[2]\AgdaField{lifting} \AgdaSymbol{=} \AgdaFunction{LIFTSUB} \AgdaSymbol{;} \<[22]%
\>[22]\<%
\\
\>[0]\AgdaIndent{2}{}\<[2]%
\>[2]\AgdaField{isLiftFamily} \AgdaSymbol{=} \AgdaKeyword{record} \AgdaSymbol{\{} \<[26]%
\>[26]\<%
\\
\>[2]\AgdaIndent{4}{}\<[4]%
\>[4]\AgdaField{liftOp-x₀} \AgdaSymbol{=} \AgdaInductiveConstructor{refl} \AgdaSymbol{;} \<[23]%
\>[23]\<%
\\
\>[2]\AgdaIndent{4}{}\<[4]%
\>[4]\AgdaField{liftOp-↑} \AgdaSymbol{=} \AgdaSymbol{λ} \AgdaSymbol{\{}\AgdaBound{\_}\AgdaSymbol{\}} \AgdaSymbol{\{}\AgdaBound{\_}\AgdaSymbol{\}} \AgdaSymbol{\{}\AgdaBound{\_}\AgdaSymbol{\}} \AgdaSymbol{\{}\AgdaBound{\_}\AgdaSymbol{\}} \AgdaSymbol{\{}\AgdaBound{σ}\AgdaSymbol{\}} \AgdaBound{x} \AgdaSymbol{→} \AgdaFunction{rep-is-sub} \AgdaSymbol{(}\AgdaBound{σ} \AgdaSymbol{\_} \AgdaBound{x}\AgdaSymbol{)} \AgdaSymbol{\}\}}\<%
\end{code}

\AgdaHide{
\begin{code}%
\>\AgdaKeyword{open} \AgdaKeyword{import} \AgdaModule{Grammar.Substitution.LiftFamily} \AgdaBound{G} \AgdaKeyword{public}\<%
\end{code}
}

\AgdaHide{
\begin{code}%
\>\AgdaKeyword{open} \AgdaKeyword{import} \AgdaModule{Grammar.Base}\<%
\\
%
\\
\>\AgdaKeyword{module} \AgdaModule{Grammar.Substitution.OpFamily} \AgdaSymbol{(}\AgdaBound{G} \AgdaSymbol{:} \AgdaRecord{Grammar}\AgdaSymbol{)} \AgdaKeyword{where}\<%
\\
\>\AgdaKeyword{open} \AgdaKeyword{import} \AgdaModule{Prelims}\<%
\\
\>\AgdaKeyword{open} \AgdaModule{Grammar} \AgdaBound{G}\<%
\\
\>\AgdaKeyword{open} \AgdaKeyword{import} \AgdaModule{Grammar.OpFamily} \AgdaBound{G}\<%
\\
\>\AgdaKeyword{open} \AgdaKeyword{import} \AgdaModule{Grammar.Replacement} \AgdaBound{G}\<%
\\
\>\AgdaKeyword{open} \AgdaKeyword{import} \AgdaModule{Grammar.Substitution.PreOpFamily} \AgdaBound{G}\<%
\\
\>\AgdaKeyword{open} \AgdaKeyword{import} \AgdaModule{Grammar.Substitution.Lifting} \AgdaBound{G}\<%
\\
\>\AgdaKeyword{open} \AgdaKeyword{import} \AgdaModule{Grammar.Substitution.LiftFamily} \AgdaBound{G}\<%
\\
\>\AgdaKeyword{open} \AgdaKeyword{import} \AgdaModule{Grammar.Substitution.RepSub} \AgdaBound{G}\<%
\end{code}
}

We now define two compositions $\bullet_1 : \mathrm{replacement} ; \mathrm{substitution} \rightarrow \mathrm{substitution}$ and $\bullet_2 : \mathrm{substitution};\mathrm{replacement} \rightarrow \mathrm{substitution}$.

\begin{code}%
\>\AgdaKeyword{infixl} \AgdaNumber{60} \AgdaFixityOp{\_•RS\_}\<%
\\
\>\AgdaFunction{\_•RS\_} \AgdaSymbol{:} \AgdaSymbol{∀} \AgdaSymbol{\{}\AgdaBound{U}\AgdaSymbol{\}} \AgdaSymbol{\{}\AgdaBound{V}\AgdaSymbol{\}} \AgdaSymbol{\{}\AgdaBound{W}\AgdaSymbol{\}} \AgdaSymbol{→} \AgdaFunction{Rep} \AgdaBound{V} \AgdaBound{W} \AgdaSymbol{→} \AgdaFunction{Sub} \AgdaBound{U} \AgdaBound{V} \AgdaSymbol{→} \AgdaFunction{Sub} \AgdaBound{U} \AgdaBound{W}\<%
\\
\>\AgdaSymbol{(}\AgdaBound{ρ} \AgdaFunction{•RS} \AgdaBound{σ}\AgdaSymbol{)} \AgdaBound{K} \AgdaBound{x} \AgdaSymbol{=} \AgdaSymbol{(}\AgdaBound{σ} \AgdaBound{K} \AgdaBound{x}\AgdaSymbol{)} \AgdaFunction{〈} \AgdaBound{ρ} \AgdaFunction{〉}\<%
\\
%
\\
\>\AgdaFunction{Sub↑-compRS} \AgdaSymbol{:} \AgdaSymbol{∀} \AgdaSymbol{\{}\AgdaBound{U}\AgdaSymbol{\}} \AgdaSymbol{\{}\AgdaBound{V}\AgdaSymbol{\}} \AgdaSymbol{\{}\AgdaBound{W}\AgdaSymbol{\}} \AgdaSymbol{\{}\AgdaBound{K}\AgdaSymbol{\}} \AgdaSymbol{\{}\AgdaBound{ρ} \AgdaSymbol{:} \AgdaFunction{Rep} \AgdaBound{V} \AgdaBound{W}\AgdaSymbol{\}} \AgdaSymbol{\{}\AgdaBound{σ} \AgdaSymbol{:} \AgdaFunction{Sub} \AgdaBound{U} \AgdaBound{V}\AgdaSymbol{\}} \AgdaSymbol{→} \AgdaFunction{Sub↑} \AgdaBound{K} \AgdaSymbol{(}\AgdaBound{ρ} \AgdaFunction{•RS} \AgdaBound{σ}\AgdaSymbol{)} \AgdaFunction{∼} \AgdaFunction{Rep↑} \AgdaBound{K} \AgdaBound{ρ} \AgdaFunction{•RS} \AgdaFunction{Sub↑} \AgdaBound{K} \AgdaBound{σ}\<%
\end{code}

\AgdaHide{
\begin{code}%
\>\AgdaFunction{Sub↑-compRS} \AgdaSymbol{\{}\AgdaArgument{K} \AgdaSymbol{=} \AgdaBound{K}\AgdaSymbol{\}} \AgdaInductiveConstructor{x₀} \AgdaSymbol{=} \AgdaInductiveConstructor{refl}\<%
\\
\>\AgdaFunction{Sub↑-compRS} \AgdaSymbol{\{}\AgdaBound{U}\AgdaSymbol{\}} \AgdaSymbol{\{}\AgdaBound{V}\AgdaSymbol{\}} \AgdaSymbol{\{}\AgdaBound{W}\AgdaSymbol{\}} \AgdaSymbol{\{}\AgdaBound{K}\AgdaSymbol{\}} \AgdaSymbol{\{}\AgdaBound{ρ}\AgdaSymbol{\}} \AgdaSymbol{\{}\AgdaBound{σ}\AgdaSymbol{\}} \AgdaSymbol{\{}\AgdaBound{L}\AgdaSymbol{\}} \AgdaSymbol{(}\AgdaInductiveConstructor{↑} \AgdaBound{x}\AgdaSymbol{)} \AgdaSymbol{=} \AgdaKeyword{let} \AgdaKeyword{open} \AgdaModule{≡-Reasoning} \AgdaSymbol{\{}\AgdaArgument{A} \AgdaSymbol{=} \AgdaFunction{Expression} \AgdaSymbol{(}\AgdaBound{W} \AgdaInductiveConstructor{,} \AgdaBound{K}\AgdaSymbol{)} \AgdaSymbol{(}\AgdaInductiveConstructor{varKind} \AgdaBound{L}\AgdaSymbol{)\}} \AgdaKeyword{in} \<[109]%
\>[109]\<%
\\
\>[0]\AgdaIndent{2}{}\<[2]%
\>[2]\AgdaFunction{begin} \<[8]%
\>[8]\<%
\\
\>[2]\AgdaIndent{4}{}\<[4]%
\>[4]\AgdaSymbol{(}\AgdaBound{σ} \AgdaBound{L} \AgdaBound{x}\AgdaSymbol{)} \AgdaFunction{〈} \AgdaBound{ρ} \AgdaFunction{〉} \AgdaFunction{〈} \AgdaFunction{upRep} \AgdaFunction{〉}\<%
\\
\>[0]\AgdaIndent{2}{}\<[2]%
\>[2]\AgdaFunction{≡⟨⟨} \AgdaFunction{rep-comp} \AgdaSymbol{(}\AgdaBound{σ} \AgdaBound{L} \AgdaBound{x}\AgdaSymbol{)} \AgdaFunction{⟩⟩}\<%
\\
\>[2]\AgdaIndent{4}{}\<[4]%
\>[4]\AgdaSymbol{(}\AgdaBound{σ} \AgdaBound{L} \AgdaBound{x}\AgdaSymbol{)} \AgdaFunction{〈} \AgdaFunction{upRep} \AgdaFunction{•R} \AgdaBound{ρ} \AgdaFunction{〉}\<%
\\
\>[0]\AgdaIndent{2}{}\<[2]%
\>[2]\AgdaFunction{≡⟨⟩}\<%
\\
\>[2]\AgdaIndent{4}{}\<[4]%
\>[4]\AgdaSymbol{(}\AgdaBound{σ} \AgdaBound{L} \AgdaBound{x}\AgdaSymbol{)} \AgdaFunction{〈} \AgdaFunction{Rep↑} \AgdaBound{K} \AgdaBound{ρ} \AgdaFunction{•R} \AgdaFunction{upRep} \AgdaFunction{〉}\<%
\\
\>[0]\AgdaIndent{2}{}\<[2]%
\>[2]\AgdaFunction{≡⟨} \AgdaFunction{rep-comp} \AgdaSymbol{(}\AgdaBound{σ} \AgdaBound{L} \AgdaBound{x}\AgdaSymbol{)} \AgdaFunction{⟩}\<%
\\
\>[2]\AgdaIndent{4}{}\<[4]%
\>[4]\AgdaSymbol{(}\AgdaBound{σ} \AgdaBound{L} \AgdaBound{x}\AgdaSymbol{)} \AgdaFunction{〈} \AgdaFunction{upRep} \AgdaFunction{〉} \AgdaFunction{〈} \AgdaFunction{Rep↑} \AgdaBound{K} \AgdaBound{ρ} \AgdaFunction{〉}\<%
\\
\>[0]\AgdaIndent{2}{}\<[2]%
\>[2]\AgdaFunction{∎}\<%
\end{code}
}

\begin{code}%
\>\AgdaFunction{COMPRS} \AgdaSymbol{:} \AgdaRecord{Composition} \AgdaFunction{proto-replacement} \AgdaFunction{proto-substitution} \AgdaFunction{proto-substitution}\<%
\\
\>\AgdaFunction{COMPRS} \AgdaSymbol{=} \AgdaKeyword{record} \AgdaSymbol{\{} \<[18]%
\>[18]\<%
\\
\>[0]\AgdaIndent{2}{}\<[2]%
\>[2]\AgdaField{circ} \AgdaSymbol{=} \AgdaFunction{\_•RS\_} \AgdaSymbol{;} \<[17]%
\>[17]\<%
\\
\>[0]\AgdaIndent{2}{}\<[2]%
\>[2]\AgdaField{liftOp-circ} \AgdaSymbol{=} \AgdaFunction{Sub↑-compRS} \AgdaSymbol{;} \<[30]%
\>[30]\<%
\\
\>[0]\AgdaIndent{2}{}\<[2]%
\>[2]\AgdaField{apV-circ} \AgdaSymbol{=} \AgdaInductiveConstructor{refl} \AgdaSymbol{\}}\<%
\\
%
\\
\>\AgdaFunction{sub-compRS} \AgdaSymbol{:} \AgdaSymbol{∀} \AgdaSymbol{\{}\AgdaBound{U}\AgdaSymbol{\}} \AgdaSymbol{\{}\AgdaBound{V}\AgdaSymbol{\}} \AgdaSymbol{\{}\AgdaBound{W}\AgdaSymbol{\}} \AgdaSymbol{\{}\AgdaBound{C}\AgdaSymbol{\}} \AgdaSymbol{\{}\AgdaBound{K}\AgdaSymbol{\}} \<[35]%
\>[35]\<%
\\
\>[0]\AgdaIndent{2}{}\<[2]%
\>[2]\AgdaSymbol{(}\AgdaBound{E} \AgdaSymbol{:} \AgdaDatatype{Subexpression} \AgdaBound{U} \AgdaBound{C} \AgdaBound{K}\AgdaSymbol{)} \AgdaSymbol{\{}\AgdaBound{ρ} \AgdaSymbol{:} \AgdaFunction{Rep} \AgdaBound{V} \AgdaBound{W}\AgdaSymbol{\}} \AgdaSymbol{\{}\AgdaBound{σ} \AgdaSymbol{:} \AgdaFunction{Sub} \AgdaBound{U} \AgdaBound{V}\AgdaSymbol{\}} \AgdaSymbol{→}\<%
\\
\>[0]\AgdaIndent{2}{}\<[2]%
\>[2]\AgdaBound{E} \AgdaFunction{⟦} \AgdaBound{ρ} \AgdaFunction{•RS} \AgdaBound{σ} \AgdaFunction{⟧} \AgdaDatatype{≡} \AgdaBound{E} \AgdaFunction{⟦} \AgdaBound{σ} \AgdaFunction{⟧} \AgdaFunction{〈} \AgdaBound{ρ} \AgdaFunction{〉}\<%
\\
\>\AgdaFunction{sub-compRS} \AgdaBound{E} \AgdaSymbol{=} \AgdaFunction{Composition.ap-circ} \AgdaFunction{COMPRS} \AgdaBound{E}\<%
\\
%
\\
\>\AgdaKeyword{infixl} \AgdaNumber{60} \AgdaFixityOp{\_•SR\_}\<%
\\
\>\AgdaFunction{\_•SR\_} \AgdaSymbol{:} \AgdaSymbol{∀} \AgdaSymbol{\{}\AgdaBound{U}\AgdaSymbol{\}} \AgdaSymbol{\{}\AgdaBound{V}\AgdaSymbol{\}} \AgdaSymbol{\{}\AgdaBound{W}\AgdaSymbol{\}} \AgdaSymbol{→} \AgdaFunction{Sub} \AgdaBound{V} \AgdaBound{W} \AgdaSymbol{→} \AgdaFunction{Rep} \AgdaBound{U} \AgdaBound{V} \AgdaSymbol{→} \AgdaFunction{Sub} \AgdaBound{U} \AgdaBound{W}\<%
\\
\>\AgdaSymbol{(}\AgdaBound{σ} \AgdaFunction{•SR} \AgdaBound{ρ}\AgdaSymbol{)} \AgdaBound{K} \AgdaBound{x} \AgdaSymbol{=} \AgdaBound{σ} \AgdaBound{K} \AgdaSymbol{(}\AgdaBound{ρ} \AgdaBound{K} \AgdaBound{x}\AgdaSymbol{)}\<%
\\
%
\\
\>\AgdaFunction{Sub↑-compSR} \AgdaSymbol{:} \AgdaSymbol{∀} \AgdaSymbol{\{}\AgdaBound{U}\AgdaSymbol{\}} \AgdaSymbol{\{}\AgdaBound{V}\AgdaSymbol{\}} \AgdaSymbol{\{}\AgdaBound{W}\AgdaSymbol{\}} \AgdaSymbol{\{}\AgdaBound{K}\AgdaSymbol{\}} \AgdaSymbol{\{}\AgdaBound{σ} \AgdaSymbol{:} \AgdaFunction{Sub} \AgdaBound{V} \AgdaBound{W}\AgdaSymbol{\}} \AgdaSymbol{\{}\AgdaBound{ρ} \AgdaSymbol{:} \AgdaFunction{Rep} \AgdaBound{U} \AgdaBound{V}\AgdaSymbol{\}} \AgdaSymbol{→} \<[62]%
\>[62]\<%
\\
\>[0]\AgdaIndent{2}{}\<[2]%
\>[2]\AgdaFunction{Sub↑} \AgdaBound{K} \AgdaSymbol{(}\AgdaBound{σ} \AgdaFunction{•SR} \AgdaBound{ρ}\AgdaSymbol{)} \AgdaFunction{∼} \AgdaFunction{Sub↑} \AgdaBound{K} \AgdaBound{σ} \AgdaFunction{•SR} \AgdaFunction{Rep↑} \AgdaBound{K} \AgdaBound{ρ}\<%
\end{code}

\AgdaHide{
\begin{code}%
\>\AgdaFunction{Sub↑-compSR} \AgdaSymbol{\{}\AgdaArgument{K} \AgdaSymbol{=} \AgdaBound{K}\AgdaSymbol{\}} \AgdaInductiveConstructor{x₀} \AgdaSymbol{=} \AgdaInductiveConstructor{refl}\<%
\\
\>\AgdaFunction{Sub↑-compSR} \AgdaSymbol{(}\AgdaInductiveConstructor{↑} \AgdaBound{x}\AgdaSymbol{)} \AgdaSymbol{=} \AgdaInductiveConstructor{refl}\<%
\end{code}
}

\begin{code}%
\>\AgdaFunction{COMPSR} \AgdaSymbol{:} \AgdaRecord{Composition} \AgdaFunction{proto-substitution} \AgdaFunction{proto-replacement} \AgdaFunction{proto-substitution}\<%
\\
\>\AgdaFunction{COMPSR} \AgdaSymbol{=} \AgdaKeyword{record} \AgdaSymbol{\{} \<[18]%
\>[18]\<%
\\
\>[0]\AgdaIndent{2}{}\<[2]%
\>[2]\AgdaField{circ} \AgdaSymbol{=} \AgdaFunction{\_•SR\_} \AgdaSymbol{;} \<[17]%
\>[17]\<%
\\
\>[0]\AgdaIndent{2}{}\<[2]%
\>[2]\AgdaField{liftOp-circ} \AgdaSymbol{=} \AgdaFunction{Sub↑-compSR} \AgdaSymbol{;} \<[30]%
\>[30]\<%
\\
\>[0]\AgdaIndent{2}{}\<[2]%
\>[2]\AgdaField{apV-circ} \AgdaSymbol{=} \AgdaInductiveConstructor{refl} \AgdaSymbol{\}}\<%
\\
%
\\
\>\AgdaFunction{sub-compSR} \AgdaSymbol{:} \AgdaSymbol{∀} \AgdaSymbol{\{}\AgdaBound{U}\AgdaSymbol{\}} \AgdaSymbol{\{}\AgdaBound{V}\AgdaSymbol{\}} \AgdaSymbol{\{}\AgdaBound{W}\AgdaSymbol{\}} \AgdaSymbol{\{}\AgdaBound{C}\AgdaSymbol{\}} \AgdaSymbol{\{}\AgdaBound{K}\AgdaSymbol{\}} \<[35]%
\>[35]\<%
\\
\>[0]\AgdaIndent{2}{}\<[2]%
\>[2]\AgdaSymbol{(}\AgdaBound{E} \AgdaSymbol{:} \AgdaDatatype{Subexpression} \AgdaBound{U} \AgdaBound{C} \AgdaBound{K}\AgdaSymbol{)} \AgdaSymbol{\{}\AgdaBound{σ} \AgdaSymbol{:} \AgdaFunction{Sub} \AgdaBound{V} \AgdaBound{W}\AgdaSymbol{\}} \AgdaSymbol{\{}\AgdaBound{ρ} \AgdaSymbol{:} \AgdaFunction{Rep} \AgdaBound{U} \AgdaBound{V}\AgdaSymbol{\}} \AgdaSymbol{→} \<[58]%
\>[58]\<%
\\
\>[0]\AgdaIndent{2}{}\<[2]%
\>[2]\AgdaBound{E} \AgdaFunction{⟦} \AgdaBound{σ} \AgdaFunction{•SR} \AgdaBound{ρ} \AgdaFunction{⟧} \AgdaDatatype{≡} \AgdaBound{E} \AgdaFunction{〈} \AgdaBound{ρ} \AgdaFunction{〉} \AgdaFunction{⟦} \AgdaBound{σ} \AgdaFunction{⟧}\<%
\end{code}

\AgdaHide{
\begin{code}%
\>\AgdaFunction{sub-compSR} \AgdaBound{E} \AgdaSymbol{=} \AgdaFunction{Composition.ap-circ} \AgdaFunction{COMPSR} \AgdaBound{E}\<%
\end{code}
}

\begin{code}%
\>\AgdaFunction{Sub↑-upRep} \AgdaSymbol{:} \AgdaSymbol{∀} \AgdaSymbol{\{}\AgdaBound{U}\AgdaSymbol{\}} \AgdaSymbol{\{}\AgdaBound{V}\AgdaSymbol{\}} \AgdaSymbol{\{}\AgdaBound{C}\AgdaSymbol{\}} \AgdaSymbol{\{}\AgdaBound{K}\AgdaSymbol{\}} \AgdaSymbol{\{}\AgdaBound{L}\AgdaSymbol{\}} \AgdaSymbol{(}\AgdaBound{E} \AgdaSymbol{:} \AgdaDatatype{Subexpression} \AgdaBound{U} \AgdaBound{C} \AgdaBound{K}\AgdaSymbol{)} \AgdaSymbol{\{}\AgdaBound{σ} \AgdaSymbol{:} \AgdaFunction{Sub} \AgdaBound{U} \AgdaBound{V}\AgdaSymbol{\}} \AgdaSymbol{→}\<%
\\
\>[0]\AgdaIndent{2}{}\<[2]%
\>[2]\AgdaBound{E} \AgdaFunction{〈} \AgdaFunction{upRep} \AgdaFunction{〉} \AgdaFunction{⟦} \AgdaFunction{Sub↑} \AgdaBound{L} \AgdaBound{σ} \AgdaFunction{⟧} \AgdaDatatype{≡} \AgdaBound{E} \AgdaFunction{⟦} \AgdaBound{σ} \AgdaFunction{⟧} \AgdaFunction{〈} \AgdaFunction{upRep} \AgdaFunction{〉}\<%
\end{code}

\AgdaHide{
\begin{code}%
\>\AgdaFunction{Sub↑-upRep} \AgdaBound{E} \AgdaSymbol{=} \AgdaFunction{liftOp-up-mixed} \AgdaFunction{COMPSR} \AgdaFunction{COMPRS} \AgdaSymbol{(λ} \AgdaSymbol{\{}\AgdaBound{\_}\AgdaSymbol{\}} \AgdaSymbol{\{}\AgdaBound{\_}\AgdaSymbol{\}} \AgdaSymbol{\{}\AgdaBound{\_}\AgdaSymbol{\}} \AgdaSymbol{\{}\AgdaBound{\_}\AgdaSymbol{\}} \AgdaSymbol{\{}\AgdaBound{E}\AgdaSymbol{\}} \AgdaSymbol{→} \AgdaFunction{sym} \AgdaSymbol{(}\AgdaFunction{up-is-up'} \AgdaSymbol{\{}\AgdaArgument{E} \AgdaSymbol{=} \AgdaBound{E}\AgdaSymbol{\}))} \AgdaSymbol{\{}\AgdaBound{E}\AgdaSymbol{\}}\<%
\end{code}
}

Composition is defined by $(\sigma \circ \rho)(x) \equiv \rho(x) [ \sigma ]$.

\begin{code}%
\>\AgdaKeyword{infixl} \AgdaNumber{60} \AgdaFixityOp{\_•\_}\<%
\\
\>\AgdaFunction{\_•\_} \AgdaSymbol{:} \AgdaSymbol{∀} \AgdaSymbol{\{}\AgdaBound{U}\AgdaSymbol{\}} \AgdaSymbol{\{}\AgdaBound{V}\AgdaSymbol{\}} \AgdaSymbol{\{}\AgdaBound{W}\AgdaSymbol{\}} \AgdaSymbol{→} \AgdaFunction{Sub} \AgdaBound{V} \AgdaBound{W} \AgdaSymbol{→} \AgdaFunction{Sub} \AgdaBound{U} \AgdaBound{V} \AgdaSymbol{→} \AgdaFunction{Sub} \AgdaBound{U} \AgdaBound{W}\<%
\\
\>\AgdaSymbol{(}\AgdaBound{σ} \AgdaFunction{•} \AgdaBound{ρ}\AgdaSymbol{)} \AgdaBound{K} \AgdaBound{x} \AgdaSymbol{=} \AgdaBound{ρ} \AgdaBound{K} \AgdaBound{x} \AgdaFunction{⟦} \AgdaBound{σ} \AgdaFunction{⟧}\<%
\end{code}

Using the fact that $\bullet_1$ and $\bullet_2$ are compositions, we are
able to prove that this is a composition $\mathrm{substitution} ; \mathrm{substitution} \rightarrow \mathrm{substitution}$, and hence that substitution is a family of operations.

\begin{code}%
\>\AgdaFunction{Sub↑-comp} \AgdaSymbol{:} \AgdaSymbol{∀} \AgdaSymbol{\{}\AgdaBound{U}\AgdaSymbol{\}} \AgdaSymbol{\{}\AgdaBound{V}\AgdaSymbol{\}} \AgdaSymbol{\{}\AgdaBound{W}\AgdaSymbol{\}} \AgdaSymbol{\{}\AgdaBound{ρ} \AgdaSymbol{:} \AgdaFunction{Sub} \AgdaBound{U} \AgdaBound{V}\AgdaSymbol{\}} \AgdaSymbol{\{}\AgdaBound{σ} \AgdaSymbol{:} \AgdaFunction{Sub} \AgdaBound{V} \AgdaBound{W}\AgdaSymbol{\}} \AgdaSymbol{\{}\AgdaBound{K}\AgdaSymbol{\}} \AgdaSymbol{→} \<[60]%
\>[60]\<%
\\
\>[0]\AgdaIndent{2}{}\<[2]%
\>[2]\AgdaFunction{Sub↑} \AgdaBound{K} \AgdaSymbol{(}\AgdaBound{σ} \AgdaFunction{•} \AgdaBound{ρ}\AgdaSymbol{)} \AgdaFunction{∼} \AgdaFunction{Sub↑} \AgdaBound{K} \AgdaBound{σ} \AgdaFunction{•} \AgdaFunction{Sub↑} \AgdaBound{K} \AgdaBound{ρ}\<%
\end{code}

\AgdaHide{
\begin{code}%
\>\AgdaFunction{Sub↑-comp} \AgdaInductiveConstructor{x₀} \AgdaSymbol{=} \AgdaInductiveConstructor{refl}\<%
\\
\>\AgdaFunction{Sub↑-comp} \AgdaSymbol{\{}\AgdaArgument{W} \AgdaSymbol{=} \AgdaBound{W}\AgdaSymbol{\}} \AgdaSymbol{\{}\AgdaArgument{ρ} \AgdaSymbol{=} \AgdaBound{ρ}\AgdaSymbol{\}} \AgdaSymbol{\{}\AgdaArgument{σ} \AgdaSymbol{=} \AgdaBound{σ}\AgdaSymbol{\}} \AgdaSymbol{\{}\AgdaArgument{K} \AgdaSymbol{=} \AgdaBound{K}\AgdaSymbol{\}} \AgdaSymbol{\{}\AgdaBound{L}\AgdaSymbol{\}} \AgdaSymbol{(}\AgdaInductiveConstructor{↑} \AgdaBound{x}\AgdaSymbol{)} \AgdaSymbol{=} \AgdaFunction{sym} \AgdaSymbol{(}\AgdaFunction{Sub↑-upRep} \AgdaSymbol{(}\AgdaBound{ρ} \AgdaBound{L} \AgdaBound{x}\AgdaSymbol{))}\<%
\\
%
\\
\>\AgdaFunction{Sub↑-upRep₂} \AgdaSymbol{:} \AgdaSymbol{∀} \AgdaSymbol{\{}\AgdaBound{U}\AgdaSymbol{\}} \AgdaSymbol{\{}\AgdaBound{V}\AgdaSymbol{\}} \AgdaSymbol{\{}\AgdaBound{C}\AgdaSymbol{\}} \AgdaSymbol{\{}\AgdaBound{K}\AgdaSymbol{\}} \AgdaSymbol{\{}\AgdaBound{L}\AgdaSymbol{\}} \AgdaSymbol{\{}\AgdaBound{M}\AgdaSymbol{\}} \AgdaSymbol{(}\AgdaBound{E} \AgdaSymbol{:} \AgdaDatatype{Subexpression} \AgdaBound{U} \AgdaBound{C} \AgdaBound{M}\AgdaSymbol{)} \AgdaSymbol{\{}\AgdaBound{σ} \AgdaSymbol{:} \AgdaFunction{Sub} \AgdaBound{U} \AgdaBound{V}\AgdaSymbol{\}} \AgdaSymbol{→} \AgdaBound{E} \AgdaFunction{⇑} \AgdaFunction{⇑} \AgdaFunction{⟦} \AgdaFunction{Sub↑} \AgdaBound{K} \AgdaSymbol{(}\AgdaFunction{Sub↑} \AgdaBound{L} \AgdaBound{σ}\AgdaSymbol{)} \AgdaFunction{⟧} \AgdaDatatype{≡} \AgdaBound{E} \AgdaFunction{⟦} \AgdaBound{σ} \AgdaFunction{⟧} \AgdaFunction{⇑} \AgdaFunction{⇑}\<%
\\
\>\AgdaFunction{Sub↑-upRep₂} \AgdaSymbol{\{}\AgdaBound{U}\AgdaSymbol{\}} \AgdaSymbol{\{}\AgdaBound{V}\AgdaSymbol{\}} \AgdaSymbol{\{}\AgdaBound{C}\AgdaSymbol{\}} \AgdaSymbol{\{}\AgdaBound{K}\AgdaSymbol{\}} \AgdaSymbol{\{}\AgdaBound{L}\AgdaSymbol{\}} \AgdaSymbol{\{}\AgdaBound{M}\AgdaSymbol{\}} \AgdaBound{E} \AgdaSymbol{\{}\AgdaBound{σ}\AgdaSymbol{\}} \AgdaSymbol{=} \AgdaKeyword{let} \AgdaKeyword{open} \AgdaModule{≡-Reasoning} \AgdaKeyword{in} \<[68]%
\>[68]\<%
\\
\>[0]\AgdaIndent{2}{}\<[2]%
\>[2]\AgdaFunction{begin}\<%
\\
\>[2]\AgdaIndent{4}{}\<[4]%
\>[4]\AgdaBound{E} \AgdaFunction{⇑} \AgdaFunction{⇑} \AgdaFunction{⟦} \AgdaFunction{Sub↑} \AgdaBound{K} \AgdaSymbol{(}\AgdaFunction{Sub↑} \AgdaBound{L} \AgdaBound{σ}\AgdaSymbol{)} \AgdaFunction{⟧}\<%
\\
\>[0]\AgdaIndent{2}{}\<[2]%
\>[2]\AgdaFunction{≡⟨} \AgdaFunction{Sub↑-upRep} \AgdaSymbol{(}\AgdaBound{E} \AgdaFunction{⇑}\AgdaSymbol{)} \AgdaFunction{⟩}\<%
\\
\>[2]\AgdaIndent{4}{}\<[4]%
\>[4]\AgdaBound{E} \AgdaFunction{⇑} \AgdaFunction{⟦} \AgdaFunction{Sub↑} \AgdaBound{L} \AgdaBound{σ} \AgdaFunction{⟧} \AgdaFunction{⇑}\<%
\\
\>[0]\AgdaIndent{2}{}\<[2]%
\>[2]\AgdaFunction{≡⟨} \AgdaFunction{rep-congl} \AgdaSymbol{(}\AgdaFunction{Sub↑-upRep} \AgdaBound{E}\AgdaSymbol{)} \AgdaFunction{⟩}\<%
\\
\>[2]\AgdaIndent{4}{}\<[4]%
\>[4]\AgdaBound{E} \AgdaFunction{⟦} \AgdaBound{σ} \AgdaFunction{⟧} \AgdaFunction{⇑} \AgdaFunction{⇑}\<%
\\
\>[0]\AgdaIndent{2}{}\<[2]%
\>[2]\AgdaFunction{∎}\<%
\\
%
\\
\>\AgdaFunction{Sub↑-upRep₃} \AgdaSymbol{:} \AgdaSymbol{∀} \AgdaSymbol{\{}\AgdaBound{U}\AgdaSymbol{\}} \AgdaSymbol{\{}\AgdaBound{V}\AgdaSymbol{\}} \AgdaSymbol{\{}\AgdaBound{C}\AgdaSymbol{\}} \AgdaSymbol{\{}\AgdaBound{K}\AgdaSymbol{\}} \AgdaSymbol{\{}\AgdaBound{L}\AgdaSymbol{\}} \AgdaSymbol{\{}\AgdaBound{M}\AgdaSymbol{\}} \AgdaSymbol{\{}\AgdaBound{N}\AgdaSymbol{\}} \AgdaSymbol{(}\AgdaBound{E} \AgdaSymbol{:} \AgdaDatatype{Subexpression} \AgdaBound{U} \AgdaBound{C} \AgdaBound{N}\AgdaSymbol{)} \AgdaSymbol{\{}\AgdaBound{σ} \AgdaSymbol{:} \AgdaFunction{Sub} \AgdaBound{U} \AgdaBound{V}\AgdaSymbol{\}} \AgdaSymbol{→} \AgdaBound{E} \AgdaFunction{⇑} \AgdaFunction{⇑} \AgdaFunction{⇑} \AgdaFunction{⟦} \AgdaFunction{Sub↑} \AgdaBound{K} \AgdaSymbol{(}\AgdaFunction{Sub↑} \AgdaBound{L} \AgdaSymbol{(}\AgdaFunction{Sub↑} \AgdaBound{M} \AgdaBound{σ}\AgdaSymbol{))} \AgdaFunction{⟧} \AgdaDatatype{≡} \AgdaBound{E} \AgdaFunction{⟦} \AgdaBound{σ} \AgdaFunction{⟧} \AgdaFunction{⇑} \AgdaFunction{⇑} \AgdaFunction{⇑}\<%
\\
\>\AgdaFunction{Sub↑-upRep₃} \AgdaSymbol{\{}\AgdaBound{U}\AgdaSymbol{\}} \AgdaSymbol{\{}\AgdaBound{V}\AgdaSymbol{\}} \AgdaSymbol{\{}\AgdaBound{C}\AgdaSymbol{\}} \AgdaSymbol{\{}\AgdaBound{K}\AgdaSymbol{\}} \AgdaSymbol{\{}\AgdaBound{L}\AgdaSymbol{\}} \AgdaSymbol{\{}\AgdaBound{M}\AgdaSymbol{\}} \AgdaSymbol{\{}\AgdaBound{N}\AgdaSymbol{\}} \AgdaBound{E} \AgdaSymbol{\{}\AgdaBound{σ}\AgdaSymbol{\}} \AgdaSymbol{=} \AgdaKeyword{let} \AgdaKeyword{open} \AgdaModule{≡-Reasoning} \AgdaKeyword{in} \<[72]%
\>[72]\<%
\\
\>[0]\AgdaIndent{2}{}\<[2]%
\>[2]\AgdaFunction{begin}\<%
\\
\>[2]\AgdaIndent{4}{}\<[4]%
\>[4]\AgdaBound{E} \AgdaFunction{⇑} \AgdaFunction{⇑} \AgdaFunction{⇑} \AgdaFunction{⟦} \AgdaFunction{Sub↑} \AgdaBound{K} \AgdaSymbol{(}\AgdaFunction{Sub↑} \AgdaBound{L} \AgdaSymbol{(}\AgdaFunction{Sub↑} \AgdaBound{M} \AgdaBound{σ}\AgdaSymbol{))} \AgdaFunction{⟧}\<%
\\
\>[0]\AgdaIndent{2}{}\<[2]%
\>[2]\AgdaFunction{≡⟨} \AgdaFunction{Sub↑-upRep₂} \AgdaSymbol{(}\AgdaBound{E} \AgdaFunction{⇑}\AgdaSymbol{)} \AgdaFunction{⟩}\<%
\\
\>[2]\AgdaIndent{4}{}\<[4]%
\>[4]\AgdaBound{E} \AgdaFunction{⇑} \AgdaFunction{⟦} \AgdaFunction{Sub↑} \AgdaBound{M} \AgdaBound{σ} \AgdaFunction{⟧} \AgdaFunction{⇑} \AgdaFunction{⇑}\<%
\\
\>[0]\AgdaIndent{2}{}\<[2]%
\>[2]\AgdaFunction{≡⟨} \AgdaFunction{rep-congl} \AgdaSymbol{(}\AgdaFunction{rep-congl} \AgdaSymbol{(}\AgdaFunction{Sub↑-upRep} \AgdaBound{E}\AgdaSymbol{))} \AgdaFunction{⟩}\<%
\\
\>[2]\AgdaIndent{4}{}\<[4]%
\>[4]\AgdaBound{E} \AgdaFunction{⟦} \AgdaBound{σ} \AgdaFunction{⟧} \AgdaFunction{⇑} \AgdaFunction{⇑} \AgdaFunction{⇑}\<%
\\
\>[0]\AgdaIndent{2}{}\<[2]%
\>[2]\AgdaFunction{∎}\<%
\\
%
\\
\>\AgdaFunction{Rep↑-Sub↑-upRep₃} \AgdaSymbol{:} \AgdaSymbol{∀} \AgdaSymbol{\{}\AgdaBound{U}\AgdaSymbol{\}} \AgdaSymbol{\{}\AgdaBound{V}\AgdaSymbol{\}} \AgdaSymbol{\{}\AgdaBound{W}\AgdaSymbol{\}} \AgdaSymbol{\{}\AgdaBound{K1}\AgdaSymbol{\}} \AgdaSymbol{\{}\AgdaBound{K2}\AgdaSymbol{\}} \AgdaSymbol{\{}\AgdaBound{K3}\AgdaSymbol{\}} \AgdaSymbol{\{}\AgdaBound{C}\AgdaSymbol{\}} \AgdaSymbol{\{}\AgdaBound{K4}\AgdaSymbol{\}} \<[57]%
\>[57]\<%
\\
\>[2]\AgdaIndent{19}{}\<[19]%
\>[19]\AgdaSymbol{(}\AgdaBound{M} \AgdaSymbol{:} \AgdaDatatype{Subexpression} \AgdaBound{U} \AgdaBound{C} \AgdaBound{K4}\AgdaSymbol{)}\<%
\\
\>[2]\AgdaIndent{19}{}\<[19]%
\>[19]\AgdaSymbol{(}\AgdaBound{σ} \AgdaSymbol{:} \AgdaFunction{Sub} \AgdaBound{U} \AgdaBound{V}\AgdaSymbol{)} \AgdaSymbol{(}\AgdaBound{ρ} \AgdaSymbol{:} \AgdaFunction{Rep} \AgdaBound{V} \AgdaBound{W}\AgdaSymbol{)} \AgdaSymbol{→}\<%
\\
\>[19]\AgdaIndent{20}{}\<[20]%
\>[20]\AgdaBound{M} \AgdaFunction{⇑} \AgdaFunction{⇑} \AgdaFunction{⇑} \AgdaFunction{⟦} \AgdaFunction{Sub↑} \AgdaBound{K1} \AgdaSymbol{(}\AgdaFunction{Sub↑} \AgdaBound{K2} \AgdaSymbol{(}\AgdaFunction{Sub↑} \AgdaBound{K3} \AgdaBound{σ}\AgdaSymbol{))} \AgdaFunction{⟧} \AgdaFunction{〈} \AgdaFunction{Rep↑} \AgdaBound{K1} \AgdaSymbol{(}\AgdaFunction{Rep↑} \AgdaBound{K2} \AgdaSymbol{(}\AgdaFunction{Rep↑} \AgdaBound{K3} \AgdaBound{ρ}\AgdaSymbol{))} \AgdaFunction{〉}\<%
\\
\>[19]\AgdaIndent{20}{}\<[20]%
\>[20]\AgdaDatatype{≡} \AgdaBound{M} \AgdaFunction{⟦} \AgdaBound{σ} \AgdaFunction{⟧} \AgdaFunction{〈} \AgdaBound{ρ} \AgdaFunction{〉} \AgdaFunction{⇑} \AgdaFunction{⇑} \AgdaFunction{⇑}\<%
\\
\>\AgdaFunction{Rep↑-Sub↑-upRep₃} \AgdaBound{M} \AgdaBound{σ} \AgdaBound{ρ} \AgdaSymbol{=} \AgdaFunction{trans} \AgdaSymbol{(}\AgdaFunction{rep-congl} \AgdaSymbol{(}\AgdaFunction{Sub↑-upRep₃} \AgdaBound{M} \AgdaSymbol{\{}\AgdaBound{σ}\AgdaSymbol{\}))} \AgdaSymbol{(}\AgdaFunction{Rep↑-upRep₃} \AgdaSymbol{(}\AgdaBound{M} \AgdaFunction{⟦} \AgdaBound{σ} \AgdaFunction{⟧}\AgdaSymbol{))}\<%
\\
%
\\
\>\AgdaFunction{assocRSSR} \AgdaSymbol{:} \AgdaSymbol{∀} \AgdaSymbol{\{}\AgdaBound{U}\AgdaSymbol{\}} \AgdaSymbol{\{}\AgdaBound{V}\AgdaSymbol{\}} \AgdaSymbol{\{}\AgdaBound{W}\AgdaSymbol{\}} \AgdaSymbol{\{}\AgdaBound{X}\AgdaSymbol{\}} \AgdaSymbol{\{}\AgdaBound{ρ} \AgdaSymbol{:} \AgdaFunction{Sub} \AgdaBound{W} \AgdaBound{X}\AgdaSymbol{\}} \AgdaSymbol{\{}\AgdaBound{σ} \AgdaSymbol{:} \AgdaFunction{Rep} \AgdaBound{V} \AgdaBound{W}\AgdaSymbol{\}} \AgdaSymbol{\{}\AgdaBound{τ} \AgdaSymbol{:} \AgdaFunction{Sub} \AgdaBound{U} \AgdaBound{V}\AgdaSymbol{\}} \AgdaSymbol{→}\<%
\\
\>[0]\AgdaIndent{12}{}\<[12]%
\>[12]\AgdaBound{ρ} \AgdaFunction{•} \AgdaSymbol{(}\AgdaBound{σ} \AgdaFunction{•RS} \AgdaBound{τ}\AgdaSymbol{)} \AgdaFunction{∼} \AgdaSymbol{(}\AgdaBound{ρ} \AgdaFunction{•SR} \AgdaBound{σ}\AgdaSymbol{)} \AgdaFunction{•} \AgdaBound{τ}\<%
\\
\>\AgdaFunction{assocRSSR} \AgdaSymbol{\{}\AgdaArgument{ρ} \AgdaSymbol{=} \AgdaBound{ρ}\AgdaSymbol{\}} \AgdaSymbol{\{}\AgdaBound{σ}\AgdaSymbol{\}} \AgdaSymbol{\{}\AgdaBound{τ}\AgdaSymbol{\}} \AgdaBound{x} \AgdaSymbol{=} \AgdaFunction{sym} \AgdaSymbol{(}\AgdaFunction{sub-compSR} \AgdaSymbol{(}\AgdaBound{τ} \AgdaSymbol{\_} \AgdaBound{x}\AgdaSymbol{))}\<%
\end{code}
}

\begin{code}%
\>\AgdaFunction{substitution} \AgdaSymbol{:} \AgdaRecord{OpFamily}\<%
\\
\>\AgdaFunction{substitution} \AgdaSymbol{=} \AgdaKeyword{record} \AgdaSymbol{\{} \<[24]%
\>[24]\<%
\\
\>[0]\AgdaIndent{2}{}\<[2]%
\>[2]\AgdaField{liftFamily} \AgdaSymbol{=} \AgdaFunction{proto-substitution} \AgdaSymbol{;} \<[36]%
\>[36]\<%
\\
\>[0]\AgdaIndent{2}{}\<[2]%
\>[2]\AgdaField{isOpFamily} \AgdaSymbol{=} \AgdaKeyword{record} \AgdaSymbol{\{} \<[24]%
\>[24]\<%
\\
\>[2]\AgdaIndent{4}{}\<[4]%
\>[4]\AgdaField{\_∘\_} \AgdaSymbol{=} \AgdaFunction{\_•\_} \AgdaSymbol{;} \<[16]%
\>[16]\<%
\\
\>[2]\AgdaIndent{4}{}\<[4]%
\>[4]\AgdaField{liftOp-comp} \AgdaSymbol{=} \AgdaFunction{Sub↑-comp} \AgdaSymbol{;} \<[30]%
\>[30]\<%
\\
\>[2]\AgdaIndent{4}{}\<[4]%
\>[4]\AgdaField{apV-comp} \AgdaSymbol{=} \AgdaInductiveConstructor{refl} \AgdaSymbol{\}} \<[22]%
\>[22]\<%
\\
\>[0]\AgdaIndent{2}{}\<[2]%
\>[2]\AgdaSymbol{\}}\<%
\end{code}

\AgdaHide{
\begin{code}%
\>\AgdaKeyword{open} \AgdaModule{OpFamily} \AgdaFunction{substitution} \AgdaKeyword{using} \AgdaSymbol{(}comp-congl\AgdaSymbol{;}comp-congr\AgdaSymbol{)}\<%
\\
\>[0]\AgdaIndent{2}{}\<[2]%
\>[2]\AgdaKeyword{renaming} \AgdaSymbol{(}liftOp-idOp \AgdaSymbol{to} Sub↑-idOp\AgdaSymbol{;}\<\\
\>           ap-idOp \AgdaSymbol{to} sub-idOp\AgdaSymbol{;}\<\\
\>           ap-congl \AgdaSymbol{to} sub-congr\AgdaSymbol{;}\<\\
\>           ap-congr \AgdaSymbol{to} sub-congl\AgdaSymbol{;}\<\\
\>           unitl \AgdaSymbol{to} sub-unitl\AgdaSymbol{;}\<\\
\>           unitr \AgdaSymbol{to} sub-unitr\AgdaSymbol{;}\<\\
\>           ∼-sym \AgdaSymbol{to} sub-sym\AgdaSymbol{;}\<\\
\>           ∼-trans \AgdaSymbol{to} sub-trans\AgdaSymbol{;}\<\\
\>           assoc \AgdaSymbol{to} sub-assoc\AgdaSymbol{)}\<%
\\
\>[0]\AgdaIndent{2}{}\<[2]%
\>[2]\AgdaKeyword{public}\<%
\end{code}
}

\begin{frame}[fragile]
\frametitle{Metatheorems}
We can now prove general results such as:

\begin{code}%
\>\AgdaFunction{sub-comp} \AgdaSymbol{:} \AgdaSymbol{∀} \AgdaSymbol{\{}\AgdaBound{U}\AgdaSymbol{\}} \AgdaSymbol{\{}\AgdaBound{V}\AgdaSymbol{\}} \AgdaSymbol{\{}\AgdaBound{W}\AgdaSymbol{\}} \AgdaSymbol{\{}\AgdaBound{C}\AgdaSymbol{\}} \AgdaSymbol{\{}\AgdaBound{K}\AgdaSymbol{\}}\<%
\\
\>[0]\AgdaIndent{2}{}\<[2]%
\>[2]\AgdaSymbol{(}\AgdaBound{E} \AgdaSymbol{:} \AgdaDatatype{Subexpression} \AgdaBound{U} \AgdaBound{C} \AgdaBound{K}\AgdaSymbol{)} \AgdaSymbol{\{}\AgdaBound{σ} \AgdaSymbol{:} \AgdaFunction{Sub} \AgdaBound{V} \AgdaBound{W}\AgdaSymbol{\}} \AgdaSymbol{\{}\AgdaBound{ρ} \AgdaSymbol{:} \AgdaFunction{Sub} \AgdaBound{U} \AgdaBound{V}\AgdaSymbol{\}} \AgdaSymbol{→}\<%
\\
\>[0]\AgdaIndent{2}{}\<[2]%
\>[2]\AgdaBound{E} \AgdaFunction{⟦} \AgdaBound{σ} \AgdaFunction{•} \AgdaBound{ρ} \AgdaFunction{⟧} \AgdaDatatype{≡} \AgdaBound{E} \AgdaFunction{⟦} \AgdaBound{ρ} \AgdaFunction{⟧} \AgdaFunction{⟦} \AgdaBound{σ} \AgdaFunction{⟧}\<%
\end{code}
\end{frame}

\AgdaHide{
\begin{code}%
\>\AgdaFunction{sub-comp} \AgdaSymbol{=} \AgdaFunction{OpFamily.ap-circ} \AgdaFunction{substitution}\<%
\end{code}
}

\AgdaHide{
\begin{code}%
\>\AgdaKeyword{open} \AgdaKeyword{import} \AgdaModule{Grammar.Substitution.OpFamily} \AgdaBound{G} \AgdaKeyword{public}\<%
\end{code}
}

\AgdaHide{
\begin{code}%
\>\AgdaKeyword{open} \AgdaKeyword{import} \AgdaModule{Grammar.Base}\<%
\\
\>[0]\AgdaIndent{2}{}\<[2]%
\>[2]\<%
\\
\>\AgdaKeyword{module} \AgdaModule{Grammar.Substitution.Botsub} \AgdaSymbol{(}\AgdaBound{G} \AgdaSymbol{:} \AgdaRecord{Grammar}\AgdaSymbol{)} \AgdaKeyword{where}\<%
\\
\>\AgdaKeyword{open} \AgdaKeyword{import} \AgdaModule{Prelims}\<%
\\
\>\AgdaKeyword{open} \AgdaModule{Grammar} \AgdaBound{G}\<%
\\
\>\AgdaKeyword{open} \AgdaKeyword{import} \AgdaModule{Grammar.OpFamily} \AgdaBound{G}\<%
\\
\>\AgdaKeyword{open} \AgdaKeyword{import} \AgdaModule{Grammar.Replacement} \AgdaBound{G}\<%
\\
\>\AgdaKeyword{open} \AgdaKeyword{import} \AgdaModule{Grammar.Substitution.PreOpFamily} \AgdaBound{G}\<%
\\
\>\AgdaKeyword{open} \AgdaKeyword{import} \AgdaModule{Grammar.Substitution.Lifting} \AgdaBound{G}\<%
\\
\>\AgdaKeyword{open} \AgdaKeyword{import} \AgdaModule{Grammar.Substitution.LiftFamily} \AgdaBound{G}\<%
\\
\>\AgdaKeyword{open} \AgdaKeyword{import} \AgdaModule{Grammar.Substitution.OpFamily} \AgdaBound{G}\<%
\end{code}
}

\subsubsection{Substitution for an Individual Variable}

Let $E$ be an expression of kind $K$ over $V$.  Then we write $[x_0 := E]$ for the following substitution
$(V , K) \Rightarrow V$:

\AgdaHide{
\begin{code}%
\>\AgdaFunction{botSub} \AgdaSymbol{:} \AgdaSymbol{∀} \AgdaSymbol{\{}\AgdaBound{V}\AgdaSymbol{\}} \AgdaSymbol{\{}\AgdaBound{A}\AgdaSymbol{\}} \AgdaSymbol{→} \AgdaDatatype{ExpList} \AgdaBound{V} \AgdaBound{A} \AgdaSymbol{→} \AgdaFunction{Sub} \AgdaSymbol{(}\AgdaFunction{snoc-extend} \AgdaBound{V} \AgdaBound{A}\AgdaSymbol{)} \AgdaBound{V}\<%
\\
\>\AgdaFunction{botSub} \AgdaSymbol{\{}\AgdaArgument{A} \AgdaSymbol{=} \AgdaInductiveConstructor{[]}\AgdaSymbol{\}} \AgdaSymbol{\_} \AgdaSymbol{\_} \AgdaBound{x} \AgdaSymbol{=} \AgdaInductiveConstructor{var} \AgdaBound{x}\<%
\\
\>\AgdaFunction{botSub} \AgdaSymbol{\{}\AgdaArgument{A} \AgdaSymbol{=} \AgdaSymbol{\_} \AgdaInductiveConstructor{snoc} \AgdaSymbol{\_\}} \AgdaSymbol{(\_} \AgdaInductiveConstructor{snoc} \AgdaBound{E}\AgdaSymbol{)} \AgdaSymbol{\_} \AgdaInductiveConstructor{x₀} \AgdaSymbol{=} \AgdaBound{E}\<%
\\
\>\AgdaFunction{botSub} \AgdaSymbol{\{}\AgdaArgument{A} \AgdaSymbol{=} \AgdaSymbol{\_} \AgdaInductiveConstructor{snoc} \AgdaSymbol{\_\}} \AgdaSymbol{(}\AgdaBound{EE} \AgdaInductiveConstructor{snoc} \AgdaSymbol{\_)} \AgdaBound{L} \AgdaSymbol{(}\AgdaInductiveConstructor{↑} \AgdaBound{x}\AgdaSymbol{)} \AgdaSymbol{=} \AgdaFunction{botSub} \AgdaBound{EE} \AgdaBound{L} \AgdaBound{x}\<%
\end{code}
}

\begin{code}%
\>\AgdaKeyword{infix} \AgdaNumber{65} \AgdaFixityOp{x₀:=\_}\<%
\\
\>\AgdaFunction{x₀:=\_} \AgdaSymbol{:} \AgdaSymbol{∀} \AgdaSymbol{\{}\AgdaBound{V}\AgdaSymbol{\}} \AgdaSymbol{\{}\AgdaBound{K}\AgdaSymbol{\}} \AgdaSymbol{→} \AgdaFunction{Expression} \AgdaBound{V} \AgdaSymbol{(}\AgdaInductiveConstructor{varKind} \AgdaBound{K}\AgdaSymbol{)} \AgdaSymbol{→} \AgdaFunction{Sub} \AgdaSymbol{(}\AgdaBound{V} \AgdaInductiveConstructor{,} \AgdaBound{K}\AgdaSymbol{)} \AgdaBound{V}\<%
\\
\>\AgdaFunction{x₀:=} \AgdaBound{E} \AgdaSymbol{=} \AgdaFunction{botSub} \AgdaSymbol{(}\AgdaInductiveConstructor{[]} \AgdaInductiveConstructor{snoc} \AgdaBound{E}\AgdaSymbol{)}\<%
\end{code}

\begin{lemma}$ $
\begin{enumerate}
\item
$ \rho \bullet_1 [x_0 := E] \sim [x_0 := E \langle \rho \rangle] \bullet_2 (\rho , K) $
\item
$ \sigma \bullet [x_0 := E] \sim [x_0 := E[\sigma]] \bullet (\sigma , K) $
\item
$ E [ \uparrow ] [ x_0 := F ] \equiv E$
\end{enumerate}
\end{lemma}

\begin{code}%
\>\AgdaKeyword{open} \AgdaModule{LiftFamily}\<%
\\
%
\\
\>\AgdaFunction{botSub-up'} \AgdaSymbol{:} \AgdaSymbol{∀} \AgdaSymbol{\{}\AgdaBound{F}\AgdaSymbol{\}} \AgdaSymbol{\{}\AgdaBound{V}\AgdaSymbol{\}} \AgdaSymbol{\{}\AgdaBound{K}\AgdaSymbol{\}} \AgdaSymbol{\{}\AgdaBound{E} \AgdaSymbol{:} \AgdaFunction{Expression} \AgdaBound{V} \AgdaSymbol{(}\AgdaInductiveConstructor{varKind} \AgdaBound{K}\AgdaSymbol{)\}} \AgdaSymbol{(}\AgdaBound{circ} \AgdaSymbol{:} \AgdaRecord{Composition} \AgdaFunction{SubLF} \AgdaBound{F} \AgdaFunction{SubLF}\AgdaSymbol{)} \AgdaSymbol{→}\<%
\\
\>[0]\AgdaIndent{2}{}\<[2]%
\>[2]\AgdaField{Composition.circ} \AgdaBound{circ} \AgdaSymbol{(}\AgdaFunction{x₀:=} \AgdaBound{E}\AgdaSymbol{)} \AgdaSymbol{(}\AgdaFunction{up} \AgdaBound{F}\AgdaSymbol{)} \AgdaFunction{∼} \AgdaFunction{idSub} \AgdaBound{V}\<%
\\
\>\AgdaFunction{botSub-up'} \AgdaSymbol{\{}\AgdaBound{F}\AgdaSymbol{\}} \AgdaSymbol{\{}\AgdaBound{V}\AgdaSymbol{\}} \AgdaSymbol{\{}\AgdaBound{K}\AgdaSymbol{\}} \AgdaSymbol{\{}\AgdaBound{E}\AgdaSymbol{\}} \AgdaBound{circ} \AgdaBound{x} \AgdaSymbol{=} \AgdaKeyword{let} \AgdaKeyword{open} \AgdaModule{≡-Reasoning} \AgdaKeyword{in} \<[60]%
\>[60]\<%
\\
\>[0]\AgdaIndent{2}{}\<[2]%
\>[2]\AgdaFunction{begin}\<%
\\
\>[2]\AgdaIndent{4}{}\<[4]%
\>[4]\AgdaSymbol{(}\AgdaField{Composition.circ} \AgdaBound{circ} \AgdaSymbol{(}\AgdaFunction{x₀:=} \AgdaBound{E}\AgdaSymbol{)} \AgdaSymbol{(}\AgdaFunction{up} \AgdaBound{F}\AgdaSymbol{))} \AgdaSymbol{\_} \AgdaBound{x}\<%
\\
\>[0]\AgdaIndent{2}{}\<[2]%
\>[2]\AgdaFunction{≡⟨} \AgdaField{Composition.apV-circ} \AgdaBound{circ} \AgdaFunction{⟩}\<%
\\
\>[2]\AgdaIndent{4}{}\<[4]%
\>[4]\AgdaFunction{apV} \AgdaBound{F} \AgdaSymbol{(}\AgdaFunction{up} \AgdaBound{F}\AgdaSymbol{)} \AgdaBound{x} \AgdaFunction{⟦} \AgdaFunction{x₀:=} \AgdaBound{E} \AgdaFunction{⟧}\<%
\\
\>[0]\AgdaIndent{2}{}\<[2]%
\>[2]\AgdaFunction{≡⟨} \AgdaFunction{sub-congl} \AgdaSymbol{(}\AgdaFunction{apV-up} \AgdaBound{F}\AgdaSymbol{)} \AgdaFunction{⟩}\<%
\\
\>[2]\AgdaIndent{4}{}\<[4]%
\>[4]\AgdaInductiveConstructor{var} \AgdaBound{x}\<%
\\
\>[0]\AgdaIndent{2}{}\<[2]%
\>[2]\AgdaFunction{∎}\<%
\\
%
\\
\>\AgdaFunction{botSub-up} \AgdaSymbol{:} \AgdaSymbol{∀} \AgdaSymbol{\{}\AgdaBound{F}\AgdaSymbol{\}} \AgdaSymbol{\{}\AgdaBound{V}\AgdaSymbol{\}} \AgdaSymbol{\{}\AgdaBound{K}\AgdaSymbol{\}} \AgdaSymbol{\{}\AgdaBound{C}\AgdaSymbol{\}} \AgdaSymbol{\{}\AgdaBound{L}\AgdaSymbol{\}} \AgdaSymbol{\{}\AgdaBound{E} \AgdaSymbol{:} \AgdaFunction{Expression} \AgdaBound{V} \AgdaSymbol{(}\AgdaInductiveConstructor{varKind} \AgdaBound{K}\AgdaSymbol{)\}} \AgdaSymbol{(}\AgdaBound{circ} \AgdaSymbol{:} \AgdaRecord{Composition} \AgdaFunction{SubLF} \AgdaBound{F} \AgdaFunction{SubLF}\AgdaSymbol{)} \AgdaSymbol{\{}\AgdaBound{E'} \AgdaSymbol{:} \AgdaDatatype{Subexpression} \AgdaBound{V} \AgdaBound{C} \AgdaBound{L}\AgdaSymbol{\}} \AgdaSymbol{→}\<%
\\
\>[0]\AgdaIndent{2}{}\<[2]%
\>[2]\AgdaFunction{ap} \AgdaBound{F} \AgdaSymbol{(}\AgdaFunction{up} \AgdaBound{F}\AgdaSymbol{)} \AgdaBound{E'} \AgdaFunction{⟦} \AgdaFunction{x₀:=} \AgdaBound{E} \AgdaFunction{⟧} \AgdaDatatype{≡} \AgdaBound{E'}\<%
\\
\>\AgdaFunction{botSub-up} \AgdaSymbol{\{}\AgdaBound{F}\AgdaSymbol{\}} \AgdaSymbol{\{}\AgdaBound{V}\AgdaSymbol{\}} \AgdaSymbol{\{}\AgdaBound{K}\AgdaSymbol{\}} \AgdaSymbol{\{}\AgdaBound{C}\AgdaSymbol{\}} \AgdaSymbol{\{}\AgdaBound{L}\AgdaSymbol{\}} \AgdaSymbol{\{}\AgdaBound{E}\AgdaSymbol{\}} \AgdaBound{circ} \AgdaSymbol{\{}\AgdaBound{E'}\AgdaSymbol{\}} \AgdaSymbol{=} \AgdaKeyword{let} \AgdaKeyword{open} \AgdaModule{≡-Reasoning} \AgdaKeyword{in}\<%
\\
\>[0]\AgdaIndent{2}{}\<[2]%
\>[2]\AgdaFunction{begin}\<%
\\
\>[2]\AgdaIndent{4}{}\<[4]%
\>[4]\AgdaFunction{ap} \AgdaBound{F} \AgdaSymbol{(}\AgdaFunction{up} \AgdaBound{F}\AgdaSymbol{)} \AgdaBound{E'} \AgdaFunction{⟦} \AgdaFunction{x₀:=} \AgdaBound{E} \AgdaFunction{⟧}\<%
\\
\>[0]\AgdaIndent{2}{}\<[2]%
\>[2]\AgdaFunction{≡⟨⟨} \AgdaFunction{Composition.ap-circ} \AgdaBound{circ} \AgdaBound{E'} \AgdaFunction{⟩⟩}\<%
\\
\>[2]\AgdaIndent{4}{}\<[4]%
\>[4]\AgdaBound{E'} \AgdaFunction{⟦} \AgdaField{Composition.circ} \AgdaBound{circ} \AgdaSymbol{(}\AgdaFunction{x₀:=} \AgdaBound{E}\AgdaSymbol{)} \AgdaSymbol{(}\AgdaFunction{up} \AgdaBound{F}\AgdaSymbol{)} \AgdaFunction{⟧}\<%
\\
\>[0]\AgdaIndent{2}{}\<[2]%
\>[2]\AgdaFunction{≡⟨} \AgdaFunction{sub-congr} \AgdaSymbol{(}\AgdaFunction{botSub-up'} \AgdaBound{circ}\AgdaSymbol{)} \AgdaBound{E'} \AgdaFunction{⟩}\<%
\\
\>[2]\AgdaIndent{4}{}\<[4]%
\>[4]\AgdaBound{E'} \AgdaFunction{⟦} \AgdaFunction{idSub} \AgdaBound{V} \AgdaFunction{⟧}\<%
\\
\>[0]\AgdaIndent{2}{}\<[2]%
\>[2]\AgdaFunction{≡⟨} \AgdaFunction{sub-idOp} \AgdaFunction{⟩}\<%
\\
\>[2]\AgdaIndent{4}{}\<[4]%
\>[4]\AgdaBound{E'}\<%
\\
\>[0]\AgdaIndent{2}{}\<[2]%
\>[2]\AgdaFunction{∎}\<%
\\
%
\\
\>\AgdaFunction{circ-botSub'} \AgdaSymbol{:} \AgdaSymbol{∀} \AgdaSymbol{\{}\AgdaBound{F}\AgdaSymbol{\}} \AgdaSymbol{\{}\AgdaBound{U}\AgdaSymbol{\}} \AgdaSymbol{\{}\AgdaBound{V}\AgdaSymbol{\}} \AgdaSymbol{\{}\AgdaBound{K}\AgdaSymbol{\}} \AgdaSymbol{\{}\AgdaBound{E} \AgdaSymbol{:} \AgdaFunction{Expression} \AgdaBound{U} \AgdaSymbol{(}\AgdaInductiveConstructor{varKind} \AgdaBound{K}\AgdaSymbol{)\}} \<[64]%
\>[64]\<%
\\
\>[0]\AgdaIndent{2}{}\<[2]%
\>[2]\AgdaSymbol{(}\AgdaBound{circ₁} \AgdaSymbol{:} \AgdaRecord{Composition} \AgdaBound{F} \AgdaFunction{SubLF} \AgdaFunction{SubLF}\AgdaSymbol{)} \<[38]%
\>[38]\<%
\\
\>[0]\AgdaIndent{2}{}\<[2]%
\>[2]\AgdaSymbol{(}\AgdaBound{circ₂} \AgdaSymbol{:} \AgdaRecord{Composition} \AgdaFunction{SubLF} \AgdaBound{F} \AgdaFunction{SubLF}\AgdaSymbol{)}\<%
\\
\>[0]\AgdaIndent{2}{}\<[2]%
\>[2]\AgdaSymbol{\{}\AgdaBound{σ} \AgdaSymbol{:} \AgdaFunction{Op} \AgdaBound{F} \AgdaBound{U} \AgdaBound{V}\AgdaSymbol{\}} \AgdaSymbol{→}\<%
\\
\>[0]\AgdaIndent{2}{}\<[2]%
\>[2]\AgdaField{Composition.circ} \AgdaBound{circ₁} \AgdaBound{σ} \AgdaSymbol{(}\AgdaFunction{x₀:=} \AgdaBound{E}\AgdaSymbol{)} \AgdaFunction{∼} \AgdaField{Composition.circ} \AgdaBound{circ₂} \AgdaSymbol{(}\AgdaFunction{x₀:=} \AgdaSymbol{(}\AgdaFunction{ap} \AgdaBound{F} \AgdaBound{σ} \AgdaBound{E}\AgdaSymbol{))} \AgdaSymbol{(}\AgdaFunction{liftOp} \AgdaBound{F} \AgdaBound{K} \AgdaBound{σ}\AgdaSymbol{)}\<%
\\
\>\AgdaFunction{circ-botSub'} \AgdaSymbol{\{}\AgdaBound{F}\AgdaSymbol{\}} \AgdaSymbol{\{}\AgdaBound{U}\AgdaSymbol{\}} \AgdaSymbol{\{}\AgdaBound{V}\AgdaSymbol{\}} \AgdaSymbol{\{}\AgdaBound{K}\AgdaSymbol{\}} \AgdaSymbol{\{}\AgdaBound{E}\AgdaSymbol{\}} \AgdaBound{circ₁} \AgdaBound{circ₂} \AgdaSymbol{\{}\AgdaBound{σ}\AgdaSymbol{\}} \AgdaInductiveConstructor{x₀} \AgdaSymbol{=} \AgdaKeyword{let} \AgdaKeyword{open} \AgdaModule{≡-Reasoning} \AgdaKeyword{in} \<[78]%
\>[78]\<%
\\
\>[0]\AgdaIndent{2}{}\<[2]%
\>[2]\AgdaFunction{begin}\<%
\\
\>[2]\AgdaIndent{4}{}\<[4]%
\>[4]\AgdaSymbol{(}\AgdaField{Composition.circ} \AgdaBound{circ₁} \AgdaBound{σ} \AgdaSymbol{(}\AgdaFunction{x₀:=} \AgdaBound{E}\AgdaSymbol{))} \AgdaSymbol{\_} \AgdaInductiveConstructor{x₀}\<%
\\
\>[0]\AgdaIndent{2}{}\<[2]%
\>[2]\AgdaFunction{≡⟨} \AgdaField{Composition.apV-circ} \AgdaBound{circ₁} \AgdaFunction{⟩}\<%
\\
\>[2]\AgdaIndent{4}{}\<[4]%
\>[4]\AgdaFunction{ap} \AgdaBound{F} \AgdaBound{σ} \AgdaBound{E}\<%
\\
\>[0]\AgdaIndent{2}{}\<[2]%
\>[2]\AgdaFunction{≡⟨⟨} \AgdaFunction{sub-congl} \AgdaSymbol{(}\AgdaFunction{liftOp-x₀} \AgdaBound{F}\AgdaSymbol{)} \AgdaFunction{⟩⟩}\<%
\\
\>[2]\AgdaIndent{4}{}\<[4]%
\>[4]\AgdaSymbol{(}\AgdaFunction{apV} \AgdaBound{F} \AgdaSymbol{(}\AgdaFunction{liftOp} \AgdaBound{F} \AgdaBound{K} \AgdaBound{σ}\AgdaSymbol{)} \AgdaInductiveConstructor{x₀}\AgdaSymbol{)} \AgdaFunction{⟦} \AgdaFunction{x₀:=} \AgdaSymbol{(}\AgdaFunction{ap} \AgdaBound{F} \AgdaBound{σ} \AgdaBound{E}\AgdaSymbol{)} \AgdaFunction{⟧}\<%
\\
\>[0]\AgdaIndent{2}{}\<[2]%
\>[2]\AgdaFunction{≡⟨⟨} \AgdaField{Composition.apV-circ} \AgdaBound{circ₂} \AgdaFunction{⟩⟩}\<%
\\
\>[2]\AgdaIndent{4}{}\<[4]%
\>[4]\AgdaSymbol{(}\AgdaField{Composition.circ} \AgdaBound{circ₂} \AgdaSymbol{(}\AgdaFunction{x₀:=} \AgdaSymbol{(}\AgdaFunction{ap} \AgdaBound{F} \AgdaBound{σ} \AgdaBound{E}\AgdaSymbol{))} \AgdaSymbol{(}\AgdaFunction{liftOp} \AgdaBound{F} \AgdaBound{K} \AgdaBound{σ}\AgdaSymbol{))} \AgdaSymbol{\_} \AgdaInductiveConstructor{x₀}\<%
\\
\>[0]\AgdaIndent{2}{}\<[2]%
\>[2]\AgdaFunction{∎}\<%
\\
\>\AgdaFunction{circ-botSub'} \AgdaSymbol{\{}\AgdaBound{F}\AgdaSymbol{\}} \AgdaSymbol{\{}\AgdaBound{U}\AgdaSymbol{\}} \AgdaSymbol{\{}\AgdaBound{V}\AgdaSymbol{\}} \AgdaSymbol{\{}\AgdaBound{K}\AgdaSymbol{\}} \AgdaSymbol{\{}\AgdaBound{E}\AgdaSymbol{\}} \AgdaBound{circ₁} \AgdaBound{circ₂} \AgdaSymbol{\{}\AgdaBound{σ}\AgdaSymbol{\}} \AgdaSymbol{(}\AgdaInductiveConstructor{↑} \AgdaBound{x}\AgdaSymbol{)} \AgdaSymbol{=} \AgdaKeyword{let} \AgdaKeyword{open} \AgdaModule{≡-Reasoning} \AgdaKeyword{in} \<[81]%
\>[81]\<%
\\
\>[0]\AgdaIndent{2}{}\<[2]%
\>[2]\AgdaFunction{begin}\<%
\\
\>[2]\AgdaIndent{4}{}\<[4]%
\>[4]\AgdaSymbol{(}\AgdaField{Composition.circ} \AgdaBound{circ₁} \AgdaBound{σ} \AgdaSymbol{(}\AgdaFunction{x₀:=} \AgdaBound{E}\AgdaSymbol{))} \AgdaSymbol{\_} \AgdaSymbol{(}\AgdaInductiveConstructor{↑} \AgdaBound{x}\AgdaSymbol{)}\<%
\\
\>[0]\AgdaIndent{2}{}\<[2]%
\>[2]\AgdaFunction{≡⟨} \AgdaField{Composition.apV-circ} \AgdaBound{circ₁} \AgdaFunction{⟩}\<%
\\
\>[2]\AgdaIndent{4}{}\<[4]%
\>[4]\AgdaFunction{apV} \AgdaBound{F} \AgdaBound{σ} \AgdaBound{x}\<%
\\
\>[0]\AgdaIndent{2}{}\<[2]%
\>[2]\AgdaFunction{≡⟨⟨} \AgdaFunction{sub-idOp} \AgdaFunction{⟩⟩}\<%
\\
\>[2]\AgdaIndent{4}{}\<[4]%
\>[4]\AgdaFunction{apV} \AgdaBound{F} \AgdaBound{σ} \AgdaBound{x} \AgdaFunction{⟦} \AgdaFunction{idSub} \AgdaBound{V} \AgdaFunction{⟧}\<%
\\
\>[0]\AgdaIndent{2}{}\<[2]%
\>[2]\AgdaFunction{≡⟨⟨} \AgdaFunction{sub-congr} \AgdaSymbol{(}\AgdaFunction{botSub-up'} \AgdaBound{circ₂}\AgdaSymbol{)} \AgdaSymbol{(}\AgdaFunction{apV} \AgdaBound{F} \AgdaBound{σ} \AgdaBound{x}\AgdaSymbol{)} \AgdaFunction{⟩⟩}\<%
\\
\>[2]\AgdaIndent{4}{}\<[4]%
\>[4]\AgdaFunction{apV} \AgdaBound{F} \AgdaBound{σ} \AgdaBound{x} \AgdaFunction{⟦} \AgdaField{Composition.circ} \AgdaBound{circ₂} \AgdaSymbol{(}\AgdaFunction{x₀:=} \AgdaSymbol{(}\AgdaFunction{ap} \AgdaBound{F} \AgdaBound{σ} \AgdaBound{E}\AgdaSymbol{))} \AgdaSymbol{(}\AgdaFunction{up} \AgdaBound{F}\AgdaSymbol{)} \AgdaFunction{⟧}\<%
\\
\>[0]\AgdaIndent{2}{}\<[2]%
\>[2]\AgdaFunction{≡⟨} \AgdaFunction{Composition.ap-circ} \AgdaBound{circ₂} \AgdaSymbol{(}\AgdaFunction{apV} \AgdaBound{F} \AgdaBound{σ} \AgdaBound{x}\AgdaSymbol{)} \AgdaFunction{⟩}\<%
\\
\>[2]\AgdaIndent{4}{}\<[4]%
\>[4]\AgdaFunction{ap} \AgdaBound{F} \AgdaSymbol{(}\AgdaFunction{up} \AgdaBound{F}\AgdaSymbol{)} \AgdaSymbol{(}\AgdaFunction{apV} \AgdaBound{F} \AgdaBound{σ} \AgdaBound{x}\AgdaSymbol{)} \AgdaFunction{⟦} \AgdaFunction{x₀:=} \AgdaSymbol{(}\AgdaFunction{ap} \AgdaBound{F} \AgdaBound{σ} \AgdaBound{E}\AgdaSymbol{)} \AgdaFunction{⟧}\<%
\\
\>[0]\AgdaIndent{2}{}\<[2]%
\>[2]\AgdaFunction{≡⟨⟨} \AgdaFunction{sub-congl} \AgdaSymbol{(}\AgdaFunction{liftOp-↑} \AgdaBound{F} \AgdaBound{x}\AgdaSymbol{)} \AgdaFunction{⟩⟩}\<%
\\
\>[2]\AgdaIndent{4}{}\<[4]%
\>[4]\AgdaSymbol{(}\AgdaFunction{apV} \AgdaBound{F} \AgdaSymbol{(}\AgdaFunction{liftOp} \AgdaBound{F} \AgdaBound{K} \AgdaBound{σ}\AgdaSymbol{)} \AgdaSymbol{(}\AgdaInductiveConstructor{↑} \AgdaBound{x}\AgdaSymbol{))} \AgdaFunction{⟦} \AgdaFunction{x₀:=} \AgdaSymbol{(}\AgdaFunction{ap} \AgdaBound{F} \AgdaBound{σ} \AgdaBound{E}\AgdaSymbol{)} \AgdaFunction{⟧}\<%
\\
\>[0]\AgdaIndent{2}{}\<[2]%
\>[2]\AgdaFunction{≡⟨⟨} \AgdaField{Composition.apV-circ} \AgdaBound{circ₂} \AgdaFunction{⟩⟩}\<%
\\
\>[2]\AgdaIndent{4}{}\<[4]%
\>[4]\AgdaSymbol{(}\AgdaField{Composition.circ} \AgdaBound{circ₂} \AgdaSymbol{(}\AgdaFunction{x₀:=} \AgdaSymbol{(}\AgdaFunction{ap} \AgdaBound{F} \AgdaBound{σ} \AgdaBound{E}\AgdaSymbol{))} \AgdaSymbol{(}\AgdaFunction{liftOp} \AgdaBound{F} \AgdaBound{K} \AgdaBound{σ}\AgdaSymbol{))} \AgdaSymbol{\_} \AgdaSymbol{(}\AgdaInductiveConstructor{↑} \AgdaBound{x}\AgdaSymbol{)}\<%
\\
\>[0]\AgdaIndent{2}{}\<[2]%
\>[2]\AgdaFunction{∎}\<%
\\
%
\\
\>\AgdaFunction{circ-botSub} \AgdaSymbol{:} \AgdaSymbol{∀} \AgdaSymbol{\{}\AgdaBound{F}\AgdaSymbol{\}} \AgdaSymbol{\{}\AgdaBound{U}\AgdaSymbol{\}} \AgdaSymbol{\{}\AgdaBound{V}\AgdaSymbol{\}} \AgdaSymbol{\{}\AgdaBound{K}\AgdaSymbol{\}} \AgdaSymbol{\{}\AgdaBound{C}\AgdaSymbol{\}} \AgdaSymbol{\{}\AgdaBound{L}\AgdaSymbol{\}} \<[40]%
\>[40]\<%
\\
\>[0]\AgdaIndent{2}{}\<[2]%
\>[2]\AgdaSymbol{\{}\AgdaBound{E} \AgdaSymbol{:} \AgdaFunction{Expression} \AgdaBound{U} \AgdaSymbol{(}\AgdaInductiveConstructor{varKind} \AgdaBound{K}\AgdaSymbol{)\}} \AgdaSymbol{\{}\AgdaBound{E'} \AgdaSymbol{:} \AgdaDatatype{Subexpression} \AgdaSymbol{(}\AgdaBound{U} \AgdaInductiveConstructor{,} \AgdaBound{K}\AgdaSymbol{)} \AgdaBound{C} \AgdaBound{L}\AgdaSymbol{\}} \AgdaSymbol{\{}\AgdaBound{σ} \AgdaSymbol{:} \AgdaFunction{Op} \AgdaBound{F} \AgdaBound{U} \AgdaBound{V}\AgdaSymbol{\}} \AgdaSymbol{→}\<%
\\
\>[0]\AgdaIndent{2}{}\<[2]%
\>[2]\AgdaRecord{Composition} \AgdaBound{F} \AgdaFunction{SubLF} \AgdaFunction{SubLF} \AgdaSymbol{→}\<%
\\
\>[0]\AgdaIndent{2}{}\<[2]%
\>[2]\AgdaRecord{Composition} \AgdaFunction{SubLF} \AgdaBound{F} \AgdaFunction{SubLF} \AgdaSymbol{→}\<%
\\
\>[0]\AgdaIndent{2}{}\<[2]%
\>[2]\AgdaFunction{ap} \AgdaBound{F} \AgdaBound{σ} \AgdaSymbol{(}\AgdaBound{E'} \AgdaFunction{⟦} \AgdaFunction{x₀:=} \AgdaBound{E} \AgdaFunction{⟧}\AgdaSymbol{)} \AgdaDatatype{≡} \AgdaSymbol{(}\AgdaFunction{ap} \AgdaBound{F} \AgdaSymbol{(}\AgdaFunction{liftOp} \AgdaBound{F} \AgdaBound{K} \AgdaBound{σ}\AgdaSymbol{)} \AgdaBound{E'}\AgdaSymbol{)} \AgdaFunction{⟦} \AgdaFunction{x₀:=} \AgdaSymbol{(}\AgdaFunction{ap} \AgdaBound{F} \AgdaBound{σ} \AgdaBound{E}\AgdaSymbol{)} \AgdaFunction{⟧}\<%
\\
\>\AgdaFunction{circ-botSub} \AgdaSymbol{\{}\AgdaArgument{E'} \AgdaSymbol{=} \AgdaBound{E'}\AgdaSymbol{\}} \AgdaBound{circ₁} \AgdaBound{circ₂} \AgdaSymbol{=} \AgdaFunction{ap-circ-sim} \AgdaBound{circ₁} \AgdaBound{circ₂} \AgdaSymbol{(}\AgdaFunction{circ-botSub'} \AgdaBound{circ₁} \AgdaBound{circ₂}\AgdaSymbol{)} \AgdaBound{E'}\<%
\\
%
\\
\>\AgdaFunction{compRS-botSub} \AgdaSymbol{:} \AgdaSymbol{∀} \AgdaSymbol{\{}\AgdaBound{U}\AgdaSymbol{\}} \AgdaSymbol{\{}\AgdaBound{V}\AgdaSymbol{\}} \AgdaSymbol{\{}\AgdaBound{C}\AgdaSymbol{\}} \AgdaSymbol{\{}\AgdaBound{K}\AgdaSymbol{\}} \AgdaSymbol{\{}\AgdaBound{L}\AgdaSymbol{\}} \AgdaSymbol{(}\AgdaBound{E} \AgdaSymbol{:} \AgdaDatatype{Subexpression} \AgdaSymbol{(}\AgdaBound{U} \AgdaInductiveConstructor{,} \AgdaBound{K}\AgdaSymbol{)} \AgdaBound{C} \AgdaBound{L}\AgdaSymbol{)} \AgdaSymbol{\{}\AgdaBound{F} \AgdaSymbol{:} \AgdaFunction{Expression} \AgdaBound{U} \AgdaSymbol{(}\AgdaInductiveConstructor{varKind} \AgdaBound{K}\AgdaSymbol{)\}} \AgdaSymbol{\{}\AgdaBound{ρ} \AgdaSymbol{:} \AgdaFunction{Rep} \AgdaBound{U} \AgdaBound{V}\AgdaSymbol{\}} \AgdaSymbol{→}\<%
\\
\>[0]\AgdaIndent{2}{}\<[2]%
\>[2]\AgdaBound{E} \AgdaFunction{⟦} \AgdaFunction{x₀:=} \AgdaBound{F} \AgdaFunction{⟧} \AgdaFunction{〈} \AgdaBound{ρ} \AgdaFunction{〉} \AgdaDatatype{≡} \AgdaBound{E} \AgdaFunction{〈} \AgdaFunction{liftRep} \AgdaBound{K} \AgdaBound{ρ} \AgdaFunction{〉} \AgdaFunction{⟦} \AgdaFunction{x₀:=} \AgdaSymbol{(}\AgdaBound{F} \AgdaFunction{〈} \AgdaBound{ρ} \AgdaFunction{〉}\AgdaSymbol{)} \AgdaFunction{⟧}\<%
\\
\>\AgdaComment{--TODO Common pattern with liftRep-botSub₃}\<%
\end{code}

\AgdaHide{
\begin{code}%
\>\AgdaFunction{compRS-botSub} \AgdaBound{E} \AgdaSymbol{=} \AgdaFunction{circ-botSub} \AgdaSymbol{\{}\AgdaArgument{E'} \AgdaSymbol{=} \AgdaBound{E}\AgdaSymbol{\}} \AgdaFunction{COMPRS} \AgdaFunction{COMPSR}\<%
\end{code}
}

\begin{code}%
\>\AgdaFunction{comp-botSub} \AgdaSymbol{:} \AgdaSymbol{∀} \AgdaSymbol{\{}\AgdaBound{U}\AgdaSymbol{\}} \AgdaSymbol{\{}\AgdaBound{V}\AgdaSymbol{\}} \AgdaSymbol{\{}\AgdaBound{C}\AgdaSymbol{\}} \AgdaSymbol{\{}\AgdaBound{K}\AgdaSymbol{\}} \AgdaSymbol{\{}\AgdaBound{L}\AgdaSymbol{\}} \<[36]%
\>[36]\<%
\\
\>[0]\AgdaIndent{2}{}\<[2]%
\>[2]\AgdaSymbol{\{}\AgdaBound{E} \AgdaSymbol{:} \AgdaFunction{Expression} \AgdaBound{U} \AgdaSymbol{(}\AgdaInductiveConstructor{varKind} \AgdaBound{K}\AgdaSymbol{)\}} \AgdaSymbol{\{}\AgdaBound{σ} \AgdaSymbol{:} \AgdaFunction{Sub} \AgdaBound{U} \AgdaBound{V}\AgdaSymbol{\}} \AgdaSymbol{(}\AgdaBound{F} \AgdaSymbol{:} \AgdaDatatype{Subexpression} \AgdaSymbol{(}\AgdaBound{U} \AgdaInductiveConstructor{,} \AgdaBound{K}\AgdaSymbol{)} \AgdaBound{C} \AgdaBound{L}\AgdaSymbol{)} \AgdaSymbol{→}\<%
\\
\>[2]\AgdaIndent{3}{}\<[3]%
\>[3]\AgdaBound{F} \AgdaFunction{⟦} \AgdaFunction{x₀:=} \AgdaBound{E} \AgdaFunction{⟧} \AgdaFunction{⟦} \AgdaBound{σ} \AgdaFunction{⟧} \AgdaDatatype{≡} \AgdaBound{F} \AgdaFunction{⟦} \AgdaFunction{liftSub} \AgdaBound{K} \AgdaBound{σ} \AgdaFunction{⟧} \AgdaFunction{⟦} \AgdaFunction{x₀:=} \AgdaSymbol{(}\AgdaBound{E} \AgdaFunction{⟦} \AgdaBound{σ} \AgdaFunction{⟧}\AgdaSymbol{)} \AgdaFunction{⟧}\<%
\end{code}

\AgdaHide{
\begin{code}%
\>\AgdaFunction{comp-botSub} \AgdaBound{F} \AgdaSymbol{=} \AgdaKeyword{let} \AgdaBound{COMP} \AgdaSymbol{=} \AgdaFunction{OpFamily.COMP} \AgdaFunction{SUB} \AgdaKeyword{in} \AgdaFunction{circ-botSub} \AgdaSymbol{\{}\AgdaArgument{E'} \AgdaSymbol{=} \AgdaBound{F}\AgdaSymbol{\}} \AgdaBound{COMP} \AgdaBound{COMP}\<%
\end{code}
}

\begin{code}%
\>\AgdaFunction{botSub-upRep} \AgdaSymbol{:} \AgdaSymbol{∀} \AgdaSymbol{\{}\AgdaBound{U}\AgdaSymbol{\}} \AgdaSymbol{\{}\AgdaBound{C}\AgdaSymbol{\}} \AgdaSymbol{\{}\AgdaBound{K}\AgdaSymbol{\}} \AgdaSymbol{\{}\AgdaBound{L}\AgdaSymbol{\}}\<%
\\
\>[0]\AgdaIndent{2}{}\<[2]%
\>[2]\AgdaSymbol{(}\AgdaBound{E} \AgdaSymbol{:} \AgdaDatatype{Subexpression} \AgdaBound{U} \AgdaBound{C} \AgdaBound{K}\AgdaSymbol{)} \AgdaSymbol{\{}\AgdaBound{F} \AgdaSymbol{:} \AgdaFunction{Expression} \AgdaBound{U} \AgdaSymbol{(}\AgdaInductiveConstructor{varKind} \AgdaBound{L}\AgdaSymbol{)\}} \AgdaSymbol{→} \<[61]%
\>[61]\<%
\\
\>[0]\AgdaIndent{2}{}\<[2]%
\>[2]\AgdaBound{E} \AgdaFunction{〈} \AgdaFunction{upRep} \AgdaFunction{〉} \AgdaFunction{⟦} \AgdaFunction{x₀:=} \AgdaBound{F} \AgdaFunction{⟧} \AgdaDatatype{≡} \AgdaBound{E}\<%
\end{code}

\AgdaHide{
\begin{code}%
\>\AgdaFunction{botSub-upRep} \AgdaSymbol{\_} \AgdaSymbol{=} \AgdaFunction{botSub-up} \AgdaFunction{COMPSR}\<%
\\
%
\\
\>\AgdaFunction{botSub-botSub'} \AgdaSymbol{:} \AgdaSymbol{∀} \AgdaSymbol{\{}\AgdaBound{V}\AgdaSymbol{\}} \AgdaSymbol{\{}\AgdaBound{K}\AgdaSymbol{\}} \AgdaSymbol{\{}\AgdaBound{L}\AgdaSymbol{\}} \AgdaSymbol{(}\AgdaBound{N} \AgdaSymbol{:} \AgdaFunction{Expression} \AgdaBound{V} \AgdaSymbol{(}\AgdaInductiveConstructor{varKind} \AgdaBound{K}\AgdaSymbol{))} \AgdaSymbol{(}\AgdaBound{N'} \AgdaSymbol{:} \AgdaFunction{Expression} \AgdaBound{V} \AgdaSymbol{(}\AgdaInductiveConstructor{varKind} \AgdaBound{L}\AgdaSymbol{))} \AgdaSymbol{→} \AgdaFunction{x₀:=} \AgdaBound{N'} \AgdaFunction{•} \AgdaFunction{liftSub} \AgdaBound{L} \AgdaSymbol{(}\AgdaFunction{x₀:=} \AgdaBound{N}\AgdaSymbol{)} \AgdaFunction{∼} \AgdaFunction{x₀:=} \AgdaBound{N} \AgdaFunction{•} \AgdaFunction{x₀:=} \AgdaSymbol{(}\AgdaBound{N'} \AgdaFunction{⇑}\AgdaSymbol{)}\<%
\\
\>\AgdaFunction{botSub-botSub'} \AgdaBound{N} \AgdaBound{N'} \AgdaInductiveConstructor{x₀} \AgdaSymbol{=} \AgdaFunction{sym} \AgdaSymbol{(}\AgdaFunction{botSub-upRep} \AgdaBound{N'}\AgdaSymbol{)}\<%
\\
\>\AgdaFunction{botSub-botSub'} \AgdaBound{N} \AgdaBound{N'} \AgdaSymbol{(}\AgdaInductiveConstructor{↑} \AgdaInductiveConstructor{x₀}\AgdaSymbol{)} \AgdaSymbol{=} \AgdaFunction{botSub-upRep} \AgdaBound{N}\<%
\\
\>\AgdaFunction{botSub-botSub'} \AgdaBound{N} \AgdaBound{N'} \AgdaSymbol{(}\AgdaInductiveConstructor{↑} \AgdaSymbol{(}\AgdaInductiveConstructor{↑} \AgdaBound{x}\AgdaSymbol{))} \AgdaSymbol{=} \AgdaInductiveConstructor{refl}\<%
\\
%
\\
\>\AgdaFunction{botSub-botSub} \AgdaSymbol{:} \AgdaSymbol{∀} \AgdaSymbol{\{}\AgdaBound{V}\AgdaSymbol{\}} \AgdaSymbol{\{}\AgdaBound{K}\AgdaSymbol{\}} \AgdaSymbol{\{}\AgdaBound{L}\AgdaSymbol{\}} \AgdaSymbol{\{}\AgdaBound{M}\AgdaSymbol{\}} \AgdaSymbol{(}\AgdaBound{E} \AgdaSymbol{:} \AgdaFunction{Expression} \AgdaSymbol{(}\AgdaBound{V} \AgdaInductiveConstructor{,} \AgdaBound{K} \AgdaInductiveConstructor{,} \AgdaBound{L}\AgdaSymbol{)} \AgdaBound{M}\AgdaSymbol{)} \AgdaBound{F} \AgdaBound{G} \AgdaSymbol{→} \AgdaBound{E} \AgdaFunction{⟦} \AgdaFunction{liftSub} \AgdaBound{L} \AgdaSymbol{(}\AgdaFunction{x₀:=} \AgdaBound{F}\AgdaSymbol{)} \AgdaFunction{⟧} \AgdaFunction{⟦} \AgdaFunction{x₀:=} \AgdaBound{G} \AgdaFunction{⟧} \AgdaDatatype{≡} \AgdaBound{E} \AgdaFunction{⟦} \AgdaFunction{x₀:=} \AgdaSymbol{(}\AgdaBound{G} \AgdaFunction{⇑}\AgdaSymbol{)} \AgdaFunction{⟧} \AgdaFunction{⟦} \AgdaFunction{x₀:=} \AgdaBound{F} \AgdaFunction{⟧}\<%
\\
\>\AgdaFunction{botSub-botSub} \AgdaSymbol{\{}\AgdaBound{V}\AgdaSymbol{\}} \AgdaSymbol{\{}\AgdaBound{K}\AgdaSymbol{\}} \AgdaSymbol{\{}\AgdaBound{L}\AgdaSymbol{\}} \AgdaSymbol{\{}\AgdaBound{M}\AgdaSymbol{\}} \AgdaBound{E} \AgdaBound{F} \AgdaBound{G} \AgdaSymbol{=} \AgdaKeyword{let} \AgdaBound{COMP} \AgdaSymbol{=} \AgdaFunction{OpFamily.COMP} \AgdaFunction{SUB} \AgdaKeyword{in} \AgdaFunction{ap-circ-sim} \AgdaBound{COMP} \AgdaBound{COMP} \AgdaSymbol{(}\AgdaFunction{botSub-botSub'} \AgdaBound{F} \AgdaBound{G}\AgdaSymbol{)} \AgdaBound{E}\<%
\\
%
\\
\>\AgdaFunction{x₂:=\_,x₁:=\_,x₀:=\_} \AgdaSymbol{:} \AgdaSymbol{∀} \AgdaSymbol{\{}\AgdaBound{V}\AgdaSymbol{\}} \AgdaSymbol{\{}\AgdaBound{K1}\AgdaSymbol{\}} \AgdaSymbol{\{}\AgdaBound{K2}\AgdaSymbol{\}} \AgdaSymbol{\{}\AgdaBound{K3}\AgdaSymbol{\}} \AgdaSymbol{→} \AgdaFunction{Expression} \AgdaBound{V} \AgdaSymbol{(}\AgdaInductiveConstructor{varKind} \AgdaBound{K1}\AgdaSymbol{)} \AgdaSymbol{→} \AgdaFunction{Expression} \AgdaBound{V} \AgdaSymbol{(}\AgdaInductiveConstructor{varKind} \AgdaBound{K2}\AgdaSymbol{)} \AgdaSymbol{→} \AgdaFunction{Expression} \AgdaBound{V} \AgdaSymbol{(}\AgdaInductiveConstructor{varKind} \AgdaBound{K3}\AgdaSymbol{)} \AgdaSymbol{→} \AgdaFunction{Sub} \AgdaSymbol{(}\AgdaBound{V} \AgdaInductiveConstructor{,} \AgdaBound{K1} \AgdaInductiveConstructor{,} \AgdaBound{K2} \AgdaInductiveConstructor{,} \AgdaBound{K3}\AgdaSymbol{)} \AgdaBound{V}\<%
\\
\>\AgdaFunction{x₂:=\_,x₁:=\_,x₀:=\_} \AgdaBound{M1} \AgdaBound{M2} \AgdaBound{M3} \AgdaSymbol{=} \AgdaFunction{botSub} \AgdaSymbol{(}\AgdaInductiveConstructor{[]} \AgdaInductiveConstructor{snoc} \AgdaBound{M1} \AgdaInductiveConstructor{snoc} \AgdaBound{M2} \AgdaInductiveConstructor{snoc} \AgdaBound{M3}\AgdaSymbol{)}\<%
\\
%
\\
\>\AgdaKeyword{postulate} \AgdaPostulate{botSub-upRep₃} \AgdaSymbol{:} \AgdaSymbol{∀} \AgdaSymbol{\{}\AgdaBound{V}\AgdaSymbol{\}} \AgdaSymbol{\{}\AgdaBound{K1}\AgdaSymbol{\}} \AgdaSymbol{\{}\AgdaBound{K2}\AgdaSymbol{\}} \AgdaSymbol{\{}\AgdaBound{K3}\AgdaSymbol{\}} \AgdaSymbol{\{}\AgdaBound{L}\AgdaSymbol{\}} \AgdaSymbol{\{}\AgdaBound{M} \AgdaSymbol{:} \AgdaFunction{Expression} \AgdaBound{V} \AgdaBound{L}\AgdaSymbol{\}} \<[72]%
\>[72]\<%
\\
\>[2]\AgdaIndent{26}{}\<[26]%
\>[26]\AgdaSymbol{\{}\AgdaBound{N1} \AgdaSymbol{:} \AgdaFunction{Expression} \AgdaBound{V} \AgdaSymbol{(}\AgdaInductiveConstructor{varKind} \AgdaBound{K1}\AgdaSymbol{)\}} \AgdaSymbol{\{}\AgdaBound{N2} \AgdaSymbol{:} \AgdaFunction{Expression} \AgdaBound{V} \AgdaSymbol{(}\AgdaInductiveConstructor{varKind} \AgdaBound{K2}\AgdaSymbol{)\}} \AgdaSymbol{\{}\AgdaBound{N3} \AgdaSymbol{:} \AgdaFunction{Expression} \AgdaBound{V} \AgdaSymbol{(}\AgdaInductiveConstructor{varKind} \AgdaBound{K3}\AgdaSymbol{)\}} \AgdaSymbol{→}\<%
\\
\>[2]\AgdaIndent{26}{}\<[26]%
\>[26]\AgdaBound{M} \AgdaFunction{⇑} \AgdaFunction{⇑} \AgdaFunction{⇑} \AgdaFunction{⟦} \AgdaFunction{x₂:=} \AgdaBound{N1} \AgdaFunction{,x₁:=} \AgdaBound{N2} \AgdaFunction{,x₀:=} \AgdaBound{N3} \AgdaFunction{⟧} \AgdaDatatype{≡} \AgdaBound{M}\<%
\\
%
\\
\>\AgdaComment{--TODO Definition for Expression varKind}\<%
\\
\>\AgdaFunction{botSub₃-liftRep₃'} \AgdaSymbol{:} \AgdaSymbol{∀} \AgdaSymbol{\{}\AgdaBound{U}\AgdaSymbol{\}} \AgdaSymbol{\{}\AgdaBound{V}\AgdaSymbol{\}} \AgdaSymbol{\{}\AgdaBound{K2}\AgdaSymbol{\}} \AgdaSymbol{\{}\AgdaBound{K1}\AgdaSymbol{\}} \AgdaSymbol{\{}\AgdaBound{K0}\AgdaSymbol{\}}\<%
\\
\>[0]\AgdaIndent{2}{}\<[2]%
\>[2]\AgdaSymbol{\{}\AgdaBound{M2} \AgdaSymbol{:} \AgdaFunction{Expression} \AgdaBound{U} \AgdaSymbol{(}\AgdaInductiveConstructor{varKind} \AgdaBound{K1}\AgdaSymbol{)\}} \AgdaSymbol{\{}\AgdaBound{M1} \AgdaSymbol{:} \AgdaFunction{Expression} \AgdaBound{U} \AgdaSymbol{(}\AgdaInductiveConstructor{varKind} \AgdaBound{K2}\AgdaSymbol{)\}} \AgdaSymbol{\{}\AgdaBound{M0} \AgdaSymbol{:} \AgdaFunction{Expression} \AgdaBound{U} \AgdaSymbol{(}\AgdaInductiveConstructor{varKind} \AgdaBound{K0}\AgdaSymbol{)\}} \AgdaSymbol{\{}\AgdaBound{ρ} \AgdaSymbol{:} \AgdaFunction{Rep} \AgdaBound{U} \AgdaBound{V}\AgdaSymbol{\}} \AgdaSymbol{→}\<%
\\
\>[0]\AgdaIndent{2}{}\<[2]%
\>[2]\AgdaSymbol{(}\AgdaFunction{x₂:=} \AgdaBound{M2} \AgdaFunction{〈} \AgdaBound{ρ} \AgdaFunction{〉} \AgdaFunction{,x₁:=} \AgdaBound{M1} \AgdaFunction{〈} \AgdaBound{ρ} \AgdaFunction{〉} \AgdaFunction{,x₀:=} \AgdaBound{M0} \AgdaFunction{〈} \AgdaBound{ρ} \AgdaFunction{〉}\AgdaSymbol{)} \AgdaFunction{•SR} \AgdaFunction{liftRep} \AgdaSymbol{\_} \AgdaSymbol{(}\AgdaFunction{liftRep} \AgdaSymbol{\_} \AgdaSymbol{(}\AgdaFunction{liftRep} \AgdaSymbol{\_} \AgdaBound{ρ}\AgdaSymbol{))}\<%
\\
\>[0]\AgdaIndent{2}{}\<[2]%
\>[2]\AgdaFunction{∼} \AgdaBound{ρ} \AgdaFunction{•RS} \AgdaSymbol{(}\AgdaFunction{x₂:=} \AgdaBound{M2} \AgdaFunction{,x₁:=} \AgdaBound{M1} \AgdaFunction{,x₀:=} \AgdaBound{M0}\AgdaSymbol{)}\<%
\\
\>\AgdaFunction{botSub₃-liftRep₃'} \AgdaInductiveConstructor{x₀} \AgdaSymbol{=} \AgdaInductiveConstructor{refl}\<%
\\
\>\AgdaFunction{botSub₃-liftRep₃'} \AgdaSymbol{(}\AgdaInductiveConstructor{↑} \AgdaInductiveConstructor{x₀}\AgdaSymbol{)} \AgdaSymbol{=} \AgdaInductiveConstructor{refl}\<%
\\
\>\AgdaFunction{botSub₃-liftRep₃'} \AgdaSymbol{(}\AgdaInductiveConstructor{↑} \AgdaSymbol{(}\AgdaInductiveConstructor{↑} \AgdaInductiveConstructor{x₀}\AgdaSymbol{))} \AgdaSymbol{=} \AgdaInductiveConstructor{refl} \<[36]%
\>[36]\<%
\\
\>\AgdaFunction{botSub₃-liftRep₃'} \AgdaSymbol{(}\AgdaInductiveConstructor{↑} \AgdaSymbol{(}\AgdaInductiveConstructor{↑} \AgdaSymbol{(}\AgdaInductiveConstructor{↑} \AgdaBound{x}\AgdaSymbol{)))} \AgdaSymbol{=} \AgdaInductiveConstructor{refl}\<%
\\
%
\\
\>\AgdaFunction{botSub₃-liftRep₃} \AgdaSymbol{:} \AgdaSymbol{∀} \AgdaSymbol{\{}\AgdaBound{U}\AgdaSymbol{\}} \AgdaSymbol{\{}\AgdaBound{V}\AgdaSymbol{\}} \AgdaSymbol{\{}\AgdaBound{K2}\AgdaSymbol{\}} \AgdaSymbol{\{}\AgdaBound{K1}\AgdaSymbol{\}} \AgdaSymbol{\{}\AgdaBound{K0}\AgdaSymbol{\}} \AgdaSymbol{\{}\AgdaBound{L}\AgdaSymbol{\}}\<%
\\
\>[0]\AgdaIndent{2}{}\<[2]%
\>[2]\AgdaSymbol{\{}\AgdaBound{M2} \AgdaSymbol{:} \AgdaFunction{Expression} \AgdaBound{U} \AgdaSymbol{(}\AgdaInductiveConstructor{varKind} \AgdaBound{K2}\AgdaSymbol{)\}} \AgdaSymbol{\{}\AgdaBound{M1} \AgdaSymbol{:} \AgdaFunction{Expression} \AgdaBound{U} \AgdaSymbol{(}\AgdaInductiveConstructor{varKind} \AgdaBound{K1}\AgdaSymbol{)\}} \AgdaSymbol{\{}\AgdaBound{M0} \AgdaSymbol{:} \AgdaFunction{Expression} \AgdaBound{U} \AgdaSymbol{(}\AgdaInductiveConstructor{varKind} \AgdaBound{K0}\AgdaSymbol{)\}} \AgdaSymbol{\{}\AgdaBound{ρ} \AgdaSymbol{:} \AgdaFunction{Rep} \AgdaBound{U} \AgdaBound{V}\AgdaSymbol{\}} \AgdaSymbol{(}\AgdaBound{N} \AgdaSymbol{:} \AgdaFunction{Expression} \AgdaSymbol{(}\AgdaBound{U} \AgdaInductiveConstructor{,} \AgdaBound{K2} \AgdaInductiveConstructor{,} \AgdaBound{K1} \AgdaInductiveConstructor{,} \AgdaBound{K0}\AgdaSymbol{)} \AgdaBound{L}\AgdaSymbol{)} \AgdaSymbol{→}\<%
\\
\>[0]\AgdaIndent{2}{}\<[2]%
\>[2]\AgdaBound{N} \AgdaFunction{〈} \AgdaFunction{liftRep} \AgdaSymbol{\_} \AgdaSymbol{(}\AgdaFunction{liftRep} \AgdaSymbol{\_} \AgdaSymbol{(}\AgdaFunction{liftRep} \AgdaSymbol{\_} \AgdaBound{ρ}\AgdaSymbol{))} \AgdaFunction{〉} \AgdaFunction{⟦} \AgdaFunction{x₂:=} \AgdaBound{M2} \AgdaFunction{〈} \AgdaBound{ρ} \AgdaFunction{〉} \AgdaFunction{,x₁:=} \AgdaBound{M1} \AgdaFunction{〈} \AgdaBound{ρ} \AgdaFunction{〉} \AgdaFunction{,x₀:=} \AgdaBound{M0} \AgdaFunction{〈} \AgdaBound{ρ} \AgdaFunction{〉} \AgdaFunction{⟧}\<%
\\
\>[0]\AgdaIndent{2}{}\<[2]%
\>[2]\AgdaDatatype{≡} \AgdaBound{N} \AgdaFunction{⟦} \AgdaFunction{x₂:=} \AgdaBound{M2} \AgdaFunction{,x₁:=} \AgdaBound{M1} \AgdaFunction{,x₀:=} \AgdaBound{M0} \AgdaFunction{⟧} \AgdaFunction{〈} \AgdaBound{ρ} \AgdaFunction{〉}\<%
\\
\>\AgdaFunction{botSub₃-liftRep₃} \AgdaSymbol{\{}\AgdaArgument{M2} \AgdaSymbol{=} \AgdaBound{M2}\AgdaSymbol{\}} \AgdaSymbol{\{}\AgdaBound{M1}\AgdaSymbol{\}} \AgdaSymbol{\{}\AgdaBound{M0}\AgdaSymbol{\}} \AgdaSymbol{\{}\AgdaBound{ρ}\AgdaSymbol{\}} \AgdaBound{N} \AgdaSymbol{=} \AgdaKeyword{let} \AgdaKeyword{open} \AgdaModule{≡-Reasoning} \AgdaKeyword{in}\<%
\\
\>[2]\AgdaIndent{14}{}\<[14]%
\>[14]\AgdaFunction{begin}\<%
\\
\>[14]\AgdaIndent{16}{}\<[16]%
\>[16]\AgdaBound{N} \AgdaFunction{〈} \AgdaFunction{liftRep} \AgdaSymbol{\_} \AgdaSymbol{(}\AgdaFunction{liftRep} \AgdaSymbol{\_} \AgdaSymbol{(}\AgdaFunction{liftRep} \AgdaSymbol{\_} \AgdaBound{ρ}\AgdaSymbol{))} \AgdaFunction{〉} \AgdaFunction{⟦} \AgdaFunction{x₂:=} \AgdaBound{M2} \AgdaFunction{〈} \AgdaBound{ρ} \AgdaFunction{〉} \AgdaFunction{,x₁:=} \AgdaBound{M1} \AgdaFunction{〈} \AgdaBound{ρ} \AgdaFunction{〉} \AgdaFunction{,x₀:=} \AgdaBound{M0} \AgdaFunction{〈} \AgdaBound{ρ} \AgdaFunction{〉} \AgdaFunction{⟧}\<%
\\
\>[0]\AgdaIndent{14}{}\<[14]%
\>[14]\AgdaFunction{≡⟨⟨} \AgdaFunction{sub-compSR} \AgdaBound{N} \AgdaFunction{⟩⟩}\<%
\\
\>[14]\AgdaIndent{16}{}\<[16]%
\>[16]\AgdaBound{N} \AgdaFunction{⟦} \AgdaSymbol{(}\AgdaFunction{x₂:=} \AgdaBound{M2} \AgdaFunction{〈} \AgdaBound{ρ} \AgdaFunction{〉} \AgdaFunction{,x₁:=} \AgdaBound{M1} \AgdaFunction{〈} \AgdaBound{ρ} \AgdaFunction{〉} \AgdaFunction{,x₀:=} \AgdaBound{M0} \AgdaFunction{〈} \AgdaBound{ρ} \AgdaFunction{〉}\AgdaSymbol{)} \AgdaFunction{•SR} \AgdaFunction{liftRep} \AgdaSymbol{\_} \AgdaSymbol{(}\AgdaFunction{liftRep} \AgdaSymbol{\_} \AgdaSymbol{(}\AgdaFunction{liftRep} \AgdaSymbol{\_} \AgdaBound{ρ}\AgdaSymbol{))} \AgdaFunction{⟧}\<%
\\
\>[0]\AgdaIndent{14}{}\<[14]%
\>[14]\AgdaFunction{≡⟨} \AgdaFunction{sub-congr} \AgdaFunction{botSub₃-liftRep₃'} \AgdaBound{N} \AgdaFunction{⟩}\<%
\\
\>[14]\AgdaIndent{16}{}\<[16]%
\>[16]\AgdaBound{N} \AgdaFunction{⟦} \AgdaBound{ρ} \AgdaFunction{•RS} \AgdaSymbol{(}\AgdaFunction{x₂:=} \AgdaBound{M2} \AgdaFunction{,x₁:=} \AgdaBound{M1} \AgdaFunction{,x₀:=} \AgdaBound{M0}\AgdaSymbol{)} \AgdaFunction{⟧}\<%
\\
\>[0]\AgdaIndent{14}{}\<[14]%
\>[14]\AgdaFunction{≡⟨} \AgdaFunction{sub-compRS} \AgdaBound{N} \AgdaFunction{⟩}\<%
\\
\>[14]\AgdaIndent{16}{}\<[16]%
\>[16]\AgdaBound{N} \AgdaFunction{⟦} \AgdaFunction{x₂:=} \AgdaBound{M2} \AgdaFunction{,x₁:=} \AgdaBound{M1} \AgdaFunction{,x₀:=} \AgdaBound{M0} \AgdaFunction{⟧} \AgdaFunction{〈} \AgdaBound{ρ} \AgdaFunction{〉}\<%
\\
\>[0]\AgdaIndent{14}{}\<[14]%
\>[14]\AgdaFunction{∎}\<%
\\
\>\AgdaComment{--TODO General lemma for this}\<%
\\
\>\AgdaComment{--TODO Deletable?}\<%
\end{code}
}

\AgdaHide{
\begin{code}%
\>\AgdaKeyword{open} \AgdaKeyword{import} \AgdaModule{Grammar.Substitution.Botsub} \AgdaBound{G} \AgdaKeyword{public}\<%
\end{code}
}

\AgdaHide{
\begin{code}%
\>\AgdaKeyword{open} \AgdaKeyword{import} \AgdaModule{Grammar.Substitution} \AgdaBound{G} \AgdaKeyword{public}\<%
\end{code}
}

\AgdaHide{
\begin{code}%
\>\AgdaKeyword{open} \AgdaKeyword{import} \AgdaModule{Grammar.Base}\<%
\\
%
\\
\>\AgdaKeyword{module} \AgdaModule{Grammar.Context} \AgdaSymbol{(}\AgdaBound{G} \AgdaSymbol{:} \AgdaRecord{Grammar}\AgdaSymbol{)} \AgdaKeyword{where}\<%
\\
%
\\
\>\AgdaKeyword{open} \AgdaKeyword{import} \AgdaModule{Data.Nat}\<%
\\
\>\AgdaKeyword{open} \AgdaKeyword{import} \AgdaModule{Data.Fin}\<%
\\
\>\AgdaKeyword{open} \AgdaKeyword{import} \AgdaModule{Relation.Binary.PropositionalEquality}\<%
\\
\>\AgdaKeyword{open} \AgdaModule{Grammar} \AgdaBound{G}\<%
\\
\>\AgdaKeyword{open} \AgdaKeyword{import} \AgdaModule{Grammar.Replacement} \AgdaBound{G}\<%
\end{code}
}

\subsection{Contexts}

A \emph{context} has the form $x_1 : A_1, \ldots, x_n : A_n$ where, for each $i$:
\begin{itemize}
\item $x_i$ is a variable of kind $K_i$ distinct from $x_1$, \ldots, $x_{i-1}$;
\item $A_i$ is an expression whose kind is the parent of $K_i$.
\end{itemize}
The \emph{domain} of this context is the alphabet $\{ x_1, \ldots, x_n \}$.

\begin{code}%
\>\AgdaKeyword{infixl} \AgdaNumber{55} \AgdaFixityOp{\_,\_}\<%
\\
\>\AgdaKeyword{data} \AgdaDatatype{Context} \AgdaSymbol{:} \AgdaDatatype{Alphabet} \AgdaSymbol{→} \AgdaPrimitiveType{Set} \AgdaKeyword{where}\<%
\\
\>[0]\AgdaIndent{2}{}\<[2]%
\>[2]\AgdaInductiveConstructor{〈〉} \AgdaSymbol{:} \AgdaDatatype{Context} \AgdaInductiveConstructor{∅}\<%
\\
\>[0]\AgdaIndent{2}{}\<[2]%
\>[2]\AgdaInductiveConstructor{\_,\_} \AgdaSymbol{:} \AgdaSymbol{∀} \AgdaSymbol{\{}\AgdaBound{V}\AgdaSymbol{\}} \AgdaSymbol{\{}\AgdaBound{K}\AgdaSymbol{\}} \AgdaSymbol{→} \AgdaDatatype{Context} \AgdaBound{V} \AgdaSymbol{→} \AgdaFunction{Expression} \AgdaBound{V} \AgdaSymbol{(}\AgdaFunction{parent} \AgdaBound{K}\AgdaSymbol{)} \AgdaSymbol{→} \<[58]%
\>[58]\<%
\\
\>[2]\AgdaIndent{4}{}\<[4]%
\>[4]\AgdaDatatype{Context} \AgdaSymbol{(}\AgdaBound{V} \AgdaInductiveConstructor{,} \AgdaBound{K}\AgdaSymbol{)}\<%
\\
%
\\
\>\AgdaComment{-- Define typeof such that, if x : A ∈ Γ, then typeof x Γ ≡ A}\<%
\\
\>\AgdaComment{-- We define it the following way so that typeof x Γ computes to an expression of the form}\<%
\\
\>\AgdaComment{-- M 〈 upRep 〉, even if x is not in canonical form}\<%
\\
\>\AgdaFunction{pretypeof} \AgdaSymbol{:} \AgdaSymbol{∀} \AgdaSymbol{\{}\AgdaBound{V}\AgdaSymbol{\}} \AgdaSymbol{\{}\AgdaBound{K}\AgdaSymbol{\}} \AgdaSymbol{\{}\AgdaBound{L}\AgdaSymbol{\}} \AgdaSymbol{(}\AgdaBound{x} \AgdaSymbol{:} \AgdaDatatype{Var} \AgdaSymbol{(}\AgdaBound{V} \AgdaInductiveConstructor{,} \AgdaBound{K}\AgdaSymbol{)} \AgdaBound{L}\AgdaSymbol{)} \AgdaSymbol{(}\AgdaBound{Γ} \AgdaSymbol{:} \AgdaDatatype{Context} \AgdaSymbol{(}\AgdaBound{V} \AgdaInductiveConstructor{,} \AgdaBound{K}\AgdaSymbol{))} \AgdaSymbol{→} \AgdaFunction{Expression} \AgdaBound{V} \AgdaSymbol{(}\AgdaFunction{parent} \AgdaBound{L}\AgdaSymbol{)}\<%
\\
\>\AgdaFunction{typeof} \AgdaSymbol{:} \AgdaSymbol{∀} \AgdaSymbol{\{}\AgdaBound{V}\AgdaSymbol{\}} \AgdaSymbol{\{}\AgdaBound{K}\AgdaSymbol{\}} \AgdaSymbol{(}\AgdaBound{x} \AgdaSymbol{:} \AgdaDatatype{Var} \AgdaBound{V} \AgdaBound{K}\AgdaSymbol{)} \AgdaSymbol{(}\AgdaBound{Γ} \AgdaSymbol{:} \AgdaDatatype{Context} \AgdaBound{V}\AgdaSymbol{)} \AgdaSymbol{→} \AgdaFunction{Expression} \AgdaBound{V} \AgdaSymbol{(}\AgdaFunction{parent} \AgdaBound{K}\AgdaSymbol{)}\<%
\\
%
\\
\>\AgdaFunction{pretypeof} \AgdaInductiveConstructor{x₀} \AgdaSymbol{(}\AgdaBound{Γ} \AgdaInductiveConstructor{,} \AgdaBound{A}\AgdaSymbol{)} \AgdaSymbol{=} \AgdaBound{A}\<%
\\
\>\AgdaFunction{pretypeof} \AgdaSymbol{(}\AgdaInductiveConstructor{↑} \AgdaBound{x}\AgdaSymbol{)} \AgdaSymbol{(}\AgdaBound{Γ} \AgdaInductiveConstructor{,} \AgdaBound{A}\AgdaSymbol{)} \AgdaSymbol{=} \AgdaFunction{typeof} \AgdaBound{x} \AgdaBound{Γ}\<%
\\
%
\\
\>\AgdaFunction{typeof} \AgdaSymbol{\{}\AgdaInductiveConstructor{∅}\AgdaSymbol{\}} \AgdaSymbol{()}\<%
\\
\>\AgdaFunction{typeof} \AgdaSymbol{\{\_} \AgdaInductiveConstructor{,} \AgdaSymbol{\_\}} \AgdaBound{x} \AgdaBound{Γ} \AgdaSymbol{=} \AgdaFunction{pretypeof} \AgdaBound{x} \AgdaBound{Γ} \AgdaFunction{⇑}\<%
\end{code}

We say that a replacement $\rho$ is a \emph{(well-typed) replacement from $\Gamma$ to $\Delta$},
$\rho : \Gamma \rightarrow \Delta$, iff, for each entry $x : A$ in $\Gamma$, we have that
$\rho(x) : A \langle ρ \rangle$ is an entry in $\Delta$.

\begin{code}%
\>\AgdaFunction{\_∶\_⇒R\_} \AgdaSymbol{:} \AgdaSymbol{∀} \AgdaSymbol{\{}\AgdaBound{U}\AgdaSymbol{\}} \AgdaSymbol{\{}\AgdaBound{V}\AgdaSymbol{\}} \AgdaSymbol{→} \AgdaFunction{Rep} \AgdaBound{U} \AgdaBound{V} \AgdaSymbol{→} \AgdaDatatype{Context} \AgdaBound{U} \AgdaSymbol{→} \AgdaDatatype{Context} \AgdaBound{V} \AgdaSymbol{→} \AgdaPrimitiveType{Set}\<%
\\
\>\AgdaBound{ρ} \AgdaFunction{∶} \AgdaBound{Γ} \AgdaFunction{⇒R} \AgdaBound{Δ} \AgdaSymbol{=} \AgdaSymbol{∀} \AgdaSymbol{\{}\AgdaBound{K}\AgdaSymbol{\}} \AgdaBound{x} \AgdaSymbol{→} \AgdaFunction{typeof} \AgdaSymbol{(}\AgdaBound{ρ} \AgdaBound{K} \AgdaBound{x}\AgdaSymbol{)} \AgdaBound{Δ} \AgdaDatatype{≡} \AgdaFunction{typeof} \AgdaBound{x} \AgdaBound{Γ} \AgdaFunction{〈} \AgdaBound{ρ} \AgdaFunction{〉}\<%
\end{code}

\AgdaHide{
\begin{code}%
\>\AgdaKeyword{open} \AgdaKeyword{import} \AgdaModule{Grammar.Context} \AgdaBound{G} \AgdaKeyword{public}\<%
\end{code}
}

\AgdaHide{
\begin{code}%
\>\AgdaKeyword{module} \AgdaModule{PL.Rules} \AgdaKeyword{where}\<%
\\
\>\AgdaKeyword{open} \AgdaKeyword{import} \AgdaModule{Data.Empty}\<%
\\
\>\AgdaKeyword{open} \AgdaKeyword{import} \AgdaModule{Prelims}\<%
\\
\>\AgdaKeyword{open} \AgdaKeyword{import} \AgdaModule{PL.Grammar}\<%
\\
\>\AgdaKeyword{open} \AgdaModule{PLgrammar}\<%
\\
\>\AgdaKeyword{open} \AgdaKeyword{import} \AgdaModule{Grammar} \AgdaFunction{Propositional-Logic}\<%
\\
\>\AgdaKeyword{open} \AgdaKeyword{import} \AgdaModule{Reduction} \AgdaFunction{Propositional-Logic} \AgdaDatatype{β}\<%
\end{code}
}

\subsection{Rules of Deduction}

The rules of deduction of the system are as follows.

\[ \infer[(p : \phi \in \Gamma)]{\Gamma \vdash p : \phi}{\Gamma \vald} \]

\[ \infer{\Gamma \vdash \delta \epsilon : \psi}{\Gamma \vdash \delta : \phi \rightarrow \psi}{\Gamma \vdash \epsilon : \phi} \]

\[ \infer{\Gamma \vdash \lambda p : \phi . \delta : \phi \rightarrow \psi}{\Gamma, p : \phi \vdash \delta : \psi} \]

\begin{code}%
\>\AgdaKeyword{infix} \AgdaNumber{10} \AgdaFixityOp{\_⊢\_∶\_}\<%
\\
\>\AgdaKeyword{data} \AgdaDatatype{\_⊢\_∶\_} \AgdaSymbol{:} \AgdaSymbol{∀} \AgdaSymbol{\{}\AgdaBound{P}\AgdaSymbol{\}} \AgdaSymbol{→} \AgdaDatatype{Context} \AgdaBound{P} \AgdaSymbol{→} \AgdaFunction{Proof} \AgdaBound{P} \AgdaSymbol{→} \AgdaDatatype{Prop} \AgdaSymbol{→} \AgdaPrimitiveType{Set} \AgdaKeyword{where}\<%
\\
\>[0]\AgdaIndent{2}{}\<[2]%
\>[2]\AgdaInductiveConstructor{var} \AgdaSymbol{:} \AgdaSymbol{∀} \AgdaSymbol{\{}\AgdaBound{P}\AgdaSymbol{\}} \AgdaSymbol{\{}\AgdaBound{Γ} \AgdaSymbol{:} \AgdaDatatype{Context} \AgdaBound{P}\AgdaSymbol{\}} \AgdaSymbol{(}\AgdaBound{p} \AgdaSymbol{:} \AgdaDatatype{Var} \AgdaBound{P} \AgdaInductiveConstructor{-proof}\AgdaSymbol{)} \AgdaSymbol{→} \<[51]%
\>[51]\<%
\\
\>[2]\AgdaIndent{4}{}\<[4]%
\>[4]\AgdaBound{Γ} \AgdaDatatype{⊢} \AgdaInductiveConstructor{var} \AgdaBound{p} \AgdaDatatype{∶} \AgdaFunction{unprp} \AgdaSymbol{(}\AgdaFunction{typeof} \AgdaBound{p} \AgdaBound{Γ}\AgdaSymbol{)}\<%
\\
\>[0]\AgdaIndent{2}{}\<[2]%
\>[2]\AgdaInductiveConstructor{app} \AgdaSymbol{:} \AgdaSymbol{∀} \AgdaSymbol{\{}\AgdaBound{P}\AgdaSymbol{\}} \AgdaSymbol{\{}\AgdaBound{Γ} \AgdaSymbol{:} \AgdaDatatype{Context} \AgdaBound{P}\AgdaSymbol{\}} \AgdaSymbol{\{}\AgdaBound{δ}\AgdaSymbol{\}} \AgdaSymbol{\{}\AgdaBound{ε}\AgdaSymbol{\}} \AgdaSymbol{\{}\AgdaBound{φ}\AgdaSymbol{\}} \AgdaSymbol{\{}\AgdaBound{ψ}\AgdaSymbol{\}} \AgdaSymbol{→} \<[48]%
\>[48]\<%
\\
\>[2]\AgdaIndent{4}{}\<[4]%
\>[4]\AgdaBound{Γ} \AgdaDatatype{⊢} \AgdaBound{δ} \AgdaDatatype{∶} \AgdaBound{φ} \AgdaInductiveConstructor{⇛} \AgdaBound{ψ} \AgdaSymbol{→} \AgdaBound{Γ} \AgdaDatatype{⊢} \AgdaBound{ε} \AgdaDatatype{∶} \AgdaBound{φ} \AgdaSymbol{→} \AgdaBound{Γ} \AgdaDatatype{⊢} \AgdaFunction{appP} \AgdaBound{δ} \AgdaBound{ε} \AgdaDatatype{∶} \AgdaBound{ψ}\<%
\\
\>[0]\AgdaIndent{2}{}\<[2]%
\>[2]\AgdaInductiveConstructor{Λ} \AgdaSymbol{:} \AgdaSymbol{∀} \AgdaSymbol{\{}\AgdaBound{P}\AgdaSymbol{\}} \AgdaSymbol{\{}\AgdaBound{Γ} \AgdaSymbol{:} \AgdaDatatype{Context} \AgdaBound{P}\AgdaSymbol{\}} \AgdaSymbol{\{}\AgdaBound{φ}\AgdaSymbol{\}} \AgdaSymbol{\{}\AgdaBound{δ}\AgdaSymbol{\}} \AgdaSymbol{\{}\AgdaBound{ψ}\AgdaSymbol{\}} \AgdaSymbol{→} \<[42]%
\>[42]\<%
\\
\>[2]\AgdaIndent{4}{}\<[4]%
\>[4]\AgdaBound{Γ} \AgdaFunction{,P} \AgdaBound{φ} \AgdaDatatype{⊢} \AgdaBound{δ} \AgdaDatatype{∶} \AgdaBound{ψ} \AgdaSymbol{→} \AgdaBound{Γ} \AgdaDatatype{⊢} \AgdaFunction{ΛP} \AgdaBound{φ} \AgdaBound{δ} \AgdaDatatype{∶} \AgdaBound{φ} \AgdaInductiveConstructor{⇛} \AgdaBound{ψ}\<%
\end{code}

\AgdaHide{
\begin{code}%
\>\AgdaFunction{change-type} \AgdaSymbol{:} \AgdaSymbol{∀} \AgdaSymbol{\{}\AgdaBound{P}\AgdaSymbol{\}} \AgdaSymbol{\{}\AgdaBound{Γ} \AgdaSymbol{:} \AgdaDatatype{Context} \AgdaBound{P}\AgdaSymbol{\}} \AgdaSymbol{\{}\AgdaBound{δ} \AgdaBound{φ} \AgdaBound{ψ}\AgdaSymbol{\}} \AgdaSymbol{→}\<%
\\
\>[0]\AgdaIndent{2}{}\<[2]%
\>[2]\AgdaBound{φ} \AgdaDatatype{≡} \AgdaBound{ψ} \AgdaSymbol{→} \AgdaBound{Γ} \AgdaDatatype{⊢} \AgdaBound{δ} \AgdaDatatype{∶} \AgdaBound{φ} \AgdaSymbol{→} \AgdaBound{Γ} \AgdaDatatype{⊢} \AgdaBound{δ} \AgdaDatatype{∶} \AgdaBound{ψ}\<%
\\
\>\AgdaFunction{change-type} \AgdaSymbol{=} \AgdaFunction{subst} \AgdaSymbol{(λ} \AgdaBound{A} \AgdaSymbol{→} \AgdaSymbol{\_} \AgdaDatatype{⊢} \AgdaSymbol{\_} \AgdaDatatype{∶} \AgdaBound{A}\AgdaSymbol{)}\<%
\end{code}
}

Let $\rho$ be a replacement.  We say $\rho$ is a replacement from $\Gamma$ to $\Delta$, $\rho : \Gamma \rightarrow \Delta$,
iff for all $x : \phi \in \Gamma$ we have $\rho(x) : \phi \in \Delta$.

\begin{code}%
\>\AgdaFunction{\_∶\_⇒R\_} \AgdaSymbol{:} \AgdaSymbol{∀} \AgdaSymbol{\{}\AgdaBound{P}\AgdaSymbol{\}} \AgdaSymbol{\{}\AgdaBound{Q}\AgdaSymbol{\}} \AgdaSymbol{→} \AgdaFunction{Rep} \AgdaBound{P} \AgdaBound{Q} \AgdaSymbol{→} \AgdaDatatype{Context} \AgdaBound{P} \AgdaSymbol{→} \AgdaDatatype{Context} \AgdaBound{Q} \AgdaSymbol{→} \AgdaPrimitiveType{Set}\<%
\\
\>\AgdaBound{ρ} \AgdaFunction{∶} \AgdaBound{Γ} \AgdaFunction{⇒R} \AgdaBound{Δ} \AgdaSymbol{=} \AgdaSymbol{∀} \AgdaBound{x} \AgdaSymbol{→} \AgdaFunction{unprp} \AgdaSymbol{(}\AgdaFunction{typeof} \AgdaSymbol{\{}\AgdaArgument{K} \AgdaSymbol{=} \AgdaInductiveConstructor{-proof}\AgdaSymbol{\}} \AgdaSymbol{(}\AgdaBound{ρ} \AgdaSymbol{\_} \AgdaBound{x}\AgdaSymbol{)} \AgdaBound{Δ}\AgdaSymbol{)} \AgdaDatatype{≡} \AgdaFunction{unprp} \AgdaSymbol{(}\AgdaFunction{typeof} \AgdaBound{x} \AgdaBound{Γ} \AgdaSymbol{)}\<%
\end{code}

\begin{lemma}$ $
\begin{enumerate}
\item
$\id{P}$ is a replacement $\Gamma \rightarrow \Gamma$.
\item
$\uparrow$ is a replacement $\Gamma \rightarrow \Gamma , \phi$.
\item
If $\rho : \Gamma \rightarrow \Delta$ then $(\rho , \mathrm{Proof}) : (\Gamma , x : \phi) \rightarrow (\Delta , x : \phi)$.
\item
If $\rho : \Gamma \rightarrow \Delta$ and $\sigma : \Delta \rightarrow \Theta$ then $\sigma \circ \rho : \Gamma \rightarrow \Delta$.
\item
(\textbf{Weakening})
If $\rho : \Gamma \rightarrow \Delta$ and $\Gamma \vdash \delta : \phi$ then $\Delta \vdash \delta \langle \rho \rangle : \phi$.
\end{enumerate}
\end{lemma}

\begin{code}%
\>\AgdaFunction{idRep-typed} \AgdaSymbol{:} \AgdaSymbol{∀} \AgdaSymbol{\{}\AgdaBound{P}\AgdaSymbol{\}} \AgdaSymbol{\{}\AgdaBound{Γ} \AgdaSymbol{:} \AgdaDatatype{Context} \AgdaBound{P}\AgdaSymbol{\}} \AgdaSymbol{→} \AgdaFunction{idRep} \AgdaBound{P} \AgdaFunction{∶} \AgdaBound{Γ} \AgdaFunction{⇒R} \AgdaBound{Γ}\<%
\end{code}

\AgdaHide{
\begin{code}%
\>\AgdaFunction{idRep-typed} \AgdaSymbol{\{}\AgdaBound{P}\AgdaSymbol{\}} \AgdaSymbol{\{}\AgdaBound{Γ}\AgdaSymbol{\}} \AgdaBound{x} \AgdaSymbol{=} \AgdaInductiveConstructor{refl}\<%
\end{code}
}

\begin{code}%
\>\AgdaFunction{unprp-rep} \AgdaSymbol{:} \AgdaSymbol{∀} \AgdaSymbol{\{}\AgdaBound{U} \AgdaBound{V}\AgdaSymbol{\}} \AgdaBound{φ} \AgdaSymbol{(}\AgdaBound{ρ} \AgdaSymbol{:} \AgdaFunction{Rep} \AgdaBound{U} \AgdaBound{V}\AgdaSymbol{)} \AgdaSymbol{→} \AgdaFunction{unprp} \AgdaSymbol{(}\AgdaBound{φ} \AgdaFunction{〈} \AgdaBound{ρ} \AgdaFunction{〉}\AgdaSymbol{)} \AgdaDatatype{≡} \AgdaFunction{unprp} \AgdaBound{φ}\<%
\\
\>\AgdaFunction{unprp-rep} \AgdaSymbol{(}\AgdaInductiveConstructor{app} \AgdaSymbol{(}\AgdaInductiveConstructor{-prp} \AgdaSymbol{\_)} \AgdaInductiveConstructor{[]}\AgdaSymbol{)} \AgdaSymbol{\_} \AgdaSymbol{=} \AgdaInductiveConstructor{refl}\<%
\\
%
\\
\>\AgdaFunction{↑-typed} \AgdaSymbol{:} \AgdaSymbol{∀} \AgdaSymbol{\{}\AgdaBound{P}\AgdaSymbol{\}} \AgdaSymbol{\{}\AgdaBound{Γ} \AgdaSymbol{:} \AgdaDatatype{Context} \AgdaBound{P}\AgdaSymbol{\}} \AgdaSymbol{\{}\AgdaBound{φ} \AgdaSymbol{:} \AgdaDatatype{Prop}\AgdaSymbol{\}} \AgdaSymbol{→} \AgdaFunction{upRep} \AgdaFunction{∶} \AgdaBound{Γ} \AgdaFunction{⇒R} \AgdaSymbol{(}\AgdaBound{Γ} \AgdaFunction{,P} \AgdaBound{φ}\AgdaSymbol{)}\<%
\end{code}

\AgdaHide{
\begin{code}%
\>\AgdaFunction{↑-typed} \AgdaSymbol{\{}\AgdaBound{P}\AgdaSymbol{\}} \AgdaSymbol{\{}\AgdaBound{Γ}\AgdaSymbol{\}} \AgdaSymbol{\{}\AgdaBound{φ}\AgdaSymbol{\}} \AgdaBound{x} \AgdaSymbol{=} \AgdaFunction{unprp-rep} \AgdaSymbol{(}\AgdaFunction{typeof} \AgdaBound{x} \AgdaBound{Γ}\AgdaSymbol{)} \AgdaFunction{upRep}\<%
\end{code}
}

\begin{code}%
\>\AgdaFunction{liftRep-typed} \AgdaSymbol{:} \AgdaSymbol{∀} \AgdaSymbol{\{}\AgdaBound{P}\AgdaSymbol{\}} \AgdaSymbol{\{}\AgdaBound{Q}\AgdaSymbol{\}} \AgdaSymbol{\{}\AgdaBound{ρ}\AgdaSymbol{\}} \AgdaSymbol{\{}\AgdaBound{Γ} \AgdaSymbol{:} \AgdaDatatype{Context} \AgdaBound{P}\AgdaSymbol{\}} \AgdaSymbol{\{}\AgdaBound{Δ} \AgdaSymbol{:} \AgdaDatatype{Context} \AgdaBound{Q}\AgdaSymbol{\}} \AgdaSymbol{\{}\AgdaBound{φ} \AgdaSymbol{:} \AgdaDatatype{Prop}\AgdaSymbol{\}} \AgdaSymbol{→} \<[75]%
\>[75]\<%
\\
\>[0]\AgdaIndent{2}{}\<[2]%
\>[2]\AgdaBound{ρ} \AgdaFunction{∶} \AgdaBound{Γ} \AgdaFunction{⇒R} \AgdaBound{Δ} \AgdaSymbol{→} \AgdaFunction{liftRep} \AgdaInductiveConstructor{-proof} \AgdaBound{ρ} \AgdaFunction{∶} \AgdaSymbol{(}\AgdaBound{Γ} \AgdaFunction{,P} \AgdaBound{φ}\AgdaSymbol{)} \AgdaFunction{⇒R} \AgdaSymbol{(}\AgdaBound{Δ} \AgdaFunction{,P} \AgdaBound{φ}\AgdaSymbol{)}\<%
\end{code}

\AgdaHide{
\begin{code}%
\>\AgdaFunction{liftRep-typed} \AgdaSymbol{\{}\AgdaBound{P}\AgdaSymbol{\}} \AgdaSymbol{\{}\AgdaArgument{Q} \AgdaSymbol{=} \AgdaBound{Q}\AgdaSymbol{\}} \AgdaSymbol{\{}\AgdaArgument{ρ} \AgdaSymbol{=} \AgdaBound{ρ}\AgdaSymbol{\}} \AgdaSymbol{\{}\AgdaBound{Γ}\AgdaSymbol{\}} \AgdaSymbol{\{}\AgdaArgument{Δ} \AgdaSymbol{=} \AgdaBound{Δ}\AgdaSymbol{\}} \AgdaSymbol{\{}\AgdaArgument{φ} \AgdaSymbol{=} \AgdaBound{φ}\AgdaSymbol{\}} \AgdaBound{ρ∶Γ→Δ} \AgdaInductiveConstructor{x₀} \AgdaSymbol{=} \AgdaInductiveConstructor{refl}\<%
\\
\>\AgdaFunction{liftRep-typed} \AgdaSymbol{\{}\AgdaArgument{Q} \AgdaSymbol{=} \AgdaBound{Q}\AgdaSymbol{\}} \AgdaSymbol{\{}\AgdaArgument{ρ} \AgdaSymbol{=} \AgdaBound{ρ}\AgdaSymbol{\}} \AgdaSymbol{\{}\AgdaArgument{Γ} \AgdaSymbol{=} \AgdaBound{Γ}\AgdaSymbol{\}} \AgdaSymbol{\{}\AgdaArgument{Δ} \AgdaSymbol{=} \AgdaBound{Δ}\AgdaSymbol{\}} \AgdaSymbol{\{}\AgdaBound{φ}\AgdaSymbol{\}} \AgdaBound{ρ∶Γ→Δ} \AgdaSymbol{(}\AgdaInductiveConstructor{↑} \AgdaBound{x}\AgdaSymbol{)} \AgdaSymbol{=} \<[64]%
\>[64]\<%
\\
\>[0]\AgdaIndent{2}{}\<[2]%
\>[2]\AgdaKeyword{let} \AgdaKeyword{open} \AgdaModule{≡-Reasoning} \AgdaKeyword{in} \<[26]%
\>[26]\<%
\\
\>[0]\AgdaIndent{2}{}\<[2]%
\>[2]\AgdaFunction{begin}\<%
\\
\>[2]\AgdaIndent{4}{}\<[4]%
\>[4]\AgdaFunction{unprp} \AgdaSymbol{(}\AgdaFunction{typeof} \AgdaSymbol{(}\AgdaFunction{liftRep} \AgdaInductiveConstructor{-proof} \AgdaBound{ρ} \AgdaInductiveConstructor{-proof} \AgdaSymbol{(}\AgdaInductiveConstructor{↑} \AgdaBound{x}\AgdaSymbol{))} \AgdaSymbol{(}\AgdaBound{Δ} \AgdaFunction{,P} \AgdaBound{φ}\AgdaSymbol{))}\<%
\\
\>[0]\AgdaIndent{2}{}\<[2]%
\>[2]\AgdaFunction{≡⟨⟩}\<%
\\
\>[2]\AgdaIndent{4}{}\<[4]%
\>[4]\AgdaFunction{unprp} \AgdaSymbol{(}\AgdaFunction{typeof} \AgdaSymbol{(}\AgdaInductiveConstructor{↑} \AgdaSymbol{(}\AgdaBound{ρ} \AgdaInductiveConstructor{-proof} \AgdaBound{x}\AgdaSymbol{))} \AgdaSymbol{(}\AgdaBound{Δ} \AgdaFunction{,P} \AgdaBound{φ}\AgdaSymbol{))}\<%
\\
\>[0]\AgdaIndent{2}{}\<[2]%
\>[2]\AgdaFunction{≡⟨⟩}\<%
\\
\>[2]\AgdaIndent{4}{}\<[4]%
\>[4]\AgdaFunction{unprp} \AgdaSymbol{(}\AgdaFunction{typeof} \AgdaSymbol{(}\AgdaBound{ρ} \AgdaInductiveConstructor{-proof} \AgdaBound{x}\AgdaSymbol{)} \AgdaBound{Δ} \AgdaFunction{〈} \AgdaFunction{upRep} \AgdaFunction{〉}\AgdaSymbol{)}\<%
\\
\>[0]\AgdaIndent{2}{}\<[2]%
\>[2]\AgdaFunction{≡⟨} \AgdaFunction{unprp-rep} \AgdaSymbol{(}\AgdaFunction{typeof} \AgdaSymbol{(}\AgdaBound{ρ} \AgdaInductiveConstructor{-proof} \AgdaBound{x}\AgdaSymbol{)} \AgdaBound{Δ}\AgdaSymbol{)} \AgdaFunction{upRep} \AgdaFunction{⟩}\<%
\\
\>[2]\AgdaIndent{4}{}\<[4]%
\>[4]\AgdaFunction{unprp} \AgdaSymbol{(}\AgdaFunction{typeof} \AgdaSymbol{(}\AgdaBound{ρ} \AgdaInductiveConstructor{-proof} \AgdaBound{x}\AgdaSymbol{)} \AgdaBound{Δ}\AgdaSymbol{)}\<%
\\
\>[0]\AgdaIndent{2}{}\<[2]%
\>[2]\AgdaFunction{≡⟨} \AgdaBound{ρ∶Γ→Δ} \AgdaBound{x} \AgdaFunction{⟩}\<%
\\
\>[2]\AgdaIndent{4}{}\<[4]%
\>[4]\AgdaFunction{unprp} \AgdaSymbol{(}\AgdaFunction{typeof} \AgdaBound{x} \AgdaBound{Γ}\AgdaSymbol{)}\<%
\\
\>[0]\AgdaIndent{2}{}\<[2]%
\>[2]\AgdaFunction{≡⟨⟨} \AgdaFunction{unprp-rep} \AgdaSymbol{(}\AgdaFunction{typeof} \AgdaBound{x} \AgdaBound{Γ}\AgdaSymbol{)} \AgdaFunction{upRep} \AgdaFunction{⟩⟩}\<%
\\
\>[2]\AgdaIndent{4}{}\<[4]%
\>[4]\AgdaFunction{unprp} \AgdaSymbol{(}\AgdaFunction{typeof} \AgdaBound{x} \AgdaBound{Γ} \AgdaFunction{〈} \AgdaFunction{upRep} \AgdaFunction{〉}\AgdaSymbol{)}\<%
\\
\>[0]\AgdaIndent{2}{}\<[2]%
\>[2]\AgdaFunction{≡⟨⟩}\<%
\\
\>[2]\AgdaIndent{4}{}\<[4]%
\>[4]\AgdaFunction{unprp} \AgdaSymbol{(}\AgdaFunction{typeof} \AgdaSymbol{(}\AgdaInductiveConstructor{↑} \AgdaBound{x}\AgdaSymbol{)} \AgdaSymbol{(}\AgdaBound{Γ} \AgdaFunction{,P} \AgdaBound{φ}\AgdaSymbol{))}\<%
\\
\>[0]\AgdaIndent{2}{}\<[2]%
\>[2]\AgdaFunction{∎}\<%
\end{code}
}

\begin{code}%
\>\AgdaFunction{•R-typed} \AgdaSymbol{:} \AgdaSymbol{∀} \AgdaSymbol{\{}\AgdaBound{P}\AgdaSymbol{\}} \AgdaSymbol{\{}\AgdaBound{Q}\AgdaSymbol{\}} \AgdaSymbol{\{}\AgdaBound{R}\AgdaSymbol{\}} \AgdaSymbol{\{}\AgdaBound{σ} \AgdaSymbol{:} \AgdaFunction{Rep} \AgdaBound{Q} \AgdaBound{R}\AgdaSymbol{\}} \AgdaSymbol{\{}\AgdaBound{ρ} \AgdaSymbol{:} \AgdaFunction{Rep} \AgdaBound{P} \AgdaBound{Q}\AgdaSymbol{\}} \AgdaSymbol{\{}\AgdaBound{Γ}\AgdaSymbol{\}} \AgdaSymbol{\{}\AgdaBound{Δ}\AgdaSymbol{\}} \AgdaSymbol{\{}\AgdaBound{Θ}\AgdaSymbol{\}} \AgdaSymbol{→} \<[67]%
\>[67]\<%
\\
\>[0]\AgdaIndent{2}{}\<[2]%
\>[2]\AgdaBound{ρ} \AgdaFunction{∶} \AgdaBound{Γ} \AgdaFunction{⇒R} \AgdaBound{Δ} \AgdaSymbol{→} \AgdaBound{σ} \AgdaFunction{∶} \AgdaBound{Δ} \AgdaFunction{⇒R} \AgdaBound{Θ} \AgdaSymbol{→} \AgdaSymbol{(}\AgdaBound{σ} \AgdaFunction{•R} \AgdaBound{ρ}\AgdaSymbol{)} \AgdaFunction{∶} \AgdaBound{Γ} \AgdaFunction{⇒R} \AgdaBound{Θ}\<%
\end{code}

\AgdaHide{
\begin{code}%
\>\AgdaFunction{•R-typed} \AgdaSymbol{\{}\AgdaArgument{R} \AgdaSymbol{=} \AgdaBound{R}\AgdaSymbol{\}} \AgdaSymbol{\{}\AgdaBound{σ}\AgdaSymbol{\}} \AgdaSymbol{\{}\AgdaBound{ρ}\AgdaSymbol{\}} \AgdaSymbol{\{}\AgdaBound{Γ}\AgdaSymbol{\}} \AgdaSymbol{\{}\AgdaBound{Δ}\AgdaSymbol{\}} \AgdaSymbol{\{}\AgdaBound{Θ}\AgdaSymbol{\}} \AgdaBound{ρ∶Γ→Δ} \AgdaBound{σ∶Δ→Θ} \AgdaBound{x} \AgdaSymbol{=} \AgdaKeyword{let} \AgdaKeyword{open} \AgdaModule{≡-Reasoning} \AgdaKeyword{in} \<[77]%
\>[77]\<%
\\
\>[0]\AgdaIndent{2}{}\<[2]%
\>[2]\AgdaFunction{begin} \<[8]%
\>[8]\<%
\\
\>[2]\AgdaIndent{4}{}\<[4]%
\>[4]\AgdaFunction{unprp} \AgdaSymbol{(}\AgdaFunction{typeof} \AgdaSymbol{(}\AgdaBound{σ} \AgdaInductiveConstructor{-proof} \AgdaSymbol{(}\AgdaBound{ρ} \AgdaInductiveConstructor{-proof} \AgdaBound{x}\AgdaSymbol{))} \AgdaBound{Θ}\AgdaSymbol{)}\<%
\\
\>[0]\AgdaIndent{2}{}\<[2]%
\>[2]\AgdaFunction{≡⟨} \AgdaBound{σ∶Δ→Θ} \AgdaSymbol{(}\AgdaBound{ρ} \AgdaInductiveConstructor{-proof} \AgdaBound{x}\AgdaSymbol{)} \AgdaFunction{⟩}\<%
\\
\>[2]\AgdaIndent{4}{}\<[4]%
\>[4]\AgdaFunction{unprp} \AgdaSymbol{(}\AgdaFunction{typeof} \AgdaSymbol{(}\AgdaBound{ρ} \AgdaInductiveConstructor{-proof} \AgdaBound{x}\AgdaSymbol{)} \AgdaBound{Δ}\AgdaSymbol{)}\<%
\\
\>[0]\AgdaIndent{2}{}\<[2]%
\>[2]\AgdaFunction{≡⟨} \AgdaBound{ρ∶Γ→Δ} \AgdaBound{x} \AgdaFunction{⟩}\<%
\\
\>[2]\AgdaIndent{4}{}\<[4]%
\>[4]\AgdaFunction{unprp} \AgdaSymbol{(}\AgdaFunction{typeof} \AgdaBound{x} \AgdaBound{Γ}\AgdaSymbol{)}\<%
\\
\>[0]\AgdaIndent{2}{}\<[2]%
\>[2]\AgdaFunction{∎}\<%
\end{code}
}

\begin{code}%
\>\AgdaFunction{weakening} \AgdaSymbol{:} \AgdaSymbol{∀} \AgdaSymbol{\{}\AgdaBound{P}\AgdaSymbol{\}} \AgdaSymbol{\{}\AgdaBound{Q}\AgdaSymbol{\}} \AgdaSymbol{\{}\AgdaBound{Γ} \AgdaSymbol{:} \AgdaDatatype{Context} \AgdaBound{P}\AgdaSymbol{\}} \AgdaSymbol{\{}\AgdaBound{Δ} \AgdaSymbol{:} \AgdaDatatype{Context} \AgdaBound{Q}\AgdaSymbol{\}} \AgdaSymbol{\{}\AgdaBound{ρ}\AgdaSymbol{\}} \AgdaSymbol{\{}\AgdaBound{δ}\AgdaSymbol{\}} \AgdaSymbol{\{}\AgdaBound{φ}\AgdaSymbol{\}} \AgdaSymbol{→} \<[68]%
\>[68]\<%
\\
\>[0]\AgdaIndent{2}{}\<[2]%
\>[2]\AgdaBound{Γ} \AgdaDatatype{⊢} \AgdaBound{δ} \AgdaDatatype{∶} \AgdaBound{φ} \AgdaSymbol{→} \AgdaBound{ρ} \AgdaFunction{∶} \AgdaBound{Γ} \AgdaFunction{⇒R} \AgdaBound{Δ} \AgdaSymbol{→} \AgdaBound{Δ} \AgdaDatatype{⊢} \AgdaBound{δ} \AgdaFunction{〈} \AgdaBound{ρ} \AgdaFunction{〉} \AgdaDatatype{∶} \AgdaBound{φ}\<%
\end{code}

\AgdaHide{
\begin{code}%
\>\AgdaFunction{weakening} \AgdaSymbol{\{}\AgdaBound{P}\AgdaSymbol{\}} \AgdaSymbol{\{}\AgdaBound{Q}\AgdaSymbol{\}} \AgdaSymbol{\{}\AgdaBound{Γ}\AgdaSymbol{\}} \AgdaSymbol{\{}\AgdaBound{Δ}\AgdaSymbol{\}} \AgdaSymbol{\{}\AgdaBound{ρ}\AgdaSymbol{\}} \AgdaSymbol{(}\AgdaInductiveConstructor{var} \AgdaBound{p}\AgdaSymbol{)} \AgdaBound{ρ∶Γ→Δ} \AgdaSymbol{=} \AgdaFunction{change-type} \AgdaSymbol{(}\AgdaBound{ρ∶Γ→Δ} \AgdaBound{p}\AgdaSymbol{)} \AgdaSymbol{(}\AgdaInductiveConstructor{var} \AgdaSymbol{(}\AgdaBound{ρ} \AgdaSymbol{\_} \AgdaBound{p}\AgdaSymbol{))}\<%
\\
\>\AgdaFunction{weakening} \AgdaSymbol{(}\AgdaInductiveConstructor{app} \AgdaBound{Γ⊢δ∶φ→ψ} \AgdaBound{Γ⊢ε∶φ}\AgdaSymbol{)} \AgdaBound{ρ∶Γ→Δ} \AgdaSymbol{=} \AgdaInductiveConstructor{app} \AgdaSymbol{(}\AgdaFunction{weakening} \AgdaBound{Γ⊢δ∶φ→ψ} \AgdaBound{ρ∶Γ→Δ}\AgdaSymbol{)} \AgdaSymbol{(}\AgdaFunction{weakening} \AgdaBound{Γ⊢ε∶φ} \AgdaBound{ρ∶Γ→Δ}\AgdaSymbol{)}\<%
\\
\>\AgdaFunction{weakening} \AgdaSymbol{.\{}\AgdaBound{P}\AgdaSymbol{\}} \AgdaSymbol{\{}\AgdaBound{Q}\AgdaSymbol{\}} \AgdaSymbol{.\{}\AgdaBound{Γ}\AgdaSymbol{\}} \AgdaSymbol{\{}\AgdaBound{Δ}\AgdaSymbol{\}} \AgdaSymbol{\{}\AgdaBound{ρ}\AgdaSymbol{\}} \AgdaSymbol{(}\AgdaInductiveConstructor{Λ} \AgdaSymbol{\{}\AgdaBound{P}\AgdaSymbol{\}} \AgdaSymbol{\{}\AgdaBound{Γ}\AgdaSymbol{\}} \AgdaSymbol{\{}\AgdaBound{φ}\AgdaSymbol{\}} \AgdaSymbol{\{}\AgdaBound{δ}\AgdaSymbol{\}} \AgdaSymbol{\{}\AgdaBound{ψ}\AgdaSymbol{\}} \AgdaBound{Γ,φ⊢δ∶ψ}\AgdaSymbol{)} \AgdaBound{ρ∶Γ→Δ} \AgdaSymbol{=} \AgdaInductiveConstructor{Λ} \<[74]%
\>[74]\<%
\\
\>[0]\AgdaIndent{2}{}\<[2]%
\>[2]\AgdaSymbol{(}\AgdaFunction{weakening} \AgdaSymbol{\{}\AgdaBound{P} \AgdaInductiveConstructor{,} \AgdaInductiveConstructor{-proof}\AgdaSymbol{\}} \AgdaSymbol{\{}\AgdaBound{Q} \AgdaInductiveConstructor{,} \AgdaInductiveConstructor{-proof}\AgdaSymbol{\}} \AgdaSymbol{\{}\AgdaBound{Γ} \AgdaFunction{,P} \AgdaBound{φ}\AgdaSymbol{\}} \AgdaSymbol{\{}\AgdaBound{Δ} \AgdaFunction{,P} \AgdaBound{φ}\AgdaSymbol{\}} \AgdaSymbol{\{}\AgdaFunction{liftRep} \AgdaInductiveConstructor{-proof} \AgdaBound{ρ}\AgdaSymbol{\}} \AgdaSymbol{\{}\AgdaBound{δ}\AgdaSymbol{\}} \AgdaSymbol{\{}\AgdaBound{ψ}\AgdaSymbol{\}} \<[84]%
\>[84]\<%
\\
\>[2]\AgdaIndent{4}{}\<[4]%
\>[4]\AgdaBound{Γ,φ⊢δ∶ψ} \AgdaSymbol{(}\AgdaFunction{liftRep-typed} \AgdaBound{ρ∶Γ→Δ}\AgdaSymbol{))}\<%
\end{code}
}
A \emph{substitution} $\sigma$ from a context $\Gamma$ to a context $\Delta$, $\sigma : \Gamma \rightarrow \Delta$,  is a substitution $\sigma$ such that
for every $x : \phi$ in $\Gamma$, we have $\Delta \vdash \sigma(x) : \phi$.

\begin{code}%
\>\AgdaFunction{\_∶\_⇒\_} \AgdaSymbol{:} \AgdaSymbol{∀} \AgdaSymbol{\{}\AgdaBound{P}\AgdaSymbol{\}} \AgdaSymbol{\{}\AgdaBound{Q}\AgdaSymbol{\}} \AgdaSymbol{→} \AgdaFunction{Sub} \AgdaBound{P} \AgdaBound{Q} \AgdaSymbol{→} \AgdaDatatype{Context} \AgdaBound{P} \AgdaSymbol{→} \AgdaDatatype{Context} \AgdaBound{Q} \AgdaSymbol{→} \AgdaPrimitiveType{Set}\<%
\\
\>\AgdaBound{σ} \AgdaFunction{∶} \AgdaBound{Γ} \AgdaFunction{⇒} \AgdaBound{Δ} \AgdaSymbol{=} \AgdaSymbol{∀} \AgdaBound{x} \AgdaSymbol{→} \AgdaBound{Δ} \AgdaDatatype{⊢} \AgdaBound{σ} \AgdaSymbol{\_} \AgdaBound{x} \AgdaDatatype{∶} \AgdaFunction{unprp} \AgdaSymbol{(}\AgdaFunction{typeof} \AgdaBound{x} \AgdaBound{Γ}\AgdaSymbol{)}\<%
\end{code}

\begin{lemma}$ $
\begin{enumerate}
\item
If $\sigma : \Gamma \rightarrow \Delta$ then $(\sigma , \mathrm{Proof}) : (\Gamma , p : \phi) \rightarrow (\Delta , p : \phi [ \sigma ])$.
\item
If $\Gamma \vdash \delta : \phi$ then $(p := \delta) : (\Gamma, p : \phi) \rightarrow \Gamma$.
\item
(\textbf{substitution Lemma})

If $\Gamma \vdash \delta : \phi$ and $\sigma : \Gamma \rightarrow \Delta$ then $\Delta \vdash \delta [ \sigma ] : \phi [ \sigma ]$.
\end{enumerate}
\end{lemma}

\begin{code}%
\>\AgdaFunction{liftSub-typed} \AgdaSymbol{:} \AgdaSymbol{∀} \AgdaSymbol{\{}\AgdaBound{P}\AgdaSymbol{\}} \AgdaSymbol{\{}\AgdaBound{Q}\AgdaSymbol{\}} \AgdaSymbol{\{}\AgdaBound{σ}\AgdaSymbol{\}} \<[30]%
\>[30]\<%
\\
\>[0]\AgdaIndent{2}{}\<[2]%
\>[2]\AgdaSymbol{\{}\AgdaBound{Γ} \AgdaSymbol{:} \AgdaDatatype{Context} \AgdaBound{P}\AgdaSymbol{\}} \AgdaSymbol{\{}\AgdaBound{Δ} \AgdaSymbol{:} \AgdaDatatype{Context} \AgdaBound{Q}\AgdaSymbol{\}} \AgdaSymbol{\{}\AgdaBound{φ} \AgdaSymbol{:} \AgdaDatatype{Prop}\AgdaSymbol{\}} \AgdaSymbol{→} \<[47]%
\>[47]\<%
\\
\>[0]\AgdaIndent{2}{}\<[2]%
\>[2]\AgdaBound{σ} \AgdaFunction{∶} \AgdaBound{Γ} \AgdaFunction{⇒} \AgdaBound{Δ} \AgdaSymbol{→} \AgdaFunction{liftSub} \AgdaInductiveConstructor{-proof} \AgdaBound{σ} \AgdaFunction{∶} \AgdaSymbol{(}\AgdaBound{Γ} \AgdaFunction{,P} \AgdaBound{φ}\AgdaSymbol{)} \AgdaFunction{⇒} \AgdaSymbol{(}\AgdaBound{Δ} \AgdaFunction{,P} \AgdaBound{φ}\AgdaSymbol{)}\<%
\end{code}

\AgdaHide{
\begin{code}%
\>\AgdaFunction{liftSub-typed} \AgdaSymbol{\{}\AgdaArgument{σ} \AgdaSymbol{=} \AgdaBound{σ}\AgdaSymbol{\}} \AgdaSymbol{\{}\AgdaBound{Γ}\AgdaSymbol{\}} \AgdaSymbol{\{}\AgdaBound{Δ}\AgdaSymbol{\}} \AgdaSymbol{\{}\AgdaBound{φ}\AgdaSymbol{\}} \AgdaBound{σ∶Γ⇒Δ} \AgdaBound{x} \AgdaSymbol{=}\<%
\\
\>[0]\AgdaIndent{2}{}\<[2]%
\>[2]\AgdaFunction{change-type} \AgdaSymbol{(}\AgdaFunction{sym} \AgdaSymbol{(}\AgdaFunction{unprp-rep} \AgdaSymbol{(}\AgdaFunction{pretypeof} \AgdaBound{x} \AgdaSymbol{(}\AgdaBound{Γ} \AgdaFunction{,P} \AgdaBound{φ}\AgdaSymbol{))} \AgdaFunction{upRep}\AgdaSymbol{))} \AgdaSymbol{(}\AgdaFunction{pre-LiftSub-typed} \AgdaBound{x}\AgdaSymbol{)} \AgdaKeyword{where}\<%
\\
\>[0]\AgdaIndent{2}{}\<[2]%
\>[2]\AgdaFunction{pre-LiftSub-typed} \AgdaSymbol{:} \AgdaSymbol{∀} \AgdaBound{x} \AgdaSymbol{→} \AgdaBound{Δ} \AgdaFunction{,P} \AgdaBound{φ} \AgdaDatatype{⊢} \AgdaFunction{liftSub} \AgdaInductiveConstructor{-proof} \AgdaBound{σ} \AgdaInductiveConstructor{-proof} \AgdaBound{x} \AgdaDatatype{∶} \AgdaFunction{unprp} \AgdaSymbol{(}\AgdaFunction{pretypeof} \AgdaBound{x} \AgdaSymbol{(}\AgdaBound{Γ} \AgdaFunction{,P} \AgdaBound{φ}\AgdaSymbol{))}\<%
\\
\>[0]\AgdaIndent{2}{}\<[2]%
\>[2]\AgdaFunction{pre-LiftSub-typed} \AgdaInductiveConstructor{x₀} \AgdaSymbol{=} \AgdaInductiveConstructor{var} \AgdaInductiveConstructor{x₀}\<%
\\
\>[0]\AgdaIndent{2}{}\<[2]%
\>[2]\AgdaFunction{pre-LiftSub-typed} \AgdaSymbol{(}\AgdaInductiveConstructor{↑} \AgdaBound{x}\AgdaSymbol{)} \AgdaSymbol{=} \AgdaFunction{weakening} \AgdaSymbol{(}\AgdaBound{σ∶Γ⇒Δ} \AgdaBound{x}\AgdaSymbol{)} \AgdaSymbol{(}\AgdaFunction{↑-typed} \AgdaSymbol{\{}\AgdaArgument{φ} \AgdaSymbol{=} \AgdaBound{φ}\AgdaSymbol{\})}\<%
\end{code}
}

\begin{code}%
\>\AgdaFunction{botSub-typed} \AgdaSymbol{:} \AgdaSymbol{∀} \AgdaSymbol{\{}\AgdaBound{P}\AgdaSymbol{\}} \AgdaSymbol{\{}\AgdaBound{Γ} \AgdaSymbol{:} \AgdaDatatype{Context} \AgdaBound{P}\AgdaSymbol{\}} \AgdaSymbol{\{}\AgdaBound{φ} \AgdaSymbol{:} \AgdaDatatype{Prop}\AgdaSymbol{\}} \AgdaSymbol{\{}\AgdaBound{δ}\AgdaSymbol{\}} \AgdaSymbol{→}\<%
\\
\>[0]\AgdaIndent{2}{}\<[2]%
\>[2]\AgdaBound{Γ} \AgdaDatatype{⊢} \AgdaBound{δ} \AgdaDatatype{∶} \AgdaBound{φ} \AgdaSymbol{→} \AgdaFunction{x₀:=} \AgdaBound{δ} \AgdaFunction{∶} \AgdaSymbol{(}\AgdaBound{Γ} \AgdaFunction{,P} \AgdaBound{φ}\AgdaSymbol{)} \AgdaFunction{⇒} \AgdaBound{Γ}\<%
\end{code}

\AgdaHide{
\begin{code}%
\>\AgdaFunction{botSub-typed} \AgdaSymbol{\{}\AgdaBound{P}\AgdaSymbol{\}} \AgdaSymbol{\{}\AgdaBound{Γ}\AgdaSymbol{\}} \AgdaSymbol{\{}\AgdaBound{φ}\AgdaSymbol{\}} \AgdaSymbol{\{}\AgdaBound{δ}\AgdaSymbol{\}} \AgdaBound{Γ⊢δ:φ} \AgdaBound{x} \AgdaSymbol{=} \<[39]%
\>[39]\<%
\\
\>[0]\AgdaIndent{2}{}\<[2]%
\>[2]\AgdaFunction{change-type} \AgdaSymbol{(}\AgdaFunction{sym} \AgdaSymbol{(}\AgdaFunction{unprp-rep} \AgdaSymbol{(}\AgdaFunction{pretypeof} \AgdaBound{x} \AgdaSymbol{(}\AgdaBound{Γ} \AgdaFunction{,P} \AgdaBound{φ}\AgdaSymbol{))} \AgdaFunction{upRep}\AgdaSymbol{))} \AgdaSymbol{(}\AgdaFunction{pre-botSub-typed} \AgdaBound{x}\AgdaSymbol{)} \AgdaKeyword{where}\<%
\\
\>[0]\AgdaIndent{2}{}\<[2]%
\>[2]\AgdaFunction{pre-botSub-typed} \AgdaSymbol{:} \AgdaSymbol{∀} \AgdaBound{x} \AgdaSymbol{→} \AgdaBound{Γ} \AgdaDatatype{⊢} \AgdaSymbol{(}\AgdaFunction{x₀:=} \AgdaBound{δ}\AgdaSymbol{)} \AgdaInductiveConstructor{-proof} \AgdaBound{x} \AgdaDatatype{∶} \AgdaFunction{unprp} \AgdaSymbol{(}\AgdaFunction{pretypeof} \AgdaBound{x} \AgdaSymbol{(}\AgdaBound{Γ} \AgdaFunction{,P} \AgdaBound{φ}\AgdaSymbol{))}\<%
\\
\>[0]\AgdaIndent{2}{}\<[2]%
\>[2]\AgdaFunction{pre-botSub-typed} \AgdaInductiveConstructor{x₀} \AgdaSymbol{=} \AgdaBound{Γ⊢δ:φ}\<%
\\
\>[0]\AgdaIndent{2}{}\<[2]%
\>[2]\AgdaFunction{pre-botSub-typed} \AgdaSymbol{(}\AgdaInductiveConstructor{↑} \AgdaBound{x}\AgdaSymbol{)} \AgdaSymbol{=} \AgdaInductiveConstructor{var} \AgdaBound{x}\<%
\end{code}
}

\begin{code}%
\>\AgdaFunction{substitution} \AgdaSymbol{:} \AgdaSymbol{∀} \AgdaSymbol{\{}\AgdaBound{P}\AgdaSymbol{\}} \AgdaSymbol{\{}\AgdaBound{Q}\AgdaSymbol{\}}\<%
\\
\>[0]\AgdaIndent{2}{}\<[2]%
\>[2]\AgdaSymbol{\{}\AgdaBound{Γ} \AgdaSymbol{:} \AgdaDatatype{Context} \AgdaBound{P}\AgdaSymbol{\}} \AgdaSymbol{\{}\AgdaBound{Δ} \AgdaSymbol{:} \AgdaDatatype{Context} \AgdaBound{Q}\AgdaSymbol{\}} \AgdaSymbol{\{}\AgdaBound{δ}\AgdaSymbol{\}} \AgdaSymbol{\{}\AgdaBound{φ}\AgdaSymbol{\}} \AgdaSymbol{\{}\AgdaBound{σ}\AgdaSymbol{\}} \AgdaSymbol{→} \<[48]%
\>[48]\<%
\\
\>[0]\AgdaIndent{2}{}\<[2]%
\>[2]\AgdaBound{Γ} \AgdaDatatype{⊢} \AgdaBound{δ} \AgdaDatatype{∶} \AgdaBound{φ} \AgdaSymbol{→} \AgdaBound{σ} \AgdaFunction{∶} \AgdaBound{Γ} \AgdaFunction{⇒} \AgdaBound{Δ} \AgdaSymbol{→} \AgdaBound{Δ} \AgdaDatatype{⊢} \AgdaBound{δ} \AgdaFunction{⟦} \AgdaBound{σ} \AgdaFunction{⟧} \AgdaDatatype{∶} \AgdaBound{φ}\<%
\end{code}

\AgdaHide{
\begin{code}%
\>\AgdaFunction{substitution} \AgdaSymbol{(}\AgdaInductiveConstructor{var} \AgdaSymbol{\_)} \AgdaBound{σ∶Γ→Δ} \AgdaSymbol{=} \AgdaBound{σ∶Γ→Δ} \AgdaSymbol{\_}\<%
\\
\>\AgdaFunction{substitution} \AgdaSymbol{(}\AgdaInductiveConstructor{app} \AgdaBound{Γ⊢δ∶φ→ψ} \AgdaBound{Γ⊢ε∶φ}\AgdaSymbol{)} \AgdaBound{σ∶Γ→Δ} \AgdaSymbol{=} \AgdaInductiveConstructor{app} \AgdaSymbol{(}\AgdaFunction{substitution} \AgdaBound{Γ⊢δ∶φ→ψ} \AgdaBound{σ∶Γ→Δ}\AgdaSymbol{)} \AgdaSymbol{(}\AgdaFunction{substitution} \AgdaBound{Γ⊢ε∶φ} \AgdaBound{σ∶Γ→Δ}\AgdaSymbol{)}\<%
\\
\>\AgdaFunction{substitution} \AgdaSymbol{\{}\AgdaArgument{Q} \AgdaSymbol{=} \AgdaBound{Q}\AgdaSymbol{\}} \AgdaSymbol{\{}\AgdaArgument{Δ} \AgdaSymbol{=} \AgdaBound{Δ}\AgdaSymbol{\}} \AgdaSymbol{\{}\AgdaArgument{σ} \AgdaSymbol{=} \AgdaBound{σ}\AgdaSymbol{\}} \AgdaSymbol{(}\AgdaInductiveConstructor{Λ} \AgdaSymbol{\{}\AgdaBound{P}\AgdaSymbol{\}} \AgdaSymbol{\{}\AgdaBound{Γ}\AgdaSymbol{\}} \AgdaSymbol{\{}\AgdaBound{φ}\AgdaSymbol{\}} \AgdaSymbol{\{}\AgdaBound{δ}\AgdaSymbol{\}} \AgdaSymbol{\{}\AgdaBound{ψ}\AgdaSymbol{\}} \AgdaBound{Γ,φ⊢δ∶ψ}\AgdaSymbol{)} \AgdaBound{σ∶Γ→Δ} \AgdaSymbol{=} \AgdaInductiveConstructor{Λ} \<[79]%
\>[79]\<%
\\
\>[0]\AgdaIndent{2}{}\<[2]%
\>[2]\AgdaSymbol{(}\AgdaFunction{substitution} \AgdaBound{Γ,φ⊢δ∶ψ} \AgdaSymbol{(}\AgdaFunction{liftSub-typed} \AgdaBound{σ∶Γ→Δ}\AgdaSymbol{))}\<%
\\
%
\\
\>\AgdaFunction{comp-typed} \AgdaSymbol{:} \AgdaSymbol{∀} \AgdaSymbol{\{}\AgdaBound{P}\AgdaSymbol{\}} \AgdaSymbol{\{}\AgdaBound{Q}\AgdaSymbol{\}} \AgdaSymbol{\{}\AgdaBound{R}\AgdaSymbol{\}}\<%
\\
\>[0]\AgdaIndent{2}{}\<[2]%
\>[2]\AgdaSymbol{\{}\AgdaBound{Γ} \AgdaSymbol{:} \AgdaDatatype{Context} \AgdaBound{P}\AgdaSymbol{\}} \AgdaSymbol{\{}\AgdaBound{Δ} \AgdaSymbol{:} \AgdaDatatype{Context} \AgdaBound{Q}\AgdaSymbol{\}} \AgdaSymbol{\{}\AgdaBound{Θ} \AgdaSymbol{:} \AgdaDatatype{Context} \AgdaBound{R}\AgdaSymbol{\}}\<%
\\
\>[0]\AgdaIndent{2}{}\<[2]%
\>[2]\AgdaSymbol{\{}\AgdaBound{τ}\AgdaSymbol{\}} \AgdaSymbol{\{}\AgdaBound{σ}\AgdaSymbol{\}} \AgdaSymbol{→} \AgdaBound{τ} \AgdaFunction{∶} \AgdaBound{Δ} \AgdaFunction{⇒} \AgdaBound{Θ} \AgdaSymbol{→} \AgdaBound{σ} \AgdaFunction{∶} \AgdaBound{Γ} \AgdaFunction{⇒} \AgdaBound{Δ} \AgdaSymbol{→} \AgdaBound{τ} \AgdaFunction{•} \AgdaBound{σ} \AgdaFunction{∶} \AgdaBound{Γ} \AgdaFunction{⇒} \AgdaBound{Θ}\<%
\\
\>\AgdaFunction{comp-typed} \AgdaBound{τ∶Δ⇒Θ} \AgdaBound{σ∶Γ⇒Δ} \AgdaBound{x} \AgdaSymbol{=} \AgdaFunction{substitution} \AgdaSymbol{(}\AgdaBound{σ∶Γ⇒Δ} \AgdaBound{x}\AgdaSymbol{)} \AgdaBound{τ∶Δ⇒Θ}\<%
\end{code}
}

\begin{lemma}[Subject Reduction]
If $\Gamma \vdash \delta : \phi$ and $\delta \rightarrow_\beta \epsilon$ then $\Gamma \vdash \epsilon : \phi$.
\end{lemma}

\begin{code}%
\>\AgdaFunction{subject-reduction} \AgdaSymbol{:} \AgdaSymbol{∀} \AgdaSymbol{\{}\AgdaBound{P}\AgdaSymbol{\}} \AgdaSymbol{\{}\AgdaBound{Γ} \AgdaSymbol{:} \AgdaDatatype{Context} \AgdaBound{P}\AgdaSymbol{\}} \AgdaSymbol{\{}\AgdaBound{δ} \AgdaBound{ε} \AgdaSymbol{:} \AgdaFunction{Proof} \AgdaSymbol{(} \AgdaBound{P}\AgdaSymbol{)\}} \AgdaSymbol{\{}\AgdaBound{φ}\AgdaSymbol{\}} \AgdaSymbol{→} \<[67]%
\>[67]\<%
\\
\>[0]\AgdaIndent{2}{}\<[2]%
\>[2]\AgdaBound{Γ} \AgdaDatatype{⊢} \AgdaBound{δ} \AgdaDatatype{∶} \AgdaBound{φ} \AgdaSymbol{→} \AgdaBound{δ} \AgdaDatatype{⇒} \AgdaBound{ε} \AgdaSymbol{→} \AgdaBound{Γ} \AgdaDatatype{⊢} \AgdaBound{ε} \AgdaDatatype{∶} \AgdaBound{φ}\<%
\end{code}

\AgdaHide{
\begin{code}%
\>\AgdaFunction{subject-reduction} \AgdaSymbol{(}\AgdaInductiveConstructor{var} \AgdaSymbol{\_)} \AgdaSymbol{()}\<%
\\
\>\AgdaFunction{subject-reduction} \AgdaSymbol{(}\AgdaInductiveConstructor{app} \AgdaSymbol{\{}\AgdaArgument{ε} \AgdaSymbol{=} \AgdaBound{ε}\AgdaSymbol{\}} \AgdaSymbol{(}\AgdaInductiveConstructor{Λ} \AgdaSymbol{\{}\AgdaBound{P}\AgdaSymbol{\}} \AgdaSymbol{\{}\AgdaBound{Γ}\AgdaSymbol{\}} \AgdaSymbol{\{}\AgdaBound{φ}\AgdaSymbol{\}} \AgdaSymbol{\{}\AgdaBound{δ}\AgdaSymbol{\}} \AgdaSymbol{\{}\AgdaBound{ψ}\AgdaSymbol{\}} \AgdaBound{Γ,φ⊢δ∶ψ}\AgdaSymbol{)} \AgdaBound{Γ⊢ε∶φ}\AgdaSymbol{)} \AgdaSymbol{(}\AgdaInductiveConstructor{redex} \AgdaInductiveConstructor{βI}\AgdaSymbol{)} \AgdaSymbol{=} \<[83]%
\>[83]\<%
\\
\>[0]\AgdaIndent{2}{}\<[2]%
\>[2]\AgdaFunction{substitution} \AgdaBound{Γ,φ⊢δ∶ψ} \AgdaSymbol{(}\AgdaFunction{botSub-typed} \AgdaBound{Γ⊢ε∶φ}\AgdaSymbol{)}\<%
\\
\>\AgdaFunction{subject-reduction} \AgdaSymbol{(}\AgdaInductiveConstructor{app} \AgdaBound{Γ⊢δ∶φ→ψ} \AgdaBound{Γ⊢ε∶φ}\AgdaSymbol{)} \AgdaSymbol{(}\AgdaInductiveConstructor{app} \AgdaSymbol{(}\AgdaInductiveConstructor{appl} \AgdaBound{δ→δ'}\AgdaSymbol{))} \AgdaSymbol{=} \AgdaInductiveConstructor{app} \AgdaSymbol{(}\AgdaFunction{subject-reduction} \AgdaBound{Γ⊢δ∶φ→ψ} \AgdaBound{δ→δ'}\AgdaSymbol{)} \AgdaBound{Γ⊢ε∶φ}\<%
\\
\>\AgdaFunction{subject-reduction} \AgdaSymbol{(}\AgdaInductiveConstructor{app} \AgdaBound{Γ⊢δ∶φ→ψ} \AgdaBound{Γ⊢ε∶φ}\AgdaSymbol{)} \AgdaSymbol{(}\AgdaInductiveConstructor{app} \AgdaSymbol{(}\AgdaInductiveConstructor{appr} \AgdaSymbol{(}\AgdaInductiveConstructor{appl} \AgdaBound{ε→ε'}\AgdaSymbol{)))} \AgdaSymbol{=} \AgdaInductiveConstructor{app} \AgdaBound{Γ⊢δ∶φ→ψ} \AgdaSymbol{(}\AgdaFunction{subject-reduction} \AgdaBound{Γ⊢ε∶φ} \AgdaBound{ε→ε'}\AgdaSymbol{)}\<%
\\
\>\AgdaFunction{subject-reduction} \AgdaSymbol{(}\AgdaInductiveConstructor{app} \AgdaBound{Γ⊢δ∶φ→ψ} \AgdaBound{Γ⊢ε∶φ}\AgdaSymbol{)} \AgdaSymbol{(}\AgdaInductiveConstructor{app} \AgdaSymbol{(}\AgdaInductiveConstructor{appr} \AgdaSymbol{(}\AgdaInductiveConstructor{appr} \AgdaSymbol{())))}\<%
\\
\>\AgdaFunction{subject-reduction} \AgdaSymbol{(}\AgdaInductiveConstructor{Λ} \AgdaSymbol{\_)} \AgdaSymbol{(}\AgdaInductiveConstructor{redex} \AgdaSymbol{())}\<%
\\
\>\AgdaFunction{subject-reduction} \AgdaSymbol{(}\AgdaInductiveConstructor{Λ} \AgdaBound{Γ,φ⊢δ∶ψ}\AgdaSymbol{)} \AgdaSymbol{(}\AgdaInductiveConstructor{app} \AgdaSymbol{(}\AgdaInductiveConstructor{appl} \AgdaBound{δ⇒ε}\AgdaSymbol{))} \AgdaSymbol{=} \AgdaInductiveConstructor{Λ} \AgdaSymbol{(}\AgdaFunction{subject-reduction} \AgdaBound{Γ,φ⊢δ∶ψ} \AgdaBound{δ⇒ε}\AgdaSymbol{)}\<%
\\
\>\AgdaFunction{subject-reduction} \AgdaSymbol{(}\AgdaInductiveConstructor{Λ} \AgdaBound{Γ⊢δ∶φ}\AgdaSymbol{)} \AgdaSymbol{(}\AgdaInductiveConstructor{app} \AgdaSymbol{(}\AgdaInductiveConstructor{appr} \AgdaSymbol{()))}\<%
\end{code}
}


\AgdaHide{
\begin{code}%
\>\AgdaKeyword{module} \AgdaModule{PL.Computable} \AgdaKeyword{where}\<%
\\
\>\AgdaKeyword{open} \AgdaKeyword{import} \AgdaModule{Data.Empty}\<%
\\
\>\AgdaKeyword{open} \AgdaKeyword{import} \AgdaModule{Data.Product} \AgdaKeyword{renaming} \AgdaSymbol{(}\AgdaInductiveConstructor{\_,\_} \AgdaSymbol{to} \AgdaInductiveConstructor{\_,p\_}\AgdaSymbol{)}\<%
\\
\>\AgdaKeyword{open} \AgdaKeyword{import} \AgdaModule{Prelims}\<%
\\
\>\AgdaKeyword{open} \AgdaKeyword{import} \AgdaModule{PL.Grammar}\<%
\\
\>\AgdaKeyword{open} \AgdaModule{PLgrammar}\<%
\\
\>\AgdaKeyword{open} \AgdaKeyword{import} \AgdaModule{Grammar} \AgdaFunction{Propositional-Logic}\<%
\\
\>\AgdaKeyword{open} \AgdaKeyword{import} \AgdaModule{Reduction} \AgdaFunction{Propositional-Logic} \AgdaDatatype{β}\<%
\\
\>\AgdaKeyword{open} \AgdaKeyword{import} \AgdaModule{PL.Rules}\<%
\end{code}
}

\subsubsection{Computable Terms}

We define the sets of \emph{computable} proofs $C_\Gamma(\phi)$ for each context $\Gamma$ and proposition $\phi$ as follows:

\begin{align*}
C_\Gamma(\bot) & = \{ \delta \mid \Gamma \vdash \delta : \bot \text{ and } \delta \in SN \} \\
C_\Gamma(\phi \rightarrow \psi) & = \{ \delta \mid \Gamma : \delta : \phi \rightarrow \psi \text{ and } \forall \Delta ⊇ \Gamma . ∀ \epsilon \in C_\Delta(\phi). \delta \epsilon \in C_\Delta(\psi) \}
\end{align*}

\begin{code}%
\>\AgdaFunction{C} \AgdaSymbol{:} \AgdaSymbol{∀} \AgdaSymbol{\{}\AgdaBound{P}\AgdaSymbol{\}} \AgdaSymbol{→} \AgdaDatatype{Context} \AgdaBound{P} \AgdaSymbol{→} \AgdaDatatype{Prop} \AgdaSymbol{→} \AgdaFunction{Proof} \AgdaBound{P} \AgdaSymbol{→} \AgdaPrimitiveType{Set}\<%
\\
\>\AgdaFunction{C} \AgdaBound{Γ} \AgdaInductiveConstructor{⊥P} \AgdaBound{δ} \AgdaSymbol{=} \AgdaBound{Γ} \AgdaDatatype{⊢} \AgdaBound{δ} \AgdaDatatype{∶} \AgdaInductiveConstructor{⊥P} \AgdaFunction{×} \AgdaDatatype{SN} \AgdaBound{δ}\<%
\\
\>\AgdaFunction{C} \AgdaBound{Γ} \AgdaSymbol{(}\AgdaBound{φ} \AgdaInductiveConstructor{⇛} \AgdaBound{ψ}\AgdaSymbol{)} \AgdaBound{δ} \AgdaSymbol{=} \AgdaBound{Γ} \AgdaDatatype{⊢} \AgdaBound{δ} \AgdaDatatype{∶} \AgdaBound{φ} \AgdaInductiveConstructor{⇛} \AgdaBound{ψ} \AgdaFunction{×} \<[32]%
\>[32]\<%
\\
\>[0]\AgdaIndent{2}{}\<[2]%
\>[2]\AgdaSymbol{(∀} \AgdaBound{Q} \AgdaSymbol{\{}\AgdaBound{Δ} \AgdaSymbol{:} \AgdaDatatype{Context} \AgdaBound{Q}\AgdaSymbol{\}} \AgdaBound{ρ} \AgdaBound{ε} \AgdaSymbol{→} \AgdaBound{ρ} \AgdaFunction{∶} \AgdaBound{Γ} \AgdaFunction{⇒R} \AgdaBound{Δ} \AgdaSymbol{→} \<[42]%
\>[42]\<%
\\
\>[2]\AgdaIndent{4}{}\<[4]%
\>[4]\AgdaFunction{C} \AgdaBound{Δ} \AgdaBound{φ} \AgdaBound{ε} \AgdaSymbol{→} \AgdaFunction{C} \AgdaBound{Δ} \AgdaBound{ψ} \AgdaSymbol{(}\AgdaFunction{appP} \AgdaSymbol{(}\AgdaBound{δ} \AgdaFunction{〈} \AgdaBound{ρ} \AgdaFunction{〉}\AgdaSymbol{)} \AgdaBound{ε}\AgdaSymbol{))}\<%
\end{code}

\begin{lemma}$ $
\begin{enumerate}
\item
If $\delta \in C_\Gamma(\phi)$ then $\Gamma \vdash \delta : \phi$.
\item
If $\delta \in C_\Gamma(\phi)$ and $\rho : \Gamma \rightarrow \Delta$ then $\delta \langle \rho \rangle \in C_\Delta(\phi)$.
\item
If $\delta \in C_\Gamma(\phi)$ and $\delta \rightarrow_\beta \epsilon$ then $\epsilon \in C_\Gamma(\phi)$.
\item
Let $\Gamma \vdash \delta : \phi$.  
If $\delta$ is neutral and, for all $\epsilon$ such that $\delta \rightarrow_\beta \epsilon$, we have $\epsilon \in C_\Gamma(\phi)$, then $\delta \in C_\Gamma(\phi)$.
\item
If $\delta \in C_\Gamma(\phi)$ then $\delta$ is strongly normalizable.
\end{enumerate}
\end{lemma}

\begin{code}%
\>\AgdaFunction{C-typed} \AgdaSymbol{:} \AgdaSymbol{∀} \AgdaSymbol{\{}\AgdaBound{P}\AgdaSymbol{\}} \AgdaSymbol{\{}\AgdaBound{Γ} \AgdaSymbol{:} \AgdaDatatype{Context} \AgdaBound{P}\AgdaSymbol{\}} \AgdaSymbol{\{}\AgdaBound{φ}\AgdaSymbol{\}} \AgdaSymbol{\{}\AgdaBound{δ}\AgdaSymbol{\}} \AgdaSymbol{→} \AgdaFunction{C} \AgdaBound{Γ} \AgdaBound{φ} \AgdaBound{δ} \AgdaSymbol{→} \AgdaBound{Γ} \AgdaDatatype{⊢} \AgdaBound{δ} \AgdaDatatype{∶} \AgdaBound{φ}\<%
\end{code}

\AgdaHide{
\begin{code}%
\>\AgdaFunction{C-typed} \AgdaSymbol{\{}\AgdaArgument{φ} \AgdaSymbol{=} \AgdaInductiveConstructor{⊥P}\AgdaSymbol{\}} \AgdaSymbol{=} \AgdaField{proj₁}\<%
\\
\>\AgdaFunction{C-typed} \AgdaSymbol{\{}\AgdaArgument{φ} \AgdaSymbol{=} \AgdaSymbol{\_} \AgdaInductiveConstructor{⇛} \AgdaSymbol{\_\}} \AgdaSymbol{=} \AgdaField{proj₁}\<%
\end{code}
}

\begin{code}%
\>\AgdaFunction{C-rep} \AgdaSymbol{:} \AgdaSymbol{∀} \AgdaSymbol{\{}\AgdaBound{P}\AgdaSymbol{\}} \AgdaSymbol{\{}\AgdaBound{Q}\AgdaSymbol{\}} \AgdaSymbol{\{}\AgdaBound{Γ} \AgdaSymbol{:} \AgdaDatatype{Context} \AgdaBound{P}\AgdaSymbol{\}} \AgdaSymbol{\{}\AgdaBound{Δ} \AgdaSymbol{:} \AgdaDatatype{Context} \AgdaBound{Q}\AgdaSymbol{\}} \AgdaBound{φ} \AgdaSymbol{\{}\AgdaBound{δ}\AgdaSymbol{\}} \AgdaSymbol{\{}\AgdaBound{ρ}\AgdaSymbol{\}} \AgdaSymbol{→} \<[62]%
\>[62]\<%
\\
\>[0]\AgdaIndent{2}{}\<[2]%
\>[2]\AgdaFunction{C} \AgdaBound{Γ} \AgdaBound{φ} \AgdaBound{δ} \AgdaSymbol{→} \AgdaBound{ρ} \AgdaFunction{∶} \AgdaBound{Γ} \AgdaFunction{⇒R} \AgdaBound{Δ} \AgdaSymbol{→} \AgdaFunction{C} \AgdaBound{Δ} \AgdaBound{φ} \AgdaSymbol{(}\AgdaBound{δ} \AgdaFunction{〈} \AgdaBound{ρ} \AgdaFunction{〉}\AgdaSymbol{)}\<%
\end{code}

\AgdaHide{
\begin{code}%
\>\AgdaFunction{C-rep} \AgdaInductiveConstructor{⊥P} \AgdaSymbol{(}\AgdaBound{Γ⊢δ∶x₀} \AgdaInductiveConstructor{,p} \AgdaBound{SNδ}\AgdaSymbol{)} \AgdaBound{ρ∶Γ→Δ} \AgdaSymbol{=} \AgdaSymbol{(}\AgdaFunction{weakening} \AgdaBound{Γ⊢δ∶x₀} \AgdaBound{ρ∶Γ→Δ}\AgdaSymbol{)} \AgdaInductiveConstructor{,p} \AgdaFunction{SNrep} \AgdaFunction{β-creates-rep} \AgdaBound{SNδ}\<%
\\
\>\AgdaFunction{C-rep} \AgdaSymbol{\{}\AgdaBound{P}\AgdaSymbol{\}} \AgdaSymbol{\{}\AgdaBound{Q}\AgdaSymbol{\}} \AgdaSymbol{\{}\AgdaBound{Γ}\AgdaSymbol{\}} \AgdaSymbol{\{}\AgdaBound{Δ}\AgdaSymbol{\}} \AgdaSymbol{(}\AgdaBound{φ} \AgdaInductiveConstructor{⇛} \AgdaBound{ψ}\AgdaSymbol{)} \AgdaSymbol{\{}\AgdaBound{δ}\AgdaSymbol{\}} \AgdaSymbol{\{}\AgdaBound{ρ}\AgdaSymbol{\}} \AgdaSymbol{(}\AgdaBound{Γ⊢δ∶φ⇒ψ} \AgdaInductiveConstructor{,p} \AgdaBound{Cδ}\AgdaSymbol{)} \AgdaBound{ρ∶Γ→Δ} \AgdaSymbol{=} \<[62]%
\>[62]\<%
\\
\>[0]\AgdaIndent{2}{}\<[2]%
\>[2]\AgdaSymbol{(}\AgdaFunction{weakening} \AgdaBound{Γ⊢δ∶φ⇒ψ} \AgdaBound{ρ∶Γ→Δ}\AgdaSymbol{)} \AgdaInductiveConstructor{,p} \AgdaSymbol{(λ} \AgdaBound{R} \AgdaSymbol{\{}\AgdaBound{Θ}\AgdaSymbol{\}} \AgdaBound{σ} \AgdaBound{ε} \AgdaBound{σ∶Δ→Θ} \AgdaBound{ε∈CΘ} \AgdaSymbol{→} \AgdaFunction{subst} \AgdaSymbol{(}\AgdaFunction{C} \AgdaBound{Θ} \AgdaBound{ψ}\AgdaSymbol{)} \<[71]%
\>[71]\<%
\\
\>[2]\AgdaIndent{4}{}\<[4]%
\>[4]\AgdaSymbol{(}\AgdaFunction{cong} \AgdaSymbol{(λ} \AgdaBound{x} \AgdaSymbol{→} \AgdaFunction{appP} \AgdaBound{x} \AgdaBound{ε}\AgdaSymbol{)} \AgdaSymbol{(}\AgdaFunction{rep-comp} \AgdaBound{δ}\AgdaSymbol{))}\<%
\\
\>[2]\AgdaIndent{4}{}\<[4]%
\>[4]\AgdaSymbol{(}\AgdaBound{Cδ} \AgdaBound{R} \AgdaSymbol{(}\AgdaBound{σ} \AgdaFunction{•R} \AgdaBound{ρ}\AgdaSymbol{)} \AgdaBound{ε} \AgdaSymbol{(}\AgdaFunction{•R-typed} \AgdaSymbol{\{}\AgdaArgument{σ} \AgdaSymbol{=} \AgdaBound{σ}\AgdaSymbol{\}} \AgdaSymbol{\{}\AgdaArgument{ρ} \AgdaSymbol{=} \AgdaBound{ρ}\AgdaSymbol{\}} \AgdaBound{ρ∶Γ→Δ} \AgdaBound{σ∶Δ→Θ}\AgdaSymbol{)} \AgdaBound{ε∈CΘ}\AgdaSymbol{))}\<%
\end{code}
}

\begin{code}%
\>\AgdaFunction{C-osr} \AgdaSymbol{:} \AgdaSymbol{∀} \AgdaSymbol{\{}\AgdaBound{P}\AgdaSymbol{\}} \AgdaSymbol{\{}\AgdaBound{Γ} \AgdaSymbol{:} \AgdaDatatype{Context} \AgdaBound{P}\AgdaSymbol{\}} \AgdaBound{φ} \AgdaSymbol{\{}\AgdaBound{δ}\AgdaSymbol{\}} \AgdaSymbol{\{}\AgdaBound{ε}\AgdaSymbol{\}} \AgdaSymbol{→} \AgdaFunction{C} \AgdaBound{Γ} \AgdaBound{φ} \AgdaBound{δ} \AgdaSymbol{→} \AgdaBound{δ} \AgdaDatatype{⇒} \AgdaBound{ε} \AgdaSymbol{→} \AgdaFunction{C} \AgdaBound{Γ} \AgdaBound{φ} \AgdaBound{ε}\<%
\end{code}

\AgdaHide{
\begin{code}%
\>\AgdaFunction{C-osr} \AgdaSymbol{(}\AgdaInductiveConstructor{⊥P}\AgdaSymbol{)} \AgdaSymbol{(}\AgdaBound{Γ⊢δ∶x₀} \AgdaInductiveConstructor{,p} \AgdaBound{SNδ}\AgdaSymbol{)} \AgdaBound{δ→ε} \AgdaSymbol{=} \AgdaSymbol{(}\AgdaFunction{subject-reduction} \AgdaBound{Γ⊢δ∶x₀} \AgdaBound{δ→ε}\AgdaSymbol{)} \AgdaInductiveConstructor{,p} \AgdaSymbol{(}\AgdaFunction{SNred} \AgdaBound{SNδ} \AgdaSymbol{(}\AgdaInductiveConstructor{osr-red} \AgdaBound{δ→ε}\AgdaSymbol{))}\<%
\\
\>\AgdaFunction{C-osr} \AgdaSymbol{\{}\AgdaArgument{Γ} \AgdaSymbol{=} \AgdaBound{Γ}\AgdaSymbol{\}} \AgdaSymbol{(}\AgdaBound{φ} \AgdaInductiveConstructor{⇛} \AgdaBound{ψ}\AgdaSymbol{)} \AgdaSymbol{\{}\AgdaArgument{δ} \AgdaSymbol{=} \AgdaBound{δ}\AgdaSymbol{\}} \AgdaSymbol{(}\AgdaBound{Γ⊢δ∶φ⇒ψ} \AgdaInductiveConstructor{,p} \AgdaBound{Cδ}\AgdaSymbol{)} \AgdaBound{δ→δ'} \AgdaSymbol{=} \AgdaFunction{subject-reduction} \AgdaBound{Γ⊢δ∶φ⇒ψ} \AgdaBound{δ→δ'} \AgdaInductiveConstructor{,p} \<[87]%
\>[87]\<%
\\
\>[0]\AgdaIndent{2}{}\<[2]%
\>[2]\AgdaSymbol{(λ} \AgdaBound{Q} \AgdaBound{ρ} \AgdaBound{ε} \AgdaBound{ρ∶Γ→Δ} \AgdaBound{ε∈Cφ} \AgdaSymbol{→} \AgdaFunction{C-osr} \AgdaBound{ψ} \AgdaSymbol{(}\AgdaBound{Cδ} \AgdaBound{Q} \AgdaBound{ρ} \AgdaBound{ε} \AgdaBound{ρ∶Γ→Δ} \AgdaBound{ε∈Cφ}\AgdaSymbol{)} \AgdaSymbol{(}\AgdaInductiveConstructor{app} \AgdaSymbol{(}\AgdaInductiveConstructor{appl} \AgdaSymbol{(}\AgdaFunction{respects-osr} \AgdaFunction{REP} \AgdaFunction{β-respects-rep} \AgdaBound{δ→δ'}\AgdaSymbol{))))}\<%
\\
%
\\
\>\AgdaFunction{C-red} \AgdaSymbol{:} \AgdaSymbol{∀} \AgdaSymbol{\{}\AgdaBound{P}\AgdaSymbol{\}} \AgdaSymbol{\{}\AgdaBound{Γ} \AgdaSymbol{:} \AgdaDatatype{Context} \AgdaBound{P}\AgdaSymbol{\}} \AgdaBound{φ} \AgdaSymbol{\{}\AgdaBound{δ}\AgdaSymbol{\}} \AgdaSymbol{\{}\AgdaBound{ε}\AgdaSymbol{\}} \AgdaSymbol{→} \AgdaFunction{C} \AgdaBound{Γ} \AgdaBound{φ} \AgdaBound{δ} \AgdaSymbol{→} \AgdaBound{δ} \AgdaDatatype{↠} \AgdaBound{ε} \AgdaSymbol{→} \AgdaFunction{C} \AgdaBound{Γ} \AgdaBound{φ} \AgdaBound{ε}\<%
\\
\>\AgdaFunction{C-red} \AgdaBound{φ} \AgdaBound{δ∈CΓφ} \AgdaSymbol{(}\AgdaInductiveConstructor{osr-red} \AgdaBound{δ⇒ε}\AgdaSymbol{)} \AgdaSymbol{=} \AgdaFunction{C-osr} \AgdaBound{φ} \AgdaBound{δ∈CΓφ} \AgdaBound{δ⇒ε}\<%
\\
\>\AgdaFunction{C-red} \AgdaSymbol{\_} \AgdaBound{δ∈CΓφ} \AgdaInductiveConstructor{ref} \AgdaSymbol{=} \AgdaBound{δ∈CΓφ}\<%
\\
\>\AgdaFunction{C-red} \AgdaBound{φ} \AgdaBound{δ∈CΓφ} \AgdaSymbol{(}\AgdaInductiveConstructor{trans-red} \AgdaBound{δ↠ε} \AgdaBound{ε↠χ}\AgdaSymbol{)} \AgdaSymbol{=} \AgdaFunction{C-red} \AgdaBound{φ} \AgdaSymbol{(}\AgdaFunction{C-red} \AgdaBound{φ} \AgdaBound{δ∈CΓφ} \AgdaBound{δ↠ε}\AgdaSymbol{)} \AgdaBound{ε↠χ}\<%
\end{code}
}

\begin{code}%
\>\AgdaFunction{NeutralC} \AgdaSymbol{:} \AgdaSymbol{∀} \AgdaSymbol{\{}\AgdaBound{P}\AgdaSymbol{\}} \AgdaSymbol{\{}\AgdaBound{Γ} \AgdaSymbol{:} \AgdaDatatype{Context} \AgdaBound{P}\AgdaSymbol{\}} \AgdaSymbol{\{}\AgdaBound{δ} \AgdaSymbol{:} \AgdaFunction{Proof} \AgdaSymbol{(} \AgdaBound{P}\AgdaSymbol{)\}} \AgdaSymbol{(}\AgdaBound{φ} \AgdaSymbol{:} \AgdaDatatype{Prop}\AgdaSymbol{)} \AgdaSymbol{→}\<%
\\
\>[0]\AgdaIndent{2}{}\<[2]%
\>[2]\AgdaBound{Γ} \AgdaDatatype{⊢} \AgdaBound{δ} \AgdaDatatype{∶} \AgdaBound{φ} \AgdaSymbol{→} \AgdaDatatype{Neutral} \AgdaBound{δ} \AgdaSymbol{→}\<%
\\
\>[0]\AgdaIndent{2}{}\<[2]%
\>[2]\AgdaSymbol{(∀} \AgdaBound{ε} \AgdaSymbol{→} \AgdaBound{δ} \AgdaDatatype{⇒} \AgdaBound{ε} \AgdaSymbol{→} \AgdaFunction{C} \AgdaBound{Γ} \AgdaBound{φ} \AgdaBound{ε}\AgdaSymbol{)} \AgdaSymbol{→}\<%
\\
\>[0]\AgdaIndent{2}{}\<[2]%
\>[2]\AgdaFunction{C} \AgdaBound{Γ} \AgdaBound{φ} \AgdaBound{δ}\<%
\\
%
\\
\>\AgdaFunction{CsubSN} \AgdaSymbol{:} \AgdaSymbol{∀} \AgdaSymbol{\{}\AgdaBound{P}\AgdaSymbol{\}} \AgdaSymbol{\{}\AgdaBound{Γ} \AgdaSymbol{:} \AgdaDatatype{Context} \AgdaBound{P}\AgdaSymbol{\}} \AgdaBound{φ} \AgdaSymbol{\{}\AgdaBound{δ}\AgdaSymbol{\}} \AgdaSymbol{→} \AgdaFunction{C} \AgdaBound{Γ} \AgdaBound{φ} \AgdaBound{δ} \AgdaSymbol{→} \AgdaDatatype{SN} \AgdaBound{δ}\<%
\end{code}

\AgdaHide{
\begin{code}%
\>\AgdaFunction{NeutralC} \AgdaSymbol{\{}\AgdaBound{P}\AgdaSymbol{\}} \AgdaSymbol{\{}\AgdaBound{Γ}\AgdaSymbol{\}} \AgdaSymbol{\{}\AgdaBound{δ}\AgdaSymbol{\}} \AgdaSymbol{\{}\AgdaInductiveConstructor{⊥P}\AgdaSymbol{\}} \AgdaBound{Γ⊢δ∶x₀} \AgdaBound{Neutralδ} \AgdaBound{hyp} \AgdaSymbol{=} \AgdaBound{Γ⊢δ∶x₀} \AgdaInductiveConstructor{,p} \AgdaInductiveConstructor{SNI} \AgdaBound{δ} \AgdaSymbol{(λ} \AgdaBound{ε} \AgdaBound{δ→ε} \AgdaSymbol{→} \AgdaField{proj₂} \AgdaSymbol{(}\AgdaBound{hyp} \AgdaBound{ε} \AgdaBound{δ→ε}\AgdaSymbol{))}\<%
\\
\>\AgdaFunction{NeutralC} \AgdaSymbol{\{}\AgdaBound{P}\AgdaSymbol{\}} \AgdaSymbol{\{}\AgdaBound{Γ}\AgdaSymbol{\}} \AgdaSymbol{\{}\AgdaBound{δ}\AgdaSymbol{\}} \AgdaSymbol{\{}\AgdaBound{φ} \AgdaInductiveConstructor{⇛} \AgdaBound{ψ}\AgdaSymbol{\}} \AgdaBound{Γ⊢δ∶φ→ψ} \AgdaBound{neutralδ} \AgdaBound{hyp} \AgdaSymbol{=} \AgdaBound{Γ⊢δ∶φ→ψ} \AgdaInductiveConstructor{,p} \<[63]%
\>[63]\<%
\\
\>[0]\AgdaIndent{2}{}\<[2]%
\>[2]\AgdaSymbol{(λ} \AgdaBound{Q} \AgdaBound{ρ} \AgdaBound{ε} \AgdaBound{ρ∶Γ→Δ} \AgdaBound{ε∈Cφ} \AgdaSymbol{→} \AgdaFunction{claim} \AgdaBound{ε} \AgdaSymbol{(}\AgdaFunction{CsubSN} \AgdaBound{φ} \AgdaSymbol{\{}\AgdaArgument{δ} \AgdaSymbol{=} \AgdaBound{ε}\AgdaSymbol{\}} \AgdaBound{ε∈Cφ}\AgdaSymbol{)} \AgdaBound{ρ∶Γ→Δ} \AgdaBound{ε∈Cφ}\AgdaSymbol{)} \AgdaKeyword{where}\<%
\\
\>[0]\AgdaIndent{2}{}\<[2]%
\>[2]\AgdaFunction{claim} \AgdaSymbol{:} \AgdaSymbol{∀} \AgdaSymbol{\{}\AgdaBound{Q}\AgdaSymbol{\}} \AgdaSymbol{\{}\AgdaBound{Δ}\AgdaSymbol{\}} \AgdaSymbol{\{}\AgdaBound{ρ} \AgdaSymbol{:} \AgdaFunction{Rep} \AgdaBound{P} \AgdaBound{Q}\AgdaSymbol{\}} \AgdaBound{ε} \AgdaSymbol{→} \AgdaDatatype{SN} \AgdaBound{ε} \AgdaSymbol{→} \AgdaBound{ρ} \AgdaFunction{∶} \AgdaBound{Γ} \AgdaFunction{⇒R} \AgdaBound{Δ} \AgdaSymbol{→} \AgdaFunction{C} \AgdaBound{Δ} \AgdaBound{φ} \AgdaBound{ε} \AgdaSymbol{→} \AgdaFunction{C} \AgdaBound{Δ} \AgdaBound{ψ} \AgdaSymbol{(}\AgdaFunction{appP} \AgdaSymbol{(}\AgdaBound{δ} \AgdaFunction{〈} \<[86]%
\>[86]\AgdaBound{ρ} \AgdaFunction{〉}\AgdaSymbol{)} \AgdaBound{ε}\AgdaSymbol{)}\<%
\\
\>[0]\AgdaIndent{2}{}\<[2]%
\>[2]\AgdaFunction{claim} \AgdaSymbol{\{}\AgdaBound{Q}\AgdaSymbol{\}} \AgdaSymbol{\{}\AgdaBound{Δ}\AgdaSymbol{\}} \AgdaSymbol{\{}\AgdaBound{ρ}\AgdaSymbol{\}} \AgdaBound{ε} \AgdaSymbol{(}\AgdaInductiveConstructor{SNI} \AgdaSymbol{.}\AgdaBound{ε} \AgdaBound{SNε}\AgdaSymbol{)} \AgdaBound{ρ∶Γ→Δ} \AgdaBound{ε∈Cφ} \AgdaSymbol{=} \AgdaFunction{NeutralC} \AgdaSymbol{\{}\AgdaBound{Q}\AgdaSymbol{\}} \AgdaSymbol{\{}\AgdaBound{Δ}\AgdaSymbol{\}} \AgdaSymbol{\{}\AgdaFunction{appP} \AgdaSymbol{(}\AgdaBound{δ} \AgdaFunction{〈} \<[77]%
\>[77]\AgdaBound{ρ} \AgdaFunction{〉}\AgdaSymbol{)} \AgdaBound{ε}\AgdaSymbol{\}} \AgdaBound{ψ}\<%
\\
\>[2]\AgdaIndent{6}{}\<[6]%
\>[6]\AgdaSymbol{(}\AgdaInductiveConstructor{app} \<[11]%
\>[11]\<%
\\
\>[2]\AgdaIndent{6}{}\<[6]%
\>[6]\AgdaSymbol{(}\AgdaFunction{weakening} \AgdaBound{Γ⊢δ∶φ→ψ} \AgdaBound{ρ∶Γ→Δ}\AgdaSymbol{)}\<%
\\
\>[2]\AgdaIndent{6}{}\<[6]%
\>[6]\AgdaSymbol{(}\AgdaFunction{C-typed} \AgdaSymbol{\{}\AgdaBound{Q}\AgdaSymbol{\}} \AgdaSymbol{\{}\AgdaBound{Δ}\AgdaSymbol{\}} \AgdaSymbol{\{}\AgdaBound{φ}\AgdaSymbol{\}} \AgdaSymbol{\{}\AgdaBound{ε}\AgdaSymbol{\}} \AgdaBound{ε∈Cφ}\AgdaSymbol{))} \<[38]%
\>[38]\<%
\\
\>[2]\AgdaIndent{6}{}\<[6]%
\>[6]\AgdaSymbol{(}\AgdaInductiveConstructor{appNeutral} \AgdaSymbol{(}\AgdaBound{δ} \AgdaFunction{〈} \<[24]%
\>[24]\AgdaBound{ρ} \AgdaFunction{〉}\AgdaSymbol{)} \AgdaBound{ε}\AgdaSymbol{)}\<%
\\
\>[2]\AgdaIndent{6}{}\<[6]%
\>[6]\AgdaSymbol{(}\AgdaFunction{NeutralC-lm} \AgdaSymbol{\{}\AgdaArgument{X} \AgdaSymbol{=} \AgdaFunction{C} \AgdaBound{Δ} \AgdaBound{ψ}\AgdaSymbol{\}} \AgdaSymbol{(}\AgdaFunction{neutral-rep} \AgdaBound{neutralδ}\AgdaSymbol{)} \<[54]%
\>[54]\<%
\\
\>[2]\AgdaIndent{6}{}\<[6]%
\>[6]\AgdaSymbol{(λ} \AgdaBound{δ'} \AgdaBound{δ〈ρ〉→δ'} \AgdaSymbol{→} \<[22]%
\>[22]\<%
\\
\>[6]\AgdaIndent{8}{}\<[8]%
\>[8]\AgdaKeyword{let} \AgdaBound{δ-creation} \AgdaSymbol{=} \AgdaFunction{create-osr} \AgdaFunction{β-creates-rep} \AgdaBound{δ} \AgdaBound{δ〈ρ〉→δ'} \AgdaKeyword{in} \<[63]%
\>[63]\<%
\\
\>[6]\AgdaIndent{8}{}\<[8]%
\>[8]\AgdaKeyword{let} \AgdaKeyword{open} \AgdaModule{creation} \AgdaBound{δ-creation} \AgdaKeyword{renaming} \AgdaSymbol{(}created \AgdaSymbol{to} δ₀\AgdaSymbol{;}red-created \AgdaSymbol{to} δ⇒δ₀\AgdaSymbol{;}ap-created \AgdaSymbol{to} δ₀〈ρ〉≡δ'\AgdaSymbol{)} \AgdaKeyword{in}\<%
\\
\>[6]\AgdaIndent{8}{}\<[8]%
\>[8]\AgdaKeyword{let} \AgdaBound{δ₀∈C[φ⇒ψ]} \AgdaSymbol{:} \AgdaFunction{C} \AgdaBound{Γ} \AgdaSymbol{(}\AgdaBound{φ} \AgdaInductiveConstructor{⇛} \AgdaBound{ψ}\AgdaSymbol{)} \AgdaFunction{δ₀}\<%
\\
\>[8]\AgdaIndent{12}{}\<[12]%
\>[12]\AgdaBound{δ₀∈C[φ⇒ψ]} \AgdaSymbol{=} \AgdaBound{hyp} \AgdaFunction{δ₀} \AgdaFunction{δ⇒δ₀}\<%
\\
\>[0]\AgdaIndent{8}{}\<[8]%
\>[8]\AgdaKeyword{in} \AgdaKeyword{let} \AgdaBound{δ'∈C[φ⇒ψ]} \AgdaSymbol{:} \AgdaFunction{C} \AgdaBound{Δ} \AgdaSymbol{(}\AgdaBound{φ} \AgdaInductiveConstructor{⇛} \AgdaBound{ψ}\AgdaSymbol{)} \AgdaBound{δ'}\<%
\\
\>[8]\AgdaIndent{15}{}\<[15]%
\>[15]\AgdaBound{δ'∈C[φ⇒ψ]} \AgdaSymbol{=} \AgdaFunction{subst} \AgdaSymbol{(}\AgdaFunction{C} \AgdaBound{Δ} \AgdaSymbol{(}\AgdaBound{φ} \AgdaInductiveConstructor{⇛} \AgdaBound{ψ}\AgdaSymbol{))} \AgdaFunction{δ₀〈ρ〉≡δ'} \AgdaSymbol{(}\AgdaFunction{C-rep} \AgdaSymbol{(}\AgdaBound{φ} \AgdaInductiveConstructor{⇛} \AgdaBound{ψ}\AgdaSymbol{)} \AgdaBound{δ₀∈C[φ⇒ψ]} \AgdaBound{ρ∶Γ→Δ}\AgdaSymbol{)}\<%
\\
\>[0]\AgdaIndent{8}{}\<[8]%
\>[8]\AgdaKeyword{in} \AgdaFunction{subst} \AgdaSymbol{(}\AgdaFunction{C} \AgdaBound{Δ} \AgdaBound{ψ}\AgdaSymbol{)} \AgdaSymbol{(}\AgdaFunction{cong} \AgdaSymbol{(λ} \AgdaBound{x} \AgdaSymbol{→} \AgdaFunction{appP} \AgdaBound{x} \AgdaBound{ε}\AgdaSymbol{)} \AgdaFunction{δ₀〈ρ〉≡δ'}\AgdaSymbol{)} \AgdaSymbol{(}\AgdaField{proj₂} \AgdaBound{δ₀∈C[φ⇒ψ]} \AgdaBound{Q} \AgdaBound{ρ} \AgdaBound{ε} \AgdaBound{ρ∶Γ→Δ} \AgdaBound{ε∈Cφ}\AgdaSymbol{))}\<%
\\
\>[0]\AgdaIndent{6}{}\<[6]%
\>[6]\AgdaSymbol{(λ} \AgdaBound{ε'} \AgdaBound{ε→ε'} \AgdaSymbol{→} \AgdaFunction{claim} \AgdaBound{ε'} \AgdaSymbol{(}\AgdaBound{SNε} \AgdaBound{ε'} \AgdaBound{ε→ε'}\AgdaSymbol{)} \AgdaBound{ρ∶Γ→Δ} \AgdaSymbol{(}\AgdaFunction{C-osr} \AgdaBound{φ} \AgdaBound{ε∈Cφ} \AgdaBound{ε→ε'}\AgdaSymbol{)))}\<%
\\
%
\\
\>\AgdaFunction{CsubSN} \AgdaSymbol{\{}\AgdaBound{P}\AgdaSymbol{\}} \AgdaSymbol{\{}\AgdaBound{Γ}\AgdaSymbol{\}} \AgdaSymbol{\{}\AgdaInductiveConstructor{⊥P}\AgdaSymbol{\}} \AgdaSymbol{=} \AgdaField{proj₂}\<%
\\
\>\AgdaFunction{CsubSN} \AgdaSymbol{\{}\AgdaBound{P}\AgdaSymbol{\}} \AgdaSymbol{\{}\AgdaBound{Γ}\AgdaSymbol{\}} \AgdaSymbol{\{}\AgdaBound{φ} \AgdaInductiveConstructor{⇛} \AgdaBound{ψ}\AgdaSymbol{\}} \AgdaSymbol{\{}\AgdaBound{δ}\AgdaSymbol{\}} \AgdaBound{P₁} \AgdaSymbol{=} \<[32]%
\>[32]\<%
\\
\>[0]\AgdaIndent{4}{}\<[4]%
\>[4]\AgdaFunction{SNap'} \AgdaSymbol{\{}\AgdaFunction{REP}\AgdaSymbol{\}} \AgdaSymbol{\{}\AgdaBound{P}\AgdaSymbol{\}} \AgdaSymbol{\{}\AgdaBound{P} \AgdaInductiveConstructor{,} \AgdaInductiveConstructor{-proof}\AgdaSymbol{\}} \AgdaSymbol{\{}\AgdaArgument{E} \AgdaSymbol{=} \AgdaBound{δ}\AgdaSymbol{\}} \AgdaSymbol{\{}\AgdaArgument{σ} \AgdaSymbol{=} \AgdaFunction{upRep}\AgdaSymbol{\}} \AgdaFunction{β-respects-rep}\<%
\\
\>[4]\AgdaIndent{6}{}\<[6]%
\>[6]\AgdaSymbol{(}\AgdaFunction{SNsubbodyl} \AgdaSymbol{(}\AgdaFunction{SNsubexp} \AgdaSymbol{(}\AgdaFunction{CsubSN} \AgdaSymbol{\{}\AgdaArgument{Γ} \AgdaSymbol{=} \AgdaBound{Γ} \AgdaFunction{,P} \AgdaBound{φ}\AgdaSymbol{\}} \AgdaBound{ψ}\<%
\\
\>[4]\AgdaIndent{6}{}\<[6]%
\>[6]\AgdaSymbol{(}\AgdaField{proj₂} \AgdaBound{P₁} \AgdaSymbol{(}\AgdaBound{P} \AgdaInductiveConstructor{,} \AgdaInductiveConstructor{-proof}\AgdaSymbol{)} \AgdaFunction{upRep} \AgdaSymbol{(}\AgdaInductiveConstructor{var} \AgdaInductiveConstructor{x₀}\AgdaSymbol{)} \AgdaSymbol{(}\AgdaFunction{↑-typed} \AgdaSymbol{\{}\AgdaArgument{φ} \AgdaSymbol{=} \AgdaBound{φ}\AgdaSymbol{\})}\<%
\\
\>[4]\AgdaIndent{6}{}\<[6]%
\>[6]\AgdaSymbol{(}\AgdaFunction{NeutralC} \AgdaBound{φ} \AgdaSymbol{(}\AgdaInductiveConstructor{var} \AgdaSymbol{\_)}\<%
\\
\>[6]\AgdaIndent{8}{}\<[8]%
\>[8]\AgdaSymbol{(}\AgdaInductiveConstructor{varNeutral} \AgdaInductiveConstructor{x₀}\AgdaSymbol{)} \<[24]%
\>[24]\<%
\\
\>[6]\AgdaIndent{8}{}\<[8]%
\>[8]\AgdaSymbol{(λ} \AgdaBound{\_} \AgdaSymbol{()))))))}\<%
\end{code}
}

\begin{corollary}
If $p : \phi \in \Gamma$ then $p \in C_\Gamma(\phi)$.
\end{corollary}

\begin{code}%
\>\AgdaFunction{varC} \AgdaSymbol{:} \AgdaSymbol{∀} \AgdaSymbol{\{}\AgdaBound{P}\AgdaSymbol{\}} \AgdaSymbol{\{}\AgdaBound{Γ} \AgdaSymbol{:} \AgdaDatatype{Context} \AgdaBound{P}\AgdaSymbol{\}} \AgdaSymbol{\{}\AgdaBound{x} \AgdaSymbol{:} \AgdaDatatype{Var} \AgdaBound{P} \AgdaInductiveConstructor{-proof}\AgdaSymbol{\}} \AgdaSymbol{→} \<[50]%
\>[50]\<%
\\
\>[0]\AgdaIndent{2}{}\<[2]%
\>[2]\AgdaFunction{C} \AgdaBound{Γ} \AgdaSymbol{(}\AgdaFunction{unprp} \AgdaSymbol{(}\AgdaFunction{typeof} \AgdaBound{x} \AgdaBound{Γ}\AgdaSymbol{))} \AgdaSymbol{(}\AgdaInductiveConstructor{var} \AgdaBound{x}\AgdaSymbol{)}\<%
\end{code}

\AgdaHide{
\begin{code}%
\>\AgdaFunction{varC} \AgdaSymbol{\{}\AgdaArgument{Γ} \AgdaSymbol{=} \AgdaBound{Γ}\AgdaSymbol{\}} \AgdaSymbol{\{}\AgdaBound{x}\AgdaSymbol{\}} \AgdaSymbol{=} \AgdaFunction{NeutralC} \AgdaSymbol{(}\AgdaFunction{unprp} \AgdaSymbol{(}\AgdaFunction{typeof} \AgdaBound{x} \AgdaBound{Γ}\AgdaSymbol{))} \AgdaSymbol{(}\AgdaInductiveConstructor{var} \AgdaBound{x}\AgdaSymbol{)} \AgdaSymbol{(}\AgdaInductiveConstructor{varNeutral} \AgdaBound{x}\AgdaSymbol{)} \AgdaSymbol{(λ} \AgdaBound{\_} \AgdaSymbol{())}\<%
\end{code}
}

\begin{lemma}[Computability is preserved under well-typed expansion]
Suppose $\Gamma, p : \phi \vdash \delta : \psi$ and $\Gamma \vdash \epsilon : \phi$.  If
$\delta[p:=\epsilon] \in C_\Gamma(\psi)$ and $\epsilon \in SN$, then $(\lambda p:\phi.\delta)\epsilon \in C_\Gamma(\psi)$.
\end{lemma}

\AgdaHide{
\begin{code}%
\>\AgdaFunction{WTEaux} \AgdaSymbol{:} \AgdaSymbol{∀} \AgdaSymbol{\{}\AgdaBound{P}\AgdaSymbol{\}} \AgdaSymbol{\{}\AgdaBound{Γ} \AgdaSymbol{:} \AgdaDatatype{Context} \AgdaBound{P}\AgdaSymbol{\}} \AgdaSymbol{\{}\AgdaBound{φ}\AgdaSymbol{\}} \AgdaSymbol{\{}\AgdaBound{δ}\AgdaSymbol{\}} \AgdaBound{ψ} \AgdaSymbol{\{}\AgdaBound{ε}\AgdaSymbol{\}} \AgdaSymbol{→}\<%
\\
\>[0]\AgdaIndent{2}{}\<[2]%
\>[2]\AgdaBound{Γ} \AgdaFunction{,P} \AgdaBound{φ} \AgdaDatatype{⊢} \AgdaBound{δ} \AgdaDatatype{∶} \AgdaBound{ψ} \AgdaSymbol{→}\<%
\\
\>[0]\AgdaIndent{2}{}\<[2]%
\>[2]\AgdaBound{Γ} \AgdaDatatype{⊢} \AgdaBound{ε} \AgdaDatatype{∶} \AgdaBound{φ} \AgdaSymbol{→}\<%
\\
\>[0]\AgdaIndent{2}{}\<[2]%
\>[2]\AgdaFunction{C} \AgdaBound{Γ} \AgdaBound{ψ} \AgdaSymbol{(}\AgdaBound{δ} \AgdaFunction{⟦} \AgdaFunction{x₀:=} \AgdaBound{ε} \AgdaFunction{⟧}\AgdaSymbol{)} \AgdaSymbol{→}\<%
\\
\>[0]\AgdaIndent{2}{}\<[2]%
\>[2]\AgdaDatatype{SN} \AgdaBound{δ} \AgdaSymbol{→} \AgdaDatatype{SN} \AgdaBound{ε} \AgdaSymbol{→}\<%
\\
\>[0]\AgdaIndent{2}{}\<[2]%
\>[2]\AgdaFunction{C} \AgdaBound{Γ} \AgdaBound{ψ} \AgdaSymbol{(}\AgdaFunction{appP} \AgdaSymbol{(}\AgdaFunction{ΛP} \AgdaBound{φ} \AgdaBound{δ}\AgdaSymbol{)} \AgdaBound{ε}\AgdaSymbol{)}\<%
\\
\>\AgdaFunction{WTEaux} \AgdaSymbol{\{}\AgdaArgument{Γ} \AgdaSymbol{=} \AgdaBound{Γ}\AgdaSymbol{\}} \AgdaSymbol{\{}\AgdaArgument{φ} \AgdaSymbol{=} \AgdaBound{φ}\AgdaSymbol{\}} \AgdaBound{ψ} \AgdaBound{Γ,p∶φ⊢δ∶ψ} \AgdaBound{Γ⊢ε∶φ} \AgdaBound{δ[p∶=ε]∈CΓψ} \AgdaSymbol{(}\AgdaInductiveConstructor{SNI} \AgdaBound{δ} \AgdaBound{SNδ}\AgdaSymbol{)} \AgdaSymbol{(}\AgdaInductiveConstructor{SNI} \AgdaBound{ε} \AgdaBound{SNε}\AgdaSymbol{)} \AgdaSymbol{=} \AgdaFunction{NeutralC} \AgdaSymbol{\{}\AgdaArgument{Γ} \AgdaSymbol{=} \AgdaBound{Γ}\AgdaSymbol{\}} \AgdaSymbol{\{}\AgdaArgument{δ} \AgdaSymbol{=} \AgdaFunction{appP} \AgdaSymbol{(}\AgdaFunction{ΛP} \AgdaBound{φ} \AgdaBound{δ}\AgdaSymbol{)} \AgdaBound{ε}\AgdaSymbol{\}} \AgdaBound{ψ}\<%
\\
\>[0]\AgdaIndent{2}{}\<[2]%
\>[2]\AgdaSymbol{(}\AgdaInductiveConstructor{app} \AgdaSymbol{(}\AgdaInductiveConstructor{Λ} \AgdaBound{Γ,p∶φ⊢δ∶ψ}\AgdaSymbol{)} \AgdaBound{Γ⊢ε∶φ}\AgdaSymbol{)} \<[28]%
\>[28]\<%
\\
\>[0]\AgdaIndent{2}{}\<[2]%
\>[2]\AgdaSymbol{(}\AgdaInductiveConstructor{appNeutral} \AgdaSymbol{\_} \AgdaSymbol{\_)} \<[19]%
\>[19]\<%
\\
\>[0]\AgdaIndent{2}{}\<[2]%
\>[2]\AgdaSymbol{(λ} \AgdaBound{χ} \AgdaBound{Λφδε⇒χ} \AgdaSymbol{→} \AgdaFunction{red-β-redex} \AgdaSymbol{(}\AgdaFunction{C} \AgdaBound{Γ} \AgdaBound{ψ}\AgdaSymbol{)} \AgdaBound{Λφδε⇒χ} \AgdaBound{δ[p∶=ε]∈CΓψ}\<%
\\
\>[2]\AgdaIndent{4}{}\<[4]%
\>[4]\AgdaSymbol{(λ} \AgdaBound{δ'} \AgdaBound{δ⇒δ'} \AgdaSymbol{→} \AgdaFunction{WTEaux} \AgdaBound{ψ}\<%
\\
\>[4]\AgdaIndent{6}{}\<[6]%
\>[6]\AgdaSymbol{(}\AgdaFunction{subject-reduction} \AgdaBound{Γ,p∶φ⊢δ∶ψ} \AgdaBound{δ⇒δ'}\AgdaSymbol{)} \<[41]%
\>[41]\<%
\\
\>[4]\AgdaIndent{6}{}\<[6]%
\>[6]\AgdaBound{Γ⊢ε∶φ} \<[12]%
\>[12]\<%
\\
\>[4]\AgdaIndent{6}{}\<[6]%
\>[6]\AgdaSymbol{(}\AgdaFunction{C-osr} \AgdaBound{ψ} \AgdaBound{δ[p∶=ε]∈CΓψ} \AgdaSymbol{(}\AgdaFunction{respects-osr} \AgdaFunction{SUB} \AgdaFunction{β-respects-sub} \AgdaBound{δ⇒δ'}\AgdaSymbol{))} \<[67]%
\>[67]\<%
\\
\>[4]\AgdaIndent{6}{}\<[6]%
\>[6]\AgdaSymbol{(}\AgdaBound{SNδ} \AgdaBound{δ'} \AgdaBound{δ⇒δ'}\AgdaSymbol{)} \AgdaSymbol{(}\AgdaInductiveConstructor{SNI} \AgdaBound{ε} \AgdaBound{SNε}\AgdaSymbol{))} \<[33]%
\>[33]\<%
\\
\>[0]\AgdaIndent{4}{}\<[4]%
\>[4]\AgdaSymbol{(λ} \AgdaBound{ε'} \AgdaBound{ε⇒ε'} \AgdaSymbol{→} \AgdaFunction{WTEaux} \AgdaBound{ψ}\<%
\\
\>[0]\AgdaIndent{4}{}\<[4]%
\>[4]\AgdaBound{Γ,p∶φ⊢δ∶ψ} \<[14]%
\>[14]\<%
\\
\>[0]\AgdaIndent{4}{}\<[4]%
\>[4]\AgdaSymbol{(}\AgdaFunction{subject-reduction} \AgdaBound{Γ⊢ε∶φ} \AgdaBound{ε⇒ε'}\AgdaSymbol{)} \<[35]%
\>[35]\<%
\\
\>[0]\AgdaIndent{4}{}\<[4]%
\>[4]\AgdaSymbol{(}\AgdaFunction{C-red} \AgdaBound{ψ} \AgdaSymbol{\{}\AgdaBound{δ} \AgdaFunction{⟦} \AgdaFunction{x₀:=} \AgdaBound{ε} \AgdaFunction{⟧}\AgdaSymbol{\}} \AgdaSymbol{\{}\AgdaBound{δ} \AgdaFunction{⟦} \AgdaFunction{x₀:=} \AgdaBound{ε'} \AgdaFunction{⟧}\AgdaSymbol{\}} \AgdaBound{δ[p∶=ε]∈CΓψ} \AgdaSymbol{(}\AgdaFunction{apredl} \AgdaFunction{SUB} \AgdaSymbol{\{}\AgdaArgument{E} \AgdaSymbol{=} \AgdaBound{δ}\AgdaSymbol{\}} \AgdaFunction{β-respects-sub} \AgdaSymbol{(}\AgdaFunction{botsub-red} \AgdaBound{ε⇒ε'}\AgdaSymbol{)))} \<[111]%
\>[111]\<%
\\
\>[0]\AgdaIndent{4}{}\<[4]%
\>[4]\AgdaSymbol{(}\AgdaInductiveConstructor{SNI} \AgdaBound{δ} \AgdaBound{SNδ}\AgdaSymbol{)} \AgdaSymbol{(}\AgdaBound{SNε} \AgdaSymbol{\_} \AgdaBound{ε⇒ε'}\AgdaSymbol{)))}\<%
\end{code}
}

\begin{code}%
\>\AgdaFunction{WTE} \AgdaSymbol{:} \AgdaSymbol{∀} \AgdaSymbol{\{}\AgdaBound{P}\AgdaSymbol{\}} \AgdaSymbol{\{}\AgdaBound{Γ} \AgdaSymbol{:} \AgdaDatatype{Context} \AgdaBound{P}\AgdaSymbol{\}} \AgdaSymbol{\{}\AgdaBound{φ}\AgdaSymbol{\}} \AgdaSymbol{\{}\AgdaBound{δ}\AgdaSymbol{\}} \AgdaSymbol{\{}\AgdaBound{ψ}\AgdaSymbol{\}} \AgdaSymbol{\{}\AgdaBound{ε}\AgdaSymbol{\}} \AgdaSymbol{→}\<%
\\
\>[0]\AgdaIndent{2}{}\<[2]%
\>[2]\AgdaBound{Γ} \AgdaFunction{,P} \AgdaBound{φ} \AgdaDatatype{⊢} \AgdaBound{δ} \AgdaDatatype{∶} \AgdaBound{ψ} \AgdaSymbol{→}\<%
\\
\>[0]\AgdaIndent{2}{}\<[2]%
\>[2]\AgdaBound{Γ} \AgdaDatatype{⊢} \AgdaBound{ε} \AgdaDatatype{∶} \AgdaBound{φ} \AgdaSymbol{→}\<%
\\
\>[0]\AgdaIndent{2}{}\<[2]%
\>[2]\AgdaFunction{C} \AgdaBound{Γ} \AgdaBound{ψ} \AgdaSymbol{(}\AgdaBound{δ} \AgdaFunction{⟦} \AgdaFunction{x₀:=} \AgdaBound{ε} \AgdaFunction{⟧}\AgdaSymbol{)} \AgdaSymbol{→}\<%
\\
\>[0]\AgdaIndent{2}{}\<[2]%
\>[2]\AgdaDatatype{SN} \AgdaBound{ε} \AgdaSymbol{→}\<%
\\
\>[0]\AgdaIndent{2}{}\<[2]%
\>[2]\AgdaFunction{C} \AgdaBound{Γ} \AgdaBound{ψ} \AgdaSymbol{(}\AgdaFunction{appP} \AgdaSymbol{(}\AgdaFunction{ΛP} \AgdaBound{φ} \AgdaBound{δ}\AgdaSymbol{)} \AgdaBound{ε}\AgdaSymbol{)}\<%
\end{code}

\AgdaHide{
\begin{code}%
\>\AgdaFunction{WTE} \AgdaSymbol{\{}\AgdaArgument{ψ} \AgdaSymbol{=} \AgdaBound{ψ}\AgdaSymbol{\}} \AgdaBound{Γ,p∶φ⊢δ∶ψ} \AgdaBound{Γ⊢ε∶φ} \AgdaBound{δ[p∶=ε]∈CΓψ} \AgdaSymbol{=} \AgdaFunction{WTEaux} \AgdaBound{ψ} \AgdaBound{Γ,p∶φ⊢δ∶ψ} \AgdaBound{Γ⊢ε∶φ} \AgdaBound{δ[p∶=ε]∈CΓψ} \AgdaSymbol{(}\AgdaFunction{SNap'} \AgdaSymbol{\{}\AgdaFunction{SUB}\AgdaSymbol{\}} \AgdaFunction{β-respects-sub} \AgdaSymbol{(}\AgdaFunction{CsubSN} \AgdaBound{ψ} \AgdaBound{δ[p∶=ε]∈CΓψ}\AgdaSymbol{))}\<%
\end{code}
}

\begin{lm}
If $\delta \in C_\Gamma(\phi \rightarrow \psi)$ and $\epsilon \in C_\Gamma(\phi)$ then $\delta \epsilon \in C_\Gamma(\psi)$.
\end{lm}

\begin{code}%
\>\AgdaFunction{C-app} \AgdaSymbol{:} \AgdaSymbol{∀} \AgdaSymbol{\{}\AgdaBound{P}\AgdaSymbol{\}} \AgdaSymbol{\{}\AgdaBound{Γ} \AgdaSymbol{:} \AgdaDatatype{Context} \AgdaBound{P}\AgdaSymbol{\}} \AgdaSymbol{\{}\AgdaBound{δ} \AgdaBound{ε} \AgdaBound{φ} \AgdaBound{ψ}\AgdaSymbol{\}} \AgdaSymbol{→} \AgdaFunction{C} \AgdaBound{Γ} \AgdaSymbol{(}\AgdaBound{φ} \AgdaInductiveConstructor{⇛} \AgdaBound{ψ}\AgdaSymbol{)} \AgdaBound{δ} \AgdaSymbol{→} \AgdaFunction{C} \AgdaBound{Γ} \AgdaBound{φ} \AgdaBound{ε} \AgdaSymbol{→} \AgdaFunction{C} \AgdaBound{Γ} \AgdaBound{ψ} \AgdaSymbol{(}\AgdaFunction{appP} \AgdaBound{δ} \AgdaBound{ε}\AgdaSymbol{)}\<%
\\
\>\AgdaFunction{C-app} \AgdaSymbol{\{}\AgdaBound{P}\AgdaSymbol{\}} \AgdaSymbol{\{}\AgdaBound{Γ}\AgdaSymbol{\}} \AgdaSymbol{\{}\AgdaBound{δ}\AgdaSymbol{\}} \AgdaSymbol{\{}\AgdaBound{ε}\AgdaSymbol{\}} \AgdaSymbol{\{}\AgdaBound{φ}\AgdaSymbol{\}} \AgdaSymbol{\{}\AgdaBound{ψ}\AgdaSymbol{\}} \AgdaSymbol{(}\AgdaBound{Γ⊢δ∶φ⇛ψ} \AgdaInductiveConstructor{,p} \AgdaBound{δ∈CΓφ⇛ψ}\AgdaSymbol{)} \AgdaBound{ε∈CΓφ} \AgdaSymbol{=}\<%
\\
\>[0]\AgdaIndent{2}{}\<[2]%
\>[2]\AgdaFunction{subst} \AgdaSymbol{(λ} \AgdaBound{x} \AgdaSymbol{→} \AgdaFunction{C} \AgdaBound{Γ} \AgdaBound{ψ} \AgdaSymbol{(}\AgdaFunction{appP} \AgdaBound{x} \AgdaBound{ε}\AgdaSymbol{))} \AgdaFunction{ap-idRep} \AgdaSymbol{(}\AgdaBound{δ∈CΓφ⇛ψ} \AgdaBound{P} \AgdaSymbol{(}\AgdaFunction{idRep} \AgdaBound{P}\AgdaSymbol{)} \AgdaBound{ε} \AgdaFunction{idRep-typed} \AgdaBound{ε∈CΓφ}\AgdaSymbol{)}\<%
\end{code}

A substitution $\sigma$ is a \emph{computable} substitution from $\Gamma$ to $\Delta$, $\sigma : \Gamma \rightarrow_C \Delta$,
iff for all $p : \phi \in \Gamma$ we have $\sigma(p) \in C_\Delta(\phi)$.

\begin{code}%
\>\AgdaFunction{\_∶\_⇒C\_} \AgdaSymbol{:} \AgdaSymbol{∀} \AgdaSymbol{\{}\AgdaBound{P}\AgdaSymbol{\}} \AgdaSymbol{\{}\AgdaBound{Q}\AgdaSymbol{\}} \AgdaSymbol{→} \AgdaFunction{Sub} \AgdaBound{P} \AgdaBound{Q} \AgdaSymbol{→} \AgdaDatatype{Context} \AgdaBound{P} \AgdaSymbol{→} \AgdaDatatype{Context} \AgdaBound{Q} \AgdaSymbol{→} \AgdaPrimitiveType{Set}\<%
\\
\>\AgdaBound{σ} \AgdaFunction{∶} \AgdaBound{Γ} \AgdaFunction{⇒C} \AgdaBound{Δ} \AgdaSymbol{=} \AgdaSymbol{∀} \AgdaBound{x} \AgdaSymbol{→} \AgdaFunction{C} \AgdaBound{Δ} \AgdaSymbol{(}\AgdaFunction{unprp} \AgdaSymbol{(}\AgdaFunction{typeof} \AgdaSymbol{\{}\AgdaArgument{K} \AgdaSymbol{=} \AgdaInductiveConstructor{-proof}\AgdaSymbol{\}} \AgdaBound{x} \AgdaBound{Γ}\AgdaSymbol{))} \AgdaSymbol{(}\AgdaBound{σ} \AgdaSymbol{\_} \AgdaBound{x}\AgdaSymbol{)}\<%
\end{code}

\begin{prop}
If $\sigma : \Gamma \rightarrow_C \Delta$ then $\sigma : \Gamma \rightarrow \Delta$.
\end{prop}

\begin{code}%
\>\AgdaFunction{SubC-typed} \AgdaSymbol{:} \AgdaSymbol{∀} \AgdaSymbol{\{}\AgdaBound{P} \AgdaBound{Q}\AgdaSymbol{\}} \AgdaSymbol{\{}\AgdaBound{σ} \AgdaSymbol{:} \AgdaFunction{Sub} \AgdaBound{P} \AgdaBound{Q}\AgdaSymbol{\}} \AgdaSymbol{\{}\AgdaBound{Γ} \AgdaBound{Δ}\AgdaSymbol{\}} \AgdaSymbol{→} \AgdaBound{σ} \AgdaFunction{∶} \AgdaBound{Γ} \AgdaFunction{⇒C} \AgdaBound{Δ} \AgdaSymbol{→} \AgdaBound{σ} \AgdaFunction{∶} \AgdaBound{Γ} \AgdaFunction{⇒} \AgdaBound{Δ}\<%
\\
\>\AgdaFunction{SubC-typed} \AgdaBound{σ∶Γ⇒CΔ} \AgdaBound{x} \AgdaSymbol{=} \AgdaFunction{C-typed} \AgdaSymbol{(}\AgdaBound{σ∶Γ⇒CΔ} \AgdaBound{x}\AgdaSymbol{)}\<%
\end{code}

\begin{prop}
If $\Gamma \vdash \delta : \phi$ and $\sigma : \Gamma \rightarrow_C \Delta$ then $\delta [ \sigma ] \in C_\Delta(\phi)$.
\end{prop}

\begin{code}%
\>\AgdaFunction{compRSC} \AgdaSymbol{:} \AgdaSymbol{∀} \AgdaSymbol{\{}\AgdaBound{P}\AgdaSymbol{\}} \AgdaSymbol{\{}\AgdaBound{Q}\AgdaSymbol{\}} \AgdaSymbol{\{}\AgdaBound{R}\AgdaSymbol{\}}\<%
\\
\>[0]\AgdaIndent{2}{}\<[2]%
\>[2]\AgdaSymbol{\{}\AgdaBound{ρ} \AgdaSymbol{:} \AgdaFunction{Rep} \AgdaBound{Q} \AgdaBound{R}\AgdaSymbol{\}} \AgdaSymbol{\{}\AgdaBound{σ} \AgdaSymbol{:} \AgdaFunction{Sub} \AgdaBound{P} \AgdaBound{Q}\AgdaSymbol{\}}\<%
\\
\>[0]\AgdaIndent{2}{}\<[2]%
\>[2]\AgdaSymbol{\{}\AgdaBound{Γ} \AgdaSymbol{:} \AgdaDatatype{Context} \AgdaBound{P}\AgdaSymbol{\}} \AgdaSymbol{\{}\AgdaBound{Δ} \AgdaSymbol{:} \AgdaDatatype{Context} \AgdaBound{Q}\AgdaSymbol{\}} \AgdaSymbol{\{}\AgdaBound{Θ} \AgdaSymbol{:} \AgdaDatatype{Context} \AgdaBound{R}\AgdaSymbol{\}} \AgdaSymbol{→}\<%
\\
\>[0]\AgdaIndent{2}{}\<[2]%
\>[2]\AgdaBound{ρ} \AgdaFunction{∶} \AgdaBound{Δ} \AgdaFunction{⇒R} \AgdaBound{Θ} \AgdaSymbol{→} \AgdaBound{σ} \AgdaFunction{∶} \AgdaBound{Γ} \AgdaFunction{⇒C} \AgdaBound{Δ} \AgdaSymbol{→} \AgdaBound{ρ} \AgdaFunction{•RS} \AgdaBound{σ} \AgdaFunction{∶} \AgdaBound{Γ} \AgdaFunction{⇒C} \AgdaBound{Θ}\<%
\\
\>\AgdaFunction{compRSC} \AgdaSymbol{\{}\AgdaArgument{Γ} \AgdaSymbol{=} \AgdaBound{Γ}\AgdaSymbol{\}} \AgdaBound{ρ∶Δ⇒RΘ} \AgdaBound{σ∶Γ⇒CΔ} \AgdaBound{x} \AgdaSymbol{=} \<[34]%
\>[34]\<%
\\
\>[2]\AgdaIndent{6}{}\<[6]%
\>[6]\AgdaFunction{C-rep} \AgdaSymbol{(}\AgdaFunction{unprp} \AgdaSymbol{(}\AgdaFunction{typeof} \AgdaBound{x} \AgdaBound{Γ}\AgdaSymbol{))} \AgdaSymbol{(}\AgdaBound{σ∶Γ⇒CΔ} \AgdaBound{x}\AgdaSymbol{)} \AgdaBound{ρ∶Δ⇒RΘ}\<%
\\
%
\\
\>\AgdaFunction{extend-sub} \AgdaSymbol{:} \AgdaSymbol{∀} \AgdaSymbol{\{}\AgdaBound{P}\AgdaSymbol{\}} \AgdaSymbol{\{}\AgdaBound{Q}\AgdaSymbol{\}} \AgdaSymbol{\{}\AgdaBound{σ} \AgdaSymbol{:} \AgdaFunction{Sub} \AgdaBound{P} \AgdaBound{Q}\AgdaSymbol{\}} \AgdaSymbol{\{}\AgdaBound{Γ}\AgdaSymbol{\}} \AgdaSymbol{\{}\AgdaBound{Δ}\AgdaSymbol{\}} \AgdaSymbol{\{}\AgdaBound{δ}\AgdaSymbol{\}} \AgdaSymbol{\{}\AgdaBound{φ}\AgdaSymbol{\}} \AgdaSymbol{→}\<%
\\
\>[6]\AgdaIndent{11}{}\<[11]%
\>[11]\AgdaBound{σ} \AgdaFunction{∶} \AgdaBound{Γ} \AgdaFunction{⇒C} \AgdaBound{Δ} \AgdaSymbol{→} \AgdaFunction{C} \AgdaBound{Δ} \AgdaBound{φ} \AgdaBound{δ} \AgdaSymbol{→} \AgdaFunction{x₀:=} \AgdaBound{δ} \AgdaFunction{•} \AgdaFunction{liftSub} \AgdaSymbol{\_} \AgdaBound{σ} \AgdaFunction{∶} \AgdaBound{Γ} \AgdaFunction{,P} \AgdaBound{φ} \AgdaFunction{⇒C} \AgdaBound{Δ}\<%
\\
\>\AgdaFunction{extend-sub} \AgdaBound{σ∶Γ⇒CΔ} \AgdaBound{δ∈CΔφ} \AgdaInductiveConstructor{x₀} \AgdaSymbol{=} \AgdaBound{δ∈CΔφ}\<%
\\
\>\AgdaFunction{extend-sub} \AgdaSymbol{\{}\AgdaArgument{σ} \AgdaSymbol{=} \AgdaBound{σ}\AgdaSymbol{\}} \AgdaSymbol{\{}\AgdaBound{Γ}\AgdaSymbol{\}} \AgdaSymbol{\{}\AgdaArgument{Δ} \AgdaSymbol{=} \AgdaBound{Δ}\AgdaSymbol{\}} \AgdaSymbol{\{}\AgdaBound{δ}\AgdaSymbol{\}} \AgdaSymbol{\{}\AgdaBound{φ}\AgdaSymbol{\}} \AgdaBound{σ∶Γ⇒CΔ} \AgdaBound{δ∈CΔφ} \AgdaSymbol{(}\AgdaInductiveConstructor{↑} \AgdaBound{x}\AgdaSymbol{)} \AgdaSymbol{=}\<%
\\
\>[0]\AgdaIndent{2}{}\<[2]%
\>[2]\AgdaKeyword{let} \AgdaBound{σx∈CΔφ} \AgdaSymbol{:} \AgdaFunction{C} \AgdaBound{Δ} \AgdaSymbol{(}\AgdaFunction{unprp} \AgdaSymbol{(}\AgdaFunction{typeof} \AgdaBound{x} \AgdaBound{Γ}\AgdaSymbol{))} \AgdaSymbol{(}\AgdaBound{σ} \AgdaSymbol{\_} \AgdaBound{x}\AgdaSymbol{)}\<%
\\
\>[2]\AgdaIndent{6}{}\<[6]%
\>[6]\AgdaBound{σx∈CΔφ} \AgdaSymbol{=} \AgdaBound{σ∶Γ⇒CΔ} \AgdaBound{x} \AgdaKeyword{in}\<%
\\
\>[0]\AgdaIndent{2}{}\<[2]%
\>[2]\AgdaFunction{subst₂} \AgdaSymbol{(}\AgdaFunction{C} \AgdaBound{Δ}\AgdaSymbol{)} \<[15]%
\>[15]\<%
\\
\>[0]\AgdaIndent{2}{}\<[2]%
\>[2]\AgdaSymbol{(}\AgdaKeyword{let} \AgdaKeyword{open} \AgdaModule{≡-Reasoning} \AgdaKeyword{in}\<%
\\
\>[0]\AgdaIndent{2}{}\<[2]%
\>[2]\AgdaFunction{begin}\<%
\\
\>[2]\AgdaIndent{4}{}\<[4]%
\>[4]\AgdaFunction{unprp} \AgdaSymbol{(}\AgdaFunction{typeof} \AgdaBound{x} \AgdaBound{Γ}\AgdaSymbol{)}\<%
\\
\>[0]\AgdaIndent{2}{}\<[2]%
\>[2]\AgdaFunction{≡⟨⟨} \AgdaFunction{unprp-rep} \AgdaSymbol{(}\AgdaFunction{typeof} \AgdaBound{x} \AgdaBound{Γ}\AgdaSymbol{)} \AgdaFunction{upRep} \AgdaFunction{⟩⟩}\<%
\\
\>[2]\AgdaIndent{4}{}\<[4]%
\>[4]\AgdaFunction{unprp} \AgdaSymbol{(}\AgdaFunction{typeof} \AgdaBound{x} \AgdaBound{Γ} \AgdaFunction{⇑}\AgdaSymbol{)}\<%
\\
\>[0]\AgdaIndent{2}{}\<[2]%
\>[2]\AgdaFunction{∎}\AgdaSymbol{)} \AgdaSymbol{(}\AgdaKeyword{let} \AgdaKeyword{open} \AgdaModule{≡-Reasoning} \AgdaKeyword{in}\<%
\\
\>[0]\AgdaIndent{2}{}\<[2]%
\>[2]\AgdaFunction{begin}\<%
\\
\>[2]\AgdaIndent{4}{}\<[4]%
\>[4]\AgdaBound{σ} \AgdaSymbol{\_} \AgdaBound{x}\<%
\\
\>[0]\AgdaIndent{2}{}\<[2]%
\>[2]\AgdaFunction{≡⟨⟨} \AgdaFunction{botSub-upRep} \AgdaSymbol{(}\AgdaBound{σ} \AgdaSymbol{\_} \AgdaBound{x}\AgdaSymbol{)} \AgdaFunction{⟩⟩}\<%
\\
\>[2]\AgdaIndent{4}{}\<[4]%
\>[4]\AgdaBound{σ} \AgdaSymbol{\_} \AgdaBound{x} \AgdaFunction{⇑} \AgdaFunction{⟦} \AgdaFunction{x₀:=} \AgdaBound{δ} \AgdaFunction{⟧}\<%
\\
\>[0]\AgdaIndent{2}{}\<[2]%
\>[2]\AgdaFunction{∎}\AgdaSymbol{)} \AgdaBound{σx∈CΔφ}\<%
\\
%
\\
\>\AgdaFunction{SNmainlemma} \AgdaSymbol{:} \AgdaSymbol{∀} \AgdaSymbol{\{}\AgdaBound{P}\AgdaSymbol{\}} \AgdaSymbol{\{}\AgdaBound{Q}\AgdaSymbol{\}} \AgdaSymbol{\{}\AgdaBound{Γ} \AgdaSymbol{:} \AgdaDatatype{Context} \AgdaBound{P}\AgdaSymbol{\}} \AgdaSymbol{\{}\AgdaBound{δ}\AgdaSymbol{\}} \AgdaSymbol{\{}\AgdaBound{φ}\AgdaSymbol{\}} \AgdaSymbol{\{}\AgdaBound{σ} \AgdaSymbol{:} \AgdaFunction{Sub} \AgdaBound{P} \AgdaBound{Q}\AgdaSymbol{\}} \AgdaSymbol{\{}\AgdaBound{Δ}\AgdaSymbol{\}} \AgdaSymbol{→}\<%
\\
\>[0]\AgdaIndent{2}{}\<[2]%
\>[2]\AgdaBound{Γ} \AgdaDatatype{⊢} \AgdaBound{δ} \AgdaDatatype{∶} \AgdaBound{φ} \AgdaSymbol{→} \AgdaBound{σ} \AgdaFunction{∶} \AgdaBound{Γ} \AgdaFunction{⇒C} \AgdaBound{Δ} \AgdaSymbol{→}\<%
\\
\>[0]\AgdaIndent{2}{}\<[2]%
\>[2]\AgdaFunction{C} \AgdaBound{Δ} \AgdaBound{φ} \AgdaSymbol{(}\AgdaBound{δ} \AgdaFunction{⟦} \AgdaBound{σ} \AgdaFunction{⟧}\AgdaSymbol{)}\<%
\end{code}

\AgdaHide{
\begin{code}%
\>\AgdaComment{--TODO Tidy up}\<%
\\
\>\AgdaFunction{SNmainlemma} \AgdaSymbol{(}\AgdaInductiveConstructor{var} \AgdaBound{p}\AgdaSymbol{)} \AgdaBound{σ∶Γ⇒CΔ} \AgdaSymbol{=} \AgdaBound{σ∶Γ⇒CΔ} \AgdaBound{p}\<%
\\
\>\AgdaFunction{SNmainlemma} \AgdaSymbol{(}\AgdaInductiveConstructor{app} \AgdaBound{Γ⊢δ∶φ⇛ψ} \AgdaBound{Γ⊢ε∶φ}\AgdaSymbol{)} \AgdaBound{σ∶Γ⇒CΔ} \AgdaSymbol{=} \<[41]%
\>[41]\<%
\\
\>[0]\AgdaIndent{2}{}\<[2]%
\>[2]\AgdaFunction{C-app} \AgdaSymbol{(}\AgdaFunction{SNmainlemma} \AgdaBound{Γ⊢δ∶φ⇛ψ} \AgdaBound{σ∶Γ⇒CΔ}\AgdaSymbol{)} \AgdaSymbol{(}\AgdaFunction{SNmainlemma} \AgdaBound{Γ⊢ε∶φ} \AgdaBound{σ∶Γ⇒CΔ}\AgdaSymbol{)}\<%
\\
\>\AgdaFunction{SNmainlemma} \AgdaSymbol{\{}\AgdaBound{P}\AgdaSymbol{\}} \AgdaSymbol{\{}\AgdaArgument{Γ} \AgdaSymbol{=} \AgdaBound{Γ}\AgdaSymbol{\}} \AgdaSymbol{\{}\AgdaArgument{σ} \AgdaSymbol{=} \AgdaBound{σ}\AgdaSymbol{\}} \AgdaSymbol{\{}\AgdaBound{Δ}\AgdaSymbol{\}} \AgdaSymbol{(}\AgdaInductiveConstructor{Λ} \AgdaSymbol{\{}\AgdaArgument{φ} \AgdaSymbol{=} \AgdaBound{φ}\AgdaSymbol{\}} \AgdaSymbol{\{}\AgdaArgument{δ} \AgdaSymbol{=} \AgdaBound{δ}\AgdaSymbol{\}} \AgdaSymbol{\{}\AgdaBound{ψ}\AgdaSymbol{\}} \AgdaBound{Γ,φ⊢δ∶ψ}\AgdaSymbol{)} \AgdaBound{σ∶Γ⇒CΔ} \AgdaSymbol{=} \AgdaSymbol{(}\AgdaFunction{substitution} \AgdaSymbol{(}\AgdaInductiveConstructor{Λ} \AgdaBound{Γ,φ⊢δ∶ψ}\AgdaSymbol{)} \AgdaSymbol{(}\AgdaFunction{SubC-typed} \AgdaSymbol{\{}\AgdaArgument{σ} \AgdaSymbol{=} \AgdaBound{σ}\AgdaSymbol{\}} \AgdaBound{σ∶Γ⇒CΔ}\AgdaSymbol{))} \AgdaInductiveConstructor{,p} \<[135]%
\>[135]\<%
\\
\>[0]\AgdaIndent{2}{}\<[2]%
\>[2]\AgdaSymbol{(λ} \AgdaBound{R} \AgdaSymbol{\{}\AgdaBound{Θ}\AgdaSymbol{\}} \AgdaBound{ρ} \AgdaBound{ε} \AgdaBound{ρ∶Δ⇒RΘ} \AgdaBound{ε∈CΘφ} \AgdaSymbol{→} \<[30]%
\>[30]\<%
\\
\>[2]\AgdaIndent{4}{}\<[4]%
\>[4]\AgdaKeyword{let} \AgdaBound{δσ[ε]∈CΘψ} \AgdaSymbol{:} \AgdaFunction{C} \AgdaBound{Θ} \AgdaBound{ψ} \AgdaSymbol{(}\AgdaBound{δ} \AgdaFunction{⟦} \AgdaFunction{liftSub} \AgdaSymbol{\_} \AgdaBound{σ} \AgdaFunction{⟧} \AgdaFunction{〈} \AgdaFunction{liftRep} \AgdaSymbol{\_} \AgdaBound{ρ} \AgdaFunction{〉} \AgdaFunction{⟦} \AgdaFunction{x₀:=} \AgdaBound{ε} \AgdaFunction{⟧}\AgdaSymbol{)}\<%
\\
\>[4]\AgdaIndent{8}{}\<[8]%
\>[8]\AgdaBound{δσ[ε]∈CΘψ} \AgdaSymbol{=} \AgdaFunction{subst} \AgdaSymbol{(}\AgdaFunction{C} \AgdaBound{Θ} \AgdaBound{ψ}\AgdaSymbol{)} \<[34]%
\>[34]\<%
\\
\>[8]\AgdaIndent{10}{}\<[10]%
\>[10]\AgdaSymbol{(}\AgdaKeyword{let} \AgdaKeyword{open} \AgdaModule{≡-Reasoning} \AgdaKeyword{in}\<%
\\
\>[10]\AgdaIndent{12}{}\<[12]%
\>[12]\AgdaBound{δ} \AgdaFunction{⟦} \AgdaFunction{x₀:=} \AgdaBound{ε} \AgdaFunction{•} \AgdaFunction{liftSub} \AgdaSymbol{\_} \AgdaSymbol{(}\AgdaBound{ρ} \AgdaFunction{•RS} \AgdaBound{σ}\AgdaSymbol{)} \AgdaFunction{⟧}\<%
\\
\>[0]\AgdaIndent{10}{}\<[10]%
\>[10]\AgdaFunction{≡⟨} \AgdaFunction{sub-comp} \AgdaBound{δ} \AgdaFunction{⟩}\<%
\\
\>[10]\AgdaIndent{12}{}\<[12]%
\>[12]\AgdaBound{δ} \AgdaFunction{⟦} \AgdaFunction{liftSub} \AgdaSymbol{\_} \AgdaSymbol{(}\AgdaBound{ρ} \AgdaFunction{•RS} \AgdaBound{σ}\AgdaSymbol{)} \AgdaFunction{⟧} \AgdaFunction{⟦} \AgdaFunction{x₀:=} \AgdaBound{ε} \AgdaFunction{⟧}\<%
\\
\>[0]\AgdaIndent{10}{}\<[10]%
\>[10]\AgdaFunction{≡⟨} \AgdaFunction{sub-congl} \AgdaSymbol{(}\AgdaFunction{sub-congr} \AgdaFunction{liftSub-compRS} \AgdaBound{δ}\AgdaSymbol{)} \AgdaFunction{⟩}\<%
\\
\>[10]\AgdaIndent{12}{}\<[12]%
\>[12]\AgdaBound{δ} \AgdaFunction{⟦} \AgdaFunction{liftRep} \AgdaSymbol{\_} \AgdaBound{ρ} \AgdaFunction{•RS} \AgdaFunction{liftSub} \AgdaSymbol{\_} \AgdaBound{σ} \AgdaFunction{⟧} \AgdaFunction{⟦} \AgdaFunction{x₀:=} \AgdaBound{ε} \AgdaFunction{⟧}\<%
\\
\>[0]\AgdaIndent{10}{}\<[10]%
\>[10]\AgdaFunction{≡⟨} \AgdaFunction{sub-congl} \AgdaSymbol{(}\AgdaFunction{sub-compRS} \AgdaBound{δ}\AgdaSymbol{)} \AgdaFunction{⟩}\<%
\\
\>[10]\AgdaIndent{12}{}\<[12]%
\>[12]\AgdaBound{δ} \AgdaFunction{⟦} \AgdaFunction{liftSub} \AgdaSymbol{\_} \AgdaBound{σ} \AgdaFunction{⟧} \AgdaFunction{〈} \AgdaFunction{liftRep} \AgdaSymbol{\_} \AgdaBound{ρ} \AgdaFunction{〉} \AgdaFunction{⟦} \AgdaFunction{x₀:=} \AgdaBound{ε} \AgdaFunction{⟧}\<%
\\
\>[0]\AgdaIndent{10}{}\<[10]%
\>[10]\AgdaFunction{∎}\AgdaSymbol{)} \<[13]%
\>[13]\<%
\\
\>[0]\AgdaIndent{10}{}\<[10]%
\>[10]\AgdaSymbol{(}\AgdaFunction{SNmainlemma} \AgdaBound{Γ,φ⊢δ∶ψ} \AgdaSymbol{(}\AgdaFunction{extend-sub} \AgdaSymbol{(}\AgdaFunction{compRSC} \AgdaSymbol{\{}\AgdaArgument{σ} \AgdaSymbol{=} \AgdaBound{σ}\AgdaSymbol{\}} \AgdaBound{ρ∶Δ⇒RΘ} \AgdaBound{σ∶Γ⇒CΔ}\AgdaSymbol{)} \AgdaBound{ε∈CΘφ}\AgdaSymbol{))} \AgdaKeyword{in}\<%
\\
\>[0]\AgdaIndent{8}{}\<[8]%
\>[8]\AgdaFunction{WTE} \AgdaSymbol{\{}\AgdaBound{R}\AgdaSymbol{\}} \AgdaSymbol{\{}\AgdaBound{Θ}\AgdaSymbol{\}} \AgdaSymbol{\{}\AgdaBound{φ}\AgdaSymbol{\}} \AgdaSymbol{\{}\AgdaBound{δ} \AgdaFunction{⟦} \AgdaFunction{liftSub} \AgdaSymbol{\_} \AgdaBound{σ} \AgdaFunction{⟧} \AgdaFunction{〈} \AgdaFunction{liftRep} \AgdaSymbol{\_} \AgdaBound{ρ} \AgdaFunction{〉}\AgdaSymbol{\}} \AgdaSymbol{\{}\AgdaBound{ψ}\AgdaSymbol{\}} \AgdaSymbol{\{}\AgdaBound{ε}\AgdaSymbol{\}}\<%
\\
\>[0]\AgdaIndent{8}{}\<[8]%
\>[8]\AgdaSymbol{(}\AgdaFunction{weakening} \AgdaSymbol{(}\AgdaFunction{substitution} \AgdaBound{Γ,φ⊢δ∶ψ} \AgdaSymbol{(}\AgdaFunction{liftSub-typed} \AgdaSymbol{(}\AgdaFunction{SubC-typed} \AgdaSymbol{\{}\AgdaArgument{σ} \AgdaSymbol{=} \AgdaBound{σ}\AgdaSymbol{\}} \AgdaSymbol{\{}\AgdaBound{Γ}\AgdaSymbol{\}} \AgdaSymbol{\{}\AgdaBound{Δ}\AgdaSymbol{\}} \AgdaBound{σ∶Γ⇒CΔ}\AgdaSymbol{)))} \AgdaSymbol{(}\AgdaFunction{liftRep-typed} \AgdaBound{ρ∶Δ⇒RΘ}\AgdaSymbol{))} \<[118]%
\>[118]\<%
\\
\>[0]\AgdaIndent{8}{}\<[8]%
\>[8]\AgdaSymbol{(}\AgdaFunction{C-typed} \AgdaBound{ε∈CΘφ}\AgdaSymbol{)} \<[24]%
\>[24]\<%
\\
\>[0]\AgdaIndent{8}{}\<[8]%
\>[8]\AgdaBound{δσ[ε]∈CΘψ} \<[18]%
\>[18]\<%
\\
\>[0]\AgdaIndent{8}{}\<[8]%
\>[8]\AgdaSymbol{(}\AgdaFunction{CsubSN} \AgdaBound{φ} \AgdaBound{ε∈CΘφ}\AgdaSymbol{))}\<%
\end{code}
}

\begin{theorem}
Propositional Logic is strongly normalizing.
\end{theorem}

\begin{code}%
\>\AgdaFunction{Strong-Normalization} \AgdaSymbol{:} \AgdaSymbol{∀} \AgdaSymbol{\{}\AgdaBound{P}\AgdaSymbol{\}} \AgdaSymbol{\{}\AgdaBound{Γ} \AgdaSymbol{:} \AgdaDatatype{Context} \AgdaBound{P}\AgdaSymbol{\}} \AgdaSymbol{\{}\AgdaBound{δ}\AgdaSymbol{\}} \AgdaSymbol{\{}\AgdaBound{φ}\AgdaSymbol{\}} \AgdaSymbol{→} \AgdaBound{Γ} \AgdaDatatype{⊢} \AgdaBound{δ} \AgdaDatatype{∶} \AgdaBound{φ} \AgdaSymbol{→} \AgdaDatatype{SN} \AgdaBound{δ}\<%
\end{code}

\AgdaHide{
\begin{code}%
\>\AgdaFunction{Strong-Normalization} \AgdaSymbol{\{}\AgdaBound{P}\AgdaSymbol{\}} \AgdaSymbol{\{}\AgdaBound{Γ}\AgdaSymbol{\}} \AgdaSymbol{\{}\AgdaBound{δ}\AgdaSymbol{\}} \AgdaSymbol{\{}\AgdaBound{φ}\AgdaSymbol{\}} \AgdaBound{Γ⊢δ:φ} \AgdaSymbol{=} \AgdaFunction{subst} \AgdaDatatype{SN} \<[54]%
\>[54]\<%
\\
\>[0]\AgdaIndent{2}{}\<[2]%
\>[2]\AgdaFunction{sub-idOp} \<[11]%
\>[11]\<%
\\
\>[0]\AgdaIndent{2}{}\<[2]%
\>[2]\AgdaSymbol{(}\AgdaFunction{CsubSN} \AgdaSymbol{(}\AgdaBound{φ}\AgdaSymbol{)} \AgdaSymbol{\{}\AgdaBound{δ} \AgdaFunction{⟦} \AgdaFunction{idSub} \AgdaBound{P} \AgdaFunction{⟧}\AgdaSymbol{\}}\<%
\\
\>[0]\AgdaIndent{2}{}\<[2]%
\>[2]\AgdaSymbol{(}\AgdaFunction{SNmainlemma} \AgdaBound{Γ⊢δ:φ} \AgdaSymbol{(λ} \AgdaBound{x} \AgdaSymbol{→} \AgdaFunction{varC} \AgdaSymbol{\{}\AgdaArgument{x} \AgdaSymbol{=} \AgdaBound{x}\AgdaSymbol{\})))}\<%
\end{code}
}


\section{Predicative Higher-Order Propositional Logic}

\mode<beamer>{}

\mode<all>{\input{PHOPL.lagda}}
\mode<all>{\AgdaHide{
\begin{code}%
\>\AgdaKeyword{module} \AgdaModule{PL.Rules} \AgdaKeyword{where}\<%
\\
\>\AgdaKeyword{open} \AgdaKeyword{import} \AgdaModule{Data.Empty}\<%
\\
\>\AgdaKeyword{open} \AgdaKeyword{import} \AgdaModule{Prelims}\<%
\\
\>\AgdaKeyword{open} \AgdaKeyword{import} \AgdaModule{PL.Grammar}\<%
\\
\>\AgdaKeyword{open} \AgdaModule{PLgrammar}\<%
\\
\>\AgdaKeyword{open} \AgdaKeyword{import} \AgdaModule{Grammar} \AgdaFunction{Propositional-Logic}\<%
\\
\>\AgdaKeyword{open} \AgdaKeyword{import} \AgdaModule{Reduction} \AgdaFunction{Propositional-Logic} \AgdaDatatype{β}\<%
\end{code}
}

\subsection{Rules of Deduction}

The rules of deduction of the system are as follows.

\[ \infer[(p : \phi \in \Gamma)]{\Gamma \vdash p : \phi}{\Gamma \vald} \]

\[ \infer{\Gamma \vdash \delta \epsilon : \psi}{\Gamma \vdash \delta : \phi \rightarrow \psi}{\Gamma \vdash \epsilon : \phi} \]

\[ \infer{\Gamma \vdash \lambda p : \phi . \delta : \phi \rightarrow \psi}{\Gamma, p : \phi \vdash \delta : \psi} \]

\begin{code}%
\>\AgdaKeyword{infix} \AgdaNumber{10} \AgdaFixityOp{\_⊢\_∶\_}\<%
\\
\>\AgdaKeyword{data} \AgdaDatatype{\_⊢\_∶\_} \AgdaSymbol{:} \AgdaSymbol{∀} \AgdaSymbol{\{}\AgdaBound{P}\AgdaSymbol{\}} \AgdaSymbol{→} \AgdaDatatype{Context} \AgdaBound{P} \AgdaSymbol{→} \AgdaFunction{Proof} \AgdaBound{P} \AgdaSymbol{→} \AgdaDatatype{Prop} \AgdaSymbol{→} \AgdaPrimitiveType{Set} \AgdaKeyword{where}\<%
\\
\>[0]\AgdaIndent{2}{}\<[2]%
\>[2]\AgdaInductiveConstructor{var} \AgdaSymbol{:} \AgdaSymbol{∀} \AgdaSymbol{\{}\AgdaBound{P}\AgdaSymbol{\}} \AgdaSymbol{\{}\AgdaBound{Γ} \AgdaSymbol{:} \AgdaDatatype{Context} \AgdaBound{P}\AgdaSymbol{\}} \AgdaSymbol{(}\AgdaBound{p} \AgdaSymbol{:} \AgdaDatatype{Var} \AgdaBound{P} \AgdaInductiveConstructor{-proof}\AgdaSymbol{)} \AgdaSymbol{→} \<[51]%
\>[51]\<%
\\
\>[2]\AgdaIndent{4}{}\<[4]%
\>[4]\AgdaBound{Γ} \AgdaDatatype{⊢} \AgdaInductiveConstructor{var} \AgdaBound{p} \AgdaDatatype{∶} \AgdaFunction{unprp} \AgdaSymbol{(}\AgdaFunction{typeof} \AgdaBound{p} \AgdaBound{Γ}\AgdaSymbol{)}\<%
\\
\>[0]\AgdaIndent{2}{}\<[2]%
\>[2]\AgdaInductiveConstructor{app} \AgdaSymbol{:} \AgdaSymbol{∀} \AgdaSymbol{\{}\AgdaBound{P}\AgdaSymbol{\}} \AgdaSymbol{\{}\AgdaBound{Γ} \AgdaSymbol{:} \AgdaDatatype{Context} \AgdaBound{P}\AgdaSymbol{\}} \AgdaSymbol{\{}\AgdaBound{δ}\AgdaSymbol{\}} \AgdaSymbol{\{}\AgdaBound{ε}\AgdaSymbol{\}} \AgdaSymbol{\{}\AgdaBound{φ}\AgdaSymbol{\}} \AgdaSymbol{\{}\AgdaBound{ψ}\AgdaSymbol{\}} \AgdaSymbol{→} \<[48]%
\>[48]\<%
\\
\>[2]\AgdaIndent{4}{}\<[4]%
\>[4]\AgdaBound{Γ} \AgdaDatatype{⊢} \AgdaBound{δ} \AgdaDatatype{∶} \AgdaBound{φ} \AgdaInductiveConstructor{⇛} \AgdaBound{ψ} \AgdaSymbol{→} \AgdaBound{Γ} \AgdaDatatype{⊢} \AgdaBound{ε} \AgdaDatatype{∶} \AgdaBound{φ} \AgdaSymbol{→} \AgdaBound{Γ} \AgdaDatatype{⊢} \AgdaFunction{appP} \AgdaBound{δ} \AgdaBound{ε} \AgdaDatatype{∶} \AgdaBound{ψ}\<%
\\
\>[0]\AgdaIndent{2}{}\<[2]%
\>[2]\AgdaInductiveConstructor{Λ} \AgdaSymbol{:} \AgdaSymbol{∀} \AgdaSymbol{\{}\AgdaBound{P}\AgdaSymbol{\}} \AgdaSymbol{\{}\AgdaBound{Γ} \AgdaSymbol{:} \AgdaDatatype{Context} \AgdaBound{P}\AgdaSymbol{\}} \AgdaSymbol{\{}\AgdaBound{φ}\AgdaSymbol{\}} \AgdaSymbol{\{}\AgdaBound{δ}\AgdaSymbol{\}} \AgdaSymbol{\{}\AgdaBound{ψ}\AgdaSymbol{\}} \AgdaSymbol{→} \<[42]%
\>[42]\<%
\\
\>[2]\AgdaIndent{4}{}\<[4]%
\>[4]\AgdaBound{Γ} \AgdaFunction{,P} \AgdaBound{φ} \AgdaDatatype{⊢} \AgdaBound{δ} \AgdaDatatype{∶} \AgdaBound{ψ} \AgdaSymbol{→} \AgdaBound{Γ} \AgdaDatatype{⊢} \AgdaFunction{ΛP} \AgdaBound{φ} \AgdaBound{δ} \AgdaDatatype{∶} \AgdaBound{φ} \AgdaInductiveConstructor{⇛} \AgdaBound{ψ}\<%
\end{code}

\AgdaHide{
\begin{code}%
\>\AgdaFunction{change-type} \AgdaSymbol{:} \AgdaSymbol{∀} \AgdaSymbol{\{}\AgdaBound{P}\AgdaSymbol{\}} \AgdaSymbol{\{}\AgdaBound{Γ} \AgdaSymbol{:} \AgdaDatatype{Context} \AgdaBound{P}\AgdaSymbol{\}} \AgdaSymbol{\{}\AgdaBound{δ} \AgdaBound{φ} \AgdaBound{ψ}\AgdaSymbol{\}} \AgdaSymbol{→}\<%
\\
\>[0]\AgdaIndent{2}{}\<[2]%
\>[2]\AgdaBound{φ} \AgdaDatatype{≡} \AgdaBound{ψ} \AgdaSymbol{→} \AgdaBound{Γ} \AgdaDatatype{⊢} \AgdaBound{δ} \AgdaDatatype{∶} \AgdaBound{φ} \AgdaSymbol{→} \AgdaBound{Γ} \AgdaDatatype{⊢} \AgdaBound{δ} \AgdaDatatype{∶} \AgdaBound{ψ}\<%
\\
\>\AgdaFunction{change-type} \AgdaSymbol{=} \AgdaFunction{subst} \AgdaSymbol{(λ} \AgdaBound{A} \AgdaSymbol{→} \AgdaSymbol{\_} \AgdaDatatype{⊢} \AgdaSymbol{\_} \AgdaDatatype{∶} \AgdaBound{A}\AgdaSymbol{)}\<%
\end{code}
}

Let $\rho$ be a replacement.  We say $\rho$ is a replacement from $\Gamma$ to $\Delta$, $\rho : \Gamma \rightarrow \Delta$,
iff for all $x : \phi \in \Gamma$ we have $\rho(x) : \phi \in \Delta$.

\begin{code}%
\>\AgdaFunction{\_∶\_⇒R\_} \AgdaSymbol{:} \AgdaSymbol{∀} \AgdaSymbol{\{}\AgdaBound{P}\AgdaSymbol{\}} \AgdaSymbol{\{}\AgdaBound{Q}\AgdaSymbol{\}} \AgdaSymbol{→} \AgdaFunction{Rep} \AgdaBound{P} \AgdaBound{Q} \AgdaSymbol{→} \AgdaDatatype{Context} \AgdaBound{P} \AgdaSymbol{→} \AgdaDatatype{Context} \AgdaBound{Q} \AgdaSymbol{→} \AgdaPrimitiveType{Set}\<%
\\
\>\AgdaBound{ρ} \AgdaFunction{∶} \AgdaBound{Γ} \AgdaFunction{⇒R} \AgdaBound{Δ} \AgdaSymbol{=} \AgdaSymbol{∀} \AgdaBound{x} \AgdaSymbol{→} \AgdaFunction{unprp} \AgdaSymbol{(}\AgdaFunction{typeof} \AgdaSymbol{\{}\AgdaArgument{K} \AgdaSymbol{=} \AgdaInductiveConstructor{-proof}\AgdaSymbol{\}} \AgdaSymbol{(}\AgdaBound{ρ} \AgdaSymbol{\_} \AgdaBound{x}\AgdaSymbol{)} \AgdaBound{Δ}\AgdaSymbol{)} \AgdaDatatype{≡} \AgdaFunction{unprp} \AgdaSymbol{(}\AgdaFunction{typeof} \AgdaBound{x} \AgdaBound{Γ} \AgdaSymbol{)}\<%
\end{code}

\begin{lemma}$ $
\begin{enumerate}
\item
$\id{P}$ is a replacement $\Gamma \rightarrow \Gamma$.
\item
$\uparrow$ is a replacement $\Gamma \rightarrow \Gamma , \phi$.
\item
If $\rho : \Gamma \rightarrow \Delta$ then $(\rho , \mathrm{Proof}) : (\Gamma , x : \phi) \rightarrow (\Delta , x : \phi)$.
\item
If $\rho : \Gamma \rightarrow \Delta$ and $\sigma : \Delta \rightarrow \Theta$ then $\sigma \circ \rho : \Gamma \rightarrow \Delta$.
\item
(\textbf{Weakening})
If $\rho : \Gamma \rightarrow \Delta$ and $\Gamma \vdash \delta : \phi$ then $\Delta \vdash \delta \langle \rho \rangle : \phi$.
\end{enumerate}
\end{lemma}

\begin{code}%
\>\AgdaFunction{idRep-typed} \AgdaSymbol{:} \AgdaSymbol{∀} \AgdaSymbol{\{}\AgdaBound{P}\AgdaSymbol{\}} \AgdaSymbol{\{}\AgdaBound{Γ} \AgdaSymbol{:} \AgdaDatatype{Context} \AgdaBound{P}\AgdaSymbol{\}} \AgdaSymbol{→} \AgdaFunction{idRep} \AgdaBound{P} \AgdaFunction{∶} \AgdaBound{Γ} \AgdaFunction{⇒R} \AgdaBound{Γ}\<%
\end{code}

\AgdaHide{
\begin{code}%
\>\AgdaFunction{idRep-typed} \AgdaSymbol{\{}\AgdaBound{P}\AgdaSymbol{\}} \AgdaSymbol{\{}\AgdaBound{Γ}\AgdaSymbol{\}} \AgdaBound{x} \AgdaSymbol{=} \AgdaInductiveConstructor{refl}\<%
\end{code}
}

\begin{code}%
\>\AgdaFunction{unprp-rep} \AgdaSymbol{:} \AgdaSymbol{∀} \AgdaSymbol{\{}\AgdaBound{U} \AgdaBound{V}\AgdaSymbol{\}} \AgdaBound{φ} \AgdaSymbol{(}\AgdaBound{ρ} \AgdaSymbol{:} \AgdaFunction{Rep} \AgdaBound{U} \AgdaBound{V}\AgdaSymbol{)} \AgdaSymbol{→} \AgdaFunction{unprp} \AgdaSymbol{(}\AgdaBound{φ} \AgdaFunction{〈} \AgdaBound{ρ} \AgdaFunction{〉}\AgdaSymbol{)} \AgdaDatatype{≡} \AgdaFunction{unprp} \AgdaBound{φ}\<%
\\
\>\AgdaFunction{unprp-rep} \AgdaSymbol{(}\AgdaInductiveConstructor{app} \AgdaSymbol{(}\AgdaInductiveConstructor{-prp} \AgdaSymbol{\_)} \AgdaInductiveConstructor{[]}\AgdaSymbol{)} \AgdaSymbol{\_} \AgdaSymbol{=} \AgdaInductiveConstructor{refl}\<%
\\
%
\\
\>\AgdaFunction{↑-typed} \AgdaSymbol{:} \AgdaSymbol{∀} \AgdaSymbol{\{}\AgdaBound{P}\AgdaSymbol{\}} \AgdaSymbol{\{}\AgdaBound{Γ} \AgdaSymbol{:} \AgdaDatatype{Context} \AgdaBound{P}\AgdaSymbol{\}} \AgdaSymbol{\{}\AgdaBound{φ} \AgdaSymbol{:} \AgdaDatatype{Prop}\AgdaSymbol{\}} \AgdaSymbol{→} \AgdaFunction{upRep} \AgdaFunction{∶} \AgdaBound{Γ} \AgdaFunction{⇒R} \AgdaSymbol{(}\AgdaBound{Γ} \AgdaFunction{,P} \AgdaBound{φ}\AgdaSymbol{)}\<%
\end{code}

\AgdaHide{
\begin{code}%
\>\AgdaFunction{↑-typed} \AgdaSymbol{\{}\AgdaBound{P}\AgdaSymbol{\}} \AgdaSymbol{\{}\AgdaBound{Γ}\AgdaSymbol{\}} \AgdaSymbol{\{}\AgdaBound{φ}\AgdaSymbol{\}} \AgdaBound{x} \AgdaSymbol{=} \AgdaFunction{unprp-rep} \AgdaSymbol{(}\AgdaFunction{typeof} \AgdaBound{x} \AgdaBound{Γ}\AgdaSymbol{)} \AgdaFunction{upRep}\<%
\end{code}
}

\begin{code}%
\>\AgdaFunction{liftRep-typed} \AgdaSymbol{:} \AgdaSymbol{∀} \AgdaSymbol{\{}\AgdaBound{P}\AgdaSymbol{\}} \AgdaSymbol{\{}\AgdaBound{Q}\AgdaSymbol{\}} \AgdaSymbol{\{}\AgdaBound{ρ}\AgdaSymbol{\}} \AgdaSymbol{\{}\AgdaBound{Γ} \AgdaSymbol{:} \AgdaDatatype{Context} \AgdaBound{P}\AgdaSymbol{\}} \AgdaSymbol{\{}\AgdaBound{Δ} \AgdaSymbol{:} \AgdaDatatype{Context} \AgdaBound{Q}\AgdaSymbol{\}} \AgdaSymbol{\{}\AgdaBound{φ} \AgdaSymbol{:} \AgdaDatatype{Prop}\AgdaSymbol{\}} \AgdaSymbol{→} \<[75]%
\>[75]\<%
\\
\>[0]\AgdaIndent{2}{}\<[2]%
\>[2]\AgdaBound{ρ} \AgdaFunction{∶} \AgdaBound{Γ} \AgdaFunction{⇒R} \AgdaBound{Δ} \AgdaSymbol{→} \AgdaFunction{liftRep} \AgdaInductiveConstructor{-proof} \AgdaBound{ρ} \AgdaFunction{∶} \AgdaSymbol{(}\AgdaBound{Γ} \AgdaFunction{,P} \AgdaBound{φ}\AgdaSymbol{)} \AgdaFunction{⇒R} \AgdaSymbol{(}\AgdaBound{Δ} \AgdaFunction{,P} \AgdaBound{φ}\AgdaSymbol{)}\<%
\end{code}

\AgdaHide{
\begin{code}%
\>\AgdaFunction{liftRep-typed} \AgdaSymbol{\{}\AgdaBound{P}\AgdaSymbol{\}} \AgdaSymbol{\{}\AgdaArgument{Q} \AgdaSymbol{=} \AgdaBound{Q}\AgdaSymbol{\}} \AgdaSymbol{\{}\AgdaArgument{ρ} \AgdaSymbol{=} \AgdaBound{ρ}\AgdaSymbol{\}} \AgdaSymbol{\{}\AgdaBound{Γ}\AgdaSymbol{\}} \AgdaSymbol{\{}\AgdaArgument{Δ} \AgdaSymbol{=} \AgdaBound{Δ}\AgdaSymbol{\}} \AgdaSymbol{\{}\AgdaArgument{φ} \AgdaSymbol{=} \AgdaBound{φ}\AgdaSymbol{\}} \AgdaBound{ρ∶Γ→Δ} \AgdaInductiveConstructor{x₀} \AgdaSymbol{=} \AgdaInductiveConstructor{refl}\<%
\\
\>\AgdaFunction{liftRep-typed} \AgdaSymbol{\{}\AgdaArgument{Q} \AgdaSymbol{=} \AgdaBound{Q}\AgdaSymbol{\}} \AgdaSymbol{\{}\AgdaArgument{ρ} \AgdaSymbol{=} \AgdaBound{ρ}\AgdaSymbol{\}} \AgdaSymbol{\{}\AgdaArgument{Γ} \AgdaSymbol{=} \AgdaBound{Γ}\AgdaSymbol{\}} \AgdaSymbol{\{}\AgdaArgument{Δ} \AgdaSymbol{=} \AgdaBound{Δ}\AgdaSymbol{\}} \AgdaSymbol{\{}\AgdaBound{φ}\AgdaSymbol{\}} \AgdaBound{ρ∶Γ→Δ} \AgdaSymbol{(}\AgdaInductiveConstructor{↑} \AgdaBound{x}\AgdaSymbol{)} \AgdaSymbol{=} \<[64]%
\>[64]\<%
\\
\>[0]\AgdaIndent{2}{}\<[2]%
\>[2]\AgdaKeyword{let} \AgdaKeyword{open} \AgdaModule{≡-Reasoning} \AgdaKeyword{in} \<[26]%
\>[26]\<%
\\
\>[0]\AgdaIndent{2}{}\<[2]%
\>[2]\AgdaFunction{begin}\<%
\\
\>[2]\AgdaIndent{4}{}\<[4]%
\>[4]\AgdaFunction{unprp} \AgdaSymbol{(}\AgdaFunction{typeof} \AgdaSymbol{(}\AgdaFunction{liftRep} \AgdaInductiveConstructor{-proof} \AgdaBound{ρ} \AgdaInductiveConstructor{-proof} \AgdaSymbol{(}\AgdaInductiveConstructor{↑} \AgdaBound{x}\AgdaSymbol{))} \AgdaSymbol{(}\AgdaBound{Δ} \AgdaFunction{,P} \AgdaBound{φ}\AgdaSymbol{))}\<%
\\
\>[0]\AgdaIndent{2}{}\<[2]%
\>[2]\AgdaFunction{≡⟨⟩}\<%
\\
\>[2]\AgdaIndent{4}{}\<[4]%
\>[4]\AgdaFunction{unprp} \AgdaSymbol{(}\AgdaFunction{typeof} \AgdaSymbol{(}\AgdaInductiveConstructor{↑} \AgdaSymbol{(}\AgdaBound{ρ} \AgdaInductiveConstructor{-proof} \AgdaBound{x}\AgdaSymbol{))} \AgdaSymbol{(}\AgdaBound{Δ} \AgdaFunction{,P} \AgdaBound{φ}\AgdaSymbol{))}\<%
\\
\>[0]\AgdaIndent{2}{}\<[2]%
\>[2]\AgdaFunction{≡⟨⟩}\<%
\\
\>[2]\AgdaIndent{4}{}\<[4]%
\>[4]\AgdaFunction{unprp} \AgdaSymbol{(}\AgdaFunction{typeof} \AgdaSymbol{(}\AgdaBound{ρ} \AgdaInductiveConstructor{-proof} \AgdaBound{x}\AgdaSymbol{)} \AgdaBound{Δ} \AgdaFunction{〈} \AgdaFunction{upRep} \AgdaFunction{〉}\AgdaSymbol{)}\<%
\\
\>[0]\AgdaIndent{2}{}\<[2]%
\>[2]\AgdaFunction{≡⟨} \AgdaFunction{unprp-rep} \AgdaSymbol{(}\AgdaFunction{typeof} \AgdaSymbol{(}\AgdaBound{ρ} \AgdaInductiveConstructor{-proof} \AgdaBound{x}\AgdaSymbol{)} \AgdaBound{Δ}\AgdaSymbol{)} \AgdaFunction{upRep} \AgdaFunction{⟩}\<%
\\
\>[2]\AgdaIndent{4}{}\<[4]%
\>[4]\AgdaFunction{unprp} \AgdaSymbol{(}\AgdaFunction{typeof} \AgdaSymbol{(}\AgdaBound{ρ} \AgdaInductiveConstructor{-proof} \AgdaBound{x}\AgdaSymbol{)} \AgdaBound{Δ}\AgdaSymbol{)}\<%
\\
\>[0]\AgdaIndent{2}{}\<[2]%
\>[2]\AgdaFunction{≡⟨} \AgdaBound{ρ∶Γ→Δ} \AgdaBound{x} \AgdaFunction{⟩}\<%
\\
\>[2]\AgdaIndent{4}{}\<[4]%
\>[4]\AgdaFunction{unprp} \AgdaSymbol{(}\AgdaFunction{typeof} \AgdaBound{x} \AgdaBound{Γ}\AgdaSymbol{)}\<%
\\
\>[0]\AgdaIndent{2}{}\<[2]%
\>[2]\AgdaFunction{≡⟨⟨} \AgdaFunction{unprp-rep} \AgdaSymbol{(}\AgdaFunction{typeof} \AgdaBound{x} \AgdaBound{Γ}\AgdaSymbol{)} \AgdaFunction{upRep} \AgdaFunction{⟩⟩}\<%
\\
\>[2]\AgdaIndent{4}{}\<[4]%
\>[4]\AgdaFunction{unprp} \AgdaSymbol{(}\AgdaFunction{typeof} \AgdaBound{x} \AgdaBound{Γ} \AgdaFunction{〈} \AgdaFunction{upRep} \AgdaFunction{〉}\AgdaSymbol{)}\<%
\\
\>[0]\AgdaIndent{2}{}\<[2]%
\>[2]\AgdaFunction{≡⟨⟩}\<%
\\
\>[2]\AgdaIndent{4}{}\<[4]%
\>[4]\AgdaFunction{unprp} \AgdaSymbol{(}\AgdaFunction{typeof} \AgdaSymbol{(}\AgdaInductiveConstructor{↑} \AgdaBound{x}\AgdaSymbol{)} \AgdaSymbol{(}\AgdaBound{Γ} \AgdaFunction{,P} \AgdaBound{φ}\AgdaSymbol{))}\<%
\\
\>[0]\AgdaIndent{2}{}\<[2]%
\>[2]\AgdaFunction{∎}\<%
\end{code}
}

\begin{code}%
\>\AgdaFunction{•R-typed} \AgdaSymbol{:} \AgdaSymbol{∀} \AgdaSymbol{\{}\AgdaBound{P}\AgdaSymbol{\}} \AgdaSymbol{\{}\AgdaBound{Q}\AgdaSymbol{\}} \AgdaSymbol{\{}\AgdaBound{R}\AgdaSymbol{\}} \AgdaSymbol{\{}\AgdaBound{σ} \AgdaSymbol{:} \AgdaFunction{Rep} \AgdaBound{Q} \AgdaBound{R}\AgdaSymbol{\}} \AgdaSymbol{\{}\AgdaBound{ρ} \AgdaSymbol{:} \AgdaFunction{Rep} \AgdaBound{P} \AgdaBound{Q}\AgdaSymbol{\}} \AgdaSymbol{\{}\AgdaBound{Γ}\AgdaSymbol{\}} \AgdaSymbol{\{}\AgdaBound{Δ}\AgdaSymbol{\}} \AgdaSymbol{\{}\AgdaBound{Θ}\AgdaSymbol{\}} \AgdaSymbol{→} \<[67]%
\>[67]\<%
\\
\>[0]\AgdaIndent{2}{}\<[2]%
\>[2]\AgdaBound{ρ} \AgdaFunction{∶} \AgdaBound{Γ} \AgdaFunction{⇒R} \AgdaBound{Δ} \AgdaSymbol{→} \AgdaBound{σ} \AgdaFunction{∶} \AgdaBound{Δ} \AgdaFunction{⇒R} \AgdaBound{Θ} \AgdaSymbol{→} \AgdaSymbol{(}\AgdaBound{σ} \AgdaFunction{•R} \AgdaBound{ρ}\AgdaSymbol{)} \AgdaFunction{∶} \AgdaBound{Γ} \AgdaFunction{⇒R} \AgdaBound{Θ}\<%
\end{code}

\AgdaHide{
\begin{code}%
\>\AgdaFunction{•R-typed} \AgdaSymbol{\{}\AgdaArgument{R} \AgdaSymbol{=} \AgdaBound{R}\AgdaSymbol{\}} \AgdaSymbol{\{}\AgdaBound{σ}\AgdaSymbol{\}} \AgdaSymbol{\{}\AgdaBound{ρ}\AgdaSymbol{\}} \AgdaSymbol{\{}\AgdaBound{Γ}\AgdaSymbol{\}} \AgdaSymbol{\{}\AgdaBound{Δ}\AgdaSymbol{\}} \AgdaSymbol{\{}\AgdaBound{Θ}\AgdaSymbol{\}} \AgdaBound{ρ∶Γ→Δ} \AgdaBound{σ∶Δ→Θ} \AgdaBound{x} \AgdaSymbol{=} \AgdaKeyword{let} \AgdaKeyword{open} \AgdaModule{≡-Reasoning} \AgdaKeyword{in} \<[77]%
\>[77]\<%
\\
\>[0]\AgdaIndent{2}{}\<[2]%
\>[2]\AgdaFunction{begin} \<[8]%
\>[8]\<%
\\
\>[2]\AgdaIndent{4}{}\<[4]%
\>[4]\AgdaFunction{unprp} \AgdaSymbol{(}\AgdaFunction{typeof} \AgdaSymbol{(}\AgdaBound{σ} \AgdaInductiveConstructor{-proof} \AgdaSymbol{(}\AgdaBound{ρ} \AgdaInductiveConstructor{-proof} \AgdaBound{x}\AgdaSymbol{))} \AgdaBound{Θ}\AgdaSymbol{)}\<%
\\
\>[0]\AgdaIndent{2}{}\<[2]%
\>[2]\AgdaFunction{≡⟨} \AgdaBound{σ∶Δ→Θ} \AgdaSymbol{(}\AgdaBound{ρ} \AgdaInductiveConstructor{-proof} \AgdaBound{x}\AgdaSymbol{)} \AgdaFunction{⟩}\<%
\\
\>[2]\AgdaIndent{4}{}\<[4]%
\>[4]\AgdaFunction{unprp} \AgdaSymbol{(}\AgdaFunction{typeof} \AgdaSymbol{(}\AgdaBound{ρ} \AgdaInductiveConstructor{-proof} \AgdaBound{x}\AgdaSymbol{)} \AgdaBound{Δ}\AgdaSymbol{)}\<%
\\
\>[0]\AgdaIndent{2}{}\<[2]%
\>[2]\AgdaFunction{≡⟨} \AgdaBound{ρ∶Γ→Δ} \AgdaBound{x} \AgdaFunction{⟩}\<%
\\
\>[2]\AgdaIndent{4}{}\<[4]%
\>[4]\AgdaFunction{unprp} \AgdaSymbol{(}\AgdaFunction{typeof} \AgdaBound{x} \AgdaBound{Γ}\AgdaSymbol{)}\<%
\\
\>[0]\AgdaIndent{2}{}\<[2]%
\>[2]\AgdaFunction{∎}\<%
\end{code}
}

\begin{code}%
\>\AgdaFunction{weakening} \AgdaSymbol{:} \AgdaSymbol{∀} \AgdaSymbol{\{}\AgdaBound{P}\AgdaSymbol{\}} \AgdaSymbol{\{}\AgdaBound{Q}\AgdaSymbol{\}} \AgdaSymbol{\{}\AgdaBound{Γ} \AgdaSymbol{:} \AgdaDatatype{Context} \AgdaBound{P}\AgdaSymbol{\}} \AgdaSymbol{\{}\AgdaBound{Δ} \AgdaSymbol{:} \AgdaDatatype{Context} \AgdaBound{Q}\AgdaSymbol{\}} \AgdaSymbol{\{}\AgdaBound{ρ}\AgdaSymbol{\}} \AgdaSymbol{\{}\AgdaBound{δ}\AgdaSymbol{\}} \AgdaSymbol{\{}\AgdaBound{φ}\AgdaSymbol{\}} \AgdaSymbol{→} \<[68]%
\>[68]\<%
\\
\>[0]\AgdaIndent{2}{}\<[2]%
\>[2]\AgdaBound{Γ} \AgdaDatatype{⊢} \AgdaBound{δ} \AgdaDatatype{∶} \AgdaBound{φ} \AgdaSymbol{→} \AgdaBound{ρ} \AgdaFunction{∶} \AgdaBound{Γ} \AgdaFunction{⇒R} \AgdaBound{Δ} \AgdaSymbol{→} \AgdaBound{Δ} \AgdaDatatype{⊢} \AgdaBound{δ} \AgdaFunction{〈} \AgdaBound{ρ} \AgdaFunction{〉} \AgdaDatatype{∶} \AgdaBound{φ}\<%
\end{code}

\AgdaHide{
\begin{code}%
\>\AgdaFunction{weakening} \AgdaSymbol{\{}\AgdaBound{P}\AgdaSymbol{\}} \AgdaSymbol{\{}\AgdaBound{Q}\AgdaSymbol{\}} \AgdaSymbol{\{}\AgdaBound{Γ}\AgdaSymbol{\}} \AgdaSymbol{\{}\AgdaBound{Δ}\AgdaSymbol{\}} \AgdaSymbol{\{}\AgdaBound{ρ}\AgdaSymbol{\}} \AgdaSymbol{(}\AgdaInductiveConstructor{var} \AgdaBound{p}\AgdaSymbol{)} \AgdaBound{ρ∶Γ→Δ} \AgdaSymbol{=} \AgdaFunction{change-type} \AgdaSymbol{(}\AgdaBound{ρ∶Γ→Δ} \AgdaBound{p}\AgdaSymbol{)} \AgdaSymbol{(}\AgdaInductiveConstructor{var} \AgdaSymbol{(}\AgdaBound{ρ} \AgdaSymbol{\_} \AgdaBound{p}\AgdaSymbol{))}\<%
\\
\>\AgdaFunction{weakening} \AgdaSymbol{(}\AgdaInductiveConstructor{app} \AgdaBound{Γ⊢δ∶φ→ψ} \AgdaBound{Γ⊢ε∶φ}\AgdaSymbol{)} \AgdaBound{ρ∶Γ→Δ} \AgdaSymbol{=} \AgdaInductiveConstructor{app} \AgdaSymbol{(}\AgdaFunction{weakening} \AgdaBound{Γ⊢δ∶φ→ψ} \AgdaBound{ρ∶Γ→Δ}\AgdaSymbol{)} \AgdaSymbol{(}\AgdaFunction{weakening} \AgdaBound{Γ⊢ε∶φ} \AgdaBound{ρ∶Γ→Δ}\AgdaSymbol{)}\<%
\\
\>\AgdaFunction{weakening} \AgdaSymbol{.\{}\AgdaBound{P}\AgdaSymbol{\}} \AgdaSymbol{\{}\AgdaBound{Q}\AgdaSymbol{\}} \AgdaSymbol{.\{}\AgdaBound{Γ}\AgdaSymbol{\}} \AgdaSymbol{\{}\AgdaBound{Δ}\AgdaSymbol{\}} \AgdaSymbol{\{}\AgdaBound{ρ}\AgdaSymbol{\}} \AgdaSymbol{(}\AgdaInductiveConstructor{Λ} \AgdaSymbol{\{}\AgdaBound{P}\AgdaSymbol{\}} \AgdaSymbol{\{}\AgdaBound{Γ}\AgdaSymbol{\}} \AgdaSymbol{\{}\AgdaBound{φ}\AgdaSymbol{\}} \AgdaSymbol{\{}\AgdaBound{δ}\AgdaSymbol{\}} \AgdaSymbol{\{}\AgdaBound{ψ}\AgdaSymbol{\}} \AgdaBound{Γ,φ⊢δ∶ψ}\AgdaSymbol{)} \AgdaBound{ρ∶Γ→Δ} \AgdaSymbol{=} \AgdaInductiveConstructor{Λ} \<[74]%
\>[74]\<%
\\
\>[0]\AgdaIndent{2}{}\<[2]%
\>[2]\AgdaSymbol{(}\AgdaFunction{weakening} \AgdaSymbol{\{}\AgdaBound{P} \AgdaInductiveConstructor{,} \AgdaInductiveConstructor{-proof}\AgdaSymbol{\}} \AgdaSymbol{\{}\AgdaBound{Q} \AgdaInductiveConstructor{,} \AgdaInductiveConstructor{-proof}\AgdaSymbol{\}} \AgdaSymbol{\{}\AgdaBound{Γ} \AgdaFunction{,P} \AgdaBound{φ}\AgdaSymbol{\}} \AgdaSymbol{\{}\AgdaBound{Δ} \AgdaFunction{,P} \AgdaBound{φ}\AgdaSymbol{\}} \AgdaSymbol{\{}\AgdaFunction{liftRep} \AgdaInductiveConstructor{-proof} \AgdaBound{ρ}\AgdaSymbol{\}} \AgdaSymbol{\{}\AgdaBound{δ}\AgdaSymbol{\}} \AgdaSymbol{\{}\AgdaBound{ψ}\AgdaSymbol{\}} \<[84]%
\>[84]\<%
\\
\>[2]\AgdaIndent{4}{}\<[4]%
\>[4]\AgdaBound{Γ,φ⊢δ∶ψ} \AgdaSymbol{(}\AgdaFunction{liftRep-typed} \AgdaBound{ρ∶Γ→Δ}\AgdaSymbol{))}\<%
\end{code}
}
A \emph{substitution} $\sigma$ from a context $\Gamma$ to a context $\Delta$, $\sigma : \Gamma \rightarrow \Delta$,  is a substitution $\sigma$ such that
for every $x : \phi$ in $\Gamma$, we have $\Delta \vdash \sigma(x) : \phi$.

\begin{code}%
\>\AgdaFunction{\_∶\_⇒\_} \AgdaSymbol{:} \AgdaSymbol{∀} \AgdaSymbol{\{}\AgdaBound{P}\AgdaSymbol{\}} \AgdaSymbol{\{}\AgdaBound{Q}\AgdaSymbol{\}} \AgdaSymbol{→} \AgdaFunction{Sub} \AgdaBound{P} \AgdaBound{Q} \AgdaSymbol{→} \AgdaDatatype{Context} \AgdaBound{P} \AgdaSymbol{→} \AgdaDatatype{Context} \AgdaBound{Q} \AgdaSymbol{→} \AgdaPrimitiveType{Set}\<%
\\
\>\AgdaBound{σ} \AgdaFunction{∶} \AgdaBound{Γ} \AgdaFunction{⇒} \AgdaBound{Δ} \AgdaSymbol{=} \AgdaSymbol{∀} \AgdaBound{x} \AgdaSymbol{→} \AgdaBound{Δ} \AgdaDatatype{⊢} \AgdaBound{σ} \AgdaSymbol{\_} \AgdaBound{x} \AgdaDatatype{∶} \AgdaFunction{unprp} \AgdaSymbol{(}\AgdaFunction{typeof} \AgdaBound{x} \AgdaBound{Γ}\AgdaSymbol{)}\<%
\end{code}

\begin{lemma}$ $
\begin{enumerate}
\item
If $\sigma : \Gamma \rightarrow \Delta$ then $(\sigma , \mathrm{Proof}) : (\Gamma , p : \phi) \rightarrow (\Delta , p : \phi [ \sigma ])$.
\item
If $\Gamma \vdash \delta : \phi$ then $(p := \delta) : (\Gamma, p : \phi) \rightarrow \Gamma$.
\item
(\textbf{substitution Lemma})

If $\Gamma \vdash \delta : \phi$ and $\sigma : \Gamma \rightarrow \Delta$ then $\Delta \vdash \delta [ \sigma ] : \phi [ \sigma ]$.
\end{enumerate}
\end{lemma}

\begin{code}%
\>\AgdaFunction{liftSub-typed} \AgdaSymbol{:} \AgdaSymbol{∀} \AgdaSymbol{\{}\AgdaBound{P}\AgdaSymbol{\}} \AgdaSymbol{\{}\AgdaBound{Q}\AgdaSymbol{\}} \AgdaSymbol{\{}\AgdaBound{σ}\AgdaSymbol{\}} \<[30]%
\>[30]\<%
\\
\>[0]\AgdaIndent{2}{}\<[2]%
\>[2]\AgdaSymbol{\{}\AgdaBound{Γ} \AgdaSymbol{:} \AgdaDatatype{Context} \AgdaBound{P}\AgdaSymbol{\}} \AgdaSymbol{\{}\AgdaBound{Δ} \AgdaSymbol{:} \AgdaDatatype{Context} \AgdaBound{Q}\AgdaSymbol{\}} \AgdaSymbol{\{}\AgdaBound{φ} \AgdaSymbol{:} \AgdaDatatype{Prop}\AgdaSymbol{\}} \AgdaSymbol{→} \<[47]%
\>[47]\<%
\\
\>[0]\AgdaIndent{2}{}\<[2]%
\>[2]\AgdaBound{σ} \AgdaFunction{∶} \AgdaBound{Γ} \AgdaFunction{⇒} \AgdaBound{Δ} \AgdaSymbol{→} \AgdaFunction{liftSub} \AgdaInductiveConstructor{-proof} \AgdaBound{σ} \AgdaFunction{∶} \AgdaSymbol{(}\AgdaBound{Γ} \AgdaFunction{,P} \AgdaBound{φ}\AgdaSymbol{)} \AgdaFunction{⇒} \AgdaSymbol{(}\AgdaBound{Δ} \AgdaFunction{,P} \AgdaBound{φ}\AgdaSymbol{)}\<%
\end{code}

\AgdaHide{
\begin{code}%
\>\AgdaFunction{liftSub-typed} \AgdaSymbol{\{}\AgdaArgument{σ} \AgdaSymbol{=} \AgdaBound{σ}\AgdaSymbol{\}} \AgdaSymbol{\{}\AgdaBound{Γ}\AgdaSymbol{\}} \AgdaSymbol{\{}\AgdaBound{Δ}\AgdaSymbol{\}} \AgdaSymbol{\{}\AgdaBound{φ}\AgdaSymbol{\}} \AgdaBound{σ∶Γ⇒Δ} \AgdaBound{x} \AgdaSymbol{=}\<%
\\
\>[0]\AgdaIndent{2}{}\<[2]%
\>[2]\AgdaFunction{change-type} \AgdaSymbol{(}\AgdaFunction{sym} \AgdaSymbol{(}\AgdaFunction{unprp-rep} \AgdaSymbol{(}\AgdaFunction{pretypeof} \AgdaBound{x} \AgdaSymbol{(}\AgdaBound{Γ} \AgdaFunction{,P} \AgdaBound{φ}\AgdaSymbol{))} \AgdaFunction{upRep}\AgdaSymbol{))} \AgdaSymbol{(}\AgdaFunction{pre-LiftSub-typed} \AgdaBound{x}\AgdaSymbol{)} \AgdaKeyword{where}\<%
\\
\>[0]\AgdaIndent{2}{}\<[2]%
\>[2]\AgdaFunction{pre-LiftSub-typed} \AgdaSymbol{:} \AgdaSymbol{∀} \AgdaBound{x} \AgdaSymbol{→} \AgdaBound{Δ} \AgdaFunction{,P} \AgdaBound{φ} \AgdaDatatype{⊢} \AgdaFunction{liftSub} \AgdaInductiveConstructor{-proof} \AgdaBound{σ} \AgdaInductiveConstructor{-proof} \AgdaBound{x} \AgdaDatatype{∶} \AgdaFunction{unprp} \AgdaSymbol{(}\AgdaFunction{pretypeof} \AgdaBound{x} \AgdaSymbol{(}\AgdaBound{Γ} \AgdaFunction{,P} \AgdaBound{φ}\AgdaSymbol{))}\<%
\\
\>[0]\AgdaIndent{2}{}\<[2]%
\>[2]\AgdaFunction{pre-LiftSub-typed} \AgdaInductiveConstructor{x₀} \AgdaSymbol{=} \AgdaInductiveConstructor{var} \AgdaInductiveConstructor{x₀}\<%
\\
\>[0]\AgdaIndent{2}{}\<[2]%
\>[2]\AgdaFunction{pre-LiftSub-typed} \AgdaSymbol{(}\AgdaInductiveConstructor{↑} \AgdaBound{x}\AgdaSymbol{)} \AgdaSymbol{=} \AgdaFunction{weakening} \AgdaSymbol{(}\AgdaBound{σ∶Γ⇒Δ} \AgdaBound{x}\AgdaSymbol{)} \AgdaSymbol{(}\AgdaFunction{↑-typed} \AgdaSymbol{\{}\AgdaArgument{φ} \AgdaSymbol{=} \AgdaBound{φ}\AgdaSymbol{\})}\<%
\end{code}
}

\begin{code}%
\>\AgdaFunction{botSub-typed} \AgdaSymbol{:} \AgdaSymbol{∀} \AgdaSymbol{\{}\AgdaBound{P}\AgdaSymbol{\}} \AgdaSymbol{\{}\AgdaBound{Γ} \AgdaSymbol{:} \AgdaDatatype{Context} \AgdaBound{P}\AgdaSymbol{\}} \AgdaSymbol{\{}\AgdaBound{φ} \AgdaSymbol{:} \AgdaDatatype{Prop}\AgdaSymbol{\}} \AgdaSymbol{\{}\AgdaBound{δ}\AgdaSymbol{\}} \AgdaSymbol{→}\<%
\\
\>[0]\AgdaIndent{2}{}\<[2]%
\>[2]\AgdaBound{Γ} \AgdaDatatype{⊢} \AgdaBound{δ} \AgdaDatatype{∶} \AgdaBound{φ} \AgdaSymbol{→} \AgdaFunction{x₀:=} \AgdaBound{δ} \AgdaFunction{∶} \AgdaSymbol{(}\AgdaBound{Γ} \AgdaFunction{,P} \AgdaBound{φ}\AgdaSymbol{)} \AgdaFunction{⇒} \AgdaBound{Γ}\<%
\end{code}

\AgdaHide{
\begin{code}%
\>\AgdaFunction{botSub-typed} \AgdaSymbol{\{}\AgdaBound{P}\AgdaSymbol{\}} \AgdaSymbol{\{}\AgdaBound{Γ}\AgdaSymbol{\}} \AgdaSymbol{\{}\AgdaBound{φ}\AgdaSymbol{\}} \AgdaSymbol{\{}\AgdaBound{δ}\AgdaSymbol{\}} \AgdaBound{Γ⊢δ:φ} \AgdaBound{x} \AgdaSymbol{=} \<[39]%
\>[39]\<%
\\
\>[0]\AgdaIndent{2}{}\<[2]%
\>[2]\AgdaFunction{change-type} \AgdaSymbol{(}\AgdaFunction{sym} \AgdaSymbol{(}\AgdaFunction{unprp-rep} \AgdaSymbol{(}\AgdaFunction{pretypeof} \AgdaBound{x} \AgdaSymbol{(}\AgdaBound{Γ} \AgdaFunction{,P} \AgdaBound{φ}\AgdaSymbol{))} \AgdaFunction{upRep}\AgdaSymbol{))} \AgdaSymbol{(}\AgdaFunction{pre-botSub-typed} \AgdaBound{x}\AgdaSymbol{)} \AgdaKeyword{where}\<%
\\
\>[0]\AgdaIndent{2}{}\<[2]%
\>[2]\AgdaFunction{pre-botSub-typed} \AgdaSymbol{:} \AgdaSymbol{∀} \AgdaBound{x} \AgdaSymbol{→} \AgdaBound{Γ} \AgdaDatatype{⊢} \AgdaSymbol{(}\AgdaFunction{x₀:=} \AgdaBound{δ}\AgdaSymbol{)} \AgdaInductiveConstructor{-proof} \AgdaBound{x} \AgdaDatatype{∶} \AgdaFunction{unprp} \AgdaSymbol{(}\AgdaFunction{pretypeof} \AgdaBound{x} \AgdaSymbol{(}\AgdaBound{Γ} \AgdaFunction{,P} \AgdaBound{φ}\AgdaSymbol{))}\<%
\\
\>[0]\AgdaIndent{2}{}\<[2]%
\>[2]\AgdaFunction{pre-botSub-typed} \AgdaInductiveConstructor{x₀} \AgdaSymbol{=} \AgdaBound{Γ⊢δ:φ}\<%
\\
\>[0]\AgdaIndent{2}{}\<[2]%
\>[2]\AgdaFunction{pre-botSub-typed} \AgdaSymbol{(}\AgdaInductiveConstructor{↑} \AgdaBound{x}\AgdaSymbol{)} \AgdaSymbol{=} \AgdaInductiveConstructor{var} \AgdaBound{x}\<%
\end{code}
}

\begin{code}%
\>\AgdaFunction{substitution} \AgdaSymbol{:} \AgdaSymbol{∀} \AgdaSymbol{\{}\AgdaBound{P}\AgdaSymbol{\}} \AgdaSymbol{\{}\AgdaBound{Q}\AgdaSymbol{\}}\<%
\\
\>[0]\AgdaIndent{2}{}\<[2]%
\>[2]\AgdaSymbol{\{}\AgdaBound{Γ} \AgdaSymbol{:} \AgdaDatatype{Context} \AgdaBound{P}\AgdaSymbol{\}} \AgdaSymbol{\{}\AgdaBound{Δ} \AgdaSymbol{:} \AgdaDatatype{Context} \AgdaBound{Q}\AgdaSymbol{\}} \AgdaSymbol{\{}\AgdaBound{δ}\AgdaSymbol{\}} \AgdaSymbol{\{}\AgdaBound{φ}\AgdaSymbol{\}} \AgdaSymbol{\{}\AgdaBound{σ}\AgdaSymbol{\}} \AgdaSymbol{→} \<[48]%
\>[48]\<%
\\
\>[0]\AgdaIndent{2}{}\<[2]%
\>[2]\AgdaBound{Γ} \AgdaDatatype{⊢} \AgdaBound{δ} \AgdaDatatype{∶} \AgdaBound{φ} \AgdaSymbol{→} \AgdaBound{σ} \AgdaFunction{∶} \AgdaBound{Γ} \AgdaFunction{⇒} \AgdaBound{Δ} \AgdaSymbol{→} \AgdaBound{Δ} \AgdaDatatype{⊢} \AgdaBound{δ} \AgdaFunction{⟦} \AgdaBound{σ} \AgdaFunction{⟧} \AgdaDatatype{∶} \AgdaBound{φ}\<%
\end{code}

\AgdaHide{
\begin{code}%
\>\AgdaFunction{substitution} \AgdaSymbol{(}\AgdaInductiveConstructor{var} \AgdaSymbol{\_)} \AgdaBound{σ∶Γ→Δ} \AgdaSymbol{=} \AgdaBound{σ∶Γ→Δ} \AgdaSymbol{\_}\<%
\\
\>\AgdaFunction{substitution} \AgdaSymbol{(}\AgdaInductiveConstructor{app} \AgdaBound{Γ⊢δ∶φ→ψ} \AgdaBound{Γ⊢ε∶φ}\AgdaSymbol{)} \AgdaBound{σ∶Γ→Δ} \AgdaSymbol{=} \AgdaInductiveConstructor{app} \AgdaSymbol{(}\AgdaFunction{substitution} \AgdaBound{Γ⊢δ∶φ→ψ} \AgdaBound{σ∶Γ→Δ}\AgdaSymbol{)} \AgdaSymbol{(}\AgdaFunction{substitution} \AgdaBound{Γ⊢ε∶φ} \AgdaBound{σ∶Γ→Δ}\AgdaSymbol{)}\<%
\\
\>\AgdaFunction{substitution} \AgdaSymbol{\{}\AgdaArgument{Q} \AgdaSymbol{=} \AgdaBound{Q}\AgdaSymbol{\}} \AgdaSymbol{\{}\AgdaArgument{Δ} \AgdaSymbol{=} \AgdaBound{Δ}\AgdaSymbol{\}} \AgdaSymbol{\{}\AgdaArgument{σ} \AgdaSymbol{=} \AgdaBound{σ}\AgdaSymbol{\}} \AgdaSymbol{(}\AgdaInductiveConstructor{Λ} \AgdaSymbol{\{}\AgdaBound{P}\AgdaSymbol{\}} \AgdaSymbol{\{}\AgdaBound{Γ}\AgdaSymbol{\}} \AgdaSymbol{\{}\AgdaBound{φ}\AgdaSymbol{\}} \AgdaSymbol{\{}\AgdaBound{δ}\AgdaSymbol{\}} \AgdaSymbol{\{}\AgdaBound{ψ}\AgdaSymbol{\}} \AgdaBound{Γ,φ⊢δ∶ψ}\AgdaSymbol{)} \AgdaBound{σ∶Γ→Δ} \AgdaSymbol{=} \AgdaInductiveConstructor{Λ} \<[79]%
\>[79]\<%
\\
\>[0]\AgdaIndent{2}{}\<[2]%
\>[2]\AgdaSymbol{(}\AgdaFunction{substitution} \AgdaBound{Γ,φ⊢δ∶ψ} \AgdaSymbol{(}\AgdaFunction{liftSub-typed} \AgdaBound{σ∶Γ→Δ}\AgdaSymbol{))}\<%
\\
%
\\
\>\AgdaFunction{comp-typed} \AgdaSymbol{:} \AgdaSymbol{∀} \AgdaSymbol{\{}\AgdaBound{P}\AgdaSymbol{\}} \AgdaSymbol{\{}\AgdaBound{Q}\AgdaSymbol{\}} \AgdaSymbol{\{}\AgdaBound{R}\AgdaSymbol{\}}\<%
\\
\>[0]\AgdaIndent{2}{}\<[2]%
\>[2]\AgdaSymbol{\{}\AgdaBound{Γ} \AgdaSymbol{:} \AgdaDatatype{Context} \AgdaBound{P}\AgdaSymbol{\}} \AgdaSymbol{\{}\AgdaBound{Δ} \AgdaSymbol{:} \AgdaDatatype{Context} \AgdaBound{Q}\AgdaSymbol{\}} \AgdaSymbol{\{}\AgdaBound{Θ} \AgdaSymbol{:} \AgdaDatatype{Context} \AgdaBound{R}\AgdaSymbol{\}}\<%
\\
\>[0]\AgdaIndent{2}{}\<[2]%
\>[2]\AgdaSymbol{\{}\AgdaBound{τ}\AgdaSymbol{\}} \AgdaSymbol{\{}\AgdaBound{σ}\AgdaSymbol{\}} \AgdaSymbol{→} \AgdaBound{τ} \AgdaFunction{∶} \AgdaBound{Δ} \AgdaFunction{⇒} \AgdaBound{Θ} \AgdaSymbol{→} \AgdaBound{σ} \AgdaFunction{∶} \AgdaBound{Γ} \AgdaFunction{⇒} \AgdaBound{Δ} \AgdaSymbol{→} \AgdaBound{τ} \AgdaFunction{•} \AgdaBound{σ} \AgdaFunction{∶} \AgdaBound{Γ} \AgdaFunction{⇒} \AgdaBound{Θ}\<%
\\
\>\AgdaFunction{comp-typed} \AgdaBound{τ∶Δ⇒Θ} \AgdaBound{σ∶Γ⇒Δ} \AgdaBound{x} \AgdaSymbol{=} \AgdaFunction{substitution} \AgdaSymbol{(}\AgdaBound{σ∶Γ⇒Δ} \AgdaBound{x}\AgdaSymbol{)} \AgdaBound{τ∶Δ⇒Θ}\<%
\end{code}
}

\begin{lemma}[Subject Reduction]
If $\Gamma \vdash \delta : \phi$ and $\delta \rightarrow_\beta \epsilon$ then $\Gamma \vdash \epsilon : \phi$.
\end{lemma}

\begin{code}%
\>\AgdaFunction{subject-reduction} \AgdaSymbol{:} \AgdaSymbol{∀} \AgdaSymbol{\{}\AgdaBound{P}\AgdaSymbol{\}} \AgdaSymbol{\{}\AgdaBound{Γ} \AgdaSymbol{:} \AgdaDatatype{Context} \AgdaBound{P}\AgdaSymbol{\}} \AgdaSymbol{\{}\AgdaBound{δ} \AgdaBound{ε} \AgdaSymbol{:} \AgdaFunction{Proof} \AgdaSymbol{(} \AgdaBound{P}\AgdaSymbol{)\}} \AgdaSymbol{\{}\AgdaBound{φ}\AgdaSymbol{\}} \AgdaSymbol{→} \<[67]%
\>[67]\<%
\\
\>[0]\AgdaIndent{2}{}\<[2]%
\>[2]\AgdaBound{Γ} \AgdaDatatype{⊢} \AgdaBound{δ} \AgdaDatatype{∶} \AgdaBound{φ} \AgdaSymbol{→} \AgdaBound{δ} \AgdaDatatype{⇒} \AgdaBound{ε} \AgdaSymbol{→} \AgdaBound{Γ} \AgdaDatatype{⊢} \AgdaBound{ε} \AgdaDatatype{∶} \AgdaBound{φ}\<%
\end{code}

\AgdaHide{
\begin{code}%
\>\AgdaFunction{subject-reduction} \AgdaSymbol{(}\AgdaInductiveConstructor{var} \AgdaSymbol{\_)} \AgdaSymbol{()}\<%
\\
\>\AgdaFunction{subject-reduction} \AgdaSymbol{(}\AgdaInductiveConstructor{app} \AgdaSymbol{\{}\AgdaArgument{ε} \AgdaSymbol{=} \AgdaBound{ε}\AgdaSymbol{\}} \AgdaSymbol{(}\AgdaInductiveConstructor{Λ} \AgdaSymbol{\{}\AgdaBound{P}\AgdaSymbol{\}} \AgdaSymbol{\{}\AgdaBound{Γ}\AgdaSymbol{\}} \AgdaSymbol{\{}\AgdaBound{φ}\AgdaSymbol{\}} \AgdaSymbol{\{}\AgdaBound{δ}\AgdaSymbol{\}} \AgdaSymbol{\{}\AgdaBound{ψ}\AgdaSymbol{\}} \AgdaBound{Γ,φ⊢δ∶ψ}\AgdaSymbol{)} \AgdaBound{Γ⊢ε∶φ}\AgdaSymbol{)} \AgdaSymbol{(}\AgdaInductiveConstructor{redex} \AgdaInductiveConstructor{βI}\AgdaSymbol{)} \AgdaSymbol{=} \<[83]%
\>[83]\<%
\\
\>[0]\AgdaIndent{2}{}\<[2]%
\>[2]\AgdaFunction{substitution} \AgdaBound{Γ,φ⊢δ∶ψ} \AgdaSymbol{(}\AgdaFunction{botSub-typed} \AgdaBound{Γ⊢ε∶φ}\AgdaSymbol{)}\<%
\\
\>\AgdaFunction{subject-reduction} \AgdaSymbol{(}\AgdaInductiveConstructor{app} \AgdaBound{Γ⊢δ∶φ→ψ} \AgdaBound{Γ⊢ε∶φ}\AgdaSymbol{)} \AgdaSymbol{(}\AgdaInductiveConstructor{app} \AgdaSymbol{(}\AgdaInductiveConstructor{appl} \AgdaBound{δ→δ'}\AgdaSymbol{))} \AgdaSymbol{=} \AgdaInductiveConstructor{app} \AgdaSymbol{(}\AgdaFunction{subject-reduction} \AgdaBound{Γ⊢δ∶φ→ψ} \AgdaBound{δ→δ'}\AgdaSymbol{)} \AgdaBound{Γ⊢ε∶φ}\<%
\\
\>\AgdaFunction{subject-reduction} \AgdaSymbol{(}\AgdaInductiveConstructor{app} \AgdaBound{Γ⊢δ∶φ→ψ} \AgdaBound{Γ⊢ε∶φ}\AgdaSymbol{)} \AgdaSymbol{(}\AgdaInductiveConstructor{app} \AgdaSymbol{(}\AgdaInductiveConstructor{appr} \AgdaSymbol{(}\AgdaInductiveConstructor{appl} \AgdaBound{ε→ε'}\AgdaSymbol{)))} \AgdaSymbol{=} \AgdaInductiveConstructor{app} \AgdaBound{Γ⊢δ∶φ→ψ} \AgdaSymbol{(}\AgdaFunction{subject-reduction} \AgdaBound{Γ⊢ε∶φ} \AgdaBound{ε→ε'}\AgdaSymbol{)}\<%
\\
\>\AgdaFunction{subject-reduction} \AgdaSymbol{(}\AgdaInductiveConstructor{app} \AgdaBound{Γ⊢δ∶φ→ψ} \AgdaBound{Γ⊢ε∶φ}\AgdaSymbol{)} \AgdaSymbol{(}\AgdaInductiveConstructor{app} \AgdaSymbol{(}\AgdaInductiveConstructor{appr} \AgdaSymbol{(}\AgdaInductiveConstructor{appr} \AgdaSymbol{())))}\<%
\\
\>\AgdaFunction{subject-reduction} \AgdaSymbol{(}\AgdaInductiveConstructor{Λ} \AgdaSymbol{\_)} \AgdaSymbol{(}\AgdaInductiveConstructor{redex} \AgdaSymbol{())}\<%
\\
\>\AgdaFunction{subject-reduction} \AgdaSymbol{(}\AgdaInductiveConstructor{Λ} \AgdaBound{Γ,φ⊢δ∶ψ}\AgdaSymbol{)} \AgdaSymbol{(}\AgdaInductiveConstructor{app} \AgdaSymbol{(}\AgdaInductiveConstructor{appl} \AgdaBound{δ⇒ε}\AgdaSymbol{))} \AgdaSymbol{=} \AgdaInductiveConstructor{Λ} \AgdaSymbol{(}\AgdaFunction{subject-reduction} \AgdaBound{Γ,φ⊢δ∶ψ} \AgdaBound{δ⇒ε}\AgdaSymbol{)}\<%
\\
\>\AgdaFunction{subject-reduction} \AgdaSymbol{(}\AgdaInductiveConstructor{Λ} \AgdaBound{Γ⊢δ∶φ}\AgdaSymbol{)} \AgdaSymbol{(}\AgdaInductiveConstructor{app} \AgdaSymbol{(}\AgdaInductiveConstructor{appr} \AgdaSymbol{()))}\<%
\end{code}
}

}

\section{Systems Two and Three}

\section{Conclusion}

\appendix

\section{Replacement and Substitution}
\label{appendix:repsub}

\AgdaHide{
\begin{code}%
\>\AgdaKeyword{open} \AgdaKeyword{import} \AgdaModule{Grammar.Base}\<%
\\
%
\\
\>\AgdaKeyword{module} \AgdaModule{Grammar.OpFamily.PreOpFamily} \AgdaSymbol{(}\AgdaBound{G} \AgdaSymbol{:} \AgdaRecord{Grammar}\AgdaSymbol{)} \AgdaKeyword{where}\<%
\\
\>\AgdaKeyword{open} \AgdaKeyword{import} \AgdaModule{Prelims}\<%
\\
\>\AgdaKeyword{open} \AgdaModule{Grammar} \AgdaBound{G}\<%
\end{code}
}

\subsection{Families of Operations}

Our aim here is to define the operations of \emph{replacement} and \emph{substitution}.  In order to organise this work, we introduce the following definitions.

A \emph{family of operations} over a grammar $G$ consists of:
\begin{enumerate}
\item
for any alphabets $U$ and $V$, a set $F[U,V]$ of \emph{operations} $\sigma$ from $U$ to $V$, $\sigma : U \rightarrow V$;
\item
for any operation $\sigma : U \rightarrow V$ and variable $x \in U$ of kind $K$, an expression $\sigma(x)$ over $V$ of kind $K$;
\item
for any alphabet $V$ and variable kind $K$, an operation $\uparrow : V \rightarrow (V , K)$, the \emph{lifting} operation;
\item
for any alphabet $V$, an operation $\id{V} : V \rightarrow V$, the \emph{identity} operation;
\item
for any operation $\sigma : U \rightarrow V$ and variable kind $K$, an operation $(\sigma , K) : (U , K) \rightarrow (V , K)$, the result of \emph{lifting} $\sigma$;
\item
for any operations $\rho : U \rightarrow V$ and $\sigma : V \rightarrow W$, an operation
$\sigma \circ \rho : U \rightarrow W$, the \emph{composition} of $\sigma$ and $\rho$;
\end{enumerate}
such that:
\begin{itemize}
\item
$\uparrow (x) \equiv x$
\item
$\id{V}(x) \equiv x$
\item
If $\rho \sim \sigma$ then $(\rho , K) \sim (\sigma , K)$
\item
$(\rho , K)(x_0) \equiv x_0$
\item
Given $\sigma : U \rightarrow V$ and $x \in U$, we have $(\sigma , K)(x) \equiv x$
\item
$(\sigma \circ \rho , K) \sim (\sigma , K) \circ (\rho , K)$
\item
$(\sigma \circ \rho)(x) \equiv \rho(x) [ \sigma ]$
\end{itemize}
where for $\sigma, \rho : U \rightarrow V$ we write $\sigma \sim \rho$ iff $\sigma(x) \equiv \rho(x)$ for all $x \in U$; and, given $\sigma : U \rightarrow V$ and $E$ an expression over $U$, we define $E[\sigma]$, the result of \emph{applying} the operation $\sigma$ to $E$, as follows:

\begin{align*}
x[\sigma] & \eqdef \sigma(x) \\
\lefteqn{c([\vec{x_1}] E_1, \ldots, [\vec{x_n}] E_n) [\sigma]} \\
 & \eqdef
c([\vec{x_1}] E_1 [(\sigma , K_{11}, \ldots, K_{1r_1})], \ldots,
[\vec{x_n}] E_n [(\sigma, K_{n1}, \ldots, K_{nr_n})])
\end{align*}
for $c$ a constructor of type (\ref{eq:conkind}).

\subsubsection{Pre-Families}
We formalize this definition in stages.  First, we define a \emph{pre-family of operations} to be a family with items of data 1--4 above that satisfies the first two axioms:

\begin{code}%
\>\AgdaKeyword{record} \AgdaRecord{PreOpFamily} \AgdaSymbol{:} \AgdaPrimitiveType{Set₂} \AgdaKeyword{where}\<%
\\
\>[0]\AgdaIndent{2}{}\<[2]%
\>[2]\AgdaKeyword{field}\<%
\\
\>[2]\AgdaIndent{4}{}\<[4]%
\>[4]\AgdaField{Op} \AgdaSymbol{:} \AgdaDatatype{Alphabet} \AgdaSymbol{→} \AgdaDatatype{Alphabet} \AgdaSymbol{→} \AgdaPrimitiveType{Set}\<%
\\
\>[2]\AgdaIndent{4}{}\<[4]%
\>[4]\AgdaField{apV} \AgdaSymbol{:} \AgdaSymbol{∀} \AgdaSymbol{\{}\AgdaBound{U}\AgdaSymbol{\}} \AgdaSymbol{\{}\AgdaBound{V}\AgdaSymbol{\}} \AgdaSymbol{\{}\AgdaBound{K}\AgdaSymbol{\}} \AgdaSymbol{→} \AgdaField{Op} \AgdaBound{U} \AgdaBound{V} \AgdaSymbol{→} \AgdaDatatype{Var} \AgdaBound{U} \AgdaBound{K} \AgdaSymbol{→} \AgdaFunction{VExpression} \AgdaBound{V} \AgdaBound{K}\<%
\\
\>[2]\AgdaIndent{4}{}\<[4]%
\>[4]\AgdaField{up} \AgdaSymbol{:} \AgdaSymbol{∀} \AgdaSymbol{\{}\AgdaBound{V}\AgdaSymbol{\}} \AgdaSymbol{\{}\AgdaBound{K}\AgdaSymbol{\}} \AgdaSymbol{→} \AgdaField{Op} \AgdaBound{V} \AgdaSymbol{(}\AgdaBound{V} \AgdaInductiveConstructor{,} \AgdaBound{K}\AgdaSymbol{)}\<%
\\
\>[2]\AgdaIndent{4}{}\<[4]%
\>[4]\AgdaField{apV-up} \AgdaSymbol{:} \AgdaSymbol{∀} \AgdaSymbol{\{}\AgdaBound{V}\AgdaSymbol{\}} \AgdaSymbol{\{}\AgdaBound{K}\AgdaSymbol{\}} \AgdaSymbol{\{}\AgdaBound{L}\AgdaSymbol{\}} \AgdaSymbol{\{}\AgdaBound{x} \AgdaSymbol{:} \AgdaDatatype{Var} \AgdaBound{V} \AgdaBound{K}\AgdaSymbol{\}} \AgdaSymbol{→} \AgdaField{apV} \AgdaSymbol{(}\AgdaField{up} \AgdaSymbol{\{}\AgdaArgument{K} \AgdaSymbol{=} \AgdaBound{L}\AgdaSymbol{\})} \AgdaBound{x} \AgdaDatatype{≡} \AgdaInductiveConstructor{var} \AgdaSymbol{(}\AgdaInductiveConstructor{↑} \AgdaBound{x}\AgdaSymbol{)}\<%
\\
\>[2]\AgdaIndent{4}{}\<[4]%
\>[4]\AgdaField{idOp} \AgdaSymbol{:} \AgdaSymbol{∀} \AgdaBound{V} \AgdaSymbol{→} \AgdaField{Op} \AgdaBound{V} \AgdaBound{V}\<%
\\
\>[2]\AgdaIndent{4}{}\<[4]%
\>[4]\AgdaField{apV-idOp} \AgdaSymbol{:} \AgdaSymbol{∀} \AgdaSymbol{\{}\AgdaBound{V}\AgdaSymbol{\}} \AgdaSymbol{\{}\AgdaBound{K}\AgdaSymbol{\}} \AgdaSymbol{(}\AgdaBound{x} \AgdaSymbol{:} \AgdaDatatype{Var} \AgdaBound{V} \AgdaBound{K}\AgdaSymbol{)} \AgdaSymbol{→} \AgdaField{apV} \AgdaSymbol{(}\AgdaField{idOp} \AgdaBound{V}\AgdaSymbol{)} \AgdaBound{x} \AgdaDatatype{≡} \AgdaInductiveConstructor{var} \AgdaBound{x}\<%
\end{code}

This allows us to define the relation $\sim$, and prove it is an equivalence relation:

\begin{code}%
\>[0]\AgdaIndent{2}{}\<[2]%
\>[2]\AgdaFunction{\_∼op\_} \AgdaSymbol{:} \AgdaSymbol{∀} \AgdaSymbol{\{}\AgdaBound{U}\AgdaSymbol{\}} \AgdaSymbol{\{}\AgdaBound{V}\AgdaSymbol{\}} \AgdaSymbol{→} \AgdaField{Op} \AgdaBound{U} \AgdaBound{V} \AgdaSymbol{→} \AgdaField{Op} \AgdaBound{U} \AgdaBound{V} \AgdaSymbol{→} \AgdaPrimitiveType{Set}\<%
\\
\>[0]\AgdaIndent{2}{}\<[2]%
\>[2]\AgdaFunction{\_∼op\_} \AgdaSymbol{\{}\AgdaBound{U}\AgdaSymbol{\}} \AgdaSymbol{\{}\AgdaBound{V}\AgdaSymbol{\}} \AgdaBound{ρ} \AgdaBound{σ} \AgdaSymbol{=} \AgdaSymbol{∀} \AgdaSymbol{\{}\AgdaBound{K}\AgdaSymbol{\}} \AgdaSymbol{(}\AgdaBound{x} \AgdaSymbol{:} \AgdaDatatype{Var} \AgdaBound{U} \AgdaBound{K}\AgdaSymbol{)} \AgdaSymbol{→} \AgdaField{apV} \AgdaBound{ρ} \AgdaBound{x} \AgdaDatatype{≡} \AgdaField{apV} \AgdaBound{σ} \AgdaBound{x}\<%
\\
\>[2]\AgdaIndent{4}{}\<[4]%
\>[4]\<%
\\
\>[0]\AgdaIndent{2}{}\<[2]%
\>[2]\AgdaFunction{∼-refl} \AgdaSymbol{:} \AgdaSymbol{∀} \AgdaSymbol{\{}\AgdaBound{U}\AgdaSymbol{\}} \AgdaSymbol{\{}\AgdaBound{V}\AgdaSymbol{\}} \AgdaSymbol{\{}\AgdaBound{σ} \AgdaSymbol{:} \AgdaField{Op} \AgdaBound{U} \AgdaBound{V}\AgdaSymbol{\}} \AgdaSymbol{→} \AgdaBound{σ} \AgdaFunction{∼op} \AgdaBound{σ}\<%
\\
\>[0]\AgdaIndent{2}{}\<[2]%
\>[2]\AgdaFunction{∼-refl} \AgdaSymbol{\_} \AgdaSymbol{=} \AgdaInductiveConstructor{refl}\<%
\\
\>[2]\AgdaIndent{4}{}\<[4]%
\>[4]\<%
\\
\>[0]\AgdaIndent{2}{}\<[2]%
\>[2]\AgdaFunction{∼-sym} \AgdaSymbol{:} \AgdaSymbol{∀} \AgdaSymbol{\{}\AgdaBound{U}\AgdaSymbol{\}} \AgdaSymbol{\{}\AgdaBound{V}\AgdaSymbol{\}} \AgdaSymbol{\{}\AgdaBound{σ} \AgdaBound{τ} \AgdaSymbol{:} \AgdaField{Op} \AgdaBound{U} \AgdaBound{V}\AgdaSymbol{\}} \AgdaSymbol{→} \AgdaBound{σ} \AgdaFunction{∼op} \AgdaBound{τ} \AgdaSymbol{→} \AgdaBound{τ} \AgdaFunction{∼op} \AgdaBound{σ}\<%
\\
\>[0]\AgdaIndent{2}{}\<[2]%
\>[2]\AgdaFunction{∼-sym} \AgdaBound{σ-is-τ} \AgdaBound{x} \AgdaSymbol{=} \AgdaFunction{sym} \AgdaSymbol{(}\AgdaBound{σ-is-τ} \AgdaBound{x}\AgdaSymbol{)}\<%
\\
%
\\
\>[0]\AgdaIndent{2}{}\<[2]%
\>[2]\AgdaFunction{∼-trans} \AgdaSymbol{:} \AgdaSymbol{∀} \AgdaSymbol{\{}\AgdaBound{U}\AgdaSymbol{\}} \AgdaSymbol{\{}\AgdaBound{V}\AgdaSymbol{\}} \AgdaSymbol{\{}\AgdaBound{ρ} \AgdaBound{σ} \AgdaBound{τ} \AgdaSymbol{:} \AgdaField{Op} \AgdaBound{U} \AgdaBound{V}\AgdaSymbol{\}} \AgdaSymbol{→} \AgdaBound{ρ} \AgdaFunction{∼op} \AgdaBound{σ} \AgdaSymbol{→} \AgdaBound{σ} \AgdaFunction{∼op} \AgdaBound{τ} \AgdaSymbol{→} \AgdaBound{ρ} \AgdaFunction{∼op} \AgdaBound{τ}\<%
\\
\>[0]\AgdaIndent{2}{}\<[2]%
\>[2]\AgdaFunction{∼-trans} \AgdaBound{ρ-is-σ} \AgdaBound{σ-is-τ} \AgdaBound{x} \AgdaSymbol{=} \AgdaFunction{trans} \AgdaSymbol{(}\AgdaBound{ρ-is-σ} \AgdaBound{x}\AgdaSymbol{)} \AgdaSymbol{(}\AgdaBound{σ-is-τ} \AgdaBound{x}\AgdaSymbol{)}\<%
\\
%
\\
\>[0]\AgdaIndent{2}{}\<[2]%
\>[2]\AgdaFunction{OP} \AgdaSymbol{:} \AgdaDatatype{Alphabet} \AgdaSymbol{→} \AgdaDatatype{Alphabet} \AgdaSymbol{→} \<[30]%
\>[30]\AgdaRecord{Setoid} \AgdaSymbol{\_} \AgdaSymbol{\_}\<%
\\
\>[0]\AgdaIndent{2}{}\<[2]%
\>[2]\AgdaFunction{OP} \AgdaBound{U} \AgdaBound{V} \AgdaSymbol{=} \AgdaKeyword{record} \AgdaSymbol{\{} \<[20]%
\>[20]\<%
\\
\>[2]\AgdaIndent{5}{}\<[5]%
\>[5]\AgdaField{Carrier} \AgdaSymbol{=} \AgdaField{Op} \AgdaBound{U} \AgdaBound{V} \AgdaSymbol{;} \<[24]%
\>[24]\<%
\\
\>[2]\AgdaIndent{5}{}\<[5]%
\>[5]\AgdaField{\_≈\_} \AgdaSymbol{=} \AgdaFunction{\_∼op\_} \AgdaSymbol{;} \<[19]%
\>[19]\<%
\\
\>[2]\AgdaIndent{5}{}\<[5]%
\>[5]\AgdaField{isEquivalence} \AgdaSymbol{=} \AgdaKeyword{record} \AgdaSymbol{\{} \<[30]%
\>[30]\<%
\\
\>[5]\AgdaIndent{7}{}\<[7]%
\>[7]\AgdaField{refl} \AgdaSymbol{=} \AgdaFunction{∼-refl} \AgdaSymbol{;} \<[23]%
\>[23]\<%
\\
\>[5]\AgdaIndent{7}{}\<[7]%
\>[7]\AgdaField{sym} \AgdaSymbol{=} \AgdaFunction{∼-sym} \AgdaSymbol{;} \<[21]%
\>[21]\<%
\\
\>[5]\AgdaIndent{7}{}\<[7]%
\>[7]\AgdaField{trans} \AgdaSymbol{=} \AgdaFunction{∼-trans} \AgdaSymbol{\}} \AgdaSymbol{\}}\<%
\end{code}


\AgdaHide{
\begin{code}%
\>\AgdaKeyword{open} \AgdaKeyword{import} \AgdaModule{Grammar.Base}\<%
\\
%
\\
\>\AgdaKeyword{module} \AgdaModule{Grammar.OpFamily.Lifting} \AgdaSymbol{(}\AgdaBound{G} \AgdaSymbol{:} \AgdaRecord{Grammar}\AgdaSymbol{)} \AgdaKeyword{where}\<%
\\
\>\AgdaKeyword{open} \AgdaKeyword{import} \AgdaModule{Data.List}\<%
\\
\>\AgdaKeyword{open} \AgdaKeyword{import} \AgdaModule{Prelims}\<%
\\
\>\AgdaKeyword{open} \AgdaModule{Grammar} \AgdaBound{G}\<%
\\
\>\AgdaKeyword{open} \AgdaKeyword{import} \AgdaModule{Grammar.OpFamily.PreOpFamily} \AgdaBound{G}\<%
\end{code}
}

\subsubsection{Liftings}

Define a \emph{lifting} on a pre-family to be an function $(- , K)$ that respects $\sim$:

\begin{code}%
\>\AgdaKeyword{record} \AgdaRecord{Lifting} \AgdaSymbol{(}\AgdaBound{F} \AgdaSymbol{:} \AgdaRecord{PreOpFamily}\AgdaSymbol{)} \AgdaSymbol{:} \AgdaPrimitiveType{Set₁} \AgdaKeyword{where}\<%
\\
\>[0]\AgdaIndent{2}{}\<[2]%
\>[2]\AgdaKeyword{open} \AgdaModule{PreOpFamily} \AgdaBound{F}\<%
\\
\>[0]\AgdaIndent{2}{}\<[2]%
\>[2]\AgdaKeyword{field}\<%
\\
\>[2]\AgdaIndent{4}{}\<[4]%
\>[4]\AgdaField{liftOp} \AgdaSymbol{:} \AgdaSymbol{∀} \AgdaSymbol{\{}\AgdaBound{U}\AgdaSymbol{\}} \AgdaSymbol{\{}\AgdaBound{V}\AgdaSymbol{\}} \AgdaBound{K} \AgdaSymbol{→} \AgdaFunction{Op} \AgdaBound{U} \AgdaBound{V} \AgdaSymbol{→} \AgdaFunction{Op} \AgdaSymbol{(}\AgdaBound{U} \AgdaInductiveConstructor{,} \AgdaBound{K}\AgdaSymbol{)} \AgdaSymbol{(}\AgdaBound{V} \AgdaInductiveConstructor{,} \AgdaBound{K}\AgdaSymbol{)}\<%
\\
\>[2]\AgdaIndent{4}{}\<[4]%
\>[4]\AgdaField{liftOp-cong} \AgdaSymbol{:} \AgdaSymbol{∀} \AgdaSymbol{\{}\AgdaBound{V}\AgdaSymbol{\}} \AgdaSymbol{\{}\AgdaBound{W}\AgdaSymbol{\}} \AgdaSymbol{\{}\AgdaBound{K}\AgdaSymbol{\}} \AgdaSymbol{\{}\AgdaBound{ρ} \AgdaBound{σ} \AgdaSymbol{:} \AgdaFunction{Op} \AgdaBound{V} \AgdaBound{W}\AgdaSymbol{\}} \AgdaSymbol{→} \<[49]%
\>[49]\<%
\\
\>[4]\AgdaIndent{6}{}\<[6]%
\>[6]\AgdaBound{ρ} \AgdaFunction{∼op} \AgdaBound{σ} \AgdaSymbol{→} \AgdaField{liftOp} \AgdaBound{K} \AgdaBound{ρ} \AgdaFunction{∼op} \AgdaField{liftOp} \AgdaBound{K} \AgdaBound{σ}\<%
\end{code}

Given an operation $\sigma : U \rightarrow V$ and a list of variable kinds $A \equiv (A_1 , \ldots , A_n)$, define
the \emph{repeated lifting} $\sigma^A$ to be $((\cdots(\sigma , A_1) , A_2) , \cdots ) , A_n)$.

\begin{code}%
\>\AgdaComment{\{-  liftOp' : ∀ \{U\} \{V\} A → Op U V → Op (extend U A) (extend V A)\<\\
\>  liftOp' [] σ = σ\<\\
\>  liftOp' (K ∷ A) σ = liftOp' A (liftOp K σ) -\}}\<%
\\
%
\\
\>[0]\AgdaIndent{2}{}\<[2]%
\>[2]\AgdaFunction{liftOp''} \AgdaSymbol{:} \AgdaSymbol{∀} \AgdaSymbol{\{}\AgdaBound{U}\AgdaSymbol{\}} \AgdaSymbol{\{}\AgdaBound{V}\AgdaSymbol{\}} \AgdaSymbol{\{}\AgdaBound{K}\AgdaSymbol{\}} \AgdaBound{A} \AgdaSymbol{→} \AgdaFunction{Op} \AgdaBound{U} \AgdaBound{V} \AgdaSymbol{→} \AgdaFunction{Op} \AgdaSymbol{(}\AgdaFunction{dom} \AgdaBound{U} \AgdaSymbol{\{}\AgdaBound{K}\AgdaSymbol{\}} \AgdaBound{A}\AgdaSymbol{)} \AgdaSymbol{(}\AgdaFunction{dom} \AgdaBound{V} \AgdaBound{A}\AgdaSymbol{)}\<%
\\
\>[0]\AgdaIndent{2}{}\<[2]%
\>[2]\AgdaFunction{liftOp''} \AgdaSymbol{(\_} \AgdaInductiveConstructor{●}\AgdaSymbol{)} \AgdaBound{σ} \AgdaSymbol{=} \AgdaBound{σ}\<%
\\
\>[0]\AgdaIndent{2}{}\<[2]%
\>[2]\AgdaFunction{liftOp''} \AgdaSymbol{(}\AgdaBound{K} \AgdaInductiveConstructor{⟶} \AgdaBound{A}\AgdaSymbol{)} \AgdaBound{σ} \AgdaSymbol{=} \AgdaFunction{liftOp''} \AgdaBound{A} \AgdaSymbol{(}\AgdaField{liftOp} \AgdaBound{K} \AgdaBound{σ}\AgdaSymbol{)}\<%
\\
%
\\
\>\AgdaComment{\{-  liftOp'-cong : ∀ \{U\} \{V\} A \{ρ σ : Op U V\} → \<\\
\>    ρ ∼op σ → liftOp' A ρ ∼op liftOp' A σ\<\\
\>}\<%
\end{code}

\AgdaHide{
\begin{code}%
\>\AgdaComment{\<\\
\>  liftOp'-cong [] ρ-is-σ = ρ-is-σ\<\\
\>  liftOp'-cong (\_ ∷ A) ρ-is-σ = liftOp'-cong A (liftOp-cong ρ-is-σ) -\}}\<%
\\
%
\\
\>[0]\AgdaIndent{2}{}\<[2]%
\>[2]\AgdaKeyword{postulate} \AgdaPostulate{liftOp''-cong} \AgdaSymbol{:} \AgdaSymbol{∀} \AgdaSymbol{\{}\AgdaBound{U}\AgdaSymbol{\}} \AgdaSymbol{\{}\AgdaBound{V}\AgdaSymbol{\}} \AgdaSymbol{\{}\AgdaBound{K}\AgdaSymbol{\}} \AgdaBound{A} \AgdaSymbol{\{}\AgdaBound{ρ} \AgdaBound{σ} \AgdaSymbol{:} \AgdaFunction{Op} \AgdaBound{U} \AgdaBound{V}\AgdaSymbol{\}} \AgdaSymbol{→} \<[61]%
\>[61]\<%
\\
\>[2]\AgdaIndent{26}{}\<[26]%
\>[26]\AgdaBound{ρ} \AgdaFunction{∼op} \AgdaBound{σ} \AgdaSymbol{→} \AgdaFunction{liftOp''} \AgdaSymbol{\{}\AgdaArgument{K} \AgdaSymbol{=} \AgdaBound{K}\AgdaSymbol{\}} \AgdaBound{A} \AgdaBound{ρ} \AgdaFunction{∼op} \AgdaFunction{liftOp''} \AgdaBound{A} \AgdaBound{σ}\<%
\end{code}
}

This allows us to define the action of \emph{application} $E[\sigma]$:

\begin{code}%
\>[0]\AgdaIndent{2}{}\<[2]%
\>[2]\AgdaFunction{ap} \AgdaSymbol{:} \AgdaSymbol{∀} \AgdaSymbol{\{}\AgdaBound{U}\AgdaSymbol{\}} \AgdaSymbol{\{}\AgdaBound{V}\AgdaSymbol{\}} \AgdaSymbol{\{}\AgdaBound{C}\AgdaSymbol{\}} \AgdaSymbol{\{}\AgdaBound{K}\AgdaSymbol{\}} \AgdaSymbol{→} \<[27]%
\>[27]\<%
\\
\>[2]\AgdaIndent{4}{}\<[4]%
\>[4]\AgdaFunction{Op} \AgdaBound{U} \AgdaBound{V} \AgdaSymbol{→} \AgdaDatatype{Subexpression} \AgdaBound{U} \AgdaBound{C} \AgdaBound{K} \AgdaSymbol{→} \AgdaDatatype{Subexpression} \AgdaBound{V} \AgdaBound{C} \AgdaBound{K}\<%
\\
\>[0]\AgdaIndent{2}{}\<[2]%
\>[2]\AgdaFunction{ap} \AgdaBound{ρ} \AgdaSymbol{(}\AgdaInductiveConstructor{var} \AgdaBound{x}\AgdaSymbol{)} \AgdaSymbol{=} \AgdaFunction{apV} \AgdaBound{ρ} \AgdaBound{x}\<%
\\
\>[0]\AgdaIndent{2}{}\<[2]%
\>[2]\AgdaFunction{ap} \AgdaBound{ρ} \AgdaSymbol{(}\AgdaInductiveConstructor{app} \AgdaBound{c} \AgdaBound{EE}\AgdaSymbol{)} \AgdaSymbol{=} \AgdaInductiveConstructor{app} \AgdaBound{c} \AgdaSymbol{(}\AgdaFunction{ap} \AgdaBound{ρ} \AgdaBound{EE}\AgdaSymbol{)}\<%
\\
\>[0]\AgdaIndent{2}{}\<[2]%
\>[2]\AgdaFunction{ap} \AgdaSymbol{\_} \AgdaInductiveConstructor{out} \AgdaSymbol{=} \AgdaInductiveConstructor{out}\<%
\\
\>[0]\AgdaIndent{2}{}\<[2]%
\>[2]\AgdaFunction{ap} \AgdaBound{ρ} \AgdaSymbol{(}\AgdaInductiveConstructor{\_,,\_} \AgdaSymbol{\{}\AgdaArgument{A} \AgdaSymbol{=} \AgdaBound{A}\AgdaSymbol{\}} \AgdaBound{E} \AgdaBound{EE}\AgdaSymbol{)} \AgdaSymbol{=} \AgdaFunction{ap} \AgdaSymbol{(}\AgdaFunction{liftOp''} \AgdaBound{A} \AgdaBound{ρ}\AgdaSymbol{)} \AgdaBound{E} \AgdaInductiveConstructor{,,} \AgdaFunction{ap} \AgdaBound{ρ} \AgdaBound{EE}\<%
\end{code}

We prove that application respects $\sim$.

\begin{code}%
\>[0]\AgdaIndent{2}{}\<[2]%
\>[2]\AgdaFunction{ap-congl} \AgdaSymbol{:} \AgdaSymbol{∀} \AgdaSymbol{\{}\AgdaBound{U}\AgdaSymbol{\}} \AgdaSymbol{\{}\AgdaBound{V}\AgdaSymbol{\}} \AgdaSymbol{\{}\AgdaBound{C}\AgdaSymbol{\}} \AgdaSymbol{\{}\AgdaBound{K}\AgdaSymbol{\}} \<[31]%
\>[31]\<%
\\
\>[2]\AgdaIndent{4}{}\<[4]%
\>[4]\AgdaSymbol{\{}\AgdaBound{ρ} \AgdaBound{σ} \AgdaSymbol{:} \AgdaFunction{Op} \AgdaBound{U} \AgdaBound{V}\AgdaSymbol{\}} \AgdaSymbol{(}\AgdaBound{E} \AgdaSymbol{:} \AgdaDatatype{Subexpression} \AgdaBound{U} \AgdaBound{C} \AgdaBound{K}\AgdaSymbol{)} \AgdaSymbol{→}\<%
\\
\>[2]\AgdaIndent{4}{}\<[4]%
\>[4]\AgdaBound{ρ} \AgdaFunction{∼op} \AgdaBound{σ} \AgdaSymbol{→} \AgdaFunction{ap} \AgdaBound{ρ} \AgdaBound{E} \AgdaDatatype{≡} \AgdaFunction{ap} \AgdaBound{σ} \AgdaBound{E}\<%
\end{code}

\AgdaHide{
\begin{code}%
\>[0]\AgdaIndent{2}{}\<[2]%
\>[2]\AgdaFunction{ap-congl} \AgdaSymbol{(}\AgdaInductiveConstructor{var} \AgdaBound{x}\AgdaSymbol{)} \AgdaBound{ρ-is-σ} \AgdaSymbol{=} \AgdaBound{ρ-is-σ} \AgdaBound{x}\<%
\\
\>[0]\AgdaIndent{2}{}\<[2]%
\>[2]\AgdaFunction{ap-congl} \AgdaSymbol{(}\AgdaInductiveConstructor{app} \AgdaBound{c} \AgdaBound{E}\AgdaSymbol{)} \AgdaBound{ρ-is-σ} \AgdaSymbol{=} \AgdaFunction{cong} \AgdaSymbol{(}\AgdaInductiveConstructor{app} \AgdaBound{c}\AgdaSymbol{)} \AgdaSymbol{(}\AgdaFunction{ap-congl} \AgdaBound{E} \AgdaBound{ρ-is-σ}\AgdaSymbol{)}\<%
\\
\>[0]\AgdaIndent{2}{}\<[2]%
\>[2]\AgdaFunction{ap-congl} \AgdaInductiveConstructor{out} \AgdaSymbol{\_} \AgdaSymbol{=} \AgdaInductiveConstructor{refl}\<%
\\
\>[0]\AgdaIndent{2}{}\<[2]%
\>[2]\AgdaFunction{ap-congl} \AgdaSymbol{(}\AgdaInductiveConstructor{\_,,\_} \AgdaSymbol{\{}\AgdaArgument{L} \AgdaSymbol{=} \AgdaBound{L}\AgdaSymbol{\}} \AgdaSymbol{\{}\AgdaArgument{A} \AgdaSymbol{=} \AgdaBound{A}\AgdaSymbol{\}} \AgdaBound{E} \AgdaBound{F}\AgdaSymbol{)} \AgdaBound{ρ-is-σ} \AgdaSymbol{=} \<[47]%
\>[47]\<%
\\
\>[2]\AgdaIndent{4}{}\<[4]%
\>[4]\AgdaFunction{cong₂} \AgdaInductiveConstructor{\_,,\_} \AgdaSymbol{(}\AgdaFunction{ap-congl} \AgdaBound{E} \AgdaSymbol{(}\AgdaPostulate{liftOp''-cong} \AgdaBound{A} \AgdaBound{ρ-is-σ}\AgdaSymbol{))} \AgdaSymbol{(}\AgdaFunction{ap-congl} \AgdaBound{F} \AgdaBound{ρ-is-σ}\AgdaSymbol{)}\<%
\\
%
\\
\>[0]\AgdaIndent{2}{}\<[2]%
\>[2]\AgdaFunction{ap-congr} \AgdaSymbol{:} \AgdaSymbol{∀} \AgdaSymbol{\{}\AgdaBound{U}\AgdaSymbol{\}} \AgdaSymbol{\{}\AgdaBound{V}\AgdaSymbol{\}} \AgdaSymbol{\{}\AgdaBound{C}\AgdaSymbol{\}} \AgdaSymbol{\{}\AgdaBound{K}\AgdaSymbol{\}}\<%
\\
\>[2]\AgdaIndent{4}{}\<[4]%
\>[4]\AgdaSymbol{\{}\AgdaBound{σ} \AgdaSymbol{:} \AgdaFunction{Op} \AgdaBound{U} \AgdaBound{V}\AgdaSymbol{\}} \AgdaSymbol{\{}\AgdaBound{E} \AgdaBound{F} \AgdaSymbol{:} \AgdaDatatype{Subexpression} \AgdaBound{U} \AgdaBound{C} \AgdaBound{K}\AgdaSymbol{\}} \AgdaSymbol{→}\<%
\\
\>[2]\AgdaIndent{4}{}\<[4]%
\>[4]\AgdaBound{E} \AgdaDatatype{≡} \AgdaBound{F} \AgdaSymbol{→} \AgdaFunction{ap} \AgdaBound{σ} \AgdaBound{E} \AgdaDatatype{≡} \AgdaFunction{ap} \AgdaBound{σ} \AgdaBound{F}\<%
\\
\>[0]\AgdaIndent{2}{}\<[2]%
\>[2]\AgdaFunction{ap-congr} \AgdaSymbol{\{}\AgdaArgument{σ} \AgdaSymbol{=} \AgdaBound{σ}\AgdaSymbol{\}} \AgdaSymbol{=} \AgdaFunction{cong} \AgdaSymbol{(}\AgdaFunction{ap} \AgdaBound{σ}\AgdaSymbol{)}\<%
\\
%
\\
\>[0]\AgdaIndent{2}{}\<[2]%
\>[2]\AgdaFunction{ap-cong} \AgdaSymbol{:} \AgdaSymbol{∀} \AgdaSymbol{\{}\AgdaBound{U}\AgdaSymbol{\}} \AgdaSymbol{\{}\AgdaBound{V}\AgdaSymbol{\}} \AgdaSymbol{\{}\AgdaBound{C}\AgdaSymbol{\}} \AgdaSymbol{\{}\AgdaBound{K}\AgdaSymbol{\}}\<%
\\
\>[2]\AgdaIndent{4}{}\<[4]%
\>[4]\AgdaSymbol{\{}\AgdaBound{ρ} \AgdaBound{σ} \AgdaSymbol{:} \AgdaFunction{Op} \AgdaBound{U} \AgdaBound{V}\AgdaSymbol{\}} \AgdaSymbol{\{}\AgdaBound{M} \AgdaBound{N} \AgdaSymbol{:} \AgdaDatatype{Subexpression} \AgdaBound{U} \AgdaBound{C} \AgdaBound{K}\AgdaSymbol{\}} \AgdaSymbol{→}\<%
\\
\>[2]\AgdaIndent{4}{}\<[4]%
\>[4]\AgdaBound{ρ} \AgdaFunction{∼op} \AgdaBound{σ} \AgdaSymbol{→} \AgdaBound{M} \AgdaDatatype{≡} \AgdaBound{N} \AgdaSymbol{→} \AgdaFunction{ap} \AgdaBound{ρ} \AgdaBound{M} \AgdaDatatype{≡} \AgdaFunction{ap} \AgdaBound{σ} \AgdaBound{N}\<%
\\
\>[0]\AgdaIndent{2}{}\<[2]%
\>[2]\AgdaFunction{ap-cong} \AgdaSymbol{\{}\AgdaArgument{ρ} \AgdaSymbol{=} \AgdaBound{ρ}\AgdaSymbol{\}} \AgdaSymbol{\{}\AgdaBound{σ}\AgdaSymbol{\}} \AgdaSymbol{\{}\AgdaBound{M}\AgdaSymbol{\}} \AgdaSymbol{\{}\AgdaBound{N}\AgdaSymbol{\}} \AgdaBound{ρ∼σ} \AgdaBound{M≡N} \AgdaSymbol{=} \AgdaKeyword{let} \AgdaKeyword{open} \AgdaModule{≡-Reasoning} \AgdaKeyword{in} \<[64]%
\>[64]\<%
\\
\>[2]\AgdaIndent{4}{}\<[4]%
\>[4]\AgdaFunction{begin}\<%
\\
\>[4]\AgdaIndent{6}{}\<[6]%
\>[6]\AgdaFunction{ap} \AgdaBound{ρ} \AgdaBound{M}\<%
\\
\>[0]\AgdaIndent{4}{}\<[4]%
\>[4]\AgdaFunction{≡⟨} \AgdaFunction{ap-congl} \AgdaBound{M} \AgdaBound{ρ∼σ} \AgdaFunction{⟩}\<%
\\
\>[4]\AgdaIndent{6}{}\<[6]%
\>[6]\AgdaFunction{ap} \AgdaBound{σ} \AgdaBound{M}\<%
\\
\>[0]\AgdaIndent{4}{}\<[4]%
\>[4]\AgdaFunction{≡⟨} \AgdaFunction{ap-congr} \AgdaBound{M≡N} \AgdaFunction{⟩}\<%
\\
\>[4]\AgdaIndent{6}{}\<[6]%
\>[6]\AgdaFunction{ap} \AgdaBound{σ} \AgdaBound{N}\<%
\\
\>[0]\AgdaIndent{4}{}\<[4]%
\>[4]\AgdaFunction{∎}\<%
\end{code}
}

\AgdaHide{
\begin{code}%
\>\AgdaKeyword{open} \AgdaKeyword{import} \AgdaModule{Grammar.Base}\<%
\\
%
\\
\>\AgdaKeyword{module} \AgdaModule{Grammar.Substitution.LiftFamily} \AgdaSymbol{(}\AgdaBound{G} \AgdaSymbol{:} \AgdaRecord{Grammar}\AgdaSymbol{)} \AgdaKeyword{where}\<%
\\
\>\AgdaKeyword{open} \AgdaKeyword{import} \AgdaModule{Prelims}\<%
\\
\>\AgdaKeyword{open} \AgdaModule{Grammar} \AgdaBound{G}\<%
\\
\>\AgdaKeyword{open} \AgdaKeyword{import} \AgdaModule{Grammar.OpFamily.LiftFamily} \AgdaBound{G}\<%
\\
\>\AgdaKeyword{open} \AgdaKeyword{import} \AgdaModule{Grammar.Substitution.PreOpFamily} \AgdaBound{G}\<%
\\
\>\AgdaKeyword{open} \AgdaKeyword{import} \AgdaModule{Grammar.Substitution.Lifting} \AgdaBound{G}\<%
\\
\>\AgdaKeyword{open} \AgdaKeyword{import} \AgdaModule{Grammar.Substitution.RepSub} \AgdaBound{G}\<%
\end{code}
}

It is now easy to show that substitution forms a pre-family with lifting.  If $\sigma : U \rightarrow V$ and $x \in U$ then $(\sigma , K)(\uparrow x) \equiv
\sigma(x) \langle \uparrow \rangle \equiv (\sigma , K)(x) [ \uparrow ]$.

\begin{code}%
\>\AgdaFunction{SubLF} \AgdaSymbol{:} \AgdaRecord{LiftFamily}\<%
\\
\>\AgdaFunction{SubLF} \AgdaSymbol{=} \AgdaKeyword{record} \AgdaSymbol{\{} \<[17]%
\>[17]\<%
\\
\>[0]\AgdaIndent{2}{}\<[2]%
\>[2]\AgdaField{preOpFamily} \AgdaSymbol{=} \AgdaFunction{pre-substitution} \AgdaSymbol{;} \<[35]%
\>[35]\<%
\\
\>[0]\AgdaIndent{2}{}\<[2]%
\>[2]\AgdaField{lifting} \AgdaSymbol{=} \AgdaFunction{LIFTSUB} \AgdaSymbol{;} \<[22]%
\>[22]\<%
\\
\>[0]\AgdaIndent{2}{}\<[2]%
\>[2]\AgdaField{isLiftFamily} \AgdaSymbol{=} \AgdaKeyword{record} \AgdaSymbol{\{} \<[26]%
\>[26]\<%
\\
\>[2]\AgdaIndent{4}{}\<[4]%
\>[4]\AgdaField{liftOp-x₀} \AgdaSymbol{=} \AgdaInductiveConstructor{refl} \AgdaSymbol{;} \<[23]%
\>[23]\<%
\\
\>[2]\AgdaIndent{4}{}\<[4]%
\>[4]\AgdaField{liftOp-↑} \AgdaSymbol{=} \AgdaSymbol{λ} \AgdaSymbol{\{}\AgdaBound{\_}\AgdaSymbol{\}} \AgdaSymbol{\{}\AgdaBound{\_}\AgdaSymbol{\}} \AgdaSymbol{\{}\AgdaBound{\_}\AgdaSymbol{\}} \AgdaSymbol{\{}\AgdaBound{\_}\AgdaSymbol{\}} \AgdaSymbol{\{}\AgdaBound{σ}\AgdaSymbol{\}} \AgdaBound{x} \AgdaSymbol{→} \AgdaFunction{rep-is-sub} \AgdaSymbol{(}\AgdaBound{σ} \AgdaSymbol{\_} \AgdaBound{x}\AgdaSymbol{)} \AgdaSymbol{\}\}}\<%
\end{code}

\AgdaHide{
\begin{code}%
\>\AgdaKeyword{open} \AgdaKeyword{import} \AgdaModule{Grammar.Base}\<%
\\
%
\\
\>\AgdaKeyword{module} \AgdaModule{Grammar.OpFamily.Composition} \AgdaSymbol{(}\AgdaBound{A} \AgdaSymbol{:} \AgdaRecord{Grammar}\AgdaSymbol{)} \AgdaKeyword{where}\<%
\\
\>\AgdaKeyword{open} \AgdaKeyword{import} \AgdaModule{Data.List}\<%
\\
\>\AgdaKeyword{open} \AgdaKeyword{import} \AgdaModule{Prelims}\<%
\\
\>\AgdaKeyword{open} \AgdaModule{Grammar} \AgdaBound{A}\<%
\\
\>\AgdaKeyword{open} \AgdaKeyword{import} \AgdaModule{Grammar.OpFamily.LiftFamily} \AgdaBound{A}\<%
\\
%
\\
\>\AgdaKeyword{open} \AgdaModule{LiftFamily}\<%
\end{code}
}

\subsubsection{Compositions}

Let $F$, $G$ and $H$ be three pre-families with lifting.  A \emph{composition} $\circ : F;G \rightarrow H$ is a family of functions
\[ \circ_{UVW} : F[V,W] \times G[U,V] \rightarrow H[U,W] \]
for all alphabets $U$, $V$ and $W$ such that:
\begin{itemize}
\item
$(\sigma \circ \rho , K) \sim (\sigma , K) \circ (\rho , K)$
\item
$(\sigma \circ \rho)(x) \equiv \rho(x) [ \sigma ]$
\end{itemize}

\begin{code}%
\>\AgdaKeyword{record} \AgdaRecord{Composition} \AgdaSymbol{(}\AgdaBound{F} \AgdaBound{G} \AgdaBound{H} \AgdaSymbol{:} \AgdaRecord{LiftFamily}\AgdaSymbol{)} \AgdaSymbol{:} \AgdaPrimitiveType{Set} \AgdaKeyword{where}\<%
\\
\>[0]\AgdaIndent{2}{}\<[2]%
\>[2]\AgdaKeyword{field}\<%
\\
\>[2]\AgdaIndent{4}{}\<[4]%
\>[4]\AgdaField{circ} \AgdaSymbol{:} \AgdaSymbol{∀} \AgdaSymbol{\{}\AgdaBound{U}\AgdaSymbol{\}} \AgdaSymbol{\{}\AgdaBound{V}\AgdaSymbol{\}} \AgdaSymbol{\{}\AgdaBound{W}\AgdaSymbol{\}} \AgdaSymbol{→} \AgdaFunction{Op} \AgdaBound{F} \AgdaBound{V} \AgdaBound{W} \AgdaSymbol{→} \AgdaFunction{Op} \AgdaBound{G} \AgdaBound{U} \AgdaBound{V} \AgdaSymbol{→} \AgdaFunction{Op} \AgdaBound{H} \AgdaBound{U} \AgdaBound{W}\<%
\\
\>[2]\AgdaIndent{4}{}\<[4]%
\>[4]\AgdaField{liftOp-circ} \AgdaSymbol{:} \AgdaSymbol{∀} \AgdaSymbol{\{}\AgdaBound{U} \AgdaBound{V} \AgdaBound{W} \AgdaBound{K} \AgdaBound{σ} \AgdaBound{ρ}\AgdaSymbol{\}} \AgdaSymbol{→} \<[36]%
\>[36]\<%
\\
\>[4]\AgdaIndent{6}{}\<[6]%
\>[6]\AgdaFunction{\_∼op\_} \AgdaBound{H} \AgdaSymbol{(}\AgdaFunction{liftOp} \AgdaBound{H} \AgdaBound{K} \AgdaSymbol{(}\AgdaField{circ} \AgdaSymbol{\{}\AgdaBound{U}\AgdaSymbol{\}} \AgdaSymbol{\{}\AgdaBound{V}\AgdaSymbol{\}} \AgdaSymbol{\{}\AgdaBound{W}\AgdaSymbol{\}} \AgdaBound{σ} \AgdaBound{ρ}\AgdaSymbol{))} \<[50]%
\>[50]\<%
\\
\>[6]\AgdaIndent{8}{}\<[8]%
\>[8]\AgdaSymbol{(}\AgdaField{circ} \AgdaSymbol{(}\AgdaFunction{liftOp} \AgdaBound{F} \AgdaBound{K} \AgdaBound{σ}\AgdaSymbol{)} \AgdaSymbol{(}\AgdaFunction{liftOp} \AgdaBound{G} \AgdaBound{K} \AgdaBound{ρ}\AgdaSymbol{))}\<%
\\
\>[0]\AgdaIndent{4}{}\<[4]%
\>[4]\AgdaField{apV-circ} \AgdaSymbol{:} \AgdaSymbol{∀} \AgdaSymbol{\{}\AgdaBound{U}\AgdaSymbol{\}} \AgdaSymbol{\{}\AgdaBound{V}\AgdaSymbol{\}} \AgdaSymbol{\{}\AgdaBound{W}\AgdaSymbol{\}} \AgdaSymbol{\{}\AgdaBound{K}\AgdaSymbol{\}} \AgdaSymbol{\{}\AgdaBound{σ}\AgdaSymbol{\}} \AgdaSymbol{\{}\AgdaBound{ρ}\AgdaSymbol{\}} \AgdaSymbol{\{}\AgdaBound{x} \AgdaSymbol{:} \AgdaDatatype{Var} \AgdaBound{U} \AgdaBound{K}\AgdaSymbol{\}} \AgdaSymbol{→} \<[57]%
\>[57]\<%
\\
\>[4]\AgdaIndent{6}{}\<[6]%
\>[6]\AgdaFunction{apV} \AgdaBound{H} \AgdaSymbol{(}\AgdaField{circ} \AgdaSymbol{\{}\AgdaBound{U}\AgdaSymbol{\}} \AgdaSymbol{\{}\AgdaBound{V}\AgdaSymbol{\}} \AgdaSymbol{\{}\AgdaBound{W}\AgdaSymbol{\}} \AgdaBound{σ} \AgdaBound{ρ}\AgdaSymbol{)} \AgdaBound{x} \AgdaDatatype{≡} \AgdaFunction{ap} \AgdaBound{F} \AgdaBound{σ} \AgdaSymbol{(}\AgdaFunction{apV} \AgdaBound{G} \AgdaBound{ρ} \AgdaBound{x}\AgdaSymbol{)}\<%
\end{code}

\begin{lemma}
For any composition $\circ$:
\begin{enumerate}
\item
If $\sigma \sim \sigma'$ and $\rho \sim \rho'$ then $\sigma \circ \rho \sim \sigma' \circ \rho'$
\item
$(\sigma \circ \rho)^A \sim \sigma^A \circ \rho^A$
\item
$E [ \sigma \circ \rho ] \equiv E [ \rho ] [ \sigma ]$
\end{enumerate}
\end{lemma}

\begin{code}%
\>[0]\AgdaIndent{2}{}\<[2]%
\>[2]\AgdaFunction{circ-cong} \AgdaSymbol{:} \AgdaSymbol{∀} \AgdaSymbol{\{}\AgdaBound{U} \AgdaBound{V} \AgdaBound{W}\AgdaSymbol{\}} \AgdaSymbol{\{}\AgdaBound{σ} \AgdaBound{σ'} \AgdaSymbol{:} \AgdaFunction{Op} \AgdaBound{F} \AgdaBound{V} \AgdaBound{W}\AgdaSymbol{\}} \AgdaSymbol{\{}\AgdaBound{ρ} \AgdaBound{ρ'} \AgdaSymbol{:} \AgdaFunction{Op} \AgdaBound{G} \AgdaBound{U} \AgdaBound{V}\AgdaSymbol{\}} \AgdaSymbol{→} \<[62]%
\>[62]\<%
\\
\>[2]\AgdaIndent{4}{}\<[4]%
\>[4]\AgdaFunction{\_∼op\_} \AgdaBound{F} \AgdaBound{σ} \AgdaBound{σ'} \AgdaSymbol{→} \AgdaFunction{\_∼op\_} \AgdaBound{G} \AgdaBound{ρ} \AgdaBound{ρ'} \AgdaSymbol{→} \AgdaFunction{\_∼op\_} \AgdaBound{H} \AgdaSymbol{(}\AgdaField{circ} \AgdaBound{σ} \AgdaBound{ρ}\AgdaSymbol{)} \AgdaSymbol{(}\AgdaField{circ} \AgdaBound{σ'} \AgdaBound{ρ'}\AgdaSymbol{)}\<%
\end{code}

\AgdaHide{
\begin{code}%
\>[0]\AgdaIndent{2}{}\<[2]%
\>[2]\AgdaFunction{circ-cong} \AgdaSymbol{\{}\AgdaBound{U}\AgdaSymbol{\}} \AgdaSymbol{\{}\AgdaBound{V}\AgdaSymbol{\}} \AgdaSymbol{\{}\AgdaBound{W}\AgdaSymbol{\}} \AgdaSymbol{\{}\AgdaBound{σ}\AgdaSymbol{\}} \AgdaSymbol{\{}\AgdaBound{σ'}\AgdaSymbol{\}} \AgdaSymbol{\{}\AgdaBound{ρ}\AgdaSymbol{\}} \AgdaSymbol{\{}\AgdaBound{ρ'}\AgdaSymbol{\}} \AgdaBound{σ∼σ'} \AgdaBound{ρ∼ρ'} \AgdaBound{x} \AgdaSymbol{=} \AgdaKeyword{let} \AgdaKeyword{open} \AgdaModule{≡-Reasoning} \AgdaKeyword{in} \<[80]%
\>[80]\<%
\\
\>[2]\AgdaIndent{4}{}\<[4]%
\>[4]\AgdaFunction{begin}\<%
\\
\>[4]\AgdaIndent{6}{}\<[6]%
\>[6]\AgdaFunction{apV} \AgdaBound{H} \AgdaSymbol{(}\AgdaField{circ} \AgdaBound{σ} \AgdaBound{ρ}\AgdaSymbol{)} \AgdaBound{x}\<%
\\
\>[0]\AgdaIndent{4}{}\<[4]%
\>[4]\AgdaFunction{≡⟨} \AgdaField{apV-circ} \AgdaFunction{⟩}\<%
\\
\>[4]\AgdaIndent{6}{}\<[6]%
\>[6]\AgdaFunction{ap} \AgdaBound{F} \AgdaBound{σ} \AgdaSymbol{(}\AgdaFunction{apV} \AgdaBound{G} \AgdaBound{ρ} \AgdaBound{x}\AgdaSymbol{)}\<%
\\
\>[0]\AgdaIndent{4}{}\<[4]%
\>[4]\AgdaFunction{≡⟨} \AgdaFunction{ap-cong} \AgdaBound{F} \AgdaBound{σ∼σ'} \AgdaSymbol{(}\AgdaBound{ρ∼ρ'} \AgdaBound{x}\AgdaSymbol{)} \AgdaFunction{⟩}\<%
\\
\>[4]\AgdaIndent{6}{}\<[6]%
\>[6]\AgdaFunction{ap} \AgdaBound{F} \AgdaBound{σ'} \AgdaSymbol{(}\AgdaFunction{apV} \AgdaBound{G} \AgdaBound{ρ'} \AgdaBound{x}\AgdaSymbol{)}\<%
\\
\>[0]\AgdaIndent{4}{}\<[4]%
\>[4]\AgdaFunction{≡⟨⟨} \AgdaField{apV-circ} \AgdaFunction{⟩⟩}\<%
\\
\>[4]\AgdaIndent{6}{}\<[6]%
\>[6]\AgdaFunction{apV} \AgdaBound{H} \AgdaSymbol{(}\AgdaField{circ} \AgdaBound{σ'} \AgdaBound{ρ'}\AgdaSymbol{)} \AgdaBound{x}\<%
\\
\>[0]\AgdaIndent{4}{}\<[4]%
\>[4]\AgdaFunction{∎}\<%
\end{code}
}

\begin{code}%
\>[0]\AgdaIndent{2}{}\<[2]%
\>[2]\AgdaFunction{liftOp'-circ} \AgdaSymbol{:} \AgdaSymbol{∀} \AgdaSymbol{\{}\AgdaBound{U} \AgdaBound{V} \AgdaBound{W}\AgdaSymbol{\}} \AgdaBound{A} \AgdaSymbol{\{}\AgdaBound{σ} \AgdaBound{ρ}\AgdaSymbol{\}} \AgdaSymbol{→} \<[37]%
\>[37]\<%
\\
\>[2]\AgdaIndent{4}{}\<[4]%
\>[4]\AgdaFunction{\_∼op\_} \AgdaBound{H} \AgdaSymbol{(}\AgdaFunction{liftOp'} \AgdaBound{H} \AgdaBound{A} \AgdaSymbol{(}\AgdaField{circ} \AgdaSymbol{\{}\AgdaBound{U}\AgdaSymbol{\}} \AgdaSymbol{\{}\AgdaBound{V}\AgdaSymbol{\}} \AgdaSymbol{\{}\AgdaBound{W}\AgdaSymbol{\}} \AgdaBound{σ} \AgdaBound{ρ}\AgdaSymbol{))} \<[49]%
\>[49]\<%
\\
\>[4]\AgdaIndent{6}{}\<[6]%
\>[6]\AgdaSymbol{(}\AgdaField{circ} \AgdaSymbol{(}\AgdaFunction{liftOp'} \AgdaBound{F} \AgdaBound{A} \AgdaBound{σ}\AgdaSymbol{)} \AgdaSymbol{(}\AgdaFunction{liftOp'} \AgdaBound{G} \AgdaBound{A} \AgdaBound{ρ}\AgdaSymbol{))}\<%
\end{code}

\AgdaHide{
\begin{code}%
\>[0]\AgdaIndent{2}{}\<[2]%
\>[2]\AgdaFunction{liftOp'-circ} \AgdaInductiveConstructor{[]} \AgdaSymbol{=} \AgdaFunction{∼-refl} \AgdaBound{H}\<%
\\
\>[0]\AgdaIndent{2}{}\<[2]%
\>[2]\AgdaFunction{liftOp'-circ} \AgdaSymbol{\{}\AgdaBound{U}\AgdaSymbol{\}} \AgdaSymbol{\{}\AgdaBound{V}\AgdaSymbol{\}} \AgdaSymbol{\{}\AgdaBound{W}\AgdaSymbol{\}} \AgdaSymbol{(}\AgdaBound{K} \AgdaInductiveConstructor{∷} \AgdaBound{A}\AgdaSymbol{)} \AgdaSymbol{\{}\AgdaBound{σ}\AgdaSymbol{\}} \AgdaSymbol{\{}\AgdaBound{ρ}\AgdaSymbol{\}} \AgdaSymbol{=} \AgdaKeyword{let} \AgdaKeyword{open} \AgdaModule{EqReasoning} \AgdaSymbol{(}\AgdaFunction{OP} \AgdaBound{H} \AgdaSymbol{\_} \AgdaSymbol{\_)} \AgdaKeyword{in} \<[80]%
\>[80]\<%
\\
\>[2]\AgdaIndent{4}{}\<[4]%
\>[4]\AgdaFunction{begin}\<%
\\
\>[4]\AgdaIndent{6}{}\<[6]%
\>[6]\AgdaFunction{liftOp'} \AgdaBound{H} \AgdaBound{A} \AgdaSymbol{(}\AgdaFunction{liftOp} \AgdaBound{H} \AgdaBound{K} \AgdaSymbol{(}\AgdaField{circ} \AgdaBound{σ} \AgdaBound{ρ}\AgdaSymbol{))}\<%
\\
\>[0]\AgdaIndent{4}{}\<[4]%
\>[4]\AgdaFunction{≈⟨} \AgdaFunction{liftOp'-cong} \AgdaBound{H} \AgdaBound{A} \AgdaField{liftOp-circ} \AgdaFunction{⟩}\<%
\\
\>[4]\AgdaIndent{6}{}\<[6]%
\>[6]\AgdaFunction{liftOp'} \AgdaBound{H} \AgdaBound{A} \AgdaSymbol{(}\AgdaField{circ} \AgdaSymbol{(}\AgdaFunction{liftOp} \AgdaBound{F} \AgdaBound{K} \AgdaBound{σ}\AgdaSymbol{)} \AgdaSymbol{(}\AgdaFunction{liftOp} \AgdaBound{G} \AgdaBound{K} \AgdaBound{ρ}\AgdaSymbol{))}\<%
\\
\>[0]\AgdaIndent{4}{}\<[4]%
\>[4]\AgdaFunction{≈⟨} \AgdaFunction{liftOp'-circ} \AgdaBound{A} \AgdaFunction{⟩}\<%
\\
\>[4]\AgdaIndent{6}{}\<[6]%
\>[6]\AgdaField{circ} \AgdaSymbol{(}\AgdaFunction{liftOp'} \AgdaBound{F} \AgdaBound{A} \AgdaSymbol{(}\AgdaFunction{liftOp} \AgdaBound{F} \AgdaBound{K} \AgdaBound{σ}\AgdaSymbol{))} \AgdaSymbol{(}\AgdaFunction{liftOp'} \AgdaBound{G} \AgdaBound{A} \AgdaSymbol{(}\AgdaFunction{liftOp} \AgdaBound{G} \AgdaBound{K} \AgdaBound{ρ}\AgdaSymbol{))}\<%
\\
\>[0]\AgdaIndent{4}{}\<[4]%
\>[4]\AgdaFunction{∎}\<%
\end{code}
}

\begin{code}%
\>[0]\AgdaIndent{2}{}\<[2]%
\>[2]\AgdaFunction{ap-circ} \AgdaSymbol{:} \AgdaSymbol{∀} \AgdaSymbol{\{}\AgdaBound{U} \AgdaBound{V} \AgdaBound{W} \AgdaBound{C} \AgdaBound{K}\AgdaSymbol{\}} \AgdaSymbol{(}\AgdaBound{E} \AgdaSymbol{:} \AgdaDatatype{Subexpression} \AgdaBound{U} \AgdaBound{C} \AgdaBound{K}\AgdaSymbol{)} \AgdaSymbol{\{}\AgdaBound{σ} \AgdaBound{ρ}\AgdaSymbol{\}} \AgdaSymbol{→} \<[60]%
\>[60]\<%
\\
\>[2]\AgdaIndent{4}{}\<[4]%
\>[4]\AgdaFunction{ap} \AgdaBound{H} \AgdaSymbol{(}\AgdaField{circ} \AgdaSymbol{\{}\AgdaBound{U}\AgdaSymbol{\}} \AgdaSymbol{\{}\AgdaBound{V}\AgdaSymbol{\}} \AgdaSymbol{\{}\AgdaBound{W}\AgdaSymbol{\}} \AgdaBound{σ} \AgdaBound{ρ}\AgdaSymbol{)} \AgdaBound{E} \AgdaDatatype{≡} \AgdaFunction{ap} \AgdaBound{F} \AgdaBound{σ} \AgdaSymbol{(}\AgdaFunction{ap} \AgdaBound{G} \AgdaBound{ρ} \AgdaBound{E}\AgdaSymbol{)}\<%
\end{code}

\AgdaHide{
\begin{code}%
\>[0]\AgdaIndent{2}{}\<[2]%
\>[2]\AgdaFunction{ap-circ} \AgdaSymbol{(}\AgdaInductiveConstructor{var} \AgdaSymbol{\_)} \AgdaSymbol{=} \AgdaField{apV-circ}\<%
\\
\>[0]\AgdaIndent{2}{}\<[2]%
\>[2]\AgdaFunction{ap-circ} \AgdaSymbol{(}\AgdaInductiveConstructor{app} \AgdaBound{c} \AgdaBound{E}\AgdaSymbol{)} \AgdaSymbol{=} \AgdaFunction{cong} \AgdaSymbol{(}\AgdaInductiveConstructor{app} \AgdaBound{c}\AgdaSymbol{)} \AgdaSymbol{(}\AgdaFunction{ap-circ} \AgdaBound{E}\AgdaSymbol{)}\<%
\\
\>[0]\AgdaIndent{2}{}\<[2]%
\>[2]\AgdaFunction{ap-circ} \AgdaInductiveConstructor{[]} \AgdaSymbol{=} \AgdaInductiveConstructor{refl}\<%
\\
\>[0]\AgdaIndent{2}{}\<[2]%
\>[2]\AgdaFunction{ap-circ} \AgdaSymbol{(}\AgdaInductiveConstructor{\_∷\_} \AgdaSymbol{\{}\AgdaArgument{A} \AgdaSymbol{=} \AgdaInductiveConstructor{SK} \AgdaBound{A} \AgdaSymbol{\_\}} \AgdaBound{E} \AgdaBound{E'}\AgdaSymbol{)} \AgdaSymbol{\{}\AgdaBound{σ}\AgdaSymbol{\}} \AgdaSymbol{\{}\AgdaBound{ρ}\AgdaSymbol{\}} \AgdaSymbol{=} \AgdaFunction{cong₂} \AgdaInductiveConstructor{\_∷\_}\<%
\\
\>[2]\AgdaIndent{4}{}\<[4]%
\>[4]\AgdaSymbol{(}\AgdaKeyword{let} \AgdaKeyword{open} \AgdaModule{≡-Reasoning} \AgdaKeyword{in} \<[29]%
\>[29]\<%
\\
\>[2]\AgdaIndent{4}{}\<[4]%
\>[4]\AgdaFunction{begin}\<%
\\
\>[4]\AgdaIndent{6}{}\<[6]%
\>[6]\AgdaFunction{ap} \AgdaBound{H} \AgdaSymbol{(}\AgdaFunction{liftOp'} \AgdaBound{H} \AgdaBound{A} \AgdaSymbol{(}\AgdaField{circ} \AgdaBound{σ} \AgdaBound{ρ}\AgdaSymbol{))} \AgdaBound{E}\<%
\\
\>[0]\AgdaIndent{4}{}\<[4]%
\>[4]\AgdaFunction{≡⟨} \AgdaFunction{ap-congl} \AgdaBound{H} \AgdaSymbol{(}\AgdaFunction{liftOp'-circ} \AgdaBound{A}\AgdaSymbol{)} \AgdaBound{E} \AgdaFunction{⟩}\<%
\\
\>[4]\AgdaIndent{6}{}\<[6]%
\>[6]\AgdaFunction{ap} \AgdaBound{H} \AgdaSymbol{(}\AgdaField{circ} \AgdaSymbol{(}\AgdaFunction{liftOp'} \AgdaBound{F} \AgdaBound{A} \AgdaBound{σ}\AgdaSymbol{)} \AgdaSymbol{(}\AgdaFunction{liftOp'} \AgdaBound{G} \AgdaBound{A} \AgdaBound{ρ}\AgdaSymbol{))} \AgdaBound{E}\<%
\\
\>[0]\AgdaIndent{4}{}\<[4]%
\>[4]\AgdaFunction{≡⟨} \AgdaFunction{ap-circ} \AgdaBound{E} \AgdaFunction{⟩}\<%
\\
\>[4]\AgdaIndent{6}{}\<[6]%
\>[6]\AgdaFunction{ap} \AgdaBound{F} \AgdaSymbol{(}\AgdaFunction{liftOp'} \AgdaBound{F} \AgdaBound{A} \AgdaBound{σ}\AgdaSymbol{)} \AgdaSymbol{(}\AgdaFunction{ap} \AgdaBound{G} \AgdaSymbol{(}\AgdaFunction{liftOp'} \AgdaBound{G} \AgdaBound{A} \AgdaBound{ρ}\AgdaSymbol{)} \AgdaBound{E}\AgdaSymbol{)}\<%
\\
\>[0]\AgdaIndent{4}{}\<[4]%
\>[4]\AgdaFunction{∎}\AgdaSymbol{)} \<[7]%
\>[7]\<%
\\
\>[0]\AgdaIndent{4}{}\<[4]%
\>[4]\AgdaSymbol{(}\AgdaFunction{ap-circ} \AgdaBound{E'}\AgdaSymbol{)}\<%
\\
%
\\
\>\AgdaFunction{ap-circ-sim} \AgdaSymbol{:} \AgdaSymbol{∀} \AgdaSymbol{\{}\AgdaBound{F} \AgdaBound{F'} \AgdaBound{G} \AgdaBound{G'} \AgdaBound{H}\AgdaSymbol{\}} \AgdaSymbol{(}\AgdaBound{circ₁} \AgdaSymbol{:} \AgdaRecord{Composition} \AgdaBound{F} \AgdaBound{G} \AgdaBound{H}\AgdaSymbol{)} \AgdaSymbol{(}\AgdaBound{circ₂} \AgdaSymbol{:} \AgdaRecord{Composition} \AgdaBound{F'} \AgdaBound{G'} \AgdaBound{H}\AgdaSymbol{)} \AgdaSymbol{\{}\AgdaBound{U}\AgdaSymbol{\}} \AgdaSymbol{\{}\AgdaBound{V}\AgdaSymbol{\}} \AgdaSymbol{\{}\AgdaBound{V'}\AgdaSymbol{\}} \AgdaSymbol{\{}\AgdaBound{W}\AgdaSymbol{\}}\<%
\\
\>[0]\AgdaIndent{2}{}\<[2]%
\>[2]\AgdaSymbol{\{}\AgdaBound{σ} \AgdaSymbol{:} \AgdaFunction{Op} \AgdaBound{F} \AgdaBound{V} \AgdaBound{W}\AgdaSymbol{\}} \AgdaSymbol{\{}\AgdaBound{ρ} \AgdaSymbol{:} \AgdaFunction{Op} \AgdaBound{G} \AgdaBound{U} \AgdaBound{V}\AgdaSymbol{\}} \AgdaSymbol{\{}\AgdaBound{σ'} \AgdaSymbol{:} \AgdaFunction{Op} \AgdaBound{F'} \AgdaBound{V'} \AgdaBound{W}\AgdaSymbol{\}} \AgdaSymbol{\{}\AgdaBound{ρ'} \AgdaSymbol{:} \AgdaFunction{Op} \AgdaBound{G'} \AgdaBound{U} \AgdaBound{V'}\AgdaSymbol{\}} \AgdaSymbol{→}\<%
\\
\>[0]\AgdaIndent{2}{}\<[2]%
\>[2]\AgdaFunction{\_∼op\_} \AgdaBound{H} \AgdaSymbol{(}\AgdaField{Composition.circ} \AgdaBound{circ₁} \AgdaBound{σ} \AgdaBound{ρ}\AgdaSymbol{)} \AgdaSymbol{(}\AgdaField{Composition.circ} \AgdaBound{circ₂} \AgdaBound{σ'} \AgdaBound{ρ'}\AgdaSymbol{)} \AgdaSymbol{→}\<%
\\
\>[0]\AgdaIndent{2}{}\<[2]%
\>[2]\AgdaSymbol{∀} \AgdaSymbol{\{}\AgdaBound{C}\AgdaSymbol{\}} \AgdaSymbol{\{}\AgdaBound{K}\AgdaSymbol{\}} \AgdaSymbol{(}\AgdaBound{E} \AgdaSymbol{:} \AgdaDatatype{Subexpression} \AgdaBound{U} \AgdaBound{C} \AgdaBound{K}\AgdaSymbol{)} \AgdaSymbol{→}\<%
\\
\>[0]\AgdaIndent{2}{}\<[2]%
\>[2]\AgdaFunction{ap} \AgdaBound{F} \AgdaBound{σ} \AgdaSymbol{(}\AgdaFunction{ap} \AgdaBound{G} \AgdaBound{ρ} \AgdaBound{E}\AgdaSymbol{)} \AgdaDatatype{≡} \AgdaFunction{ap} \AgdaBound{F'} \AgdaBound{σ'} \AgdaSymbol{(}\AgdaFunction{ap} \AgdaBound{G'} \AgdaBound{ρ'} \AgdaBound{E}\AgdaSymbol{)}\<%
\\
\>\AgdaFunction{ap-circ-sim} \AgdaSymbol{\{}\AgdaBound{F}\AgdaSymbol{\}} \AgdaSymbol{\{}\AgdaBound{F'}\AgdaSymbol{\}} \AgdaSymbol{\{}\AgdaBound{G}\AgdaSymbol{\}} \AgdaSymbol{\{}\AgdaBound{G'}\AgdaSymbol{\}} \AgdaSymbol{\{}\AgdaBound{H}\AgdaSymbol{\}} \AgdaBound{circ₁} \AgdaBound{circ₂} \AgdaSymbol{\{}\AgdaBound{U}\AgdaSymbol{\}} \AgdaSymbol{\{}\AgdaBound{V}\AgdaSymbol{\}} \AgdaSymbol{\{}\AgdaBound{V'}\AgdaSymbol{\}} \AgdaSymbol{\{}\AgdaBound{W}\AgdaSymbol{\}} \AgdaSymbol{\{}\AgdaBound{σ}\AgdaSymbol{\}} \AgdaSymbol{\{}\AgdaBound{ρ}\AgdaSymbol{\}} \AgdaSymbol{\{}\AgdaBound{σ'}\AgdaSymbol{\}} \AgdaSymbol{\{}\AgdaBound{ρ'}\AgdaSymbol{\}} \AgdaBound{hyp} \AgdaSymbol{\{}\AgdaBound{C}\AgdaSymbol{\}} \AgdaSymbol{\{}\AgdaBound{K}\AgdaSymbol{\}} \AgdaBound{E} \AgdaSymbol{=}\<%
\\
\>[0]\AgdaIndent{2}{}\<[2]%
\>[2]\AgdaKeyword{let} \AgdaKeyword{open} \AgdaModule{≡-Reasoning} \AgdaKeyword{in} \<[26]%
\>[26]\<%
\\
\>[0]\AgdaIndent{2}{}\<[2]%
\>[2]\AgdaFunction{begin}\<%
\\
\>[2]\AgdaIndent{4}{}\<[4]%
\>[4]\AgdaFunction{ap} \AgdaBound{F} \AgdaBound{σ} \AgdaSymbol{(}\AgdaFunction{ap} \AgdaBound{G} \AgdaBound{ρ} \AgdaBound{E}\AgdaSymbol{)}\<%
\\
\>[0]\AgdaIndent{2}{}\<[2]%
\>[2]\AgdaFunction{≡⟨⟨} \AgdaFunction{Composition.ap-circ} \AgdaBound{circ₁} \AgdaBound{E} \AgdaSymbol{\{}\AgdaBound{σ}\AgdaSymbol{\}} \AgdaSymbol{\{}\AgdaBound{ρ}\AgdaSymbol{\}} \AgdaFunction{⟩⟩}\<%
\\
\>[2]\AgdaIndent{4}{}\<[4]%
\>[4]\AgdaFunction{ap} \AgdaBound{H} \AgdaSymbol{(}\AgdaField{Composition.circ} \AgdaBound{circ₁} \AgdaBound{σ} \AgdaBound{ρ}\AgdaSymbol{)} \AgdaBound{E}\<%
\\
\>[0]\AgdaIndent{2}{}\<[2]%
\>[2]\AgdaFunction{≡⟨} \AgdaFunction{ap-congl} \AgdaBound{H} \AgdaBound{hyp} \AgdaBound{E} \AgdaFunction{⟩}\<%
\\
\>[2]\AgdaIndent{4}{}\<[4]%
\>[4]\AgdaFunction{ap} \AgdaBound{H} \AgdaSymbol{(}\AgdaField{Composition.circ} \AgdaBound{circ₂} \AgdaBound{σ'} \AgdaBound{ρ'}\AgdaSymbol{)} \AgdaBound{E}\<%
\\
\>[0]\AgdaIndent{2}{}\<[2]%
\>[2]\AgdaFunction{≡⟨} \AgdaFunction{Composition.ap-circ} \AgdaBound{circ₂} \AgdaBound{E} \AgdaSymbol{\{}\AgdaBound{σ'}\AgdaSymbol{\}} \AgdaSymbol{\{}\AgdaBound{ρ'}\AgdaSymbol{\}} \AgdaFunction{⟩}\<%
\\
\>[2]\AgdaIndent{4}{}\<[4]%
\>[4]\AgdaFunction{ap} \AgdaBound{F'} \AgdaBound{σ'} \AgdaSymbol{(}\AgdaFunction{ap} \AgdaBound{G'} \AgdaBound{ρ'} \AgdaBound{E}\AgdaSymbol{)}\<%
\\
\>[0]\AgdaIndent{2}{}\<[2]%
\>[2]\AgdaFunction{∎}\<%
\\
%
\\
\>\AgdaFunction{liftOp-up-mixed'} \AgdaSymbol{:} \AgdaSymbol{∀} \AgdaSymbol{\{}\AgdaBound{F}\AgdaSymbol{\}} \AgdaSymbol{\{}\AgdaBound{G}\AgdaSymbol{\}} \AgdaSymbol{\{}\AgdaBound{H}\AgdaSymbol{\}} \AgdaSymbol{\{}\AgdaBound{F'}\AgdaSymbol{\}} \AgdaSymbol{(}\AgdaBound{circ₁} \AgdaSymbol{:} \AgdaRecord{Composition} \AgdaBound{F} \AgdaBound{G} \AgdaBound{H}\AgdaSymbol{)} \AgdaSymbol{(}\AgdaBound{circ₂} \AgdaSymbol{:} \AgdaRecord{Composition} \AgdaBound{F'} \AgdaBound{F} \AgdaBound{H}\AgdaSymbol{)}\<%
\\
\>[0]\AgdaIndent{2}{}\<[2]%
\>[2]\AgdaSymbol{\{}\AgdaBound{U}\AgdaSymbol{\}} \AgdaSymbol{\{}\AgdaBound{V}\AgdaSymbol{\}} \AgdaSymbol{\{}\AgdaBound{K}\AgdaSymbol{\}} \AgdaSymbol{\{}\AgdaBound{σ} \AgdaSymbol{:} \AgdaFunction{Op} \AgdaBound{F} \AgdaBound{U} \AgdaBound{V}\AgdaSymbol{\}} \AgdaSymbol{→}\<%
\\
\>[0]\AgdaIndent{2}{}\<[2]%
\>[2]\AgdaSymbol{(∀} \AgdaSymbol{\{}\AgdaBound{V}\AgdaSymbol{\}} \AgdaSymbol{\{}\AgdaBound{C}\AgdaSymbol{\}} \AgdaSymbol{\{}\AgdaBound{K}\AgdaSymbol{\}} \AgdaSymbol{\{}\AgdaBound{L}\AgdaSymbol{\}} \AgdaSymbol{\{}\AgdaBound{E} \AgdaSymbol{:} \AgdaDatatype{Subexpression} \AgdaBound{V} \AgdaBound{C} \AgdaBound{K}\AgdaSymbol{\}} \AgdaSymbol{→} \AgdaFunction{ap} \AgdaBound{F} \AgdaSymbol{(}\AgdaFunction{up} \AgdaBound{F} \AgdaSymbol{\{}\AgdaBound{V}\AgdaSymbol{\}} \AgdaSymbol{\{}\AgdaBound{L}\AgdaSymbol{\})} \AgdaBound{E} \AgdaDatatype{≡} \AgdaFunction{ap} \AgdaBound{F'} \AgdaSymbol{(}\AgdaFunction{up} \AgdaBound{F'} \AgdaSymbol{\{}\AgdaBound{V}\AgdaSymbol{\}} \AgdaSymbol{\{}\AgdaBound{L}\AgdaSymbol{\})} \AgdaBound{E}\AgdaSymbol{)} \AgdaSymbol{→}\<%
\\
\>[0]\AgdaIndent{2}{}\<[2]%
\>[2]\AgdaFunction{\_∼op\_} \AgdaBound{H} \AgdaSymbol{(}\AgdaField{Composition.circ} \AgdaBound{circ₁} \AgdaSymbol{(}\AgdaFunction{liftOp} \AgdaBound{F} \AgdaBound{K} \AgdaBound{σ}\AgdaSymbol{)} \AgdaSymbol{(}\AgdaFunction{up} \AgdaBound{G}\AgdaSymbol{))} \AgdaSymbol{(}\AgdaField{Composition.circ} \AgdaBound{circ₂} \AgdaSymbol{(}\AgdaFunction{up} \AgdaBound{F'}\AgdaSymbol{)} \AgdaBound{σ}\AgdaSymbol{)}\<%
\\
\>\AgdaFunction{liftOp-up-mixed'} \AgdaSymbol{\{}\AgdaBound{F}\AgdaSymbol{\}} \AgdaSymbol{\{}\AgdaBound{G}\AgdaSymbol{\}} \AgdaSymbol{\{}\AgdaBound{H}\AgdaSymbol{\}} \AgdaSymbol{\{}\AgdaBound{F'}\AgdaSymbol{\}} \AgdaBound{circ₁} \AgdaBound{circ₂} \AgdaSymbol{\{}\AgdaBound{U}\AgdaSymbol{\}} \AgdaSymbol{\{}\AgdaBound{V}\AgdaSymbol{\}} \AgdaSymbol{\{}\AgdaBound{K}\AgdaSymbol{\}} \AgdaSymbol{\{}\AgdaBound{σ}\AgdaSymbol{\}} \AgdaBound{hyp} \AgdaSymbol{\{}\AgdaBound{L}\AgdaSymbol{\}} \AgdaBound{x} \AgdaSymbol{=} \<[74]%
\>[74]\<%
\\
\>[0]\AgdaIndent{2}{}\<[2]%
\>[2]\AgdaKeyword{let} \AgdaKeyword{open} \AgdaModule{≡-Reasoning} \AgdaKeyword{in} \<[26]%
\>[26]\<%
\\
\>[0]\AgdaIndent{2}{}\<[2]%
\>[2]\AgdaFunction{begin}\<%
\\
\>[2]\AgdaIndent{4}{}\<[4]%
\>[4]\AgdaFunction{apV} \AgdaBound{H} \AgdaSymbol{(}\AgdaField{Composition.circ} \AgdaBound{circ₁} \AgdaSymbol{(}\AgdaFunction{liftOp} \AgdaBound{F} \AgdaBound{K} \AgdaBound{σ}\AgdaSymbol{)} \AgdaSymbol{(}\AgdaFunction{up} \AgdaBound{G}\AgdaSymbol{))} \AgdaBound{x}\<%
\\
\>[0]\AgdaIndent{2}{}\<[2]%
\>[2]\AgdaFunction{≡⟨} \AgdaField{Composition.apV-circ} \AgdaBound{circ₁} \AgdaFunction{⟩}\<%
\\
\>[2]\AgdaIndent{4}{}\<[4]%
\>[4]\AgdaFunction{ap} \AgdaBound{F} \AgdaSymbol{(}\AgdaFunction{liftOp} \AgdaBound{F} \AgdaBound{K} \AgdaBound{σ}\AgdaSymbol{)} \AgdaSymbol{(}\AgdaFunction{apV} \AgdaBound{G} \AgdaSymbol{(}\AgdaFunction{up} \AgdaBound{G}\AgdaSymbol{)} \AgdaBound{x}\AgdaSymbol{)}\<%
\\
\>[0]\AgdaIndent{2}{}\<[2]%
\>[2]\AgdaFunction{≡⟨} \AgdaFunction{cong} \AgdaSymbol{(}\AgdaFunction{ap} \AgdaBound{F} \AgdaSymbol{(}\AgdaFunction{liftOp} \AgdaBound{F} \AgdaBound{K} \AgdaBound{σ}\AgdaSymbol{))} \AgdaSymbol{(}\AgdaFunction{apV-up} \AgdaBound{G}\AgdaSymbol{)} \AgdaFunction{⟩}\<%
\\
\>[2]\AgdaIndent{4}{}\<[4]%
\>[4]\AgdaFunction{apV} \AgdaBound{F} \AgdaSymbol{(}\AgdaFunction{liftOp} \AgdaBound{F} \AgdaBound{K} \AgdaBound{σ}\AgdaSymbol{)} \AgdaSymbol{(}\AgdaInductiveConstructor{↑} \AgdaBound{x}\AgdaSymbol{)}\<%
\\
\>[0]\AgdaIndent{2}{}\<[2]%
\>[2]\AgdaFunction{≡⟨} \AgdaFunction{liftOp-↑} \AgdaBound{F} \AgdaBound{x} \AgdaFunction{⟩}\<%
\\
\>[2]\AgdaIndent{4}{}\<[4]%
\>[4]\AgdaFunction{ap} \AgdaBound{F} \AgdaSymbol{(}\AgdaFunction{up} \AgdaBound{F}\AgdaSymbol{)} \AgdaSymbol{(}\AgdaFunction{apV} \AgdaBound{F} \AgdaBound{σ} \AgdaBound{x}\AgdaSymbol{)}\<%
\\
\>[0]\AgdaIndent{2}{}\<[2]%
\>[2]\AgdaFunction{≡⟨} \AgdaBound{hyp} \AgdaSymbol{\{}\AgdaArgument{E} \AgdaSymbol{=} \AgdaFunction{apV} \AgdaBound{F} \AgdaBound{σ} \AgdaBound{x}\AgdaSymbol{\}}\AgdaFunction{⟩}\<%
\\
\>[2]\AgdaIndent{4}{}\<[4]%
\>[4]\AgdaFunction{ap} \AgdaBound{F'} \AgdaSymbol{(}\AgdaFunction{up} \AgdaBound{F'}\AgdaSymbol{)} \AgdaSymbol{(}\AgdaFunction{apV} \AgdaBound{F} \AgdaBound{σ} \AgdaBound{x}\AgdaSymbol{)}\<%
\\
\>[0]\AgdaIndent{2}{}\<[2]%
\>[2]\AgdaFunction{≡⟨⟨} \AgdaField{Composition.apV-circ} \AgdaBound{circ₂} \AgdaFunction{⟩⟩}\<%
\\
\>[2]\AgdaIndent{4}{}\<[4]%
\>[4]\AgdaFunction{apV} \AgdaBound{H} \AgdaSymbol{(}\AgdaField{Composition.circ} \AgdaBound{circ₂} \AgdaSymbol{(}\AgdaFunction{up} \AgdaBound{F'}\AgdaSymbol{)} \AgdaBound{σ}\AgdaSymbol{)} \AgdaBound{x}\<%
\\
\>[0]\AgdaIndent{2}{}\<[2]%
\>[2]\AgdaFunction{∎}\<%
\\
%
\\
\>\AgdaFunction{liftOp-up-mixed} \AgdaSymbol{:} \AgdaSymbol{∀} \AgdaSymbol{\{}\AgdaBound{F}\AgdaSymbol{\}} \AgdaSymbol{\{}\AgdaBound{G}\AgdaSymbol{\}} \AgdaSymbol{\{}\AgdaBound{H}\AgdaSymbol{\}} \AgdaSymbol{\{}\AgdaBound{F'}\AgdaSymbol{\}} \AgdaSymbol{(}\AgdaBound{circ₁} \AgdaSymbol{:} \AgdaRecord{Composition} \AgdaBound{F} \AgdaBound{G} \AgdaBound{H}\AgdaSymbol{)} \AgdaSymbol{(}\AgdaBound{circ₂} \AgdaSymbol{:} \AgdaRecord{Composition} \AgdaBound{F'} \AgdaBound{F} \AgdaBound{H}\AgdaSymbol{)}\<%
\\
\>[0]\AgdaIndent{2}{}\<[2]%
\>[2]\AgdaSymbol{\{}\AgdaBound{U}\AgdaSymbol{\}} \AgdaSymbol{\{}\AgdaBound{V}\AgdaSymbol{\}} \AgdaSymbol{\{}\AgdaBound{C}\AgdaSymbol{\}} \AgdaSymbol{\{}\AgdaBound{K}\AgdaSymbol{\}} \AgdaSymbol{\{}\AgdaBound{L}\AgdaSymbol{\}} \AgdaSymbol{\{}\AgdaBound{σ} \AgdaSymbol{:} \AgdaFunction{Op} \AgdaBound{F} \AgdaBound{U} \AgdaBound{V}\AgdaSymbol{\}} \AgdaSymbol{→}\<%
\\
\>[0]\AgdaIndent{2}{}\<[2]%
\>[2]\AgdaSymbol{(∀} \AgdaSymbol{\{}\AgdaBound{V}\AgdaSymbol{\}} \AgdaSymbol{\{}\AgdaBound{C}\AgdaSymbol{\}} \AgdaSymbol{\{}\AgdaBound{K}\AgdaSymbol{\}} \AgdaSymbol{\{}\AgdaBound{L}\AgdaSymbol{\}} \AgdaSymbol{\{}\AgdaBound{E} \AgdaSymbol{:} \AgdaDatatype{Subexpression} \AgdaBound{V} \AgdaBound{C} \AgdaBound{K}\AgdaSymbol{\}} \AgdaSymbol{→} \AgdaFunction{ap} \AgdaBound{F} \AgdaSymbol{(}\AgdaFunction{up} \AgdaBound{F} \AgdaSymbol{\{}\AgdaBound{V}\AgdaSymbol{\}} \AgdaSymbol{\{}\AgdaBound{L}\AgdaSymbol{\})} \AgdaBound{E} \AgdaDatatype{≡} \AgdaFunction{ap} \AgdaBound{F'} \AgdaSymbol{(}\AgdaFunction{up} \AgdaBound{F'} \AgdaSymbol{\{}\AgdaBound{V}\AgdaSymbol{\}} \AgdaSymbol{\{}\AgdaBound{L}\AgdaSymbol{\})} \AgdaBound{E}\AgdaSymbol{)} \AgdaSymbol{→}\<%
\\
\>[0]\AgdaIndent{2}{}\<[2]%
\>[2]\AgdaSymbol{∀} \AgdaSymbol{\{}\AgdaBound{E} \AgdaSymbol{:} \AgdaDatatype{Subexpression} \AgdaBound{U} \AgdaBound{C} \AgdaBound{K}\AgdaSymbol{\}} \AgdaSymbol{→} \AgdaFunction{ap} \AgdaBound{F} \AgdaSymbol{(}\AgdaFunction{liftOp} \AgdaBound{F} \AgdaBound{L} \AgdaBound{σ}\AgdaSymbol{)} \AgdaSymbol{(}\AgdaFunction{ap} \AgdaBound{G} \AgdaSymbol{(}\AgdaFunction{up} \AgdaBound{G}\AgdaSymbol{)} \AgdaBound{E}\AgdaSymbol{)} \AgdaDatatype{≡} \AgdaFunction{ap} \AgdaBound{F'} \AgdaSymbol{(}\AgdaFunction{up} \AgdaBound{F'}\AgdaSymbol{)} \AgdaSymbol{(}\AgdaFunction{ap} \AgdaBound{F} \AgdaBound{σ} \AgdaBound{E}\AgdaSymbol{)}\<%
\\
\>\AgdaFunction{liftOp-up-mixed} \AgdaBound{circ₁} \AgdaBound{circ₂} \AgdaBound{hyp} \AgdaSymbol{\{}\AgdaArgument{E} \AgdaSymbol{=} \AgdaBound{E}\AgdaSymbol{\}} \AgdaSymbol{=} \AgdaFunction{ap-circ-sim} \AgdaBound{circ₁} \AgdaBound{circ₂} \AgdaSymbol{(}\AgdaFunction{liftOp-up-mixed'} \AgdaBound{circ₁} \AgdaBound{circ₂} \AgdaSymbol{(λ} \AgdaSymbol{\{}\AgdaBound{\_}\AgdaSymbol{\}} \AgdaSymbol{\{}\AgdaBound{\_}\AgdaSymbol{\}} \AgdaSymbol{\{}\AgdaBound{\_}\AgdaSymbol{\}} \AgdaSymbol{\{}\AgdaBound{\_}\AgdaSymbol{\}} \AgdaSymbol{\{}\AgdaBound{E}\AgdaSymbol{\}} \AgdaSymbol{→} \AgdaBound{hyp} \AgdaSymbol{\{}\AgdaArgument{E} \AgdaSymbol{=} \AgdaBound{E}\AgdaSymbol{\}))} \AgdaBound{E}\<%
\end{code}
}

\AgdaHide{
\begin{code}%
\>\AgdaKeyword{open} \AgdaKeyword{import} \AgdaModule{Grammar.Base}\<%
\\
%
\\
\>\AgdaKeyword{module} \AgdaModule{Grammar.Substitution.OpFamily} \AgdaSymbol{(}\AgdaBound{G} \AgdaSymbol{:} \AgdaRecord{Grammar}\AgdaSymbol{)} \AgdaKeyword{where}\<%
\\
\>\AgdaKeyword{open} \AgdaKeyword{import} \AgdaModule{Prelims}\<%
\\
\>\AgdaKeyword{open} \AgdaModule{Grammar} \AgdaBound{G}\<%
\\
\>\AgdaKeyword{open} \AgdaKeyword{import} \AgdaModule{Grammar.OpFamily} \AgdaBound{G}\<%
\\
\>\AgdaKeyword{open} \AgdaKeyword{import} \AgdaModule{Grammar.Replacement} \AgdaBound{G}\<%
\\
\>\AgdaKeyword{open} \AgdaKeyword{import} \AgdaModule{Grammar.Substitution.PreOpFamily} \AgdaBound{G}\<%
\\
\>\AgdaKeyword{open} \AgdaKeyword{import} \AgdaModule{Grammar.Substitution.Lifting} \AgdaBound{G}\<%
\\
\>\AgdaKeyword{open} \AgdaKeyword{import} \AgdaModule{Grammar.Substitution.LiftFamily} \AgdaBound{G}\<%
\\
\>\AgdaKeyword{open} \AgdaKeyword{import} \AgdaModule{Grammar.Substitution.RepSub} \AgdaBound{G}\<%
\end{code}
}

We now define two compositions $\bullet_1 : \mathrm{replacement} ; \mathrm{substitution} \rightarrow \mathrm{substitution}$ and $\bullet_2 : \mathrm{substitution};\mathrm{replacement} \rightarrow \mathrm{substitution}$.

\begin{code}%
\>\AgdaKeyword{infixl} \AgdaNumber{60} \AgdaFixityOp{\_•RS\_}\<%
\\
\>\AgdaFunction{\_•RS\_} \AgdaSymbol{:} \AgdaSymbol{∀} \AgdaSymbol{\{}\AgdaBound{U}\AgdaSymbol{\}} \AgdaSymbol{\{}\AgdaBound{V}\AgdaSymbol{\}} \AgdaSymbol{\{}\AgdaBound{W}\AgdaSymbol{\}} \AgdaSymbol{→} \AgdaFunction{Rep} \AgdaBound{V} \AgdaBound{W} \AgdaSymbol{→} \AgdaFunction{Sub} \AgdaBound{U} \AgdaBound{V} \AgdaSymbol{→} \AgdaFunction{Sub} \AgdaBound{U} \AgdaBound{W}\<%
\\
\>\AgdaSymbol{(}\AgdaBound{ρ} \AgdaFunction{•RS} \AgdaBound{σ}\AgdaSymbol{)} \AgdaBound{K} \AgdaBound{x} \AgdaSymbol{=} \AgdaSymbol{(}\AgdaBound{σ} \AgdaBound{K} \AgdaBound{x}\AgdaSymbol{)} \AgdaFunction{〈} \AgdaBound{ρ} \AgdaFunction{〉}\<%
\\
%
\\
\>\AgdaFunction{Sub↑-compRS} \AgdaSymbol{:} \AgdaSymbol{∀} \AgdaSymbol{\{}\AgdaBound{U}\AgdaSymbol{\}} \AgdaSymbol{\{}\AgdaBound{V}\AgdaSymbol{\}} \AgdaSymbol{\{}\AgdaBound{W}\AgdaSymbol{\}} \AgdaSymbol{\{}\AgdaBound{K}\AgdaSymbol{\}} \AgdaSymbol{\{}\AgdaBound{ρ} \AgdaSymbol{:} \AgdaFunction{Rep} \AgdaBound{V} \AgdaBound{W}\AgdaSymbol{\}} \AgdaSymbol{\{}\AgdaBound{σ} \AgdaSymbol{:} \AgdaFunction{Sub} \AgdaBound{U} \AgdaBound{V}\AgdaSymbol{\}} \AgdaSymbol{→} \AgdaFunction{Sub↑} \AgdaBound{K} \AgdaSymbol{(}\AgdaBound{ρ} \AgdaFunction{•RS} \AgdaBound{σ}\AgdaSymbol{)} \AgdaFunction{∼} \AgdaFunction{Rep↑} \AgdaBound{K} \AgdaBound{ρ} \AgdaFunction{•RS} \AgdaFunction{Sub↑} \AgdaBound{K} \AgdaBound{σ}\<%
\end{code}

\AgdaHide{
\begin{code}%
\>\AgdaFunction{Sub↑-compRS} \AgdaSymbol{\{}\AgdaArgument{K} \AgdaSymbol{=} \AgdaBound{K}\AgdaSymbol{\}} \AgdaInductiveConstructor{x₀} \AgdaSymbol{=} \AgdaInductiveConstructor{refl}\<%
\\
\>\AgdaFunction{Sub↑-compRS} \AgdaSymbol{\{}\AgdaBound{U}\AgdaSymbol{\}} \AgdaSymbol{\{}\AgdaBound{V}\AgdaSymbol{\}} \AgdaSymbol{\{}\AgdaBound{W}\AgdaSymbol{\}} \AgdaSymbol{\{}\AgdaBound{K}\AgdaSymbol{\}} \AgdaSymbol{\{}\AgdaBound{ρ}\AgdaSymbol{\}} \AgdaSymbol{\{}\AgdaBound{σ}\AgdaSymbol{\}} \AgdaSymbol{\{}\AgdaBound{L}\AgdaSymbol{\}} \AgdaSymbol{(}\AgdaInductiveConstructor{↑} \AgdaBound{x}\AgdaSymbol{)} \AgdaSymbol{=} \AgdaKeyword{let} \AgdaKeyword{open} \AgdaModule{≡-Reasoning} \AgdaSymbol{\{}\AgdaArgument{A} \AgdaSymbol{=} \AgdaFunction{Expression} \AgdaSymbol{(}\AgdaBound{W} \AgdaInductiveConstructor{,} \AgdaBound{K}\AgdaSymbol{)} \AgdaSymbol{(}\AgdaInductiveConstructor{varKind} \AgdaBound{L}\AgdaSymbol{)\}} \AgdaKeyword{in} \<[109]%
\>[109]\<%
\\
\>[0]\AgdaIndent{2}{}\<[2]%
\>[2]\AgdaFunction{begin} \<[8]%
\>[8]\<%
\\
\>[2]\AgdaIndent{4}{}\<[4]%
\>[4]\AgdaSymbol{(}\AgdaBound{σ} \AgdaBound{L} \AgdaBound{x}\AgdaSymbol{)} \AgdaFunction{〈} \AgdaBound{ρ} \AgdaFunction{〉} \AgdaFunction{〈} \AgdaFunction{upRep} \AgdaFunction{〉}\<%
\\
\>[0]\AgdaIndent{2}{}\<[2]%
\>[2]\AgdaFunction{≡⟨⟨} \AgdaFunction{rep-comp} \AgdaSymbol{(}\AgdaBound{σ} \AgdaBound{L} \AgdaBound{x}\AgdaSymbol{)} \AgdaFunction{⟩⟩}\<%
\\
\>[2]\AgdaIndent{4}{}\<[4]%
\>[4]\AgdaSymbol{(}\AgdaBound{σ} \AgdaBound{L} \AgdaBound{x}\AgdaSymbol{)} \AgdaFunction{〈} \AgdaFunction{upRep} \AgdaFunction{•R} \AgdaBound{ρ} \AgdaFunction{〉}\<%
\\
\>[0]\AgdaIndent{2}{}\<[2]%
\>[2]\AgdaFunction{≡⟨⟩}\<%
\\
\>[2]\AgdaIndent{4}{}\<[4]%
\>[4]\AgdaSymbol{(}\AgdaBound{σ} \AgdaBound{L} \AgdaBound{x}\AgdaSymbol{)} \AgdaFunction{〈} \AgdaFunction{Rep↑} \AgdaBound{K} \AgdaBound{ρ} \AgdaFunction{•R} \AgdaFunction{upRep} \AgdaFunction{〉}\<%
\\
\>[0]\AgdaIndent{2}{}\<[2]%
\>[2]\AgdaFunction{≡⟨} \AgdaFunction{rep-comp} \AgdaSymbol{(}\AgdaBound{σ} \AgdaBound{L} \AgdaBound{x}\AgdaSymbol{)} \AgdaFunction{⟩}\<%
\\
\>[2]\AgdaIndent{4}{}\<[4]%
\>[4]\AgdaSymbol{(}\AgdaBound{σ} \AgdaBound{L} \AgdaBound{x}\AgdaSymbol{)} \AgdaFunction{〈} \AgdaFunction{upRep} \AgdaFunction{〉} \AgdaFunction{〈} \AgdaFunction{Rep↑} \AgdaBound{K} \AgdaBound{ρ} \AgdaFunction{〉}\<%
\\
\>[0]\AgdaIndent{2}{}\<[2]%
\>[2]\AgdaFunction{∎}\<%
\end{code}
}

\begin{code}%
\>\AgdaFunction{COMPRS} \AgdaSymbol{:} \AgdaRecord{Composition} \AgdaFunction{proto-replacement} \AgdaFunction{proto-substitution} \AgdaFunction{proto-substitution}\<%
\\
\>\AgdaFunction{COMPRS} \AgdaSymbol{=} \AgdaKeyword{record} \AgdaSymbol{\{} \<[18]%
\>[18]\<%
\\
\>[0]\AgdaIndent{2}{}\<[2]%
\>[2]\AgdaField{circ} \AgdaSymbol{=} \AgdaFunction{\_•RS\_} \AgdaSymbol{;} \<[17]%
\>[17]\<%
\\
\>[0]\AgdaIndent{2}{}\<[2]%
\>[2]\AgdaField{liftOp-circ} \AgdaSymbol{=} \AgdaFunction{Sub↑-compRS} \AgdaSymbol{;} \<[30]%
\>[30]\<%
\\
\>[0]\AgdaIndent{2}{}\<[2]%
\>[2]\AgdaField{apV-circ} \AgdaSymbol{=} \AgdaInductiveConstructor{refl} \AgdaSymbol{\}}\<%
\\
%
\\
\>\AgdaFunction{sub-compRS} \AgdaSymbol{:} \AgdaSymbol{∀} \AgdaSymbol{\{}\AgdaBound{U}\AgdaSymbol{\}} \AgdaSymbol{\{}\AgdaBound{V}\AgdaSymbol{\}} \AgdaSymbol{\{}\AgdaBound{W}\AgdaSymbol{\}} \AgdaSymbol{\{}\AgdaBound{C}\AgdaSymbol{\}} \AgdaSymbol{\{}\AgdaBound{K}\AgdaSymbol{\}} \<[35]%
\>[35]\<%
\\
\>[0]\AgdaIndent{2}{}\<[2]%
\>[2]\AgdaSymbol{(}\AgdaBound{E} \AgdaSymbol{:} \AgdaDatatype{Subexpression} \AgdaBound{U} \AgdaBound{C} \AgdaBound{K}\AgdaSymbol{)} \AgdaSymbol{\{}\AgdaBound{ρ} \AgdaSymbol{:} \AgdaFunction{Rep} \AgdaBound{V} \AgdaBound{W}\AgdaSymbol{\}} \AgdaSymbol{\{}\AgdaBound{σ} \AgdaSymbol{:} \AgdaFunction{Sub} \AgdaBound{U} \AgdaBound{V}\AgdaSymbol{\}} \AgdaSymbol{→}\<%
\\
\>[0]\AgdaIndent{2}{}\<[2]%
\>[2]\AgdaBound{E} \AgdaFunction{⟦} \AgdaBound{ρ} \AgdaFunction{•RS} \AgdaBound{σ} \AgdaFunction{⟧} \AgdaDatatype{≡} \AgdaBound{E} \AgdaFunction{⟦} \AgdaBound{σ} \AgdaFunction{⟧} \AgdaFunction{〈} \AgdaBound{ρ} \AgdaFunction{〉}\<%
\\
\>\AgdaFunction{sub-compRS} \AgdaBound{E} \AgdaSymbol{=} \AgdaFunction{Composition.ap-circ} \AgdaFunction{COMPRS} \AgdaBound{E}\<%
\\
%
\\
\>\AgdaKeyword{infixl} \AgdaNumber{60} \AgdaFixityOp{\_•SR\_}\<%
\\
\>\AgdaFunction{\_•SR\_} \AgdaSymbol{:} \AgdaSymbol{∀} \AgdaSymbol{\{}\AgdaBound{U}\AgdaSymbol{\}} \AgdaSymbol{\{}\AgdaBound{V}\AgdaSymbol{\}} \AgdaSymbol{\{}\AgdaBound{W}\AgdaSymbol{\}} \AgdaSymbol{→} \AgdaFunction{Sub} \AgdaBound{V} \AgdaBound{W} \AgdaSymbol{→} \AgdaFunction{Rep} \AgdaBound{U} \AgdaBound{V} \AgdaSymbol{→} \AgdaFunction{Sub} \AgdaBound{U} \AgdaBound{W}\<%
\\
\>\AgdaSymbol{(}\AgdaBound{σ} \AgdaFunction{•SR} \AgdaBound{ρ}\AgdaSymbol{)} \AgdaBound{K} \AgdaBound{x} \AgdaSymbol{=} \AgdaBound{σ} \AgdaBound{K} \AgdaSymbol{(}\AgdaBound{ρ} \AgdaBound{K} \AgdaBound{x}\AgdaSymbol{)}\<%
\\
%
\\
\>\AgdaFunction{Sub↑-compSR} \AgdaSymbol{:} \AgdaSymbol{∀} \AgdaSymbol{\{}\AgdaBound{U}\AgdaSymbol{\}} \AgdaSymbol{\{}\AgdaBound{V}\AgdaSymbol{\}} \AgdaSymbol{\{}\AgdaBound{W}\AgdaSymbol{\}} \AgdaSymbol{\{}\AgdaBound{K}\AgdaSymbol{\}} \AgdaSymbol{\{}\AgdaBound{σ} \AgdaSymbol{:} \AgdaFunction{Sub} \AgdaBound{V} \AgdaBound{W}\AgdaSymbol{\}} \AgdaSymbol{\{}\AgdaBound{ρ} \AgdaSymbol{:} \AgdaFunction{Rep} \AgdaBound{U} \AgdaBound{V}\AgdaSymbol{\}} \AgdaSymbol{→} \<[62]%
\>[62]\<%
\\
\>[0]\AgdaIndent{2}{}\<[2]%
\>[2]\AgdaFunction{Sub↑} \AgdaBound{K} \AgdaSymbol{(}\AgdaBound{σ} \AgdaFunction{•SR} \AgdaBound{ρ}\AgdaSymbol{)} \AgdaFunction{∼} \AgdaFunction{Sub↑} \AgdaBound{K} \AgdaBound{σ} \AgdaFunction{•SR} \AgdaFunction{Rep↑} \AgdaBound{K} \AgdaBound{ρ}\<%
\end{code}

\AgdaHide{
\begin{code}%
\>\AgdaFunction{Sub↑-compSR} \AgdaSymbol{\{}\AgdaArgument{K} \AgdaSymbol{=} \AgdaBound{K}\AgdaSymbol{\}} \AgdaInductiveConstructor{x₀} \AgdaSymbol{=} \AgdaInductiveConstructor{refl}\<%
\\
\>\AgdaFunction{Sub↑-compSR} \AgdaSymbol{(}\AgdaInductiveConstructor{↑} \AgdaBound{x}\AgdaSymbol{)} \AgdaSymbol{=} \AgdaInductiveConstructor{refl}\<%
\end{code}
}

\begin{code}%
\>\AgdaFunction{COMPSR} \AgdaSymbol{:} \AgdaRecord{Composition} \AgdaFunction{proto-substitution} \AgdaFunction{proto-replacement} \AgdaFunction{proto-substitution}\<%
\\
\>\AgdaFunction{COMPSR} \AgdaSymbol{=} \AgdaKeyword{record} \AgdaSymbol{\{} \<[18]%
\>[18]\<%
\\
\>[0]\AgdaIndent{2}{}\<[2]%
\>[2]\AgdaField{circ} \AgdaSymbol{=} \AgdaFunction{\_•SR\_} \AgdaSymbol{;} \<[17]%
\>[17]\<%
\\
\>[0]\AgdaIndent{2}{}\<[2]%
\>[2]\AgdaField{liftOp-circ} \AgdaSymbol{=} \AgdaFunction{Sub↑-compSR} \AgdaSymbol{;} \<[30]%
\>[30]\<%
\\
\>[0]\AgdaIndent{2}{}\<[2]%
\>[2]\AgdaField{apV-circ} \AgdaSymbol{=} \AgdaInductiveConstructor{refl} \AgdaSymbol{\}}\<%
\\
%
\\
\>\AgdaFunction{sub-compSR} \AgdaSymbol{:} \AgdaSymbol{∀} \AgdaSymbol{\{}\AgdaBound{U}\AgdaSymbol{\}} \AgdaSymbol{\{}\AgdaBound{V}\AgdaSymbol{\}} \AgdaSymbol{\{}\AgdaBound{W}\AgdaSymbol{\}} \AgdaSymbol{\{}\AgdaBound{C}\AgdaSymbol{\}} \AgdaSymbol{\{}\AgdaBound{K}\AgdaSymbol{\}} \<[35]%
\>[35]\<%
\\
\>[0]\AgdaIndent{2}{}\<[2]%
\>[2]\AgdaSymbol{(}\AgdaBound{E} \AgdaSymbol{:} \AgdaDatatype{Subexpression} \AgdaBound{U} \AgdaBound{C} \AgdaBound{K}\AgdaSymbol{)} \AgdaSymbol{\{}\AgdaBound{σ} \AgdaSymbol{:} \AgdaFunction{Sub} \AgdaBound{V} \AgdaBound{W}\AgdaSymbol{\}} \AgdaSymbol{\{}\AgdaBound{ρ} \AgdaSymbol{:} \AgdaFunction{Rep} \AgdaBound{U} \AgdaBound{V}\AgdaSymbol{\}} \AgdaSymbol{→} \<[58]%
\>[58]\<%
\\
\>[0]\AgdaIndent{2}{}\<[2]%
\>[2]\AgdaBound{E} \AgdaFunction{⟦} \AgdaBound{σ} \AgdaFunction{•SR} \AgdaBound{ρ} \AgdaFunction{⟧} \AgdaDatatype{≡} \AgdaBound{E} \AgdaFunction{〈} \AgdaBound{ρ} \AgdaFunction{〉} \AgdaFunction{⟦} \AgdaBound{σ} \AgdaFunction{⟧}\<%
\end{code}

\AgdaHide{
\begin{code}%
\>\AgdaFunction{sub-compSR} \AgdaBound{E} \AgdaSymbol{=} \AgdaFunction{Composition.ap-circ} \AgdaFunction{COMPSR} \AgdaBound{E}\<%
\end{code}
}

\begin{code}%
\>\AgdaFunction{Sub↑-upRep} \AgdaSymbol{:} \AgdaSymbol{∀} \AgdaSymbol{\{}\AgdaBound{U}\AgdaSymbol{\}} \AgdaSymbol{\{}\AgdaBound{V}\AgdaSymbol{\}} \AgdaSymbol{\{}\AgdaBound{C}\AgdaSymbol{\}} \AgdaSymbol{\{}\AgdaBound{K}\AgdaSymbol{\}} \AgdaSymbol{\{}\AgdaBound{L}\AgdaSymbol{\}} \AgdaSymbol{(}\AgdaBound{E} \AgdaSymbol{:} \AgdaDatatype{Subexpression} \AgdaBound{U} \AgdaBound{C} \AgdaBound{K}\AgdaSymbol{)} \AgdaSymbol{\{}\AgdaBound{σ} \AgdaSymbol{:} \AgdaFunction{Sub} \AgdaBound{U} \AgdaBound{V}\AgdaSymbol{\}} \AgdaSymbol{→}\<%
\\
\>[0]\AgdaIndent{2}{}\<[2]%
\>[2]\AgdaBound{E} \AgdaFunction{〈} \AgdaFunction{upRep} \AgdaFunction{〉} \AgdaFunction{⟦} \AgdaFunction{Sub↑} \AgdaBound{L} \AgdaBound{σ} \AgdaFunction{⟧} \AgdaDatatype{≡} \AgdaBound{E} \AgdaFunction{⟦} \AgdaBound{σ} \AgdaFunction{⟧} \AgdaFunction{〈} \AgdaFunction{upRep} \AgdaFunction{〉}\<%
\end{code}

\AgdaHide{
\begin{code}%
\>\AgdaFunction{Sub↑-upRep} \AgdaBound{E} \AgdaSymbol{=} \AgdaFunction{liftOp-up-mixed} \AgdaFunction{COMPSR} \AgdaFunction{COMPRS} \AgdaSymbol{(λ} \AgdaSymbol{\{}\AgdaBound{\_}\AgdaSymbol{\}} \AgdaSymbol{\{}\AgdaBound{\_}\AgdaSymbol{\}} \AgdaSymbol{\{}\AgdaBound{\_}\AgdaSymbol{\}} \AgdaSymbol{\{}\AgdaBound{\_}\AgdaSymbol{\}} \AgdaSymbol{\{}\AgdaBound{E}\AgdaSymbol{\}} \AgdaSymbol{→} \AgdaFunction{sym} \AgdaSymbol{(}\AgdaFunction{up-is-up'} \AgdaSymbol{\{}\AgdaArgument{E} \AgdaSymbol{=} \AgdaBound{E}\AgdaSymbol{\}))} \AgdaSymbol{\{}\AgdaBound{E}\AgdaSymbol{\}}\<%
\end{code}
}

Composition is defined by $(\sigma \circ \rho)(x) \equiv \rho(x) [ \sigma ]$.

\begin{code}%
\>\AgdaKeyword{infixl} \AgdaNumber{60} \AgdaFixityOp{\_•\_}\<%
\\
\>\AgdaFunction{\_•\_} \AgdaSymbol{:} \AgdaSymbol{∀} \AgdaSymbol{\{}\AgdaBound{U}\AgdaSymbol{\}} \AgdaSymbol{\{}\AgdaBound{V}\AgdaSymbol{\}} \AgdaSymbol{\{}\AgdaBound{W}\AgdaSymbol{\}} \AgdaSymbol{→} \AgdaFunction{Sub} \AgdaBound{V} \AgdaBound{W} \AgdaSymbol{→} \AgdaFunction{Sub} \AgdaBound{U} \AgdaBound{V} \AgdaSymbol{→} \AgdaFunction{Sub} \AgdaBound{U} \AgdaBound{W}\<%
\\
\>\AgdaSymbol{(}\AgdaBound{σ} \AgdaFunction{•} \AgdaBound{ρ}\AgdaSymbol{)} \AgdaBound{K} \AgdaBound{x} \AgdaSymbol{=} \AgdaBound{ρ} \AgdaBound{K} \AgdaBound{x} \AgdaFunction{⟦} \AgdaBound{σ} \AgdaFunction{⟧}\<%
\end{code}

Using the fact that $\bullet_1$ and $\bullet_2$ are compositions, we are
able to prove that this is a composition $\mathrm{substitution} ; \mathrm{substitution} \rightarrow \mathrm{substitution}$, and hence that substitution is a family of operations.

\begin{code}%
\>\AgdaFunction{Sub↑-comp} \AgdaSymbol{:} \AgdaSymbol{∀} \AgdaSymbol{\{}\AgdaBound{U}\AgdaSymbol{\}} \AgdaSymbol{\{}\AgdaBound{V}\AgdaSymbol{\}} \AgdaSymbol{\{}\AgdaBound{W}\AgdaSymbol{\}} \AgdaSymbol{\{}\AgdaBound{ρ} \AgdaSymbol{:} \AgdaFunction{Sub} \AgdaBound{U} \AgdaBound{V}\AgdaSymbol{\}} \AgdaSymbol{\{}\AgdaBound{σ} \AgdaSymbol{:} \AgdaFunction{Sub} \AgdaBound{V} \AgdaBound{W}\AgdaSymbol{\}} \AgdaSymbol{\{}\AgdaBound{K}\AgdaSymbol{\}} \AgdaSymbol{→} \<[60]%
\>[60]\<%
\\
\>[0]\AgdaIndent{2}{}\<[2]%
\>[2]\AgdaFunction{Sub↑} \AgdaBound{K} \AgdaSymbol{(}\AgdaBound{σ} \AgdaFunction{•} \AgdaBound{ρ}\AgdaSymbol{)} \AgdaFunction{∼} \AgdaFunction{Sub↑} \AgdaBound{K} \AgdaBound{σ} \AgdaFunction{•} \AgdaFunction{Sub↑} \AgdaBound{K} \AgdaBound{ρ}\<%
\end{code}

\AgdaHide{
\begin{code}%
\>\AgdaFunction{Sub↑-comp} \AgdaInductiveConstructor{x₀} \AgdaSymbol{=} \AgdaInductiveConstructor{refl}\<%
\\
\>\AgdaFunction{Sub↑-comp} \AgdaSymbol{\{}\AgdaArgument{W} \AgdaSymbol{=} \AgdaBound{W}\AgdaSymbol{\}} \AgdaSymbol{\{}\AgdaArgument{ρ} \AgdaSymbol{=} \AgdaBound{ρ}\AgdaSymbol{\}} \AgdaSymbol{\{}\AgdaArgument{σ} \AgdaSymbol{=} \AgdaBound{σ}\AgdaSymbol{\}} \AgdaSymbol{\{}\AgdaArgument{K} \AgdaSymbol{=} \AgdaBound{K}\AgdaSymbol{\}} \AgdaSymbol{\{}\AgdaBound{L}\AgdaSymbol{\}} \AgdaSymbol{(}\AgdaInductiveConstructor{↑} \AgdaBound{x}\AgdaSymbol{)} \AgdaSymbol{=} \AgdaFunction{sym} \AgdaSymbol{(}\AgdaFunction{Sub↑-upRep} \AgdaSymbol{(}\AgdaBound{ρ} \AgdaBound{L} \AgdaBound{x}\AgdaSymbol{))}\<%
\\
%
\\
\>\AgdaFunction{Sub↑-upRep₂} \AgdaSymbol{:} \AgdaSymbol{∀} \AgdaSymbol{\{}\AgdaBound{U}\AgdaSymbol{\}} \AgdaSymbol{\{}\AgdaBound{V}\AgdaSymbol{\}} \AgdaSymbol{\{}\AgdaBound{C}\AgdaSymbol{\}} \AgdaSymbol{\{}\AgdaBound{K}\AgdaSymbol{\}} \AgdaSymbol{\{}\AgdaBound{L}\AgdaSymbol{\}} \AgdaSymbol{\{}\AgdaBound{M}\AgdaSymbol{\}} \AgdaSymbol{(}\AgdaBound{E} \AgdaSymbol{:} \AgdaDatatype{Subexpression} \AgdaBound{U} \AgdaBound{C} \AgdaBound{M}\AgdaSymbol{)} \AgdaSymbol{\{}\AgdaBound{σ} \AgdaSymbol{:} \AgdaFunction{Sub} \AgdaBound{U} \AgdaBound{V}\AgdaSymbol{\}} \AgdaSymbol{→} \AgdaBound{E} \AgdaFunction{⇑} \AgdaFunction{⇑} \AgdaFunction{⟦} \AgdaFunction{Sub↑} \AgdaBound{K} \AgdaSymbol{(}\AgdaFunction{Sub↑} \AgdaBound{L} \AgdaBound{σ}\AgdaSymbol{)} \AgdaFunction{⟧} \AgdaDatatype{≡} \AgdaBound{E} \AgdaFunction{⟦} \AgdaBound{σ} \AgdaFunction{⟧} \AgdaFunction{⇑} \AgdaFunction{⇑}\<%
\\
\>\AgdaFunction{Sub↑-upRep₂} \AgdaSymbol{\{}\AgdaBound{U}\AgdaSymbol{\}} \AgdaSymbol{\{}\AgdaBound{V}\AgdaSymbol{\}} \AgdaSymbol{\{}\AgdaBound{C}\AgdaSymbol{\}} \AgdaSymbol{\{}\AgdaBound{K}\AgdaSymbol{\}} \AgdaSymbol{\{}\AgdaBound{L}\AgdaSymbol{\}} \AgdaSymbol{\{}\AgdaBound{M}\AgdaSymbol{\}} \AgdaBound{E} \AgdaSymbol{\{}\AgdaBound{σ}\AgdaSymbol{\}} \AgdaSymbol{=} \AgdaKeyword{let} \AgdaKeyword{open} \AgdaModule{≡-Reasoning} \AgdaKeyword{in} \<[68]%
\>[68]\<%
\\
\>[0]\AgdaIndent{2}{}\<[2]%
\>[2]\AgdaFunction{begin}\<%
\\
\>[2]\AgdaIndent{4}{}\<[4]%
\>[4]\AgdaBound{E} \AgdaFunction{⇑} \AgdaFunction{⇑} \AgdaFunction{⟦} \AgdaFunction{Sub↑} \AgdaBound{K} \AgdaSymbol{(}\AgdaFunction{Sub↑} \AgdaBound{L} \AgdaBound{σ}\AgdaSymbol{)} \AgdaFunction{⟧}\<%
\\
\>[0]\AgdaIndent{2}{}\<[2]%
\>[2]\AgdaFunction{≡⟨} \AgdaFunction{Sub↑-upRep} \AgdaSymbol{(}\AgdaBound{E} \AgdaFunction{⇑}\AgdaSymbol{)} \AgdaFunction{⟩}\<%
\\
\>[2]\AgdaIndent{4}{}\<[4]%
\>[4]\AgdaBound{E} \AgdaFunction{⇑} \AgdaFunction{⟦} \AgdaFunction{Sub↑} \AgdaBound{L} \AgdaBound{σ} \AgdaFunction{⟧} \AgdaFunction{⇑}\<%
\\
\>[0]\AgdaIndent{2}{}\<[2]%
\>[2]\AgdaFunction{≡⟨} \AgdaFunction{rep-congl} \AgdaSymbol{(}\AgdaFunction{Sub↑-upRep} \AgdaBound{E}\AgdaSymbol{)} \AgdaFunction{⟩}\<%
\\
\>[2]\AgdaIndent{4}{}\<[4]%
\>[4]\AgdaBound{E} \AgdaFunction{⟦} \AgdaBound{σ} \AgdaFunction{⟧} \AgdaFunction{⇑} \AgdaFunction{⇑}\<%
\\
\>[0]\AgdaIndent{2}{}\<[2]%
\>[2]\AgdaFunction{∎}\<%
\\
%
\\
\>\AgdaFunction{Sub↑-upRep₃} \AgdaSymbol{:} \AgdaSymbol{∀} \AgdaSymbol{\{}\AgdaBound{U}\AgdaSymbol{\}} \AgdaSymbol{\{}\AgdaBound{V}\AgdaSymbol{\}} \AgdaSymbol{\{}\AgdaBound{C}\AgdaSymbol{\}} \AgdaSymbol{\{}\AgdaBound{K}\AgdaSymbol{\}} \AgdaSymbol{\{}\AgdaBound{L}\AgdaSymbol{\}} \AgdaSymbol{\{}\AgdaBound{M}\AgdaSymbol{\}} \AgdaSymbol{\{}\AgdaBound{N}\AgdaSymbol{\}} \AgdaSymbol{(}\AgdaBound{E} \AgdaSymbol{:} \AgdaDatatype{Subexpression} \AgdaBound{U} \AgdaBound{C} \AgdaBound{N}\AgdaSymbol{)} \AgdaSymbol{\{}\AgdaBound{σ} \AgdaSymbol{:} \AgdaFunction{Sub} \AgdaBound{U} \AgdaBound{V}\AgdaSymbol{\}} \AgdaSymbol{→} \AgdaBound{E} \AgdaFunction{⇑} \AgdaFunction{⇑} \AgdaFunction{⇑} \AgdaFunction{⟦} \AgdaFunction{Sub↑} \AgdaBound{K} \AgdaSymbol{(}\AgdaFunction{Sub↑} \AgdaBound{L} \AgdaSymbol{(}\AgdaFunction{Sub↑} \AgdaBound{M} \AgdaBound{σ}\AgdaSymbol{))} \AgdaFunction{⟧} \AgdaDatatype{≡} \AgdaBound{E} \AgdaFunction{⟦} \AgdaBound{σ} \AgdaFunction{⟧} \AgdaFunction{⇑} \AgdaFunction{⇑} \AgdaFunction{⇑}\<%
\\
\>\AgdaFunction{Sub↑-upRep₃} \AgdaSymbol{\{}\AgdaBound{U}\AgdaSymbol{\}} \AgdaSymbol{\{}\AgdaBound{V}\AgdaSymbol{\}} \AgdaSymbol{\{}\AgdaBound{C}\AgdaSymbol{\}} \AgdaSymbol{\{}\AgdaBound{K}\AgdaSymbol{\}} \AgdaSymbol{\{}\AgdaBound{L}\AgdaSymbol{\}} \AgdaSymbol{\{}\AgdaBound{M}\AgdaSymbol{\}} \AgdaSymbol{\{}\AgdaBound{N}\AgdaSymbol{\}} \AgdaBound{E} \AgdaSymbol{\{}\AgdaBound{σ}\AgdaSymbol{\}} \AgdaSymbol{=} \AgdaKeyword{let} \AgdaKeyword{open} \AgdaModule{≡-Reasoning} \AgdaKeyword{in} \<[72]%
\>[72]\<%
\\
\>[0]\AgdaIndent{2}{}\<[2]%
\>[2]\AgdaFunction{begin}\<%
\\
\>[2]\AgdaIndent{4}{}\<[4]%
\>[4]\AgdaBound{E} \AgdaFunction{⇑} \AgdaFunction{⇑} \AgdaFunction{⇑} \AgdaFunction{⟦} \AgdaFunction{Sub↑} \AgdaBound{K} \AgdaSymbol{(}\AgdaFunction{Sub↑} \AgdaBound{L} \AgdaSymbol{(}\AgdaFunction{Sub↑} \AgdaBound{M} \AgdaBound{σ}\AgdaSymbol{))} \AgdaFunction{⟧}\<%
\\
\>[0]\AgdaIndent{2}{}\<[2]%
\>[2]\AgdaFunction{≡⟨} \AgdaFunction{Sub↑-upRep₂} \AgdaSymbol{(}\AgdaBound{E} \AgdaFunction{⇑}\AgdaSymbol{)} \AgdaFunction{⟩}\<%
\\
\>[2]\AgdaIndent{4}{}\<[4]%
\>[4]\AgdaBound{E} \AgdaFunction{⇑} \AgdaFunction{⟦} \AgdaFunction{Sub↑} \AgdaBound{M} \AgdaBound{σ} \AgdaFunction{⟧} \AgdaFunction{⇑} \AgdaFunction{⇑}\<%
\\
\>[0]\AgdaIndent{2}{}\<[2]%
\>[2]\AgdaFunction{≡⟨} \AgdaFunction{rep-congl} \AgdaSymbol{(}\AgdaFunction{rep-congl} \AgdaSymbol{(}\AgdaFunction{Sub↑-upRep} \AgdaBound{E}\AgdaSymbol{))} \AgdaFunction{⟩}\<%
\\
\>[2]\AgdaIndent{4}{}\<[4]%
\>[4]\AgdaBound{E} \AgdaFunction{⟦} \AgdaBound{σ} \AgdaFunction{⟧} \AgdaFunction{⇑} \AgdaFunction{⇑} \AgdaFunction{⇑}\<%
\\
\>[0]\AgdaIndent{2}{}\<[2]%
\>[2]\AgdaFunction{∎}\<%
\\
%
\\
\>\AgdaFunction{Rep↑-Sub↑-upRep₃} \AgdaSymbol{:} \AgdaSymbol{∀} \AgdaSymbol{\{}\AgdaBound{U}\AgdaSymbol{\}} \AgdaSymbol{\{}\AgdaBound{V}\AgdaSymbol{\}} \AgdaSymbol{\{}\AgdaBound{W}\AgdaSymbol{\}} \AgdaSymbol{\{}\AgdaBound{K1}\AgdaSymbol{\}} \AgdaSymbol{\{}\AgdaBound{K2}\AgdaSymbol{\}} \AgdaSymbol{\{}\AgdaBound{K3}\AgdaSymbol{\}} \AgdaSymbol{\{}\AgdaBound{C}\AgdaSymbol{\}} \AgdaSymbol{\{}\AgdaBound{K4}\AgdaSymbol{\}} \<[57]%
\>[57]\<%
\\
\>[2]\AgdaIndent{19}{}\<[19]%
\>[19]\AgdaSymbol{(}\AgdaBound{M} \AgdaSymbol{:} \AgdaDatatype{Subexpression} \AgdaBound{U} \AgdaBound{C} \AgdaBound{K4}\AgdaSymbol{)}\<%
\\
\>[2]\AgdaIndent{19}{}\<[19]%
\>[19]\AgdaSymbol{(}\AgdaBound{σ} \AgdaSymbol{:} \AgdaFunction{Sub} \AgdaBound{U} \AgdaBound{V}\AgdaSymbol{)} \AgdaSymbol{(}\AgdaBound{ρ} \AgdaSymbol{:} \AgdaFunction{Rep} \AgdaBound{V} \AgdaBound{W}\AgdaSymbol{)} \AgdaSymbol{→}\<%
\\
\>[19]\AgdaIndent{20}{}\<[20]%
\>[20]\AgdaBound{M} \AgdaFunction{⇑} \AgdaFunction{⇑} \AgdaFunction{⇑} \AgdaFunction{⟦} \AgdaFunction{Sub↑} \AgdaBound{K1} \AgdaSymbol{(}\AgdaFunction{Sub↑} \AgdaBound{K2} \AgdaSymbol{(}\AgdaFunction{Sub↑} \AgdaBound{K3} \AgdaBound{σ}\AgdaSymbol{))} \AgdaFunction{⟧} \AgdaFunction{〈} \AgdaFunction{Rep↑} \AgdaBound{K1} \AgdaSymbol{(}\AgdaFunction{Rep↑} \AgdaBound{K2} \AgdaSymbol{(}\AgdaFunction{Rep↑} \AgdaBound{K3} \AgdaBound{ρ}\AgdaSymbol{))} \AgdaFunction{〉}\<%
\\
\>[19]\AgdaIndent{20}{}\<[20]%
\>[20]\AgdaDatatype{≡} \AgdaBound{M} \AgdaFunction{⟦} \AgdaBound{σ} \AgdaFunction{⟧} \AgdaFunction{〈} \AgdaBound{ρ} \AgdaFunction{〉} \AgdaFunction{⇑} \AgdaFunction{⇑} \AgdaFunction{⇑}\<%
\\
\>\AgdaFunction{Rep↑-Sub↑-upRep₃} \AgdaBound{M} \AgdaBound{σ} \AgdaBound{ρ} \AgdaSymbol{=} \AgdaFunction{trans} \AgdaSymbol{(}\AgdaFunction{rep-congl} \AgdaSymbol{(}\AgdaFunction{Sub↑-upRep₃} \AgdaBound{M} \AgdaSymbol{\{}\AgdaBound{σ}\AgdaSymbol{\}))} \AgdaSymbol{(}\AgdaFunction{Rep↑-upRep₃} \AgdaSymbol{(}\AgdaBound{M} \AgdaFunction{⟦} \AgdaBound{σ} \AgdaFunction{⟧}\AgdaSymbol{))}\<%
\\
%
\\
\>\AgdaFunction{assocRSSR} \AgdaSymbol{:} \AgdaSymbol{∀} \AgdaSymbol{\{}\AgdaBound{U}\AgdaSymbol{\}} \AgdaSymbol{\{}\AgdaBound{V}\AgdaSymbol{\}} \AgdaSymbol{\{}\AgdaBound{W}\AgdaSymbol{\}} \AgdaSymbol{\{}\AgdaBound{X}\AgdaSymbol{\}} \AgdaSymbol{\{}\AgdaBound{ρ} \AgdaSymbol{:} \AgdaFunction{Sub} \AgdaBound{W} \AgdaBound{X}\AgdaSymbol{\}} \AgdaSymbol{\{}\AgdaBound{σ} \AgdaSymbol{:} \AgdaFunction{Rep} \AgdaBound{V} \AgdaBound{W}\AgdaSymbol{\}} \AgdaSymbol{\{}\AgdaBound{τ} \AgdaSymbol{:} \AgdaFunction{Sub} \AgdaBound{U} \AgdaBound{V}\AgdaSymbol{\}} \AgdaSymbol{→}\<%
\\
\>[0]\AgdaIndent{12}{}\<[12]%
\>[12]\AgdaBound{ρ} \AgdaFunction{•} \AgdaSymbol{(}\AgdaBound{σ} \AgdaFunction{•RS} \AgdaBound{τ}\AgdaSymbol{)} \AgdaFunction{∼} \AgdaSymbol{(}\AgdaBound{ρ} \AgdaFunction{•SR} \AgdaBound{σ}\AgdaSymbol{)} \AgdaFunction{•} \AgdaBound{τ}\<%
\\
\>\AgdaFunction{assocRSSR} \AgdaSymbol{\{}\AgdaArgument{ρ} \AgdaSymbol{=} \AgdaBound{ρ}\AgdaSymbol{\}} \AgdaSymbol{\{}\AgdaBound{σ}\AgdaSymbol{\}} \AgdaSymbol{\{}\AgdaBound{τ}\AgdaSymbol{\}} \AgdaBound{x} \AgdaSymbol{=} \AgdaFunction{sym} \AgdaSymbol{(}\AgdaFunction{sub-compSR} \AgdaSymbol{(}\AgdaBound{τ} \AgdaSymbol{\_} \AgdaBound{x}\AgdaSymbol{))}\<%
\end{code}
}

\begin{code}%
\>\AgdaFunction{substitution} \AgdaSymbol{:} \AgdaRecord{OpFamily}\<%
\\
\>\AgdaFunction{substitution} \AgdaSymbol{=} \AgdaKeyword{record} \AgdaSymbol{\{} \<[24]%
\>[24]\<%
\\
\>[0]\AgdaIndent{2}{}\<[2]%
\>[2]\AgdaField{liftFamily} \AgdaSymbol{=} \AgdaFunction{proto-substitution} \AgdaSymbol{;} \<[36]%
\>[36]\<%
\\
\>[0]\AgdaIndent{2}{}\<[2]%
\>[2]\AgdaField{isOpFamily} \AgdaSymbol{=} \AgdaKeyword{record} \AgdaSymbol{\{} \<[24]%
\>[24]\<%
\\
\>[2]\AgdaIndent{4}{}\<[4]%
\>[4]\AgdaField{\_∘\_} \AgdaSymbol{=} \AgdaFunction{\_•\_} \AgdaSymbol{;} \<[16]%
\>[16]\<%
\\
\>[2]\AgdaIndent{4}{}\<[4]%
\>[4]\AgdaField{liftOp-comp} \AgdaSymbol{=} \AgdaFunction{Sub↑-comp} \AgdaSymbol{;} \<[30]%
\>[30]\<%
\\
\>[2]\AgdaIndent{4}{}\<[4]%
\>[4]\AgdaField{apV-comp} \AgdaSymbol{=} \AgdaInductiveConstructor{refl} \AgdaSymbol{\}} \<[22]%
\>[22]\<%
\\
\>[0]\AgdaIndent{2}{}\<[2]%
\>[2]\AgdaSymbol{\}}\<%
\end{code}

\AgdaHide{
\begin{code}%
\>\AgdaKeyword{open} \AgdaModule{OpFamily} \AgdaFunction{substitution} \AgdaKeyword{using} \AgdaSymbol{(}comp-congl\AgdaSymbol{;}comp-congr\AgdaSymbol{)}\<%
\\
\>[0]\AgdaIndent{2}{}\<[2]%
\>[2]\AgdaKeyword{renaming} \AgdaSymbol{(}liftOp-idOp \AgdaSymbol{to} Sub↑-idOp\AgdaSymbol{;}\<\\
\>           ap-idOp \AgdaSymbol{to} sub-idOp\AgdaSymbol{;}\<\\
\>           ap-congl \AgdaSymbol{to} sub-congr\AgdaSymbol{;}\<\\
\>           ap-congr \AgdaSymbol{to} sub-congl\AgdaSymbol{;}\<\\
\>           unitl \AgdaSymbol{to} sub-unitl\AgdaSymbol{;}\<\\
\>           unitr \AgdaSymbol{to} sub-unitr\AgdaSymbol{;}\<\\
\>           ∼-sym \AgdaSymbol{to} sub-sym\AgdaSymbol{;}\<\\
\>           ∼-trans \AgdaSymbol{to} sub-trans\AgdaSymbol{;}\<\\
\>           assoc \AgdaSymbol{to} sub-assoc\AgdaSymbol{)}\<%
\\
\>[0]\AgdaIndent{2}{}\<[2]%
\>[2]\AgdaKeyword{public}\<%
\end{code}
}

\begin{frame}[fragile]
\frametitle{Metatheorems}
We can now prove general results such as:

\begin{code}%
\>\AgdaFunction{sub-comp} \AgdaSymbol{:} \AgdaSymbol{∀} \AgdaSymbol{\{}\AgdaBound{U}\AgdaSymbol{\}} \AgdaSymbol{\{}\AgdaBound{V}\AgdaSymbol{\}} \AgdaSymbol{\{}\AgdaBound{W}\AgdaSymbol{\}} \AgdaSymbol{\{}\AgdaBound{C}\AgdaSymbol{\}} \AgdaSymbol{\{}\AgdaBound{K}\AgdaSymbol{\}}\<%
\\
\>[0]\AgdaIndent{2}{}\<[2]%
\>[2]\AgdaSymbol{(}\AgdaBound{E} \AgdaSymbol{:} \AgdaDatatype{Subexpression} \AgdaBound{U} \AgdaBound{C} \AgdaBound{K}\AgdaSymbol{)} \AgdaSymbol{\{}\AgdaBound{σ} \AgdaSymbol{:} \AgdaFunction{Sub} \AgdaBound{V} \AgdaBound{W}\AgdaSymbol{\}} \AgdaSymbol{\{}\AgdaBound{ρ} \AgdaSymbol{:} \AgdaFunction{Sub} \AgdaBound{U} \AgdaBound{V}\AgdaSymbol{\}} \AgdaSymbol{→}\<%
\\
\>[0]\AgdaIndent{2}{}\<[2]%
\>[2]\AgdaBound{E} \AgdaFunction{⟦} \AgdaBound{σ} \AgdaFunction{•} \AgdaBound{ρ} \AgdaFunction{⟧} \AgdaDatatype{≡} \AgdaBound{E} \AgdaFunction{⟦} \AgdaBound{ρ} \AgdaFunction{⟧} \AgdaFunction{⟦} \AgdaBound{σ} \AgdaFunction{⟧}\<%
\end{code}
\end{frame}

\AgdaHide{
\begin{code}%
\>\AgdaFunction{sub-comp} \AgdaSymbol{=} \AgdaFunction{OpFamily.ap-circ} \AgdaFunction{substitution}\<%
\end{code}
}

\AgdaHide{
\begin{code}%
\>\AgdaComment{--Variable convention: ρ ranges over replacements}\<%
\\
\>\AgdaKeyword{open} \AgdaKeyword{import} \AgdaModule{Grammar.Base}\<%
\\
%
\\
\>\AgdaKeyword{module} \AgdaModule{Grammar.Replacement} \AgdaSymbol{(}\AgdaBound{G} \AgdaSymbol{:} \AgdaRecord{Grammar}\AgdaSymbol{)} \AgdaKeyword{where}\<%
\\
%
\\
\>\AgdaKeyword{open} \AgdaKeyword{import} \AgdaModule{Function}\<%
\\
\>\AgdaKeyword{open} \AgdaKeyword{import} \AgdaModule{Prelims}\<%
\\
\>\AgdaKeyword{open} \AgdaModule{Grammar} \AgdaBound{G}\<%
\\
\>\AgdaKeyword{open} \AgdaKeyword{import} \AgdaModule{Grammar.OpFamily.PreOpFamily} \AgdaBound{G}\<%
\\
\>\AgdaKeyword{open} \AgdaKeyword{import} \AgdaModule{Grammar.OpFamily.LiftFamily} \AgdaBound{G}\<%
\\
\>\AgdaKeyword{open} \AgdaKeyword{import} \AgdaModule{Grammar.OpFamily.OpFamily} \AgdaBound{G}\<%
\end{code}
}

\subsection{Replacement}

The operation family of \emph{replacement} is defined as follows.  A replacement $\rho : U \rightarrow V$ is a function
that maps every variable in $U$ to a variable in $V$ of the same kind.  Application, identity and composition are simply
function application, the identity function and function composition.  The successor is the canonical injection $V \rightarrow (V, K)$,
and $(\sigma , K)$ is the extension of $\sigma$ that maps $x_0$ to $x_0$.

\begin{code}%
\>\AgdaFunction{Rep} \AgdaSymbol{:} \AgdaDatatype{Alphabet} \AgdaSymbol{→} \AgdaDatatype{Alphabet} \AgdaSymbol{→} \AgdaPrimitiveType{Set}\<%
\\
\>\AgdaFunction{Rep} \AgdaBound{U} \AgdaBound{V} \AgdaSymbol{=} \AgdaSymbol{∀} \AgdaBound{K} \AgdaSymbol{→} \AgdaDatatype{Var} \AgdaBound{U} \AgdaBound{K} \AgdaSymbol{→} \AgdaDatatype{Var} \AgdaBound{V} \AgdaBound{K}\<%
\\
%
\\
\>\AgdaFunction{upRep} \AgdaSymbol{:} \AgdaSymbol{∀} \AgdaSymbol{\{}\AgdaBound{V}\AgdaSymbol{\}} \AgdaSymbol{\{}\AgdaBound{K}\AgdaSymbol{\}} \AgdaSymbol{→} \AgdaFunction{Rep} \AgdaBound{V} \AgdaSymbol{(}\AgdaBound{V} \AgdaInductiveConstructor{,} \AgdaBound{K}\AgdaSymbol{)}\<%
\\
\>\AgdaFunction{upRep} \AgdaSymbol{\_} \AgdaSymbol{=} \AgdaInductiveConstructor{↑}\<%
\\
%
\\
\>\AgdaFunction{idRep} \AgdaSymbol{:} \AgdaSymbol{∀} \AgdaBound{V} \AgdaSymbol{→} \AgdaFunction{Rep} \AgdaBound{V} \AgdaBound{V}\<%
\\
\>\AgdaFunction{idRep} \AgdaSymbol{\_} \AgdaSymbol{\_} \AgdaBound{x} \AgdaSymbol{=} \AgdaBound{x}\<%
\\
%
\\
\>\AgdaFunction{Rep∶POF} \AgdaSymbol{:} \AgdaRecord{PreOpFamily}\<%
\\
\>\AgdaFunction{Rep∶POF} \AgdaSymbol{=} \AgdaKeyword{record} \AgdaSymbol{\{} \<[19]%
\>[19]\<%
\\
\>[0]\AgdaIndent{2}{}\<[2]%
\>[2]\AgdaField{Op} \AgdaSymbol{=} \AgdaFunction{Rep}\AgdaSymbol{;} \<[12]%
\>[12]\<%
\\
\>[0]\AgdaIndent{2}{}\<[2]%
\>[2]\AgdaField{apV} \AgdaSymbol{=} \AgdaSymbol{λ} \AgdaBound{ρ} \AgdaBound{x} \AgdaSymbol{→} \AgdaInductiveConstructor{var} \AgdaSymbol{(}\AgdaBound{ρ} \AgdaSymbol{\_} \AgdaBound{x}\AgdaSymbol{);} \<[29]%
\>[29]\<%
\\
\>[0]\AgdaIndent{2}{}\<[2]%
\>[2]\AgdaField{up} \AgdaSymbol{=} \AgdaFunction{upRep}\AgdaSymbol{;} \<[14]%
\>[14]\<%
\\
\>[0]\AgdaIndent{2}{}\<[2]%
\>[2]\AgdaField{apV-up} \AgdaSymbol{=} \AgdaInductiveConstructor{refl}\AgdaSymbol{;} \<[17]%
\>[17]\<%
\\
\>[0]\AgdaIndent{2}{}\<[2]%
\>[2]\AgdaField{idOp} \AgdaSymbol{=} \AgdaFunction{idRep}\AgdaSymbol{;} \<[16]%
\>[16]\<%
\\
\>[0]\AgdaIndent{2}{}\<[2]%
\>[2]\AgdaField{apV-idOp} \AgdaSymbol{=} \AgdaSymbol{λ} \AgdaBound{\_} \AgdaSymbol{→} \AgdaInductiveConstructor{refl} \AgdaSymbol{\}}\<%
\\
%
\\
\>\AgdaFunction{\_∼R\_} \AgdaSymbol{:} \AgdaSymbol{∀} \AgdaSymbol{\{}\AgdaBound{U}\AgdaSymbol{\}} \AgdaSymbol{\{}\AgdaBound{V}\AgdaSymbol{\}} \AgdaSymbol{→} \AgdaFunction{Rep} \AgdaBound{U} \AgdaBound{V} \AgdaSymbol{→} \AgdaFunction{Rep} \AgdaBound{U} \AgdaBound{V} \AgdaSymbol{→} \AgdaPrimitiveType{Set}\<%
\\
\>\AgdaFunction{\_∼R\_} \AgdaSymbol{=} \AgdaFunction{PreOpFamily.\_∼op\_} \AgdaFunction{Rep∶POF}\<%
\\
%
\\
\>\AgdaFunction{liftRep} \AgdaSymbol{:} \AgdaSymbol{∀} \AgdaSymbol{\{}\AgdaBound{U}\AgdaSymbol{\}} \AgdaSymbol{\{}\AgdaBound{V}\AgdaSymbol{\}} \AgdaBound{K} \AgdaSymbol{→} \AgdaFunction{Rep} \AgdaBound{U} \AgdaBound{V} \AgdaSymbol{→} \AgdaFunction{Rep} \AgdaSymbol{(}\AgdaBound{U} \AgdaInductiveConstructor{,} \AgdaBound{K}\AgdaSymbol{)} \AgdaSymbol{(}\AgdaBound{V} \AgdaInductiveConstructor{,} \AgdaBound{K}\AgdaSymbol{)}\<%
\\
\>\AgdaFunction{liftRep} \AgdaSymbol{\_} \AgdaSymbol{\_} \AgdaSymbol{\_} \AgdaInductiveConstructor{x₀} \AgdaSymbol{=} \AgdaInductiveConstructor{x₀}\<%
\\
\>\AgdaFunction{liftRep} \AgdaSymbol{\_} \AgdaBound{ρ} \AgdaBound{K} \AgdaSymbol{(}\AgdaInductiveConstructor{↑} \AgdaBound{x}\AgdaSymbol{)} \AgdaSymbol{=} \AgdaInductiveConstructor{↑} \AgdaSymbol{(}\AgdaBound{ρ} \AgdaBound{K} \AgdaBound{x}\AgdaSymbol{)}\<%
\\
%
\\
\>\AgdaFunction{liftRep-cong} \AgdaSymbol{:} \AgdaSymbol{∀} \AgdaSymbol{\{}\AgdaBound{U}\AgdaSymbol{\}} \AgdaSymbol{\{}\AgdaBound{V}\AgdaSymbol{\}} \AgdaSymbol{\{}\AgdaBound{K}\AgdaSymbol{\}} \AgdaSymbol{\{}\AgdaBound{ρ} \AgdaBound{ρ'} \AgdaSymbol{:} \AgdaFunction{Rep} \AgdaBound{U} \AgdaBound{V}\AgdaSymbol{\}} \AgdaSymbol{→} \<[48]%
\>[48]\<%
\\
\>[0]\AgdaIndent{2}{}\<[2]%
\>[2]\AgdaBound{ρ} \AgdaFunction{∼R} \AgdaBound{ρ'} \AgdaSymbol{→} \AgdaFunction{liftRep} \AgdaBound{K} \AgdaBound{ρ} \AgdaFunction{∼R} \AgdaFunction{liftRep} \AgdaBound{K} \AgdaBound{ρ'}\<%
\end{code}

\AgdaHide{
\begin{code}%
\>\AgdaFunction{liftRep-cong} \AgdaBound{ρ-is-ρ'} \AgdaInductiveConstructor{x₀} \AgdaSymbol{=} \AgdaInductiveConstructor{refl}\<%
\\
\>\AgdaFunction{liftRep-cong} \AgdaBound{ρ-is-ρ'} \AgdaSymbol{(}\AgdaInductiveConstructor{↑} \AgdaBound{x}\AgdaSymbol{)} \AgdaSymbol{=} \AgdaFunction{cong} \AgdaSymbol{(}\AgdaInductiveConstructor{var} \AgdaFunction{∘} \AgdaInductiveConstructor{↑}\AgdaSymbol{)} \AgdaSymbol{(}\AgdaFunction{var-inj} \AgdaSymbol{(}\AgdaBound{ρ-is-ρ'} \AgdaBound{x}\AgdaSymbol{))}\<%
\end{code}
}

\begin{code}%
\>\AgdaFunction{Rep∶LF} \AgdaSymbol{:} \AgdaRecord{LiftFamily}\<%
\\
\>\AgdaFunction{Rep∶LF} \AgdaSymbol{=} \AgdaKeyword{record} \AgdaSymbol{\{} \<[18]%
\>[18]\<%
\\
\>[0]\AgdaIndent{2}{}\<[2]%
\>[2]\AgdaField{preOpFamily} \AgdaSymbol{=} \AgdaFunction{Rep∶POF} \AgdaSymbol{;} \<[26]%
\>[26]\<%
\\
\>[0]\AgdaIndent{2}{}\<[2]%
\>[2]\AgdaField{lifting} \AgdaSymbol{=} \AgdaKeyword{record} \AgdaSymbol{\{} \<[21]%
\>[21]\<%
\\
\>[2]\AgdaIndent{4}{}\<[4]%
\>[4]\AgdaField{liftOp} \AgdaSymbol{=} \AgdaFunction{liftRep} \AgdaSymbol{;} \<[23]%
\>[23]\<%
\\
\>[2]\AgdaIndent{4}{}\<[4]%
\>[4]\AgdaField{liftOp-cong} \AgdaSymbol{=} \AgdaFunction{liftRep-cong} \AgdaSymbol{\}} \AgdaSymbol{;} \<[35]%
\>[35]\<%
\\
\>[0]\AgdaIndent{2}{}\<[2]%
\>[2]\AgdaField{isLiftFamily} \AgdaSymbol{=} \AgdaKeyword{record} \AgdaSymbol{\{} \<[26]%
\>[26]\<%
\\
\>[2]\AgdaIndent{4}{}\<[4]%
\>[4]\AgdaField{liftOp-x₀} \AgdaSymbol{=} \AgdaInductiveConstructor{refl} \AgdaSymbol{;} \<[23]%
\>[23]\<%
\\
\>[2]\AgdaIndent{4}{}\<[4]%
\>[4]\AgdaField{liftOp-↑} \AgdaSymbol{=} \AgdaSymbol{λ} \AgdaBound{\_} \AgdaSymbol{→} \AgdaInductiveConstructor{refl} \AgdaSymbol{\}} \AgdaSymbol{\}}\<%
\\
%
\\
\>\AgdaKeyword{infix} \AgdaNumber{70} \AgdaFixityOp{\_〈\_〉}\<%
\\
\>\AgdaFunction{\_〈\_〉} \AgdaSymbol{:} \AgdaSymbol{∀} \AgdaSymbol{\{}\AgdaBound{U}\AgdaSymbol{\}} \AgdaSymbol{\{}\AgdaBound{V}\AgdaSymbol{\}} \AgdaSymbol{\{}\AgdaBound{C}\AgdaSymbol{\}} \AgdaSymbol{\{}\AgdaBound{K}\AgdaSymbol{\}} \AgdaSymbol{→} \AgdaDatatype{Subexpression} \AgdaBound{U} \AgdaBound{C} \AgdaBound{K} \AgdaSymbol{→} \AgdaFunction{Rep} \AgdaBound{U} \AgdaBound{V} \AgdaSymbol{→} \AgdaDatatype{Subexpression} \AgdaBound{V} \AgdaBound{C} \AgdaBound{K}\<%
\\
\>\AgdaBound{E} \AgdaFunction{〈} \AgdaBound{ρ} \AgdaFunction{〉} \AgdaSymbol{=} \AgdaFunction{LiftFamily.ap} \AgdaFunction{Rep∶LF} \AgdaBound{ρ} \AgdaBound{E}\<%
\\
%
\\
\>\AgdaKeyword{infixl} \AgdaNumber{75} \AgdaFixityOp{\_•R\_}\<%
\\
\>\AgdaFunction{\_•R\_} \AgdaSymbol{:} \AgdaSymbol{∀} \AgdaSymbol{\{}\AgdaBound{U}\AgdaSymbol{\}} \AgdaSymbol{\{}\AgdaBound{V}\AgdaSymbol{\}} \AgdaSymbol{\{}\AgdaBound{W}\AgdaSymbol{\}} \AgdaSymbol{→} \AgdaFunction{Rep} \AgdaBound{V} \AgdaBound{W} \AgdaSymbol{→} \AgdaFunction{Rep} \AgdaBound{U} \AgdaBound{V} \AgdaSymbol{→} \AgdaFunction{Rep} \AgdaBound{U} \AgdaBound{W}\<%
\\
\>\AgdaSymbol{(}\AgdaBound{ρ'} \AgdaFunction{•R} \AgdaBound{ρ}\AgdaSymbol{)} \AgdaBound{K} \AgdaBound{x} \AgdaSymbol{=} \AgdaBound{ρ'} \AgdaBound{K} \AgdaSymbol{(}\AgdaBound{ρ} \AgdaBound{K} \AgdaBound{x}\AgdaSymbol{)}\<%
\\
%
\\
\>\AgdaFunction{liftRep-comp} \AgdaSymbol{:} \AgdaSymbol{∀} \AgdaSymbol{\{}\AgdaBound{U}\AgdaSymbol{\}} \AgdaSymbol{\{}\AgdaBound{V}\AgdaSymbol{\}} \AgdaSymbol{\{}\AgdaBound{W}\AgdaSymbol{\}} \AgdaSymbol{\{}\AgdaBound{K}\AgdaSymbol{\}} \AgdaSymbol{\{}\AgdaBound{ρ'} \AgdaSymbol{:} \AgdaFunction{Rep} \AgdaBound{V} \AgdaBound{W}\AgdaSymbol{\}} \AgdaSymbol{\{}\AgdaBound{ρ} \AgdaSymbol{:} \AgdaFunction{Rep} \AgdaBound{U} \AgdaBound{V}\AgdaSymbol{\}} \AgdaSymbol{→} \<[64]%
\>[64]\<%
\\
\>[0]\AgdaIndent{2}{}\<[2]%
\>[2]\AgdaFunction{liftRep} \AgdaBound{K} \AgdaSymbol{(}\AgdaBound{ρ'} \AgdaFunction{•R} \AgdaBound{ρ}\AgdaSymbol{)} \AgdaFunction{∼R} \AgdaFunction{liftRep} \AgdaBound{K} \AgdaBound{ρ'} \AgdaFunction{•R} \AgdaFunction{liftRep} \AgdaBound{K} \AgdaBound{ρ}\<%
\end{code}

\AgdaHide{
\begin{code}%
\>\AgdaFunction{liftRep-comp} \AgdaInductiveConstructor{x₀} \AgdaSymbol{=} \AgdaInductiveConstructor{refl}\<%
\\
\>\AgdaFunction{liftRep-comp} \AgdaSymbol{(}\AgdaInductiveConstructor{↑} \AgdaSymbol{\_)} \AgdaSymbol{=} \AgdaInductiveConstructor{refl}\<%
\end{code}
}

\begin{code}%
\>\AgdaFunction{REP} \AgdaSymbol{:} \AgdaRecord{OpFamily}\<%
\\
\>\AgdaFunction{REP} \AgdaSymbol{=} \AgdaKeyword{record} \AgdaSymbol{\{} \<[15]%
\>[15]\<%
\\
\>[0]\AgdaIndent{2}{}\<[2]%
\>[2]\AgdaField{liftFamily} \AgdaSymbol{=} \AgdaFunction{Rep∶LF} \AgdaSymbol{;} \<[24]%
\>[24]\<%
\\
\>[0]\AgdaIndent{2}{}\<[2]%
\>[2]\AgdaField{comp} \AgdaSymbol{=} \AgdaKeyword{record} \AgdaSymbol{\{} \<[18]%
\>[18]\<%
\\
\>[2]\AgdaIndent{4}{}\<[4]%
\>[4]\AgdaField{\_∘\_} \AgdaSymbol{=} \AgdaFunction{\_•R\_} \AgdaSymbol{;} \<[17]%
\>[17]\<%
\\
\>[2]\AgdaIndent{4}{}\<[4]%
\>[4]\AgdaField{apV-comp} \AgdaSymbol{=} \AgdaInductiveConstructor{refl} \AgdaSymbol{;} \<[22]%
\>[22]\<%
\\
\>[2]\AgdaIndent{4}{}\<[4]%
\>[4]\AgdaField{liftOp-comp} \AgdaSymbol{=} \AgdaFunction{liftRep-comp} \AgdaSymbol{\}} \AgdaSymbol{\}}\<%
\end{code}

\AgdaHide{
\begin{code}%
\>\AgdaFunction{•R-congl} \AgdaSymbol{:} \AgdaSymbol{∀} \AgdaSymbol{\{}\AgdaBound{U} \AgdaBound{V} \AgdaBound{W}\AgdaSymbol{\}} \AgdaSymbol{\{}\AgdaBound{ρ₁} \AgdaBound{ρ₂} \AgdaSymbol{:} \AgdaFunction{Rep} \AgdaBound{V} \AgdaBound{W}\AgdaSymbol{\}} \AgdaSymbol{→} \AgdaBound{ρ₁} \AgdaFunction{∼R} \AgdaBound{ρ₂} \AgdaSymbol{→} \AgdaSymbol{∀} \AgdaSymbol{(}\AgdaBound{ρ₃} \AgdaSymbol{:} \AgdaFunction{Rep} \AgdaBound{U} \AgdaBound{V}\AgdaSymbol{)} \AgdaSymbol{→} \AgdaBound{ρ₁} \AgdaFunction{•R} \AgdaBound{ρ₃} \AgdaFunction{∼R} \AgdaBound{ρ₂} \AgdaFunction{•R} \AgdaBound{ρ₃}\<%
\\
\>\AgdaFunction{•R-congl} \AgdaSymbol{=} \AgdaFunction{OpFamily.comp-congl} \AgdaFunction{REP}\<%
\\
%
\\
\>\AgdaFunction{•R-congr} \AgdaSymbol{:} \AgdaSymbol{∀} \AgdaSymbol{\{}\AgdaBound{U} \AgdaBound{V} \AgdaBound{W}\AgdaSymbol{\}} \AgdaSymbol{\{}\AgdaBound{ρ₁} \AgdaSymbol{:} \AgdaFunction{Rep} \AgdaBound{V} \AgdaBound{W}\AgdaSymbol{\}} \AgdaSymbol{\{}\AgdaBound{ρ₂} \AgdaBound{ρ₃} \AgdaSymbol{:} \AgdaFunction{Rep} \AgdaBound{U} \AgdaBound{V}\AgdaSymbol{\}} \AgdaSymbol{→} \AgdaBound{ρ₂} \AgdaFunction{∼R} \AgdaBound{ρ₃} \AgdaSymbol{→} \AgdaBound{ρ₁} \AgdaFunction{•R} \AgdaBound{ρ₂} \AgdaFunction{∼R} \AgdaBound{ρ₁} \AgdaFunction{•R} \AgdaBound{ρ₃}\<%
\\
\>\AgdaFunction{•R-congr} \AgdaSymbol{\{}\AgdaArgument{ρ₁} \AgdaSymbol{=} \AgdaBound{ρ₁}\AgdaSymbol{\}} \AgdaSymbol{=} \AgdaFunction{OpFamily.comp-congr} \AgdaFunction{REP} \AgdaBound{ρ₁}\<%
\\
%
\\
\>\AgdaFunction{rep-congr} \AgdaSymbol{:} \AgdaSymbol{∀} \AgdaSymbol{\{}\AgdaBound{U} \AgdaBound{V} \AgdaBound{C} \AgdaBound{K}\AgdaSymbol{\}} \AgdaSymbol{\{}\AgdaBound{ρ} \AgdaBound{ρ'} \AgdaSymbol{:} \AgdaFunction{Rep} \AgdaBound{U} \AgdaBound{V}\AgdaSymbol{\}} \AgdaSymbol{→} \AgdaBound{ρ} \AgdaFunction{∼R} \AgdaBound{ρ'} \AgdaSymbol{→} \AgdaSymbol{∀} \AgdaSymbol{(}\AgdaBound{E} \AgdaSymbol{:} \AgdaDatatype{Subexpression} \AgdaBound{U} \AgdaBound{C} \AgdaBound{K}\AgdaSymbol{)} \AgdaSymbol{→} \AgdaBound{E} \AgdaFunction{〈} \AgdaBound{ρ} \AgdaFunction{〉} \AgdaDatatype{≡} \AgdaBound{E} \AgdaFunction{〈} \AgdaBound{ρ'} \AgdaFunction{〉}\<%
\\
\>\AgdaFunction{rep-congr} \AgdaSymbol{=} \AgdaFunction{OpFamily.ap-congl} \AgdaFunction{REP}\<%
\\
%
\\
\>\AgdaFunction{rep-congl} \AgdaSymbol{:} \AgdaSymbol{∀} \AgdaSymbol{\{}\AgdaBound{U} \AgdaBound{V} \AgdaBound{C} \AgdaBound{K}\AgdaSymbol{\}} \AgdaSymbol{\{}\AgdaBound{ρ} \AgdaSymbol{:} \AgdaFunction{Rep} \AgdaBound{U} \AgdaBound{V}\AgdaSymbol{\}} \AgdaSymbol{\{}\AgdaBound{E} \AgdaBound{F} \AgdaSymbol{:} \AgdaDatatype{Subexpression} \AgdaBound{U} \AgdaBound{C} \AgdaBound{K}\AgdaSymbol{\}} \AgdaSymbol{→} \AgdaBound{E} \AgdaDatatype{≡} \AgdaBound{F} \AgdaSymbol{→} \AgdaBound{E} \AgdaFunction{〈} \AgdaBound{ρ} \AgdaFunction{〉} \AgdaDatatype{≡} \AgdaBound{F} \AgdaFunction{〈} \AgdaBound{ρ} \AgdaFunction{〉}\<%
\\
\>\AgdaFunction{rep-congl} \AgdaSymbol{=} \AgdaFunction{OpFamily.ap-congr} \AgdaFunction{REP}\<%
\\
%
\\
\>\AgdaFunction{rep-idOp} \AgdaSymbol{:} \AgdaSymbol{∀} \AgdaSymbol{\{}\AgdaBound{V} \AgdaBound{C} \AgdaBound{K}\AgdaSymbol{\}} \AgdaSymbol{\{}\AgdaBound{E} \AgdaSymbol{:} \AgdaDatatype{Subexpression} \AgdaBound{V} \AgdaBound{C} \AgdaBound{K}\AgdaSymbol{\}} \AgdaSymbol{→} \AgdaBound{E} \AgdaFunction{〈} \AgdaFunction{idRep} \AgdaBound{V} \AgdaFunction{〉} \AgdaDatatype{≡} \AgdaBound{E}\<%
\\
\>\AgdaFunction{rep-idOp} \AgdaSymbol{=} \AgdaFunction{OpFamily.ap-idOp} \AgdaFunction{REP}\<%
\\
%
\\
\>\AgdaFunction{rep-comp} \AgdaSymbol{:} \AgdaSymbol{∀} \AgdaSymbol{\{}\AgdaBound{U} \AgdaBound{V} \AgdaBound{W} \AgdaBound{C} \AgdaBound{K}\AgdaSymbol{\}} \AgdaSymbol{(}\AgdaBound{E} \AgdaSymbol{:} \AgdaDatatype{Subexpression} \AgdaBound{U} \AgdaBound{C} \AgdaBound{K}\AgdaSymbol{)} \AgdaSymbol{\{}\AgdaBound{σ} \AgdaSymbol{:} \AgdaFunction{Rep} \AgdaBound{V} \AgdaBound{W}\AgdaSymbol{\}} \AgdaSymbol{\{}\AgdaBound{ρ}\AgdaSymbol{\}} \AgdaSymbol{→} \AgdaBound{E} \AgdaFunction{〈} \AgdaBound{σ} \AgdaFunction{•R} \AgdaBound{ρ} \AgdaFunction{〉} \AgdaDatatype{≡} \AgdaBound{E} \AgdaFunction{〈} \AgdaBound{ρ} \AgdaFunction{〉} \AgdaFunction{〈} \AgdaBound{σ} \AgdaFunction{〉}\<%
\\
\>\AgdaFunction{rep-comp} \AgdaSymbol{=} \AgdaFunction{OpFamily.ap-comp} \AgdaFunction{REP}\<%
\\
%
\\
\>\AgdaFunction{liftRep-idOp} \AgdaSymbol{:} \AgdaSymbol{∀} \AgdaSymbol{\{}\AgdaBound{V} \AgdaBound{K}\AgdaSymbol{\}} \AgdaSymbol{→} \AgdaFunction{liftRep} \AgdaBound{K} \AgdaSymbol{(}\AgdaFunction{idRep} \AgdaBound{V}\AgdaSymbol{)} \AgdaFunction{∼R} \AgdaFunction{idRep} \AgdaSymbol{(}\AgdaBound{V} \AgdaInductiveConstructor{,} \AgdaBound{K}\AgdaSymbol{)}\<%
\\
\>\AgdaFunction{liftRep-idOp} \AgdaSymbol{=} \AgdaFunction{OpFamily.liftOp-idOp} \AgdaFunction{REP}\<%
\\
%
\\
\>\AgdaFunction{liftRep-upRep} \AgdaSymbol{:} \AgdaSymbol{∀} \AgdaSymbol{\{}\AgdaBound{U} \AgdaBound{V} \AgdaBound{C} \AgdaBound{K} \AgdaBound{L}\AgdaSymbol{\}} \AgdaSymbol{\{}\AgdaBound{σ} \AgdaSymbol{:} \AgdaFunction{Rep} \AgdaBound{U} \AgdaBound{V}\AgdaSymbol{\}} \AgdaSymbol{(}\AgdaBound{E} \AgdaSymbol{:} \AgdaDatatype{Subexpression} \AgdaBound{U} \AgdaBound{C} \AgdaBound{K}\AgdaSymbol{)} \AgdaSymbol{→} \AgdaBound{E} \AgdaFunction{〈} \AgdaFunction{upRep} \AgdaFunction{〉} \AgdaFunction{〈} \AgdaFunction{liftRep} \AgdaBound{L} \AgdaBound{σ} \AgdaFunction{〉} \AgdaDatatype{≡} \AgdaBound{E} \AgdaFunction{〈} \AgdaBound{σ} \AgdaFunction{〉} \AgdaFunction{〈} \AgdaFunction{upRep} \AgdaFunction{〉}\<%
\\
\>\AgdaFunction{liftRep-upRep} \AgdaSymbol{=} \AgdaFunction{OpFamily.liftOp-up} \AgdaFunction{REP}\<%
\\
%
\\
\>\AgdaFunction{liftRep-comp₄} \AgdaSymbol{:} \AgdaSymbol{∀} \AgdaSymbol{\{}\AgdaBound{U}\AgdaSymbol{\}} \AgdaSymbol{\{}\AgdaBound{V1}\AgdaSymbol{\}} \AgdaSymbol{\{}\AgdaBound{V2}\AgdaSymbol{\}} \AgdaSymbol{\{}\AgdaBound{V3}\AgdaSymbol{\}} \AgdaSymbol{\{}\AgdaBound{V4}\AgdaSymbol{\}} \AgdaSymbol{\{}\AgdaBound{K}\AgdaSymbol{\}} \AgdaSymbol{\{}\AgdaBound{ρ1} \AgdaSymbol{:} \AgdaFunction{Rep} \AgdaBound{U} \AgdaBound{V1}\AgdaSymbol{\}} \AgdaSymbol{\{}\AgdaBound{ρ2} \AgdaSymbol{:} \AgdaFunction{Rep} \AgdaBound{V1} \AgdaBound{V2}\AgdaSymbol{\}} \AgdaSymbol{\{}\AgdaBound{ρ3} \AgdaSymbol{:} \AgdaFunction{Rep} \AgdaBound{V2} \AgdaBound{V3}\AgdaSymbol{\}} \AgdaSymbol{\{}\AgdaBound{ρ4} \AgdaSymbol{:} \AgdaFunction{Rep} \AgdaBound{V3} \AgdaBound{V4}\AgdaSymbol{\}} \AgdaSymbol{→}\<%
\\
\>[4]\AgdaIndent{16}{}\<[16]%
\>[16]\AgdaFunction{liftRep} \AgdaBound{K} \AgdaSymbol{(}\AgdaBound{ρ4} \AgdaFunction{•R} \AgdaBound{ρ3} \AgdaFunction{•R} \AgdaBound{ρ2} \AgdaFunction{•R} \AgdaBound{ρ1}\AgdaSymbol{)} \AgdaFunction{∼R} \AgdaFunction{liftRep} \AgdaBound{K} \AgdaBound{ρ4} \AgdaFunction{•R} \AgdaFunction{liftRep} \AgdaBound{K} \AgdaBound{ρ3} \AgdaFunction{•R} \AgdaFunction{liftRep} \AgdaBound{K} \AgdaBound{ρ2} \AgdaFunction{•R} \AgdaFunction{liftRep} \AgdaBound{K} \AgdaBound{ρ1}\<%
\\
\>\AgdaFunction{liftRep-comp₄} \AgdaSymbol{\{}\AgdaBound{U}\AgdaSymbol{\}} \AgdaSymbol{\{}\AgdaBound{V1}\AgdaSymbol{\}} \AgdaSymbol{\{}\AgdaBound{V2}\AgdaSymbol{\}} \AgdaSymbol{\{}\AgdaBound{V3}\AgdaSymbol{\}} \AgdaSymbol{\{}\AgdaBound{V4}\AgdaSymbol{\}} \AgdaSymbol{\{}\AgdaBound{K}\AgdaSymbol{\}} \AgdaSymbol{\{}\AgdaBound{ρ1}\AgdaSymbol{\}} \AgdaSymbol{\{}\AgdaBound{ρ2}\AgdaSymbol{\}} \AgdaSymbol{\{}\AgdaBound{ρ3}\AgdaSymbol{\}} \AgdaSymbol{\{}\AgdaBound{ρ4}\AgdaSymbol{\}} \AgdaSymbol{=}\<%
\\
\>[0]\AgdaIndent{2}{}\<[2]%
\>[2]\AgdaKeyword{let} \AgdaKeyword{open} \AgdaModule{Prelims.}\AgdaModule{EqReasoning} \AgdaSymbol{(}\AgdaFunction{PreOpFamily.OP} \AgdaFunction{Rep∶POF} \AgdaSymbol{(}\AgdaBound{U} \AgdaInductiveConstructor{,} \AgdaBound{K}\AgdaSymbol{)} \AgdaSymbol{(}\AgdaBound{V4} \AgdaInductiveConstructor{,} \AgdaBound{K}\AgdaSymbol{))} \AgdaKeyword{in} \<[76]%
\>[76]\<%
\\
\>[0]\AgdaIndent{2}{}\<[2]%
\>[2]\AgdaFunction{begin}\<%
\\
\>[2]\AgdaIndent{4}{}\<[4]%
\>[4]\AgdaFunction{liftRep} \AgdaBound{K} \AgdaSymbol{(}\AgdaBound{ρ4} \AgdaFunction{•R} \AgdaBound{ρ3} \AgdaFunction{•R} \AgdaBound{ρ2} \AgdaFunction{•R} \AgdaBound{ρ1}\AgdaSymbol{)}\<%
\\
\>[0]\AgdaIndent{2}{}\<[2]%
\>[2]\AgdaFunction{≈⟨} \AgdaFunction{liftRep-comp} \AgdaFunction{⟩}\<%
\\
\>[2]\AgdaIndent{4}{}\<[4]%
\>[4]\AgdaFunction{liftRep} \AgdaBound{K} \AgdaSymbol{(}\AgdaBound{ρ4} \AgdaFunction{•R} \AgdaBound{ρ3} \AgdaFunction{•R} \AgdaBound{ρ2}\AgdaSymbol{)} \AgdaFunction{•R} \AgdaFunction{liftRep} \AgdaBound{K} \AgdaBound{ρ1}\<%
\\
\>[0]\AgdaIndent{2}{}\<[2]%
\>[2]\AgdaFunction{≈⟨} \AgdaFunction{•R-congl} \AgdaFunction{liftRep-comp} \AgdaSymbol{(}\AgdaFunction{liftRep} \AgdaBound{K} \AgdaBound{ρ1}\AgdaSymbol{)}\AgdaFunction{⟩}\<%
\\
\>[2]\AgdaIndent{4}{}\<[4]%
\>[4]\AgdaFunction{liftRep} \AgdaBound{K} \AgdaSymbol{(}\AgdaBound{ρ4} \AgdaFunction{•R} \AgdaBound{ρ3}\AgdaSymbol{)} \AgdaFunction{•R} \AgdaFunction{liftRep} \AgdaBound{K} \AgdaBound{ρ2} \AgdaFunction{•R} \AgdaFunction{liftRep} \AgdaBound{K} \AgdaBound{ρ1}\<%
\\
\>[0]\AgdaIndent{2}{}\<[2]%
\>[2]\AgdaFunction{≈⟨} \AgdaFunction{•R-congl} \AgdaSymbol{(}\AgdaFunction{•R-congl} \AgdaFunction{liftRep-comp} \AgdaSymbol{(}\AgdaFunction{liftRep} \AgdaBound{K} \AgdaBound{ρ2}\AgdaSymbol{))} \AgdaSymbol{(}\AgdaFunction{liftRep} \AgdaBound{K} \AgdaBound{ρ1}\AgdaSymbol{)}\AgdaFunction{⟩}\<%
\\
\>[2]\AgdaIndent{4}{}\<[4]%
\>[4]\AgdaFunction{liftRep} \AgdaBound{K} \AgdaBound{ρ4} \AgdaFunction{•R} \AgdaFunction{liftRep} \AgdaBound{K} \AgdaBound{ρ3} \AgdaFunction{•R} \AgdaFunction{liftRep} \AgdaBound{K} \AgdaBound{ρ2} \AgdaFunction{•R} \AgdaFunction{liftRep} \AgdaBound{K} \AgdaBound{ρ1}\<%
\\
\>[0]\AgdaIndent{2}{}\<[2]%
\>[2]\AgdaFunction{∎}\<%
\\
%
\\
\>\AgdaFunction{rep-comp₄} \AgdaSymbol{:} \AgdaSymbol{∀} \AgdaSymbol{\{}\AgdaBound{U}\AgdaSymbol{\}} \AgdaSymbol{\{}\AgdaBound{V1}\AgdaSymbol{\}} \AgdaSymbol{\{}\AgdaBound{V2}\AgdaSymbol{\}} \AgdaSymbol{\{}\AgdaBound{V3}\AgdaSymbol{\}} \AgdaSymbol{\{}\AgdaBound{V4}\AgdaSymbol{\}} \<[38]%
\>[38]\<%
\\
\>[2]\AgdaIndent{12}{}\<[12]%
\>[12]\AgdaSymbol{\{}\AgdaBound{ρ1} \AgdaSymbol{:} \AgdaFunction{Rep} \AgdaBound{U} \AgdaBound{V1}\AgdaSymbol{\}} \AgdaSymbol{\{}\AgdaBound{ρ2} \AgdaSymbol{:} \AgdaFunction{Rep} \AgdaBound{V1} \AgdaBound{V2}\AgdaSymbol{\}} \AgdaSymbol{\{}\AgdaBound{ρ3} \AgdaSymbol{:} \AgdaFunction{Rep} \AgdaBound{V2} \AgdaBound{V3}\AgdaSymbol{\}} \AgdaSymbol{\{}\AgdaBound{ρ4} \AgdaSymbol{:} \AgdaFunction{Rep} \AgdaBound{V3} \AgdaBound{V4}\AgdaSymbol{\}} \<[79]%
\>[79]\<%
\\
\>[2]\AgdaIndent{12}{}\<[12]%
\>[12]\AgdaSymbol{\{}\AgdaBound{C}\AgdaSymbol{\}} \AgdaSymbol{\{}\AgdaBound{K}\AgdaSymbol{\}} \AgdaSymbol{(}\AgdaBound{E} \AgdaSymbol{:} \AgdaDatatype{Subexpression} \AgdaBound{U} \AgdaBound{C} \AgdaBound{K}\AgdaSymbol{)} \AgdaSymbol{→}\<%
\\
\>[2]\AgdaIndent{12}{}\<[12]%
\>[12]\AgdaBound{E} \AgdaFunction{〈} \AgdaBound{ρ4} \AgdaFunction{•R} \AgdaBound{ρ3} \AgdaFunction{•R} \AgdaBound{ρ2} \AgdaFunction{•R} \AgdaBound{ρ1} \AgdaFunction{〉} \AgdaDatatype{≡} \AgdaBound{E} \AgdaFunction{〈} \AgdaBound{ρ1} \AgdaFunction{〉} \AgdaFunction{〈} \AgdaBound{ρ2} \AgdaFunction{〉} \AgdaFunction{〈} \AgdaBound{ρ3} \AgdaFunction{〉} \AgdaFunction{〈} \AgdaBound{ρ4} \AgdaFunction{〉}\<%
\\
\>\AgdaFunction{rep-comp₄} \AgdaSymbol{\{}\AgdaBound{U}\AgdaSymbol{\}} \AgdaSymbol{\{}\AgdaBound{V1}\AgdaSymbol{\}} \AgdaSymbol{\{}\AgdaBound{V2}\AgdaSymbol{\}} \AgdaSymbol{\{}\AgdaBound{V3}\AgdaSymbol{\}} \AgdaSymbol{\{}\AgdaBound{V4}\AgdaSymbol{\}} \AgdaSymbol{\{}\AgdaBound{ρ1}\AgdaSymbol{\}} \AgdaSymbol{\{}\AgdaBound{ρ2}\AgdaSymbol{\}} \AgdaSymbol{\{}\AgdaBound{ρ3}\AgdaSymbol{\}} \AgdaSymbol{\{}\AgdaBound{ρ4}\AgdaSymbol{\}} \AgdaSymbol{\{}\AgdaBound{C}\AgdaSymbol{\}} \AgdaSymbol{\{}\AgdaBound{K}\AgdaSymbol{\}} \AgdaBound{E} \AgdaSymbol{=} \<[66]%
\>[66]\<%
\\
\>[0]\AgdaIndent{2}{}\<[2]%
\>[2]\AgdaKeyword{let} \AgdaKeyword{open} \AgdaModule{≡-Reasoning} \AgdaKeyword{in} \<[26]%
\>[26]\<%
\\
\>[0]\AgdaIndent{2}{}\<[2]%
\>[2]\AgdaFunction{begin}\<%
\\
\>[2]\AgdaIndent{4}{}\<[4]%
\>[4]\AgdaBound{E} \AgdaFunction{〈} \AgdaBound{ρ4} \AgdaFunction{•R} \AgdaBound{ρ3} \AgdaFunction{•R} \AgdaBound{ρ2} \AgdaFunction{•R} \AgdaBound{ρ1} \AgdaFunction{〉}\<%
\\
\>[4]\AgdaIndent{6}{}\<[6]%
\>[6]\AgdaFunction{≡⟨} \AgdaFunction{rep-comp} \AgdaBound{E} \AgdaFunction{⟩}\<%
\\
\>[0]\AgdaIndent{4}{}\<[4]%
\>[4]\AgdaBound{E} \AgdaFunction{〈} \AgdaBound{ρ1} \AgdaFunction{〉} \AgdaFunction{〈} \AgdaBound{ρ4} \AgdaFunction{•R} \AgdaBound{ρ3} \AgdaFunction{•R} \AgdaBound{ρ2} \AgdaFunction{〉}\<%
\\
\>[4]\AgdaIndent{6}{}\<[6]%
\>[6]\AgdaFunction{≡⟨} \AgdaFunction{rep-comp} \AgdaSymbol{(}\AgdaBound{E} \AgdaFunction{〈} \AgdaBound{ρ1} \AgdaFunction{〉}\AgdaSymbol{)} \AgdaFunction{⟩}\<%
\\
\>[0]\AgdaIndent{4}{}\<[4]%
\>[4]\AgdaBound{E} \AgdaFunction{〈} \AgdaBound{ρ1} \AgdaFunction{〉} \AgdaFunction{〈} \AgdaBound{ρ2} \AgdaFunction{〉} \AgdaFunction{〈} \AgdaBound{ρ4} \AgdaFunction{•R} \AgdaBound{ρ3} \AgdaFunction{〉}\<%
\\
\>[4]\AgdaIndent{6}{}\<[6]%
\>[6]\AgdaFunction{≡⟨} \AgdaFunction{rep-comp} \AgdaSymbol{(}\AgdaBound{E} \AgdaFunction{〈} \AgdaBound{ρ1} \AgdaFunction{〉} \AgdaFunction{〈} \AgdaBound{ρ2} \AgdaFunction{〉}\AgdaSymbol{)} \AgdaFunction{⟩}\<%
\\
\>[0]\AgdaIndent{4}{}\<[4]%
\>[4]\AgdaBound{E} \AgdaFunction{〈} \AgdaBound{ρ1} \AgdaFunction{〉} \AgdaFunction{〈} \AgdaBound{ρ2} \AgdaFunction{〉} \AgdaFunction{〈} \AgdaBound{ρ3} \AgdaFunction{〉} \AgdaFunction{〈} \AgdaBound{ρ4} \AgdaFunction{〉}\<%
\\
\>[0]\AgdaIndent{2}{}\<[2]%
\>[2]\AgdaFunction{∎}\<%
\end{code}
}

We write $E \uparrow$ for $E \langle \uparrow \rangle$.

\begin{code}%
\>\AgdaKeyword{infixl} \AgdaNumber{70} \AgdaFixityOp{\_⇑}\<%
\\
\>\AgdaFunction{\_⇑} \AgdaSymbol{:} \AgdaSymbol{∀} \AgdaSymbol{\{}\AgdaBound{V}\AgdaSymbol{\}} \AgdaSymbol{\{}\AgdaBound{K}\AgdaSymbol{\}} \AgdaSymbol{\{}\AgdaBound{C}\AgdaSymbol{\}} \AgdaSymbol{\{}\AgdaBound{L}\AgdaSymbol{\}} \AgdaSymbol{→} \AgdaDatatype{Subexpression} \AgdaBound{V} \AgdaBound{C} \AgdaBound{L} \AgdaSymbol{→} \AgdaDatatype{Subexpression} \AgdaSymbol{(}\AgdaBound{V} \AgdaInductiveConstructor{,} \AgdaBound{K}\AgdaSymbol{)} \AgdaBound{C} \AgdaBound{L}\<%
\\
\>\AgdaBound{E} \AgdaFunction{⇑} \AgdaSymbol{=} \AgdaBound{E} \AgdaFunction{〈} \AgdaFunction{upRep} \AgdaFunction{〉}\<%
\end{code}

We define the unique replacement $\emptyset \rightarrow V$ for any V, and prove it unique:

\begin{code}%
\>\AgdaFunction{magic} \AgdaSymbol{:} \AgdaSymbol{∀} \AgdaSymbol{\{}\AgdaBound{V}\AgdaSymbol{\}} \AgdaSymbol{→} \AgdaFunction{Rep} \AgdaInductiveConstructor{∅} \AgdaBound{V}\<%
\\
\>\AgdaFunction{magic} \AgdaSymbol{\_} \AgdaSymbol{()}\<%
\\
%
\\
\>\AgdaFunction{magic-unique} \AgdaSymbol{:} \AgdaSymbol{∀} \AgdaSymbol{\{}\AgdaBound{V}\AgdaSymbol{\}} \AgdaSymbol{\{}\AgdaBound{ρ} \AgdaSymbol{:} \AgdaFunction{Rep} \AgdaInductiveConstructor{∅} \AgdaBound{V}\AgdaSymbol{\}} \AgdaSymbol{→} \AgdaBound{ρ} \AgdaFunction{∼R} \AgdaFunction{magic}\<%
\end{code}

\AgdaHide{
\begin{code}%
\>\AgdaFunction{magic-unique} \AgdaSymbol{\{}\AgdaBound{V}\AgdaSymbol{\}} \AgdaSymbol{\{}\AgdaBound{ρ}\AgdaSymbol{\}} \AgdaSymbol{()}\<%
\end{code}
}

\begin{code}%
\>\AgdaFunction{magic-unique'} \AgdaSymbol{:} \AgdaSymbol{∀} \AgdaSymbol{\{}\AgdaBound{U}\AgdaSymbol{\}} \AgdaSymbol{\{}\AgdaBound{V}\AgdaSymbol{\}} \AgdaSymbol{\{}\AgdaBound{C}\AgdaSymbol{\}} \AgdaSymbol{\{}\AgdaBound{K}\AgdaSymbol{\}}\<%
\\
\>[0]\AgdaIndent{2}{}\<[2]%
\>[2]\AgdaSymbol{(}\AgdaBound{E} \AgdaSymbol{:} \AgdaDatatype{Subexpression} \AgdaInductiveConstructor{∅} \AgdaBound{C} \AgdaBound{K}\AgdaSymbol{)} \AgdaSymbol{\{}\AgdaBound{ρ} \AgdaSymbol{:} \AgdaFunction{Rep} \AgdaBound{U} \AgdaBound{V}\AgdaSymbol{\}} \AgdaSymbol{→} \<[44]%
\>[44]\<%
\\
\>[0]\AgdaIndent{2}{}\<[2]%
\>[2]\AgdaBound{E} \AgdaFunction{〈} \AgdaFunction{magic} \AgdaFunction{〉} \AgdaFunction{〈} \AgdaBound{ρ} \AgdaFunction{〉} \AgdaDatatype{≡} \AgdaBound{E} \AgdaFunction{〈} \AgdaFunction{magic} \AgdaFunction{〉}\<%
\end{code}

\AgdaHide{
\begin{code}%
\>\AgdaFunction{magic-unique'} \AgdaBound{E} \AgdaSymbol{\{}\AgdaBound{ρ}\AgdaSymbol{\}} \AgdaSymbol{=} \AgdaKeyword{let} \AgdaKeyword{open} \AgdaModule{≡-Reasoning} \AgdaKeyword{in}\<%
\\
\>[0]\AgdaIndent{2}{}\<[2]%
\>[2]\AgdaFunction{begin}\<%
\\
\>[2]\AgdaIndent{4}{}\<[4]%
\>[4]\AgdaBound{E} \AgdaFunction{〈} \AgdaFunction{magic} \AgdaFunction{〉} \AgdaFunction{〈} \AgdaBound{ρ} \AgdaFunction{〉}\<%
\\
\>[0]\AgdaIndent{2}{}\<[2]%
\>[2]\AgdaFunction{≡⟨⟨} \AgdaFunction{rep-comp} \AgdaBound{E} \AgdaFunction{⟩⟩}\<%
\\
\>[2]\AgdaIndent{4}{}\<[4]%
\>[4]\AgdaBound{E} \AgdaFunction{〈} \AgdaBound{ρ} \AgdaFunction{•R} \AgdaFunction{magic} \AgdaFunction{〉}\<%
\\
\>[0]\AgdaIndent{2}{}\<[2]%
\>[2]\AgdaFunction{≡⟨} \AgdaFunction{rep-congr} \AgdaSymbol{(}\AgdaFunction{magic-unique} \AgdaSymbol{\{}\AgdaArgument{ρ} \AgdaSymbol{=} \AgdaBound{ρ} \AgdaFunction{•R} \AgdaFunction{magic}\AgdaSymbol{\})} \AgdaBound{E} \AgdaFunction{⟩}\<%
\\
\>[2]\AgdaIndent{4}{}\<[4]%
\>[4]\AgdaBound{E} \AgdaFunction{〈} \AgdaFunction{magic} \AgdaFunction{〉}\<%
\\
\>[0]\AgdaIndent{2}{}\<[2]%
\>[2]\AgdaFunction{∎}\<%
\\
%
\\
\>\AgdaFunction{liftRep-upRep₂} \AgdaSymbol{:} \AgdaSymbol{∀} \AgdaSymbol{\{}\AgdaBound{U}\AgdaSymbol{\}} \AgdaSymbol{\{}\AgdaBound{V}\AgdaSymbol{\}} \AgdaSymbol{\{}\AgdaBound{C}\AgdaSymbol{\}} \AgdaSymbol{\{}\AgdaBound{K}\AgdaSymbol{\}} \AgdaSymbol{\{}\AgdaBound{L}\AgdaSymbol{\}} \AgdaSymbol{\{}\AgdaBound{M}\AgdaSymbol{\}} \AgdaSymbol{(}\AgdaBound{E} \AgdaSymbol{:} \AgdaDatatype{Subexpression} \AgdaBound{U} \AgdaBound{C} \AgdaBound{M}\AgdaSymbol{)} \AgdaSymbol{\{}\AgdaBound{ρ} \AgdaSymbol{:} \AgdaFunction{Rep} \AgdaBound{U} \AgdaBound{V}\AgdaSymbol{\}} \AgdaSymbol{→} \AgdaBound{E} \AgdaFunction{⇑} \AgdaFunction{⇑} \AgdaFunction{〈} \AgdaFunction{liftRep} \AgdaBound{K} \AgdaSymbol{(}\AgdaFunction{liftRep} \AgdaBound{L} \AgdaBound{ρ}\AgdaSymbol{)} \AgdaFunction{〉} \AgdaDatatype{≡} \AgdaBound{E} \AgdaFunction{〈} \AgdaBound{ρ} \AgdaFunction{〉} \AgdaFunction{⇑} \AgdaFunction{⇑}\<%
\\
\>\AgdaFunction{liftRep-upRep₂} \AgdaSymbol{\{}\AgdaBound{U}\AgdaSymbol{\}} \AgdaSymbol{\{}\AgdaBound{V}\AgdaSymbol{\}} \AgdaSymbol{\{}\AgdaBound{C}\AgdaSymbol{\}} \AgdaSymbol{\{}\AgdaBound{K}\AgdaSymbol{\}} \AgdaSymbol{\{}\AgdaBound{L}\AgdaSymbol{\}} \AgdaSymbol{\{}\AgdaBound{M}\AgdaSymbol{\}} \AgdaBound{E} \AgdaSymbol{\{}\AgdaBound{ρ}\AgdaSymbol{\}} \AgdaSymbol{=} \AgdaKeyword{let} \AgdaKeyword{open} \AgdaModule{≡-Reasoning} \AgdaKeyword{in} \<[71]%
\>[71]\<%
\\
\>[0]\AgdaIndent{2}{}\<[2]%
\>[2]\AgdaFunction{begin}\<%
\\
\>[2]\AgdaIndent{4}{}\<[4]%
\>[4]\AgdaBound{E} \AgdaFunction{⇑} \AgdaFunction{⇑} \AgdaFunction{〈} \AgdaFunction{liftRep} \AgdaBound{K} \AgdaSymbol{(}\AgdaFunction{liftRep} \AgdaBound{L} \AgdaBound{ρ}\AgdaSymbol{)} \AgdaFunction{〉}\<%
\\
\>[0]\AgdaIndent{2}{}\<[2]%
\>[2]\AgdaFunction{≡⟨} \AgdaFunction{liftRep-upRep} \AgdaSymbol{(}\AgdaBound{E} \AgdaFunction{⇑}\AgdaSymbol{)} \AgdaFunction{⟩}\<%
\\
\>[2]\AgdaIndent{4}{}\<[4]%
\>[4]\AgdaBound{E} \AgdaFunction{⇑} \AgdaFunction{〈} \AgdaFunction{liftRep} \AgdaBound{L} \AgdaBound{ρ} \AgdaFunction{〉} \AgdaFunction{⇑}\<%
\\
\>[0]\AgdaIndent{2}{}\<[2]%
\>[2]\AgdaFunction{≡⟨} \AgdaFunction{rep-congl} \AgdaSymbol{(}\AgdaFunction{liftRep-upRep} \AgdaBound{E}\AgdaSymbol{)} \AgdaFunction{⟩}\<%
\\
\>[2]\AgdaIndent{4}{}\<[4]%
\>[4]\AgdaBound{E} \AgdaFunction{〈} \AgdaBound{ρ} \AgdaFunction{〉} \AgdaFunction{⇑} \AgdaFunction{⇑}\<%
\\
\>[0]\AgdaIndent{2}{}\<[2]%
\>[2]\AgdaFunction{∎}\<%
\\
%
\\
\>\AgdaFunction{liftRep-upRep₃} \AgdaSymbol{:} \AgdaSymbol{∀} \AgdaSymbol{\{}\AgdaBound{U}\AgdaSymbol{\}} \AgdaSymbol{\{}\AgdaBound{V}\AgdaSymbol{\}} \AgdaSymbol{\{}\AgdaBound{C}\AgdaSymbol{\}} \AgdaSymbol{\{}\AgdaBound{K}\AgdaSymbol{\}} \AgdaSymbol{\{}\AgdaBound{L}\AgdaSymbol{\}} \AgdaSymbol{\{}\AgdaBound{M}\AgdaSymbol{\}} \AgdaSymbol{\{}\AgdaBound{N}\AgdaSymbol{\}} \AgdaSymbol{(}\AgdaBound{E} \AgdaSymbol{:} \AgdaDatatype{Subexpression} \AgdaBound{U} \AgdaBound{C} \AgdaBound{N}\AgdaSymbol{)} \AgdaSymbol{\{}\AgdaBound{ρ} \AgdaSymbol{:} \AgdaFunction{Rep} \AgdaBound{U} \AgdaBound{V}\AgdaSymbol{\}} \AgdaSymbol{→} \<[89]%
\>[89]\<%
\\
\>[0]\AgdaIndent{2}{}\<[2]%
\>[2]\AgdaBound{E} \AgdaFunction{⇑} \AgdaFunction{⇑} \AgdaFunction{⇑} \AgdaFunction{〈} \AgdaFunction{liftRep} \AgdaBound{K} \AgdaSymbol{(}\AgdaFunction{liftRep} \AgdaBound{L} \AgdaSymbol{(}\AgdaFunction{liftRep} \AgdaBound{M} \AgdaBound{ρ}\AgdaSymbol{))} \AgdaFunction{〉} \AgdaDatatype{≡} \AgdaBound{E} \AgdaFunction{〈} \AgdaBound{ρ} \AgdaFunction{〉} \AgdaFunction{⇑} \AgdaFunction{⇑} \AgdaFunction{⇑}\<%
\\
\>\AgdaFunction{liftRep-upRep₃} \AgdaSymbol{\{}\AgdaBound{U}\AgdaSymbol{\}} \AgdaSymbol{\{}\AgdaBound{V}\AgdaSymbol{\}} \AgdaSymbol{\{}\AgdaBound{C}\AgdaSymbol{\}} \AgdaSymbol{\{}\AgdaBound{K}\AgdaSymbol{\}} \AgdaSymbol{\{}\AgdaBound{L}\AgdaSymbol{\}} \AgdaSymbol{\{}\AgdaBound{M}\AgdaSymbol{\}} \AgdaBound{E} \AgdaSymbol{\{}\AgdaBound{ρ}\AgdaSymbol{\}} \AgdaSymbol{=} \AgdaKeyword{let} \AgdaKeyword{open} \AgdaModule{≡-Reasoning} \AgdaKeyword{in} \<[71]%
\>[71]\<%
\\
\>[0]\AgdaIndent{2}{}\<[2]%
\>[2]\AgdaFunction{begin}\<%
\\
\>[2]\AgdaIndent{4}{}\<[4]%
\>[4]\AgdaBound{E} \AgdaFunction{⇑} \AgdaFunction{⇑} \AgdaFunction{⇑} \AgdaFunction{〈} \AgdaFunction{liftRep} \AgdaBound{K} \AgdaSymbol{(}\AgdaFunction{liftRep} \AgdaBound{L} \AgdaSymbol{(}\AgdaFunction{liftRep} \AgdaBound{M} \AgdaBound{ρ}\AgdaSymbol{))} \AgdaFunction{〉}\<%
\\
\>[0]\AgdaIndent{2}{}\<[2]%
\>[2]\AgdaFunction{≡⟨} \AgdaFunction{liftRep-upRep₂} \AgdaSymbol{(}\AgdaBound{E} \AgdaFunction{⇑}\AgdaSymbol{)} \AgdaFunction{⟩}\<%
\\
\>[2]\AgdaIndent{4}{}\<[4]%
\>[4]\AgdaBound{E} \AgdaFunction{⇑} \AgdaFunction{〈} \AgdaFunction{liftRep} \AgdaBound{M} \AgdaBound{ρ} \AgdaFunction{〉} \AgdaFunction{⇑} \AgdaFunction{⇑}\<%
\\
\>[0]\AgdaIndent{2}{}\<[2]%
\>[2]\AgdaFunction{≡⟨} \AgdaFunction{rep-congl} \AgdaSymbol{(}\AgdaFunction{rep-congl} \AgdaSymbol{(}\AgdaFunction{liftRep-upRep} \AgdaBound{E}\AgdaSymbol{))} \AgdaFunction{⟩}\<%
\\
\>[2]\AgdaIndent{4}{}\<[4]%
\>[4]\AgdaBound{E} \AgdaFunction{〈} \AgdaBound{ρ} \AgdaFunction{〉} \AgdaFunction{⇑} \AgdaFunction{⇑} \AgdaFunction{⇑}\<%
\\
\>[0]\AgdaIndent{2}{}\<[2]%
\>[2]\AgdaFunction{∎}\<%
\\
%
\\
\>\AgdaKeyword{postulate} \AgdaPostulate{liftRep-upRep₄'} \AgdaSymbol{:} \AgdaSymbol{∀} \AgdaSymbol{\{}\AgdaBound{U}\AgdaSymbol{\}} \AgdaSymbol{\{}\AgdaBound{V}\AgdaSymbol{\}} \AgdaSymbol{(}\AgdaBound{ρ} \AgdaSymbol{:} \AgdaFunction{Rep} \AgdaBound{U} \AgdaBound{V}\AgdaSymbol{)} \AgdaSymbol{\{}\AgdaBound{K1}\AgdaSymbol{\}} \AgdaSymbol{\{}\AgdaBound{K2}\AgdaSymbol{\}} \AgdaSymbol{\{}\AgdaBound{K3}\AgdaSymbol{\}} \AgdaSymbol{→} \AgdaFunction{upRep} \AgdaFunction{•R} \AgdaFunction{upRep} \AgdaFunction{•R} \AgdaFunction{upRep} \AgdaFunction{•R} \AgdaBound{ρ} \AgdaFunction{∼R} \AgdaFunction{liftRep} \AgdaBound{K1} \AgdaSymbol{(}\AgdaFunction{liftRep} \AgdaBound{K2} \AgdaSymbol{(}\AgdaFunction{liftRep} \AgdaBound{K3} \AgdaBound{ρ}\AgdaSymbol{))} \AgdaFunction{•R} \AgdaFunction{upRep} \AgdaFunction{•R} \AgdaFunction{upRep} \AgdaFunction{•R} \AgdaFunction{upRep}\<%
\end{code}
}

\AgdaHide{
\begin{code}%
\>\AgdaKeyword{open} \AgdaKeyword{import} \AgdaModule{Grammar.Base}\<%
\\
%
\\
\>\AgdaKeyword{module} \AgdaModule{Grammar.OpFamily.PreOpFamily} \AgdaSymbol{(}\AgdaBound{G} \AgdaSymbol{:} \AgdaRecord{Grammar}\AgdaSymbol{)} \AgdaKeyword{where}\<%
\\
\>\AgdaKeyword{open} \AgdaKeyword{import} \AgdaModule{Prelims}\<%
\\
\>\AgdaKeyword{open} \AgdaModule{Grammar} \AgdaBound{G}\<%
\end{code}
}

\subsection{Families of Operations}

Our aim here is to define the operations of \emph{replacement} and \emph{substitution}.  In order to organise this work, we introduce the following definitions.

A \emph{family of operations} over a grammar $G$ consists of:
\begin{enumerate}
\item
for any alphabets $U$ and $V$, a set $F[U,V]$ of \emph{operations} $\sigma$ from $U$ to $V$, $\sigma : U \rightarrow V$;
\item
for any operation $\sigma : U \rightarrow V$ and variable $x \in U$ of kind $K$, an expression $\sigma(x)$ over $V$ of kind $K$;
\item
for any alphabet $V$ and variable kind $K$, an operation $\uparrow : V \rightarrow (V , K)$, the \emph{lifting} operation;
\item
for any alphabet $V$, an operation $\id{V} : V \rightarrow V$, the \emph{identity} operation;
\item
for any operation $\sigma : U \rightarrow V$ and variable kind $K$, an operation $(\sigma , K) : (U , K) \rightarrow (V , K)$, the result of \emph{lifting} $\sigma$;
\item
for any operations $\rho : U \rightarrow V$ and $\sigma : V \rightarrow W$, an operation
$\sigma \circ \rho : U \rightarrow W$, the \emph{composition} of $\sigma$ and $\rho$;
\end{enumerate}
such that:
\begin{itemize}
\item
$\uparrow (x) \equiv x$
\item
$\id{V}(x) \equiv x$
\item
If $\rho \sim \sigma$ then $(\rho , K) \sim (\sigma , K)$
\item
$(\rho , K)(x_0) \equiv x_0$
\item
Given $\sigma : U \rightarrow V$ and $x \in U$, we have $(\sigma , K)(x) \equiv x$
\item
$(\sigma \circ \rho , K) \sim (\sigma , K) \circ (\rho , K)$
\item
$(\sigma \circ \rho)(x) \equiv \rho(x) [ \sigma ]$
\end{itemize}
where for $\sigma, \rho : U \rightarrow V$ we write $\sigma \sim \rho$ iff $\sigma(x) \equiv \rho(x)$ for all $x \in U$; and, given $\sigma : U \rightarrow V$ and $E$ an expression over $U$, we define $E[\sigma]$, the result of \emph{applying} the operation $\sigma$ to $E$, as follows:

\begin{align*}
x[\sigma] & \eqdef \sigma(x) \\
\lefteqn{c([\vec{x_1}] E_1, \ldots, [\vec{x_n}] E_n) [\sigma]} \\
 & \eqdef
c([\vec{x_1}] E_1 [(\sigma , K_{11}, \ldots, K_{1r_1})], \ldots,
[\vec{x_n}] E_n [(\sigma, K_{n1}, \ldots, K_{nr_n})])
\end{align*}
for $c$ a constructor of type (\ref{eq:conkind}).

\subsubsection{Pre-Families}
We formalize this definition in stages.  First, we define a \emph{pre-family of operations} to be a family with items of data 1--4 above that satisfies the first two axioms:

\begin{code}%
\>\AgdaKeyword{record} \AgdaRecord{PreOpFamily} \AgdaSymbol{:} \AgdaPrimitiveType{Set₂} \AgdaKeyword{where}\<%
\\
\>[0]\AgdaIndent{2}{}\<[2]%
\>[2]\AgdaKeyword{field}\<%
\\
\>[2]\AgdaIndent{4}{}\<[4]%
\>[4]\AgdaField{Op} \AgdaSymbol{:} \AgdaDatatype{Alphabet} \AgdaSymbol{→} \AgdaDatatype{Alphabet} \AgdaSymbol{→} \AgdaPrimitiveType{Set}\<%
\\
\>[2]\AgdaIndent{4}{}\<[4]%
\>[4]\AgdaField{apV} \AgdaSymbol{:} \AgdaSymbol{∀} \AgdaSymbol{\{}\AgdaBound{U}\AgdaSymbol{\}} \AgdaSymbol{\{}\AgdaBound{V}\AgdaSymbol{\}} \AgdaSymbol{\{}\AgdaBound{K}\AgdaSymbol{\}} \AgdaSymbol{→} \AgdaField{Op} \AgdaBound{U} \AgdaBound{V} \AgdaSymbol{→} \AgdaDatatype{Var} \AgdaBound{U} \AgdaBound{K} \AgdaSymbol{→} \AgdaFunction{VExpression} \AgdaBound{V} \AgdaBound{K}\<%
\\
\>[2]\AgdaIndent{4}{}\<[4]%
\>[4]\AgdaField{up} \AgdaSymbol{:} \AgdaSymbol{∀} \AgdaSymbol{\{}\AgdaBound{V}\AgdaSymbol{\}} \AgdaSymbol{\{}\AgdaBound{K}\AgdaSymbol{\}} \AgdaSymbol{→} \AgdaField{Op} \AgdaBound{V} \AgdaSymbol{(}\AgdaBound{V} \AgdaInductiveConstructor{,} \AgdaBound{K}\AgdaSymbol{)}\<%
\\
\>[2]\AgdaIndent{4}{}\<[4]%
\>[4]\AgdaField{apV-up} \AgdaSymbol{:} \AgdaSymbol{∀} \AgdaSymbol{\{}\AgdaBound{V}\AgdaSymbol{\}} \AgdaSymbol{\{}\AgdaBound{K}\AgdaSymbol{\}} \AgdaSymbol{\{}\AgdaBound{L}\AgdaSymbol{\}} \AgdaSymbol{\{}\AgdaBound{x} \AgdaSymbol{:} \AgdaDatatype{Var} \AgdaBound{V} \AgdaBound{K}\AgdaSymbol{\}} \AgdaSymbol{→} \AgdaField{apV} \AgdaSymbol{(}\AgdaField{up} \AgdaSymbol{\{}\AgdaArgument{K} \AgdaSymbol{=} \AgdaBound{L}\AgdaSymbol{\})} \AgdaBound{x} \AgdaDatatype{≡} \AgdaInductiveConstructor{var} \AgdaSymbol{(}\AgdaInductiveConstructor{↑} \AgdaBound{x}\AgdaSymbol{)}\<%
\\
\>[2]\AgdaIndent{4}{}\<[4]%
\>[4]\AgdaField{idOp} \AgdaSymbol{:} \AgdaSymbol{∀} \AgdaBound{V} \AgdaSymbol{→} \AgdaField{Op} \AgdaBound{V} \AgdaBound{V}\<%
\\
\>[2]\AgdaIndent{4}{}\<[4]%
\>[4]\AgdaField{apV-idOp} \AgdaSymbol{:} \AgdaSymbol{∀} \AgdaSymbol{\{}\AgdaBound{V}\AgdaSymbol{\}} \AgdaSymbol{\{}\AgdaBound{K}\AgdaSymbol{\}} \AgdaSymbol{(}\AgdaBound{x} \AgdaSymbol{:} \AgdaDatatype{Var} \AgdaBound{V} \AgdaBound{K}\AgdaSymbol{)} \AgdaSymbol{→} \AgdaField{apV} \AgdaSymbol{(}\AgdaField{idOp} \AgdaBound{V}\AgdaSymbol{)} \AgdaBound{x} \AgdaDatatype{≡} \AgdaInductiveConstructor{var} \AgdaBound{x}\<%
\end{code}

This allows us to define the relation $\sim$, and prove it is an equivalence relation:

\begin{code}%
\>[0]\AgdaIndent{2}{}\<[2]%
\>[2]\AgdaFunction{\_∼op\_} \AgdaSymbol{:} \AgdaSymbol{∀} \AgdaSymbol{\{}\AgdaBound{U}\AgdaSymbol{\}} \AgdaSymbol{\{}\AgdaBound{V}\AgdaSymbol{\}} \AgdaSymbol{→} \AgdaField{Op} \AgdaBound{U} \AgdaBound{V} \AgdaSymbol{→} \AgdaField{Op} \AgdaBound{U} \AgdaBound{V} \AgdaSymbol{→} \AgdaPrimitiveType{Set}\<%
\\
\>[0]\AgdaIndent{2}{}\<[2]%
\>[2]\AgdaFunction{\_∼op\_} \AgdaSymbol{\{}\AgdaBound{U}\AgdaSymbol{\}} \AgdaSymbol{\{}\AgdaBound{V}\AgdaSymbol{\}} \AgdaBound{ρ} \AgdaBound{σ} \AgdaSymbol{=} \AgdaSymbol{∀} \AgdaSymbol{\{}\AgdaBound{K}\AgdaSymbol{\}} \AgdaSymbol{(}\AgdaBound{x} \AgdaSymbol{:} \AgdaDatatype{Var} \AgdaBound{U} \AgdaBound{K}\AgdaSymbol{)} \AgdaSymbol{→} \AgdaField{apV} \AgdaBound{ρ} \AgdaBound{x} \AgdaDatatype{≡} \AgdaField{apV} \AgdaBound{σ} \AgdaBound{x}\<%
\\
\>[2]\AgdaIndent{4}{}\<[4]%
\>[4]\<%
\\
\>[0]\AgdaIndent{2}{}\<[2]%
\>[2]\AgdaFunction{∼-refl} \AgdaSymbol{:} \AgdaSymbol{∀} \AgdaSymbol{\{}\AgdaBound{U}\AgdaSymbol{\}} \AgdaSymbol{\{}\AgdaBound{V}\AgdaSymbol{\}} \AgdaSymbol{\{}\AgdaBound{σ} \AgdaSymbol{:} \AgdaField{Op} \AgdaBound{U} \AgdaBound{V}\AgdaSymbol{\}} \AgdaSymbol{→} \AgdaBound{σ} \AgdaFunction{∼op} \AgdaBound{σ}\<%
\\
\>[0]\AgdaIndent{2}{}\<[2]%
\>[2]\AgdaFunction{∼-refl} \AgdaSymbol{\_} \AgdaSymbol{=} \AgdaInductiveConstructor{refl}\<%
\\
\>[2]\AgdaIndent{4}{}\<[4]%
\>[4]\<%
\\
\>[0]\AgdaIndent{2}{}\<[2]%
\>[2]\AgdaFunction{∼-sym} \AgdaSymbol{:} \AgdaSymbol{∀} \AgdaSymbol{\{}\AgdaBound{U}\AgdaSymbol{\}} \AgdaSymbol{\{}\AgdaBound{V}\AgdaSymbol{\}} \AgdaSymbol{\{}\AgdaBound{σ} \AgdaBound{τ} \AgdaSymbol{:} \AgdaField{Op} \AgdaBound{U} \AgdaBound{V}\AgdaSymbol{\}} \AgdaSymbol{→} \AgdaBound{σ} \AgdaFunction{∼op} \AgdaBound{τ} \AgdaSymbol{→} \AgdaBound{τ} \AgdaFunction{∼op} \AgdaBound{σ}\<%
\\
\>[0]\AgdaIndent{2}{}\<[2]%
\>[2]\AgdaFunction{∼-sym} \AgdaBound{σ-is-τ} \AgdaBound{x} \AgdaSymbol{=} \AgdaFunction{sym} \AgdaSymbol{(}\AgdaBound{σ-is-τ} \AgdaBound{x}\AgdaSymbol{)}\<%
\\
%
\\
\>[0]\AgdaIndent{2}{}\<[2]%
\>[2]\AgdaFunction{∼-trans} \AgdaSymbol{:} \AgdaSymbol{∀} \AgdaSymbol{\{}\AgdaBound{U}\AgdaSymbol{\}} \AgdaSymbol{\{}\AgdaBound{V}\AgdaSymbol{\}} \AgdaSymbol{\{}\AgdaBound{ρ} \AgdaBound{σ} \AgdaBound{τ} \AgdaSymbol{:} \AgdaField{Op} \AgdaBound{U} \AgdaBound{V}\AgdaSymbol{\}} \AgdaSymbol{→} \AgdaBound{ρ} \AgdaFunction{∼op} \AgdaBound{σ} \AgdaSymbol{→} \AgdaBound{σ} \AgdaFunction{∼op} \AgdaBound{τ} \AgdaSymbol{→} \AgdaBound{ρ} \AgdaFunction{∼op} \AgdaBound{τ}\<%
\\
\>[0]\AgdaIndent{2}{}\<[2]%
\>[2]\AgdaFunction{∼-trans} \AgdaBound{ρ-is-σ} \AgdaBound{σ-is-τ} \AgdaBound{x} \AgdaSymbol{=} \AgdaFunction{trans} \AgdaSymbol{(}\AgdaBound{ρ-is-σ} \AgdaBound{x}\AgdaSymbol{)} \AgdaSymbol{(}\AgdaBound{σ-is-τ} \AgdaBound{x}\AgdaSymbol{)}\<%
\\
%
\\
\>[0]\AgdaIndent{2}{}\<[2]%
\>[2]\AgdaFunction{OP} \AgdaSymbol{:} \AgdaDatatype{Alphabet} \AgdaSymbol{→} \AgdaDatatype{Alphabet} \AgdaSymbol{→} \<[30]%
\>[30]\AgdaRecord{Setoid} \AgdaSymbol{\_} \AgdaSymbol{\_}\<%
\\
\>[0]\AgdaIndent{2}{}\<[2]%
\>[2]\AgdaFunction{OP} \AgdaBound{U} \AgdaBound{V} \AgdaSymbol{=} \AgdaKeyword{record} \AgdaSymbol{\{} \<[20]%
\>[20]\<%
\\
\>[2]\AgdaIndent{5}{}\<[5]%
\>[5]\AgdaField{Carrier} \AgdaSymbol{=} \AgdaField{Op} \AgdaBound{U} \AgdaBound{V} \AgdaSymbol{;} \<[24]%
\>[24]\<%
\\
\>[2]\AgdaIndent{5}{}\<[5]%
\>[5]\AgdaField{\_≈\_} \AgdaSymbol{=} \AgdaFunction{\_∼op\_} \AgdaSymbol{;} \<[19]%
\>[19]\<%
\\
\>[2]\AgdaIndent{5}{}\<[5]%
\>[5]\AgdaField{isEquivalence} \AgdaSymbol{=} \AgdaKeyword{record} \AgdaSymbol{\{} \<[30]%
\>[30]\<%
\\
\>[5]\AgdaIndent{7}{}\<[7]%
\>[7]\AgdaField{refl} \AgdaSymbol{=} \AgdaFunction{∼-refl} \AgdaSymbol{;} \<[23]%
\>[23]\<%
\\
\>[5]\AgdaIndent{7}{}\<[7]%
\>[7]\AgdaField{sym} \AgdaSymbol{=} \AgdaFunction{∼-sym} \AgdaSymbol{;} \<[21]%
\>[21]\<%
\\
\>[5]\AgdaIndent{7}{}\<[7]%
\>[7]\AgdaField{trans} \AgdaSymbol{=} \AgdaFunction{∼-trans} \AgdaSymbol{\}} \AgdaSymbol{\}}\<%
\end{code}


\AgdaHide{
\begin{code}%
\>\AgdaKeyword{open} \AgdaKeyword{import} \AgdaModule{Grammar.Base}\<%
\\
%
\\
\>\AgdaKeyword{module} \AgdaModule{Grammar.OpFamily.Lifting} \AgdaSymbol{(}\AgdaBound{G} \AgdaSymbol{:} \AgdaRecord{Grammar}\AgdaSymbol{)} \AgdaKeyword{where}\<%
\\
\>\AgdaKeyword{open} \AgdaKeyword{import} \AgdaModule{Data.List}\<%
\\
\>\AgdaKeyword{open} \AgdaKeyword{import} \AgdaModule{Prelims}\<%
\\
\>\AgdaKeyword{open} \AgdaModule{Grammar} \AgdaBound{G}\<%
\\
\>\AgdaKeyword{open} \AgdaKeyword{import} \AgdaModule{Grammar.OpFamily.PreOpFamily} \AgdaBound{G}\<%
\end{code}
}

\subsubsection{Liftings}

Define a \emph{lifting} on a pre-family to be an function $(- , K)$ that respects $\sim$:

\begin{code}%
\>\AgdaKeyword{record} \AgdaRecord{Lifting} \AgdaSymbol{(}\AgdaBound{F} \AgdaSymbol{:} \AgdaRecord{PreOpFamily}\AgdaSymbol{)} \AgdaSymbol{:} \AgdaPrimitiveType{Set₁} \AgdaKeyword{where}\<%
\\
\>[0]\AgdaIndent{2}{}\<[2]%
\>[2]\AgdaKeyword{open} \AgdaModule{PreOpFamily} \AgdaBound{F}\<%
\\
\>[0]\AgdaIndent{2}{}\<[2]%
\>[2]\AgdaKeyword{field}\<%
\\
\>[2]\AgdaIndent{4}{}\<[4]%
\>[4]\AgdaField{liftOp} \AgdaSymbol{:} \AgdaSymbol{∀} \AgdaSymbol{\{}\AgdaBound{U}\AgdaSymbol{\}} \AgdaSymbol{\{}\AgdaBound{V}\AgdaSymbol{\}} \AgdaBound{K} \AgdaSymbol{→} \AgdaFunction{Op} \AgdaBound{U} \AgdaBound{V} \AgdaSymbol{→} \AgdaFunction{Op} \AgdaSymbol{(}\AgdaBound{U} \AgdaInductiveConstructor{,} \AgdaBound{K}\AgdaSymbol{)} \AgdaSymbol{(}\AgdaBound{V} \AgdaInductiveConstructor{,} \AgdaBound{K}\AgdaSymbol{)}\<%
\\
\>[2]\AgdaIndent{4}{}\<[4]%
\>[4]\AgdaField{liftOp-cong} \AgdaSymbol{:} \AgdaSymbol{∀} \AgdaSymbol{\{}\AgdaBound{V}\AgdaSymbol{\}} \AgdaSymbol{\{}\AgdaBound{W}\AgdaSymbol{\}} \AgdaSymbol{\{}\AgdaBound{K}\AgdaSymbol{\}} \AgdaSymbol{\{}\AgdaBound{ρ} \AgdaBound{σ} \AgdaSymbol{:} \AgdaFunction{Op} \AgdaBound{V} \AgdaBound{W}\AgdaSymbol{\}} \AgdaSymbol{→} \<[49]%
\>[49]\<%
\\
\>[4]\AgdaIndent{6}{}\<[6]%
\>[6]\AgdaBound{ρ} \AgdaFunction{∼op} \AgdaBound{σ} \AgdaSymbol{→} \AgdaField{liftOp} \AgdaBound{K} \AgdaBound{ρ} \AgdaFunction{∼op} \AgdaField{liftOp} \AgdaBound{K} \AgdaBound{σ}\<%
\end{code}

Given an operation $\sigma : U \rightarrow V$ and a list of variable kinds $A \equiv (A_1 , \ldots , A_n)$, define
the \emph{repeated lifting} $\sigma^A$ to be $((\cdots(\sigma , A_1) , A_2) , \cdots ) , A_n)$.

\begin{code}%
\>\AgdaComment{\{-  liftOp' : ∀ \{U\} \{V\} A → Op U V → Op (extend U A) (extend V A)\<\\
\>  liftOp' [] σ = σ\<\\
\>  liftOp' (K ∷ A) σ = liftOp' A (liftOp K σ) -\}}\<%
\\
%
\\
\>[0]\AgdaIndent{2}{}\<[2]%
\>[2]\AgdaFunction{liftOp''} \AgdaSymbol{:} \AgdaSymbol{∀} \AgdaSymbol{\{}\AgdaBound{U}\AgdaSymbol{\}} \AgdaSymbol{\{}\AgdaBound{V}\AgdaSymbol{\}} \AgdaSymbol{\{}\AgdaBound{K}\AgdaSymbol{\}} \AgdaBound{A} \AgdaSymbol{→} \AgdaFunction{Op} \AgdaBound{U} \AgdaBound{V} \AgdaSymbol{→} \AgdaFunction{Op} \AgdaSymbol{(}\AgdaFunction{dom} \AgdaBound{U} \AgdaSymbol{\{}\AgdaBound{K}\AgdaSymbol{\}} \AgdaBound{A}\AgdaSymbol{)} \AgdaSymbol{(}\AgdaFunction{dom} \AgdaBound{V} \AgdaBound{A}\AgdaSymbol{)}\<%
\\
\>[0]\AgdaIndent{2}{}\<[2]%
\>[2]\AgdaFunction{liftOp''} \AgdaSymbol{(\_} \AgdaInductiveConstructor{●}\AgdaSymbol{)} \AgdaBound{σ} \AgdaSymbol{=} \AgdaBound{σ}\<%
\\
\>[0]\AgdaIndent{2}{}\<[2]%
\>[2]\AgdaFunction{liftOp''} \AgdaSymbol{(}\AgdaBound{K} \AgdaInductiveConstructor{⟶} \AgdaBound{A}\AgdaSymbol{)} \AgdaBound{σ} \AgdaSymbol{=} \AgdaFunction{liftOp''} \AgdaBound{A} \AgdaSymbol{(}\AgdaField{liftOp} \AgdaBound{K} \AgdaBound{σ}\AgdaSymbol{)}\<%
\\
%
\\
\>\AgdaComment{\{-  liftOp'-cong : ∀ \{U\} \{V\} A \{ρ σ : Op U V\} → \<\\
\>    ρ ∼op σ → liftOp' A ρ ∼op liftOp' A σ\<\\
\>}\<%
\end{code}

\AgdaHide{
\begin{code}%
\>\AgdaComment{\<\\
\>  liftOp'-cong [] ρ-is-σ = ρ-is-σ\<\\
\>  liftOp'-cong (\_ ∷ A) ρ-is-σ = liftOp'-cong A (liftOp-cong ρ-is-σ) -\}}\<%
\\
%
\\
\>[0]\AgdaIndent{2}{}\<[2]%
\>[2]\AgdaKeyword{postulate} \AgdaPostulate{liftOp''-cong} \AgdaSymbol{:} \AgdaSymbol{∀} \AgdaSymbol{\{}\AgdaBound{U}\AgdaSymbol{\}} \AgdaSymbol{\{}\AgdaBound{V}\AgdaSymbol{\}} \AgdaSymbol{\{}\AgdaBound{K}\AgdaSymbol{\}} \AgdaBound{A} \AgdaSymbol{\{}\AgdaBound{ρ} \AgdaBound{σ} \AgdaSymbol{:} \AgdaFunction{Op} \AgdaBound{U} \AgdaBound{V}\AgdaSymbol{\}} \AgdaSymbol{→} \<[61]%
\>[61]\<%
\\
\>[2]\AgdaIndent{26}{}\<[26]%
\>[26]\AgdaBound{ρ} \AgdaFunction{∼op} \AgdaBound{σ} \AgdaSymbol{→} \AgdaFunction{liftOp''} \AgdaSymbol{\{}\AgdaArgument{K} \AgdaSymbol{=} \AgdaBound{K}\AgdaSymbol{\}} \AgdaBound{A} \AgdaBound{ρ} \AgdaFunction{∼op} \AgdaFunction{liftOp''} \AgdaBound{A} \AgdaBound{σ}\<%
\end{code}
}

This allows us to define the action of \emph{application} $E[\sigma]$:

\begin{code}%
\>[0]\AgdaIndent{2}{}\<[2]%
\>[2]\AgdaFunction{ap} \AgdaSymbol{:} \AgdaSymbol{∀} \AgdaSymbol{\{}\AgdaBound{U}\AgdaSymbol{\}} \AgdaSymbol{\{}\AgdaBound{V}\AgdaSymbol{\}} \AgdaSymbol{\{}\AgdaBound{C}\AgdaSymbol{\}} \AgdaSymbol{\{}\AgdaBound{K}\AgdaSymbol{\}} \AgdaSymbol{→} \<[27]%
\>[27]\<%
\\
\>[2]\AgdaIndent{4}{}\<[4]%
\>[4]\AgdaFunction{Op} \AgdaBound{U} \AgdaBound{V} \AgdaSymbol{→} \AgdaDatatype{Subexpression} \AgdaBound{U} \AgdaBound{C} \AgdaBound{K} \AgdaSymbol{→} \AgdaDatatype{Subexpression} \AgdaBound{V} \AgdaBound{C} \AgdaBound{K}\<%
\\
\>[0]\AgdaIndent{2}{}\<[2]%
\>[2]\AgdaFunction{ap} \AgdaBound{ρ} \AgdaSymbol{(}\AgdaInductiveConstructor{var} \AgdaBound{x}\AgdaSymbol{)} \AgdaSymbol{=} \AgdaFunction{apV} \AgdaBound{ρ} \AgdaBound{x}\<%
\\
\>[0]\AgdaIndent{2}{}\<[2]%
\>[2]\AgdaFunction{ap} \AgdaBound{ρ} \AgdaSymbol{(}\AgdaInductiveConstructor{app} \AgdaBound{c} \AgdaBound{EE}\AgdaSymbol{)} \AgdaSymbol{=} \AgdaInductiveConstructor{app} \AgdaBound{c} \AgdaSymbol{(}\AgdaFunction{ap} \AgdaBound{ρ} \AgdaBound{EE}\AgdaSymbol{)}\<%
\\
\>[0]\AgdaIndent{2}{}\<[2]%
\>[2]\AgdaFunction{ap} \AgdaSymbol{\_} \AgdaInductiveConstructor{out} \AgdaSymbol{=} \AgdaInductiveConstructor{out}\<%
\\
\>[0]\AgdaIndent{2}{}\<[2]%
\>[2]\AgdaFunction{ap} \AgdaBound{ρ} \AgdaSymbol{(}\AgdaInductiveConstructor{\_,,\_} \AgdaSymbol{\{}\AgdaArgument{A} \AgdaSymbol{=} \AgdaBound{A}\AgdaSymbol{\}} \AgdaBound{E} \AgdaBound{EE}\AgdaSymbol{)} \AgdaSymbol{=} \AgdaFunction{ap} \AgdaSymbol{(}\AgdaFunction{liftOp''} \AgdaBound{A} \AgdaBound{ρ}\AgdaSymbol{)} \AgdaBound{E} \AgdaInductiveConstructor{,,} \AgdaFunction{ap} \AgdaBound{ρ} \AgdaBound{EE}\<%
\end{code}

We prove that application respects $\sim$.

\begin{code}%
\>[0]\AgdaIndent{2}{}\<[2]%
\>[2]\AgdaFunction{ap-congl} \AgdaSymbol{:} \AgdaSymbol{∀} \AgdaSymbol{\{}\AgdaBound{U}\AgdaSymbol{\}} \AgdaSymbol{\{}\AgdaBound{V}\AgdaSymbol{\}} \AgdaSymbol{\{}\AgdaBound{C}\AgdaSymbol{\}} \AgdaSymbol{\{}\AgdaBound{K}\AgdaSymbol{\}} \<[31]%
\>[31]\<%
\\
\>[2]\AgdaIndent{4}{}\<[4]%
\>[4]\AgdaSymbol{\{}\AgdaBound{ρ} \AgdaBound{σ} \AgdaSymbol{:} \AgdaFunction{Op} \AgdaBound{U} \AgdaBound{V}\AgdaSymbol{\}} \AgdaSymbol{(}\AgdaBound{E} \AgdaSymbol{:} \AgdaDatatype{Subexpression} \AgdaBound{U} \AgdaBound{C} \AgdaBound{K}\AgdaSymbol{)} \AgdaSymbol{→}\<%
\\
\>[2]\AgdaIndent{4}{}\<[4]%
\>[4]\AgdaBound{ρ} \AgdaFunction{∼op} \AgdaBound{σ} \AgdaSymbol{→} \AgdaFunction{ap} \AgdaBound{ρ} \AgdaBound{E} \AgdaDatatype{≡} \AgdaFunction{ap} \AgdaBound{σ} \AgdaBound{E}\<%
\end{code}

\AgdaHide{
\begin{code}%
\>[0]\AgdaIndent{2}{}\<[2]%
\>[2]\AgdaFunction{ap-congl} \AgdaSymbol{(}\AgdaInductiveConstructor{var} \AgdaBound{x}\AgdaSymbol{)} \AgdaBound{ρ-is-σ} \AgdaSymbol{=} \AgdaBound{ρ-is-σ} \AgdaBound{x}\<%
\\
\>[0]\AgdaIndent{2}{}\<[2]%
\>[2]\AgdaFunction{ap-congl} \AgdaSymbol{(}\AgdaInductiveConstructor{app} \AgdaBound{c} \AgdaBound{E}\AgdaSymbol{)} \AgdaBound{ρ-is-σ} \AgdaSymbol{=} \AgdaFunction{cong} \AgdaSymbol{(}\AgdaInductiveConstructor{app} \AgdaBound{c}\AgdaSymbol{)} \AgdaSymbol{(}\AgdaFunction{ap-congl} \AgdaBound{E} \AgdaBound{ρ-is-σ}\AgdaSymbol{)}\<%
\\
\>[0]\AgdaIndent{2}{}\<[2]%
\>[2]\AgdaFunction{ap-congl} \AgdaInductiveConstructor{out} \AgdaSymbol{\_} \AgdaSymbol{=} \AgdaInductiveConstructor{refl}\<%
\\
\>[0]\AgdaIndent{2}{}\<[2]%
\>[2]\AgdaFunction{ap-congl} \AgdaSymbol{(}\AgdaInductiveConstructor{\_,,\_} \AgdaSymbol{\{}\AgdaArgument{L} \AgdaSymbol{=} \AgdaBound{L}\AgdaSymbol{\}} \AgdaSymbol{\{}\AgdaArgument{A} \AgdaSymbol{=} \AgdaBound{A}\AgdaSymbol{\}} \AgdaBound{E} \AgdaBound{F}\AgdaSymbol{)} \AgdaBound{ρ-is-σ} \AgdaSymbol{=} \<[47]%
\>[47]\<%
\\
\>[2]\AgdaIndent{4}{}\<[4]%
\>[4]\AgdaFunction{cong₂} \AgdaInductiveConstructor{\_,,\_} \AgdaSymbol{(}\AgdaFunction{ap-congl} \AgdaBound{E} \AgdaSymbol{(}\AgdaPostulate{liftOp''-cong} \AgdaBound{A} \AgdaBound{ρ-is-σ}\AgdaSymbol{))} \AgdaSymbol{(}\AgdaFunction{ap-congl} \AgdaBound{F} \AgdaBound{ρ-is-σ}\AgdaSymbol{)}\<%
\\
%
\\
\>[0]\AgdaIndent{2}{}\<[2]%
\>[2]\AgdaFunction{ap-congr} \AgdaSymbol{:} \AgdaSymbol{∀} \AgdaSymbol{\{}\AgdaBound{U}\AgdaSymbol{\}} \AgdaSymbol{\{}\AgdaBound{V}\AgdaSymbol{\}} \AgdaSymbol{\{}\AgdaBound{C}\AgdaSymbol{\}} \AgdaSymbol{\{}\AgdaBound{K}\AgdaSymbol{\}}\<%
\\
\>[2]\AgdaIndent{4}{}\<[4]%
\>[4]\AgdaSymbol{\{}\AgdaBound{σ} \AgdaSymbol{:} \AgdaFunction{Op} \AgdaBound{U} \AgdaBound{V}\AgdaSymbol{\}} \AgdaSymbol{\{}\AgdaBound{E} \AgdaBound{F} \AgdaSymbol{:} \AgdaDatatype{Subexpression} \AgdaBound{U} \AgdaBound{C} \AgdaBound{K}\AgdaSymbol{\}} \AgdaSymbol{→}\<%
\\
\>[2]\AgdaIndent{4}{}\<[4]%
\>[4]\AgdaBound{E} \AgdaDatatype{≡} \AgdaBound{F} \AgdaSymbol{→} \AgdaFunction{ap} \AgdaBound{σ} \AgdaBound{E} \AgdaDatatype{≡} \AgdaFunction{ap} \AgdaBound{σ} \AgdaBound{F}\<%
\\
\>[0]\AgdaIndent{2}{}\<[2]%
\>[2]\AgdaFunction{ap-congr} \AgdaSymbol{\{}\AgdaArgument{σ} \AgdaSymbol{=} \AgdaBound{σ}\AgdaSymbol{\}} \AgdaSymbol{=} \AgdaFunction{cong} \AgdaSymbol{(}\AgdaFunction{ap} \AgdaBound{σ}\AgdaSymbol{)}\<%
\\
%
\\
\>[0]\AgdaIndent{2}{}\<[2]%
\>[2]\AgdaFunction{ap-cong} \AgdaSymbol{:} \AgdaSymbol{∀} \AgdaSymbol{\{}\AgdaBound{U}\AgdaSymbol{\}} \AgdaSymbol{\{}\AgdaBound{V}\AgdaSymbol{\}} \AgdaSymbol{\{}\AgdaBound{C}\AgdaSymbol{\}} \AgdaSymbol{\{}\AgdaBound{K}\AgdaSymbol{\}}\<%
\\
\>[2]\AgdaIndent{4}{}\<[4]%
\>[4]\AgdaSymbol{\{}\AgdaBound{ρ} \AgdaBound{σ} \AgdaSymbol{:} \AgdaFunction{Op} \AgdaBound{U} \AgdaBound{V}\AgdaSymbol{\}} \AgdaSymbol{\{}\AgdaBound{M} \AgdaBound{N} \AgdaSymbol{:} \AgdaDatatype{Subexpression} \AgdaBound{U} \AgdaBound{C} \AgdaBound{K}\AgdaSymbol{\}} \AgdaSymbol{→}\<%
\\
\>[2]\AgdaIndent{4}{}\<[4]%
\>[4]\AgdaBound{ρ} \AgdaFunction{∼op} \AgdaBound{σ} \AgdaSymbol{→} \AgdaBound{M} \AgdaDatatype{≡} \AgdaBound{N} \AgdaSymbol{→} \AgdaFunction{ap} \AgdaBound{ρ} \AgdaBound{M} \AgdaDatatype{≡} \AgdaFunction{ap} \AgdaBound{σ} \AgdaBound{N}\<%
\\
\>[0]\AgdaIndent{2}{}\<[2]%
\>[2]\AgdaFunction{ap-cong} \AgdaSymbol{\{}\AgdaArgument{ρ} \AgdaSymbol{=} \AgdaBound{ρ}\AgdaSymbol{\}} \AgdaSymbol{\{}\AgdaBound{σ}\AgdaSymbol{\}} \AgdaSymbol{\{}\AgdaBound{M}\AgdaSymbol{\}} \AgdaSymbol{\{}\AgdaBound{N}\AgdaSymbol{\}} \AgdaBound{ρ∼σ} \AgdaBound{M≡N} \AgdaSymbol{=} \AgdaKeyword{let} \AgdaKeyword{open} \AgdaModule{≡-Reasoning} \AgdaKeyword{in} \<[64]%
\>[64]\<%
\\
\>[2]\AgdaIndent{4}{}\<[4]%
\>[4]\AgdaFunction{begin}\<%
\\
\>[4]\AgdaIndent{6}{}\<[6]%
\>[6]\AgdaFunction{ap} \AgdaBound{ρ} \AgdaBound{M}\<%
\\
\>[0]\AgdaIndent{4}{}\<[4]%
\>[4]\AgdaFunction{≡⟨} \AgdaFunction{ap-congl} \AgdaBound{M} \AgdaBound{ρ∼σ} \AgdaFunction{⟩}\<%
\\
\>[4]\AgdaIndent{6}{}\<[6]%
\>[6]\AgdaFunction{ap} \AgdaBound{σ} \AgdaBound{M}\<%
\\
\>[0]\AgdaIndent{4}{}\<[4]%
\>[4]\AgdaFunction{≡⟨} \AgdaFunction{ap-congr} \AgdaBound{M≡N} \AgdaFunction{⟩}\<%
\\
\>[4]\AgdaIndent{6}{}\<[6]%
\>[6]\AgdaFunction{ap} \AgdaBound{σ} \AgdaBound{N}\<%
\\
\>[0]\AgdaIndent{4}{}\<[4]%
\>[4]\AgdaFunction{∎}\<%
\end{code}
}

\begin{code}%
\>\AgdaKeyword{open} \AgdaKeyword{import} \AgdaModule{Grammar.Base}\<%
\\
%
\\
\>\AgdaKeyword{module} \AgdaModule{Grammar.Substitution.RepSub} \AgdaSymbol{(}\AgdaBound{G} \AgdaSymbol{:} \AgdaRecord{Grammar}\AgdaSymbol{)} \AgdaKeyword{where}\<%
\\
\>\AgdaKeyword{open} \AgdaKeyword{import} \AgdaModule{Data.List}\<%
\\
\>\AgdaKeyword{open} \AgdaKeyword{import} \AgdaModule{Prelims}\<%
\\
\>\AgdaKeyword{open} \AgdaModule{Grammar} \AgdaBound{G}\<%
\\
\>\AgdaKeyword{open} \AgdaKeyword{import} \AgdaModule{Grammar.OpFamily} \AgdaBound{G}\<%
\\
\>\AgdaKeyword{open} \AgdaKeyword{import} \AgdaModule{Grammar.Replacement} \AgdaBound{G}\<%
\\
\>\AgdaKeyword{open} \AgdaKeyword{import} \AgdaModule{Grammar.Substitution.PreOpFamily} \AgdaBound{G}\<%
\\
\>\AgdaKeyword{open} \AgdaKeyword{import} \AgdaModule{Grammar.Substitution.Lifting} \AgdaBound{G}\<%
\\
%
\\
\>\AgdaKeyword{open} \AgdaModule{OpFamily} \AgdaFunction{REP} \AgdaKeyword{using} \AgdaSymbol{()} \AgdaKeyword{renaming} \AgdaSymbol{(}liftsOp \AgdaSymbol{to} liftsOpR\AgdaSymbol{)}\<%
\\
\>\AgdaKeyword{open} \AgdaModule{PreOpFamily} \AgdaFunction{pre-substitution}\<%
\\
\>\AgdaKeyword{open} \AgdaModule{Lifting} \AgdaFunction{LIFTSUB}\<%
\end{code}

We can consider replacement to be a special case of substitution.  That is,
we can identify every replacement $\rho : U \rightarrow V$ with the substitution
that maps $x$ to $\rho(x)$.  
\begin{lemma}
Let $\rho$ be a replacement $U \rightarrow V$.
\begin{enumerate}
\item
The replacement $(\rho , K)$ and the substitution $(\rho , K)$ are equal.
\item
The replacement $\uparrow$ and the substitution $\uparrow$ are equal.
\item
The replacement $\rho^A$ and the substitution $\rho^A$ are equal.
\item
$ E \langle \rho \rangle \equiv E [ \rho ] $
\item
Hence $ E \langle \uparrow \rangle \equiv E [ \uparrow ]$.
\item
Substitution is a pre-family with lifting.
\end{enumerate}
\end{lemma}

\begin{code}%
\>\AgdaFunction{rep2sub} \AgdaSymbol{:} \AgdaSymbol{∀} \AgdaSymbol{\{}\AgdaBound{U}\AgdaSymbol{\}} \AgdaSymbol{\{}\AgdaBound{V}\AgdaSymbol{\}} \AgdaSymbol{→} \AgdaFunction{Rep} \AgdaBound{U} \AgdaBound{V} \AgdaSymbol{→} \AgdaFunction{Sub} \AgdaBound{U} \AgdaBound{V}\<%
\\
\>\AgdaFunction{rep2sub} \AgdaBound{ρ} \AgdaBound{K} \AgdaBound{x} \AgdaSymbol{=} \AgdaInductiveConstructor{var} \AgdaSymbol{(}\AgdaBound{ρ} \AgdaBound{K} \AgdaBound{x}\AgdaSymbol{)}\<%
\\
%
\\
\>\AgdaFunction{liftRep-is-liftSub} \AgdaSymbol{:} \AgdaSymbol{∀} \AgdaSymbol{\{}\AgdaBound{U}\AgdaSymbol{\}} \AgdaSymbol{\{}\AgdaBound{V}\AgdaSymbol{\}} \AgdaSymbol{\{}\AgdaBound{ρ} \AgdaSymbol{:} \AgdaFunction{Rep} \AgdaBound{U} \AgdaBound{V}\AgdaSymbol{\}} \AgdaSymbol{\{}\AgdaBound{K}\AgdaSymbol{\}} \AgdaSymbol{→} \<[51]%
\>[51]\<%
\\
\>[0]\AgdaIndent{2}{}\<[2]%
\>[2]\AgdaFunction{rep2sub} \AgdaSymbol{(}\AgdaFunction{liftRep} \AgdaBound{K} \AgdaBound{ρ}\AgdaSymbol{)} \AgdaFunction{∼} \AgdaFunction{liftSub} \AgdaBound{K} \AgdaSymbol{(}\AgdaFunction{rep2sub} \AgdaBound{ρ}\AgdaSymbol{)}\<%
\end{code}

\AgdaHide{
\begin{code}%
\>\AgdaFunction{liftRep-is-liftSub} \AgdaInductiveConstructor{x₀} \AgdaSymbol{=} \AgdaInductiveConstructor{refl}\<%
\\
\>\AgdaFunction{liftRep-is-liftSub} \AgdaSymbol{(}\AgdaInductiveConstructor{↑} \AgdaSymbol{\_)} \AgdaSymbol{=} \AgdaInductiveConstructor{refl}\<%
\end{code}
}

\begin{code}%
\>\AgdaFunction{up-is-up} \AgdaSymbol{:} \AgdaSymbol{∀} \AgdaSymbol{\{}\AgdaBound{V}\AgdaSymbol{\}} \AgdaSymbol{\{}\AgdaBound{K}\AgdaSymbol{\}} \AgdaSymbol{→} \AgdaFunction{rep2sub} \AgdaSymbol{(}\AgdaFunction{upRep} \AgdaSymbol{\{}\AgdaBound{V}\AgdaSymbol{\}} \AgdaSymbol{\{}\AgdaBound{K}\AgdaSymbol{\})} \AgdaFunction{∼} \AgdaFunction{upSub}\<%
\end{code}

\AgdaHide{
\begin{code}%
\>\AgdaFunction{up-is-up} \AgdaSymbol{\_} \AgdaSymbol{=} \AgdaInductiveConstructor{refl}\<%
\end{code}
}

\begin{code}%
\>\AgdaFunction{liftsOp-is-liftsOp} \AgdaSymbol{:} \AgdaSymbol{∀} \AgdaSymbol{\{}\AgdaBound{U}\AgdaSymbol{\}} \AgdaSymbol{\{}\AgdaBound{V}\AgdaSymbol{\}} \AgdaSymbol{\{}\AgdaBound{ρ} \AgdaSymbol{:} \AgdaFunction{Rep} \AgdaBound{U} \AgdaBound{V}\AgdaSymbol{\}} \AgdaSymbol{\{}\AgdaBound{A}\AgdaSymbol{\}} \AgdaSymbol{→} \<[51]%
\>[51]\<%
\\
\>[0]\AgdaIndent{2}{}\<[2]%
\>[2]\AgdaFunction{rep2sub} \AgdaSymbol{(}\AgdaFunction{liftsOpR} \<[21]%
\>[21]\AgdaBound{A} \AgdaBound{ρ}\AgdaSymbol{)} \AgdaFunction{∼} \AgdaFunction{liftsOp} \AgdaBound{A} \AgdaSymbol{(}\AgdaFunction{rep2sub} \AgdaBound{ρ}\AgdaSymbol{)}\<%
\end{code}

\AgdaHide{
\begin{code}%
\>\AgdaFunction{liftsOp-is-liftsOp} \AgdaSymbol{\{}\AgdaArgument{ρ} \AgdaSymbol{=} \AgdaBound{ρ}\AgdaSymbol{\}} \AgdaSymbol{\{}\AgdaArgument{A} \AgdaSymbol{=} \AgdaInductiveConstructor{[]}\AgdaSymbol{\}} \AgdaSymbol{=} \AgdaFunction{∼-refl} \AgdaSymbol{\{}\AgdaArgument{σ} \AgdaSymbol{=} \AgdaFunction{rep2sub} \AgdaBound{ρ}\AgdaSymbol{\}}\<%
\\
\>\AgdaFunction{liftsOp-is-liftsOp} \AgdaSymbol{\{}\AgdaBound{U}\AgdaSymbol{\}} \AgdaSymbol{\{}\AgdaBound{V}\AgdaSymbol{\}} \AgdaSymbol{\{}\AgdaBound{ρ}\AgdaSymbol{\}} \AgdaSymbol{\{}\AgdaBound{K} \AgdaInductiveConstructor{∷} \AgdaBound{A}\AgdaSymbol{\}} \AgdaSymbol{=} \AgdaKeyword{let} \AgdaKeyword{open} \AgdaModule{EqReasoning} \AgdaSymbol{(}\AgdaFunction{OP} \AgdaSymbol{\_} \AgdaSymbol{\_)} \AgdaKeyword{in} \<[74]%
\>[74]\<%
\\
\>[0]\AgdaIndent{2}{}\<[2]%
\>[2]\AgdaFunction{begin}\<%
\\
\>[2]\AgdaIndent{4}{}\<[4]%
\>[4]\AgdaFunction{rep2sub} \AgdaSymbol{(}\AgdaFunction{liftsOpR} \AgdaBound{A} \AgdaSymbol{(}\AgdaFunction{liftRep} \AgdaBound{K} \AgdaBound{ρ}\AgdaSymbol{))}\<%
\\
\>[0]\AgdaIndent{2}{}\<[2]%
\>[2]\AgdaFunction{≈⟨} \AgdaFunction{liftsOp-is-liftsOp} \AgdaSymbol{\{}\AgdaArgument{A} \AgdaSymbol{=} \AgdaBound{A}\AgdaSymbol{\}} \AgdaFunction{⟩}\<%
\\
\>[2]\AgdaIndent{4}{}\<[4]%
\>[4]\AgdaFunction{liftsOp} \AgdaBound{A} \AgdaSymbol{(}\AgdaFunction{rep2sub} \AgdaSymbol{(}\AgdaFunction{liftRep} \AgdaBound{K} \AgdaBound{ρ}\AgdaSymbol{))}\<%
\\
\>[0]\AgdaIndent{2}{}\<[2]%
\>[2]\AgdaFunction{≈⟨} \AgdaFunction{liftsOp-cong} \AgdaBound{A} \AgdaFunction{liftRep-is-liftSub} \AgdaFunction{⟩}\<%
\\
\>[2]\AgdaIndent{4}{}\<[4]%
\>[4]\AgdaFunction{liftsOp} \AgdaBound{A} \AgdaSymbol{(}\AgdaFunction{liftSub} \AgdaBound{K} \AgdaSymbol{(}\AgdaFunction{rep2sub} \AgdaBound{ρ}\AgdaSymbol{))}\<%
\\
\>[0]\AgdaIndent{2}{}\<[2]%
\>[2]\AgdaFunction{∎}\<%
\end{code}
}

\begin{code}%
\>\AgdaFunction{rep-is-sub} \AgdaSymbol{:} \AgdaSymbol{∀} \AgdaSymbol{\{}\AgdaBound{U}\AgdaSymbol{\}} \AgdaSymbol{\{}\AgdaBound{V}\AgdaSymbol{\}} \AgdaSymbol{\{}\AgdaBound{K}\AgdaSymbol{\}} \AgdaSymbol{\{}\AgdaBound{C}\AgdaSymbol{\}} \AgdaSymbol{(}\AgdaBound{E} \AgdaSymbol{:} \AgdaDatatype{Subexp} \AgdaBound{U} \AgdaBound{K} \AgdaBound{C}\AgdaSymbol{)} \AgdaSymbol{\{}\AgdaBound{ρ} \AgdaSymbol{:} \AgdaFunction{Rep} \AgdaBound{U} \AgdaBound{V}\AgdaSymbol{\}} \AgdaSymbol{→} \<[66]%
\>[66]\<%
\\
\>[0]\AgdaIndent{2}{}\<[2]%
\>[2]\AgdaBound{E} \AgdaFunction{〈} \AgdaBound{ρ} \AgdaFunction{〉} \AgdaDatatype{≡} \AgdaBound{E} \AgdaFunction{⟦} \AgdaFunction{rep2sub} \AgdaBound{ρ} \AgdaFunction{⟧}\<%
\end{code}

\AgdaHide{
\begin{code}%
\>\AgdaFunction{rep-is-sub} \AgdaSymbol{(}\AgdaInductiveConstructor{var} \AgdaSymbol{\_)} \AgdaSymbol{=} \AgdaInductiveConstructor{refl}\<%
\\
\>\AgdaFunction{rep-is-sub} \AgdaSymbol{(}\AgdaInductiveConstructor{app} \AgdaBound{c} \AgdaBound{E}\AgdaSymbol{)} \AgdaSymbol{=} \AgdaFunction{cong} \AgdaSymbol{(}\AgdaInductiveConstructor{app} \AgdaBound{c}\AgdaSymbol{)} \AgdaSymbol{(}\AgdaFunction{rep-is-sub} \AgdaBound{E}\AgdaSymbol{)}\<%
\\
\>\AgdaFunction{rep-is-sub} \AgdaInductiveConstructor{[]} \AgdaSymbol{=} \AgdaInductiveConstructor{refl}\<%
\\
\>\AgdaFunction{rep-is-sub} \AgdaSymbol{\{}\AgdaBound{U}\AgdaSymbol{\}} \AgdaSymbol{\{}\AgdaBound{V}\AgdaSymbol{\}} \AgdaSymbol{(}\AgdaInductiveConstructor{\_∷\_} \AgdaSymbol{\{}\AgdaArgument{A} \AgdaSymbol{=} \AgdaInductiveConstructor{SK} \AgdaBound{A} \AgdaSymbol{\_\}} \AgdaBound{E} \AgdaBound{F}\AgdaSymbol{)} \AgdaSymbol{\{}\AgdaBound{ρ}\AgdaSymbol{\}} \AgdaSymbol{=} \AgdaFunction{cong₂} \AgdaInductiveConstructor{\_∷\_} \<[58]%
\>[58]\<%
\\
\>[0]\AgdaIndent{2}{}\<[2]%
\>[2]\AgdaSymbol{(}\AgdaKeyword{let} \AgdaKeyword{open} \AgdaModule{≡-Reasoning} \AgdaSymbol{\{}\AgdaArgument{A} \AgdaSymbol{=} \AgdaDatatype{Subexp} \AgdaSymbol{(}\AgdaFunction{extend} \AgdaBound{V} \AgdaBound{A}\AgdaSymbol{)} \AgdaSymbol{\_} \AgdaSymbol{\_\}} \AgdaKeyword{in}\<%
\\
\>[0]\AgdaIndent{2}{}\<[2]%
\>[2]\AgdaFunction{begin} \<[8]%
\>[8]\<%
\\
\>[2]\AgdaIndent{4}{}\<[4]%
\>[4]\AgdaBound{E} \AgdaFunction{〈} \AgdaFunction{liftsOpR} \AgdaBound{A} \AgdaBound{ρ} \AgdaFunction{〉}\<%
\\
\>[0]\AgdaIndent{2}{}\<[2]%
\>[2]\AgdaFunction{≡⟨} \AgdaFunction{rep-is-sub} \AgdaBound{E} \AgdaFunction{⟩}\<%
\\
\>[2]\AgdaIndent{4}{}\<[4]%
\>[4]\AgdaBound{E} \AgdaFunction{⟦} \AgdaSymbol{(λ} \AgdaBound{K} \AgdaBound{x} \AgdaSymbol{→} \AgdaInductiveConstructor{var} \AgdaSymbol{(}\AgdaFunction{liftsOpR} \AgdaBound{A} \AgdaBound{ρ} \AgdaBound{K} \AgdaBound{x}\AgdaSymbol{))} \AgdaFunction{⟧} \<[43]%
\>[43]\<%
\\
\>[0]\AgdaIndent{2}{}\<[2]%
\>[2]\AgdaFunction{≡⟨} \AgdaFunction{ap-congl} \AgdaSymbol{(}\AgdaFunction{liftsOp-is-liftsOp} \AgdaSymbol{\{}\AgdaArgument{A} \AgdaSymbol{=} \AgdaBound{A}\AgdaSymbol{\})} \AgdaBound{E} \AgdaFunction{⟩}\<%
\\
\>[2]\AgdaIndent{4}{}\<[4]%
\>[4]\AgdaBound{E} \AgdaFunction{⟦} \AgdaFunction{liftsOp} \AgdaBound{A} \AgdaSymbol{(λ} \AgdaBound{K} \AgdaBound{x} \AgdaSymbol{→} \AgdaInductiveConstructor{var} \AgdaSymbol{(}\AgdaBound{ρ} \AgdaBound{K} \AgdaBound{x}\AgdaSymbol{))} \AgdaFunction{⟧}\<%
\\
\>[0]\AgdaIndent{2}{}\<[2]%
\>[2]\AgdaFunction{∎}\AgdaSymbol{)}\<%
\\
\>[0]\AgdaIndent{2}{}\<[2]%
\>[2]\AgdaSymbol{(}\AgdaFunction{rep-is-sub} \AgdaBound{F}\AgdaSymbol{)}\<%
\end{code}
}

\begin{code}%
\>\AgdaFunction{up-is-up'} \AgdaSymbol{:} \AgdaSymbol{∀} \AgdaSymbol{\{}\AgdaBound{V}\AgdaSymbol{\}} \AgdaSymbol{\{}\AgdaBound{C}\AgdaSymbol{\}} \AgdaSymbol{\{}\AgdaBound{K}\AgdaSymbol{\}} \AgdaSymbol{\{}\AgdaBound{L}\AgdaSymbol{\}} \AgdaSymbol{\{}\AgdaBound{E} \AgdaSymbol{:} \AgdaDatatype{Subexp} \AgdaBound{V} \AgdaBound{C} \AgdaBound{K}\AgdaSymbol{\}} \AgdaSymbol{→} \<[51]%
\>[51]\<%
\\
\>[0]\AgdaIndent{2}{}\<[2]%
\>[2]\AgdaBound{E} \AgdaFunction{〈} \AgdaFunction{upRep} \AgdaSymbol{\{}\AgdaArgument{K} \AgdaSymbol{=} \AgdaBound{L}\AgdaSymbol{\}} \AgdaFunction{〉} \AgdaDatatype{≡} \AgdaBound{E} \AgdaFunction{⟦} \AgdaFunction{upSub} \AgdaFunction{⟧}\<%
\end{code}

\AgdaHide{
\begin{code}%
\>\AgdaFunction{up-is-up'} \AgdaSymbol{\{}\AgdaArgument{E} \AgdaSymbol{=} \AgdaBound{E}\AgdaSymbol{\}} \AgdaSymbol{=} \AgdaFunction{rep-is-sub} \AgdaBound{E}\<%
\end{code}
}

\AgdaHide{
\begin{code}%
\>\AgdaKeyword{open} \AgdaKeyword{import} \AgdaModule{Grammar.Base}\<%
\\
%
\\
\>\AgdaKeyword{module} \AgdaModule{Grammar.Substitution.LiftFamily} \AgdaSymbol{(}\AgdaBound{G} \AgdaSymbol{:} \AgdaRecord{Grammar}\AgdaSymbol{)} \AgdaKeyword{where}\<%
\\
\>\AgdaKeyword{open} \AgdaKeyword{import} \AgdaModule{Prelims}\<%
\\
\>\AgdaKeyword{open} \AgdaModule{Grammar} \AgdaBound{G}\<%
\\
\>\AgdaKeyword{open} \AgdaKeyword{import} \AgdaModule{Grammar.OpFamily.LiftFamily} \AgdaBound{G}\<%
\\
\>\AgdaKeyword{open} \AgdaKeyword{import} \AgdaModule{Grammar.Substitution.PreOpFamily} \AgdaBound{G}\<%
\\
\>\AgdaKeyword{open} \AgdaKeyword{import} \AgdaModule{Grammar.Substitution.Lifting} \AgdaBound{G}\<%
\\
\>\AgdaKeyword{open} \AgdaKeyword{import} \AgdaModule{Grammar.Substitution.RepSub} \AgdaBound{G}\<%
\end{code}
}

It is now easy to show that substitution forms a pre-family with lifting.  If $\sigma : U \rightarrow V$ and $x \in U$ then $(\sigma , K)(\uparrow x) \equiv
\sigma(x) \langle \uparrow \rangle \equiv (\sigma , K)(x) [ \uparrow ]$.

\begin{code}%
\>\AgdaFunction{SubLF} \AgdaSymbol{:} \AgdaRecord{LiftFamily}\<%
\\
\>\AgdaFunction{SubLF} \AgdaSymbol{=} \AgdaKeyword{record} \AgdaSymbol{\{} \<[17]%
\>[17]\<%
\\
\>[0]\AgdaIndent{2}{}\<[2]%
\>[2]\AgdaField{preOpFamily} \AgdaSymbol{=} \AgdaFunction{pre-substitution} \AgdaSymbol{;} \<[35]%
\>[35]\<%
\\
\>[0]\AgdaIndent{2}{}\<[2]%
\>[2]\AgdaField{lifting} \AgdaSymbol{=} \AgdaFunction{LIFTSUB} \AgdaSymbol{;} \<[22]%
\>[22]\<%
\\
\>[0]\AgdaIndent{2}{}\<[2]%
\>[2]\AgdaField{isLiftFamily} \AgdaSymbol{=} \AgdaKeyword{record} \AgdaSymbol{\{} \<[26]%
\>[26]\<%
\\
\>[2]\AgdaIndent{4}{}\<[4]%
\>[4]\AgdaField{liftOp-x₀} \AgdaSymbol{=} \AgdaInductiveConstructor{refl} \AgdaSymbol{;} \<[23]%
\>[23]\<%
\\
\>[2]\AgdaIndent{4}{}\<[4]%
\>[4]\AgdaField{liftOp-↑} \AgdaSymbol{=} \AgdaSymbol{λ} \AgdaSymbol{\{}\AgdaBound{\_}\AgdaSymbol{\}} \AgdaSymbol{\{}\AgdaBound{\_}\AgdaSymbol{\}} \AgdaSymbol{\{}\AgdaBound{\_}\AgdaSymbol{\}} \AgdaSymbol{\{}\AgdaBound{\_}\AgdaSymbol{\}} \AgdaSymbol{\{}\AgdaBound{σ}\AgdaSymbol{\}} \AgdaBound{x} \AgdaSymbol{→} \AgdaFunction{rep-is-sub} \AgdaSymbol{(}\AgdaBound{σ} \AgdaSymbol{\_} \AgdaBound{x}\AgdaSymbol{)} \AgdaSymbol{\}\}}\<%
\end{code}

\AgdaHide{
\begin{code}%
\>\AgdaKeyword{open} \AgdaKeyword{import} \AgdaModule{Grammar.Base}\<%
\\
%
\\
\>\AgdaKeyword{module} \AgdaModule{Grammar.Substitution.OpFamily} \AgdaSymbol{(}\AgdaBound{G} \AgdaSymbol{:} \AgdaRecord{Grammar}\AgdaSymbol{)} \AgdaKeyword{where}\<%
\\
\>\AgdaKeyword{open} \AgdaKeyword{import} \AgdaModule{Prelims}\<%
\\
\>\AgdaKeyword{open} \AgdaModule{Grammar} \AgdaBound{G}\<%
\\
\>\AgdaKeyword{open} \AgdaKeyword{import} \AgdaModule{Grammar.OpFamily} \AgdaBound{G}\<%
\\
\>\AgdaKeyword{open} \AgdaKeyword{import} \AgdaModule{Grammar.Replacement} \AgdaBound{G}\<%
\\
\>\AgdaKeyword{open} \AgdaKeyword{import} \AgdaModule{Grammar.Substitution.PreOpFamily} \AgdaBound{G}\<%
\\
\>\AgdaKeyword{open} \AgdaKeyword{import} \AgdaModule{Grammar.Substitution.Lifting} \AgdaBound{G}\<%
\\
\>\AgdaKeyword{open} \AgdaKeyword{import} \AgdaModule{Grammar.Substitution.LiftFamily} \AgdaBound{G}\<%
\\
\>\AgdaKeyword{open} \AgdaKeyword{import} \AgdaModule{Grammar.Substitution.RepSub} \AgdaBound{G}\<%
\end{code}
}

We now define two compositions $\bullet_1 : \mathrm{replacement} ; \mathrm{substitution} \rightarrow \mathrm{substitution}$ and $\bullet_2 : \mathrm{substitution};\mathrm{replacement} \rightarrow \mathrm{substitution}$.

\begin{code}%
\>\AgdaKeyword{infixl} \AgdaNumber{60} \AgdaFixityOp{\_•RS\_}\<%
\\
\>\AgdaFunction{\_•RS\_} \AgdaSymbol{:} \AgdaSymbol{∀} \AgdaSymbol{\{}\AgdaBound{U}\AgdaSymbol{\}} \AgdaSymbol{\{}\AgdaBound{V}\AgdaSymbol{\}} \AgdaSymbol{\{}\AgdaBound{W}\AgdaSymbol{\}} \AgdaSymbol{→} \AgdaFunction{Rep} \AgdaBound{V} \AgdaBound{W} \AgdaSymbol{→} \AgdaFunction{Sub} \AgdaBound{U} \AgdaBound{V} \AgdaSymbol{→} \AgdaFunction{Sub} \AgdaBound{U} \AgdaBound{W}\<%
\\
\>\AgdaSymbol{(}\AgdaBound{ρ} \AgdaFunction{•RS} \AgdaBound{σ}\AgdaSymbol{)} \AgdaBound{K} \AgdaBound{x} \AgdaSymbol{=} \AgdaSymbol{(}\AgdaBound{σ} \AgdaBound{K} \AgdaBound{x}\AgdaSymbol{)} \AgdaFunction{〈} \AgdaBound{ρ} \AgdaFunction{〉}\<%
\\
%
\\
\>\AgdaFunction{Sub↑-compRS} \AgdaSymbol{:} \AgdaSymbol{∀} \AgdaSymbol{\{}\AgdaBound{U}\AgdaSymbol{\}} \AgdaSymbol{\{}\AgdaBound{V}\AgdaSymbol{\}} \AgdaSymbol{\{}\AgdaBound{W}\AgdaSymbol{\}} \AgdaSymbol{\{}\AgdaBound{K}\AgdaSymbol{\}} \AgdaSymbol{\{}\AgdaBound{ρ} \AgdaSymbol{:} \AgdaFunction{Rep} \AgdaBound{V} \AgdaBound{W}\AgdaSymbol{\}} \AgdaSymbol{\{}\AgdaBound{σ} \AgdaSymbol{:} \AgdaFunction{Sub} \AgdaBound{U} \AgdaBound{V}\AgdaSymbol{\}} \AgdaSymbol{→} \AgdaFunction{Sub↑} \AgdaBound{K} \AgdaSymbol{(}\AgdaBound{ρ} \AgdaFunction{•RS} \AgdaBound{σ}\AgdaSymbol{)} \AgdaFunction{∼} \AgdaFunction{Rep↑} \AgdaBound{K} \AgdaBound{ρ} \AgdaFunction{•RS} \AgdaFunction{Sub↑} \AgdaBound{K} \AgdaBound{σ}\<%
\end{code}

\AgdaHide{
\begin{code}%
\>\AgdaFunction{Sub↑-compRS} \AgdaSymbol{\{}\AgdaArgument{K} \AgdaSymbol{=} \AgdaBound{K}\AgdaSymbol{\}} \AgdaInductiveConstructor{x₀} \AgdaSymbol{=} \AgdaInductiveConstructor{refl}\<%
\\
\>\AgdaFunction{Sub↑-compRS} \AgdaSymbol{\{}\AgdaBound{U}\AgdaSymbol{\}} \AgdaSymbol{\{}\AgdaBound{V}\AgdaSymbol{\}} \AgdaSymbol{\{}\AgdaBound{W}\AgdaSymbol{\}} \AgdaSymbol{\{}\AgdaBound{K}\AgdaSymbol{\}} \AgdaSymbol{\{}\AgdaBound{ρ}\AgdaSymbol{\}} \AgdaSymbol{\{}\AgdaBound{σ}\AgdaSymbol{\}} \AgdaSymbol{\{}\AgdaBound{L}\AgdaSymbol{\}} \AgdaSymbol{(}\AgdaInductiveConstructor{↑} \AgdaBound{x}\AgdaSymbol{)} \AgdaSymbol{=} \AgdaKeyword{let} \AgdaKeyword{open} \AgdaModule{≡-Reasoning} \AgdaSymbol{\{}\AgdaArgument{A} \AgdaSymbol{=} \AgdaFunction{Expression} \AgdaSymbol{(}\AgdaBound{W} \AgdaInductiveConstructor{,} \AgdaBound{K}\AgdaSymbol{)} \AgdaSymbol{(}\AgdaInductiveConstructor{varKind} \AgdaBound{L}\AgdaSymbol{)\}} \AgdaKeyword{in} \<[109]%
\>[109]\<%
\\
\>[0]\AgdaIndent{2}{}\<[2]%
\>[2]\AgdaFunction{begin} \<[8]%
\>[8]\<%
\\
\>[2]\AgdaIndent{4}{}\<[4]%
\>[4]\AgdaSymbol{(}\AgdaBound{σ} \AgdaBound{L} \AgdaBound{x}\AgdaSymbol{)} \AgdaFunction{〈} \AgdaBound{ρ} \AgdaFunction{〉} \AgdaFunction{〈} \AgdaFunction{upRep} \AgdaFunction{〉}\<%
\\
\>[0]\AgdaIndent{2}{}\<[2]%
\>[2]\AgdaFunction{≡⟨⟨} \AgdaFunction{rep-comp} \AgdaSymbol{(}\AgdaBound{σ} \AgdaBound{L} \AgdaBound{x}\AgdaSymbol{)} \AgdaFunction{⟩⟩}\<%
\\
\>[2]\AgdaIndent{4}{}\<[4]%
\>[4]\AgdaSymbol{(}\AgdaBound{σ} \AgdaBound{L} \AgdaBound{x}\AgdaSymbol{)} \AgdaFunction{〈} \AgdaFunction{upRep} \AgdaFunction{•R} \AgdaBound{ρ} \AgdaFunction{〉}\<%
\\
\>[0]\AgdaIndent{2}{}\<[2]%
\>[2]\AgdaFunction{≡⟨⟩}\<%
\\
\>[2]\AgdaIndent{4}{}\<[4]%
\>[4]\AgdaSymbol{(}\AgdaBound{σ} \AgdaBound{L} \AgdaBound{x}\AgdaSymbol{)} \AgdaFunction{〈} \AgdaFunction{Rep↑} \AgdaBound{K} \AgdaBound{ρ} \AgdaFunction{•R} \AgdaFunction{upRep} \AgdaFunction{〉}\<%
\\
\>[0]\AgdaIndent{2}{}\<[2]%
\>[2]\AgdaFunction{≡⟨} \AgdaFunction{rep-comp} \AgdaSymbol{(}\AgdaBound{σ} \AgdaBound{L} \AgdaBound{x}\AgdaSymbol{)} \AgdaFunction{⟩}\<%
\\
\>[2]\AgdaIndent{4}{}\<[4]%
\>[4]\AgdaSymbol{(}\AgdaBound{σ} \AgdaBound{L} \AgdaBound{x}\AgdaSymbol{)} \AgdaFunction{〈} \AgdaFunction{upRep} \AgdaFunction{〉} \AgdaFunction{〈} \AgdaFunction{Rep↑} \AgdaBound{K} \AgdaBound{ρ} \AgdaFunction{〉}\<%
\\
\>[0]\AgdaIndent{2}{}\<[2]%
\>[2]\AgdaFunction{∎}\<%
\end{code}
}

\begin{code}%
\>\AgdaFunction{COMPRS} \AgdaSymbol{:} \AgdaRecord{Composition} \AgdaFunction{proto-replacement} \AgdaFunction{proto-substitution} \AgdaFunction{proto-substitution}\<%
\\
\>\AgdaFunction{COMPRS} \AgdaSymbol{=} \AgdaKeyword{record} \AgdaSymbol{\{} \<[18]%
\>[18]\<%
\\
\>[0]\AgdaIndent{2}{}\<[2]%
\>[2]\AgdaField{circ} \AgdaSymbol{=} \AgdaFunction{\_•RS\_} \AgdaSymbol{;} \<[17]%
\>[17]\<%
\\
\>[0]\AgdaIndent{2}{}\<[2]%
\>[2]\AgdaField{liftOp-circ} \AgdaSymbol{=} \AgdaFunction{Sub↑-compRS} \AgdaSymbol{;} \<[30]%
\>[30]\<%
\\
\>[0]\AgdaIndent{2}{}\<[2]%
\>[2]\AgdaField{apV-circ} \AgdaSymbol{=} \AgdaInductiveConstructor{refl} \AgdaSymbol{\}}\<%
\\
%
\\
\>\AgdaFunction{sub-compRS} \AgdaSymbol{:} \AgdaSymbol{∀} \AgdaSymbol{\{}\AgdaBound{U}\AgdaSymbol{\}} \AgdaSymbol{\{}\AgdaBound{V}\AgdaSymbol{\}} \AgdaSymbol{\{}\AgdaBound{W}\AgdaSymbol{\}} \AgdaSymbol{\{}\AgdaBound{C}\AgdaSymbol{\}} \AgdaSymbol{\{}\AgdaBound{K}\AgdaSymbol{\}} \<[35]%
\>[35]\<%
\\
\>[0]\AgdaIndent{2}{}\<[2]%
\>[2]\AgdaSymbol{(}\AgdaBound{E} \AgdaSymbol{:} \AgdaDatatype{Subexpression} \AgdaBound{U} \AgdaBound{C} \AgdaBound{K}\AgdaSymbol{)} \AgdaSymbol{\{}\AgdaBound{ρ} \AgdaSymbol{:} \AgdaFunction{Rep} \AgdaBound{V} \AgdaBound{W}\AgdaSymbol{\}} \AgdaSymbol{\{}\AgdaBound{σ} \AgdaSymbol{:} \AgdaFunction{Sub} \AgdaBound{U} \AgdaBound{V}\AgdaSymbol{\}} \AgdaSymbol{→}\<%
\\
\>[0]\AgdaIndent{2}{}\<[2]%
\>[2]\AgdaBound{E} \AgdaFunction{⟦} \AgdaBound{ρ} \AgdaFunction{•RS} \AgdaBound{σ} \AgdaFunction{⟧} \AgdaDatatype{≡} \AgdaBound{E} \AgdaFunction{⟦} \AgdaBound{σ} \AgdaFunction{⟧} \AgdaFunction{〈} \AgdaBound{ρ} \AgdaFunction{〉}\<%
\\
\>\AgdaFunction{sub-compRS} \AgdaBound{E} \AgdaSymbol{=} \AgdaFunction{Composition.ap-circ} \AgdaFunction{COMPRS} \AgdaBound{E}\<%
\\
%
\\
\>\AgdaKeyword{infixl} \AgdaNumber{60} \AgdaFixityOp{\_•SR\_}\<%
\\
\>\AgdaFunction{\_•SR\_} \AgdaSymbol{:} \AgdaSymbol{∀} \AgdaSymbol{\{}\AgdaBound{U}\AgdaSymbol{\}} \AgdaSymbol{\{}\AgdaBound{V}\AgdaSymbol{\}} \AgdaSymbol{\{}\AgdaBound{W}\AgdaSymbol{\}} \AgdaSymbol{→} \AgdaFunction{Sub} \AgdaBound{V} \AgdaBound{W} \AgdaSymbol{→} \AgdaFunction{Rep} \AgdaBound{U} \AgdaBound{V} \AgdaSymbol{→} \AgdaFunction{Sub} \AgdaBound{U} \AgdaBound{W}\<%
\\
\>\AgdaSymbol{(}\AgdaBound{σ} \AgdaFunction{•SR} \AgdaBound{ρ}\AgdaSymbol{)} \AgdaBound{K} \AgdaBound{x} \AgdaSymbol{=} \AgdaBound{σ} \AgdaBound{K} \AgdaSymbol{(}\AgdaBound{ρ} \AgdaBound{K} \AgdaBound{x}\AgdaSymbol{)}\<%
\\
%
\\
\>\AgdaFunction{Sub↑-compSR} \AgdaSymbol{:} \AgdaSymbol{∀} \AgdaSymbol{\{}\AgdaBound{U}\AgdaSymbol{\}} \AgdaSymbol{\{}\AgdaBound{V}\AgdaSymbol{\}} \AgdaSymbol{\{}\AgdaBound{W}\AgdaSymbol{\}} \AgdaSymbol{\{}\AgdaBound{K}\AgdaSymbol{\}} \AgdaSymbol{\{}\AgdaBound{σ} \AgdaSymbol{:} \AgdaFunction{Sub} \AgdaBound{V} \AgdaBound{W}\AgdaSymbol{\}} \AgdaSymbol{\{}\AgdaBound{ρ} \AgdaSymbol{:} \AgdaFunction{Rep} \AgdaBound{U} \AgdaBound{V}\AgdaSymbol{\}} \AgdaSymbol{→} \<[62]%
\>[62]\<%
\\
\>[0]\AgdaIndent{2}{}\<[2]%
\>[2]\AgdaFunction{Sub↑} \AgdaBound{K} \AgdaSymbol{(}\AgdaBound{σ} \AgdaFunction{•SR} \AgdaBound{ρ}\AgdaSymbol{)} \AgdaFunction{∼} \AgdaFunction{Sub↑} \AgdaBound{K} \AgdaBound{σ} \AgdaFunction{•SR} \AgdaFunction{Rep↑} \AgdaBound{K} \AgdaBound{ρ}\<%
\end{code}

\AgdaHide{
\begin{code}%
\>\AgdaFunction{Sub↑-compSR} \AgdaSymbol{\{}\AgdaArgument{K} \AgdaSymbol{=} \AgdaBound{K}\AgdaSymbol{\}} \AgdaInductiveConstructor{x₀} \AgdaSymbol{=} \AgdaInductiveConstructor{refl}\<%
\\
\>\AgdaFunction{Sub↑-compSR} \AgdaSymbol{(}\AgdaInductiveConstructor{↑} \AgdaBound{x}\AgdaSymbol{)} \AgdaSymbol{=} \AgdaInductiveConstructor{refl}\<%
\end{code}
}

\begin{code}%
\>\AgdaFunction{COMPSR} \AgdaSymbol{:} \AgdaRecord{Composition} \AgdaFunction{proto-substitution} \AgdaFunction{proto-replacement} \AgdaFunction{proto-substitution}\<%
\\
\>\AgdaFunction{COMPSR} \AgdaSymbol{=} \AgdaKeyword{record} \AgdaSymbol{\{} \<[18]%
\>[18]\<%
\\
\>[0]\AgdaIndent{2}{}\<[2]%
\>[2]\AgdaField{circ} \AgdaSymbol{=} \AgdaFunction{\_•SR\_} \AgdaSymbol{;} \<[17]%
\>[17]\<%
\\
\>[0]\AgdaIndent{2}{}\<[2]%
\>[2]\AgdaField{liftOp-circ} \AgdaSymbol{=} \AgdaFunction{Sub↑-compSR} \AgdaSymbol{;} \<[30]%
\>[30]\<%
\\
\>[0]\AgdaIndent{2}{}\<[2]%
\>[2]\AgdaField{apV-circ} \AgdaSymbol{=} \AgdaInductiveConstructor{refl} \AgdaSymbol{\}}\<%
\\
%
\\
\>\AgdaFunction{sub-compSR} \AgdaSymbol{:} \AgdaSymbol{∀} \AgdaSymbol{\{}\AgdaBound{U}\AgdaSymbol{\}} \AgdaSymbol{\{}\AgdaBound{V}\AgdaSymbol{\}} \AgdaSymbol{\{}\AgdaBound{W}\AgdaSymbol{\}} \AgdaSymbol{\{}\AgdaBound{C}\AgdaSymbol{\}} \AgdaSymbol{\{}\AgdaBound{K}\AgdaSymbol{\}} \<[35]%
\>[35]\<%
\\
\>[0]\AgdaIndent{2}{}\<[2]%
\>[2]\AgdaSymbol{(}\AgdaBound{E} \AgdaSymbol{:} \AgdaDatatype{Subexpression} \AgdaBound{U} \AgdaBound{C} \AgdaBound{K}\AgdaSymbol{)} \AgdaSymbol{\{}\AgdaBound{σ} \AgdaSymbol{:} \AgdaFunction{Sub} \AgdaBound{V} \AgdaBound{W}\AgdaSymbol{\}} \AgdaSymbol{\{}\AgdaBound{ρ} \AgdaSymbol{:} \AgdaFunction{Rep} \AgdaBound{U} \AgdaBound{V}\AgdaSymbol{\}} \AgdaSymbol{→} \<[58]%
\>[58]\<%
\\
\>[0]\AgdaIndent{2}{}\<[2]%
\>[2]\AgdaBound{E} \AgdaFunction{⟦} \AgdaBound{σ} \AgdaFunction{•SR} \AgdaBound{ρ} \AgdaFunction{⟧} \AgdaDatatype{≡} \AgdaBound{E} \AgdaFunction{〈} \AgdaBound{ρ} \AgdaFunction{〉} \AgdaFunction{⟦} \AgdaBound{σ} \AgdaFunction{⟧}\<%
\end{code}

\AgdaHide{
\begin{code}%
\>\AgdaFunction{sub-compSR} \AgdaBound{E} \AgdaSymbol{=} \AgdaFunction{Composition.ap-circ} \AgdaFunction{COMPSR} \AgdaBound{E}\<%
\end{code}
}

\begin{code}%
\>\AgdaFunction{Sub↑-upRep} \AgdaSymbol{:} \AgdaSymbol{∀} \AgdaSymbol{\{}\AgdaBound{U}\AgdaSymbol{\}} \AgdaSymbol{\{}\AgdaBound{V}\AgdaSymbol{\}} \AgdaSymbol{\{}\AgdaBound{C}\AgdaSymbol{\}} \AgdaSymbol{\{}\AgdaBound{K}\AgdaSymbol{\}} \AgdaSymbol{\{}\AgdaBound{L}\AgdaSymbol{\}} \AgdaSymbol{(}\AgdaBound{E} \AgdaSymbol{:} \AgdaDatatype{Subexpression} \AgdaBound{U} \AgdaBound{C} \AgdaBound{K}\AgdaSymbol{)} \AgdaSymbol{\{}\AgdaBound{σ} \AgdaSymbol{:} \AgdaFunction{Sub} \AgdaBound{U} \AgdaBound{V}\AgdaSymbol{\}} \AgdaSymbol{→}\<%
\\
\>[0]\AgdaIndent{2}{}\<[2]%
\>[2]\AgdaBound{E} \AgdaFunction{〈} \AgdaFunction{upRep} \AgdaFunction{〉} \AgdaFunction{⟦} \AgdaFunction{Sub↑} \AgdaBound{L} \AgdaBound{σ} \AgdaFunction{⟧} \AgdaDatatype{≡} \AgdaBound{E} \AgdaFunction{⟦} \AgdaBound{σ} \AgdaFunction{⟧} \AgdaFunction{〈} \AgdaFunction{upRep} \AgdaFunction{〉}\<%
\end{code}

\AgdaHide{
\begin{code}%
\>\AgdaFunction{Sub↑-upRep} \AgdaBound{E} \AgdaSymbol{=} \AgdaFunction{liftOp-up-mixed} \AgdaFunction{COMPSR} \AgdaFunction{COMPRS} \AgdaSymbol{(λ} \AgdaSymbol{\{}\AgdaBound{\_}\AgdaSymbol{\}} \AgdaSymbol{\{}\AgdaBound{\_}\AgdaSymbol{\}} \AgdaSymbol{\{}\AgdaBound{\_}\AgdaSymbol{\}} \AgdaSymbol{\{}\AgdaBound{\_}\AgdaSymbol{\}} \AgdaSymbol{\{}\AgdaBound{E}\AgdaSymbol{\}} \AgdaSymbol{→} \AgdaFunction{sym} \AgdaSymbol{(}\AgdaFunction{up-is-up'} \AgdaSymbol{\{}\AgdaArgument{E} \AgdaSymbol{=} \AgdaBound{E}\AgdaSymbol{\}))} \AgdaSymbol{\{}\AgdaBound{E}\AgdaSymbol{\}}\<%
\end{code}
}

Composition is defined by $(\sigma \circ \rho)(x) \equiv \rho(x) [ \sigma ]$.

\begin{code}%
\>\AgdaKeyword{infixl} \AgdaNumber{60} \AgdaFixityOp{\_•\_}\<%
\\
\>\AgdaFunction{\_•\_} \AgdaSymbol{:} \AgdaSymbol{∀} \AgdaSymbol{\{}\AgdaBound{U}\AgdaSymbol{\}} \AgdaSymbol{\{}\AgdaBound{V}\AgdaSymbol{\}} \AgdaSymbol{\{}\AgdaBound{W}\AgdaSymbol{\}} \AgdaSymbol{→} \AgdaFunction{Sub} \AgdaBound{V} \AgdaBound{W} \AgdaSymbol{→} \AgdaFunction{Sub} \AgdaBound{U} \AgdaBound{V} \AgdaSymbol{→} \AgdaFunction{Sub} \AgdaBound{U} \AgdaBound{W}\<%
\\
\>\AgdaSymbol{(}\AgdaBound{σ} \AgdaFunction{•} \AgdaBound{ρ}\AgdaSymbol{)} \AgdaBound{K} \AgdaBound{x} \AgdaSymbol{=} \AgdaBound{ρ} \AgdaBound{K} \AgdaBound{x} \AgdaFunction{⟦} \AgdaBound{σ} \AgdaFunction{⟧}\<%
\end{code}

Using the fact that $\bullet_1$ and $\bullet_2$ are compositions, we are
able to prove that this is a composition $\mathrm{substitution} ; \mathrm{substitution} \rightarrow \mathrm{substitution}$, and hence that substitution is a family of operations.

\begin{code}%
\>\AgdaFunction{Sub↑-comp} \AgdaSymbol{:} \AgdaSymbol{∀} \AgdaSymbol{\{}\AgdaBound{U}\AgdaSymbol{\}} \AgdaSymbol{\{}\AgdaBound{V}\AgdaSymbol{\}} \AgdaSymbol{\{}\AgdaBound{W}\AgdaSymbol{\}} \AgdaSymbol{\{}\AgdaBound{ρ} \AgdaSymbol{:} \AgdaFunction{Sub} \AgdaBound{U} \AgdaBound{V}\AgdaSymbol{\}} \AgdaSymbol{\{}\AgdaBound{σ} \AgdaSymbol{:} \AgdaFunction{Sub} \AgdaBound{V} \AgdaBound{W}\AgdaSymbol{\}} \AgdaSymbol{\{}\AgdaBound{K}\AgdaSymbol{\}} \AgdaSymbol{→} \<[60]%
\>[60]\<%
\\
\>[0]\AgdaIndent{2}{}\<[2]%
\>[2]\AgdaFunction{Sub↑} \AgdaBound{K} \AgdaSymbol{(}\AgdaBound{σ} \AgdaFunction{•} \AgdaBound{ρ}\AgdaSymbol{)} \AgdaFunction{∼} \AgdaFunction{Sub↑} \AgdaBound{K} \AgdaBound{σ} \AgdaFunction{•} \AgdaFunction{Sub↑} \AgdaBound{K} \AgdaBound{ρ}\<%
\end{code}

\AgdaHide{
\begin{code}%
\>\AgdaFunction{Sub↑-comp} \AgdaInductiveConstructor{x₀} \AgdaSymbol{=} \AgdaInductiveConstructor{refl}\<%
\\
\>\AgdaFunction{Sub↑-comp} \AgdaSymbol{\{}\AgdaArgument{W} \AgdaSymbol{=} \AgdaBound{W}\AgdaSymbol{\}} \AgdaSymbol{\{}\AgdaArgument{ρ} \AgdaSymbol{=} \AgdaBound{ρ}\AgdaSymbol{\}} \AgdaSymbol{\{}\AgdaArgument{σ} \AgdaSymbol{=} \AgdaBound{σ}\AgdaSymbol{\}} \AgdaSymbol{\{}\AgdaArgument{K} \AgdaSymbol{=} \AgdaBound{K}\AgdaSymbol{\}} \AgdaSymbol{\{}\AgdaBound{L}\AgdaSymbol{\}} \AgdaSymbol{(}\AgdaInductiveConstructor{↑} \AgdaBound{x}\AgdaSymbol{)} \AgdaSymbol{=} \AgdaFunction{sym} \AgdaSymbol{(}\AgdaFunction{Sub↑-upRep} \AgdaSymbol{(}\AgdaBound{ρ} \AgdaBound{L} \AgdaBound{x}\AgdaSymbol{))}\<%
\\
%
\\
\>\AgdaFunction{Sub↑-upRep₂} \AgdaSymbol{:} \AgdaSymbol{∀} \AgdaSymbol{\{}\AgdaBound{U}\AgdaSymbol{\}} \AgdaSymbol{\{}\AgdaBound{V}\AgdaSymbol{\}} \AgdaSymbol{\{}\AgdaBound{C}\AgdaSymbol{\}} \AgdaSymbol{\{}\AgdaBound{K}\AgdaSymbol{\}} \AgdaSymbol{\{}\AgdaBound{L}\AgdaSymbol{\}} \AgdaSymbol{\{}\AgdaBound{M}\AgdaSymbol{\}} \AgdaSymbol{(}\AgdaBound{E} \AgdaSymbol{:} \AgdaDatatype{Subexpression} \AgdaBound{U} \AgdaBound{C} \AgdaBound{M}\AgdaSymbol{)} \AgdaSymbol{\{}\AgdaBound{σ} \AgdaSymbol{:} \AgdaFunction{Sub} \AgdaBound{U} \AgdaBound{V}\AgdaSymbol{\}} \AgdaSymbol{→} \AgdaBound{E} \AgdaFunction{⇑} \AgdaFunction{⇑} \AgdaFunction{⟦} \AgdaFunction{Sub↑} \AgdaBound{K} \AgdaSymbol{(}\AgdaFunction{Sub↑} \AgdaBound{L} \AgdaBound{σ}\AgdaSymbol{)} \AgdaFunction{⟧} \AgdaDatatype{≡} \AgdaBound{E} \AgdaFunction{⟦} \AgdaBound{σ} \AgdaFunction{⟧} \AgdaFunction{⇑} \AgdaFunction{⇑}\<%
\\
\>\AgdaFunction{Sub↑-upRep₂} \AgdaSymbol{\{}\AgdaBound{U}\AgdaSymbol{\}} \AgdaSymbol{\{}\AgdaBound{V}\AgdaSymbol{\}} \AgdaSymbol{\{}\AgdaBound{C}\AgdaSymbol{\}} \AgdaSymbol{\{}\AgdaBound{K}\AgdaSymbol{\}} \AgdaSymbol{\{}\AgdaBound{L}\AgdaSymbol{\}} \AgdaSymbol{\{}\AgdaBound{M}\AgdaSymbol{\}} \AgdaBound{E} \AgdaSymbol{\{}\AgdaBound{σ}\AgdaSymbol{\}} \AgdaSymbol{=} \AgdaKeyword{let} \AgdaKeyword{open} \AgdaModule{≡-Reasoning} \AgdaKeyword{in} \<[68]%
\>[68]\<%
\\
\>[0]\AgdaIndent{2}{}\<[2]%
\>[2]\AgdaFunction{begin}\<%
\\
\>[2]\AgdaIndent{4}{}\<[4]%
\>[4]\AgdaBound{E} \AgdaFunction{⇑} \AgdaFunction{⇑} \AgdaFunction{⟦} \AgdaFunction{Sub↑} \AgdaBound{K} \AgdaSymbol{(}\AgdaFunction{Sub↑} \AgdaBound{L} \AgdaBound{σ}\AgdaSymbol{)} \AgdaFunction{⟧}\<%
\\
\>[0]\AgdaIndent{2}{}\<[2]%
\>[2]\AgdaFunction{≡⟨} \AgdaFunction{Sub↑-upRep} \AgdaSymbol{(}\AgdaBound{E} \AgdaFunction{⇑}\AgdaSymbol{)} \AgdaFunction{⟩}\<%
\\
\>[2]\AgdaIndent{4}{}\<[4]%
\>[4]\AgdaBound{E} \AgdaFunction{⇑} \AgdaFunction{⟦} \AgdaFunction{Sub↑} \AgdaBound{L} \AgdaBound{σ} \AgdaFunction{⟧} \AgdaFunction{⇑}\<%
\\
\>[0]\AgdaIndent{2}{}\<[2]%
\>[2]\AgdaFunction{≡⟨} \AgdaFunction{rep-congl} \AgdaSymbol{(}\AgdaFunction{Sub↑-upRep} \AgdaBound{E}\AgdaSymbol{)} \AgdaFunction{⟩}\<%
\\
\>[2]\AgdaIndent{4}{}\<[4]%
\>[4]\AgdaBound{E} \AgdaFunction{⟦} \AgdaBound{σ} \AgdaFunction{⟧} \AgdaFunction{⇑} \AgdaFunction{⇑}\<%
\\
\>[0]\AgdaIndent{2}{}\<[2]%
\>[2]\AgdaFunction{∎}\<%
\\
%
\\
\>\AgdaFunction{Sub↑-upRep₃} \AgdaSymbol{:} \AgdaSymbol{∀} \AgdaSymbol{\{}\AgdaBound{U}\AgdaSymbol{\}} \AgdaSymbol{\{}\AgdaBound{V}\AgdaSymbol{\}} \AgdaSymbol{\{}\AgdaBound{C}\AgdaSymbol{\}} \AgdaSymbol{\{}\AgdaBound{K}\AgdaSymbol{\}} \AgdaSymbol{\{}\AgdaBound{L}\AgdaSymbol{\}} \AgdaSymbol{\{}\AgdaBound{M}\AgdaSymbol{\}} \AgdaSymbol{\{}\AgdaBound{N}\AgdaSymbol{\}} \AgdaSymbol{(}\AgdaBound{E} \AgdaSymbol{:} \AgdaDatatype{Subexpression} \AgdaBound{U} \AgdaBound{C} \AgdaBound{N}\AgdaSymbol{)} \AgdaSymbol{\{}\AgdaBound{σ} \AgdaSymbol{:} \AgdaFunction{Sub} \AgdaBound{U} \AgdaBound{V}\AgdaSymbol{\}} \AgdaSymbol{→} \AgdaBound{E} \AgdaFunction{⇑} \AgdaFunction{⇑} \AgdaFunction{⇑} \AgdaFunction{⟦} \AgdaFunction{Sub↑} \AgdaBound{K} \AgdaSymbol{(}\AgdaFunction{Sub↑} \AgdaBound{L} \AgdaSymbol{(}\AgdaFunction{Sub↑} \AgdaBound{M} \AgdaBound{σ}\AgdaSymbol{))} \AgdaFunction{⟧} \AgdaDatatype{≡} \AgdaBound{E} \AgdaFunction{⟦} \AgdaBound{σ} \AgdaFunction{⟧} \AgdaFunction{⇑} \AgdaFunction{⇑} \AgdaFunction{⇑}\<%
\\
\>\AgdaFunction{Sub↑-upRep₃} \AgdaSymbol{\{}\AgdaBound{U}\AgdaSymbol{\}} \AgdaSymbol{\{}\AgdaBound{V}\AgdaSymbol{\}} \AgdaSymbol{\{}\AgdaBound{C}\AgdaSymbol{\}} \AgdaSymbol{\{}\AgdaBound{K}\AgdaSymbol{\}} \AgdaSymbol{\{}\AgdaBound{L}\AgdaSymbol{\}} \AgdaSymbol{\{}\AgdaBound{M}\AgdaSymbol{\}} \AgdaSymbol{\{}\AgdaBound{N}\AgdaSymbol{\}} \AgdaBound{E} \AgdaSymbol{\{}\AgdaBound{σ}\AgdaSymbol{\}} \AgdaSymbol{=} \AgdaKeyword{let} \AgdaKeyword{open} \AgdaModule{≡-Reasoning} \AgdaKeyword{in} \<[72]%
\>[72]\<%
\\
\>[0]\AgdaIndent{2}{}\<[2]%
\>[2]\AgdaFunction{begin}\<%
\\
\>[2]\AgdaIndent{4}{}\<[4]%
\>[4]\AgdaBound{E} \AgdaFunction{⇑} \AgdaFunction{⇑} \AgdaFunction{⇑} \AgdaFunction{⟦} \AgdaFunction{Sub↑} \AgdaBound{K} \AgdaSymbol{(}\AgdaFunction{Sub↑} \AgdaBound{L} \AgdaSymbol{(}\AgdaFunction{Sub↑} \AgdaBound{M} \AgdaBound{σ}\AgdaSymbol{))} \AgdaFunction{⟧}\<%
\\
\>[0]\AgdaIndent{2}{}\<[2]%
\>[2]\AgdaFunction{≡⟨} \AgdaFunction{Sub↑-upRep₂} \AgdaSymbol{(}\AgdaBound{E} \AgdaFunction{⇑}\AgdaSymbol{)} \AgdaFunction{⟩}\<%
\\
\>[2]\AgdaIndent{4}{}\<[4]%
\>[4]\AgdaBound{E} \AgdaFunction{⇑} \AgdaFunction{⟦} \AgdaFunction{Sub↑} \AgdaBound{M} \AgdaBound{σ} \AgdaFunction{⟧} \AgdaFunction{⇑} \AgdaFunction{⇑}\<%
\\
\>[0]\AgdaIndent{2}{}\<[2]%
\>[2]\AgdaFunction{≡⟨} \AgdaFunction{rep-congl} \AgdaSymbol{(}\AgdaFunction{rep-congl} \AgdaSymbol{(}\AgdaFunction{Sub↑-upRep} \AgdaBound{E}\AgdaSymbol{))} \AgdaFunction{⟩}\<%
\\
\>[2]\AgdaIndent{4}{}\<[4]%
\>[4]\AgdaBound{E} \AgdaFunction{⟦} \AgdaBound{σ} \AgdaFunction{⟧} \AgdaFunction{⇑} \AgdaFunction{⇑} \AgdaFunction{⇑}\<%
\\
\>[0]\AgdaIndent{2}{}\<[2]%
\>[2]\AgdaFunction{∎}\<%
\\
%
\\
\>\AgdaFunction{Rep↑-Sub↑-upRep₃} \AgdaSymbol{:} \AgdaSymbol{∀} \AgdaSymbol{\{}\AgdaBound{U}\AgdaSymbol{\}} \AgdaSymbol{\{}\AgdaBound{V}\AgdaSymbol{\}} \AgdaSymbol{\{}\AgdaBound{W}\AgdaSymbol{\}} \AgdaSymbol{\{}\AgdaBound{K1}\AgdaSymbol{\}} \AgdaSymbol{\{}\AgdaBound{K2}\AgdaSymbol{\}} \AgdaSymbol{\{}\AgdaBound{K3}\AgdaSymbol{\}} \AgdaSymbol{\{}\AgdaBound{C}\AgdaSymbol{\}} \AgdaSymbol{\{}\AgdaBound{K4}\AgdaSymbol{\}} \<[57]%
\>[57]\<%
\\
\>[2]\AgdaIndent{19}{}\<[19]%
\>[19]\AgdaSymbol{(}\AgdaBound{M} \AgdaSymbol{:} \AgdaDatatype{Subexpression} \AgdaBound{U} \AgdaBound{C} \AgdaBound{K4}\AgdaSymbol{)}\<%
\\
\>[2]\AgdaIndent{19}{}\<[19]%
\>[19]\AgdaSymbol{(}\AgdaBound{σ} \AgdaSymbol{:} \AgdaFunction{Sub} \AgdaBound{U} \AgdaBound{V}\AgdaSymbol{)} \AgdaSymbol{(}\AgdaBound{ρ} \AgdaSymbol{:} \AgdaFunction{Rep} \AgdaBound{V} \AgdaBound{W}\AgdaSymbol{)} \AgdaSymbol{→}\<%
\\
\>[19]\AgdaIndent{20}{}\<[20]%
\>[20]\AgdaBound{M} \AgdaFunction{⇑} \AgdaFunction{⇑} \AgdaFunction{⇑} \AgdaFunction{⟦} \AgdaFunction{Sub↑} \AgdaBound{K1} \AgdaSymbol{(}\AgdaFunction{Sub↑} \AgdaBound{K2} \AgdaSymbol{(}\AgdaFunction{Sub↑} \AgdaBound{K3} \AgdaBound{σ}\AgdaSymbol{))} \AgdaFunction{⟧} \AgdaFunction{〈} \AgdaFunction{Rep↑} \AgdaBound{K1} \AgdaSymbol{(}\AgdaFunction{Rep↑} \AgdaBound{K2} \AgdaSymbol{(}\AgdaFunction{Rep↑} \AgdaBound{K3} \AgdaBound{ρ}\AgdaSymbol{))} \AgdaFunction{〉}\<%
\\
\>[19]\AgdaIndent{20}{}\<[20]%
\>[20]\AgdaDatatype{≡} \AgdaBound{M} \AgdaFunction{⟦} \AgdaBound{σ} \AgdaFunction{⟧} \AgdaFunction{〈} \AgdaBound{ρ} \AgdaFunction{〉} \AgdaFunction{⇑} \AgdaFunction{⇑} \AgdaFunction{⇑}\<%
\\
\>\AgdaFunction{Rep↑-Sub↑-upRep₃} \AgdaBound{M} \AgdaBound{σ} \AgdaBound{ρ} \AgdaSymbol{=} \AgdaFunction{trans} \AgdaSymbol{(}\AgdaFunction{rep-congl} \AgdaSymbol{(}\AgdaFunction{Sub↑-upRep₃} \AgdaBound{M} \AgdaSymbol{\{}\AgdaBound{σ}\AgdaSymbol{\}))} \AgdaSymbol{(}\AgdaFunction{Rep↑-upRep₃} \AgdaSymbol{(}\AgdaBound{M} \AgdaFunction{⟦} \AgdaBound{σ} \AgdaFunction{⟧}\AgdaSymbol{))}\<%
\\
%
\\
\>\AgdaFunction{assocRSSR} \AgdaSymbol{:} \AgdaSymbol{∀} \AgdaSymbol{\{}\AgdaBound{U}\AgdaSymbol{\}} \AgdaSymbol{\{}\AgdaBound{V}\AgdaSymbol{\}} \AgdaSymbol{\{}\AgdaBound{W}\AgdaSymbol{\}} \AgdaSymbol{\{}\AgdaBound{X}\AgdaSymbol{\}} \AgdaSymbol{\{}\AgdaBound{ρ} \AgdaSymbol{:} \AgdaFunction{Sub} \AgdaBound{W} \AgdaBound{X}\AgdaSymbol{\}} \AgdaSymbol{\{}\AgdaBound{σ} \AgdaSymbol{:} \AgdaFunction{Rep} \AgdaBound{V} \AgdaBound{W}\AgdaSymbol{\}} \AgdaSymbol{\{}\AgdaBound{τ} \AgdaSymbol{:} \AgdaFunction{Sub} \AgdaBound{U} \AgdaBound{V}\AgdaSymbol{\}} \AgdaSymbol{→}\<%
\\
\>[0]\AgdaIndent{12}{}\<[12]%
\>[12]\AgdaBound{ρ} \AgdaFunction{•} \AgdaSymbol{(}\AgdaBound{σ} \AgdaFunction{•RS} \AgdaBound{τ}\AgdaSymbol{)} \AgdaFunction{∼} \AgdaSymbol{(}\AgdaBound{ρ} \AgdaFunction{•SR} \AgdaBound{σ}\AgdaSymbol{)} \AgdaFunction{•} \AgdaBound{τ}\<%
\\
\>\AgdaFunction{assocRSSR} \AgdaSymbol{\{}\AgdaArgument{ρ} \AgdaSymbol{=} \AgdaBound{ρ}\AgdaSymbol{\}} \AgdaSymbol{\{}\AgdaBound{σ}\AgdaSymbol{\}} \AgdaSymbol{\{}\AgdaBound{τ}\AgdaSymbol{\}} \AgdaBound{x} \AgdaSymbol{=} \AgdaFunction{sym} \AgdaSymbol{(}\AgdaFunction{sub-compSR} \AgdaSymbol{(}\AgdaBound{τ} \AgdaSymbol{\_} \AgdaBound{x}\AgdaSymbol{))}\<%
\end{code}
}

\begin{code}%
\>\AgdaFunction{substitution} \AgdaSymbol{:} \AgdaRecord{OpFamily}\<%
\\
\>\AgdaFunction{substitution} \AgdaSymbol{=} \AgdaKeyword{record} \AgdaSymbol{\{} \<[24]%
\>[24]\<%
\\
\>[0]\AgdaIndent{2}{}\<[2]%
\>[2]\AgdaField{liftFamily} \AgdaSymbol{=} \AgdaFunction{proto-substitution} \AgdaSymbol{;} \<[36]%
\>[36]\<%
\\
\>[0]\AgdaIndent{2}{}\<[2]%
\>[2]\AgdaField{isOpFamily} \AgdaSymbol{=} \AgdaKeyword{record} \AgdaSymbol{\{} \<[24]%
\>[24]\<%
\\
\>[2]\AgdaIndent{4}{}\<[4]%
\>[4]\AgdaField{\_∘\_} \AgdaSymbol{=} \AgdaFunction{\_•\_} \AgdaSymbol{;} \<[16]%
\>[16]\<%
\\
\>[2]\AgdaIndent{4}{}\<[4]%
\>[4]\AgdaField{liftOp-comp} \AgdaSymbol{=} \AgdaFunction{Sub↑-comp} \AgdaSymbol{;} \<[30]%
\>[30]\<%
\\
\>[2]\AgdaIndent{4}{}\<[4]%
\>[4]\AgdaField{apV-comp} \AgdaSymbol{=} \AgdaInductiveConstructor{refl} \AgdaSymbol{\}} \<[22]%
\>[22]\<%
\\
\>[0]\AgdaIndent{2}{}\<[2]%
\>[2]\AgdaSymbol{\}}\<%
\end{code}

\AgdaHide{
\begin{code}%
\>\AgdaKeyword{open} \AgdaModule{OpFamily} \AgdaFunction{substitution} \AgdaKeyword{using} \AgdaSymbol{(}comp-congl\AgdaSymbol{;}comp-congr\AgdaSymbol{)}\<%
\\
\>[0]\AgdaIndent{2}{}\<[2]%
\>[2]\AgdaKeyword{renaming} \AgdaSymbol{(}liftOp-idOp \AgdaSymbol{to} Sub↑-idOp\AgdaSymbol{;}\<\\
\>           ap-idOp \AgdaSymbol{to} sub-idOp\AgdaSymbol{;}\<\\
\>           ap-congl \AgdaSymbol{to} sub-congr\AgdaSymbol{;}\<\\
\>           ap-congr \AgdaSymbol{to} sub-congl\AgdaSymbol{;}\<\\
\>           unitl \AgdaSymbol{to} sub-unitl\AgdaSymbol{;}\<\\
\>           unitr \AgdaSymbol{to} sub-unitr\AgdaSymbol{;}\<\\
\>           ∼-sym \AgdaSymbol{to} sub-sym\AgdaSymbol{;}\<\\
\>           ∼-trans \AgdaSymbol{to} sub-trans\AgdaSymbol{;}\<\\
\>           assoc \AgdaSymbol{to} sub-assoc\AgdaSymbol{)}\<%
\\
\>[0]\AgdaIndent{2}{}\<[2]%
\>[2]\AgdaKeyword{public}\<%
\end{code}
}

\begin{frame}[fragile]
\frametitle{Metatheorems}
We can now prove general results such as:

\begin{code}%
\>\AgdaFunction{sub-comp} \AgdaSymbol{:} \AgdaSymbol{∀} \AgdaSymbol{\{}\AgdaBound{U}\AgdaSymbol{\}} \AgdaSymbol{\{}\AgdaBound{V}\AgdaSymbol{\}} \AgdaSymbol{\{}\AgdaBound{W}\AgdaSymbol{\}} \AgdaSymbol{\{}\AgdaBound{C}\AgdaSymbol{\}} \AgdaSymbol{\{}\AgdaBound{K}\AgdaSymbol{\}}\<%
\\
\>[0]\AgdaIndent{2}{}\<[2]%
\>[2]\AgdaSymbol{(}\AgdaBound{E} \AgdaSymbol{:} \AgdaDatatype{Subexpression} \AgdaBound{U} \AgdaBound{C} \AgdaBound{K}\AgdaSymbol{)} \AgdaSymbol{\{}\AgdaBound{σ} \AgdaSymbol{:} \AgdaFunction{Sub} \AgdaBound{V} \AgdaBound{W}\AgdaSymbol{\}} \AgdaSymbol{\{}\AgdaBound{ρ} \AgdaSymbol{:} \AgdaFunction{Sub} \AgdaBound{U} \AgdaBound{V}\AgdaSymbol{\}} \AgdaSymbol{→}\<%
\\
\>[0]\AgdaIndent{2}{}\<[2]%
\>[2]\AgdaBound{E} \AgdaFunction{⟦} \AgdaBound{σ} \AgdaFunction{•} \AgdaBound{ρ} \AgdaFunction{⟧} \AgdaDatatype{≡} \AgdaBound{E} \AgdaFunction{⟦} \AgdaBound{ρ} \AgdaFunction{⟧} \AgdaFunction{⟦} \AgdaBound{σ} \AgdaFunction{⟧}\<%
\end{code}
\end{frame}

\AgdaHide{
\begin{code}%
\>\AgdaFunction{sub-comp} \AgdaSymbol{=} \AgdaFunction{OpFamily.ap-circ} \AgdaFunction{substitution}\<%
\end{code}
}

\AgdaHide{
\begin{code}%
\>\AgdaKeyword{open} \AgdaKeyword{import} \AgdaModule{Grammar.Base}\<%
\\
\>[0]\AgdaIndent{2}{}\<[2]%
\>[2]\<%
\\
\>\AgdaKeyword{module} \AgdaModule{Grammar.Substitution.Botsub} \AgdaSymbol{(}\AgdaBound{G} \AgdaSymbol{:} \AgdaRecord{Grammar}\AgdaSymbol{)} \AgdaKeyword{where}\<%
\\
\>\AgdaKeyword{open} \AgdaKeyword{import} \AgdaModule{Prelims}\<%
\\
\>\AgdaKeyword{open} \AgdaModule{Grammar} \AgdaBound{G}\<%
\\
\>\AgdaKeyword{open} \AgdaKeyword{import} \AgdaModule{Grammar.OpFamily} \AgdaBound{G}\<%
\\
\>\AgdaKeyword{open} \AgdaKeyword{import} \AgdaModule{Grammar.Replacement} \AgdaBound{G}\<%
\\
\>\AgdaKeyword{open} \AgdaKeyword{import} \AgdaModule{Grammar.Substitution.PreOpFamily} \AgdaBound{G}\<%
\\
\>\AgdaKeyword{open} \AgdaKeyword{import} \AgdaModule{Grammar.Substitution.Lifting} \AgdaBound{G}\<%
\\
\>\AgdaKeyword{open} \AgdaKeyword{import} \AgdaModule{Grammar.Substitution.LiftFamily} \AgdaBound{G}\<%
\\
\>\AgdaKeyword{open} \AgdaKeyword{import} \AgdaModule{Grammar.Substitution.OpFamily} \AgdaBound{G}\<%
\end{code}
}

\subsubsection{Substitution for an Individual Variable}

Let $E$ be an expression of kind $K$ over $V$.  Then we write $[x_0 := E]$ for the following substitution
$(V , K) \Rightarrow V$:

\AgdaHide{
\begin{code}%
\>\AgdaFunction{botSub} \AgdaSymbol{:} \AgdaSymbol{∀} \AgdaSymbol{\{}\AgdaBound{V}\AgdaSymbol{\}} \AgdaSymbol{\{}\AgdaBound{A}\AgdaSymbol{\}} \AgdaSymbol{→} \AgdaDatatype{ExpList} \AgdaBound{V} \AgdaBound{A} \AgdaSymbol{→} \AgdaFunction{Sub} \AgdaSymbol{(}\AgdaFunction{snoc-extend} \AgdaBound{V} \AgdaBound{A}\AgdaSymbol{)} \AgdaBound{V}\<%
\\
\>\AgdaFunction{botSub} \AgdaSymbol{\{}\AgdaArgument{A} \AgdaSymbol{=} \AgdaInductiveConstructor{[]}\AgdaSymbol{\}} \AgdaSymbol{\_} \AgdaSymbol{\_} \AgdaBound{x} \AgdaSymbol{=} \AgdaInductiveConstructor{var} \AgdaBound{x}\<%
\\
\>\AgdaFunction{botSub} \AgdaSymbol{\{}\AgdaArgument{A} \AgdaSymbol{=} \AgdaSymbol{\_} \AgdaInductiveConstructor{snoc} \AgdaSymbol{\_\}} \AgdaSymbol{(\_} \AgdaInductiveConstructor{snoc} \AgdaBound{E}\AgdaSymbol{)} \AgdaSymbol{\_} \AgdaInductiveConstructor{x₀} \AgdaSymbol{=} \AgdaBound{E}\<%
\\
\>\AgdaFunction{botSub} \AgdaSymbol{\{}\AgdaArgument{A} \AgdaSymbol{=} \AgdaSymbol{\_} \AgdaInductiveConstructor{snoc} \AgdaSymbol{\_\}} \AgdaSymbol{(}\AgdaBound{EE} \AgdaInductiveConstructor{snoc} \AgdaSymbol{\_)} \AgdaBound{L} \AgdaSymbol{(}\AgdaInductiveConstructor{↑} \AgdaBound{x}\AgdaSymbol{)} \AgdaSymbol{=} \AgdaFunction{botSub} \AgdaBound{EE} \AgdaBound{L} \AgdaBound{x}\<%
\end{code}
}

\begin{code}%
\>\AgdaKeyword{infix} \AgdaNumber{65} \AgdaFixityOp{x₀:=\_}\<%
\\
\>\AgdaFunction{x₀:=\_} \AgdaSymbol{:} \AgdaSymbol{∀} \AgdaSymbol{\{}\AgdaBound{V}\AgdaSymbol{\}} \AgdaSymbol{\{}\AgdaBound{K}\AgdaSymbol{\}} \AgdaSymbol{→} \AgdaFunction{Expression} \AgdaBound{V} \AgdaSymbol{(}\AgdaInductiveConstructor{varKind} \AgdaBound{K}\AgdaSymbol{)} \AgdaSymbol{→} \AgdaFunction{Sub} \AgdaSymbol{(}\AgdaBound{V} \AgdaInductiveConstructor{,} \AgdaBound{K}\AgdaSymbol{)} \AgdaBound{V}\<%
\\
\>\AgdaFunction{x₀:=} \AgdaBound{E} \AgdaSymbol{=} \AgdaFunction{botSub} \AgdaSymbol{(}\AgdaInductiveConstructor{[]} \AgdaInductiveConstructor{snoc} \AgdaBound{E}\AgdaSymbol{)}\<%
\end{code}

\begin{lemma}$ $
\begin{enumerate}
\item
$ \rho \bullet_1 [x_0 := E] \sim [x_0 := E \langle \rho \rangle] \bullet_2 (\rho , K) $
\item
$ \sigma \bullet [x_0 := E] \sim [x_0 := E[\sigma]] \bullet (\sigma , K) $
\item
$ E [ \uparrow ] [ x_0 := F ] \equiv E$
\end{enumerate}
\end{lemma}

\begin{code}%
\>\AgdaKeyword{open} \AgdaModule{LiftFamily}\<%
\\
%
\\
\>\AgdaFunction{botSub-up'} \AgdaSymbol{:} \AgdaSymbol{∀} \AgdaSymbol{\{}\AgdaBound{F}\AgdaSymbol{\}} \AgdaSymbol{\{}\AgdaBound{V}\AgdaSymbol{\}} \AgdaSymbol{\{}\AgdaBound{K}\AgdaSymbol{\}} \AgdaSymbol{\{}\AgdaBound{E} \AgdaSymbol{:} \AgdaFunction{Expression} \AgdaBound{V} \AgdaSymbol{(}\AgdaInductiveConstructor{varKind} \AgdaBound{K}\AgdaSymbol{)\}} \AgdaSymbol{(}\AgdaBound{circ} \AgdaSymbol{:} \AgdaRecord{Composition} \AgdaFunction{SubLF} \AgdaBound{F} \AgdaFunction{SubLF}\AgdaSymbol{)} \AgdaSymbol{→}\<%
\\
\>[0]\AgdaIndent{2}{}\<[2]%
\>[2]\AgdaField{Composition.circ} \AgdaBound{circ} \AgdaSymbol{(}\AgdaFunction{x₀:=} \AgdaBound{E}\AgdaSymbol{)} \AgdaSymbol{(}\AgdaFunction{up} \AgdaBound{F}\AgdaSymbol{)} \AgdaFunction{∼} \AgdaFunction{idSub} \AgdaBound{V}\<%
\\
\>\AgdaFunction{botSub-up'} \AgdaSymbol{\{}\AgdaBound{F}\AgdaSymbol{\}} \AgdaSymbol{\{}\AgdaBound{V}\AgdaSymbol{\}} \AgdaSymbol{\{}\AgdaBound{K}\AgdaSymbol{\}} \AgdaSymbol{\{}\AgdaBound{E}\AgdaSymbol{\}} \AgdaBound{circ} \AgdaBound{x} \AgdaSymbol{=} \AgdaKeyword{let} \AgdaKeyword{open} \AgdaModule{≡-Reasoning} \AgdaKeyword{in} \<[60]%
\>[60]\<%
\\
\>[0]\AgdaIndent{2}{}\<[2]%
\>[2]\AgdaFunction{begin}\<%
\\
\>[2]\AgdaIndent{4}{}\<[4]%
\>[4]\AgdaSymbol{(}\AgdaField{Composition.circ} \AgdaBound{circ} \AgdaSymbol{(}\AgdaFunction{x₀:=} \AgdaBound{E}\AgdaSymbol{)} \AgdaSymbol{(}\AgdaFunction{up} \AgdaBound{F}\AgdaSymbol{))} \AgdaSymbol{\_} \AgdaBound{x}\<%
\\
\>[0]\AgdaIndent{2}{}\<[2]%
\>[2]\AgdaFunction{≡⟨} \AgdaField{Composition.apV-circ} \AgdaBound{circ} \AgdaFunction{⟩}\<%
\\
\>[2]\AgdaIndent{4}{}\<[4]%
\>[4]\AgdaFunction{apV} \AgdaBound{F} \AgdaSymbol{(}\AgdaFunction{up} \AgdaBound{F}\AgdaSymbol{)} \AgdaBound{x} \AgdaFunction{⟦} \AgdaFunction{x₀:=} \AgdaBound{E} \AgdaFunction{⟧}\<%
\\
\>[0]\AgdaIndent{2}{}\<[2]%
\>[2]\AgdaFunction{≡⟨} \AgdaFunction{sub-congl} \AgdaSymbol{(}\AgdaFunction{apV-up} \AgdaBound{F}\AgdaSymbol{)} \AgdaFunction{⟩}\<%
\\
\>[2]\AgdaIndent{4}{}\<[4]%
\>[4]\AgdaInductiveConstructor{var} \AgdaBound{x}\<%
\\
\>[0]\AgdaIndent{2}{}\<[2]%
\>[2]\AgdaFunction{∎}\<%
\\
%
\\
\>\AgdaFunction{botSub-up} \AgdaSymbol{:} \AgdaSymbol{∀} \AgdaSymbol{\{}\AgdaBound{F}\AgdaSymbol{\}} \AgdaSymbol{\{}\AgdaBound{V}\AgdaSymbol{\}} \AgdaSymbol{\{}\AgdaBound{K}\AgdaSymbol{\}} \AgdaSymbol{\{}\AgdaBound{C}\AgdaSymbol{\}} \AgdaSymbol{\{}\AgdaBound{L}\AgdaSymbol{\}} \AgdaSymbol{\{}\AgdaBound{E} \AgdaSymbol{:} \AgdaFunction{Expression} \AgdaBound{V} \AgdaSymbol{(}\AgdaInductiveConstructor{varKind} \AgdaBound{K}\AgdaSymbol{)\}} \AgdaSymbol{(}\AgdaBound{circ} \AgdaSymbol{:} \AgdaRecord{Composition} \AgdaFunction{SubLF} \AgdaBound{F} \AgdaFunction{SubLF}\AgdaSymbol{)} \AgdaSymbol{\{}\AgdaBound{E'} \AgdaSymbol{:} \AgdaDatatype{Subexpression} \AgdaBound{V} \AgdaBound{C} \AgdaBound{L}\AgdaSymbol{\}} \AgdaSymbol{→}\<%
\\
\>[0]\AgdaIndent{2}{}\<[2]%
\>[2]\AgdaFunction{ap} \AgdaBound{F} \AgdaSymbol{(}\AgdaFunction{up} \AgdaBound{F}\AgdaSymbol{)} \AgdaBound{E'} \AgdaFunction{⟦} \AgdaFunction{x₀:=} \AgdaBound{E} \AgdaFunction{⟧} \AgdaDatatype{≡} \AgdaBound{E'}\<%
\\
\>\AgdaFunction{botSub-up} \AgdaSymbol{\{}\AgdaBound{F}\AgdaSymbol{\}} \AgdaSymbol{\{}\AgdaBound{V}\AgdaSymbol{\}} \AgdaSymbol{\{}\AgdaBound{K}\AgdaSymbol{\}} \AgdaSymbol{\{}\AgdaBound{C}\AgdaSymbol{\}} \AgdaSymbol{\{}\AgdaBound{L}\AgdaSymbol{\}} \AgdaSymbol{\{}\AgdaBound{E}\AgdaSymbol{\}} \AgdaBound{circ} \AgdaSymbol{\{}\AgdaBound{E'}\AgdaSymbol{\}} \AgdaSymbol{=} \AgdaKeyword{let} \AgdaKeyword{open} \AgdaModule{≡-Reasoning} \AgdaKeyword{in}\<%
\\
\>[0]\AgdaIndent{2}{}\<[2]%
\>[2]\AgdaFunction{begin}\<%
\\
\>[2]\AgdaIndent{4}{}\<[4]%
\>[4]\AgdaFunction{ap} \AgdaBound{F} \AgdaSymbol{(}\AgdaFunction{up} \AgdaBound{F}\AgdaSymbol{)} \AgdaBound{E'} \AgdaFunction{⟦} \AgdaFunction{x₀:=} \AgdaBound{E} \AgdaFunction{⟧}\<%
\\
\>[0]\AgdaIndent{2}{}\<[2]%
\>[2]\AgdaFunction{≡⟨⟨} \AgdaFunction{Composition.ap-circ} \AgdaBound{circ} \AgdaBound{E'} \AgdaFunction{⟩⟩}\<%
\\
\>[2]\AgdaIndent{4}{}\<[4]%
\>[4]\AgdaBound{E'} \AgdaFunction{⟦} \AgdaField{Composition.circ} \AgdaBound{circ} \AgdaSymbol{(}\AgdaFunction{x₀:=} \AgdaBound{E}\AgdaSymbol{)} \AgdaSymbol{(}\AgdaFunction{up} \AgdaBound{F}\AgdaSymbol{)} \AgdaFunction{⟧}\<%
\\
\>[0]\AgdaIndent{2}{}\<[2]%
\>[2]\AgdaFunction{≡⟨} \AgdaFunction{sub-congr} \AgdaSymbol{(}\AgdaFunction{botSub-up'} \AgdaBound{circ}\AgdaSymbol{)} \AgdaBound{E'} \AgdaFunction{⟩}\<%
\\
\>[2]\AgdaIndent{4}{}\<[4]%
\>[4]\AgdaBound{E'} \AgdaFunction{⟦} \AgdaFunction{idSub} \AgdaBound{V} \AgdaFunction{⟧}\<%
\\
\>[0]\AgdaIndent{2}{}\<[2]%
\>[2]\AgdaFunction{≡⟨} \AgdaFunction{sub-idOp} \AgdaFunction{⟩}\<%
\\
\>[2]\AgdaIndent{4}{}\<[4]%
\>[4]\AgdaBound{E'}\<%
\\
\>[0]\AgdaIndent{2}{}\<[2]%
\>[2]\AgdaFunction{∎}\<%
\\
%
\\
\>\AgdaFunction{circ-botSub'} \AgdaSymbol{:} \AgdaSymbol{∀} \AgdaSymbol{\{}\AgdaBound{F}\AgdaSymbol{\}} \AgdaSymbol{\{}\AgdaBound{U}\AgdaSymbol{\}} \AgdaSymbol{\{}\AgdaBound{V}\AgdaSymbol{\}} \AgdaSymbol{\{}\AgdaBound{K}\AgdaSymbol{\}} \AgdaSymbol{\{}\AgdaBound{E} \AgdaSymbol{:} \AgdaFunction{Expression} \AgdaBound{U} \AgdaSymbol{(}\AgdaInductiveConstructor{varKind} \AgdaBound{K}\AgdaSymbol{)\}} \<[64]%
\>[64]\<%
\\
\>[0]\AgdaIndent{2}{}\<[2]%
\>[2]\AgdaSymbol{(}\AgdaBound{circ₁} \AgdaSymbol{:} \AgdaRecord{Composition} \AgdaBound{F} \AgdaFunction{SubLF} \AgdaFunction{SubLF}\AgdaSymbol{)} \<[38]%
\>[38]\<%
\\
\>[0]\AgdaIndent{2}{}\<[2]%
\>[2]\AgdaSymbol{(}\AgdaBound{circ₂} \AgdaSymbol{:} \AgdaRecord{Composition} \AgdaFunction{SubLF} \AgdaBound{F} \AgdaFunction{SubLF}\AgdaSymbol{)}\<%
\\
\>[0]\AgdaIndent{2}{}\<[2]%
\>[2]\AgdaSymbol{\{}\AgdaBound{σ} \AgdaSymbol{:} \AgdaFunction{Op} \AgdaBound{F} \AgdaBound{U} \AgdaBound{V}\AgdaSymbol{\}} \AgdaSymbol{→}\<%
\\
\>[0]\AgdaIndent{2}{}\<[2]%
\>[2]\AgdaField{Composition.circ} \AgdaBound{circ₁} \AgdaBound{σ} \AgdaSymbol{(}\AgdaFunction{x₀:=} \AgdaBound{E}\AgdaSymbol{)} \AgdaFunction{∼} \AgdaField{Composition.circ} \AgdaBound{circ₂} \AgdaSymbol{(}\AgdaFunction{x₀:=} \AgdaSymbol{(}\AgdaFunction{ap} \AgdaBound{F} \AgdaBound{σ} \AgdaBound{E}\AgdaSymbol{))} \AgdaSymbol{(}\AgdaFunction{liftOp} \AgdaBound{F} \AgdaBound{K} \AgdaBound{σ}\AgdaSymbol{)}\<%
\\
\>\AgdaFunction{circ-botSub'} \AgdaSymbol{\{}\AgdaBound{F}\AgdaSymbol{\}} \AgdaSymbol{\{}\AgdaBound{U}\AgdaSymbol{\}} \AgdaSymbol{\{}\AgdaBound{V}\AgdaSymbol{\}} \AgdaSymbol{\{}\AgdaBound{K}\AgdaSymbol{\}} \AgdaSymbol{\{}\AgdaBound{E}\AgdaSymbol{\}} \AgdaBound{circ₁} \AgdaBound{circ₂} \AgdaSymbol{\{}\AgdaBound{σ}\AgdaSymbol{\}} \AgdaInductiveConstructor{x₀} \AgdaSymbol{=} \AgdaKeyword{let} \AgdaKeyword{open} \AgdaModule{≡-Reasoning} \AgdaKeyword{in} \<[78]%
\>[78]\<%
\\
\>[0]\AgdaIndent{2}{}\<[2]%
\>[2]\AgdaFunction{begin}\<%
\\
\>[2]\AgdaIndent{4}{}\<[4]%
\>[4]\AgdaSymbol{(}\AgdaField{Composition.circ} \AgdaBound{circ₁} \AgdaBound{σ} \AgdaSymbol{(}\AgdaFunction{x₀:=} \AgdaBound{E}\AgdaSymbol{))} \AgdaSymbol{\_} \AgdaInductiveConstructor{x₀}\<%
\\
\>[0]\AgdaIndent{2}{}\<[2]%
\>[2]\AgdaFunction{≡⟨} \AgdaField{Composition.apV-circ} \AgdaBound{circ₁} \AgdaFunction{⟩}\<%
\\
\>[2]\AgdaIndent{4}{}\<[4]%
\>[4]\AgdaFunction{ap} \AgdaBound{F} \AgdaBound{σ} \AgdaBound{E}\<%
\\
\>[0]\AgdaIndent{2}{}\<[2]%
\>[2]\AgdaFunction{≡⟨⟨} \AgdaFunction{sub-congl} \AgdaSymbol{(}\AgdaFunction{liftOp-x₀} \AgdaBound{F}\AgdaSymbol{)} \AgdaFunction{⟩⟩}\<%
\\
\>[2]\AgdaIndent{4}{}\<[4]%
\>[4]\AgdaSymbol{(}\AgdaFunction{apV} \AgdaBound{F} \AgdaSymbol{(}\AgdaFunction{liftOp} \AgdaBound{F} \AgdaBound{K} \AgdaBound{σ}\AgdaSymbol{)} \AgdaInductiveConstructor{x₀}\AgdaSymbol{)} \AgdaFunction{⟦} \AgdaFunction{x₀:=} \AgdaSymbol{(}\AgdaFunction{ap} \AgdaBound{F} \AgdaBound{σ} \AgdaBound{E}\AgdaSymbol{)} \AgdaFunction{⟧}\<%
\\
\>[0]\AgdaIndent{2}{}\<[2]%
\>[2]\AgdaFunction{≡⟨⟨} \AgdaField{Composition.apV-circ} \AgdaBound{circ₂} \AgdaFunction{⟩⟩}\<%
\\
\>[2]\AgdaIndent{4}{}\<[4]%
\>[4]\AgdaSymbol{(}\AgdaField{Composition.circ} \AgdaBound{circ₂} \AgdaSymbol{(}\AgdaFunction{x₀:=} \AgdaSymbol{(}\AgdaFunction{ap} \AgdaBound{F} \AgdaBound{σ} \AgdaBound{E}\AgdaSymbol{))} \AgdaSymbol{(}\AgdaFunction{liftOp} \AgdaBound{F} \AgdaBound{K} \AgdaBound{σ}\AgdaSymbol{))} \AgdaSymbol{\_} \AgdaInductiveConstructor{x₀}\<%
\\
\>[0]\AgdaIndent{2}{}\<[2]%
\>[2]\AgdaFunction{∎}\<%
\\
\>\AgdaFunction{circ-botSub'} \AgdaSymbol{\{}\AgdaBound{F}\AgdaSymbol{\}} \AgdaSymbol{\{}\AgdaBound{U}\AgdaSymbol{\}} \AgdaSymbol{\{}\AgdaBound{V}\AgdaSymbol{\}} \AgdaSymbol{\{}\AgdaBound{K}\AgdaSymbol{\}} \AgdaSymbol{\{}\AgdaBound{E}\AgdaSymbol{\}} \AgdaBound{circ₁} \AgdaBound{circ₂} \AgdaSymbol{\{}\AgdaBound{σ}\AgdaSymbol{\}} \AgdaSymbol{(}\AgdaInductiveConstructor{↑} \AgdaBound{x}\AgdaSymbol{)} \AgdaSymbol{=} \AgdaKeyword{let} \AgdaKeyword{open} \AgdaModule{≡-Reasoning} \AgdaKeyword{in} \<[81]%
\>[81]\<%
\\
\>[0]\AgdaIndent{2}{}\<[2]%
\>[2]\AgdaFunction{begin}\<%
\\
\>[2]\AgdaIndent{4}{}\<[4]%
\>[4]\AgdaSymbol{(}\AgdaField{Composition.circ} \AgdaBound{circ₁} \AgdaBound{σ} \AgdaSymbol{(}\AgdaFunction{x₀:=} \AgdaBound{E}\AgdaSymbol{))} \AgdaSymbol{\_} \AgdaSymbol{(}\AgdaInductiveConstructor{↑} \AgdaBound{x}\AgdaSymbol{)}\<%
\\
\>[0]\AgdaIndent{2}{}\<[2]%
\>[2]\AgdaFunction{≡⟨} \AgdaField{Composition.apV-circ} \AgdaBound{circ₁} \AgdaFunction{⟩}\<%
\\
\>[2]\AgdaIndent{4}{}\<[4]%
\>[4]\AgdaFunction{apV} \AgdaBound{F} \AgdaBound{σ} \AgdaBound{x}\<%
\\
\>[0]\AgdaIndent{2}{}\<[2]%
\>[2]\AgdaFunction{≡⟨⟨} \AgdaFunction{sub-idOp} \AgdaFunction{⟩⟩}\<%
\\
\>[2]\AgdaIndent{4}{}\<[4]%
\>[4]\AgdaFunction{apV} \AgdaBound{F} \AgdaBound{σ} \AgdaBound{x} \AgdaFunction{⟦} \AgdaFunction{idSub} \AgdaBound{V} \AgdaFunction{⟧}\<%
\\
\>[0]\AgdaIndent{2}{}\<[2]%
\>[2]\AgdaFunction{≡⟨⟨} \AgdaFunction{sub-congr} \AgdaSymbol{(}\AgdaFunction{botSub-up'} \AgdaBound{circ₂}\AgdaSymbol{)} \AgdaSymbol{(}\AgdaFunction{apV} \AgdaBound{F} \AgdaBound{σ} \AgdaBound{x}\AgdaSymbol{)} \AgdaFunction{⟩⟩}\<%
\\
\>[2]\AgdaIndent{4}{}\<[4]%
\>[4]\AgdaFunction{apV} \AgdaBound{F} \AgdaBound{σ} \AgdaBound{x} \AgdaFunction{⟦} \AgdaField{Composition.circ} \AgdaBound{circ₂} \AgdaSymbol{(}\AgdaFunction{x₀:=} \AgdaSymbol{(}\AgdaFunction{ap} \AgdaBound{F} \AgdaBound{σ} \AgdaBound{E}\AgdaSymbol{))} \AgdaSymbol{(}\AgdaFunction{up} \AgdaBound{F}\AgdaSymbol{)} \AgdaFunction{⟧}\<%
\\
\>[0]\AgdaIndent{2}{}\<[2]%
\>[2]\AgdaFunction{≡⟨} \AgdaFunction{Composition.ap-circ} \AgdaBound{circ₂} \AgdaSymbol{(}\AgdaFunction{apV} \AgdaBound{F} \AgdaBound{σ} \AgdaBound{x}\AgdaSymbol{)} \AgdaFunction{⟩}\<%
\\
\>[2]\AgdaIndent{4}{}\<[4]%
\>[4]\AgdaFunction{ap} \AgdaBound{F} \AgdaSymbol{(}\AgdaFunction{up} \AgdaBound{F}\AgdaSymbol{)} \AgdaSymbol{(}\AgdaFunction{apV} \AgdaBound{F} \AgdaBound{σ} \AgdaBound{x}\AgdaSymbol{)} \AgdaFunction{⟦} \AgdaFunction{x₀:=} \AgdaSymbol{(}\AgdaFunction{ap} \AgdaBound{F} \AgdaBound{σ} \AgdaBound{E}\AgdaSymbol{)} \AgdaFunction{⟧}\<%
\\
\>[0]\AgdaIndent{2}{}\<[2]%
\>[2]\AgdaFunction{≡⟨⟨} \AgdaFunction{sub-congl} \AgdaSymbol{(}\AgdaFunction{liftOp-↑} \AgdaBound{F} \AgdaBound{x}\AgdaSymbol{)} \AgdaFunction{⟩⟩}\<%
\\
\>[2]\AgdaIndent{4}{}\<[4]%
\>[4]\AgdaSymbol{(}\AgdaFunction{apV} \AgdaBound{F} \AgdaSymbol{(}\AgdaFunction{liftOp} \AgdaBound{F} \AgdaBound{K} \AgdaBound{σ}\AgdaSymbol{)} \AgdaSymbol{(}\AgdaInductiveConstructor{↑} \AgdaBound{x}\AgdaSymbol{))} \AgdaFunction{⟦} \AgdaFunction{x₀:=} \AgdaSymbol{(}\AgdaFunction{ap} \AgdaBound{F} \AgdaBound{σ} \AgdaBound{E}\AgdaSymbol{)} \AgdaFunction{⟧}\<%
\\
\>[0]\AgdaIndent{2}{}\<[2]%
\>[2]\AgdaFunction{≡⟨⟨} \AgdaField{Composition.apV-circ} \AgdaBound{circ₂} \AgdaFunction{⟩⟩}\<%
\\
\>[2]\AgdaIndent{4}{}\<[4]%
\>[4]\AgdaSymbol{(}\AgdaField{Composition.circ} \AgdaBound{circ₂} \AgdaSymbol{(}\AgdaFunction{x₀:=} \AgdaSymbol{(}\AgdaFunction{ap} \AgdaBound{F} \AgdaBound{σ} \AgdaBound{E}\AgdaSymbol{))} \AgdaSymbol{(}\AgdaFunction{liftOp} \AgdaBound{F} \AgdaBound{K} \AgdaBound{σ}\AgdaSymbol{))} \AgdaSymbol{\_} \AgdaSymbol{(}\AgdaInductiveConstructor{↑} \AgdaBound{x}\AgdaSymbol{)}\<%
\\
\>[0]\AgdaIndent{2}{}\<[2]%
\>[2]\AgdaFunction{∎}\<%
\\
%
\\
\>\AgdaFunction{circ-botSub} \AgdaSymbol{:} \AgdaSymbol{∀} \AgdaSymbol{\{}\AgdaBound{F}\AgdaSymbol{\}} \AgdaSymbol{\{}\AgdaBound{U}\AgdaSymbol{\}} \AgdaSymbol{\{}\AgdaBound{V}\AgdaSymbol{\}} \AgdaSymbol{\{}\AgdaBound{K}\AgdaSymbol{\}} \AgdaSymbol{\{}\AgdaBound{C}\AgdaSymbol{\}} \AgdaSymbol{\{}\AgdaBound{L}\AgdaSymbol{\}} \<[40]%
\>[40]\<%
\\
\>[0]\AgdaIndent{2}{}\<[2]%
\>[2]\AgdaSymbol{\{}\AgdaBound{E} \AgdaSymbol{:} \AgdaFunction{Expression} \AgdaBound{U} \AgdaSymbol{(}\AgdaInductiveConstructor{varKind} \AgdaBound{K}\AgdaSymbol{)\}} \AgdaSymbol{\{}\AgdaBound{E'} \AgdaSymbol{:} \AgdaDatatype{Subexpression} \AgdaSymbol{(}\AgdaBound{U} \AgdaInductiveConstructor{,} \AgdaBound{K}\AgdaSymbol{)} \AgdaBound{C} \AgdaBound{L}\AgdaSymbol{\}} \AgdaSymbol{\{}\AgdaBound{σ} \AgdaSymbol{:} \AgdaFunction{Op} \AgdaBound{F} \AgdaBound{U} \AgdaBound{V}\AgdaSymbol{\}} \AgdaSymbol{→}\<%
\\
\>[0]\AgdaIndent{2}{}\<[2]%
\>[2]\AgdaRecord{Composition} \AgdaBound{F} \AgdaFunction{SubLF} \AgdaFunction{SubLF} \AgdaSymbol{→}\<%
\\
\>[0]\AgdaIndent{2}{}\<[2]%
\>[2]\AgdaRecord{Composition} \AgdaFunction{SubLF} \AgdaBound{F} \AgdaFunction{SubLF} \AgdaSymbol{→}\<%
\\
\>[0]\AgdaIndent{2}{}\<[2]%
\>[2]\AgdaFunction{ap} \AgdaBound{F} \AgdaBound{σ} \AgdaSymbol{(}\AgdaBound{E'} \AgdaFunction{⟦} \AgdaFunction{x₀:=} \AgdaBound{E} \AgdaFunction{⟧}\AgdaSymbol{)} \AgdaDatatype{≡} \AgdaSymbol{(}\AgdaFunction{ap} \AgdaBound{F} \AgdaSymbol{(}\AgdaFunction{liftOp} \AgdaBound{F} \AgdaBound{K} \AgdaBound{σ}\AgdaSymbol{)} \AgdaBound{E'}\AgdaSymbol{)} \AgdaFunction{⟦} \AgdaFunction{x₀:=} \AgdaSymbol{(}\AgdaFunction{ap} \AgdaBound{F} \AgdaBound{σ} \AgdaBound{E}\AgdaSymbol{)} \AgdaFunction{⟧}\<%
\\
\>\AgdaFunction{circ-botSub} \AgdaSymbol{\{}\AgdaArgument{E'} \AgdaSymbol{=} \AgdaBound{E'}\AgdaSymbol{\}} \AgdaBound{circ₁} \AgdaBound{circ₂} \AgdaSymbol{=} \AgdaFunction{ap-circ-sim} \AgdaBound{circ₁} \AgdaBound{circ₂} \AgdaSymbol{(}\AgdaFunction{circ-botSub'} \AgdaBound{circ₁} \AgdaBound{circ₂}\AgdaSymbol{)} \AgdaBound{E'}\<%
\\
%
\\
\>\AgdaFunction{compRS-botSub} \AgdaSymbol{:} \AgdaSymbol{∀} \AgdaSymbol{\{}\AgdaBound{U}\AgdaSymbol{\}} \AgdaSymbol{\{}\AgdaBound{V}\AgdaSymbol{\}} \AgdaSymbol{\{}\AgdaBound{C}\AgdaSymbol{\}} \AgdaSymbol{\{}\AgdaBound{K}\AgdaSymbol{\}} \AgdaSymbol{\{}\AgdaBound{L}\AgdaSymbol{\}} \AgdaSymbol{(}\AgdaBound{E} \AgdaSymbol{:} \AgdaDatatype{Subexpression} \AgdaSymbol{(}\AgdaBound{U} \AgdaInductiveConstructor{,} \AgdaBound{K}\AgdaSymbol{)} \AgdaBound{C} \AgdaBound{L}\AgdaSymbol{)} \AgdaSymbol{\{}\AgdaBound{F} \AgdaSymbol{:} \AgdaFunction{Expression} \AgdaBound{U} \AgdaSymbol{(}\AgdaInductiveConstructor{varKind} \AgdaBound{K}\AgdaSymbol{)\}} \AgdaSymbol{\{}\AgdaBound{ρ} \AgdaSymbol{:} \AgdaFunction{Rep} \AgdaBound{U} \AgdaBound{V}\AgdaSymbol{\}} \AgdaSymbol{→}\<%
\\
\>[0]\AgdaIndent{2}{}\<[2]%
\>[2]\AgdaBound{E} \AgdaFunction{⟦} \AgdaFunction{x₀:=} \AgdaBound{F} \AgdaFunction{⟧} \AgdaFunction{〈} \AgdaBound{ρ} \AgdaFunction{〉} \AgdaDatatype{≡} \AgdaBound{E} \AgdaFunction{〈} \AgdaFunction{liftRep} \AgdaBound{K} \AgdaBound{ρ} \AgdaFunction{〉} \AgdaFunction{⟦} \AgdaFunction{x₀:=} \AgdaSymbol{(}\AgdaBound{F} \AgdaFunction{〈} \AgdaBound{ρ} \AgdaFunction{〉}\AgdaSymbol{)} \AgdaFunction{⟧}\<%
\\
\>\AgdaComment{--TODO Common pattern with liftRep-botSub₃}\<%
\end{code}

\AgdaHide{
\begin{code}%
\>\AgdaFunction{compRS-botSub} \AgdaBound{E} \AgdaSymbol{=} \AgdaFunction{circ-botSub} \AgdaSymbol{\{}\AgdaArgument{E'} \AgdaSymbol{=} \AgdaBound{E}\AgdaSymbol{\}} \AgdaFunction{COMPRS} \AgdaFunction{COMPSR}\<%
\end{code}
}

\begin{code}%
\>\AgdaFunction{comp-botSub} \AgdaSymbol{:} \AgdaSymbol{∀} \AgdaSymbol{\{}\AgdaBound{U}\AgdaSymbol{\}} \AgdaSymbol{\{}\AgdaBound{V}\AgdaSymbol{\}} \AgdaSymbol{\{}\AgdaBound{C}\AgdaSymbol{\}} \AgdaSymbol{\{}\AgdaBound{K}\AgdaSymbol{\}} \AgdaSymbol{\{}\AgdaBound{L}\AgdaSymbol{\}} \<[36]%
\>[36]\<%
\\
\>[0]\AgdaIndent{2}{}\<[2]%
\>[2]\AgdaSymbol{\{}\AgdaBound{E} \AgdaSymbol{:} \AgdaFunction{Expression} \AgdaBound{U} \AgdaSymbol{(}\AgdaInductiveConstructor{varKind} \AgdaBound{K}\AgdaSymbol{)\}} \AgdaSymbol{\{}\AgdaBound{σ} \AgdaSymbol{:} \AgdaFunction{Sub} \AgdaBound{U} \AgdaBound{V}\AgdaSymbol{\}} \AgdaSymbol{(}\AgdaBound{F} \AgdaSymbol{:} \AgdaDatatype{Subexpression} \AgdaSymbol{(}\AgdaBound{U} \AgdaInductiveConstructor{,} \AgdaBound{K}\AgdaSymbol{)} \AgdaBound{C} \AgdaBound{L}\AgdaSymbol{)} \AgdaSymbol{→}\<%
\\
\>[2]\AgdaIndent{3}{}\<[3]%
\>[3]\AgdaBound{F} \AgdaFunction{⟦} \AgdaFunction{x₀:=} \AgdaBound{E} \AgdaFunction{⟧} \AgdaFunction{⟦} \AgdaBound{σ} \AgdaFunction{⟧} \AgdaDatatype{≡} \AgdaBound{F} \AgdaFunction{⟦} \AgdaFunction{liftSub} \AgdaBound{K} \AgdaBound{σ} \AgdaFunction{⟧} \AgdaFunction{⟦} \AgdaFunction{x₀:=} \AgdaSymbol{(}\AgdaBound{E} \AgdaFunction{⟦} \AgdaBound{σ} \AgdaFunction{⟧}\AgdaSymbol{)} \AgdaFunction{⟧}\<%
\end{code}

\AgdaHide{
\begin{code}%
\>\AgdaFunction{comp-botSub} \AgdaBound{F} \AgdaSymbol{=} \AgdaKeyword{let} \AgdaBound{COMP} \AgdaSymbol{=} \AgdaFunction{OpFamily.COMP} \AgdaFunction{SUB} \AgdaKeyword{in} \AgdaFunction{circ-botSub} \AgdaSymbol{\{}\AgdaArgument{E'} \AgdaSymbol{=} \AgdaBound{F}\AgdaSymbol{\}} \AgdaBound{COMP} \AgdaBound{COMP}\<%
\end{code}
}

\begin{code}%
\>\AgdaFunction{botSub-upRep} \AgdaSymbol{:} \AgdaSymbol{∀} \AgdaSymbol{\{}\AgdaBound{U}\AgdaSymbol{\}} \AgdaSymbol{\{}\AgdaBound{C}\AgdaSymbol{\}} \AgdaSymbol{\{}\AgdaBound{K}\AgdaSymbol{\}} \AgdaSymbol{\{}\AgdaBound{L}\AgdaSymbol{\}}\<%
\\
\>[0]\AgdaIndent{2}{}\<[2]%
\>[2]\AgdaSymbol{(}\AgdaBound{E} \AgdaSymbol{:} \AgdaDatatype{Subexpression} \AgdaBound{U} \AgdaBound{C} \AgdaBound{K}\AgdaSymbol{)} \AgdaSymbol{\{}\AgdaBound{F} \AgdaSymbol{:} \AgdaFunction{Expression} \AgdaBound{U} \AgdaSymbol{(}\AgdaInductiveConstructor{varKind} \AgdaBound{L}\AgdaSymbol{)\}} \AgdaSymbol{→} \<[61]%
\>[61]\<%
\\
\>[0]\AgdaIndent{2}{}\<[2]%
\>[2]\AgdaBound{E} \AgdaFunction{〈} \AgdaFunction{upRep} \AgdaFunction{〉} \AgdaFunction{⟦} \AgdaFunction{x₀:=} \AgdaBound{F} \AgdaFunction{⟧} \AgdaDatatype{≡} \AgdaBound{E}\<%
\end{code}

\AgdaHide{
\begin{code}%
\>\AgdaFunction{botSub-upRep} \AgdaSymbol{\_} \AgdaSymbol{=} \AgdaFunction{botSub-up} \AgdaFunction{COMPSR}\<%
\\
%
\\
\>\AgdaFunction{botSub-botSub'} \AgdaSymbol{:} \AgdaSymbol{∀} \AgdaSymbol{\{}\AgdaBound{V}\AgdaSymbol{\}} \AgdaSymbol{\{}\AgdaBound{K}\AgdaSymbol{\}} \AgdaSymbol{\{}\AgdaBound{L}\AgdaSymbol{\}} \AgdaSymbol{(}\AgdaBound{N} \AgdaSymbol{:} \AgdaFunction{Expression} \AgdaBound{V} \AgdaSymbol{(}\AgdaInductiveConstructor{varKind} \AgdaBound{K}\AgdaSymbol{))} \AgdaSymbol{(}\AgdaBound{N'} \AgdaSymbol{:} \AgdaFunction{Expression} \AgdaBound{V} \AgdaSymbol{(}\AgdaInductiveConstructor{varKind} \AgdaBound{L}\AgdaSymbol{))} \AgdaSymbol{→} \AgdaFunction{x₀:=} \AgdaBound{N'} \AgdaFunction{•} \AgdaFunction{liftSub} \AgdaBound{L} \AgdaSymbol{(}\AgdaFunction{x₀:=} \AgdaBound{N}\AgdaSymbol{)} \AgdaFunction{∼} \AgdaFunction{x₀:=} \AgdaBound{N} \AgdaFunction{•} \AgdaFunction{x₀:=} \AgdaSymbol{(}\AgdaBound{N'} \AgdaFunction{⇑}\AgdaSymbol{)}\<%
\\
\>\AgdaFunction{botSub-botSub'} \AgdaBound{N} \AgdaBound{N'} \AgdaInductiveConstructor{x₀} \AgdaSymbol{=} \AgdaFunction{sym} \AgdaSymbol{(}\AgdaFunction{botSub-upRep} \AgdaBound{N'}\AgdaSymbol{)}\<%
\\
\>\AgdaFunction{botSub-botSub'} \AgdaBound{N} \AgdaBound{N'} \AgdaSymbol{(}\AgdaInductiveConstructor{↑} \AgdaInductiveConstructor{x₀}\AgdaSymbol{)} \AgdaSymbol{=} \AgdaFunction{botSub-upRep} \AgdaBound{N}\<%
\\
\>\AgdaFunction{botSub-botSub'} \AgdaBound{N} \AgdaBound{N'} \AgdaSymbol{(}\AgdaInductiveConstructor{↑} \AgdaSymbol{(}\AgdaInductiveConstructor{↑} \AgdaBound{x}\AgdaSymbol{))} \AgdaSymbol{=} \AgdaInductiveConstructor{refl}\<%
\\
%
\\
\>\AgdaFunction{botSub-botSub} \AgdaSymbol{:} \AgdaSymbol{∀} \AgdaSymbol{\{}\AgdaBound{V}\AgdaSymbol{\}} \AgdaSymbol{\{}\AgdaBound{K}\AgdaSymbol{\}} \AgdaSymbol{\{}\AgdaBound{L}\AgdaSymbol{\}} \AgdaSymbol{\{}\AgdaBound{M}\AgdaSymbol{\}} \AgdaSymbol{(}\AgdaBound{E} \AgdaSymbol{:} \AgdaFunction{Expression} \AgdaSymbol{(}\AgdaBound{V} \AgdaInductiveConstructor{,} \AgdaBound{K} \AgdaInductiveConstructor{,} \AgdaBound{L}\AgdaSymbol{)} \AgdaBound{M}\AgdaSymbol{)} \AgdaBound{F} \AgdaBound{G} \AgdaSymbol{→} \AgdaBound{E} \AgdaFunction{⟦} \AgdaFunction{liftSub} \AgdaBound{L} \AgdaSymbol{(}\AgdaFunction{x₀:=} \AgdaBound{F}\AgdaSymbol{)} \AgdaFunction{⟧} \AgdaFunction{⟦} \AgdaFunction{x₀:=} \AgdaBound{G} \AgdaFunction{⟧} \AgdaDatatype{≡} \AgdaBound{E} \AgdaFunction{⟦} \AgdaFunction{x₀:=} \AgdaSymbol{(}\AgdaBound{G} \AgdaFunction{⇑}\AgdaSymbol{)} \AgdaFunction{⟧} \AgdaFunction{⟦} \AgdaFunction{x₀:=} \AgdaBound{F} \AgdaFunction{⟧}\<%
\\
\>\AgdaFunction{botSub-botSub} \AgdaSymbol{\{}\AgdaBound{V}\AgdaSymbol{\}} \AgdaSymbol{\{}\AgdaBound{K}\AgdaSymbol{\}} \AgdaSymbol{\{}\AgdaBound{L}\AgdaSymbol{\}} \AgdaSymbol{\{}\AgdaBound{M}\AgdaSymbol{\}} \AgdaBound{E} \AgdaBound{F} \AgdaBound{G} \AgdaSymbol{=} \AgdaKeyword{let} \AgdaBound{COMP} \AgdaSymbol{=} \AgdaFunction{OpFamily.COMP} \AgdaFunction{SUB} \AgdaKeyword{in} \AgdaFunction{ap-circ-sim} \AgdaBound{COMP} \AgdaBound{COMP} \AgdaSymbol{(}\AgdaFunction{botSub-botSub'} \AgdaBound{F} \AgdaBound{G}\AgdaSymbol{)} \AgdaBound{E}\<%
\\
%
\\
\>\AgdaFunction{x₂:=\_,x₁:=\_,x₀:=\_} \AgdaSymbol{:} \AgdaSymbol{∀} \AgdaSymbol{\{}\AgdaBound{V}\AgdaSymbol{\}} \AgdaSymbol{\{}\AgdaBound{K1}\AgdaSymbol{\}} \AgdaSymbol{\{}\AgdaBound{K2}\AgdaSymbol{\}} \AgdaSymbol{\{}\AgdaBound{K3}\AgdaSymbol{\}} \AgdaSymbol{→} \AgdaFunction{Expression} \AgdaBound{V} \AgdaSymbol{(}\AgdaInductiveConstructor{varKind} \AgdaBound{K1}\AgdaSymbol{)} \AgdaSymbol{→} \AgdaFunction{Expression} \AgdaBound{V} \AgdaSymbol{(}\AgdaInductiveConstructor{varKind} \AgdaBound{K2}\AgdaSymbol{)} \AgdaSymbol{→} \AgdaFunction{Expression} \AgdaBound{V} \AgdaSymbol{(}\AgdaInductiveConstructor{varKind} \AgdaBound{K3}\AgdaSymbol{)} \AgdaSymbol{→} \AgdaFunction{Sub} \AgdaSymbol{(}\AgdaBound{V} \AgdaInductiveConstructor{,} \AgdaBound{K1} \AgdaInductiveConstructor{,} \AgdaBound{K2} \AgdaInductiveConstructor{,} \AgdaBound{K3}\AgdaSymbol{)} \AgdaBound{V}\<%
\\
\>\AgdaFunction{x₂:=\_,x₁:=\_,x₀:=\_} \AgdaBound{M1} \AgdaBound{M2} \AgdaBound{M3} \AgdaSymbol{=} \AgdaFunction{botSub} \AgdaSymbol{(}\AgdaInductiveConstructor{[]} \AgdaInductiveConstructor{snoc} \AgdaBound{M1} \AgdaInductiveConstructor{snoc} \AgdaBound{M2} \AgdaInductiveConstructor{snoc} \AgdaBound{M3}\AgdaSymbol{)}\<%
\\
%
\\
\>\AgdaKeyword{postulate} \AgdaPostulate{botSub-upRep₃} \AgdaSymbol{:} \AgdaSymbol{∀} \AgdaSymbol{\{}\AgdaBound{V}\AgdaSymbol{\}} \AgdaSymbol{\{}\AgdaBound{K1}\AgdaSymbol{\}} \AgdaSymbol{\{}\AgdaBound{K2}\AgdaSymbol{\}} \AgdaSymbol{\{}\AgdaBound{K3}\AgdaSymbol{\}} \AgdaSymbol{\{}\AgdaBound{L}\AgdaSymbol{\}} \AgdaSymbol{\{}\AgdaBound{M} \AgdaSymbol{:} \AgdaFunction{Expression} \AgdaBound{V} \AgdaBound{L}\AgdaSymbol{\}} \<[72]%
\>[72]\<%
\\
\>[2]\AgdaIndent{26}{}\<[26]%
\>[26]\AgdaSymbol{\{}\AgdaBound{N1} \AgdaSymbol{:} \AgdaFunction{Expression} \AgdaBound{V} \AgdaSymbol{(}\AgdaInductiveConstructor{varKind} \AgdaBound{K1}\AgdaSymbol{)\}} \AgdaSymbol{\{}\AgdaBound{N2} \AgdaSymbol{:} \AgdaFunction{Expression} \AgdaBound{V} \AgdaSymbol{(}\AgdaInductiveConstructor{varKind} \AgdaBound{K2}\AgdaSymbol{)\}} \AgdaSymbol{\{}\AgdaBound{N3} \AgdaSymbol{:} \AgdaFunction{Expression} \AgdaBound{V} \AgdaSymbol{(}\AgdaInductiveConstructor{varKind} \AgdaBound{K3}\AgdaSymbol{)\}} \AgdaSymbol{→}\<%
\\
\>[2]\AgdaIndent{26}{}\<[26]%
\>[26]\AgdaBound{M} \AgdaFunction{⇑} \AgdaFunction{⇑} \AgdaFunction{⇑} \AgdaFunction{⟦} \AgdaFunction{x₂:=} \AgdaBound{N1} \AgdaFunction{,x₁:=} \AgdaBound{N2} \AgdaFunction{,x₀:=} \AgdaBound{N3} \AgdaFunction{⟧} \AgdaDatatype{≡} \AgdaBound{M}\<%
\\
%
\\
\>\AgdaComment{--TODO Definition for Expression varKind}\<%
\\
\>\AgdaFunction{botSub₃-liftRep₃'} \AgdaSymbol{:} \AgdaSymbol{∀} \AgdaSymbol{\{}\AgdaBound{U}\AgdaSymbol{\}} \AgdaSymbol{\{}\AgdaBound{V}\AgdaSymbol{\}} \AgdaSymbol{\{}\AgdaBound{K2}\AgdaSymbol{\}} \AgdaSymbol{\{}\AgdaBound{K1}\AgdaSymbol{\}} \AgdaSymbol{\{}\AgdaBound{K0}\AgdaSymbol{\}}\<%
\\
\>[0]\AgdaIndent{2}{}\<[2]%
\>[2]\AgdaSymbol{\{}\AgdaBound{M2} \AgdaSymbol{:} \AgdaFunction{Expression} \AgdaBound{U} \AgdaSymbol{(}\AgdaInductiveConstructor{varKind} \AgdaBound{K1}\AgdaSymbol{)\}} \AgdaSymbol{\{}\AgdaBound{M1} \AgdaSymbol{:} \AgdaFunction{Expression} \AgdaBound{U} \AgdaSymbol{(}\AgdaInductiveConstructor{varKind} \AgdaBound{K2}\AgdaSymbol{)\}} \AgdaSymbol{\{}\AgdaBound{M0} \AgdaSymbol{:} \AgdaFunction{Expression} \AgdaBound{U} \AgdaSymbol{(}\AgdaInductiveConstructor{varKind} \AgdaBound{K0}\AgdaSymbol{)\}} \AgdaSymbol{\{}\AgdaBound{ρ} \AgdaSymbol{:} \AgdaFunction{Rep} \AgdaBound{U} \AgdaBound{V}\AgdaSymbol{\}} \AgdaSymbol{→}\<%
\\
\>[0]\AgdaIndent{2}{}\<[2]%
\>[2]\AgdaSymbol{(}\AgdaFunction{x₂:=} \AgdaBound{M2} \AgdaFunction{〈} \AgdaBound{ρ} \AgdaFunction{〉} \AgdaFunction{,x₁:=} \AgdaBound{M1} \AgdaFunction{〈} \AgdaBound{ρ} \AgdaFunction{〉} \AgdaFunction{,x₀:=} \AgdaBound{M0} \AgdaFunction{〈} \AgdaBound{ρ} \AgdaFunction{〉}\AgdaSymbol{)} \AgdaFunction{•SR} \AgdaFunction{liftRep} \AgdaSymbol{\_} \AgdaSymbol{(}\AgdaFunction{liftRep} \AgdaSymbol{\_} \AgdaSymbol{(}\AgdaFunction{liftRep} \AgdaSymbol{\_} \AgdaBound{ρ}\AgdaSymbol{))}\<%
\\
\>[0]\AgdaIndent{2}{}\<[2]%
\>[2]\AgdaFunction{∼} \AgdaBound{ρ} \AgdaFunction{•RS} \AgdaSymbol{(}\AgdaFunction{x₂:=} \AgdaBound{M2} \AgdaFunction{,x₁:=} \AgdaBound{M1} \AgdaFunction{,x₀:=} \AgdaBound{M0}\AgdaSymbol{)}\<%
\\
\>\AgdaFunction{botSub₃-liftRep₃'} \AgdaInductiveConstructor{x₀} \AgdaSymbol{=} \AgdaInductiveConstructor{refl}\<%
\\
\>\AgdaFunction{botSub₃-liftRep₃'} \AgdaSymbol{(}\AgdaInductiveConstructor{↑} \AgdaInductiveConstructor{x₀}\AgdaSymbol{)} \AgdaSymbol{=} \AgdaInductiveConstructor{refl}\<%
\\
\>\AgdaFunction{botSub₃-liftRep₃'} \AgdaSymbol{(}\AgdaInductiveConstructor{↑} \AgdaSymbol{(}\AgdaInductiveConstructor{↑} \AgdaInductiveConstructor{x₀}\AgdaSymbol{))} \AgdaSymbol{=} \AgdaInductiveConstructor{refl} \<[36]%
\>[36]\<%
\\
\>\AgdaFunction{botSub₃-liftRep₃'} \AgdaSymbol{(}\AgdaInductiveConstructor{↑} \AgdaSymbol{(}\AgdaInductiveConstructor{↑} \AgdaSymbol{(}\AgdaInductiveConstructor{↑} \AgdaBound{x}\AgdaSymbol{)))} \AgdaSymbol{=} \AgdaInductiveConstructor{refl}\<%
\\
%
\\
\>\AgdaFunction{botSub₃-liftRep₃} \AgdaSymbol{:} \AgdaSymbol{∀} \AgdaSymbol{\{}\AgdaBound{U}\AgdaSymbol{\}} \AgdaSymbol{\{}\AgdaBound{V}\AgdaSymbol{\}} \AgdaSymbol{\{}\AgdaBound{K2}\AgdaSymbol{\}} \AgdaSymbol{\{}\AgdaBound{K1}\AgdaSymbol{\}} \AgdaSymbol{\{}\AgdaBound{K0}\AgdaSymbol{\}} \AgdaSymbol{\{}\AgdaBound{L}\AgdaSymbol{\}}\<%
\\
\>[0]\AgdaIndent{2}{}\<[2]%
\>[2]\AgdaSymbol{\{}\AgdaBound{M2} \AgdaSymbol{:} \AgdaFunction{Expression} \AgdaBound{U} \AgdaSymbol{(}\AgdaInductiveConstructor{varKind} \AgdaBound{K2}\AgdaSymbol{)\}} \AgdaSymbol{\{}\AgdaBound{M1} \AgdaSymbol{:} \AgdaFunction{Expression} \AgdaBound{U} \AgdaSymbol{(}\AgdaInductiveConstructor{varKind} \AgdaBound{K1}\AgdaSymbol{)\}} \AgdaSymbol{\{}\AgdaBound{M0} \AgdaSymbol{:} \AgdaFunction{Expression} \AgdaBound{U} \AgdaSymbol{(}\AgdaInductiveConstructor{varKind} \AgdaBound{K0}\AgdaSymbol{)\}} \AgdaSymbol{\{}\AgdaBound{ρ} \AgdaSymbol{:} \AgdaFunction{Rep} \AgdaBound{U} \AgdaBound{V}\AgdaSymbol{\}} \AgdaSymbol{(}\AgdaBound{N} \AgdaSymbol{:} \AgdaFunction{Expression} \AgdaSymbol{(}\AgdaBound{U} \AgdaInductiveConstructor{,} \AgdaBound{K2} \AgdaInductiveConstructor{,} \AgdaBound{K1} \AgdaInductiveConstructor{,} \AgdaBound{K0}\AgdaSymbol{)} \AgdaBound{L}\AgdaSymbol{)} \AgdaSymbol{→}\<%
\\
\>[0]\AgdaIndent{2}{}\<[2]%
\>[2]\AgdaBound{N} \AgdaFunction{〈} \AgdaFunction{liftRep} \AgdaSymbol{\_} \AgdaSymbol{(}\AgdaFunction{liftRep} \AgdaSymbol{\_} \AgdaSymbol{(}\AgdaFunction{liftRep} \AgdaSymbol{\_} \AgdaBound{ρ}\AgdaSymbol{))} \AgdaFunction{〉} \AgdaFunction{⟦} \AgdaFunction{x₂:=} \AgdaBound{M2} \AgdaFunction{〈} \AgdaBound{ρ} \AgdaFunction{〉} \AgdaFunction{,x₁:=} \AgdaBound{M1} \AgdaFunction{〈} \AgdaBound{ρ} \AgdaFunction{〉} \AgdaFunction{,x₀:=} \AgdaBound{M0} \AgdaFunction{〈} \AgdaBound{ρ} \AgdaFunction{〉} \AgdaFunction{⟧}\<%
\\
\>[0]\AgdaIndent{2}{}\<[2]%
\>[2]\AgdaDatatype{≡} \AgdaBound{N} \AgdaFunction{⟦} \AgdaFunction{x₂:=} \AgdaBound{M2} \AgdaFunction{,x₁:=} \AgdaBound{M1} \AgdaFunction{,x₀:=} \AgdaBound{M0} \AgdaFunction{⟧} \AgdaFunction{〈} \AgdaBound{ρ} \AgdaFunction{〉}\<%
\\
\>\AgdaFunction{botSub₃-liftRep₃} \AgdaSymbol{\{}\AgdaArgument{M2} \AgdaSymbol{=} \AgdaBound{M2}\AgdaSymbol{\}} \AgdaSymbol{\{}\AgdaBound{M1}\AgdaSymbol{\}} \AgdaSymbol{\{}\AgdaBound{M0}\AgdaSymbol{\}} \AgdaSymbol{\{}\AgdaBound{ρ}\AgdaSymbol{\}} \AgdaBound{N} \AgdaSymbol{=} \AgdaKeyword{let} \AgdaKeyword{open} \AgdaModule{≡-Reasoning} \AgdaKeyword{in}\<%
\\
\>[2]\AgdaIndent{14}{}\<[14]%
\>[14]\AgdaFunction{begin}\<%
\\
\>[14]\AgdaIndent{16}{}\<[16]%
\>[16]\AgdaBound{N} \AgdaFunction{〈} \AgdaFunction{liftRep} \AgdaSymbol{\_} \AgdaSymbol{(}\AgdaFunction{liftRep} \AgdaSymbol{\_} \AgdaSymbol{(}\AgdaFunction{liftRep} \AgdaSymbol{\_} \AgdaBound{ρ}\AgdaSymbol{))} \AgdaFunction{〉} \AgdaFunction{⟦} \AgdaFunction{x₂:=} \AgdaBound{M2} \AgdaFunction{〈} \AgdaBound{ρ} \AgdaFunction{〉} \AgdaFunction{,x₁:=} \AgdaBound{M1} \AgdaFunction{〈} \AgdaBound{ρ} \AgdaFunction{〉} \AgdaFunction{,x₀:=} \AgdaBound{M0} \AgdaFunction{〈} \AgdaBound{ρ} \AgdaFunction{〉} \AgdaFunction{⟧}\<%
\\
\>[0]\AgdaIndent{14}{}\<[14]%
\>[14]\AgdaFunction{≡⟨⟨} \AgdaFunction{sub-compSR} \AgdaBound{N} \AgdaFunction{⟩⟩}\<%
\\
\>[14]\AgdaIndent{16}{}\<[16]%
\>[16]\AgdaBound{N} \AgdaFunction{⟦} \AgdaSymbol{(}\AgdaFunction{x₂:=} \AgdaBound{M2} \AgdaFunction{〈} \AgdaBound{ρ} \AgdaFunction{〉} \AgdaFunction{,x₁:=} \AgdaBound{M1} \AgdaFunction{〈} \AgdaBound{ρ} \AgdaFunction{〉} \AgdaFunction{,x₀:=} \AgdaBound{M0} \AgdaFunction{〈} \AgdaBound{ρ} \AgdaFunction{〉}\AgdaSymbol{)} \AgdaFunction{•SR} \AgdaFunction{liftRep} \AgdaSymbol{\_} \AgdaSymbol{(}\AgdaFunction{liftRep} \AgdaSymbol{\_} \AgdaSymbol{(}\AgdaFunction{liftRep} \AgdaSymbol{\_} \AgdaBound{ρ}\AgdaSymbol{))} \AgdaFunction{⟧}\<%
\\
\>[0]\AgdaIndent{14}{}\<[14]%
\>[14]\AgdaFunction{≡⟨} \AgdaFunction{sub-congr} \AgdaFunction{botSub₃-liftRep₃'} \AgdaBound{N} \AgdaFunction{⟩}\<%
\\
\>[14]\AgdaIndent{16}{}\<[16]%
\>[16]\AgdaBound{N} \AgdaFunction{⟦} \AgdaBound{ρ} \AgdaFunction{•RS} \AgdaSymbol{(}\AgdaFunction{x₂:=} \AgdaBound{M2} \AgdaFunction{,x₁:=} \AgdaBound{M1} \AgdaFunction{,x₀:=} \AgdaBound{M0}\AgdaSymbol{)} \AgdaFunction{⟧}\<%
\\
\>[0]\AgdaIndent{14}{}\<[14]%
\>[14]\AgdaFunction{≡⟨} \AgdaFunction{sub-compRS} \AgdaBound{N} \AgdaFunction{⟩}\<%
\\
\>[14]\AgdaIndent{16}{}\<[16]%
\>[16]\AgdaBound{N} \AgdaFunction{⟦} \AgdaFunction{x₂:=} \AgdaBound{M2} \AgdaFunction{,x₁:=} \AgdaBound{M1} \AgdaFunction{,x₀:=} \AgdaBound{M0} \AgdaFunction{⟧} \AgdaFunction{〈} \AgdaBound{ρ} \AgdaFunction{〉}\<%
\\
\>[0]\AgdaIndent{14}{}\<[14]%
\>[14]\AgdaFunction{∎}\<%
\\
\>\AgdaComment{--TODO General lemma for this}\<%
\\
\>\AgdaComment{--TODO Deletable?}\<%
\end{code}
}


\end{document}
