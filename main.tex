\usepackage{amsmath}
\usepackage{amssymb}
\usepackage{bbm}
\usepackage[greek,english]{babel}
\usepackage{ucs}
\usepackage[utf8x]{inputenc}
\usepackage{autofe}
\usepackage{fancyvrb}
\usepackage{proof}
\usepackage{stmaryrd}

\DeclareUnicodeCharacter{8598}{\ensuremath{\nwarrow}}
\DeclareUnicodeCharacter{8599}{\ensuremath{\nearrow}}
\DeclareUnicodeCharacter{8608}{\ensuremath{\twoheadrightarrow}}
\DeclareUnicodeCharacter{8657}{\ensuremath{\Uparrow}}
\DeclareUnicodeCharacter{8667}{\ensuremath{\Rrightarrow}}
\DeclareUnicodeCharacter{8718}{\ensuremath{\qed}}
\DeclareUnicodeCharacter{8759}{\ensuremath{::}}
\DeclareUnicodeCharacter{8988}{\ensuremath{\ulcorner}}
\DeclareUnicodeCharacter{8989}{\ensuremath{\urcorner}}
\DeclareUnicodeCharacter{8803}{\ensuremath{\overline{\equiv}}}
\DeclareUnicodeCharacter{9001}{\ensuremath{\langle}}
\DeclareUnicodeCharacter{9002}{\ensuremath{\rangle}}
\DeclareUnicodeCharacter{9655}{\ensuremath{\rhd}}
\DeclareUnicodeCharacter{10214}{\ensuremath{[}}
\DeclareUnicodeCharacter{10215}{\ensuremath{]}}
\DeclareUnicodeCharacter{10219}{\ensuremath{\rangle\rangle}}


\usepackage{textalpha}

\DefineVerbatimEnvironment{code}{Verbatim}{fontsize=\small}

% \newtheorem{lemma}{Lemma}[section]
% \newtheorem{corollary}[lemma]{Corollary}
\newtheorem{prop}{Proposition}[section]
% \newtheorem{theorem}{Theorem}[section]
% \theoremstyle{definition}
% \newtheorem{definition}[lemma]{Definition}

\newcommand{\Set}{\mathbf{Set}}
\newcommand{\eqdef}{\mathrel{\smash{\stackrel{\text{def}}{=}}}}
\newcommand{\AgdaHide}[1]{}
\newcommand{\isotoid}{\ensuremath{isotoid}}
\newcommand{\vald}{\ensuremath{\ \mathrm{valid}}}
\newcommand{\reff}[1]{\ensuremath{\mathsf{ref} \left( {#1} \right)}}
\newcommand{\univ}[4]{\ensuremath{\mathsf{univ}_{{#1} , {#2}} \left( {#3} , {#4} \right)}}
\newcommand{\triplelambda}{\lambda \!\! \lambda \!\! \lambda}
\newcommand{\SN}{\mathbf{SN}}

\begin{document}

\begin{frame}
\maketitle

\mode<beamer>{
\begin{small}
\begin{center}
This talk is a literate Agda file
\texttt{https://github.com/radams78/Univalence}
\end{center}
\end{small}
}

%TODO Full author and conference information

\end{frame}

\section{Introduction}

\begin{frame}
\frametitle{The problem}
\begin{itemize}
\item
A type theory should enjoy three properties:
\begin{description}
\item[Canonicity] Every well-typed term of type $A$ reduces to a canonical form of $A$.
\item[Confluence] Reduction is confluent.  (Therefore, the canonical form is unique.)
\item[Strong Normalization] Every reduction strategy terminates.
\end{description}
\item
Every term of type $\mathbb{N}$ reduces to a unique numeral.
\item 
The \emph{univalence axiom} postulates a function
\[ \isotoid : A \simeq B \rightarrow A = B \]
that is an inverse to the obvious function $A = B \rightarrow A \simeq B$.
\item
This breaks canonicity.
\end{itemize}
\end{frame}
%TODO Define canonical form

\begin{frame}
\frametitle{Possible Solutions}
\begin{itemize}
\item
Lower our standards (propositional canonicity).
\item
Use a type theory in which $\isotoid$ is definable (e.g. Cubical Type Theory, Polonsky) %TODO Cite
\item
Introduce a reduction rule for $\isotoid$.
\end{itemize}
\end{frame}
%TODO Look up other possible solutions.

\begin{frame}
\frametitle{Our Approach}
We begin with a small type theory, and work our way up to the full HoTT. %Too much?
\begin{enumerate}
\item \emph{Predicative Higher-Order Propositional Logic} A type theory with:
  \begin{itemize}
  \item a universe $\Omega$ of \emph{propositions} with $\bot$ and $\supset$
  \item a universe $U$ of \emph{small types} with $\Omega$ and $\rightarrow$
  \item for any two terms $M, N : A$, a (large) type $M =_A N$.
  \end{itemize}
\item \emph{PHOPL with Equality}
Make $\delta =_\phi \epsilon$ a proposition.  (So we can form propositions like $M =_A N \epsilon \supset N =_A M$.)
\item \emph{Predicative Higher-Order Predicate Logic} Close $\Omega$ under $\forall$.  (So we can form propositions like $\forall x : \phi. x =_A x$.)
\end{enumerate}
For the future: natural numbers, inductive types, path elimination, \ldots
\end{frame}

\section{About the Formalization}

\begin{frame}
\frametitle{About the Formalization}
This work is being formalized in Agda (work in progress).

\begin{itemize}
\item
General definition of a grammar, a reduction rule.
\item
Expressions are divided into \emph{kinds} (e.g. terms, types).
\item
An \emph{alphabet} provides a finite set of variables of each kind.
\item
\texttt{Expression V K} is the type of expressions of kind $K$ with variables in $V$.
\end{itemize}

Follow the progress here: \url{www.github.com/radams78/Univalence}
\end{frame}

\AgdaHide{
\begin{code}%
\>\AgdaKeyword{module} \AgdaModule{Prelims} \AgdaKeyword{where}\<%
\\
\>\AgdaKeyword{open} \AgdaKeyword{import} \AgdaModule{Relation.Binary.PropositionalEquality}\<%
\\
\>\AgdaKeyword{open} \AgdaKeyword{import} \AgdaModule{Data.List}\<%
\\
\>\AgdaKeyword{open} \AgdaKeyword{import} \AgdaModule{Prelims.Bifunction} \AgdaKeyword{public}\<%
\\
\>\AgdaKeyword{open} \AgdaKeyword{import} \AgdaModule{Prelims.EqReasoning} \AgdaKeyword{public}\<%
\\
\>\AgdaKeyword{open} \AgdaKeyword{import} \AgdaModule{Prelims.Snoclist} \AgdaKeyword{public}\<%
\end{code}
}

\newcommand{\id}[1]{\mathsf{id}_{#1}}

\section{Grammars}

\subsection{Taxonomy}

\AgdaHide{
\begin{code}%
\>\AgdaKeyword{module} \AgdaModule{Grammar.Taxonomy} \AgdaKeyword{where}\<%
\\
\>\AgdaKeyword{open} \AgdaKeyword{import} \AgdaModule{Data.List} \AgdaKeyword{public}\<%
\\
\>\AgdaKeyword{open} \AgdaKeyword{import} \AgdaModule{Prelims}\<%
\end{code}
}

Before we begin investigating the several theories we wish to consider, we present a general theory of syntax and
capture-avoiding substitution.

A \emph{taxononmy} consists of:
\begin{itemize}
\item a set of \emph{expression kinds};
\item a subset of expression kinds, called the \emph{variable kinds}.  We refer to the other expession kinds as \emph{non-variable kinds}.
\end{itemize}

%<*Taxonomy>
\begin{code}%
\>\AgdaKeyword{record} \AgdaRecord{Taxonomy} \AgdaSymbol{:} \AgdaPrimitiveType{Set₁} \AgdaKeyword{where}\<%
\\
\>[0]\AgdaIndent{2}{}\<[2]%
\>[2]\AgdaKeyword{field}\<%
\\
\>[2]\AgdaIndent{4}{}\<[4]%
\>[4]\AgdaField{VarKind} \AgdaSymbol{:} \AgdaPrimitiveType{Set}\<%
\\
\>[2]\AgdaIndent{4}{}\<[4]%
\>[4]\AgdaField{NonVarKind} \AgdaSymbol{:} \AgdaPrimitiveType{Set}\<%
\\
%
\\
\>[0]\AgdaIndent{2}{}\<[2]%
\>[2]\AgdaKeyword{data} \AgdaDatatype{ExpressionKind} \AgdaSymbol{:} \AgdaPrimitiveType{Set} \AgdaKeyword{where}\<%
\\
\>[2]\AgdaIndent{4}{}\<[4]%
\>[4]\AgdaInductiveConstructor{varKind} \AgdaSymbol{:} \AgdaField{VarKind} \AgdaSymbol{→} \AgdaDatatype{ExpressionKind}\<%
\\
\>[2]\AgdaIndent{4}{}\<[4]%
\>[4]\AgdaInductiveConstructor{nonVarKind} \AgdaSymbol{:} \AgdaField{NonVarKind} \AgdaSymbol{→} \AgdaDatatype{ExpressionKind}\<%
\\
\>\AgdaComment{--TODO: Maybe get rid of NonVarKind, and just have parent : VarKind -> Set?}\<%
\end{code}
%</Taxonomy>

\begin{frame}[fragile]
\frametitle{Alphabets}
An \emph{alphabet} $A$ consists of a finite set of \emph{variables},
\mode<article>{to each of which is assigned a variable kind $K$.
Let $\emptyset$ be the empty alphabet, and $(A , K)$ be the result of extending the alphabet $A$ with one
fresh variable $x₀$ of kind $K$.  We write $\mathsf{Var}\ A\ K$ for the set of all variables in $A$ of kind $K$.}
\mode<beamer>{each with a variable kind.}

\begin{code}%
\>[0]\AgdaIndent{2}{}\<[2]%
\>[2]\AgdaKeyword{infixl} \AgdaNumber{55} \AgdaFixityOp{\_,\_}\<%
\\
\>[0]\AgdaIndent{2}{}\<[2]%
\>[2]\AgdaKeyword{data} \AgdaDatatype{Alphabet} \AgdaSymbol{:} \AgdaPrimitiveType{Set} \AgdaKeyword{where}\<%
\\
\>[2]\AgdaIndent{4}{}\<[4]%
\>[4]\AgdaInductiveConstructor{∅} \AgdaSymbol{:} \AgdaDatatype{Alphabet}\<%
\\
\>[2]\AgdaIndent{4}{}\<[4]%
\>[4]\AgdaInductiveConstructor{\_,\_} \AgdaSymbol{:} \AgdaDatatype{Alphabet} \AgdaSymbol{→} \AgdaField{VarKind} \AgdaSymbol{→} \AgdaDatatype{Alphabet}\<%
\end{code}

\AgdaHide{
\begin{code}%
\>[0]\AgdaIndent{2}{}\<[2]%
\>[2]\AgdaFunction{extend} \AgdaSymbol{:} \AgdaDatatype{Alphabet} \AgdaSymbol{→} \AgdaDatatype{List} \AgdaField{VarKind} \AgdaSymbol{→} \AgdaDatatype{Alphabet}\<%
\\
\>[0]\AgdaIndent{2}{}\<[2]%
\>[2]\AgdaFunction{extend} \AgdaBound{A} \AgdaInductiveConstructor{[]} \AgdaSymbol{=} \AgdaBound{A}\<%
\\
\>[0]\AgdaIndent{2}{}\<[2]%
\>[2]\AgdaFunction{extend} \AgdaBound{A} \AgdaSymbol{(}\AgdaBound{K} \AgdaInductiveConstructor{∷} \AgdaBound{KK}\AgdaSymbol{)} \AgdaSymbol{=} \AgdaFunction{extend} \AgdaSymbol{(}\AgdaBound{A} \AgdaInductiveConstructor{,} \AgdaBound{K}\AgdaSymbol{)} \AgdaBound{KK}\<%
\\
%
\\
\>[0]\AgdaIndent{2}{}\<[2]%
\>[2]\AgdaFunction{snoc-extend} \AgdaSymbol{:} \AgdaDatatype{Alphabet} \AgdaSymbol{→} \AgdaDatatype{snocList} \AgdaField{VarKind} \AgdaSymbol{→} \AgdaDatatype{Alphabet}\<%
\\
\>[0]\AgdaIndent{2}{}\<[2]%
\>[2]\AgdaFunction{snoc-extend} \AgdaBound{A} \AgdaInductiveConstructor{[]} \AgdaSymbol{=} \AgdaBound{A}\<%
\\
\>[0]\AgdaIndent{2}{}\<[2]%
\>[2]\AgdaFunction{snoc-extend} \AgdaBound{A} \AgdaSymbol{(}\AgdaBound{KK} \AgdaInductiveConstructor{snoc} \AgdaBound{K}\AgdaSymbol{)} \AgdaSymbol{=} \AgdaFunction{snoc-extend} \AgdaBound{A} \AgdaBound{KK} \AgdaInductiveConstructor{,} \AgdaBound{K}\<%
\end{code}
}

\begin{code}%
\>[0]\AgdaIndent{2}{}\<[2]%
\>[2]\AgdaKeyword{data} \AgdaDatatype{Var} \AgdaSymbol{:} \AgdaDatatype{Alphabet} \AgdaSymbol{→} \AgdaField{VarKind} \AgdaSymbol{→} \AgdaPrimitiveType{Set} \AgdaKeyword{where}\<%
\\
\>[2]\AgdaIndent{4}{}\<[4]%
\>[4]\AgdaInductiveConstructor{x₀} \AgdaSymbol{:} \AgdaSymbol{∀} \AgdaSymbol{\{}\AgdaBound{V}\AgdaSymbol{\}} \AgdaSymbol{\{}\AgdaBound{K}\AgdaSymbol{\}} \AgdaSymbol{→} \AgdaDatatype{Var} \AgdaSymbol{(}\AgdaBound{V} \AgdaInductiveConstructor{,} \AgdaBound{K}\AgdaSymbol{)} \AgdaBound{K}\<%
\\
\>[2]\AgdaIndent{4}{}\<[4]%
\>[4]\AgdaInductiveConstructor{↑} \AgdaSymbol{:} \AgdaSymbol{∀} \AgdaSymbol{\{}\AgdaBound{V}\AgdaSymbol{\}} \AgdaSymbol{\{}\AgdaBound{K}\AgdaSymbol{\}} \AgdaSymbol{\{}\AgdaBound{L}\AgdaSymbol{\}} \AgdaSymbol{→} \AgdaDatatype{Var} \AgdaBound{V} \AgdaBound{L} \AgdaSymbol{→} \AgdaDatatype{Var} \AgdaSymbol{(}\AgdaBound{V} \AgdaInductiveConstructor{,} \AgdaBound{K}\AgdaSymbol{)} \AgdaBound{L}\<%
\end{code}

\AgdaHide{
\begin{code}%
\>[0]\AgdaIndent{2}{}\<[2]%
\>[2]\AgdaFunction{x₁} \AgdaSymbol{:} \AgdaSymbol{∀} \AgdaSymbol{\{}\AgdaBound{V}\AgdaSymbol{\}} \AgdaSymbol{\{}\AgdaBound{K}\AgdaSymbol{\}} \AgdaSymbol{\{}\AgdaBound{L}\AgdaSymbol{\}} \AgdaSymbol{→} \AgdaDatatype{Var} \AgdaSymbol{(}\AgdaBound{V} \AgdaInductiveConstructor{,} \AgdaBound{K} \AgdaInductiveConstructor{,} \AgdaBound{L}\AgdaSymbol{)} \AgdaBound{K}\<%
\end{code}
}

\begin{code}%
\>[0]\AgdaIndent{2}{}\<[2]%
\>[2]\AgdaFunction{x₁} \AgdaSymbol{=} \AgdaInductiveConstructor{↑} \AgdaInductiveConstructor{x₀}\<%
\end{code}

\AgdaHide{
\begin{code}%
\>[0]\AgdaIndent{2}{}\<[2]%
\>[2]\AgdaFunction{x₂} \AgdaSymbol{:} \AgdaSymbol{∀} \AgdaSymbol{\{}\AgdaBound{V}\AgdaSymbol{\}} \AgdaSymbol{\{}\AgdaBound{K}\AgdaSymbol{\}} \AgdaSymbol{\{}\AgdaBound{L}\AgdaSymbol{\}} \AgdaSymbol{\{}\AgdaBound{L'}\AgdaSymbol{\}} \AgdaSymbol{→} \AgdaDatatype{Var} \AgdaSymbol{(}\AgdaBound{V} \AgdaInductiveConstructor{,} \AgdaBound{K} \AgdaInductiveConstructor{,} \AgdaBound{L} \AgdaInductiveConstructor{,} \AgdaBound{L'}\AgdaSymbol{)} \AgdaBound{K}\<%
\end{code}
}

\begin{code}%
\>[0]\AgdaIndent{2}{}\<[2]%
\>[2]\AgdaFunction{x₂} \AgdaSymbol{=} \AgdaInductiveConstructor{↑} \AgdaFunction{x₁}\<%
\end{code}
\end{frame}

A constructor $c$ of kind (\ref{eq:conkind}) is a constructor that takes $m$ arguments of kind $B_1$, \ldots, $B_m$, and binds $r_i$ variables in its $i$th argument of kind $A_{ij}$,
producing an expression of kind $C$.  We write this expression as

\begin{equation}
\label{eq:expression}
c([x_{11}, \ldots, x_{1r_1}]E_1, \ldots, [x_{m1}, \ldots, x_{mr_m}]E_m) \enspace .
\end{equation}

The subexpressions of the form $[x_{i1}, \ldots, x_{ir_i}]E_i$ shall be called \emph{abstractions}.

When giving a specific grammar, we shall feel free to use BNF notation.  

We formalise this as follows.  First, we construct the sets of expression kinds and constructor kinds over a taxonomy:

\begin{frame}[fragile]
There are two \emph{classes} of kinds: expression kinds and constructor kinds.

\begin{code}%
\>[0]\AgdaIndent{2}{}\<[2]%
\>[2]\AgdaKeyword{infix} \AgdaNumber{22} \AgdaFixityOp{\_✧}\<%
\\
\>[0]\AgdaIndent{2}{}\<[2]%
\>[2]\AgdaKeyword{infixr} \AgdaNumber{21} \AgdaFixityOp{\_abs\_}\<%
\\
\>[0]\AgdaIndent{2}{}\<[2]%
\>[2]\AgdaKeyword{data} \AgdaDatatype{AbstractionKind} \AgdaSymbol{:} \AgdaDatatype{ExpressionKind} \AgdaSymbol{→} \AgdaPrimitiveType{Set} \AgdaKeyword{where}\<%
\\
\>[2]\AgdaIndent{4}{}\<[4]%
\>[4]\AgdaInductiveConstructor{\_✧} \AgdaSymbol{:} \AgdaSymbol{∀} \AgdaBound{K} \AgdaSymbol{→} \AgdaDatatype{AbstractionKind} \AgdaBound{K}\<%
\\
\>[2]\AgdaIndent{4}{}\<[4]%
\>[4]\AgdaInductiveConstructor{\_abs\_} \AgdaSymbol{:} \AgdaSymbol{∀} \AgdaSymbol{\{}\AgdaBound{K}\AgdaSymbol{\}} \AgdaSymbol{→} \AgdaField{VarKind} \AgdaSymbol{→} \AgdaDatatype{AbstractionKind} \AgdaBound{K} \AgdaSymbol{→} \AgdaDatatype{AbstractionKind} \AgdaBound{K}\<%
\\
%
\\
\>[0]\AgdaIndent{2}{}\<[2]%
\>[2]\AgdaKeyword{infix} \AgdaNumber{22} \AgdaFixityOp{\_●}\<%
\\
\>[0]\AgdaIndent{2}{}\<[2]%
\>[2]\AgdaKeyword{infixr} \AgdaNumber{20} \AgdaFixityOp{\_⟶\_}\<%
\\
\>[0]\AgdaIndent{2}{}\<[2]%
\>[2]\AgdaKeyword{data} \AgdaDatatype{ConstructorKind} \AgdaSymbol{:} \AgdaDatatype{ExpressionKind} \AgdaSymbol{→} \AgdaPrimitiveType{Set} \AgdaKeyword{where}\<%
\\
\>[2]\AgdaIndent{4}{}\<[4]%
\>[4]\AgdaInductiveConstructor{\_●} \AgdaSymbol{:} \AgdaSymbol{∀} \AgdaBound{K} \AgdaSymbol{→} \AgdaDatatype{ConstructorKind} \AgdaBound{K}\<%
\\
\>[2]\AgdaIndent{4}{}\<[4]%
\>[4]\AgdaInductiveConstructor{\_⟶\_} \AgdaSymbol{:} \AgdaSymbol{∀} \AgdaSymbol{\{}\AgdaBound{L}\AgdaSymbol{\}} \AgdaSymbol{\{}\AgdaBound{K}\AgdaSymbol{\}} \AgdaSymbol{→} \AgdaDatatype{AbstractionKind} \AgdaBound{L} \AgdaSymbol{→} \AgdaDatatype{ConstructorKind} \AgdaBound{K} \AgdaSymbol{→} \AgdaDatatype{ConstructorKind} \AgdaBound{K}\<%
\\
%
\\
\>[0]\AgdaIndent{2}{}\<[2]%
\>[2]\AgdaKeyword{data} \AgdaDatatype{KindClass} \AgdaSymbol{:} \AgdaPrimitiveType{Set} \AgdaKeyword{where}\<%
\\
\>[2]\AgdaIndent{4}{}\<[4]%
\>[4]\AgdaInductiveConstructor{-Expression} \AgdaSymbol{:} \AgdaDatatype{KindClass}\<%
\\
\>[2]\AgdaIndent{4}{}\<[4]%
\>[4]\AgdaInductiveConstructor{-Constructor} \AgdaSymbol{:} \AgdaDatatype{ExpressionKind} \AgdaSymbol{→} \AgdaDatatype{KindClass}\<%
\\
%
\\
\>[0]\AgdaIndent{2}{}\<[2]%
\>[2]\AgdaFunction{Kind} \AgdaSymbol{:} \AgdaDatatype{KindClass} \AgdaSymbol{→} \AgdaPrimitiveType{Set}\<%
\\
\>[0]\AgdaIndent{2}{}\<[2]%
\>[2]\AgdaFunction{Kind} \AgdaInductiveConstructor{-Expression} \AgdaSymbol{=} \AgdaDatatype{ExpressionKind}\<%
\\
\>[0]\AgdaIndent{2}{}\<[2]%
\>[2]\AgdaFunction{Kind} \AgdaSymbol{(}\AgdaInductiveConstructor{-Constructor} \AgdaBound{K}\AgdaSymbol{)} \AgdaSymbol{=} \AgdaDatatype{ConstructorKind} \AgdaBound{K}\<%
\end{code}
\end{frame}
%TODO Colours in Agda code?

\AgdaHide{
\begin{code}%
\>\AgdaComment{\{- Metavariable conventions:\<\\
\>  A, B    range over abstraction kinds\<\\
\>  C       range over kind classes\<\\
\>  AA, BB  range over lists of abstraction kinds\<\\
\>  E, F, G range over subexpressions\<\\
\>  K, L    range over expression kinds including variable kinds\<\\
\>  M, N, P range over expressions\<\\
\>  U, V, W range over alphabets -\}}\<%
\\
\>\AgdaKeyword{open} \AgdaKeyword{import} \AgdaModule{Function}\<%
\\
\>\AgdaKeyword{open} \AgdaKeyword{import} \AgdaModule{Data.List}\<%
\\
\>\AgdaKeyword{open} \AgdaKeyword{import} \AgdaModule{Prelims}\<%
\\
\>\AgdaKeyword{open} \AgdaKeyword{import} \AgdaModule{Grammar.Taxonomy}\<%
\\
%
\\
\>\AgdaKeyword{module} \AgdaModule{Grammar.Base} \AgdaKeyword{where}\<%
\\
%
\\
\>\AgdaKeyword{record} \AgdaRecord{IsGrammar} \AgdaSymbol{(}\AgdaBound{T} \AgdaSymbol{:} \AgdaRecord{Taxonomy}\AgdaSymbol{)} \AgdaSymbol{:} \AgdaPrimitiveType{Set₁} \AgdaKeyword{where}\<%
\\
\>[0]\AgdaIndent{2}{}\<[2]%
\>[2]\AgdaKeyword{open} \AgdaModule{Taxonomy} \AgdaBound{T}\<%
\\
\>[0]\AgdaIndent{2}{}\<[2]%
\>[2]\AgdaKeyword{field}\<%
\\
\>[2]\AgdaIndent{4}{}\<[4]%
\>[4]\AgdaField{Constructor} \<[19]%
\>[19]\AgdaSymbol{:} \AgdaFunction{ConKind} \AgdaSymbol{→} \AgdaPrimitiveType{Set}\<%
\\
\>[2]\AgdaIndent{4}{}\<[4]%
\>[4]\AgdaField{parent} \<[19]%
\>[19]\AgdaSymbol{:} \AgdaFunction{VarKind} \AgdaSymbol{→} \AgdaDatatype{ExpKind}\<%
\\
%
\\
\>\AgdaKeyword{record} \AgdaRecord{Grammar} \AgdaSymbol{:} \AgdaPrimitiveType{Set₁} \AgdaKeyword{where}\<%
\\
\>[0]\AgdaIndent{2}{}\<[2]%
\>[2]\AgdaKeyword{field}\<%
\\
\>[2]\AgdaIndent{4}{}\<[4]%
\>[4]\AgdaField{taxonomy} \AgdaSymbol{:} \AgdaRecord{Taxonomy}\<%
\\
\>[2]\AgdaIndent{4}{}\<[4]%
\>[4]\AgdaField{isGrammar} \AgdaSymbol{:} \AgdaRecord{IsGrammar} \AgdaField{taxonomy}\<%
\\
\>[0]\AgdaIndent{2}{}\<[2]%
\>[2]\AgdaKeyword{open} \AgdaModule{Taxonomy} \AgdaField{taxonomy} \AgdaKeyword{public}\<%
\\
\>[0]\AgdaIndent{2}{}\<[2]%
\>[2]\AgdaKeyword{open} \AgdaModule{IsGrammar} \AgdaField{isGrammar} \AgdaKeyword{public}\<%
\end{code}
}

%<*Expression>
\begin{code}%
\>[0]\AgdaIndent{2}{}\<[2]%
\>[2]\AgdaKeyword{data} \AgdaDatatype{Subexpression} \AgdaSymbol{(}\AgdaBound{V} \AgdaSymbol{:} \AgdaDatatype{Alphabet}\AgdaSymbol{)} \AgdaSymbol{:} \AgdaSymbol{∀} \AgdaBound{C} \AgdaSymbol{→} \AgdaFunction{Kind} \AgdaBound{C} \AgdaSymbol{→} \AgdaPrimitiveType{Set}\<%
\\
\>[0]\AgdaIndent{2}{}\<[2]%
\>[2]\AgdaFunction{Expression} \AgdaSymbol{:} \AgdaDatatype{Alphabet} \AgdaSymbol{→} \AgdaDatatype{ExpKind} \AgdaSymbol{→} \AgdaPrimitiveType{Set}\<%
\\
\>[0]\AgdaIndent{2}{}\<[2]%
\>[2]\AgdaFunction{VExpression} \AgdaSymbol{:} \AgdaDatatype{Alphabet} \AgdaSymbol{→} \AgdaFunction{VarKind} \AgdaSymbol{→} \AgdaPrimitiveType{Set}\<%
\\
\>[0]\AgdaIndent{2}{}\<[2]%
\>[2]\AgdaFunction{Abstraction} \AgdaSymbol{:} \AgdaDatatype{Alphabet} \AgdaSymbol{→} \AgdaFunction{AbsKind} \AgdaSymbol{→} \AgdaPrimitiveType{Set}\<%
\\
\>[0]\AgdaIndent{2}{}\<[2]%
\>[2]\AgdaFunction{ListAbs} \AgdaSymbol{:} \AgdaDatatype{Alphabet} \AgdaSymbol{→} \AgdaDatatype{List} \AgdaFunction{AbsKind} \AgdaSymbol{→} \AgdaPrimitiveType{Set}\<%
\\
%
\\
\>[0]\AgdaIndent{2}{}\<[2]%
\>[2]\AgdaFunction{Expression} \AgdaBound{V} \AgdaBound{K} \AgdaSymbol{=} \AgdaDatatype{Subexpression} \AgdaBound{V} \AgdaInductiveConstructor{-Expression} \AgdaBound{K}\<%
\\
\>[0]\AgdaIndent{2}{}\<[2]%
\>[2]\AgdaFunction{VExpression} \AgdaBound{V} \AgdaBound{K} \AgdaSymbol{=} \AgdaFunction{Expression} \AgdaBound{V} \AgdaSymbol{(}\AgdaInductiveConstructor{varKind} \AgdaBound{K}\AgdaSymbol{)}\<%
\\
\>[0]\AgdaIndent{2}{}\<[2]%
\>[2]\AgdaFunction{Abstraction} \AgdaBound{V} \AgdaSymbol{(}\AgdaInductiveConstructor{SK} \AgdaBound{AA} \AgdaBound{K}\AgdaSymbol{)} \AgdaSymbol{=} \AgdaFunction{Expression} \AgdaSymbol{(}\AgdaFunction{extend} \AgdaBound{V} \AgdaBound{AA}\AgdaSymbol{)} \AgdaBound{K}\<%
\\
\>[0]\AgdaIndent{2}{}\<[2]%
\>[2]\AgdaFunction{ListAbs} \AgdaBound{V} \AgdaBound{AA} \AgdaSymbol{=} \AgdaDatatype{Subexpression} \AgdaBound{V} \AgdaInductiveConstructor{-ListAbs} \AgdaBound{AA}\<%
\\
%
\\
\>[0]\AgdaIndent{2}{}\<[2]%
\>[2]\AgdaKeyword{infixr} \AgdaNumber{5} \AgdaFixityOp{\_∷\_}\<%
\\
\>[0]\AgdaIndent{2}{}\<[2]%
\>[2]\AgdaKeyword{data} \AgdaDatatype{Subexpression} \AgdaBound{V} \AgdaKeyword{where}\<%
\\
\>[2]\AgdaIndent{4}{}\<[4]%
\>[4]\AgdaInductiveConstructor{var} \AgdaSymbol{:} \AgdaSymbol{∀} \AgdaSymbol{\{}\AgdaBound{K}\AgdaSymbol{\}} \AgdaSymbol{→} \AgdaDatatype{Var} \AgdaBound{V} \AgdaBound{K} \AgdaSymbol{→} \AgdaFunction{VExpression} \AgdaBound{V} \AgdaBound{K}\<%
\\
\>[2]\AgdaIndent{4}{}\<[4]%
\>[4]\AgdaInductiveConstructor{app} \AgdaSymbol{:} \AgdaSymbol{∀} \AgdaSymbol{\{}\AgdaBound{AA}\AgdaSymbol{\}} \AgdaSymbol{\{}\AgdaBound{K}\AgdaSymbol{\}} \AgdaSymbol{→} \AgdaFunction{Constructor} \AgdaSymbol{(}\AgdaInductiveConstructor{SK} \AgdaBound{AA} \AgdaBound{K}\AgdaSymbol{)} \AgdaSymbol{→} \AgdaFunction{ListAbs} \AgdaBound{V} \AgdaBound{AA} \AgdaSymbol{→} \AgdaFunction{Expression} \AgdaBound{V} \AgdaBound{K}\<%
\\
\>[2]\AgdaIndent{4}{}\<[4]%
\>[4]\AgdaInductiveConstructor{[]} \AgdaSymbol{:} \AgdaFunction{ListAbs} \AgdaBound{V} \AgdaInductiveConstructor{[]}\<%
\\
\>[2]\AgdaIndent{4}{}\<[4]%
\>[4]\AgdaInductiveConstructor{\_∷\_} \AgdaSymbol{:} \AgdaSymbol{∀} \AgdaSymbol{\{}\AgdaBound{A}\AgdaSymbol{\}} \AgdaSymbol{\{}\AgdaBound{AA}\AgdaSymbol{\}} \AgdaSymbol{→} \AgdaFunction{Abstraction} \AgdaBound{V} \AgdaBound{A} \AgdaSymbol{→} \AgdaFunction{ListAbs} \AgdaBound{V} \AgdaBound{AA} \AgdaSymbol{→} \AgdaFunction{ListAbs} \AgdaBound{V} \AgdaSymbol{(}\AgdaBound{A} \AgdaInductiveConstructor{∷} \AgdaBound{AA}\AgdaSymbol{)}\<%
\end{code}
%</Expression>

We prove that the constructor \AgdaRef{var} is injective.

\begin{code}%
\>[0]\AgdaIndent{2}{}\<[2]%
\>[2]\AgdaFunction{var-inj} \AgdaSymbol{:} \AgdaSymbol{∀} \AgdaSymbol{\{}\AgdaBound{V}\AgdaSymbol{\}} \AgdaSymbol{\{}\AgdaBound{K}\AgdaSymbol{\}} \AgdaSymbol{\{}\AgdaBound{x} \AgdaBound{y} \AgdaSymbol{:} \AgdaDatatype{Var} \AgdaBound{V} \AgdaBound{K}\AgdaSymbol{\}} \AgdaSymbol{→} \AgdaInductiveConstructor{var} \AgdaBound{x} \AgdaDatatype{≡} \AgdaInductiveConstructor{var} \AgdaBound{y} \AgdaSymbol{→} \AgdaBound{x} \AgdaDatatype{≡} \AgdaBound{y}\<%
\\
\>[0]\AgdaIndent{2}{}\<[2]%
\>[2]\AgdaFunction{var-inj} \AgdaInductiveConstructor{refl} \AgdaSymbol{=} \AgdaInductiveConstructor{refl}\<%
\end{code}

For the future, we also define the type of all snoc-lists of expressions $(M_1, \ldots, M_n)$
such that $M_i$ is of type $K_i$, given a snoc-list of variable kinds $(K_1, \ldots, K_n)$.

\begin{code}%
\>[0]\AgdaIndent{2}{}\<[2]%
\>[2]\AgdaKeyword{infixl} \AgdaNumber{20} \AgdaFixityOp{\_snoc\_}\<%
\\
\>[0]\AgdaIndent{2}{}\<[2]%
\>[2]\AgdaKeyword{data} \AgdaDatatype{snocListExp} \AgdaBound{V} \AgdaSymbol{:} \AgdaDatatype{snocList} \AgdaFunction{VarKind} \AgdaSymbol{→} \AgdaPrimitiveType{Set} \AgdaKeyword{where}\<%
\\
\>[2]\AgdaIndent{4}{}\<[4]%
\>[4]\AgdaInductiveConstructor{[]} \AgdaSymbol{:} \AgdaDatatype{snocListExp} \AgdaBound{V} \AgdaInductiveConstructor{[]}\<%
\\
\>[2]\AgdaIndent{4}{}\<[4]%
\>[4]\AgdaInductiveConstructor{\_snoc\_} \AgdaSymbol{:} \AgdaSymbol{∀} \AgdaSymbol{\{}\AgdaBound{A}\AgdaSymbol{\}} \AgdaSymbol{\{}\AgdaBound{K}\AgdaSymbol{\}} \AgdaSymbol{→} \AgdaDatatype{snocListExp} \AgdaBound{V} \AgdaBound{A} \AgdaSymbol{→} \AgdaFunction{Expression} \AgdaBound{V} \AgdaSymbol{(}\AgdaInductiveConstructor{varKind} \AgdaBound{K}\AgdaSymbol{)} \AgdaSymbol{→} \AgdaDatatype{snocListExp} \AgdaBound{V} \AgdaSymbol{(}\AgdaBound{A} \AgdaInductiveConstructor{snoc} \AgdaBound{K}\AgdaSymbol{)}\<%
\end{code}

A \emph{reduction} is a relation $\rhd$ between expressions such that, if $E \rhd F$,
then $E$ is not a variable.  It is given by a term $R : \AgdaRef{Reduction}$ such that
$R\, c\, MM\, N$ iff $c[MM] \rhd N$.

\begin{code}%
\>[0]\AgdaIndent{2}{}\<[2]%
\>[2]\AgdaFunction{Reduction} \AgdaSymbol{:} \AgdaPrimitiveType{Set₁}\<%
\\
\>[0]\AgdaIndent{2}{}\<[2]%
\>[2]\AgdaFunction{Reduction} \AgdaSymbol{=} \AgdaSymbol{∀} \AgdaSymbol{\{}\AgdaBound{V}\AgdaSymbol{\}} \AgdaSymbol{\{}\AgdaBound{AA}\AgdaSymbol{\}} \AgdaSymbol{\{}\AgdaBound{K}\AgdaSymbol{\}} \AgdaSymbol{→} \AgdaFunction{Constructor} \AgdaSymbol{(}\AgdaInductiveConstructor{SK} \AgdaBound{AA} \AgdaBound{K}\AgdaSymbol{)} \AgdaSymbol{→} \AgdaFunction{ListAbs} \AgdaBound{V} \AgdaBound{AA} \AgdaSymbol{→} \AgdaFunction{Expression} \AgdaBound{V} \AgdaBound{K} \AgdaSymbol{→} \AgdaPrimitiveType{Set}\<%
\end{code}
}


We define the operations of replacement and substitution on
expressions.  The details are given in Appendix \ref{appendix:repsub}.

\AgdaHide{
\begin{code}%
\>\AgdaKeyword{open} \AgdaKeyword{import} \AgdaModule{Grammar.Base}\<%
\\
%
\\
\>\AgdaKeyword{module} \AgdaModule{Grammar.Context} \AgdaSymbol{(}\AgdaBound{G} \AgdaSymbol{:} \AgdaRecord{Grammar}\AgdaSymbol{)} \AgdaKeyword{where}\<%
\\
%
\\
\>\AgdaKeyword{open} \AgdaKeyword{import} \AgdaModule{Data.Nat}\<%
\\
\>\AgdaKeyword{open} \AgdaKeyword{import} \AgdaModule{Data.Fin}\<%
\\
\>\AgdaKeyword{open} \AgdaKeyword{import} \AgdaModule{Prelims}\<%
\\
\>\AgdaKeyword{open} \AgdaModule{Grammar} \AgdaBound{G}\<%
\\
\>\AgdaKeyword{open} \AgdaKeyword{import} \AgdaModule{Grammar.Replacement} \AgdaBound{G}\<%
\end{code}
}

\subsection{Contexts}

A \emph{context} has the form $x_1 : A_1, \ldots, x_n : A_n$ where, for each $i$:
\begin{itemize}
\item $x_i$ is a variable of kind $K_i$ distinct from $x_1$, \ldots, $x_{i-1}$;
\item $A_i$ is an expression whose kind is the parent of $K_i$.
\end{itemize}
The \emph{domain} of this context is the alphabet $\{ x_1, \ldots, x_n \}$.

\begin{code}%
\>\AgdaKeyword{infixl} \AgdaNumber{55} \AgdaFixityOp{\_,\_}\<%
\\
\>\AgdaKeyword{data} \AgdaDatatype{Context} \AgdaSymbol{:} \AgdaDatatype{Alphabet} \AgdaSymbol{→} \AgdaPrimitiveType{Set} \AgdaKeyword{where}\<%
\\
\>[0]\AgdaIndent{2}{}\<[2]%
\>[2]\AgdaInductiveConstructor{〈〉} \AgdaSymbol{:} \AgdaDatatype{Context} \AgdaInductiveConstructor{∅}\<%
\\
\>[0]\AgdaIndent{2}{}\<[2]%
\>[2]\AgdaInductiveConstructor{\_,\_} \AgdaSymbol{:} \AgdaSymbol{∀} \AgdaSymbol{\{}\AgdaBound{V}\AgdaSymbol{\}} \AgdaSymbol{\{}\AgdaBound{K}\AgdaSymbol{\}} \AgdaSymbol{→} \AgdaDatatype{Context} \AgdaBound{V} \AgdaSymbol{→} \AgdaFunction{Expression} \AgdaBound{V} \AgdaSymbol{(}\AgdaFunction{parent} \AgdaBound{K}\AgdaSymbol{)} \AgdaSymbol{→} \<[58]%
\>[58]\<%
\\
\>[2]\AgdaIndent{4}{}\<[4]%
\>[4]\AgdaDatatype{Context} \AgdaSymbol{(}\AgdaBound{V} \AgdaInductiveConstructor{,} \AgdaBound{K}\AgdaSymbol{)}\<%
\\
%
\\
\>\AgdaComment{-- Define typeof such that, if x : A ∈ Γ, then typeof x Γ ≡ A}\<%
\\
\>\AgdaComment{-- We define it the following way so that typeof x Γ computes to an expression of the form}\<%
\\
\>\AgdaComment{-- M 〈 upRep 〉, even if x is not in canonical form}\<%
\\
\>\AgdaFunction{pretypeof} \AgdaSymbol{:} \AgdaSymbol{∀} \AgdaSymbol{\{}\AgdaBound{V}\AgdaSymbol{\}} \AgdaSymbol{\{}\AgdaBound{K}\AgdaSymbol{\}} \AgdaSymbol{\{}\AgdaBound{L}\AgdaSymbol{\}} \AgdaSymbol{(}\AgdaBound{x} \AgdaSymbol{:} \AgdaDatatype{Var} \AgdaSymbol{(}\AgdaBound{V} \AgdaInductiveConstructor{,} \AgdaBound{K}\AgdaSymbol{)} \AgdaBound{L}\AgdaSymbol{)} \AgdaSymbol{(}\AgdaBound{Γ} \AgdaSymbol{:} \AgdaDatatype{Context} \AgdaSymbol{(}\AgdaBound{V} \AgdaInductiveConstructor{,} \AgdaBound{K}\AgdaSymbol{))} \AgdaSymbol{→} \AgdaFunction{Expression} \AgdaBound{V} \AgdaSymbol{(}\AgdaFunction{parent} \AgdaBound{L}\AgdaSymbol{)}\<%
\\
\>\AgdaFunction{typeof} \AgdaSymbol{:} \AgdaSymbol{∀} \AgdaSymbol{\{}\AgdaBound{V}\AgdaSymbol{\}} \AgdaSymbol{\{}\AgdaBound{K}\AgdaSymbol{\}} \AgdaSymbol{(}\AgdaBound{x} \AgdaSymbol{:} \AgdaDatatype{Var} \AgdaBound{V} \AgdaBound{K}\AgdaSymbol{)} \AgdaSymbol{(}\AgdaBound{Γ} \AgdaSymbol{:} \AgdaDatatype{Context} \AgdaBound{V}\AgdaSymbol{)} \AgdaSymbol{→} \AgdaFunction{Expression} \AgdaBound{V} \AgdaSymbol{(}\AgdaFunction{parent} \AgdaBound{K}\AgdaSymbol{)}\<%
\\
%
\\
\>\AgdaFunction{pretypeof} \AgdaInductiveConstructor{x₀} \AgdaSymbol{(}\AgdaBound{Γ} \AgdaInductiveConstructor{,} \AgdaBound{A}\AgdaSymbol{)} \AgdaSymbol{=} \AgdaBound{A}\<%
\\
\>\AgdaFunction{pretypeof} \AgdaSymbol{(}\AgdaInductiveConstructor{↑} \AgdaBound{x}\AgdaSymbol{)} \AgdaSymbol{(}\AgdaBound{Γ} \AgdaInductiveConstructor{,} \AgdaBound{A}\AgdaSymbol{)} \AgdaSymbol{=} \AgdaFunction{typeof} \AgdaBound{x} \AgdaBound{Γ}\<%
\\
%
\\
\>\AgdaFunction{typeof} \AgdaSymbol{\{}\AgdaInductiveConstructor{∅}\AgdaSymbol{\}} \AgdaSymbol{()}\<%
\\
\>\AgdaFunction{typeof} \AgdaSymbol{\{\_} \AgdaInductiveConstructor{,} \AgdaSymbol{\_\}} \AgdaBound{x} \AgdaBound{Γ} \AgdaSymbol{=} \AgdaFunction{pretypeof} \AgdaBound{x} \AgdaBound{Γ} \AgdaFunction{⇑}\<%
\end{code}

We say that a replacement $\rho$ is a \emph{(well-typed) replacement from $\Gamma$ to $\Delta$},
$\rho : \Gamma \rightarrow \Delta$, iff, for each entry $x : A$ in $\Gamma$, we have that
$\rho(x) : A \langle ρ \rangle$ is an entry in $\Delta$.

\begin{code}%
\>\AgdaFunction{\_∶\_⇒R\_} \AgdaSymbol{:} \AgdaSymbol{∀} \AgdaSymbol{\{}\AgdaBound{U}\AgdaSymbol{\}} \AgdaSymbol{\{}\AgdaBound{V}\AgdaSymbol{\}} \AgdaSymbol{→} \AgdaFunction{Rep} \AgdaBound{U} \AgdaBound{V} \AgdaSymbol{→} \AgdaDatatype{Context} \AgdaBound{U} \AgdaSymbol{→} \AgdaDatatype{Context} \AgdaBound{V} \AgdaSymbol{→} \AgdaPrimitiveType{Set}\<%
\\
\>\AgdaBound{ρ} \AgdaFunction{∶} \AgdaBound{Γ} \AgdaFunction{⇒R} \AgdaBound{Δ} \AgdaSymbol{=} \AgdaSymbol{∀} \AgdaSymbol{\{}\AgdaBound{K}\AgdaSymbol{\}} \AgdaBound{x} \AgdaSymbol{→} \AgdaFunction{typeof} \AgdaSymbol{(}\AgdaBound{ρ} \AgdaBound{K} \AgdaBound{x}\AgdaSymbol{)} \AgdaBound{Δ} \AgdaDatatype{≡} \AgdaFunction{typeof} \AgdaBound{x} \AgdaBound{Γ} \AgdaFunction{〈} \AgdaBound{ρ} \AgdaFunction{〉}\<%
\\
%
\\
\>\AgdaKeyword{infix} \AgdaNumber{25} \AgdaFixityOp{\_,,\_}\<%
\\
\>\AgdaFunction{\_,,\_} \AgdaSymbol{:} \AgdaSymbol{∀} \AgdaSymbol{\{}\AgdaBound{V}\AgdaSymbol{\}} \AgdaSymbol{\{}\AgdaBound{AA}\AgdaSymbol{\}} \AgdaSymbol{→} \AgdaDatatype{Context} \AgdaBound{V} \AgdaSymbol{→} \AgdaDatatype{Types} \AgdaBound{V} \AgdaBound{AA} \AgdaSymbol{→} \AgdaDatatype{Context} \AgdaSymbol{(}\AgdaFunction{extend} \AgdaBound{V} \AgdaBound{AA}\AgdaSymbol{)}\<%
\\
\>\AgdaBound{Γ} \AgdaFunction{,,} \AgdaInductiveConstructor{[]} \AgdaSymbol{=} \AgdaBound{Γ}\<%
\\
\>\AgdaBound{Γ} \AgdaFunction{,,} \AgdaSymbol{(}\AgdaBound{A} \AgdaInductiveConstructor{,} \AgdaBound{AA}\AgdaSymbol{)} \AgdaSymbol{=} \AgdaSymbol{(}\AgdaBound{Γ} \AgdaInductiveConstructor{,} \AgdaBound{A}\AgdaSymbol{)} \AgdaFunction{,,} \AgdaBound{AA}\<%
\\
%
\\
\>\AgdaKeyword{infix} \AgdaNumber{25} \AgdaFixityOp{\_,,,\_}\<%
\\
\>\AgdaFunction{\_,,,\_} \AgdaSymbol{:} \AgdaSymbol{∀} \AgdaSymbol{\{}\AgdaBound{V} \AgdaBound{AA}\AgdaSymbol{\}} \AgdaSymbol{→} \AgdaDatatype{Context} \AgdaBound{V} \AgdaSymbol{→} \AgdaDatatype{snocTypes} \AgdaBound{V} \AgdaBound{AA} \AgdaSymbol{→} \AgdaDatatype{Context} \AgdaSymbol{(}\AgdaFunction{snoc-extend} \AgdaBound{V} \AgdaBound{AA}\AgdaSymbol{)}\<%
\\
\>\AgdaBound{Γ} \AgdaFunction{,,,} \AgdaInductiveConstructor{[]} \AgdaSymbol{=} \AgdaBound{Γ}\<%
\\
\>\AgdaBound{Γ} \AgdaFunction{,,,} \AgdaSymbol{(}\AgdaBound{AA} \AgdaInductiveConstructor{snoc} \AgdaBound{A}\AgdaSymbol{)} \AgdaSymbol{=} \AgdaSymbol{(}\AgdaBound{Γ} \AgdaFunction{,,,} \AgdaBound{AA}\AgdaSymbol{)} \AgdaInductiveConstructor{,} \AgdaBound{A}\<%
\end{code}

\begin{lemma}
\begin{enumerate}
\item
$\id{P}$ is a replacement $\Gamma \rightarrow \Gamma$.

\begin{code}%
\>\AgdaFunction{idRep-typed} \AgdaSymbol{:} \AgdaSymbol{∀} \AgdaSymbol{\{}\AgdaBound{V}\AgdaSymbol{\}} \AgdaSymbol{\{}\AgdaBound{Γ} \AgdaSymbol{:} \AgdaDatatype{Context} \AgdaBound{V}\AgdaSymbol{\}} \AgdaSymbol{→} \AgdaFunction{idRep} \AgdaBound{V} \AgdaFunction{∶} \AgdaBound{Γ} \AgdaFunction{⇒R} \AgdaBound{Γ}\<%
\end{code}

\AgdaHide{
\begin{code}%
\>\AgdaFunction{idRep-typed} \AgdaSymbol{\_} \AgdaSymbol{=} \AgdaFunction{sym} \AgdaFunction{rep-idRep}\<%
\end{code}
}

\item
$\uparrow$ is a replacement $\Gamma \rightarrow \Gamma , \phi$.

\begin{code}%
\>\AgdaFunction{↑-typed} \AgdaSymbol{:} \AgdaSymbol{∀} \AgdaSymbol{\{}\AgdaBound{V} \AgdaBound{Γ} \AgdaBound{K}\AgdaSymbol{\}} \AgdaSymbol{\{}\AgdaBound{A} \AgdaSymbol{:} \AgdaFunction{Expression} \AgdaBound{V} \AgdaSymbol{(}\AgdaFunction{parent} \AgdaBound{K}\AgdaSymbol{)\}} \AgdaSymbol{→} \AgdaFunction{upRep} \AgdaFunction{∶} \AgdaBound{Γ} \AgdaFunction{⇒R} \AgdaSymbol{(}\AgdaBound{Γ} \AgdaInductiveConstructor{,} \AgdaBound{A}\AgdaSymbol{)}\<%
\end{code}

\AgdaHide{
\begin{code}%
\>\AgdaFunction{↑-typed} \AgdaSymbol{\_} \AgdaSymbol{=} \AgdaInductiveConstructor{refl}\<%
\end{code}
}

\item
If $\rho : \Gamma \rightarrow \Delta$ then $(\rho , K) : (\Gamma , x : A) \rightarrow (\Delta , x : A 〈 ρ 〉)$.

\begin{code}%
\>\AgdaFunction{liftRep-typed} \AgdaSymbol{:} \AgdaSymbol{∀} \AgdaSymbol{\{}\AgdaBound{U} \AgdaBound{V} \AgdaBound{ρ} \AgdaBound{K}\AgdaSymbol{\}} \AgdaSymbol{\{}\AgdaBound{Γ} \AgdaSymbol{:} \AgdaDatatype{Context} \AgdaBound{U}\AgdaSymbol{\}} \AgdaSymbol{\{}\AgdaBound{Δ} \AgdaSymbol{:} \AgdaDatatype{Context} \AgdaBound{V}\AgdaSymbol{\}} \AgdaSymbol{\{}\AgdaBound{A} \AgdaSymbol{:} \AgdaFunction{Expression} \AgdaBound{U} \AgdaSymbol{(}\AgdaFunction{parent} \AgdaBound{K}\AgdaSymbol{)\}} \AgdaSymbol{→} \<[92]%
\>[92]\<%
\\
\>[0]\AgdaIndent{2}{}\<[2]%
\>[2]\AgdaBound{ρ} \AgdaFunction{∶} \AgdaBound{Γ} \AgdaFunction{⇒R} \AgdaBound{Δ} \AgdaSymbol{→} \AgdaFunction{liftRep} \AgdaBound{K} \AgdaBound{ρ} \AgdaFunction{∶} \AgdaSymbol{(}\AgdaBound{Γ} \AgdaInductiveConstructor{,} \AgdaBound{A}\AgdaSymbol{)} \AgdaFunction{⇒R} \AgdaSymbol{(}\AgdaBound{Δ} \AgdaInductiveConstructor{,} \AgdaBound{A} \AgdaFunction{〈} \AgdaBound{ρ} \AgdaFunction{〉}\AgdaSymbol{)}\<%
\end{code}

\AgdaHide{
\begin{code}%
\>\AgdaComment{--TODO Refactor?}\<%
\\
\>\AgdaFunction{liftRep-typed} \AgdaSymbol{\{}\AgdaArgument{A} \AgdaSymbol{=} \AgdaBound{A}\AgdaSymbol{\}} \AgdaBound{ρ∶Γ⇒Δ} \AgdaInductiveConstructor{x₀} \AgdaSymbol{=} \AgdaFunction{sym} \AgdaSymbol{(}\AgdaFunction{liftRep-upRep} \AgdaBound{A}\AgdaSymbol{)}\<%
\\
\>\AgdaFunction{liftRep-typed} \AgdaSymbol{\{}\AgdaArgument{ρ} \AgdaSymbol{=} \AgdaBound{ρ}\AgdaSymbol{\}} \AgdaSymbol{\{}\AgdaBound{K}\AgdaSymbol{\}} \AgdaSymbol{\{}\AgdaBound{Γ}\AgdaSymbol{\}} \AgdaSymbol{\{}\AgdaBound{Δ}\AgdaSymbol{\}} \AgdaSymbol{\{}\AgdaBound{A}\AgdaSymbol{\}} \AgdaBound{ρ∶Γ⇒Δ} \AgdaSymbol{\{}\AgdaBound{L}\AgdaSymbol{\}} \AgdaSymbol{(}\AgdaInductiveConstructor{↑} \AgdaBound{x}\AgdaSymbol{)} \AgdaSymbol{=} \AgdaKeyword{let} \AgdaKeyword{open} \AgdaModule{≡-Reasoning} \AgdaKeyword{in} \<[80]%
\>[80]\<%
\\
\>[0]\AgdaIndent{2}{}\<[2]%
\>[2]\AgdaFunction{begin}\<%
\\
\>[2]\AgdaIndent{4}{}\<[4]%
\>[4]\AgdaFunction{typeof} \AgdaSymbol{(}\AgdaBound{ρ} \AgdaBound{L} \AgdaBound{x}\AgdaSymbol{)} \AgdaBound{Δ} \AgdaFunction{⇑}\<%
\\
\>[0]\AgdaIndent{2}{}\<[2]%
\>[2]\AgdaFunction{≡⟨} \AgdaFunction{rep-congl} \AgdaSymbol{(}\AgdaBound{ρ∶Γ⇒Δ} \AgdaBound{x}\AgdaSymbol{)} \AgdaFunction{⟩}\<%
\\
\>[2]\AgdaIndent{4}{}\<[4]%
\>[4]\AgdaFunction{typeof} \AgdaBound{x} \AgdaBound{Γ} \AgdaFunction{〈} \AgdaBound{ρ} \AgdaFunction{〉} \AgdaFunction{⇑}\<%
\\
\>[0]\AgdaIndent{2}{}\<[2]%
\>[2]\AgdaFunction{≡⟨⟨} \AgdaFunction{liftRep-upRep} \AgdaSymbol{(}\AgdaFunction{typeof} \AgdaBound{x} \AgdaBound{Γ}\AgdaSymbol{)} \AgdaFunction{⟩⟩}\<%
\\
\>[2]\AgdaIndent{4}{}\<[4]%
\>[4]\AgdaSymbol{(}\AgdaFunction{typeof} \AgdaBound{x} \AgdaBound{Γ} \AgdaFunction{⇑}\AgdaSymbol{)} \AgdaFunction{〈} \AgdaFunction{liftRep} \AgdaBound{K} \AgdaBound{ρ} \AgdaFunction{〉}\<%
\\
\>[0]\AgdaIndent{2}{}\<[2]%
\>[2]\AgdaFunction{∎}\<%
\end{code}
}

\item
If $\rho : \Gamma \rightarrow \Delta$ and $\sigma : \Delta \rightarrow \Theta$ then $\sigma \circ \rho : \Gamma \rightarrow \Delta$.

\begin{code}%
\>\AgdaFunction{•R-typed} \AgdaSymbol{:} \AgdaSymbol{∀} \AgdaSymbol{\{}\AgdaBound{U} \AgdaBound{V} \AgdaBound{W}\AgdaSymbol{\}} \AgdaSymbol{\{}\AgdaBound{σ} \AgdaSymbol{:} \AgdaFunction{Rep} \AgdaBound{V} \AgdaBound{W}\AgdaSymbol{\}} \AgdaSymbol{\{}\AgdaBound{ρ} \AgdaSymbol{:} \AgdaFunction{Rep} \AgdaBound{U} \AgdaBound{V}\AgdaSymbol{\}} \AgdaSymbol{\{}\AgdaBound{Γ}\AgdaSymbol{\}} \AgdaSymbol{\{}\AgdaBound{Δ}\AgdaSymbol{\}} \AgdaSymbol{\{}\AgdaBound{Θ}\AgdaSymbol{\}} \AgdaSymbol{→} \<[63]%
\>[63]\<%
\\
\>[0]\AgdaIndent{2}{}\<[2]%
\>[2]\AgdaBound{ρ} \AgdaFunction{∶} \AgdaBound{Γ} \AgdaFunction{⇒R} \AgdaBound{Δ} \AgdaSymbol{→} \AgdaBound{σ} \AgdaFunction{∶} \AgdaBound{Δ} \AgdaFunction{⇒R} \AgdaBound{Θ} \AgdaSymbol{→} \AgdaSymbol{(}\AgdaBound{σ} \AgdaFunction{•R} \AgdaBound{ρ}\AgdaSymbol{)} \AgdaFunction{∶} \AgdaBound{Γ} \AgdaFunction{⇒R} \AgdaBound{Θ}\<%
\end{code}

\AgdaHide{
\begin{code}%
\>\AgdaFunction{•R-typed} \AgdaSymbol{\{}\AgdaBound{U}\AgdaSymbol{\}} \AgdaSymbol{\{}\AgdaBound{V}\AgdaSymbol{\}} \AgdaSymbol{\{}\AgdaBound{W}\AgdaSymbol{\}} \AgdaSymbol{\{}\AgdaBound{σ}\AgdaSymbol{\}} \AgdaSymbol{\{}\AgdaBound{ρ}\AgdaSymbol{\}} \AgdaSymbol{\{}\AgdaBound{Γ}\AgdaSymbol{\}} \AgdaSymbol{\{}\AgdaBound{Δ}\AgdaSymbol{\}} \AgdaSymbol{\{}\AgdaBound{Θ}\AgdaSymbol{\}} \AgdaBound{ρ∶Γ⇒RΔ} \AgdaBound{σ∶Δ⇒RΘ} \AgdaSymbol{\{}\AgdaBound{K}\AgdaSymbol{\}} \AgdaBound{x} \AgdaSymbol{=} \AgdaKeyword{let} \AgdaKeyword{open} \AgdaModule{≡-Reasoning} \AgdaKeyword{in} \<[87]%
\>[87]\<%
\\
\>[0]\AgdaIndent{2}{}\<[2]%
\>[2]\AgdaFunction{begin}\<%
\\
\>[2]\AgdaIndent{4}{}\<[4]%
\>[4]\AgdaFunction{typeof} \AgdaSymbol{(}\AgdaBound{σ} \AgdaBound{K} \AgdaSymbol{(}\AgdaBound{ρ} \AgdaBound{K} \AgdaBound{x}\AgdaSymbol{))} \AgdaBound{Θ}\<%
\\
\>[0]\AgdaIndent{2}{}\<[2]%
\>[2]\AgdaFunction{≡⟨} \AgdaBound{σ∶Δ⇒RΘ} \AgdaSymbol{(}\AgdaBound{ρ} \AgdaBound{K} \AgdaBound{x}\AgdaSymbol{)} \AgdaFunction{⟩}\<%
\\
\>[2]\AgdaIndent{4}{}\<[4]%
\>[4]\AgdaFunction{typeof} \AgdaSymbol{(}\AgdaBound{ρ} \AgdaBound{K} \AgdaBound{x}\AgdaSymbol{)} \AgdaBound{Δ} \AgdaFunction{〈} \AgdaBound{σ} \AgdaFunction{〉}\<%
\\
\>[0]\AgdaIndent{2}{}\<[2]%
\>[2]\AgdaFunction{≡⟨} \AgdaFunction{rep-congl} \AgdaSymbol{(}\AgdaBound{ρ∶Γ⇒RΔ} \AgdaBound{x}\AgdaSymbol{)} \AgdaFunction{⟩}\<%
\\
\>[2]\AgdaIndent{4}{}\<[4]%
\>[4]\AgdaFunction{typeof} \AgdaBound{x} \AgdaBound{Γ} \AgdaFunction{〈} \AgdaBound{ρ} \AgdaFunction{〉} \AgdaFunction{〈} \AgdaBound{σ} \AgdaFunction{〉}\<%
\\
\>[0]\AgdaIndent{2}{}\<[2]%
\>[2]\AgdaFunction{≡⟨⟨} \AgdaFunction{rep-comp} \AgdaSymbol{(}\AgdaFunction{typeof} \AgdaBound{x} \AgdaBound{Γ}\AgdaSymbol{)} \AgdaFunction{⟩⟩}\<%
\\
\>[2]\AgdaIndent{4}{}\<[4]%
\>[4]\AgdaFunction{typeof} \AgdaBound{x} \AgdaBound{Γ} \AgdaFunction{〈} \AgdaBound{σ} \AgdaFunction{•R} \AgdaBound{ρ} \AgdaFunction{〉}\<%
\\
\>[0]\AgdaIndent{2}{}\<[2]%
\>[2]\AgdaFunction{∎}\<%
\end{code}
}

\end{enumerate}
\end{lemma}


\AgdaHide{
\begin{code}%
\>\AgdaComment{\{- Metavariable conventions:\<\\
\>  A, B    range over abstraction kinds\<\\
\>  C       range over kind classes\<\\
\>  AA, BB  range over lists of abstraction kinds\<\\
\>  E, F, G range over subexpressions\<\\
\>  K, L    range over expression kinds including variable kinds\<\\
\>  M, N, P range over expressions\<\\
\>  U, V, W range over alphabets -\}}\<%
\\
\>\AgdaKeyword{open} \AgdaKeyword{import} \AgdaModule{Function}\<%
\\
\>\AgdaKeyword{open} \AgdaKeyword{import} \AgdaModule{Data.List}\<%
\\
\>\AgdaKeyword{open} \AgdaKeyword{import} \AgdaModule{Prelims}\<%
\\
\>\AgdaKeyword{open} \AgdaKeyword{import} \AgdaModule{Grammar.Taxonomy}\<%
\\
%
\\
\>\AgdaKeyword{module} \AgdaModule{Grammar.Base} \AgdaKeyword{where}\<%
\\
%
\\
\>\AgdaKeyword{record} \AgdaRecord{IsGrammar} \AgdaSymbol{(}\AgdaBound{T} \AgdaSymbol{:} \AgdaRecord{Taxonomy}\AgdaSymbol{)} \AgdaSymbol{:} \AgdaPrimitiveType{Set₁} \AgdaKeyword{where}\<%
\\
\>[0]\AgdaIndent{2}{}\<[2]%
\>[2]\AgdaKeyword{open} \AgdaModule{Taxonomy} \AgdaBound{T}\<%
\\
\>[0]\AgdaIndent{2}{}\<[2]%
\>[2]\AgdaKeyword{field}\<%
\\
\>[2]\AgdaIndent{4}{}\<[4]%
\>[4]\AgdaField{Constructor} \<[19]%
\>[19]\AgdaSymbol{:} \AgdaFunction{ConKind} \AgdaSymbol{→} \AgdaPrimitiveType{Set}\<%
\\
\>[2]\AgdaIndent{4}{}\<[4]%
\>[4]\AgdaField{parent} \<[19]%
\>[19]\AgdaSymbol{:} \AgdaFunction{VarKind} \AgdaSymbol{→} \AgdaDatatype{ExpKind}\<%
\\
%
\\
\>\AgdaKeyword{record} \AgdaRecord{Grammar} \AgdaSymbol{:} \AgdaPrimitiveType{Set₁} \AgdaKeyword{where}\<%
\\
\>[0]\AgdaIndent{2}{}\<[2]%
\>[2]\AgdaKeyword{field}\<%
\\
\>[2]\AgdaIndent{4}{}\<[4]%
\>[4]\AgdaField{taxonomy} \AgdaSymbol{:} \AgdaRecord{Taxonomy}\<%
\\
\>[2]\AgdaIndent{4}{}\<[4]%
\>[4]\AgdaField{isGrammar} \AgdaSymbol{:} \AgdaRecord{IsGrammar} \AgdaField{taxonomy}\<%
\\
\>[0]\AgdaIndent{2}{}\<[2]%
\>[2]\AgdaKeyword{open} \AgdaModule{Taxonomy} \AgdaField{taxonomy} \AgdaKeyword{public}\<%
\\
\>[0]\AgdaIndent{2}{}\<[2]%
\>[2]\AgdaKeyword{open} \AgdaModule{IsGrammar} \AgdaField{isGrammar} \AgdaKeyword{public}\<%
\end{code}
}

%<*Expression>
\begin{code}%
\>[0]\AgdaIndent{2}{}\<[2]%
\>[2]\AgdaKeyword{data} \AgdaDatatype{Subexpression} \AgdaSymbol{(}\AgdaBound{V} \AgdaSymbol{:} \AgdaDatatype{Alphabet}\AgdaSymbol{)} \AgdaSymbol{:} \AgdaSymbol{∀} \AgdaBound{C} \AgdaSymbol{→} \AgdaFunction{Kind} \AgdaBound{C} \AgdaSymbol{→} \AgdaPrimitiveType{Set}\<%
\\
\>[0]\AgdaIndent{2}{}\<[2]%
\>[2]\AgdaFunction{Expression} \AgdaSymbol{:} \AgdaDatatype{Alphabet} \AgdaSymbol{→} \AgdaDatatype{ExpKind} \AgdaSymbol{→} \AgdaPrimitiveType{Set}\<%
\\
\>[0]\AgdaIndent{2}{}\<[2]%
\>[2]\AgdaFunction{VExpression} \AgdaSymbol{:} \AgdaDatatype{Alphabet} \AgdaSymbol{→} \AgdaFunction{VarKind} \AgdaSymbol{→} \AgdaPrimitiveType{Set}\<%
\\
\>[0]\AgdaIndent{2}{}\<[2]%
\>[2]\AgdaFunction{Abstraction} \AgdaSymbol{:} \AgdaDatatype{Alphabet} \AgdaSymbol{→} \AgdaFunction{AbsKind} \AgdaSymbol{→} \AgdaPrimitiveType{Set}\<%
\\
\>[0]\AgdaIndent{2}{}\<[2]%
\>[2]\AgdaFunction{ListAbs} \AgdaSymbol{:} \AgdaDatatype{Alphabet} \AgdaSymbol{→} \AgdaDatatype{List} \AgdaFunction{AbsKind} \AgdaSymbol{→} \AgdaPrimitiveType{Set}\<%
\\
%
\\
\>[0]\AgdaIndent{2}{}\<[2]%
\>[2]\AgdaFunction{Expression} \AgdaBound{V} \AgdaBound{K} \AgdaSymbol{=} \AgdaDatatype{Subexpression} \AgdaBound{V} \AgdaInductiveConstructor{-Expression} \AgdaBound{K}\<%
\\
\>[0]\AgdaIndent{2}{}\<[2]%
\>[2]\AgdaFunction{VExpression} \AgdaBound{V} \AgdaBound{K} \AgdaSymbol{=} \AgdaFunction{Expression} \AgdaBound{V} \AgdaSymbol{(}\AgdaInductiveConstructor{varKind} \AgdaBound{K}\AgdaSymbol{)}\<%
\\
\>[0]\AgdaIndent{2}{}\<[2]%
\>[2]\AgdaFunction{Abstraction} \AgdaBound{V} \AgdaSymbol{(}\AgdaInductiveConstructor{SK} \AgdaBound{AA} \AgdaBound{K}\AgdaSymbol{)} \AgdaSymbol{=} \AgdaFunction{Expression} \AgdaSymbol{(}\AgdaFunction{extend} \AgdaBound{V} \AgdaBound{AA}\AgdaSymbol{)} \AgdaBound{K}\<%
\\
\>[0]\AgdaIndent{2}{}\<[2]%
\>[2]\AgdaFunction{ListAbs} \AgdaBound{V} \AgdaBound{AA} \AgdaSymbol{=} \AgdaDatatype{Subexpression} \AgdaBound{V} \AgdaInductiveConstructor{-ListAbs} \AgdaBound{AA}\<%
\\
%
\\
\>[0]\AgdaIndent{2}{}\<[2]%
\>[2]\AgdaKeyword{infixr} \AgdaNumber{5} \AgdaFixityOp{\_∷\_}\<%
\\
\>[0]\AgdaIndent{2}{}\<[2]%
\>[2]\AgdaKeyword{data} \AgdaDatatype{Subexpression} \AgdaBound{V} \AgdaKeyword{where}\<%
\\
\>[2]\AgdaIndent{4}{}\<[4]%
\>[4]\AgdaInductiveConstructor{var} \AgdaSymbol{:} \AgdaSymbol{∀} \AgdaSymbol{\{}\AgdaBound{K}\AgdaSymbol{\}} \AgdaSymbol{→} \AgdaDatatype{Var} \AgdaBound{V} \AgdaBound{K} \AgdaSymbol{→} \AgdaFunction{VExpression} \AgdaBound{V} \AgdaBound{K}\<%
\\
\>[2]\AgdaIndent{4}{}\<[4]%
\>[4]\AgdaInductiveConstructor{app} \AgdaSymbol{:} \AgdaSymbol{∀} \AgdaSymbol{\{}\AgdaBound{AA}\AgdaSymbol{\}} \AgdaSymbol{\{}\AgdaBound{K}\AgdaSymbol{\}} \AgdaSymbol{→} \AgdaFunction{Constructor} \AgdaSymbol{(}\AgdaInductiveConstructor{SK} \AgdaBound{AA} \AgdaBound{K}\AgdaSymbol{)} \AgdaSymbol{→} \AgdaFunction{ListAbs} \AgdaBound{V} \AgdaBound{AA} \AgdaSymbol{→} \AgdaFunction{Expression} \AgdaBound{V} \AgdaBound{K}\<%
\\
\>[2]\AgdaIndent{4}{}\<[4]%
\>[4]\AgdaInductiveConstructor{[]} \AgdaSymbol{:} \AgdaFunction{ListAbs} \AgdaBound{V} \AgdaInductiveConstructor{[]}\<%
\\
\>[2]\AgdaIndent{4}{}\<[4]%
\>[4]\AgdaInductiveConstructor{\_∷\_} \AgdaSymbol{:} \AgdaSymbol{∀} \AgdaSymbol{\{}\AgdaBound{A}\AgdaSymbol{\}} \AgdaSymbol{\{}\AgdaBound{AA}\AgdaSymbol{\}} \AgdaSymbol{→} \AgdaFunction{Abstraction} \AgdaBound{V} \AgdaBound{A} \AgdaSymbol{→} \AgdaFunction{ListAbs} \AgdaBound{V} \AgdaBound{AA} \AgdaSymbol{→} \AgdaFunction{ListAbs} \AgdaBound{V} \AgdaSymbol{(}\AgdaBound{A} \AgdaInductiveConstructor{∷} \AgdaBound{AA}\AgdaSymbol{)}\<%
\end{code}
%</Expression>

We prove that the constructor \AgdaRef{var} is injective.

\begin{code}%
\>[0]\AgdaIndent{2}{}\<[2]%
\>[2]\AgdaFunction{var-inj} \AgdaSymbol{:} \AgdaSymbol{∀} \AgdaSymbol{\{}\AgdaBound{V}\AgdaSymbol{\}} \AgdaSymbol{\{}\AgdaBound{K}\AgdaSymbol{\}} \AgdaSymbol{\{}\AgdaBound{x} \AgdaBound{y} \AgdaSymbol{:} \AgdaDatatype{Var} \AgdaBound{V} \AgdaBound{K}\AgdaSymbol{\}} \AgdaSymbol{→} \AgdaInductiveConstructor{var} \AgdaBound{x} \AgdaDatatype{≡} \AgdaInductiveConstructor{var} \AgdaBound{y} \AgdaSymbol{→} \AgdaBound{x} \AgdaDatatype{≡} \AgdaBound{y}\<%
\\
\>[0]\AgdaIndent{2}{}\<[2]%
\>[2]\AgdaFunction{var-inj} \AgdaInductiveConstructor{refl} \AgdaSymbol{=} \AgdaInductiveConstructor{refl}\<%
\end{code}

For the future, we also define the type of all snoc-lists of expressions $(M_1, \ldots, M_n)$
such that $M_i$ is of type $K_i$, given a snoc-list of variable kinds $(K_1, \ldots, K_n)$.

\begin{code}%
\>[0]\AgdaIndent{2}{}\<[2]%
\>[2]\AgdaKeyword{infixl} \AgdaNumber{20} \AgdaFixityOp{\_snoc\_}\<%
\\
\>[0]\AgdaIndent{2}{}\<[2]%
\>[2]\AgdaKeyword{data} \AgdaDatatype{snocListExp} \AgdaBound{V} \AgdaSymbol{:} \AgdaDatatype{snocList} \AgdaFunction{VarKind} \AgdaSymbol{→} \AgdaPrimitiveType{Set} \AgdaKeyword{where}\<%
\\
\>[2]\AgdaIndent{4}{}\<[4]%
\>[4]\AgdaInductiveConstructor{[]} \AgdaSymbol{:} \AgdaDatatype{snocListExp} \AgdaBound{V} \AgdaInductiveConstructor{[]}\<%
\\
\>[2]\AgdaIndent{4}{}\<[4]%
\>[4]\AgdaInductiveConstructor{\_snoc\_} \AgdaSymbol{:} \AgdaSymbol{∀} \AgdaSymbol{\{}\AgdaBound{A}\AgdaSymbol{\}} \AgdaSymbol{\{}\AgdaBound{K}\AgdaSymbol{\}} \AgdaSymbol{→} \AgdaDatatype{snocListExp} \AgdaBound{V} \AgdaBound{A} \AgdaSymbol{→} \AgdaFunction{Expression} \AgdaBound{V} \AgdaSymbol{(}\AgdaInductiveConstructor{varKind} \AgdaBound{K}\AgdaSymbol{)} \AgdaSymbol{→} \AgdaDatatype{snocListExp} \AgdaBound{V} \AgdaSymbol{(}\AgdaBound{A} \AgdaInductiveConstructor{snoc} \AgdaBound{K}\AgdaSymbol{)}\<%
\end{code}

A \emph{reduction} is a relation $\rhd$ between expressions such that, if $E \rhd F$,
then $E$ is not a variable.  It is given by a term $R : \AgdaRef{Reduction}$ such that
$R\, c\, MM\, N$ iff $c[MM] \rhd N$.

\begin{code}%
\>[0]\AgdaIndent{2}{}\<[2]%
\>[2]\AgdaFunction{Reduction} \AgdaSymbol{:} \AgdaPrimitiveType{Set₁}\<%
\\
\>[0]\AgdaIndent{2}{}\<[2]%
\>[2]\AgdaFunction{Reduction} \AgdaSymbol{=} \AgdaSymbol{∀} \AgdaSymbol{\{}\AgdaBound{V}\AgdaSymbol{\}} \AgdaSymbol{\{}\AgdaBound{AA}\AgdaSymbol{\}} \AgdaSymbol{\{}\AgdaBound{K}\AgdaSymbol{\}} \AgdaSymbol{→} \AgdaFunction{Constructor} \AgdaSymbol{(}\AgdaInductiveConstructor{SK} \AgdaBound{AA} \AgdaBound{K}\AgdaSymbol{)} \AgdaSymbol{→} \AgdaFunction{ListAbs} \AgdaBound{V} \AgdaBound{AA} \AgdaSymbol{→} \AgdaFunction{Expression} \AgdaBound{V} \AgdaBound{K} \AgdaSymbol{→} \AgdaPrimitiveType{Set}\<%
\end{code}
}

\begin{code}%
\>\AgdaKeyword{module} \AgdaModule{PHOPL.SN} \AgdaKeyword{where}\<%
\\
\>\AgdaKeyword{open} \AgdaKeyword{import} \AgdaModule{Data.Empty} \AgdaKeyword{renaming} \AgdaSymbol{(}\AgdaDatatype{⊥} \AgdaSymbol{to} \AgdaDatatype{Empty}\AgdaSymbol{)}\<%
\\
\>\AgdaKeyword{open} \AgdaKeyword{import} \AgdaModule{PHOPL.Grammar}\<%
\\
\>\AgdaKeyword{open} \AgdaKeyword{import} \AgdaModule{PHOPL.Red}\<%
\\
\>\AgdaKeyword{open} \AgdaKeyword{import} \AgdaModule{Reduction.Botsub} \AgdaFunction{PHOPL} \AgdaDatatype{R}\<%
\\
%
\\
\>\AgdaKeyword{private} \AgdaFunction{βR-exp'} \AgdaSymbol{:} \AgdaSymbol{∀} \AgdaSymbol{\{}\AgdaBound{V}\AgdaSymbol{\}} \AgdaSymbol{\{}\AgdaBound{φ} \AgdaSymbol{:} \AgdaFunction{Term} \AgdaBound{V}\AgdaSymbol{\}} \AgdaSymbol{\{}\AgdaBound{δ}\AgdaSymbol{\}} \AgdaSymbol{\{}\AgdaBound{ε}\AgdaSymbol{\}} \AgdaSymbol{\{}\AgdaBound{χ}\AgdaSymbol{\}} \AgdaSymbol{→} \AgdaDatatype{SN} \AgdaBound{φ} \AgdaSymbol{→} \AgdaDatatype{SN} \AgdaBound{δ} \AgdaSymbol{→} \AgdaDatatype{SN} \AgdaBound{ε} \AgdaSymbol{→}\<%
\\
\>[0]\AgdaIndent{9}{}\<[9]%
\>[9]\AgdaDatatype{SN} \AgdaSymbol{(}\AgdaBound{δ} \AgdaFunction{⟦} \AgdaFunction{x₀:=} \AgdaBound{ε} \AgdaFunction{⟧}\AgdaSymbol{)} \AgdaSymbol{→} \AgdaFunction{appP} \AgdaSymbol{(}\AgdaFunction{ΛP} \AgdaBound{φ} \AgdaBound{δ}\AgdaSymbol{)} \AgdaBound{ε} \AgdaDatatype{⇒} \AgdaBound{χ} \AgdaSymbol{→} \AgdaDatatype{SN} \AgdaBound{χ}\<%
\\
\>\AgdaFunction{βR-exp'} \AgdaBound{SNφ} \AgdaBound{SNδ} \AgdaBound{SNε} \AgdaBound{SNδε} \AgdaSymbol{(}\AgdaInductiveConstructor{redex} \AgdaInductiveConstructor{βR}\AgdaSymbol{)} \AgdaSymbol{=} \AgdaBound{SNδε}\<%
\\
\>\AgdaFunction{βR-exp'} \AgdaBound{SNφ} \AgdaBound{SNδ} \AgdaBound{SNε} \AgdaBound{SNδε} \AgdaSymbol{(}\AgdaInductiveConstructor{app} \AgdaSymbol{(}\AgdaInductiveConstructor{appl} \AgdaSymbol{(}\AgdaInductiveConstructor{redex} \AgdaSymbol{())))}\<%
\\
\>\AgdaFunction{βR-exp'} \AgdaSymbol{(}\AgdaInductiveConstructor{SNI} \AgdaBound{φ} \AgdaBound{SNφ}\AgdaSymbol{)} \AgdaBound{SNδ} \AgdaBound{SNε} \AgdaBound{SNδε} \AgdaSymbol{(}\AgdaInductiveConstructor{app} \AgdaSymbol{(}\AgdaInductiveConstructor{appl} \AgdaSymbol{(}\AgdaInductiveConstructor{app} \AgdaSymbol{(}\AgdaInductiveConstructor{appl} \AgdaBound{φ⇒φ'}\AgdaSymbol{))))} \AgdaSymbol{=} \<[66]%
\>[66]\<%
\\
\>[0]\AgdaIndent{2}{}\<[2]%
\>[2]\AgdaInductiveConstructor{SNI} \AgdaSymbol{\_} \AgdaSymbol{(λ} \AgdaBound{\_} \AgdaSymbol{→} \AgdaFunction{βR-exp'} \AgdaSymbol{(}\AgdaBound{SNφ} \AgdaSymbol{\_} \AgdaBound{φ⇒φ'}\AgdaSymbol{)} \AgdaBound{SNδ} \AgdaBound{SNε} \AgdaBound{SNδε}\AgdaSymbol{)} \<[50]%
\>[50]\<%
\\
\>\AgdaFunction{βR-exp'} \AgdaBound{SNφ} \AgdaSymbol{(}\AgdaInductiveConstructor{SNI} \AgdaBound{δ} \AgdaBound{SNδ}\AgdaSymbol{)} \AgdaBound{SNε} \AgdaBound{SNδε} \AgdaSymbol{(}\AgdaInductiveConstructor{app} \AgdaSymbol{(}\AgdaInductiveConstructor{appl} \AgdaSymbol{(}\AgdaInductiveConstructor{app} \AgdaSymbol{(}\AgdaInductiveConstructor{appr} \AgdaSymbol{(}\AgdaInductiveConstructor{appl} \AgdaBound{δ⇒δ'}\AgdaSymbol{)))))} \AgdaSymbol{=} \<[73]%
\>[73]\<%
\\
\>[0]\AgdaIndent{2}{}\<[2]%
\>[2]\AgdaInductiveConstructor{SNI} \AgdaSymbol{\_} \AgdaSymbol{(λ} \AgdaBound{\_} \AgdaSymbol{→} \AgdaFunction{βR-exp'} \AgdaBound{SNφ} \AgdaSymbol{(}\AgdaBound{SNδ} \AgdaSymbol{\_} \AgdaBound{δ⇒δ'}\AgdaSymbol{)} \AgdaBound{SNε} \AgdaSymbol{(}\AgdaFunction{SNred} \AgdaBound{SNδε} \AgdaSymbol{(}\AgdaFunction{apredr} \AgdaFunction{substitution} \AgdaPostulate{R-respects-sub} \AgdaSymbol{(}\AgdaInductiveConstructor{osr-red} \AgdaBound{δ⇒δ'}\AgdaSymbol{))))}\<%
\\
\>\AgdaFunction{βR-exp'} \AgdaBound{SNφ} \AgdaBound{SNδ} \AgdaBound{SNε} \AgdaBound{SNδε} \AgdaSymbol{(}\AgdaInductiveConstructor{app} \AgdaSymbol{(}\AgdaInductiveConstructor{appl} \AgdaSymbol{(}\AgdaInductiveConstructor{app} \AgdaSymbol{(}\AgdaInductiveConstructor{appr} \AgdaSymbol{(}\AgdaInductiveConstructor{appr} \AgdaSymbol{())))))}\<%
\\
\>\AgdaFunction{βR-exp'} \AgdaSymbol{\{}\AgdaArgument{δ} \AgdaSymbol{=} \AgdaBound{δ}\AgdaSymbol{\}} \AgdaBound{SNφ} \AgdaBound{SNδ} \AgdaSymbol{(}\AgdaInductiveConstructor{SNI} \AgdaBound{ε} \AgdaBound{SNε}\AgdaSymbol{)} \AgdaBound{SNδε} \AgdaSymbol{(}\AgdaInductiveConstructor{app} \AgdaSymbol{(}\AgdaInductiveConstructor{appr} \AgdaSymbol{(}\AgdaInductiveConstructor{appl} \AgdaBound{ε⇒ε'}\AgdaSymbol{)))} \AgdaSymbol{=} \<[68]%
\>[68]\<%
\\
\>[0]\AgdaIndent{2}{}\<[2]%
\>[2]\AgdaInductiveConstructor{SNI} \AgdaSymbol{\_} \AgdaSymbol{(λ} \AgdaBound{\_} \AgdaSymbol{→} \AgdaFunction{βR-exp'} \AgdaBound{SNφ} \AgdaBound{SNδ} \AgdaSymbol{(}\AgdaBound{SNε} \AgdaSymbol{\_} \AgdaBound{ε⇒ε'}\AgdaSymbol{)} \AgdaSymbol{(}\AgdaFunction{SNred} \AgdaBound{SNδε} \AgdaSymbol{(}\AgdaFunction{apredl} \AgdaFunction{substitution} \AgdaSymbol{\{}\AgdaArgument{E} \AgdaSymbol{=} \AgdaBound{δ}\AgdaSymbol{\}} \AgdaPostulate{R-respects-sub} \AgdaSymbol{(}\AgdaFunction{botsub-red} \AgdaBound{ε⇒ε'}\AgdaSymbol{))))}\<%
\\
\>\AgdaFunction{βR-exp'} \AgdaBound{SNφ} \AgdaBound{SNδ} \AgdaBound{SNε} \AgdaBound{SNδε} \AgdaSymbol{(}\AgdaInductiveConstructor{app} \AgdaSymbol{(}\AgdaInductiveConstructor{appr} \AgdaSymbol{(}\AgdaInductiveConstructor{appr} \AgdaSymbol{())))}\<%
\\
%
\\
\>\AgdaFunction{βR-exp} \AgdaSymbol{:} \AgdaSymbol{∀} \AgdaSymbol{\{}\AgdaBound{V}\AgdaSymbol{\}} \AgdaSymbol{\{}\AgdaBound{φ} \AgdaSymbol{:} \AgdaFunction{Term} \AgdaBound{V}\AgdaSymbol{\}} \AgdaSymbol{\{}\AgdaBound{δ}\AgdaSymbol{\}} \AgdaSymbol{\{}\AgdaBound{ε}\AgdaSymbol{\}} \AgdaSymbol{→} \AgdaDatatype{SN} \AgdaBound{φ} \AgdaSymbol{→} \AgdaDatatype{SN} \AgdaBound{ε} \AgdaSymbol{→}\<%
\\
\>[2]\AgdaIndent{9}{}\<[9]%
\>[9]\AgdaDatatype{SN} \AgdaSymbol{(}\AgdaBound{δ} \AgdaFunction{⟦} \AgdaFunction{x₀:=} \AgdaBound{ε} \AgdaFunction{⟧}\AgdaSymbol{)} \AgdaSymbol{→} \AgdaDatatype{SN} \AgdaSymbol{(}\AgdaFunction{appP} \AgdaSymbol{(}\AgdaFunction{ΛP} \AgdaBound{φ} \AgdaBound{δ}\AgdaSymbol{)} \AgdaBound{ε}\AgdaSymbol{)}\<%
\\
\>\AgdaFunction{βR-exp} \AgdaSymbol{\{}\AgdaArgument{φ} \AgdaSymbol{=} \AgdaBound{φ}\AgdaSymbol{\}} \AgdaSymbol{\{}\AgdaBound{δ}\AgdaSymbol{\}} \AgdaSymbol{\{}\AgdaBound{ε}\AgdaSymbol{\}} \AgdaBound{SNφ} \AgdaBound{SNε} \AgdaBound{SNδε} \AgdaSymbol{=} \AgdaInductiveConstructor{SNI} \AgdaSymbol{(}\AgdaFunction{appP} \AgdaSymbol{(}\AgdaFunction{ΛP} \AgdaBound{φ} \AgdaBound{δ}\AgdaSymbol{)} \AgdaBound{ε}\AgdaSymbol{)} \AgdaSymbol{(λ} \AgdaBound{χ} \AgdaSymbol{→} \AgdaFunction{βR-exp'} \AgdaBound{SNφ} \<[79]%
\>[79]\<%
\\
\>[0]\AgdaIndent{2}{}\<[2]%
\>[2]\AgdaSymbol{(}\AgdaFunction{SNap'} \AgdaSymbol{\{}\AgdaArgument{Ops} \AgdaSymbol{=} \AgdaFunction{substitution}\AgdaSymbol{\}} \AgdaPostulate{R-respects-sub} \AgdaBound{SNδε}\AgdaSymbol{)} \AgdaBound{SNε} \AgdaBound{SNδε}\AgdaSymbol{)}\<%
\\
%
\\
\>\AgdaKeyword{private} \AgdaFunction{βE-exp'} \AgdaSymbol{:} \AgdaSymbol{∀} \AgdaSymbol{\{}\AgdaBound{V}\AgdaSymbol{\}} \AgdaSymbol{\{}\AgdaBound{A}\AgdaSymbol{\}} \AgdaSymbol{\{}\AgdaBound{M} \AgdaBound{N} \AgdaSymbol{:} \AgdaFunction{Term} \AgdaBound{V}\AgdaSymbol{\}} \AgdaSymbol{\{}\AgdaBound{P}\AgdaSymbol{\}} \AgdaSymbol{\{}\AgdaBound{Q}\AgdaSymbol{\}} \AgdaSymbol{\{}\AgdaBound{R}\AgdaSymbol{\}} \AgdaSymbol{→}\<%
\\
\>[2]\AgdaIndent{17}{}\<[17]%
\>[17]\AgdaDatatype{SN} \AgdaBound{M} \AgdaSymbol{→} \AgdaDatatype{SN} \AgdaBound{N} \AgdaSymbol{→} \AgdaDatatype{SN} \AgdaBound{P} \AgdaSymbol{→} \AgdaDatatype{SN} \AgdaBound{Q} \AgdaSymbol{→}\<%
\\
\>[2]\AgdaIndent{17}{}\<[17]%
\>[17]\AgdaDatatype{SN} \AgdaSymbol{(}\AgdaBound{P} \AgdaFunction{⟦} \AgdaFunction{x₂:=} \AgdaBound{M} \AgdaFunction{,x₁:=} \AgdaBound{N} \AgdaFunction{,x₀:=} \AgdaBound{Q} \AgdaFunction{⟧}\AgdaSymbol{)} \AgdaSymbol{→}\<%
\\
\>[2]\AgdaIndent{17}{}\<[17]%
\>[17]\AgdaFunction{app*} \AgdaBound{M} \AgdaBound{N} \AgdaSymbol{(}\AgdaFunction{λλλ} \AgdaBound{A} \AgdaBound{P}\AgdaSymbol{)} \AgdaBound{Q} \AgdaDatatype{⇒} \AgdaBound{R} \AgdaSymbol{→}\<%
\\
\>[2]\AgdaIndent{17}{}\<[17]%
\>[17]\AgdaDatatype{SN} \AgdaBound{R}\<%
\\
\>\AgdaFunction{βE-exp'} \AgdaBound{SNM} \AgdaBound{SNN} \AgdaBound{SNP} \AgdaBound{SNQ} \AgdaBound{SNPMNQ} \AgdaSymbol{(}\AgdaInductiveConstructor{redex} \AgdaInductiveConstructor{βE}\AgdaSymbol{)} \AgdaSymbol{=} \AgdaBound{SNPMNQ}\<%
\\
\>\AgdaFunction{βE-exp'} \AgdaSymbol{\{}\AgdaArgument{P} \AgdaSymbol{=} \AgdaBound{P}\AgdaSymbol{\}} \AgdaSymbol{(}\AgdaInductiveConstructor{SNI} \AgdaBound{M} \AgdaBound{SNM}\AgdaSymbol{)} \AgdaBound{SNN} \AgdaBound{SNP} \AgdaBound{SNQ} \AgdaBound{SNPMNQ} \AgdaSymbol{(}\AgdaInductiveConstructor{app} \AgdaSymbol{(}\AgdaInductiveConstructor{appl} \AgdaBound{M⇒M'}\AgdaSymbol{))} \AgdaSymbol{=} \<[67]%
\>[67]\<%
\\
\>[0]\AgdaIndent{2}{}\<[2]%
\>[2]\AgdaInductiveConstructor{SNI} \AgdaSymbol{\_} \AgdaSymbol{(λ} \AgdaBound{\_} \AgdaSymbol{→} \AgdaFunction{βE-exp'} \AgdaSymbol{(}\AgdaBound{SNM} \AgdaSymbol{\_} \AgdaBound{M⇒M'}\AgdaSymbol{)} \AgdaBound{SNN} \AgdaBound{SNP} \AgdaBound{SNQ} \<[48]%
\>[48]\<%
\\
\>[2]\AgdaIndent{4}{}\<[4]%
\>[4]\AgdaSymbol{(}\AgdaFunction{SNred} \AgdaBound{SNPMNQ} \AgdaSymbol{(}\AgdaFunction{apredl} \AgdaFunction{substitution} \AgdaSymbol{\{}\AgdaArgument{E} \AgdaSymbol{=} \AgdaBound{P}\AgdaSymbol{\}} \AgdaPostulate{R-respects-sub} \<[62]%
\>[62]\<%
\\
\>[4]\AgdaIndent{6}{}\<[6]%
\>[6]\AgdaSymbol{(}\AgdaFunction{botsub₃-red} \AgdaSymbol{(}\AgdaInductiveConstructor{osr-red} \AgdaBound{M⇒M'}\AgdaSymbol{)} \AgdaInductiveConstructor{ref} \AgdaInductiveConstructor{ref}\AgdaSymbol{))))} \<[46]%
\>[46]\<%
\\
\>\AgdaFunction{βE-exp'} \AgdaSymbol{\{}\AgdaArgument{P} \AgdaSymbol{=} \AgdaBound{P}\AgdaSymbol{\}} \AgdaBound{SNM} \AgdaSymbol{(}\AgdaInductiveConstructor{SNI} \AgdaSymbol{\_} \AgdaBound{SNN}\AgdaSymbol{)} \AgdaBound{SNP} \AgdaBound{SNQ} \AgdaBound{SNPMNQ} \AgdaSymbol{(}\AgdaInductiveConstructor{app} \AgdaSymbol{(}\AgdaInductiveConstructor{appr} \AgdaSymbol{(}\AgdaInductiveConstructor{appl} \AgdaBound{N⇒N'}\AgdaSymbol{)))} \AgdaSymbol{=} \<[74]%
\>[74]\<%
\\
\>[0]\AgdaIndent{2}{}\<[2]%
\>[2]\AgdaInductiveConstructor{SNI} \AgdaSymbol{\_} \AgdaSymbol{(λ} \AgdaBound{\_} \AgdaSymbol{→} \AgdaFunction{βE-exp'} \AgdaBound{SNM} \AgdaSymbol{(}\AgdaBound{SNN} \AgdaSymbol{\_} \AgdaBound{N⇒N'}\AgdaSymbol{)} \AgdaBound{SNP} \AgdaBound{SNQ} \<[48]%
\>[48]\<%
\\
\>[2]\AgdaIndent{4}{}\<[4]%
\>[4]\AgdaSymbol{(}\AgdaFunction{SNred} \AgdaBound{SNPMNQ} \AgdaSymbol{(}\AgdaFunction{apredl} \AgdaFunction{substitution} \AgdaSymbol{\{}\AgdaArgument{E} \AgdaSymbol{=} \AgdaBound{P}\AgdaSymbol{\}} \AgdaPostulate{R-respects-sub} \<[62]%
\>[62]\<%
\\
\>[4]\AgdaIndent{6}{}\<[6]%
\>[6]\AgdaSymbol{(}\AgdaFunction{botsub₃-red} \AgdaInductiveConstructor{ref} \AgdaSymbol{(}\AgdaInductiveConstructor{osr-red} \AgdaBound{N⇒N'}\AgdaSymbol{)} \AgdaInductiveConstructor{ref}\AgdaSymbol{))))}\<%
\\
\>\AgdaFunction{βE-exp'} \AgdaBound{SNM} \AgdaBound{SNN} \AgdaBound{SNP} \AgdaBound{SNQ} \AgdaBound{SNPMNQ} \AgdaSymbol{(}\AgdaInductiveConstructor{app} \AgdaSymbol{(}\AgdaInductiveConstructor{appr} \AgdaSymbol{(}\AgdaInductiveConstructor{appr} \AgdaSymbol{(}\AgdaInductiveConstructor{appl} \AgdaSymbol{(}\AgdaInductiveConstructor{redex} \AgdaSymbol{())))))}\<%
\\
\>\AgdaFunction{βE-exp'} \AgdaBound{SNM} \AgdaBound{SNN} \AgdaSymbol{(}\AgdaInductiveConstructor{SNI} \AgdaSymbol{\_} \AgdaBound{SNP}\AgdaSymbol{)} \AgdaBound{SNQ} \AgdaBound{SNPMNQ} \AgdaSymbol{(}\AgdaInductiveConstructor{app} \AgdaSymbol{(}\AgdaInductiveConstructor{appr} \AgdaSymbol{(}\AgdaInductiveConstructor{appr} \AgdaSymbol{(}\AgdaInductiveConstructor{appl} \AgdaSymbol{(}\AgdaInductiveConstructor{app} \AgdaSymbol{(}\AgdaInductiveConstructor{appl} \AgdaBound{P⇒P'}\AgdaSymbol{))))))} \AgdaSymbol{=} \<[86]%
\>[86]\<%
\\
\>[0]\AgdaIndent{2}{}\<[2]%
\>[2]\AgdaInductiveConstructor{SNI} \AgdaSymbol{\_} \AgdaSymbol{(λ} \AgdaBound{\_} \AgdaSymbol{→} \AgdaFunction{βE-exp'} \AgdaBound{SNM} \AgdaBound{SNN} \AgdaSymbol{(}\AgdaBound{SNP} \AgdaSymbol{\_} \AgdaBound{P⇒P'}\AgdaSymbol{)} \AgdaBound{SNQ}\<%
\\
\>[2]\AgdaIndent{4}{}\<[4]%
\>[4]\AgdaSymbol{(}\AgdaFunction{SNred} \AgdaBound{SNPMNQ} \AgdaSymbol{(}\AgdaFunction{apredr} \AgdaFunction{substitution} \AgdaPostulate{R-respects-sub} \AgdaSymbol{(}\AgdaInductiveConstructor{osr-red} \AgdaBound{P⇒P'}\AgdaSymbol{))))}\<%
\\
\>\AgdaFunction{βE-exp'} \AgdaBound{SNM} \AgdaBound{SNN} \AgdaBound{SNP} \AgdaBound{SNQ} \AgdaBound{SNPMNQ} \AgdaSymbol{(}\AgdaInductiveConstructor{app} \AgdaSymbol{(}\AgdaInductiveConstructor{appr} \AgdaSymbol{(}\AgdaInductiveConstructor{appr} \AgdaSymbol{(}\AgdaInductiveConstructor{appl} \AgdaSymbol{(}\AgdaInductiveConstructor{app} \AgdaSymbol{(}\AgdaInductiveConstructor{appr} \AgdaSymbol{()))))))}\<%
\\
\>\AgdaFunction{βE-exp'} \AgdaSymbol{\{}\AgdaArgument{P} \AgdaSymbol{=} \AgdaBound{P}\AgdaSymbol{\}} \AgdaBound{SNM} \AgdaBound{SNN} \AgdaBound{SNP} \AgdaSymbol{(}\AgdaInductiveConstructor{SNI} \AgdaSymbol{\_} \AgdaBound{SNQ}\AgdaSymbol{)} \AgdaBound{SNPMNQ} \AgdaSymbol{(}\AgdaInductiveConstructor{app} \AgdaSymbol{(}\AgdaInductiveConstructor{appr} \AgdaSymbol{(}\AgdaInductiveConstructor{appr} \AgdaSymbol{(}\AgdaInductiveConstructor{appr} \AgdaSymbol{(}\AgdaInductiveConstructor{appl} \AgdaBound{Q⇒Q'}\AgdaSymbol{)))))} \AgdaSymbol{=} \<[88]%
\>[88]\<%
\\
\>[0]\AgdaIndent{2}{}\<[2]%
\>[2]\AgdaInductiveConstructor{SNI} \AgdaSymbol{\_} \AgdaSymbol{(λ} \AgdaBound{\_} \AgdaSymbol{→} \AgdaFunction{βE-exp'} \AgdaBound{SNM} \AgdaBound{SNN} \AgdaBound{SNP} \AgdaSymbol{(}\AgdaBound{SNQ} \AgdaSymbol{\_} \AgdaBound{Q⇒Q'}\AgdaSymbol{)} \<[48]%
\>[48]\<%
\\
\>[2]\AgdaIndent{4}{}\<[4]%
\>[4]\AgdaSymbol{(}\AgdaFunction{SNred} \AgdaBound{SNPMNQ} \AgdaSymbol{(}\AgdaFunction{apredl} \AgdaFunction{substitution} \AgdaSymbol{\{}\AgdaArgument{E} \AgdaSymbol{=} \AgdaBound{P}\AgdaSymbol{\}} \AgdaPostulate{R-respects-sub} \<[62]%
\>[62]\<%
\\
\>[4]\AgdaIndent{6}{}\<[6]%
\>[6]\AgdaSymbol{(}\AgdaFunction{botsub₃-red} \AgdaInductiveConstructor{ref} \AgdaInductiveConstructor{ref} \AgdaSymbol{(}\AgdaInductiveConstructor{osr-red} \AgdaBound{Q⇒Q'}\AgdaSymbol{)))))}\<%
\\
\>\AgdaFunction{βE-exp'} \AgdaBound{SNM} \AgdaBound{SNN} \AgdaBound{SNP} \AgdaBound{SNQ} \AgdaBound{SNPMNQ} \AgdaSymbol{(}\AgdaInductiveConstructor{app} \AgdaSymbol{(}\AgdaInductiveConstructor{appr} \AgdaSymbol{(}\AgdaInductiveConstructor{appr} \AgdaSymbol{(}\AgdaInductiveConstructor{appr} \AgdaSymbol{(}\AgdaInductiveConstructor{appr} \AgdaSymbol{())))))}\<%
\\
%
\\
\>\AgdaFunction{βE-exp} \AgdaSymbol{:} \AgdaSymbol{∀} \AgdaSymbol{\{}\AgdaBound{V}\AgdaSymbol{\}} \AgdaSymbol{\{}\AgdaBound{A}\AgdaSymbol{\}} \AgdaSymbol{\{}\AgdaBound{M} \AgdaBound{N} \AgdaSymbol{:} \AgdaFunction{Term} \AgdaBound{V}\AgdaSymbol{\}} \AgdaSymbol{\{}\AgdaBound{P}\AgdaSymbol{\}} \AgdaSymbol{\{}\AgdaBound{Q}\AgdaSymbol{\}} \AgdaSymbol{→}\<%
\\
\>[6]\AgdaIndent{9}{}\<[9]%
\>[9]\AgdaDatatype{SN} \AgdaBound{M} \AgdaSymbol{→} \AgdaDatatype{SN} \AgdaBound{N} \AgdaSymbol{→} \AgdaDatatype{SN} \AgdaBound{Q} \AgdaSymbol{→}\<%
\\
\>[6]\AgdaIndent{9}{}\<[9]%
\>[9]\AgdaDatatype{SN} \AgdaSymbol{(}\AgdaBound{P} \AgdaFunction{⟦} \AgdaFunction{x₂:=} \AgdaBound{M} \AgdaFunction{,x₁:=} \AgdaBound{N} \AgdaFunction{,x₀:=} \AgdaBound{Q} \AgdaFunction{⟧}\AgdaSymbol{)} \AgdaSymbol{→}\<%
\\
\>[6]\AgdaIndent{9}{}\<[9]%
\>[9]\AgdaDatatype{SN} \AgdaSymbol{(}\AgdaFunction{app*} \AgdaBound{M} \AgdaBound{N} \AgdaSymbol{(}\AgdaFunction{λλλ} \AgdaBound{A} \AgdaBound{P}\AgdaSymbol{)} \AgdaBound{Q}\AgdaSymbol{)}\<%
\\
\>\AgdaFunction{βE-exp} \AgdaBound{SNM} \AgdaBound{SNN} \AgdaBound{SNQ} \AgdaBound{SNPQ} \AgdaSymbol{=} \AgdaInductiveConstructor{SNI} \AgdaSymbol{\_} \AgdaSymbol{(λ} \AgdaBound{R} \AgdaBound{PQ⇒R} \AgdaSymbol{→} \AgdaFunction{βE-exp'} \AgdaBound{SNM} \AgdaBound{SNN} \AgdaSymbol{(}\AgdaFunction{SNap'} \AgdaSymbol{\{}\AgdaArgument{Ops} \AgdaSymbol{=} \AgdaFunction{substitution}\AgdaSymbol{\}} \AgdaPostulate{R-respects-sub} \AgdaBound{SNPQ}\AgdaSymbol{)} \AgdaBound{SNQ} \AgdaBound{SNPQ} \AgdaBound{PQ⇒R}\AgdaSymbol{)}\<%
\\
%
\\
\>\AgdaComment{--REFACTOR Common pattern}\<%
\\
%
\\
\>\AgdaKeyword{private} \AgdaFunction{SN'} \AgdaSymbol{:} \AgdaSymbol{∀} \AgdaSymbol{\{}\AgdaBound{V}\AgdaSymbol{\}} \AgdaSymbol{\{}\AgdaBound{φ} \AgdaSymbol{:} \AgdaFunction{Term} \AgdaBound{V}\AgdaSymbol{\}} \AgdaSymbol{→} \AgdaFunction{⊥} \AgdaDatatype{⇒} \AgdaBound{φ} \AgdaSymbol{→} \AgdaDatatype{Empty}\<%
\\
\>\AgdaFunction{SN'} \AgdaSymbol{(}\AgdaInductiveConstructor{redex} \AgdaSymbol{())}\<%
\\
\>\AgdaFunction{SN'} \AgdaSymbol{(}\AgdaInductiveConstructor{app} \AgdaSymbol{())}\<%
\\
%
\\
\>\AgdaFunction{SN⊥} \AgdaSymbol{:} \AgdaSymbol{∀} \AgdaSymbol{\{}\AgdaBound{V}\AgdaSymbol{\}} \AgdaSymbol{→} \AgdaDatatype{SN} \AgdaSymbol{\{}\AgdaBound{V}\AgdaSymbol{\}} \AgdaFunction{⊥}\<%
\\
\>\AgdaFunction{SN⊥} \AgdaSymbol{\{}\AgdaBound{V}\AgdaSymbol{\}} \AgdaSymbol{=} \AgdaInductiveConstructor{SNI} \AgdaFunction{⊥} \AgdaSymbol{(λ} \AgdaBound{\_} \AgdaBound{⊥⇒F} \AgdaSymbol{→} \AgdaFunction{⊥-elim} \AgdaSymbol{(}\AgdaFunction{SN'} \AgdaBound{⊥⇒F}\AgdaSymbol{))}\<%
\end{code}

\AgdaHide{
\begin{code}%
\>\AgdaKeyword{module} \AgdaModule{PHOPL.Grammar} \AgdaKeyword{where}\<%
\\
%
\\
\>\AgdaKeyword{open} \AgdaKeyword{import} \AgdaModule{Data.Nat}\<%
\\
\>\AgdaKeyword{open} \AgdaKeyword{import} \AgdaModule{Data.Empty} \AgdaKeyword{renaming} \AgdaSymbol{(}\AgdaDatatype{⊥} \AgdaSymbol{to} \AgdaDatatype{Empty}\AgdaSymbol{)}\<%
\\
\>\AgdaKeyword{open} \AgdaKeyword{import} \AgdaModule{Data.List} \AgdaKeyword{hiding} \AgdaSymbol{(}\AgdaFunction{replicate}\AgdaSymbol{)}\<%
\\
\>\AgdaKeyword{open} \AgdaKeyword{import} \AgdaModule{Data.Vec} \AgdaKeyword{hiding} \AgdaSymbol{(}\AgdaFunction{replicate}\AgdaSymbol{)}\<%
\\
\>\AgdaKeyword{open} \AgdaKeyword{import} \AgdaModule{Prelims}\<%
\\
\>\AgdaKeyword{open} \AgdaKeyword{import} \AgdaModule{Grammar.Taxonomy}\<%
\\
\>\AgdaKeyword{open} \AgdaKeyword{import} \AgdaModule{Grammar.Base}\<%
\end{code}
}

\subsection{Syntax}

Fix three disjoint, infinite sets of variables, which we shall call \emph{term variables}, \emph{proof variables}
and \emph{path variables}.  We shall use $x$ and $y$ as term variables, $p$ and $q$ as proof variables,
$e$ as a path variable, and $z$ for a variable that may come from any of these three sets.

The syntax of $\lambda o e$ is given by the grammar:

\[
\begin{array}{lrcl}
\text{Type} & A,B,C & ::= & \Omega \mid A \rightarrow B \\
\text{Term} & L,M,N, \phi,\psi,\chi & ::= & x \mid \bot \mid \phi \supset \psi \mid \lambda x:A.M \mid MN \\
\text{Proof} & \delta, \epsilon & ::= & p \mid \lambda p:\phi.\delta \mid \delta \epsilon \mid P^+ \mid P^- \\
\text{Path} & P, Q & ::= & e \mid \reff{M} \mid P \supset^* Q \mid \univ{\phi}{\psi}{P}{Q} \mid \\
& & & \triplelambda e : x =_A y. P \mid P_{MN} Q \\
\text{Context} & \Gamma, \Delta, \Theta & ::= & \langle \rangle \mid \Gamma, x : A \mid \Gamma, p : \phi \mid \Gamma, e : M =_A N \\
\text{Judgement} & \mathbf{J} & ::= & \Gamma \vald \mid \Gamma \vdash M : A \mid \Gamma \vdash \delta : \phi \mid \\
& & & \Gamma \vdash P : M =_A N
\end{array}
\]

In the path $\triplelambda e : x =_A y . P$, the term variables $x$ and $y$ must be distinct.  (We also have $x \not\equiv e \not\equiv y$, thanks to our
stipulation that term variables and path variables are disjoint.)  The term variable $x$ is bound within $M$ in the term $\lambda x:A.M$,
and the proof variable $p$ is bound within $\delta$ in $\lambda p:\phi.\delta$.  The three variables $e$, $x$ and $y$ are bound within $P$ in the path
$\triplelambda e:x =_A y.P$.  We identify terms, proofs and paths up to $\alpha$-conversion.

We shall use the word 'expression' to mean either a type, term, proof, path, or equation (an equation having the form $M =_A N$).  We shall use $r$, $s$, $t$, $S$ and $T$ as metavariables that range over expressions.

Note that we use both Roman letters $M$, $N$ and Greek letters $\phi$, $\psi$, $\chi$ to range over terms.  Intuitively, a term is understood as either a proposition or a function,
and we shall use Greek letters for terms that are intended to be propositions.  Formally, there is no significance to which letter we choose.

Note also that the types of $\lambda o e$ are just the simple types over $\Omega$; therefore, no variable can occur in a type.

The intuition behind the new expressions is as follows (see also the rules of deduction in Figure \ref{fig:lambdaoe}).  For any object $M : A$, there is the trivial path $\reff{M} : M =_A M$.  The constructor $\supset^*$ ensures congruence for $\supset$ --- if $P : \phi =_\Omega \phi'$ and $Q : \psi =_\Omega \psi'$ then $P \supset^* Q : \phi \supset \psi =_\Omega \phi' \supset \psi'$.  The constructor $\mathsf{univ}$ gives univalence for our propositions: if $\delta : \phi \supset \psi$ and $\epsilon : \psi \supset \phi$, then $\univ{\phi}{\psi}{\delta}{\epsilon}$ is a path of type $\phi =_\Omega \psi$.  The constructors $^+$ and $^-$ are the converses: if $P$ is a path of type $\phi =_\Omega \psi$, then $P^+$ is a proof of $\phi \supset \psi$, and $P^-$ is a proof of $\psi \supset \phi$.

The constructor $\triplelambda$ gives functional extensionality.  Let $F$ and $G$ be functions of type $A \rightarrow B$.  If $F x =_B G y$ whenever $x =_A y$, then $F =_{A \rightarrow B} G$.  More formally, if $P$ is a path of type $Fx =_B Gy$ that depends on $x : A$, $y : A$ and $e : x =_A y$, then $\triplelambda e : x =_A y . P$ is a path of type $F =_{A \rightarrow B} G$.

Finally, if $P$ is a path of type $F =_{A \rightarrow B} G$, and $Q$ is a path $M =_A N$, then $P_{MN} Q$ is a path $FM =_B G N$.

\begin{code}%
\>\AgdaKeyword{data} \AgdaDatatype{PHOPLVarKind} \AgdaSymbol{:} \AgdaPrimitiveType{Set} \AgdaKeyword{where}\<%
\\
\>[0]\AgdaIndent{2}{}\<[2]%
\>[2]\AgdaInductiveConstructor{-Proof} \AgdaSymbol{:} \AgdaDatatype{PHOPLVarKind}\<%
\\
\>[0]\AgdaIndent{2}{}\<[2]%
\>[2]\AgdaInductiveConstructor{-Term} \AgdaSymbol{:} \AgdaDatatype{PHOPLVarKind}\<%
\\
\>[0]\AgdaIndent{2}{}\<[2]%
\>[2]\AgdaInductiveConstructor{-Path} \AgdaSymbol{:} \AgdaDatatype{PHOPLVarKind}\<%
\\
%
\\
\>\AgdaKeyword{data} \AgdaDatatype{PHOPLNonVarKind} \AgdaSymbol{:} \AgdaPrimitiveType{Set} \AgdaKeyword{where}\<%
\\
\>[0]\AgdaIndent{2}{}\<[2]%
\>[2]\AgdaInductiveConstructor{-Type} \AgdaSymbol{:} \AgdaDatatype{PHOPLNonVarKind}\<%
\\
\>[0]\AgdaIndent{2}{}\<[2]%
\>[2]\AgdaInductiveConstructor{-Equation} \AgdaSymbol{:} \AgdaDatatype{PHOPLNonVarKind}\<%
\\
%
\\
\>\AgdaFunction{PHOPLTaxonomy} \AgdaSymbol{:} \AgdaRecord{Taxonomy}\<%
\\
\>\AgdaFunction{PHOPLTaxonomy} \AgdaSymbol{=} \AgdaKeyword{record} \AgdaSymbol{\{} \<[25]%
\>[25]\<%
\\
\>[0]\AgdaIndent{2}{}\<[2]%
\>[2]\AgdaField{VarKind} \AgdaSymbol{=} \AgdaDatatype{PHOPLVarKind}\AgdaSymbol{;} \<[26]%
\>[26]\<%
\\
\>[0]\AgdaIndent{2}{}\<[2]%
\>[2]\AgdaField{NonVarKind} \AgdaSymbol{=} \AgdaDatatype{PHOPLNonVarKind} \AgdaSymbol{\}}\<%
\\
%
\\
\>\AgdaKeyword{module} \AgdaModule{PHOPLgrammar} \AgdaKeyword{where}\<%
\\
\>[0]\AgdaIndent{2}{}\<[2]%
\>[2]\AgdaKeyword{open} \AgdaModule{Taxonomy} \AgdaFunction{PHOPLTaxonomy}\<%
\\
%
\\
\>[0]\AgdaIndent{2}{}\<[2]%
\>[2]\AgdaFunction{-vProof} \AgdaSymbol{:} \AgdaDatatype{ExpKind}\<%
\\
\>[0]\AgdaIndent{2}{}\<[2]%
\>[2]\AgdaFunction{-vProof} \AgdaSymbol{=} \AgdaInductiveConstructor{varKind} \AgdaInductiveConstructor{-Proof}\<%
\\
%
\\
\>[0]\AgdaIndent{2}{}\<[2]%
\>[2]\AgdaFunction{-vTerm} \AgdaSymbol{:} \AgdaDatatype{ExpKind}\<%
\\
\>[0]\AgdaIndent{2}{}\<[2]%
\>[2]\AgdaFunction{-vTerm} \AgdaSymbol{=} \AgdaInductiveConstructor{varKind} \AgdaInductiveConstructor{-Term}\<%
\\
%
\\
\>[0]\AgdaIndent{2}{}\<[2]%
\>[2]\AgdaFunction{-vPath} \AgdaSymbol{:} \AgdaDatatype{ExpKind}\<%
\\
\>[0]\AgdaIndent{2}{}\<[2]%
\>[2]\AgdaFunction{-vPath} \AgdaSymbol{=} \AgdaInductiveConstructor{varKind} \AgdaInductiveConstructor{-Path}\<%
\\
%
\\
\>[0]\AgdaIndent{2}{}\<[2]%
\>[2]\AgdaFunction{-nvType} \AgdaSymbol{:} \AgdaDatatype{ExpKind}\<%
\\
\>[0]\AgdaIndent{2}{}\<[2]%
\>[2]\AgdaFunction{-nvType} \AgdaSymbol{=} \AgdaInductiveConstructor{nonVarKind} \AgdaInductiveConstructor{-Type}\<%
\\
%
\\
\>[0]\AgdaIndent{2}{}\<[2]%
\>[2]\AgdaFunction{-nvEq} \AgdaSymbol{:} \AgdaDatatype{ExpKind}\<%
\\
\>[0]\AgdaIndent{2}{}\<[2]%
\>[2]\AgdaFunction{-nvEq} \AgdaSymbol{=} \AgdaInductiveConstructor{nonVarKind} \AgdaInductiveConstructor{-Equation}\<%
\\
%
\\
\>[0]\AgdaIndent{2}{}\<[2]%
\>[2]\AgdaKeyword{data} \AgdaDatatype{Type} \AgdaSymbol{:} \AgdaPrimitiveType{Set} \AgdaKeyword{where}\<%
\\
\>[2]\AgdaIndent{4}{}\<[4]%
\>[4]\AgdaInductiveConstructor{Ω} \AgdaSymbol{:} \AgdaDatatype{Type}\<%
\\
\>[2]\AgdaIndent{4}{}\<[4]%
\>[4]\AgdaInductiveConstructor{\_⇛\_} \AgdaSymbol{:} \AgdaDatatype{Type} \AgdaSymbol{→} \AgdaDatatype{Type} \AgdaSymbol{→} \AgdaDatatype{Type}\<%
\\
%
\\
\>[0]\AgdaIndent{2}{}\<[2]%
\>[2]\AgdaKeyword{data} \AgdaDatatype{Dir} \AgdaSymbol{:} \AgdaPrimitiveType{Set} \AgdaKeyword{where}\<%
\\
\>[2]\AgdaIndent{4}{}\<[4]%
\>[4]\AgdaInductiveConstructor{-plus} \AgdaSymbol{:} \AgdaDatatype{Dir}\<%
\\
\>[2]\AgdaIndent{4}{}\<[4]%
\>[4]\AgdaInductiveConstructor{-minus} \AgdaSymbol{:} \AgdaDatatype{Dir}\<%
\\
%
\\
\>[0]\AgdaIndent{2}{}\<[2]%
\>[2]\AgdaFunction{pathDom} \AgdaSymbol{:} \AgdaDatatype{List} \AgdaFunction{VarKind}\<%
\\
\>[0]\AgdaIndent{2}{}\<[2]%
\>[2]\AgdaFunction{pathDom} \AgdaSymbol{=} \AgdaInductiveConstructor{-Term} \AgdaInductiveConstructor{∷} \AgdaInductiveConstructor{-Term} \AgdaInductiveConstructor{∷} \AgdaInductiveConstructor{-Path} \AgdaInductiveConstructor{∷} \AgdaInductiveConstructor{[]}\<%
\\
%
\\
\>[0]\AgdaIndent{2}{}\<[2]%
\>[2]\AgdaKeyword{data} \AgdaDatatype{PHOPLcon} \AgdaSymbol{:} \AgdaFunction{ConKind} \AgdaSymbol{→} \AgdaPrimitiveType{Set} \AgdaKeyword{where}\<%
\\
\>[2]\AgdaIndent{4}{}\<[4]%
\>[4]\AgdaInductiveConstructor{-ty} \AgdaSymbol{:} \AgdaDatatype{Type} \AgdaSymbol{→} \AgdaDatatype{PHOPLcon} \AgdaSymbol{(}\AgdaFunction{-nvType} \AgdaFunction{✧}\AgdaSymbol{)}\<%
\\
\>[2]\AgdaIndent{4}{}\<[4]%
\>[4]\AgdaInductiveConstructor{-bot} \AgdaSymbol{:} \AgdaDatatype{PHOPLcon} \AgdaSymbol{(}\AgdaFunction{-vTerm} \AgdaFunction{✧}\AgdaSymbol{)}\<%
\\
\>[2]\AgdaIndent{4}{}\<[4]%
\>[4]\AgdaInductiveConstructor{-imp} \AgdaSymbol{:} \AgdaDatatype{PHOPLcon} \AgdaSymbol{(}\AgdaFunction{-vTerm} \AgdaFunction{✧} \AgdaFunction{⟶} \AgdaFunction{-vTerm} \AgdaFunction{✧} \AgdaFunction{⟶} \AgdaFunction{-vTerm} \AgdaFunction{✧}\AgdaSymbol{)}\<%
\\
\>[2]\AgdaIndent{4}{}\<[4]%
\>[4]\AgdaInductiveConstructor{-lamTerm} \AgdaSymbol{:} \AgdaDatatype{Type} \AgdaSymbol{→} \AgdaDatatype{PHOPLcon} \AgdaSymbol{((}\AgdaInductiveConstructor{-Term} \AgdaFunction{⟶} \AgdaFunction{-vTerm} \AgdaFunction{✧}\AgdaSymbol{)} \AgdaFunction{⟶} \AgdaFunction{-vTerm} \AgdaFunction{✧}\AgdaSymbol{)}\<%
\\
\>[2]\AgdaIndent{4}{}\<[4]%
\>[4]\AgdaInductiveConstructor{-appTerm} \AgdaSymbol{:} \AgdaDatatype{PHOPLcon} \AgdaSymbol{(}\AgdaFunction{-vTerm} \AgdaFunction{✧} \AgdaFunction{⟶} \AgdaFunction{-vTerm} \AgdaFunction{✧} \AgdaFunction{⟶} \AgdaFunction{-vTerm} \AgdaFunction{✧}\AgdaSymbol{)}\<%
\\
\>[2]\AgdaIndent{4}{}\<[4]%
\>[4]\AgdaInductiveConstructor{-lamProof} \AgdaSymbol{:} \AgdaDatatype{PHOPLcon} \AgdaSymbol{(}\AgdaFunction{-vTerm} \AgdaFunction{✧} \AgdaFunction{⟶} \AgdaSymbol{(}\AgdaInductiveConstructor{-Proof} \AgdaFunction{⟶} \AgdaFunction{-vProof} \AgdaFunction{✧}\AgdaSymbol{)} \AgdaFunction{⟶} \AgdaFunction{-vProof} \AgdaFunction{✧}\AgdaSymbol{)}\<%
\\
\>[2]\AgdaIndent{4}{}\<[4]%
\>[4]\AgdaInductiveConstructor{-appProof} \AgdaSymbol{:} \AgdaDatatype{PHOPLcon} \AgdaSymbol{(}\AgdaFunction{-vProof} \AgdaFunction{✧} \AgdaFunction{⟶} \AgdaFunction{-vProof} \AgdaFunction{✧} \AgdaFunction{⟶} \AgdaFunction{-vProof} \AgdaFunction{✧}\AgdaSymbol{)}\<%
\\
\>[2]\AgdaIndent{4}{}\<[4]%
\>[4]\AgdaInductiveConstructor{-dir} \AgdaSymbol{:} \AgdaDatatype{Dir} \AgdaSymbol{→} \AgdaDatatype{PHOPLcon} \AgdaSymbol{(}\AgdaFunction{-vPath} \AgdaFunction{✧} \AgdaFunction{⟶} \AgdaFunction{-vProof} \AgdaFunction{✧}\AgdaSymbol{)}\<%
\\
\>[2]\AgdaIndent{4}{}\<[4]%
\>[4]\AgdaInductiveConstructor{-ref} \AgdaSymbol{:} \AgdaDatatype{PHOPLcon} \AgdaSymbol{(}\AgdaFunction{-vTerm} \AgdaFunction{✧} \AgdaFunction{⟶} \AgdaFunction{-vPath} \AgdaFunction{✧}\AgdaSymbol{)}\<%
\\
\>[2]\AgdaIndent{4}{}\<[4]%
\>[4]\AgdaInductiveConstructor{-imp*} \AgdaSymbol{:} \AgdaDatatype{PHOPLcon} \AgdaSymbol{(}\AgdaFunction{-vPath} \AgdaFunction{✧} \AgdaFunction{⟶} \AgdaFunction{-vPath} \AgdaFunction{✧} \AgdaFunction{⟶} \AgdaFunction{-vPath} \AgdaFunction{✧}\AgdaSymbol{)}\<%
\\
\>[2]\AgdaIndent{4}{}\<[4]%
\>[4]\AgdaInductiveConstructor{-univ} \AgdaSymbol{:} \AgdaDatatype{PHOPLcon} \AgdaSymbol{(}\AgdaFunction{-vTerm} \AgdaFunction{✧} \AgdaFunction{⟶} \AgdaFunction{-vTerm} \AgdaFunction{✧} \AgdaFunction{⟶} \AgdaFunction{-vProof} \AgdaFunction{✧} \AgdaFunction{⟶} \AgdaFunction{-vProof} \AgdaFunction{✧} \AgdaFunction{⟶} \AgdaFunction{-vPath} \AgdaFunction{✧}\AgdaSymbol{)}\<%
\\
\>[2]\AgdaIndent{4}{}\<[4]%
\>[4]\AgdaInductiveConstructor{-lll} \AgdaSymbol{:} \AgdaDatatype{Type} \AgdaSymbol{→} \AgdaDatatype{PHOPLcon} \AgdaSymbol{(}\AgdaInductiveConstructor{SK} \AgdaFunction{pathDom} \AgdaFunction{-vPath} \AgdaFunction{⟶} \AgdaFunction{-vPath} \AgdaFunction{✧}\AgdaSymbol{)}\<%
\\
\>[2]\AgdaIndent{4}{}\<[4]%
\>[4]\AgdaInductiveConstructor{-app*} \AgdaSymbol{:} \AgdaDatatype{PHOPLcon} \AgdaSymbol{(}\AgdaFunction{-vTerm} \AgdaFunction{✧} \AgdaFunction{⟶} \AgdaFunction{-vTerm} \AgdaFunction{✧} \AgdaFunction{⟶} \AgdaFunction{-vPath} \AgdaFunction{✧} \AgdaFunction{⟶} \AgdaFunction{-vPath} \AgdaFunction{✧} \AgdaFunction{⟶} \AgdaFunction{-vPath} \AgdaFunction{✧}\AgdaSymbol{)}\<%
\\
\>[2]\AgdaIndent{4}{}\<[4]%
\>[4]\AgdaInductiveConstructor{-eq} \AgdaSymbol{:} \AgdaDatatype{Type} \AgdaSymbol{→} \AgdaDatatype{PHOPLcon} \AgdaSymbol{(}\AgdaFunction{-vTerm} \AgdaFunction{✧} \AgdaFunction{⟶} \AgdaFunction{-vTerm} \AgdaFunction{✧} \AgdaFunction{⟶} \AgdaFunction{-nvEq} \AgdaFunction{✧}\AgdaSymbol{)}\<%
\\
%
\\
\>[0]\AgdaIndent{2}{}\<[2]%
\>[2]\AgdaFunction{PHOPLparent} \AgdaSymbol{:} \AgdaDatatype{PHOPLVarKind} \AgdaSymbol{→} \AgdaDatatype{ExpKind}\<%
\\
\>[0]\AgdaIndent{2}{}\<[2]%
\>[2]\AgdaFunction{PHOPLparent} \AgdaInductiveConstructor{-Proof} \AgdaSymbol{=} \AgdaFunction{-vTerm}\<%
\\
\>[0]\AgdaIndent{2}{}\<[2]%
\>[2]\AgdaFunction{PHOPLparent} \AgdaInductiveConstructor{-Term} \AgdaSymbol{=} \AgdaFunction{-nvType}\<%
\\
\>[0]\AgdaIndent{2}{}\<[2]%
\>[2]\AgdaFunction{PHOPLparent} \AgdaInductiveConstructor{-Path} \AgdaSymbol{=} \AgdaFunction{-nvEq}\<%
\\
%
\\
\>[0]\AgdaIndent{2}{}\<[2]%
\>[2]\AgdaFunction{PHOPL} \AgdaSymbol{:} \AgdaRecord{Grammar}\<%
\\
\>[0]\AgdaIndent{2}{}\<[2]%
\>[2]\AgdaFunction{PHOPL} \AgdaSymbol{=} \AgdaKeyword{record} \AgdaSymbol{\{} \<[19]%
\>[19]\<%
\\
\>[2]\AgdaIndent{4}{}\<[4]%
\>[4]\AgdaField{taxonomy} \AgdaSymbol{=} \AgdaFunction{PHOPLTaxonomy}\AgdaSymbol{;}\<%
\\
\>[2]\AgdaIndent{4}{}\<[4]%
\>[4]\AgdaField{isGrammar} \AgdaSymbol{=} \AgdaKeyword{record} \AgdaSymbol{\{} \<[25]%
\>[25]\<%
\\
\>[4]\AgdaIndent{6}{}\<[6]%
\>[6]\AgdaField{Con} \AgdaSymbol{=} \AgdaDatatype{PHOPLcon}\AgdaSymbol{;} \<[22]%
\>[22]\<%
\\
\>[4]\AgdaIndent{6}{}\<[6]%
\>[6]\AgdaField{parent} \AgdaSymbol{=} \AgdaFunction{PHOPLparent} \AgdaSymbol{\}} \AgdaSymbol{\}}\<%
\end{code}

\AgdaHide{
\begin{code}%
\>\AgdaKeyword{open} \AgdaModule{PHOPLgrammar} \AgdaKeyword{public}\<%
\\
\>\AgdaKeyword{open} \AgdaKeyword{import} \AgdaModule{Grammar} \AgdaFunction{PHOPL} \AgdaKeyword{public}\<%
\\
%
\\
\>\AgdaFunction{Proof} \AgdaSymbol{:} \AgdaDatatype{Alphabet} \AgdaSymbol{→} \AgdaPrimitiveType{Set}\<%
\\
\>\AgdaFunction{Proof} \AgdaBound{V} \AgdaSymbol{=} \AgdaFunction{Expression} \AgdaBound{V} \AgdaFunction{-vProof}\<%
\\
%
\\
\>\AgdaFunction{Term} \AgdaSymbol{:} \AgdaDatatype{Alphabet} \AgdaSymbol{→} \AgdaPrimitiveType{Set}\<%
\\
\>\AgdaFunction{Term} \AgdaBound{V} \AgdaSymbol{=} \AgdaFunction{Expression} \AgdaBound{V} \AgdaFunction{-vTerm}\<%
\\
%
\\
\>\AgdaFunction{Path} \AgdaSymbol{:} \AgdaDatatype{Alphabet} \AgdaSymbol{→} \AgdaPrimitiveType{Set}\<%
\\
\>\AgdaFunction{Path} \AgdaBound{V} \AgdaSymbol{=} \AgdaFunction{Expression} \AgdaBound{V} \AgdaFunction{-vPath}\<%
\\
%
\\
\>\AgdaFunction{Equation} \AgdaSymbol{:} \AgdaDatatype{Alphabet} \AgdaSymbol{→} \AgdaPrimitiveType{Set}\<%
\\
\>\AgdaFunction{Equation} \AgdaBound{V} \AgdaSymbol{=} \AgdaFunction{Expression} \AgdaBound{V} \AgdaFunction{-nvEq}\<%
\\
%
\\
\>\AgdaFunction{ty} \AgdaSymbol{:} \AgdaSymbol{∀} \AgdaSymbol{\{}\AgdaBound{V}\AgdaSymbol{\}} \AgdaSymbol{→} \AgdaDatatype{Type} \AgdaSymbol{→} \AgdaFunction{Expression} \AgdaBound{V} \AgdaSymbol{(}\AgdaInductiveConstructor{nonVarKind} \AgdaInductiveConstructor{-Type}\AgdaSymbol{)}\<%
\\
\>\AgdaFunction{ty} \AgdaBound{A} \AgdaSymbol{=} \AgdaInductiveConstructor{app} \AgdaSymbol{(}\AgdaInductiveConstructor{-ty} \AgdaBound{A}\AgdaSymbol{)} \AgdaInductiveConstructor{[]}\<%
\\
%
\\
\>\AgdaFunction{⊥} \AgdaSymbol{:} \AgdaSymbol{∀} \AgdaSymbol{\{}\AgdaBound{V}\AgdaSymbol{\}} \AgdaSymbol{→} \AgdaFunction{Term} \AgdaBound{V}\<%
\\
\>\AgdaFunction{⊥} \AgdaSymbol{=} \AgdaInductiveConstructor{app} \AgdaInductiveConstructor{-bot} \AgdaInductiveConstructor{[]}\<%
\\
%
\\
\>\AgdaKeyword{infix} \AgdaNumber{65} \AgdaFixityOp{\_⊃\_}\<%
\\
\>\AgdaFunction{\_⊃\_} \AgdaSymbol{:} \AgdaSymbol{∀} \AgdaSymbol{\{}\AgdaBound{V}\AgdaSymbol{\}} \AgdaSymbol{→} \AgdaFunction{Term} \AgdaBound{V} \AgdaSymbol{→} \AgdaFunction{Term} \AgdaBound{V} \AgdaSymbol{→} \AgdaFunction{Term} \AgdaBound{V}\<%
\\
\>\AgdaBound{φ} \AgdaFunction{⊃} \AgdaBound{ψ} \AgdaSymbol{=} \AgdaInductiveConstructor{app} \AgdaInductiveConstructor{-imp} \AgdaSymbol{(}\AgdaBound{φ} \AgdaInductiveConstructor{∷} \AgdaBound{ψ} \AgdaInductiveConstructor{∷} \AgdaInductiveConstructor{[]}\AgdaSymbol{)}\<%
\\
%
\\
\>\AgdaFunction{ΛT} \AgdaSymbol{:} \AgdaSymbol{∀} \AgdaSymbol{\{}\AgdaBound{V}\AgdaSymbol{\}} \AgdaSymbol{→} \AgdaDatatype{Type} \AgdaSymbol{→} \AgdaFunction{Term} \AgdaSymbol{(}\AgdaBound{V} \AgdaInductiveConstructor{,} \AgdaInductiveConstructor{-Term}\AgdaSymbol{)} \AgdaSymbol{→} \AgdaFunction{Term} \AgdaBound{V}\<%
\\
\>\AgdaFunction{ΛT} \AgdaBound{A} \AgdaBound{M} \AgdaSymbol{=} \AgdaInductiveConstructor{app} \AgdaSymbol{(}\AgdaInductiveConstructor{-lamTerm} \AgdaBound{A}\AgdaSymbol{)} \AgdaSymbol{(}\AgdaBound{M} \AgdaInductiveConstructor{∷} \AgdaInductiveConstructor{[]}\AgdaSymbol{)}\<%
\\
%
\\
\>\AgdaFunction{appT} \AgdaSymbol{:} \AgdaSymbol{∀} \AgdaSymbol{\{}\AgdaBound{V}\AgdaSymbol{\}} \AgdaSymbol{→} \AgdaFunction{Term} \AgdaBound{V} \AgdaSymbol{→} \AgdaFunction{Term} \AgdaBound{V} \AgdaSymbol{→} \AgdaFunction{Term} \AgdaBound{V}\<%
\\
\>\AgdaFunction{appT} \AgdaBound{M} \AgdaBound{N} \AgdaSymbol{=} \AgdaInductiveConstructor{app} \AgdaInductiveConstructor{-appTerm} \AgdaSymbol{(}\AgdaBound{M} \AgdaInductiveConstructor{∷} \AgdaBound{N} \AgdaInductiveConstructor{∷} \AgdaInductiveConstructor{[]}\AgdaSymbol{)}\<%
\\
%
\\
\>\AgdaFunction{ΛP} \AgdaSymbol{:} \AgdaSymbol{∀} \AgdaSymbol{\{}\AgdaBound{V}\AgdaSymbol{\}} \AgdaSymbol{→} \AgdaFunction{Term} \AgdaBound{V} \AgdaSymbol{→} \AgdaFunction{Proof} \AgdaSymbol{(}\AgdaBound{V} \AgdaInductiveConstructor{,} \AgdaInductiveConstructor{-Proof}\AgdaSymbol{)} \AgdaSymbol{→} \AgdaFunction{Proof} \AgdaBound{V}\<%
\\
\>\AgdaFunction{ΛP} \AgdaBound{φ} \AgdaBound{δ} \AgdaSymbol{=} \AgdaInductiveConstructor{app} \AgdaInductiveConstructor{-lamProof} \AgdaSymbol{(}\AgdaBound{φ} \AgdaInductiveConstructor{∷} \AgdaBound{δ} \AgdaInductiveConstructor{∷} \AgdaInductiveConstructor{[]}\AgdaSymbol{)}\<%
\\
%
\\
\>\AgdaFunction{appP} \AgdaSymbol{:} \AgdaSymbol{∀} \AgdaSymbol{\{}\AgdaBound{V}\AgdaSymbol{\}} \AgdaSymbol{→} \AgdaFunction{Proof} \AgdaBound{V} \AgdaSymbol{→} \AgdaFunction{Proof} \AgdaBound{V} \AgdaSymbol{→} \AgdaFunction{Proof} \AgdaBound{V}\<%
\\
\>\AgdaFunction{appP} \AgdaBound{δ} \AgdaBound{ε} \AgdaSymbol{=} \AgdaInductiveConstructor{app} \AgdaInductiveConstructor{-appProof} \AgdaSymbol{(}\AgdaBound{δ} \AgdaInductiveConstructor{∷} \AgdaBound{ε} \AgdaInductiveConstructor{∷} \AgdaInductiveConstructor{[]}\AgdaSymbol{)}\<%
\\
%
\\
\>\AgdaFunction{dir} \AgdaSymbol{:} \AgdaSymbol{∀} \AgdaSymbol{\{}\AgdaBound{V}\AgdaSymbol{\}} \AgdaSymbol{→} \AgdaDatatype{Dir} \AgdaSymbol{→} \AgdaFunction{Path} \AgdaBound{V} \AgdaSymbol{→} \AgdaFunction{Proof} \AgdaBound{V}\<%
\\
\>\AgdaFunction{dir} \AgdaBound{d} \AgdaBound{P} \AgdaSymbol{=} \AgdaInductiveConstructor{app} \AgdaSymbol{(}\AgdaInductiveConstructor{-dir} \AgdaBound{d}\AgdaSymbol{)} \AgdaSymbol{(}\AgdaBound{P} \AgdaInductiveConstructor{∷} \AgdaInductiveConstructor{[]}\AgdaSymbol{)}\<%
\\
%
\\
\>\AgdaFunction{plus} \AgdaSymbol{:} \AgdaSymbol{∀} \AgdaSymbol{\{}\AgdaBound{V}\AgdaSymbol{\}} \AgdaSymbol{→} \AgdaFunction{Path} \AgdaBound{V} \AgdaSymbol{→} \AgdaFunction{Proof} \AgdaBound{V}\<%
\\
\>\AgdaFunction{plus} \AgdaBound{P} \AgdaSymbol{=} \AgdaFunction{dir} \AgdaInductiveConstructor{-plus} \AgdaBound{P}\<%
\\
%
\\
\>\AgdaFunction{minus} \AgdaSymbol{:} \AgdaSymbol{∀} \AgdaSymbol{\{}\AgdaBound{V}\AgdaSymbol{\}} \AgdaSymbol{→} \AgdaFunction{Path} \AgdaBound{V} \AgdaSymbol{→} \AgdaFunction{Proof} \AgdaBound{V}\<%
\\
\>\AgdaFunction{minus} \AgdaBound{P} \AgdaSymbol{=} \AgdaFunction{dir} \AgdaInductiveConstructor{-minus} \AgdaBound{P}\<%
\\
%
\\
\>\AgdaFunction{reff} \AgdaSymbol{:} \AgdaSymbol{∀} \AgdaSymbol{\{}\AgdaBound{V}\AgdaSymbol{\}} \AgdaSymbol{→} \AgdaFunction{Term} \AgdaBound{V} \AgdaSymbol{→} \AgdaFunction{Path} \AgdaBound{V}\<%
\\
\>\AgdaFunction{reff} \AgdaBound{M} \AgdaSymbol{=} \AgdaInductiveConstructor{app} \AgdaInductiveConstructor{-ref} \AgdaSymbol{(}\AgdaBound{M} \AgdaInductiveConstructor{∷} \AgdaInductiveConstructor{[]}\AgdaSymbol{)}\<%
\\
%
\\
\>\AgdaKeyword{infix} \AgdaNumber{15} \AgdaFixityOp{\_⊃*\_}\<%
\\
\>\AgdaFunction{\_⊃*\_} \AgdaSymbol{:} \AgdaSymbol{∀} \AgdaSymbol{\{}\AgdaBound{V}\AgdaSymbol{\}} \AgdaSymbol{→} \AgdaFunction{Path} \AgdaBound{V} \AgdaSymbol{→} \AgdaFunction{Path} \AgdaBound{V} \AgdaSymbol{→} \AgdaFunction{Path} \AgdaBound{V}\<%
\\
\>\AgdaBound{P} \AgdaFunction{⊃*} \AgdaBound{Q} \AgdaSymbol{=} \AgdaInductiveConstructor{app} \AgdaInductiveConstructor{-imp*} \AgdaSymbol{(}\AgdaBound{P} \AgdaInductiveConstructor{∷} \AgdaBound{Q} \AgdaInductiveConstructor{∷} \AgdaInductiveConstructor{[]}\AgdaSymbol{)}\<%
\\
%
\\
\>\AgdaFunction{univ} \AgdaSymbol{:} \AgdaSymbol{∀} \AgdaSymbol{\{}\AgdaBound{V}\AgdaSymbol{\}} \AgdaSymbol{→} \AgdaFunction{Term} \AgdaBound{V} \AgdaSymbol{→} \AgdaFunction{Term} \AgdaBound{V} \AgdaSymbol{→} \AgdaFunction{Proof} \AgdaBound{V} \AgdaSymbol{→} \AgdaFunction{Proof} \AgdaBound{V} \AgdaSymbol{→} \AgdaFunction{Path} \AgdaBound{V}\<%
\\
\>\AgdaFunction{univ} \AgdaBound{φ} \AgdaBound{ψ} \AgdaBound{P} \AgdaBound{Q} \AgdaSymbol{=} \AgdaInductiveConstructor{app} \AgdaInductiveConstructor{-univ} \AgdaSymbol{(}\AgdaBound{φ} \AgdaInductiveConstructor{∷} \AgdaBound{ψ} \AgdaInductiveConstructor{∷} \AgdaBound{P} \AgdaInductiveConstructor{∷} \AgdaBound{Q} \AgdaInductiveConstructor{∷} \AgdaInductiveConstructor{[]}\AgdaSymbol{)}\<%
\\
%
\\
\>\AgdaFunction{λλλ} \AgdaSymbol{:} \AgdaSymbol{∀} \AgdaSymbol{\{}\AgdaBound{V}\AgdaSymbol{\}} \AgdaSymbol{→} \AgdaDatatype{Type} \AgdaSymbol{→} \AgdaFunction{Path} \AgdaSymbol{(}\AgdaBound{V} \AgdaInductiveConstructor{,} \AgdaInductiveConstructor{-Term} \AgdaInductiveConstructor{,} \AgdaInductiveConstructor{-Term} \AgdaInductiveConstructor{,} \AgdaInductiveConstructor{-Path}\AgdaSymbol{)} \AgdaSymbol{→} \AgdaFunction{Path} \AgdaBound{V}\<%
\\
\>\AgdaFunction{λλλ} \AgdaBound{A} \AgdaBound{P} \AgdaSymbol{=} \AgdaInductiveConstructor{app} \AgdaSymbol{(}\AgdaInductiveConstructor{-lll} \AgdaBound{A}\AgdaSymbol{)} \AgdaSymbol{(}\AgdaBound{P} \AgdaInductiveConstructor{∷} \AgdaInductiveConstructor{[]}\AgdaSymbol{)}\<%
\\
%
\\
\>\AgdaFunction{app*} \AgdaSymbol{:} \AgdaSymbol{∀} \AgdaSymbol{\{}\AgdaBound{V}\AgdaSymbol{\}} \AgdaSymbol{→} \AgdaFunction{Term} \AgdaBound{V} \AgdaSymbol{→} \AgdaFunction{Term} \AgdaBound{V} \AgdaSymbol{→} \AgdaFunction{Path} \AgdaBound{V} \AgdaSymbol{→} \AgdaFunction{Path} \AgdaBound{V} \AgdaSymbol{→} \AgdaFunction{Path} \AgdaBound{V}\<%
\\
\>\AgdaFunction{app*} \AgdaBound{M} \AgdaBound{N} \AgdaBound{P} \AgdaBound{Q} \AgdaSymbol{=} \AgdaInductiveConstructor{app} \AgdaInductiveConstructor{-app*} \AgdaSymbol{(}\AgdaBound{M} \AgdaInductiveConstructor{∷} \AgdaBound{N} \AgdaInductiveConstructor{∷} \AgdaBound{P} \AgdaInductiveConstructor{∷} \AgdaBound{Q} \AgdaInductiveConstructor{∷} \AgdaInductiveConstructor{[]}\AgdaSymbol{)}\<%
\\
%
\\
\>\AgdaKeyword{infix} \AgdaNumber{60} \AgdaFixityOp{\_≡〈\_〉\_}\<%
\\
\>\AgdaFunction{\_≡〈\_〉\_} \AgdaSymbol{:} \AgdaSymbol{∀} \AgdaSymbol{\{}\AgdaBound{V}\AgdaSymbol{\}} \AgdaSymbol{→} \AgdaFunction{Term} \AgdaBound{V} \AgdaSymbol{→} \AgdaDatatype{Type} \AgdaSymbol{→} \AgdaFunction{Term} \AgdaBound{V} \AgdaSymbol{→} \AgdaFunction{Equation} \AgdaBound{V}\<%
\\
\>\AgdaBound{M} \AgdaFunction{≡〈} \AgdaBound{A} \AgdaFunction{〉} \AgdaBound{N} \AgdaSymbol{=} \AgdaInductiveConstructor{app} \AgdaSymbol{(}\AgdaInductiveConstructor{-eq} \AgdaBound{A}\AgdaSymbol{)} \AgdaSymbol{(}\AgdaBound{M} \AgdaInductiveConstructor{∷} \AgdaBound{N} \AgdaInductiveConstructor{∷} \AgdaInductiveConstructor{[]}\AgdaSymbol{)}\<%
\\
%
\\
\>\AgdaKeyword{infixl} \AgdaNumber{59} \AgdaFixityOp{\_,T\_}\<%
\\
\>\AgdaFunction{\_,T\_} \AgdaSymbol{:} \AgdaSymbol{∀} \AgdaSymbol{\{}\AgdaBound{V}\AgdaSymbol{\}} \AgdaSymbol{→} \AgdaDatatype{Context} \AgdaBound{V} \AgdaSymbol{→} \AgdaDatatype{Type} \AgdaSymbol{→} \AgdaDatatype{Context} \AgdaSymbol{(}\AgdaBound{V} \AgdaInductiveConstructor{,} \AgdaInductiveConstructor{-Term}\AgdaSymbol{)}\<%
\\
\>\AgdaBound{Γ} \AgdaFunction{,T} \AgdaBound{A} \AgdaSymbol{=} \AgdaBound{Γ} \AgdaInductiveConstructor{,} \AgdaFunction{ty} \AgdaBound{A}\<%
\\
%
\\
\>\AgdaKeyword{infixl} \AgdaNumber{59} \AgdaFixityOp{\_,P\_}\<%
\\
\>\AgdaFunction{\_,P\_} \AgdaSymbol{:} \AgdaSymbol{∀} \AgdaSymbol{\{}\AgdaBound{V}\AgdaSymbol{\}} \AgdaSymbol{→} \AgdaDatatype{Context} \AgdaBound{V} \AgdaSymbol{→} \AgdaFunction{Term} \AgdaBound{V} \AgdaSymbol{→} \AgdaDatatype{Context} \AgdaSymbol{(}\AgdaBound{V} \AgdaInductiveConstructor{,} \AgdaInductiveConstructor{-Proof}\AgdaSymbol{)}\<%
\\
\>\AgdaFunction{\_,P\_} \AgdaSymbol{=} \AgdaInductiveConstructor{\_,\_}\<%
\\
%
\\
\>\AgdaKeyword{infixl} \AgdaNumber{59} \AgdaFixityOp{\_,E\_}\<%
\\
\>\AgdaFunction{\_,E\_} \AgdaSymbol{:} \AgdaSymbol{∀} \AgdaSymbol{\{}\AgdaBound{V}\AgdaSymbol{\}} \AgdaSymbol{→} \AgdaDatatype{Context} \AgdaBound{V} \AgdaSymbol{→} \AgdaFunction{Equation} \AgdaBound{V} \AgdaSymbol{→} \AgdaDatatype{Context} \AgdaSymbol{(}\AgdaBound{V} \AgdaInductiveConstructor{,} \AgdaInductiveConstructor{-Path}\AgdaSymbol{)}\<%
\\
\>\AgdaFunction{\_,E\_} \AgdaSymbol{=} \AgdaInductiveConstructor{\_,\_}\<%
\\
%
\\
\>\AgdaFunction{yt} \AgdaSymbol{:} \AgdaSymbol{∀} \AgdaSymbol{\{}\AgdaBound{V}\AgdaSymbol{\}} \AgdaSymbol{→} \AgdaFunction{Expression} \AgdaBound{V} \AgdaSymbol{(}\AgdaInductiveConstructor{nonVarKind} \AgdaInductiveConstructor{-Type}\AgdaSymbol{)} \AgdaSymbol{→} \AgdaDatatype{Type}\<%
\\
\>\AgdaFunction{yt} \AgdaSymbol{(}\AgdaInductiveConstructor{app} \AgdaSymbol{(}\AgdaInductiveConstructor{-ty} \AgdaBound{A}\AgdaSymbol{)} \AgdaInductiveConstructor{[]}\AgdaSymbol{)} \AgdaSymbol{=} \AgdaBound{A}\<%
\\
%
\\
\>\AgdaFunction{ty-yt} \AgdaSymbol{:} \AgdaSymbol{∀} \AgdaSymbol{\{}\AgdaBound{V}\AgdaSymbol{\}} \AgdaSymbol{\{}\AgdaBound{A} \AgdaSymbol{:} \AgdaFunction{Expression} \AgdaBound{V} \AgdaSymbol{(}\AgdaInductiveConstructor{nonVarKind} \AgdaInductiveConstructor{-Type}\AgdaSymbol{)\}} \AgdaSymbol{→} \AgdaFunction{ty} \AgdaSymbol{(}\AgdaFunction{yt} \AgdaBound{A}\AgdaSymbol{)} \AgdaDatatype{≡} \AgdaBound{A}\<%
\\
\>\AgdaFunction{ty-yt} \AgdaSymbol{\{}\AgdaArgument{A} \AgdaSymbol{=} \AgdaInductiveConstructor{app} \AgdaSymbol{(}\AgdaInductiveConstructor{-ty} \AgdaSymbol{\_)} \AgdaInductiveConstructor{[]}\AgdaSymbol{\}} \AgdaSymbol{=} \AgdaInductiveConstructor{refl} \AgdaComment{--TODO Remove?}\<%
\\
%
\\
\>\AgdaFunction{appT-injl} \AgdaSymbol{:} \AgdaSymbol{∀} \AgdaSymbol{\{}\AgdaBound{V}\AgdaSymbol{\}} \AgdaSymbol{\{}\AgdaBound{M} \AgdaBound{M'} \AgdaBound{N} \AgdaBound{N'} \AgdaSymbol{:} \AgdaFunction{Term} \AgdaBound{V}\AgdaSymbol{\}} \AgdaSymbol{→} \AgdaFunction{appT} \AgdaBound{M} \AgdaBound{N} \AgdaDatatype{≡} \AgdaFunction{appT} \AgdaBound{M'} \AgdaBound{N'} \AgdaSymbol{→} \AgdaBound{M} \AgdaDatatype{≡} \AgdaBound{M'}\<%
\\
\>\AgdaFunction{appT-injl} \AgdaInductiveConstructor{refl} \AgdaSymbol{=} \AgdaInductiveConstructor{refl}\<%
\\
%
\\
\>\AgdaFunction{Pi} \AgdaSymbol{:} \AgdaSymbol{∀} \AgdaSymbol{\{}\AgdaBound{n}\AgdaSymbol{\}} \AgdaSymbol{→} \AgdaDatatype{snocVec} \AgdaDatatype{Type} \AgdaBound{n} \AgdaSymbol{→} \AgdaDatatype{Type} \AgdaSymbol{→} \AgdaDatatype{Type}\<%
\\
\>\AgdaFunction{Pi} \AgdaInductiveConstructor{[]} \AgdaBound{B} \AgdaSymbol{=} \AgdaBound{B}\<%
\\
\>\AgdaFunction{Pi} \AgdaSymbol{(}\AgdaBound{AA} \AgdaInductiveConstructor{snoc} \AgdaBound{A}\AgdaSymbol{)} \AgdaBound{B} \AgdaSymbol{=} \AgdaFunction{Pi} \AgdaBound{AA} \AgdaSymbol{(}\AgdaBound{A} \AgdaInductiveConstructor{⇛} \AgdaBound{B}\AgdaSymbol{)}\<%
\\
%
\\
\>\AgdaFunction{APP} \AgdaSymbol{:} \AgdaSymbol{∀} \AgdaSymbol{\{}\AgdaBound{V} \AgdaBound{n}\AgdaSymbol{\}} \AgdaSymbol{→} \AgdaFunction{Term} \AgdaBound{V} \AgdaSymbol{→} \AgdaDatatype{snocVec} \AgdaSymbol{(}\AgdaFunction{Term} \AgdaBound{V}\AgdaSymbol{)} \AgdaBound{n} \AgdaSymbol{→} \AgdaFunction{Term} \AgdaBound{V}\<%
\\
\>\AgdaFunction{APP} \AgdaBound{M} \AgdaInductiveConstructor{[]} \AgdaSymbol{=} \AgdaBound{M}\<%
\\
\>\AgdaFunction{APP} \AgdaBound{M} \AgdaSymbol{(}\AgdaBound{NN} \AgdaInductiveConstructor{snoc} \AgdaBound{N}\AgdaSymbol{)} \AgdaSymbol{=} \AgdaFunction{appT} \AgdaSymbol{(}\AgdaFunction{APP} \AgdaBound{M} \AgdaBound{NN}\AgdaSymbol{)} \AgdaBound{N}\<%
\\
%
\\
\>\AgdaKeyword{postulate} \AgdaPostulate{APP-rep} \AgdaSymbol{:} \AgdaSymbol{∀} \AgdaSymbol{\{}\AgdaBound{U} \AgdaBound{V} \AgdaBound{n} \AgdaBound{M}\AgdaSymbol{\}} \AgdaSymbol{(}\AgdaBound{NN} \AgdaSymbol{:} \AgdaDatatype{snocVec} \AgdaSymbol{(}\AgdaFunction{Term} \AgdaBound{U}\AgdaSymbol{)} \AgdaBound{n}\AgdaSymbol{)} \AgdaSymbol{\{}\AgdaBound{ρ} \AgdaSymbol{:} \AgdaFunction{Rep} \AgdaBound{U} \AgdaBound{V}\AgdaSymbol{\}} \AgdaSymbol{→}\<%
\\
\>[6]\AgdaIndent{18}{}\<[18]%
\>[18]\AgdaSymbol{(}\AgdaFunction{APP} \AgdaBound{M} \AgdaBound{NN}\AgdaSymbol{)} \AgdaFunction{〈} \AgdaBound{ρ} \AgdaFunction{〉} \AgdaDatatype{≡} \AgdaFunction{APP} \AgdaSymbol{(}\AgdaBound{M} \AgdaFunction{〈} \AgdaBound{ρ} \AgdaFunction{〉}\AgdaSymbol{)} \AgdaSymbol{(}\AgdaFunction{snocVec-rep} \AgdaBound{NN} \AgdaBound{ρ}\AgdaSymbol{)}\<%
\\
%
\\
\>\AgdaFunction{APPP} \AgdaSymbol{:} \AgdaSymbol{∀} \AgdaSymbol{\{}\AgdaBound{V}\AgdaSymbol{\}} \AgdaSymbol{\{}\AgdaBound{n}\AgdaSymbol{\}} \AgdaSymbol{→} \AgdaFunction{Proof} \AgdaBound{V} \AgdaSymbol{→} \AgdaDatatype{snocVec} \AgdaSymbol{(}\AgdaFunction{Proof} \AgdaBound{V}\AgdaSymbol{)} \AgdaBound{n} \AgdaSymbol{→} \AgdaFunction{Proof} \AgdaBound{V}\<%
\\
\>\AgdaFunction{APPP} \AgdaBound{δ} \AgdaInductiveConstructor{[]} \AgdaSymbol{=} \AgdaBound{δ}\<%
\\
\>\AgdaFunction{APPP} \AgdaBound{δ} \AgdaSymbol{(}\AgdaBound{εε} \AgdaInductiveConstructor{snoc} \AgdaBound{ε}\AgdaSymbol{)} \AgdaSymbol{=} \AgdaFunction{appP} \AgdaSymbol{(}\AgdaFunction{APPP} \AgdaBound{δ} \AgdaBound{εε}\AgdaSymbol{)} \AgdaBound{ε}\<%
\\
%
\\
\>\AgdaFunction{APPP-rep} \AgdaSymbol{:} \AgdaSymbol{∀} \AgdaSymbol{\{}\AgdaBound{U} \AgdaBound{V} \AgdaBound{n} \AgdaBound{δ}\AgdaSymbol{\}} \AgdaSymbol{(}\AgdaBound{εε} \AgdaSymbol{:} \AgdaDatatype{snocVec} \AgdaSymbol{(}\AgdaFunction{Proof} \AgdaBound{U}\AgdaSymbol{)} \AgdaBound{n}\AgdaSymbol{)} \AgdaSymbol{\{}\AgdaBound{ρ} \AgdaSymbol{:} \AgdaFunction{Rep} \AgdaBound{U} \AgdaBound{V}\AgdaSymbol{\}} \AgdaSymbol{→}\<%
\\
\>[0]\AgdaIndent{2}{}\<[2]%
\>[2]\AgdaSymbol{(}\AgdaFunction{APPP} \AgdaBound{δ} \AgdaBound{εε}\AgdaSymbol{)} \AgdaFunction{〈} \AgdaBound{ρ} \AgdaFunction{〉} \AgdaDatatype{≡} \AgdaFunction{APPP} \AgdaSymbol{(}\AgdaBound{δ} \AgdaFunction{〈} \AgdaBound{ρ} \AgdaFunction{〉}\AgdaSymbol{)} \AgdaSymbol{(}\AgdaFunction{snocVec-rep} \AgdaBound{εε} \AgdaBound{ρ}\AgdaSymbol{)}\<%
\\
\>\AgdaFunction{APPP-rep} \AgdaInductiveConstructor{[]} \AgdaSymbol{=} \AgdaInductiveConstructor{refl}\<%
\\
\>\AgdaFunction{APPP-rep} \AgdaSymbol{(}\AgdaBound{εε} \AgdaInductiveConstructor{snoc} \AgdaBound{ε}\AgdaSymbol{)} \AgdaSymbol{\{}\AgdaBound{ρ}\AgdaSymbol{\}} \AgdaSymbol{=} \AgdaFunction{cong} \AgdaSymbol{(λ} \AgdaBound{x} \AgdaSymbol{→} \AgdaFunction{appP} \AgdaBound{x} \AgdaSymbol{(}\AgdaBound{ε} \AgdaFunction{〈} \AgdaBound{ρ} \AgdaFunction{〉}\AgdaSymbol{))} \AgdaSymbol{(}\AgdaFunction{APPP-rep} \AgdaBound{εε}\AgdaSymbol{)}\<%
\\
%
\\
\>\AgdaFunction{APP*} \AgdaSymbol{:} \AgdaSymbol{∀} \AgdaSymbol{\{}\AgdaBound{V} \AgdaBound{n}\AgdaSymbol{\}} \AgdaSymbol{→} \AgdaDatatype{snocVec} \AgdaSymbol{(}\AgdaFunction{Term} \AgdaBound{V}\AgdaSymbol{)} \AgdaBound{n} \AgdaSymbol{→} \AgdaDatatype{snocVec} \AgdaSymbol{(}\AgdaFunction{Term} \AgdaBound{V}\AgdaSymbol{)} \AgdaBound{n} \AgdaSymbol{→} \AgdaFunction{Path} \AgdaBound{V} \AgdaSymbol{→} \AgdaDatatype{snocVec} \AgdaSymbol{(}\AgdaFunction{Path} \AgdaBound{V}\AgdaSymbol{)} \AgdaBound{n} \AgdaSymbol{→} \AgdaFunction{Path} \AgdaBound{V}\<%
\\
\>\AgdaFunction{APP*} \AgdaInductiveConstructor{[]} \AgdaInductiveConstructor{[]} \AgdaBound{P} \AgdaInductiveConstructor{[]} \AgdaSymbol{=} \AgdaBound{P}\<%
\\
\>\AgdaFunction{APP*} \AgdaSymbol{(}\AgdaBound{MM} \AgdaInductiveConstructor{snoc} \AgdaBound{M}\AgdaSymbol{)} \AgdaSymbol{(}\AgdaBound{NN} \AgdaInductiveConstructor{snoc} \AgdaBound{N}\AgdaSymbol{)} \AgdaBound{P} \AgdaSymbol{(}\AgdaBound{QQ} \AgdaInductiveConstructor{snoc} \AgdaBound{Q}\AgdaSymbol{)} \AgdaSymbol{=} \AgdaFunction{app*} \AgdaBound{M} \AgdaBound{N} \AgdaSymbol{(}\AgdaFunction{APP*} \AgdaBound{MM} \AgdaBound{NN} \AgdaBound{P} \AgdaBound{QQ}\AgdaSymbol{)} \AgdaBound{Q}\<%
\\
%
\\
\>\AgdaFunction{APP*-rep} \AgdaSymbol{:} \AgdaSymbol{∀} \AgdaSymbol{\{}\AgdaBound{U} \AgdaBound{V} \AgdaBound{n}\AgdaSymbol{\}} \AgdaBound{MM} \AgdaSymbol{\{}\AgdaBound{NN} \AgdaSymbol{:} \AgdaDatatype{snocVec} \AgdaSymbol{(}\AgdaFunction{Term} \AgdaBound{U}\AgdaSymbol{)} \AgdaBound{n}\AgdaSymbol{\}} \AgdaSymbol{\{}\AgdaBound{P} \AgdaBound{QQ}\AgdaSymbol{\}} \AgdaSymbol{\{}\AgdaBound{ρ} \AgdaSymbol{:} \AgdaFunction{Rep} \AgdaBound{U} \AgdaBound{V}\AgdaSymbol{\}} \AgdaSymbol{→}\<%
\\
\>[0]\AgdaIndent{2}{}\<[2]%
\>[2]\AgdaSymbol{(}\AgdaFunction{APP*} \AgdaBound{MM} \AgdaBound{NN} \AgdaBound{P} \AgdaBound{QQ}\AgdaSymbol{)} \AgdaFunction{〈} \AgdaBound{ρ} \AgdaFunction{〉} \AgdaDatatype{≡} \AgdaFunction{APP*} \AgdaSymbol{(}\AgdaFunction{snocVec-rep} \AgdaBound{MM} \AgdaBound{ρ}\AgdaSymbol{)} \AgdaSymbol{(}\AgdaFunction{snocVec-rep} \AgdaBound{NN} \AgdaBound{ρ}\AgdaSymbol{)} \AgdaSymbol{(}\AgdaBound{P} \AgdaFunction{〈} \AgdaBound{ρ} \AgdaFunction{〉}\AgdaSymbol{)} \AgdaSymbol{(}\AgdaFunction{snocVec-rep} \AgdaBound{QQ} \AgdaBound{ρ}\AgdaSymbol{)}\<%
\\
\>\AgdaFunction{APP*-rep} \AgdaInductiveConstructor{[]} \AgdaSymbol{\{}\AgdaInductiveConstructor{[]}\AgdaSymbol{\}} \AgdaSymbol{\{}\AgdaArgument{QQ} \AgdaSymbol{=} \AgdaInductiveConstructor{[]}\AgdaSymbol{\}} \AgdaSymbol{=} \AgdaInductiveConstructor{refl}\<%
\\
\>\AgdaFunction{APP*-rep} \AgdaSymbol{(}\AgdaBound{MM} \AgdaInductiveConstructor{snoc} \AgdaBound{M}\AgdaSymbol{)} \AgdaSymbol{\{}\AgdaBound{NN} \AgdaInductiveConstructor{snoc} \AgdaBound{N}\AgdaSymbol{\}} \AgdaSymbol{\{}\AgdaArgument{QQ} \AgdaSymbol{=} \AgdaBound{QQ} \AgdaInductiveConstructor{snoc} \AgdaBound{Q}\AgdaSymbol{\}} \AgdaSymbol{\{}\AgdaArgument{ρ} \AgdaSymbol{=} \AgdaBound{ρ}\AgdaSymbol{\}} \AgdaSymbol{=} \<[60]%
\>[60]\<%
\\
\>[0]\AgdaIndent{2}{}\<[2]%
\>[2]\AgdaFunction{cong} \AgdaSymbol{(λ} \AgdaBound{x} \AgdaSymbol{→} \AgdaFunction{app*} \AgdaSymbol{(}\AgdaBound{M} \AgdaFunction{〈} \AgdaBound{ρ} \AgdaFunction{〉}\AgdaSymbol{)} \AgdaSymbol{(}\AgdaBound{N} \AgdaFunction{〈} \AgdaBound{ρ} \AgdaFunction{〉}\AgdaSymbol{)} \AgdaBound{x} \AgdaSymbol{(}\AgdaBound{Q} \AgdaFunction{〈} \AgdaBound{ρ} \AgdaFunction{〉}\AgdaSymbol{))} \AgdaSymbol{(}\AgdaFunction{APP*-rep} \AgdaBound{MM}\AgdaSymbol{)}\<%
\\
%
\\
\>\AgdaFunction{typeof'} \AgdaSymbol{:} \AgdaSymbol{∀} \AgdaSymbol{\{}\AgdaBound{V}\AgdaSymbol{\}} \AgdaSymbol{→} \AgdaDatatype{Var} \AgdaBound{V} \AgdaInductiveConstructor{-Term} \AgdaSymbol{→} \AgdaDatatype{Context} \AgdaBound{V} \AgdaSymbol{→} \AgdaDatatype{Type}\<%
\\
\>\AgdaFunction{typeof'} \AgdaBound{x} \AgdaBound{Γ} \<[13]%
\>[13]\AgdaSymbol{=} \AgdaFunction{yt} \AgdaSymbol{(}\AgdaFunction{typeof} \AgdaBound{x} \AgdaBound{Γ}\AgdaSymbol{)}\<%
\\
%
\\
\>\AgdaFunction{typeof-typeof'} \AgdaSymbol{:} \AgdaSymbol{∀} \AgdaSymbol{\{}\AgdaBound{V}\AgdaSymbol{\}} \AgdaSymbol{\{}\AgdaBound{x} \AgdaSymbol{:} \AgdaDatatype{Var} \AgdaBound{V} \AgdaInductiveConstructor{-Term}\AgdaSymbol{\}} \AgdaSymbol{\{}\AgdaBound{Γ}\AgdaSymbol{\}} \AgdaSymbol{→} \AgdaFunction{typeof} \AgdaBound{x} \AgdaBound{Γ} \AgdaDatatype{≡} \AgdaFunction{ty} \AgdaSymbol{(}\AgdaFunction{typeof'} \AgdaBound{x} \AgdaBound{Γ}\AgdaSymbol{)}\<%
\\
\>\AgdaFunction{typeof-typeof'} \AgdaSymbol{=} \AgdaFunction{sym} \AgdaFunction{ty-yt} \AgdaComment{-- TODO Remove?}\<%
\\
%
\\
\>\AgdaFunction{addpath} \AgdaSymbol{:} \AgdaSymbol{∀} \AgdaSymbol{\{}\AgdaBound{V}\AgdaSymbol{\}} \AgdaSymbol{→} \AgdaDatatype{Context} \AgdaBound{V} \AgdaSymbol{→} \AgdaDatatype{Type} \AgdaSymbol{→} \AgdaDatatype{Context} \AgdaSymbol{(}\AgdaBound{V} \AgdaInductiveConstructor{,} \AgdaInductiveConstructor{-Term} \AgdaInductiveConstructor{,} \AgdaInductiveConstructor{-Term} \AgdaInductiveConstructor{,} \AgdaInductiveConstructor{-Path}\AgdaSymbol{)}\<%
\\
\>\AgdaFunction{addpath} \AgdaBound{Γ} \AgdaBound{A} \AgdaSymbol{=} \AgdaBound{Γ} \AgdaFunction{,T} \AgdaBound{A} \AgdaFunction{,T} \AgdaBound{A} \AgdaFunction{,E} \AgdaInductiveConstructor{var} \AgdaFunction{x₁} \AgdaFunction{≡〈} \AgdaBound{A} \AgdaFunction{〉} \AgdaInductiveConstructor{var} \AgdaInductiveConstructor{x₀}\<%
\\
%
\\
\>\AgdaFunction{sub↖} \AgdaSymbol{:} \AgdaSymbol{∀} \AgdaSymbol{\{}\AgdaBound{U}\AgdaSymbol{\}} \AgdaSymbol{\{}\AgdaBound{V}\AgdaSymbol{\}} \AgdaSymbol{→} \AgdaFunction{Sub} \AgdaBound{U} \AgdaBound{V} \AgdaSymbol{→} \AgdaFunction{Sub} \AgdaSymbol{(}\AgdaBound{U} \AgdaInductiveConstructor{,} \AgdaInductiveConstructor{-Term}\AgdaSymbol{)} \AgdaSymbol{(}\AgdaBound{V} \AgdaInductiveConstructor{,} \AgdaInductiveConstructor{-Term} \AgdaInductiveConstructor{,} \AgdaInductiveConstructor{-Term} \AgdaInductiveConstructor{,} \AgdaInductiveConstructor{-Path}\AgdaSymbol{)}\<%
\\
\>\AgdaFunction{sub↖} \AgdaBound{σ} \AgdaSymbol{\_} \AgdaInductiveConstructor{x₀} \AgdaSymbol{=} \AgdaInductiveConstructor{var} \AgdaFunction{x₂}\<%
\\
\>\AgdaFunction{sub↖} \AgdaBound{σ} \AgdaSymbol{\_} \AgdaSymbol{(}\AgdaInductiveConstructor{↑} \AgdaBound{x}\AgdaSymbol{)} \AgdaSymbol{=} \AgdaBound{σ} \AgdaSymbol{\_} \AgdaBound{x} \AgdaFunction{⇑} \AgdaFunction{⇑} \AgdaFunction{⇑}\<%
\\
%
\\
\>\AgdaKeyword{postulate} \AgdaPostulate{sub↖-cong} \AgdaSymbol{:} \AgdaSymbol{∀} \AgdaSymbol{\{}\AgdaBound{U}\AgdaSymbol{\}} \AgdaSymbol{\{}\AgdaBound{V}\AgdaSymbol{\}} \AgdaSymbol{\{}\AgdaBound{ρ} \AgdaBound{σ} \AgdaSymbol{:} \AgdaFunction{Sub} \AgdaBound{U} \AgdaBound{V}\AgdaSymbol{\}} \AgdaSymbol{→} \AgdaBound{ρ} \AgdaFunction{∼} \AgdaBound{σ} \AgdaSymbol{→} \AgdaFunction{sub↖} \AgdaBound{ρ} \AgdaFunction{∼} \AgdaFunction{sub↖} \AgdaBound{σ}\<%
\\
%
\\
\>\AgdaKeyword{postulate} \AgdaPostulate{sub↖-comp₁} \AgdaSymbol{:} \AgdaSymbol{∀} \AgdaSymbol{\{}\AgdaBound{U}\AgdaSymbol{\}} \AgdaSymbol{\{}\AgdaBound{V}\AgdaSymbol{\}} \AgdaSymbol{\{}\AgdaBound{W}\AgdaSymbol{\}} \AgdaSymbol{\{}\AgdaBound{ρ} \AgdaSymbol{:} \AgdaFunction{Rep} \AgdaBound{V} \AgdaBound{W}\AgdaSymbol{\}} \AgdaSymbol{\{}\AgdaBound{σ} \AgdaSymbol{:} \AgdaFunction{Sub} \AgdaBound{U} \AgdaBound{V}\AgdaSymbol{\}} \AgdaSymbol{→}\<%
\\
\>[2]\AgdaIndent{21}{}\<[21]%
\>[21]\AgdaFunction{sub↖} \AgdaSymbol{(}\AgdaBound{ρ} \AgdaFunction{•RS} \AgdaBound{σ}\AgdaSymbol{)} \AgdaFunction{∼} \AgdaFunction{liftRep} \AgdaInductiveConstructor{-Path} \AgdaSymbol{(}\AgdaFunction{liftRep} \AgdaInductiveConstructor{-Term} \AgdaSymbol{(}\AgdaFunction{liftRep} \AgdaInductiveConstructor{-Term} \AgdaBound{ρ}\AgdaSymbol{))} \AgdaFunction{•RS} \AgdaFunction{sub↖} \AgdaBound{σ}\<%
\\
%
\\
\>\AgdaFunction{sub↗} \AgdaSymbol{:} \AgdaSymbol{∀} \AgdaSymbol{\{}\AgdaBound{U}\AgdaSymbol{\}} \AgdaSymbol{\{}\AgdaBound{V}\AgdaSymbol{\}} \AgdaSymbol{→} \AgdaFunction{Sub} \AgdaBound{U} \AgdaBound{V} \AgdaSymbol{→} \AgdaFunction{Sub} \AgdaSymbol{(}\AgdaBound{U} \AgdaInductiveConstructor{,} \AgdaInductiveConstructor{-Term}\AgdaSymbol{)} \AgdaSymbol{(}\AgdaBound{V} \AgdaInductiveConstructor{,} \AgdaInductiveConstructor{-Term} \AgdaInductiveConstructor{,} \AgdaInductiveConstructor{-Term} \AgdaInductiveConstructor{,} \AgdaInductiveConstructor{-Path}\AgdaSymbol{)}\<%
\\
\>\AgdaFunction{sub↗} \AgdaBound{σ} \AgdaSymbol{\_} \AgdaInductiveConstructor{x₀} \AgdaSymbol{=} \AgdaInductiveConstructor{var} \AgdaFunction{x₁}\<%
\\
\>\AgdaFunction{sub↗} \AgdaBound{σ} \AgdaSymbol{\_} \AgdaSymbol{(}\AgdaInductiveConstructor{↑} \AgdaBound{x}\AgdaSymbol{)} \AgdaSymbol{=} \AgdaBound{σ} \AgdaSymbol{\_} \AgdaBound{x} \AgdaFunction{⇑} \AgdaFunction{⇑} \AgdaFunction{⇑}\<%
\\
%
\\
\>\AgdaKeyword{postulate} \AgdaPostulate{sub↗-cong} \AgdaSymbol{:} \AgdaSymbol{∀} \AgdaSymbol{\{}\AgdaBound{U}\AgdaSymbol{\}} \AgdaSymbol{\{}\AgdaBound{V}\AgdaSymbol{\}} \AgdaSymbol{\{}\AgdaBound{ρ} \AgdaBound{σ} \AgdaSymbol{:} \AgdaFunction{Sub} \AgdaBound{U} \AgdaBound{V}\AgdaSymbol{\}} \AgdaSymbol{→} \AgdaBound{ρ} \AgdaFunction{∼} \AgdaBound{σ} \AgdaSymbol{→} \AgdaFunction{sub↗} \AgdaBound{ρ} \AgdaFunction{∼} \AgdaFunction{sub↗} \AgdaBound{σ}\<%
\\
%
\\
\>\AgdaKeyword{postulate} \AgdaPostulate{sub↗-comp₁} \AgdaSymbol{:} \AgdaSymbol{∀} \AgdaSymbol{\{}\AgdaBound{U}\AgdaSymbol{\}} \AgdaSymbol{\{}\AgdaBound{V}\AgdaSymbol{\}} \AgdaSymbol{\{}\AgdaBound{W}\AgdaSymbol{\}} \AgdaSymbol{\{}\AgdaBound{ρ} \AgdaSymbol{:} \AgdaFunction{Rep} \AgdaBound{V} \AgdaBound{W}\AgdaSymbol{\}} \AgdaSymbol{\{}\AgdaBound{σ} \AgdaSymbol{:} \AgdaFunction{Sub} \AgdaBound{U} \AgdaBound{V}\AgdaSymbol{\}} \AgdaSymbol{→}\<%
\\
\>[2]\AgdaIndent{21}{}\<[21]%
\>[21]\AgdaFunction{sub↗} \AgdaSymbol{(}\AgdaBound{ρ} \AgdaFunction{•RS} \AgdaBound{σ}\AgdaSymbol{)} \AgdaFunction{∼} \AgdaFunction{liftRep} \AgdaInductiveConstructor{-Path} \AgdaSymbol{(}\AgdaFunction{liftRep} \AgdaInductiveConstructor{-Term} \AgdaSymbol{(}\AgdaFunction{liftRep} \AgdaInductiveConstructor{-Term} \AgdaBound{ρ}\AgdaSymbol{))} \AgdaFunction{•RS} \AgdaFunction{sub↗} \AgdaBound{σ}\<%
\\
%
\\
\>\AgdaComment{--REFACTOR Duplication}\<%
\\
%
\\
\>\AgdaFunction{var-not-Λ} \AgdaSymbol{:} \AgdaSymbol{∀} \AgdaSymbol{\{}\AgdaBound{V}\AgdaSymbol{\}} \AgdaSymbol{\{}\AgdaBound{x} \AgdaSymbol{:} \AgdaDatatype{Var} \AgdaBound{V} \AgdaInductiveConstructor{-Term}\AgdaSymbol{\}} \AgdaSymbol{\{}\AgdaBound{A}\AgdaSymbol{\}} \AgdaSymbol{\{}\AgdaBound{M} \AgdaSymbol{:} \AgdaFunction{Term} \AgdaSymbol{(}\AgdaBound{V} \AgdaInductiveConstructor{,} \AgdaInductiveConstructor{-Term}\AgdaSymbol{)\}} \AgdaSymbol{→} \AgdaInductiveConstructor{var} \AgdaBound{x} \AgdaDatatype{≡} \AgdaFunction{ΛT} \AgdaBound{A} \AgdaBound{M} \AgdaSymbol{→} \AgdaDatatype{Empty}\<%
\\
\>\AgdaFunction{var-not-Λ} \AgdaSymbol{()}\<%
\\
%
\\
\>\AgdaFunction{app-not-Λ} \AgdaSymbol{:} \AgdaSymbol{∀} \AgdaSymbol{\{}\AgdaBound{V}\AgdaSymbol{\}} \AgdaSymbol{\{}\AgdaBound{M} \AgdaBound{N} \AgdaSymbol{:} \AgdaFunction{Term} \AgdaBound{V}\AgdaSymbol{\}} \AgdaSymbol{\{}\AgdaBound{A}\AgdaSymbol{\}} \AgdaSymbol{\{}\AgdaBound{P} \AgdaSymbol{:} \AgdaFunction{Term} \AgdaSymbol{(}\AgdaBound{V} \AgdaInductiveConstructor{,} \AgdaInductiveConstructor{-Term}\AgdaSymbol{)\}} \AgdaSymbol{→} \AgdaFunction{appT} \AgdaBound{M} \AgdaBound{N} \AgdaDatatype{≡} \AgdaFunction{ΛT} \AgdaBound{A} \AgdaBound{P} \AgdaSymbol{→} \AgdaDatatype{Empty}\<%
\\
\>\AgdaFunction{app-not-Λ} \AgdaSymbol{()}\<%
\\
%
\\
\>\AgdaFunction{appT-injr} \AgdaSymbol{:} \AgdaSymbol{∀} \AgdaSymbol{\{}\AgdaBound{V}\AgdaSymbol{\}} \AgdaSymbol{\{}\AgdaBound{M} \AgdaBound{N} \AgdaBound{P} \AgdaBound{Q} \AgdaSymbol{:} \AgdaFunction{Term} \AgdaBound{V}\AgdaSymbol{\}} \AgdaSymbol{→} \AgdaFunction{appT} \AgdaBound{M} \AgdaBound{N} \AgdaDatatype{≡} \AgdaFunction{appT} \AgdaBound{P} \AgdaBound{Q} \AgdaSymbol{→} \AgdaBound{N} \AgdaDatatype{≡} \AgdaBound{Q}\<%
\\
\>\AgdaFunction{appT-injr} \AgdaInductiveConstructor{refl} \AgdaSymbol{=} \AgdaInductiveConstructor{refl}\<%
\\
%
\\
\>\AgdaFunction{imp-injl} \AgdaSymbol{:} \AgdaSymbol{∀} \AgdaSymbol{\{}\AgdaBound{V}\AgdaSymbol{\}} \AgdaSymbol{\{}\AgdaBound{φ} \AgdaBound{φ'} \AgdaBound{ψ} \AgdaBound{ψ'} \AgdaSymbol{:} \AgdaFunction{Term} \AgdaBound{V}\AgdaSymbol{\}} \AgdaSymbol{→} \AgdaBound{φ} \AgdaFunction{⊃} \AgdaBound{ψ} \AgdaDatatype{≡} \AgdaBound{φ'} \AgdaFunction{⊃} \AgdaBound{ψ'} \AgdaSymbol{→} \AgdaBound{φ} \AgdaDatatype{≡} \AgdaBound{φ'}\<%
\\
\>\AgdaFunction{imp-injl} \AgdaInductiveConstructor{refl} \AgdaSymbol{=} \AgdaInductiveConstructor{refl}\<%
\\
%
\\
\>\AgdaFunction{imp-injr} \AgdaSymbol{:} \AgdaSymbol{∀} \AgdaSymbol{\{}\AgdaBound{V}\AgdaSymbol{\}} \AgdaSymbol{\{}\AgdaBound{φ} \AgdaBound{φ'} \AgdaBound{ψ} \AgdaBound{ψ'} \AgdaSymbol{:} \AgdaFunction{Term} \AgdaBound{V}\AgdaSymbol{\}} \AgdaSymbol{→} \AgdaBound{φ} \AgdaFunction{⊃} \AgdaBound{ψ} \AgdaDatatype{≡} \AgdaBound{φ'} \AgdaFunction{⊃} \AgdaBound{ψ'} \AgdaSymbol{→} \AgdaBound{ψ} \AgdaDatatype{≡} \AgdaBound{ψ'}\<%
\\
\>\AgdaFunction{imp-injr} \AgdaInductiveConstructor{refl} \AgdaSymbol{=} \AgdaInductiveConstructor{refl}\<%
\\
\>\AgdaComment{--REFACTOR General pattern}\<%
\\
%
\\
\>\AgdaFunction{toSnocTypes} \AgdaSymbol{:} \AgdaSymbol{∀} \AgdaSymbol{\{}\AgdaBound{V} \AgdaBound{n}\AgdaSymbol{\}} \AgdaSymbol{→} \AgdaDatatype{snocVec} \AgdaDatatype{Type} \AgdaBound{n} \AgdaSymbol{→} \AgdaDatatype{snocTypes} \AgdaBound{V} \AgdaSymbol{(}\AgdaFunction{replicate} \AgdaBound{n} \AgdaInductiveConstructor{-Term}\AgdaSymbol{)}\<%
\\
\>\AgdaFunction{toSnocTypes} \AgdaInductiveConstructor{[]} \AgdaSymbol{=} \AgdaInductiveConstructor{[]}\<%
\\
\>\AgdaFunction{toSnocTypes} \AgdaSymbol{(}\AgdaBound{AA} \AgdaInductiveConstructor{snoc} \AgdaBound{A}\AgdaSymbol{)} \AgdaSymbol{=} \AgdaFunction{toSnocTypes} \AgdaBound{AA} \AgdaInductiveConstructor{snoc} \AgdaFunction{ty} \AgdaBound{A}\<%
\\
%
\\
\>\AgdaFunction{toSnocTypes-rep} \AgdaSymbol{:} \AgdaSymbol{∀} \AgdaSymbol{\{}\AgdaBound{U} \AgdaBound{V} \AgdaBound{n}\AgdaSymbol{\}} \AgdaSymbol{\{}\AgdaBound{AA} \AgdaSymbol{:} \AgdaDatatype{snocVec} \AgdaDatatype{Type} \AgdaBound{n}\AgdaSymbol{\}} \AgdaSymbol{\{}\AgdaBound{ρ} \AgdaSymbol{:} \AgdaFunction{Rep} \AgdaBound{U} \AgdaBound{V}\AgdaSymbol{\}} \AgdaSymbol{→} \AgdaFunction{snocTypes-rep} \AgdaSymbol{(}\AgdaFunction{toSnocTypes} \AgdaBound{AA}\AgdaSymbol{)} \AgdaBound{ρ} \AgdaDatatype{≡} \AgdaFunction{toSnocTypes} \AgdaBound{AA}\<%
\\
\>\AgdaFunction{toSnocTypes-rep} \AgdaSymbol{\{}\AgdaArgument{AA} \AgdaSymbol{=} \AgdaInductiveConstructor{[]}\AgdaSymbol{\}} \AgdaSymbol{=} \AgdaInductiveConstructor{refl}\<%
\\
\>\AgdaFunction{toSnocTypes-rep} \AgdaSymbol{\{}\AgdaArgument{AA} \AgdaSymbol{=} \AgdaBound{AA} \AgdaInductiveConstructor{snoc} \AgdaBound{A}\AgdaSymbol{\}} \AgdaSymbol{=} \AgdaFunction{cong} \AgdaSymbol{(λ} \AgdaBound{x} \AgdaSymbol{→} \AgdaBound{x} \AgdaInductiveConstructor{snoc} \AgdaFunction{ty} \AgdaBound{A}\AgdaSymbol{)} \AgdaFunction{toSnocTypes-rep}\<%
\\
%
\\
\>\AgdaFunction{eqmult} \AgdaSymbol{:} \AgdaSymbol{∀} \AgdaSymbol{\{}\AgdaBound{V} \AgdaBound{n}\AgdaSymbol{\}} \AgdaSymbol{→} \AgdaDatatype{snocVec} \AgdaSymbol{(}\AgdaFunction{Term} \AgdaBound{V}\AgdaSymbol{)} \AgdaBound{n} \AgdaSymbol{→} \AgdaDatatype{snocVec} \AgdaDatatype{Type} \AgdaBound{n} \AgdaSymbol{→} \AgdaDatatype{snocVec} \AgdaSymbol{(}\AgdaFunction{Term} \AgdaBound{V}\AgdaSymbol{)} \AgdaBound{n} \AgdaSymbol{→} \AgdaDatatype{snocTypes} \AgdaBound{V} \AgdaSymbol{(}\AgdaFunction{Prelims.replicate} \AgdaBound{n} \AgdaInductiveConstructor{-Path}\AgdaSymbol{)}\<%
\\
\>\AgdaFunction{eqmult} \AgdaInductiveConstructor{[]} \AgdaInductiveConstructor{[]} \AgdaInductiveConstructor{[]} \AgdaSymbol{=} \AgdaInductiveConstructor{[]}\<%
\\
\>\AgdaFunction{eqmult} \AgdaSymbol{\{}\AgdaArgument{n} \AgdaSymbol{=} \AgdaInductiveConstructor{suc} \AgdaBound{n}\AgdaSymbol{\}} \AgdaSymbol{(}\AgdaBound{MM} \AgdaInductiveConstructor{snoc} \AgdaBound{M}\AgdaSymbol{)} \AgdaSymbol{(}\AgdaBound{AA} \AgdaInductiveConstructor{snoc} \AgdaBound{A}\AgdaSymbol{)} \AgdaSymbol{(}\AgdaBound{NN} \AgdaInductiveConstructor{snoc} \AgdaBound{N}\AgdaSymbol{)} \AgdaSymbol{=} \AgdaFunction{eqmult} \AgdaBound{MM} \AgdaBound{AA} \AgdaBound{NN} \AgdaInductiveConstructor{snoc} \AgdaSymbol{(}\AgdaFunction{\_⇑⇑} \AgdaSymbol{\{}\AgdaArgument{KK} \AgdaSymbol{=} \AgdaFunction{Prelims.replicate} \AgdaBound{n} \AgdaInductiveConstructor{-Path}\AgdaSymbol{\}} \AgdaBound{M}\AgdaSymbol{)} \AgdaFunction{≡〈} \AgdaBound{A} \AgdaFunction{〉} \AgdaSymbol{(}\AgdaFunction{\_⇑⇑} \AgdaSymbol{\{}\AgdaArgument{KK} \AgdaSymbol{=} \AgdaFunction{Prelims.replicate} \AgdaBound{n} \AgdaInductiveConstructor{-Path}\AgdaSymbol{\}} \AgdaBound{N}\AgdaSymbol{)}\<%
\\
%
\\
\>\AgdaFunction{eqmult-rep} \AgdaSymbol{:} \AgdaSymbol{∀} \AgdaSymbol{\{}\AgdaBound{U} \AgdaBound{V} \AgdaBound{n}\AgdaSymbol{\}} \AgdaSymbol{\{}\AgdaBound{MM} \AgdaSymbol{:} \AgdaDatatype{snocVec} \AgdaSymbol{(}\AgdaFunction{Term} \AgdaBound{U}\AgdaSymbol{)} \AgdaBound{n}\AgdaSymbol{\}} \AgdaSymbol{\{}\AgdaBound{AA} \AgdaBound{NN}\AgdaSymbol{\}} \AgdaSymbol{\{}\AgdaBound{ρ} \AgdaSymbol{:} \AgdaFunction{Rep} \AgdaBound{U} \AgdaBound{V}\AgdaSymbol{\}} \AgdaSymbol{→}\<%
\\
\>[0]\AgdaIndent{2}{}\<[2]%
\>[2]\AgdaFunction{snocTypes-rep} \AgdaSymbol{(}\AgdaFunction{eqmult} \AgdaBound{MM} \AgdaBound{AA} \AgdaBound{NN}\AgdaSymbol{)} \AgdaBound{ρ} \AgdaDatatype{≡} \AgdaFunction{eqmult} \AgdaSymbol{(}\AgdaFunction{snocVec-rep} \AgdaBound{MM} \AgdaBound{ρ}\AgdaSymbol{)} \AgdaBound{AA} \AgdaSymbol{(}\AgdaFunction{snocVec-rep} \AgdaBound{NN} \AgdaBound{ρ}\AgdaSymbol{)}\<%
\\
\>\AgdaFunction{eqmult-rep} \AgdaSymbol{\{}\AgdaArgument{MM} \AgdaSymbol{=} \AgdaInductiveConstructor{[]}\AgdaSymbol{\}} \AgdaSymbol{\{}\AgdaInductiveConstructor{[]}\AgdaSymbol{\}} \AgdaSymbol{\{}\AgdaInductiveConstructor{[]}\AgdaSymbol{\}} \AgdaSymbol{=} \AgdaInductiveConstructor{refl}\<%
\\
\>\AgdaFunction{eqmult-rep} \AgdaSymbol{\{}\AgdaArgument{n} \AgdaSymbol{=} \AgdaInductiveConstructor{suc} \AgdaBound{n}\AgdaSymbol{\}} \AgdaSymbol{\{}\AgdaArgument{MM} \AgdaSymbol{=} \AgdaBound{MM} \AgdaInductiveConstructor{snoc} \AgdaBound{M}\AgdaSymbol{\}} \AgdaSymbol{\{}\AgdaBound{AA} \AgdaInductiveConstructor{snoc} \AgdaBound{A}\AgdaSymbol{\}} \AgdaSymbol{\{}\AgdaBound{NN} \AgdaInductiveConstructor{snoc} \AgdaBound{N}\AgdaSymbol{\}} \AgdaSymbol{=} \AgdaFunction{cong₃} \AgdaSymbol{(λ} \AgdaBound{a} \AgdaBound{b} \AgdaBound{c} \AgdaSymbol{→} \AgdaBound{a} \AgdaInductiveConstructor{snoc} \AgdaBound{b} \AgdaFunction{≡〈} \AgdaBound{A} \AgdaFunction{〉} \AgdaBound{c}\AgdaSymbol{)} \<[102]%
\>[102]\<%
\\
\>[0]\AgdaIndent{2}{}\<[2]%
\>[2]\AgdaFunction{eqmult-rep} \<[13]%
\>[13]\<%
\\
\>[0]\AgdaIndent{2}{}\<[2]%
\>[2]\AgdaSymbol{(}\AgdaFunction{liftsnocRep-ups} \AgdaSymbol{(}\AgdaFunction{Prelims.replicate} \AgdaBound{n} \AgdaInductiveConstructor{-Path}\AgdaSymbol{)} \AgdaBound{M}\AgdaSymbol{)} \AgdaSymbol{(}\AgdaFunction{liftsnocRep-ups} \AgdaSymbol{(}\AgdaFunction{Prelims.replicate} \AgdaBound{n} \AgdaInductiveConstructor{-Path}\AgdaSymbol{)} \AgdaBound{N}\AgdaSymbol{)}\<%
\\
%
\\
\>\AgdaFunction{toSnocListExp} \AgdaSymbol{:} \AgdaSymbol{∀} \AgdaSymbol{\{}\AgdaBound{V} \AgdaBound{K} \AgdaBound{n}\AgdaSymbol{\}} \AgdaSymbol{→} \AgdaDatatype{snocVec} \AgdaSymbol{(}\AgdaFunction{Expression} \AgdaBound{V} \AgdaSymbol{(}\AgdaInductiveConstructor{varKind} \AgdaBound{K}\AgdaSymbol{))} \AgdaBound{n} \AgdaSymbol{→} \AgdaDatatype{snocListExp} \AgdaBound{V} \AgdaSymbol{(}\AgdaFunction{replicate} \AgdaBound{n} \AgdaBound{K}\AgdaSymbol{)}\<%
\\
\>\AgdaFunction{toSnocListExp} \AgdaInductiveConstructor{[]} \AgdaSymbol{=} \AgdaInductiveConstructor{[]}\<%
\\
\>\AgdaFunction{toSnocListExp} \AgdaSymbol{(}\AgdaBound{MM} \AgdaInductiveConstructor{snoc} \AgdaBound{M}\AgdaSymbol{)} \AgdaSymbol{=} \AgdaFunction{toSnocListExp} \AgdaBound{MM} \AgdaInductiveConstructor{snoc} \AgdaBound{M}\<%
\\
%
\\
\>\AgdaFunction{toSnocListExp-rep} \AgdaSymbol{:} \AgdaSymbol{∀} \AgdaSymbol{\{}\AgdaBound{U} \AgdaBound{V} \AgdaBound{K} \AgdaBound{n}\AgdaSymbol{\}} \AgdaSymbol{\{}\AgdaBound{MM} \AgdaSymbol{:} \AgdaDatatype{snocVec} \AgdaSymbol{(}\AgdaFunction{Expression} \AgdaBound{U} \AgdaSymbol{(}\AgdaInductiveConstructor{varKind} \AgdaBound{K}\AgdaSymbol{))} \AgdaBound{n}\AgdaSymbol{\}} \AgdaSymbol{\{}\AgdaBound{ρ} \AgdaSymbol{:} \AgdaFunction{Rep} \AgdaBound{U} \AgdaBound{V}\AgdaSymbol{\}} \AgdaSymbol{→}\<%
\\
\>[0]\AgdaIndent{2}{}\<[2]%
\>[2]\AgdaFunction{snocListExp-rep} \AgdaSymbol{(}\AgdaFunction{toSnocListExp} \AgdaBound{MM}\AgdaSymbol{)} \AgdaBound{ρ} \AgdaDatatype{≡} \AgdaFunction{toSnocListExp} \AgdaSymbol{(}\AgdaFunction{snocVec-rep} \AgdaBound{MM} \AgdaBound{ρ}\AgdaSymbol{)}\<%
\\
\>\AgdaFunction{toSnocListExp-rep} \AgdaSymbol{\{}\AgdaArgument{MM} \AgdaSymbol{=} \AgdaInductiveConstructor{[]}\AgdaSymbol{\}} \AgdaSymbol{=} \AgdaInductiveConstructor{refl}\<%
\\
\>\AgdaFunction{toSnocListExp-rep} \AgdaSymbol{\{}\AgdaArgument{MM} \AgdaSymbol{=} \AgdaBound{MM} \AgdaInductiveConstructor{snoc} \AgdaBound{M}\AgdaSymbol{\}} \AgdaSymbol{\{}\AgdaBound{ρ}\AgdaSymbol{\}} \AgdaSymbol{=} \AgdaFunction{cong} \AgdaSymbol{(λ} \AgdaBound{x} \AgdaSymbol{→} \AgdaBound{x} \AgdaInductiveConstructor{snoc} \AgdaBound{M} \AgdaFunction{〈} \AgdaBound{ρ} \AgdaFunction{〉}\AgdaSymbol{)} \AgdaFunction{toSnocListExp-rep}\<%
\end{code}
}

\paragraph{Substitution}

We write $t[z:=s]$ for the result of substituting $s$ for $z$ in $t$,
renaming bound variables to avoid capture.  We write $s[z_1 := t_1, \ldots, z_n := t_n]$
or $s[\vec{z} := \vec{t}]$ for the result of simultaneously substituting
each $t_i$ for $z_i$ in $s$.

A \emph{substitution} $\sigma$ is a function whose domain is a finite set of variables, and
which maps term variables to terms, proof variables to proofs, and path variables to paths.
Given a substitution $\sigma$ and an expression $t$, we write $t[\sigma]$ for the result
of simultaneously substituting $\sigma(z)$ for $z$ within $t$, for each variable $z$ in the domain of $\sigma$.

Given two substitutions $\sigma$ and $\rho$, we define their \emph{composition} $\sigma \circ \rho$ to
be the substitution with the same domain an $\rho$, such that
\[ (\sigma \circ \rho)(x) \eqdef \rho(x)[\sigma] \enspace . \]
An easy induction on $t$ shows that we have $t [\sigma \circ \rho] \equiv t [ \rho ] [ \sigma ]$.

\AgdaHide{
\begin{code}%
\>\AgdaKeyword{module} \AgdaModule{PL.Rules} \AgdaKeyword{where}\<%
\\
\>\AgdaKeyword{open} \AgdaKeyword{import} \AgdaModule{Data.Empty}\<%
\\
\>\AgdaKeyword{open} \AgdaKeyword{import} \AgdaModule{Prelims}\<%
\\
\>\AgdaKeyword{open} \AgdaKeyword{import} \AgdaModule{PL.Grammar}\<%
\\
\>\AgdaKeyword{open} \AgdaModule{PLgrammar}\<%
\\
\>\AgdaKeyword{open} \AgdaKeyword{import} \AgdaModule{Grammar} \AgdaFunction{Propositional-Logic}\<%
\\
\>\AgdaKeyword{open} \AgdaKeyword{import} \AgdaModule{Reduction} \AgdaFunction{Propositional-Logic} \AgdaDatatype{β}\<%
\end{code}
}

\subsection{Rules of Deduction}

The rules of deduction of the system are as follows.

\[ \infer[(p : \phi \in \Gamma)]{\Gamma \vdash p : \phi}{\Gamma \vald} \]

\[ \infer{\Gamma \vdash \delta \epsilon : \psi}{\Gamma \vdash \delta : \phi \rightarrow \psi}{\Gamma \vdash \epsilon : \phi} \]

\[ \infer{\Gamma \vdash \lambda p : \phi . \delta : \phi \rightarrow \psi}{\Gamma, p : \phi \vdash \delta : \psi} \]

\begin{code}%
\>\AgdaKeyword{infix} \AgdaNumber{10} \AgdaFixityOp{\_⊢\_∶\_}\<%
\\
\>\AgdaKeyword{data} \AgdaDatatype{\_⊢\_∶\_} \AgdaSymbol{:} \AgdaSymbol{∀} \AgdaSymbol{\{}\AgdaBound{P}\AgdaSymbol{\}} \AgdaSymbol{→} \AgdaDatatype{Context} \AgdaBound{P} \AgdaSymbol{→} \AgdaFunction{Proof} \AgdaBound{P} \AgdaSymbol{→} \AgdaDatatype{Prop} \AgdaSymbol{→} \AgdaPrimitiveType{Set} \AgdaKeyword{where}\<%
\\
\>[0]\AgdaIndent{2}{}\<[2]%
\>[2]\AgdaInductiveConstructor{var} \AgdaSymbol{:} \AgdaSymbol{∀} \AgdaSymbol{\{}\AgdaBound{P}\AgdaSymbol{\}} \AgdaSymbol{\{}\AgdaBound{Γ} \AgdaSymbol{:} \AgdaDatatype{Context} \AgdaBound{P}\AgdaSymbol{\}} \AgdaSymbol{(}\AgdaBound{p} \AgdaSymbol{:} \AgdaDatatype{Var} \AgdaBound{P} \AgdaInductiveConstructor{-proof}\AgdaSymbol{)} \AgdaSymbol{→} \<[51]%
\>[51]\<%
\\
\>[2]\AgdaIndent{4}{}\<[4]%
\>[4]\AgdaBound{Γ} \AgdaDatatype{⊢} \AgdaInductiveConstructor{var} \AgdaBound{p} \AgdaDatatype{∶} \AgdaFunction{unprp} \AgdaSymbol{(}\AgdaFunction{typeof} \AgdaBound{p} \AgdaBound{Γ}\AgdaSymbol{)}\<%
\\
\>[0]\AgdaIndent{2}{}\<[2]%
\>[2]\AgdaInductiveConstructor{app} \AgdaSymbol{:} \AgdaSymbol{∀} \AgdaSymbol{\{}\AgdaBound{P}\AgdaSymbol{\}} \AgdaSymbol{\{}\AgdaBound{Γ} \AgdaSymbol{:} \AgdaDatatype{Context} \AgdaBound{P}\AgdaSymbol{\}} \AgdaSymbol{\{}\AgdaBound{δ}\AgdaSymbol{\}} \AgdaSymbol{\{}\AgdaBound{ε}\AgdaSymbol{\}} \AgdaSymbol{\{}\AgdaBound{φ}\AgdaSymbol{\}} \AgdaSymbol{\{}\AgdaBound{ψ}\AgdaSymbol{\}} \AgdaSymbol{→} \<[48]%
\>[48]\<%
\\
\>[2]\AgdaIndent{4}{}\<[4]%
\>[4]\AgdaBound{Γ} \AgdaDatatype{⊢} \AgdaBound{δ} \AgdaDatatype{∶} \AgdaBound{φ} \AgdaInductiveConstructor{⇛} \AgdaBound{ψ} \AgdaSymbol{→} \AgdaBound{Γ} \AgdaDatatype{⊢} \AgdaBound{ε} \AgdaDatatype{∶} \AgdaBound{φ} \AgdaSymbol{→} \AgdaBound{Γ} \AgdaDatatype{⊢} \AgdaFunction{appP} \AgdaBound{δ} \AgdaBound{ε} \AgdaDatatype{∶} \AgdaBound{ψ}\<%
\\
\>[0]\AgdaIndent{2}{}\<[2]%
\>[2]\AgdaInductiveConstructor{Λ} \AgdaSymbol{:} \AgdaSymbol{∀} \AgdaSymbol{\{}\AgdaBound{P}\AgdaSymbol{\}} \AgdaSymbol{\{}\AgdaBound{Γ} \AgdaSymbol{:} \AgdaDatatype{Context} \AgdaBound{P}\AgdaSymbol{\}} \AgdaSymbol{\{}\AgdaBound{φ}\AgdaSymbol{\}} \AgdaSymbol{\{}\AgdaBound{δ}\AgdaSymbol{\}} \AgdaSymbol{\{}\AgdaBound{ψ}\AgdaSymbol{\}} \AgdaSymbol{→} \<[42]%
\>[42]\<%
\\
\>[2]\AgdaIndent{4}{}\<[4]%
\>[4]\AgdaBound{Γ} \AgdaFunction{,P} \AgdaBound{φ} \AgdaDatatype{⊢} \AgdaBound{δ} \AgdaDatatype{∶} \AgdaBound{ψ} \AgdaSymbol{→} \AgdaBound{Γ} \AgdaDatatype{⊢} \AgdaFunction{ΛP} \AgdaBound{φ} \AgdaBound{δ} \AgdaDatatype{∶} \AgdaBound{φ} \AgdaInductiveConstructor{⇛} \AgdaBound{ψ}\<%
\end{code}

\AgdaHide{
\begin{code}%
\>\AgdaFunction{change-type} \AgdaSymbol{:} \AgdaSymbol{∀} \AgdaSymbol{\{}\AgdaBound{P}\AgdaSymbol{\}} \AgdaSymbol{\{}\AgdaBound{Γ} \AgdaSymbol{:} \AgdaDatatype{Context} \AgdaBound{P}\AgdaSymbol{\}} \AgdaSymbol{\{}\AgdaBound{δ} \AgdaBound{φ} \AgdaBound{ψ}\AgdaSymbol{\}} \AgdaSymbol{→}\<%
\\
\>[0]\AgdaIndent{2}{}\<[2]%
\>[2]\AgdaBound{φ} \AgdaDatatype{≡} \AgdaBound{ψ} \AgdaSymbol{→} \AgdaBound{Γ} \AgdaDatatype{⊢} \AgdaBound{δ} \AgdaDatatype{∶} \AgdaBound{φ} \AgdaSymbol{→} \AgdaBound{Γ} \AgdaDatatype{⊢} \AgdaBound{δ} \AgdaDatatype{∶} \AgdaBound{ψ}\<%
\\
\>\AgdaFunction{change-type} \AgdaSymbol{=} \AgdaFunction{subst} \AgdaSymbol{(λ} \AgdaBound{A} \AgdaSymbol{→} \AgdaSymbol{\_} \AgdaDatatype{⊢} \AgdaSymbol{\_} \AgdaDatatype{∶} \AgdaBound{A}\AgdaSymbol{)}\<%
\end{code}
}

Let $\rho$ be a replacement.  We say $\rho$ is a replacement from $\Gamma$ to $\Delta$, $\rho : \Gamma \rightarrow \Delta$,
iff for all $x : \phi \in \Gamma$ we have $\rho(x) : \phi \in \Delta$.

\begin{code}%
\>\AgdaFunction{\_∶\_⇒R\_} \AgdaSymbol{:} \AgdaSymbol{∀} \AgdaSymbol{\{}\AgdaBound{P}\AgdaSymbol{\}} \AgdaSymbol{\{}\AgdaBound{Q}\AgdaSymbol{\}} \AgdaSymbol{→} \AgdaFunction{Rep} \AgdaBound{P} \AgdaBound{Q} \AgdaSymbol{→} \AgdaDatatype{Context} \AgdaBound{P} \AgdaSymbol{→} \AgdaDatatype{Context} \AgdaBound{Q} \AgdaSymbol{→} \AgdaPrimitiveType{Set}\<%
\\
\>\AgdaBound{ρ} \AgdaFunction{∶} \AgdaBound{Γ} \AgdaFunction{⇒R} \AgdaBound{Δ} \AgdaSymbol{=} \AgdaSymbol{∀} \AgdaBound{x} \AgdaSymbol{→} \AgdaFunction{unprp} \AgdaSymbol{(}\AgdaFunction{typeof} \AgdaSymbol{\{}\AgdaArgument{K} \AgdaSymbol{=} \AgdaInductiveConstructor{-proof}\AgdaSymbol{\}} \AgdaSymbol{(}\AgdaBound{ρ} \AgdaSymbol{\_} \AgdaBound{x}\AgdaSymbol{)} \AgdaBound{Δ}\AgdaSymbol{)} \AgdaDatatype{≡} \AgdaFunction{unprp} \AgdaSymbol{(}\AgdaFunction{typeof} \AgdaBound{x} \AgdaBound{Γ} \AgdaSymbol{)}\<%
\end{code}

\begin{lemma}$ $
\begin{enumerate}
\item
$\id{P}$ is a replacement $\Gamma \rightarrow \Gamma$.
\item
$\uparrow$ is a replacement $\Gamma \rightarrow \Gamma , \phi$.
\item
If $\rho : \Gamma \rightarrow \Delta$ then $(\rho , \mathrm{Proof}) : (\Gamma , x : \phi) \rightarrow (\Delta , x : \phi)$.
\item
If $\rho : \Gamma \rightarrow \Delta$ and $\sigma : \Delta \rightarrow \Theta$ then $\sigma \circ \rho : \Gamma \rightarrow \Delta$.
\item
(\textbf{Weakening})
If $\rho : \Gamma \rightarrow \Delta$ and $\Gamma \vdash \delta : \phi$ then $\Delta \vdash \delta \langle \rho \rangle : \phi$.
\end{enumerate}
\end{lemma}

\begin{code}%
\>\AgdaFunction{idRep-typed} \AgdaSymbol{:} \AgdaSymbol{∀} \AgdaSymbol{\{}\AgdaBound{P}\AgdaSymbol{\}} \AgdaSymbol{\{}\AgdaBound{Γ} \AgdaSymbol{:} \AgdaDatatype{Context} \AgdaBound{P}\AgdaSymbol{\}} \AgdaSymbol{→} \AgdaFunction{idRep} \AgdaBound{P} \AgdaFunction{∶} \AgdaBound{Γ} \AgdaFunction{⇒R} \AgdaBound{Γ}\<%
\end{code}

\AgdaHide{
\begin{code}%
\>\AgdaFunction{idRep-typed} \AgdaSymbol{\{}\AgdaBound{P}\AgdaSymbol{\}} \AgdaSymbol{\{}\AgdaBound{Γ}\AgdaSymbol{\}} \AgdaBound{x} \AgdaSymbol{=} \AgdaInductiveConstructor{refl}\<%
\end{code}
}

\begin{code}%
\>\AgdaFunction{unprp-rep} \AgdaSymbol{:} \AgdaSymbol{∀} \AgdaSymbol{\{}\AgdaBound{U} \AgdaBound{V}\AgdaSymbol{\}} \AgdaBound{φ} \AgdaSymbol{(}\AgdaBound{ρ} \AgdaSymbol{:} \AgdaFunction{Rep} \AgdaBound{U} \AgdaBound{V}\AgdaSymbol{)} \AgdaSymbol{→} \AgdaFunction{unprp} \AgdaSymbol{(}\AgdaBound{φ} \AgdaFunction{〈} \AgdaBound{ρ} \AgdaFunction{〉}\AgdaSymbol{)} \AgdaDatatype{≡} \AgdaFunction{unprp} \AgdaBound{φ}\<%
\\
\>\AgdaFunction{unprp-rep} \AgdaSymbol{(}\AgdaInductiveConstructor{app} \AgdaSymbol{(}\AgdaInductiveConstructor{-prp} \AgdaSymbol{\_)} \AgdaInductiveConstructor{[]}\AgdaSymbol{)} \AgdaSymbol{\_} \AgdaSymbol{=} \AgdaInductiveConstructor{refl}\<%
\\
%
\\
\>\AgdaFunction{↑-typed} \AgdaSymbol{:} \AgdaSymbol{∀} \AgdaSymbol{\{}\AgdaBound{P}\AgdaSymbol{\}} \AgdaSymbol{\{}\AgdaBound{Γ} \AgdaSymbol{:} \AgdaDatatype{Context} \AgdaBound{P}\AgdaSymbol{\}} \AgdaSymbol{\{}\AgdaBound{φ} \AgdaSymbol{:} \AgdaDatatype{Prop}\AgdaSymbol{\}} \AgdaSymbol{→} \AgdaFunction{upRep} \AgdaFunction{∶} \AgdaBound{Γ} \AgdaFunction{⇒R} \AgdaSymbol{(}\AgdaBound{Γ} \AgdaFunction{,P} \AgdaBound{φ}\AgdaSymbol{)}\<%
\end{code}

\AgdaHide{
\begin{code}%
\>\AgdaFunction{↑-typed} \AgdaSymbol{\{}\AgdaBound{P}\AgdaSymbol{\}} \AgdaSymbol{\{}\AgdaBound{Γ}\AgdaSymbol{\}} \AgdaSymbol{\{}\AgdaBound{φ}\AgdaSymbol{\}} \AgdaBound{x} \AgdaSymbol{=} \AgdaFunction{unprp-rep} \AgdaSymbol{(}\AgdaFunction{typeof} \AgdaBound{x} \AgdaBound{Γ}\AgdaSymbol{)} \AgdaFunction{upRep}\<%
\end{code}
}

\begin{code}%
\>\AgdaFunction{liftRep-typed} \AgdaSymbol{:} \AgdaSymbol{∀} \AgdaSymbol{\{}\AgdaBound{P}\AgdaSymbol{\}} \AgdaSymbol{\{}\AgdaBound{Q}\AgdaSymbol{\}} \AgdaSymbol{\{}\AgdaBound{ρ}\AgdaSymbol{\}} \AgdaSymbol{\{}\AgdaBound{Γ} \AgdaSymbol{:} \AgdaDatatype{Context} \AgdaBound{P}\AgdaSymbol{\}} \AgdaSymbol{\{}\AgdaBound{Δ} \AgdaSymbol{:} \AgdaDatatype{Context} \AgdaBound{Q}\AgdaSymbol{\}} \AgdaSymbol{\{}\AgdaBound{φ} \AgdaSymbol{:} \AgdaDatatype{Prop}\AgdaSymbol{\}} \AgdaSymbol{→} \<[75]%
\>[75]\<%
\\
\>[0]\AgdaIndent{2}{}\<[2]%
\>[2]\AgdaBound{ρ} \AgdaFunction{∶} \AgdaBound{Γ} \AgdaFunction{⇒R} \AgdaBound{Δ} \AgdaSymbol{→} \AgdaFunction{liftRep} \AgdaInductiveConstructor{-proof} \AgdaBound{ρ} \AgdaFunction{∶} \AgdaSymbol{(}\AgdaBound{Γ} \AgdaFunction{,P} \AgdaBound{φ}\AgdaSymbol{)} \AgdaFunction{⇒R} \AgdaSymbol{(}\AgdaBound{Δ} \AgdaFunction{,P} \AgdaBound{φ}\AgdaSymbol{)}\<%
\end{code}

\AgdaHide{
\begin{code}%
\>\AgdaFunction{liftRep-typed} \AgdaSymbol{\{}\AgdaBound{P}\AgdaSymbol{\}} \AgdaSymbol{\{}\AgdaArgument{Q} \AgdaSymbol{=} \AgdaBound{Q}\AgdaSymbol{\}} \AgdaSymbol{\{}\AgdaArgument{ρ} \AgdaSymbol{=} \AgdaBound{ρ}\AgdaSymbol{\}} \AgdaSymbol{\{}\AgdaBound{Γ}\AgdaSymbol{\}} \AgdaSymbol{\{}\AgdaArgument{Δ} \AgdaSymbol{=} \AgdaBound{Δ}\AgdaSymbol{\}} \AgdaSymbol{\{}\AgdaArgument{φ} \AgdaSymbol{=} \AgdaBound{φ}\AgdaSymbol{\}} \AgdaBound{ρ∶Γ→Δ} \AgdaInductiveConstructor{x₀} \AgdaSymbol{=} \AgdaInductiveConstructor{refl}\<%
\\
\>\AgdaFunction{liftRep-typed} \AgdaSymbol{\{}\AgdaArgument{Q} \AgdaSymbol{=} \AgdaBound{Q}\AgdaSymbol{\}} \AgdaSymbol{\{}\AgdaArgument{ρ} \AgdaSymbol{=} \AgdaBound{ρ}\AgdaSymbol{\}} \AgdaSymbol{\{}\AgdaArgument{Γ} \AgdaSymbol{=} \AgdaBound{Γ}\AgdaSymbol{\}} \AgdaSymbol{\{}\AgdaArgument{Δ} \AgdaSymbol{=} \AgdaBound{Δ}\AgdaSymbol{\}} \AgdaSymbol{\{}\AgdaBound{φ}\AgdaSymbol{\}} \AgdaBound{ρ∶Γ→Δ} \AgdaSymbol{(}\AgdaInductiveConstructor{↑} \AgdaBound{x}\AgdaSymbol{)} \AgdaSymbol{=} \<[64]%
\>[64]\<%
\\
\>[0]\AgdaIndent{2}{}\<[2]%
\>[2]\AgdaKeyword{let} \AgdaKeyword{open} \AgdaModule{≡-Reasoning} \AgdaKeyword{in} \<[26]%
\>[26]\<%
\\
\>[0]\AgdaIndent{2}{}\<[2]%
\>[2]\AgdaFunction{begin}\<%
\\
\>[2]\AgdaIndent{4}{}\<[4]%
\>[4]\AgdaFunction{unprp} \AgdaSymbol{(}\AgdaFunction{typeof} \AgdaSymbol{(}\AgdaFunction{liftRep} \AgdaInductiveConstructor{-proof} \AgdaBound{ρ} \AgdaInductiveConstructor{-proof} \AgdaSymbol{(}\AgdaInductiveConstructor{↑} \AgdaBound{x}\AgdaSymbol{))} \AgdaSymbol{(}\AgdaBound{Δ} \AgdaFunction{,P} \AgdaBound{φ}\AgdaSymbol{))}\<%
\\
\>[0]\AgdaIndent{2}{}\<[2]%
\>[2]\AgdaFunction{≡⟨⟩}\<%
\\
\>[2]\AgdaIndent{4}{}\<[4]%
\>[4]\AgdaFunction{unprp} \AgdaSymbol{(}\AgdaFunction{typeof} \AgdaSymbol{(}\AgdaInductiveConstructor{↑} \AgdaSymbol{(}\AgdaBound{ρ} \AgdaInductiveConstructor{-proof} \AgdaBound{x}\AgdaSymbol{))} \AgdaSymbol{(}\AgdaBound{Δ} \AgdaFunction{,P} \AgdaBound{φ}\AgdaSymbol{))}\<%
\\
\>[0]\AgdaIndent{2}{}\<[2]%
\>[2]\AgdaFunction{≡⟨⟩}\<%
\\
\>[2]\AgdaIndent{4}{}\<[4]%
\>[4]\AgdaFunction{unprp} \AgdaSymbol{(}\AgdaFunction{typeof} \AgdaSymbol{(}\AgdaBound{ρ} \AgdaInductiveConstructor{-proof} \AgdaBound{x}\AgdaSymbol{)} \AgdaBound{Δ} \AgdaFunction{〈} \AgdaFunction{upRep} \AgdaFunction{〉}\AgdaSymbol{)}\<%
\\
\>[0]\AgdaIndent{2}{}\<[2]%
\>[2]\AgdaFunction{≡⟨} \AgdaFunction{unprp-rep} \AgdaSymbol{(}\AgdaFunction{typeof} \AgdaSymbol{(}\AgdaBound{ρ} \AgdaInductiveConstructor{-proof} \AgdaBound{x}\AgdaSymbol{)} \AgdaBound{Δ}\AgdaSymbol{)} \AgdaFunction{upRep} \AgdaFunction{⟩}\<%
\\
\>[2]\AgdaIndent{4}{}\<[4]%
\>[4]\AgdaFunction{unprp} \AgdaSymbol{(}\AgdaFunction{typeof} \AgdaSymbol{(}\AgdaBound{ρ} \AgdaInductiveConstructor{-proof} \AgdaBound{x}\AgdaSymbol{)} \AgdaBound{Δ}\AgdaSymbol{)}\<%
\\
\>[0]\AgdaIndent{2}{}\<[2]%
\>[2]\AgdaFunction{≡⟨} \AgdaBound{ρ∶Γ→Δ} \AgdaBound{x} \AgdaFunction{⟩}\<%
\\
\>[2]\AgdaIndent{4}{}\<[4]%
\>[4]\AgdaFunction{unprp} \AgdaSymbol{(}\AgdaFunction{typeof} \AgdaBound{x} \AgdaBound{Γ}\AgdaSymbol{)}\<%
\\
\>[0]\AgdaIndent{2}{}\<[2]%
\>[2]\AgdaFunction{≡⟨⟨} \AgdaFunction{unprp-rep} \AgdaSymbol{(}\AgdaFunction{typeof} \AgdaBound{x} \AgdaBound{Γ}\AgdaSymbol{)} \AgdaFunction{upRep} \AgdaFunction{⟩⟩}\<%
\\
\>[2]\AgdaIndent{4}{}\<[4]%
\>[4]\AgdaFunction{unprp} \AgdaSymbol{(}\AgdaFunction{typeof} \AgdaBound{x} \AgdaBound{Γ} \AgdaFunction{〈} \AgdaFunction{upRep} \AgdaFunction{〉}\AgdaSymbol{)}\<%
\\
\>[0]\AgdaIndent{2}{}\<[2]%
\>[2]\AgdaFunction{≡⟨⟩}\<%
\\
\>[2]\AgdaIndent{4}{}\<[4]%
\>[4]\AgdaFunction{unprp} \AgdaSymbol{(}\AgdaFunction{typeof} \AgdaSymbol{(}\AgdaInductiveConstructor{↑} \AgdaBound{x}\AgdaSymbol{)} \AgdaSymbol{(}\AgdaBound{Γ} \AgdaFunction{,P} \AgdaBound{φ}\AgdaSymbol{))}\<%
\\
\>[0]\AgdaIndent{2}{}\<[2]%
\>[2]\AgdaFunction{∎}\<%
\end{code}
}

\begin{code}%
\>\AgdaFunction{•R-typed} \AgdaSymbol{:} \AgdaSymbol{∀} \AgdaSymbol{\{}\AgdaBound{P}\AgdaSymbol{\}} \AgdaSymbol{\{}\AgdaBound{Q}\AgdaSymbol{\}} \AgdaSymbol{\{}\AgdaBound{R}\AgdaSymbol{\}} \AgdaSymbol{\{}\AgdaBound{σ} \AgdaSymbol{:} \AgdaFunction{Rep} \AgdaBound{Q} \AgdaBound{R}\AgdaSymbol{\}} \AgdaSymbol{\{}\AgdaBound{ρ} \AgdaSymbol{:} \AgdaFunction{Rep} \AgdaBound{P} \AgdaBound{Q}\AgdaSymbol{\}} \AgdaSymbol{\{}\AgdaBound{Γ}\AgdaSymbol{\}} \AgdaSymbol{\{}\AgdaBound{Δ}\AgdaSymbol{\}} \AgdaSymbol{\{}\AgdaBound{Θ}\AgdaSymbol{\}} \AgdaSymbol{→} \<[67]%
\>[67]\<%
\\
\>[0]\AgdaIndent{2}{}\<[2]%
\>[2]\AgdaBound{ρ} \AgdaFunction{∶} \AgdaBound{Γ} \AgdaFunction{⇒R} \AgdaBound{Δ} \AgdaSymbol{→} \AgdaBound{σ} \AgdaFunction{∶} \AgdaBound{Δ} \AgdaFunction{⇒R} \AgdaBound{Θ} \AgdaSymbol{→} \AgdaSymbol{(}\AgdaBound{σ} \AgdaFunction{•R} \AgdaBound{ρ}\AgdaSymbol{)} \AgdaFunction{∶} \AgdaBound{Γ} \AgdaFunction{⇒R} \AgdaBound{Θ}\<%
\end{code}

\AgdaHide{
\begin{code}%
\>\AgdaFunction{•R-typed} \AgdaSymbol{\{}\AgdaArgument{R} \AgdaSymbol{=} \AgdaBound{R}\AgdaSymbol{\}} \AgdaSymbol{\{}\AgdaBound{σ}\AgdaSymbol{\}} \AgdaSymbol{\{}\AgdaBound{ρ}\AgdaSymbol{\}} \AgdaSymbol{\{}\AgdaBound{Γ}\AgdaSymbol{\}} \AgdaSymbol{\{}\AgdaBound{Δ}\AgdaSymbol{\}} \AgdaSymbol{\{}\AgdaBound{Θ}\AgdaSymbol{\}} \AgdaBound{ρ∶Γ→Δ} \AgdaBound{σ∶Δ→Θ} \AgdaBound{x} \AgdaSymbol{=} \AgdaKeyword{let} \AgdaKeyword{open} \AgdaModule{≡-Reasoning} \AgdaKeyword{in} \<[77]%
\>[77]\<%
\\
\>[0]\AgdaIndent{2}{}\<[2]%
\>[2]\AgdaFunction{begin} \<[8]%
\>[8]\<%
\\
\>[2]\AgdaIndent{4}{}\<[4]%
\>[4]\AgdaFunction{unprp} \AgdaSymbol{(}\AgdaFunction{typeof} \AgdaSymbol{(}\AgdaBound{σ} \AgdaInductiveConstructor{-proof} \AgdaSymbol{(}\AgdaBound{ρ} \AgdaInductiveConstructor{-proof} \AgdaBound{x}\AgdaSymbol{))} \AgdaBound{Θ}\AgdaSymbol{)}\<%
\\
\>[0]\AgdaIndent{2}{}\<[2]%
\>[2]\AgdaFunction{≡⟨} \AgdaBound{σ∶Δ→Θ} \AgdaSymbol{(}\AgdaBound{ρ} \AgdaInductiveConstructor{-proof} \AgdaBound{x}\AgdaSymbol{)} \AgdaFunction{⟩}\<%
\\
\>[2]\AgdaIndent{4}{}\<[4]%
\>[4]\AgdaFunction{unprp} \AgdaSymbol{(}\AgdaFunction{typeof} \AgdaSymbol{(}\AgdaBound{ρ} \AgdaInductiveConstructor{-proof} \AgdaBound{x}\AgdaSymbol{)} \AgdaBound{Δ}\AgdaSymbol{)}\<%
\\
\>[0]\AgdaIndent{2}{}\<[2]%
\>[2]\AgdaFunction{≡⟨} \AgdaBound{ρ∶Γ→Δ} \AgdaBound{x} \AgdaFunction{⟩}\<%
\\
\>[2]\AgdaIndent{4}{}\<[4]%
\>[4]\AgdaFunction{unprp} \AgdaSymbol{(}\AgdaFunction{typeof} \AgdaBound{x} \AgdaBound{Γ}\AgdaSymbol{)}\<%
\\
\>[0]\AgdaIndent{2}{}\<[2]%
\>[2]\AgdaFunction{∎}\<%
\end{code}
}

\begin{code}%
\>\AgdaFunction{Weakening} \AgdaSymbol{:} \AgdaSymbol{∀} \AgdaSymbol{\{}\AgdaBound{P}\AgdaSymbol{\}} \AgdaSymbol{\{}\AgdaBound{Q}\AgdaSymbol{\}} \AgdaSymbol{\{}\AgdaBound{Γ} \AgdaSymbol{:} \AgdaDatatype{Context} \AgdaBound{P}\AgdaSymbol{\}} \AgdaSymbol{\{}\AgdaBound{Δ} \AgdaSymbol{:} \AgdaDatatype{Context} \AgdaBound{Q}\AgdaSymbol{\}} \AgdaSymbol{\{}\AgdaBound{ρ}\AgdaSymbol{\}} \AgdaSymbol{\{}\AgdaBound{δ}\AgdaSymbol{\}} \AgdaSymbol{\{}\AgdaBound{φ}\AgdaSymbol{\}} \AgdaSymbol{→} \<[68]%
\>[68]\<%
\\
\>[0]\AgdaIndent{2}{}\<[2]%
\>[2]\AgdaBound{Γ} \AgdaDatatype{⊢} \AgdaBound{δ} \AgdaDatatype{∶} \AgdaBound{φ} \AgdaSymbol{→} \AgdaBound{ρ} \AgdaFunction{∶} \AgdaBound{Γ} \AgdaFunction{⇒R} \AgdaBound{Δ} \AgdaSymbol{→} \AgdaBound{Δ} \AgdaDatatype{⊢} \AgdaBound{δ} \AgdaFunction{〈} \AgdaBound{ρ} \AgdaFunction{〉} \AgdaDatatype{∶} \AgdaBound{φ}\<%
\end{code}

\AgdaHide{
\begin{code}%
\>\AgdaFunction{Weakening} \AgdaSymbol{\{}\AgdaBound{P}\AgdaSymbol{\}} \AgdaSymbol{\{}\AgdaBound{Q}\AgdaSymbol{\}} \AgdaSymbol{\{}\AgdaBound{Γ}\AgdaSymbol{\}} \AgdaSymbol{\{}\AgdaBound{Δ}\AgdaSymbol{\}} \AgdaSymbol{\{}\AgdaBound{ρ}\AgdaSymbol{\}} \AgdaSymbol{(}\AgdaInductiveConstructor{var} \AgdaBound{p}\AgdaSymbol{)} \AgdaBound{ρ∶Γ→Δ} \AgdaSymbol{=} \AgdaFunction{change-type} \AgdaSymbol{(}\AgdaBound{ρ∶Γ→Δ} \AgdaBound{p}\AgdaSymbol{)} \AgdaSymbol{(}\AgdaInductiveConstructor{var} \AgdaSymbol{(}\AgdaBound{ρ} \AgdaSymbol{\_} \AgdaBound{p}\AgdaSymbol{))}\<%
\\
\>\AgdaFunction{Weakening} \AgdaSymbol{(}\AgdaInductiveConstructor{app} \AgdaBound{Γ⊢δ∶φ→ψ} \AgdaBound{Γ⊢ε∶φ}\AgdaSymbol{)} \AgdaBound{ρ∶Γ→Δ} \AgdaSymbol{=} \AgdaInductiveConstructor{app} \AgdaSymbol{(}\AgdaFunction{Weakening} \AgdaBound{Γ⊢δ∶φ→ψ} \AgdaBound{ρ∶Γ→Δ}\AgdaSymbol{)} \AgdaSymbol{(}\AgdaFunction{Weakening} \AgdaBound{Γ⊢ε∶φ} \AgdaBound{ρ∶Γ→Δ}\AgdaSymbol{)}\<%
\\
\>\AgdaFunction{Weakening} \AgdaSymbol{.\{}\AgdaBound{P}\AgdaSymbol{\}} \AgdaSymbol{\{}\AgdaBound{Q}\AgdaSymbol{\}} \AgdaSymbol{.\{}\AgdaBound{Γ}\AgdaSymbol{\}} \AgdaSymbol{\{}\AgdaBound{Δ}\AgdaSymbol{\}} \AgdaSymbol{\{}\AgdaBound{ρ}\AgdaSymbol{\}} \AgdaSymbol{(}\AgdaInductiveConstructor{Λ} \AgdaSymbol{\{}\AgdaBound{P}\AgdaSymbol{\}} \AgdaSymbol{\{}\AgdaBound{Γ}\AgdaSymbol{\}} \AgdaSymbol{\{}\AgdaBound{φ}\AgdaSymbol{\}} \AgdaSymbol{\{}\AgdaBound{δ}\AgdaSymbol{\}} \AgdaSymbol{\{}\AgdaBound{ψ}\AgdaSymbol{\}} \AgdaBound{Γ,φ⊢δ∶ψ}\AgdaSymbol{)} \AgdaBound{ρ∶Γ→Δ} \AgdaSymbol{=} \AgdaInductiveConstructor{Λ} \<[74]%
\>[74]\<%
\\
\>[0]\AgdaIndent{2}{}\<[2]%
\>[2]\AgdaSymbol{(}\AgdaFunction{Weakening} \AgdaSymbol{\{}\AgdaBound{P} \AgdaInductiveConstructor{,} \AgdaInductiveConstructor{-proof}\AgdaSymbol{\}} \AgdaSymbol{\{}\AgdaBound{Q} \AgdaInductiveConstructor{,} \AgdaInductiveConstructor{-proof}\AgdaSymbol{\}} \AgdaSymbol{\{}\AgdaBound{Γ} \AgdaFunction{,P} \AgdaBound{φ}\AgdaSymbol{\}} \AgdaSymbol{\{}\AgdaBound{Δ} \AgdaFunction{,P} \AgdaBound{φ}\AgdaSymbol{\}} \AgdaSymbol{\{}\AgdaFunction{liftRep} \AgdaInductiveConstructor{-proof} \AgdaBound{ρ}\AgdaSymbol{\}} \AgdaSymbol{\{}\AgdaBound{δ}\AgdaSymbol{\}} \AgdaSymbol{\{}\AgdaBound{ψ}\AgdaSymbol{\}} \<[84]%
\>[84]\<%
\\
\>[2]\AgdaIndent{4}{}\<[4]%
\>[4]\AgdaBound{Γ,φ⊢δ∶ψ} \AgdaSymbol{(}\AgdaFunction{liftRep-typed} \AgdaBound{ρ∶Γ→Δ}\AgdaSymbol{))}\<%
\end{code}
}
A \emph{substitution} $\sigma$ from a context $\Gamma$ to a context $\Delta$, $\sigma : \Gamma \rightarrow \Delta$,  is a substitution $\sigma$ such that
for every $x : \phi$ in $\Gamma$, we have $\Delta \vdash \sigma(x) : \phi$.

\begin{code}%
\>\AgdaFunction{\_∶\_⇒\_} \AgdaSymbol{:} \AgdaSymbol{∀} \AgdaSymbol{\{}\AgdaBound{P}\AgdaSymbol{\}} \AgdaSymbol{\{}\AgdaBound{Q}\AgdaSymbol{\}} \AgdaSymbol{→} \AgdaFunction{Sub} \AgdaBound{P} \AgdaBound{Q} \AgdaSymbol{→} \AgdaDatatype{Context} \AgdaBound{P} \AgdaSymbol{→} \AgdaDatatype{Context} \AgdaBound{Q} \AgdaSymbol{→} \AgdaPrimitiveType{Set}\<%
\\
\>\AgdaBound{σ} \AgdaFunction{∶} \AgdaBound{Γ} \AgdaFunction{⇒} \AgdaBound{Δ} \AgdaSymbol{=} \AgdaSymbol{∀} \AgdaBound{x} \AgdaSymbol{→} \AgdaBound{Δ} \AgdaDatatype{⊢} \AgdaBound{σ} \AgdaSymbol{\_} \AgdaBound{x} \AgdaDatatype{∶} \AgdaFunction{unprp} \AgdaSymbol{(}\AgdaFunction{typeof} \AgdaBound{x} \AgdaBound{Γ}\AgdaSymbol{)}\<%
\end{code}

\begin{lemma}$ $
\begin{enumerate}
\item
If $\sigma : \Gamma \rightarrow \Delta$ then $(\sigma , \mathrm{Proof}) : (\Gamma , p : \phi) \rightarrow (\Delta , p : \phi [ \sigma ])$.
\item
If $\Gamma \vdash \delta : \phi$ then $(p := \delta) : (\Gamma, p : \phi) \rightarrow \Gamma$.
\item
(\textbf{substitution Lemma})

If $\Gamma \vdash \delta : \phi$ and $\sigma : \Gamma \rightarrow \Delta$ then $\Delta \vdash \delta [ \sigma ] : \phi [ \sigma ]$.
\end{enumerate}
\end{lemma}

\begin{code}%
\>\AgdaFunction{liftSub-typed} \AgdaSymbol{:} \AgdaSymbol{∀} \AgdaSymbol{\{}\AgdaBound{P}\AgdaSymbol{\}} \AgdaSymbol{\{}\AgdaBound{Q}\AgdaSymbol{\}} \AgdaSymbol{\{}\AgdaBound{σ}\AgdaSymbol{\}} \<[30]%
\>[30]\<%
\\
\>[0]\AgdaIndent{2}{}\<[2]%
\>[2]\AgdaSymbol{\{}\AgdaBound{Γ} \AgdaSymbol{:} \AgdaDatatype{Context} \AgdaBound{P}\AgdaSymbol{\}} \AgdaSymbol{\{}\AgdaBound{Δ} \AgdaSymbol{:} \AgdaDatatype{Context} \AgdaBound{Q}\AgdaSymbol{\}} \AgdaSymbol{\{}\AgdaBound{φ} \AgdaSymbol{:} \AgdaDatatype{Prop}\AgdaSymbol{\}} \AgdaSymbol{→} \<[47]%
\>[47]\<%
\\
\>[0]\AgdaIndent{2}{}\<[2]%
\>[2]\AgdaBound{σ} \AgdaFunction{∶} \AgdaBound{Γ} \AgdaFunction{⇒} \AgdaBound{Δ} \AgdaSymbol{→} \AgdaFunction{liftSub} \AgdaInductiveConstructor{-proof} \AgdaBound{σ} \AgdaFunction{∶} \AgdaSymbol{(}\AgdaBound{Γ} \AgdaFunction{,P} \AgdaBound{φ}\AgdaSymbol{)} \AgdaFunction{⇒} \AgdaSymbol{(}\AgdaBound{Δ} \AgdaFunction{,P} \AgdaBound{φ}\AgdaSymbol{)}\<%
\end{code}

\AgdaHide{
\begin{code}%
\>\AgdaFunction{liftSub-typed} \AgdaSymbol{\{}\AgdaArgument{σ} \AgdaSymbol{=} \AgdaBound{σ}\AgdaSymbol{\}} \AgdaSymbol{\{}\AgdaBound{Γ}\AgdaSymbol{\}} \AgdaSymbol{\{}\AgdaBound{Δ}\AgdaSymbol{\}} \AgdaSymbol{\{}\AgdaBound{φ}\AgdaSymbol{\}} \AgdaBound{σ∶Γ⇒Δ} \AgdaBound{x} \AgdaSymbol{=}\<%
\\
\>[0]\AgdaIndent{2}{}\<[2]%
\>[2]\AgdaFunction{change-type} \AgdaSymbol{(}\AgdaFunction{sym} \AgdaSymbol{(}\AgdaFunction{unprp-rep} \AgdaSymbol{(}\AgdaFunction{pretypeof} \AgdaBound{x} \AgdaSymbol{(}\AgdaBound{Γ} \AgdaFunction{,P} \AgdaBound{φ}\AgdaSymbol{))} \AgdaFunction{upRep}\AgdaSymbol{))} \AgdaSymbol{(}\AgdaFunction{pre-LiftSub-typed} \AgdaBound{x}\AgdaSymbol{)} \AgdaKeyword{where}\<%
\\
\>[0]\AgdaIndent{2}{}\<[2]%
\>[2]\AgdaFunction{pre-LiftSub-typed} \AgdaSymbol{:} \AgdaSymbol{∀} \AgdaBound{x} \AgdaSymbol{→} \AgdaBound{Δ} \AgdaFunction{,P} \AgdaBound{φ} \AgdaDatatype{⊢} \AgdaFunction{liftSub} \AgdaInductiveConstructor{-proof} \AgdaBound{σ} \AgdaInductiveConstructor{-proof} \AgdaBound{x} \AgdaDatatype{∶} \AgdaFunction{unprp} \AgdaSymbol{(}\AgdaFunction{pretypeof} \AgdaBound{x} \AgdaSymbol{(}\AgdaBound{Γ} \AgdaFunction{,P} \AgdaBound{φ}\AgdaSymbol{))}\<%
\\
\>[0]\AgdaIndent{2}{}\<[2]%
\>[2]\AgdaFunction{pre-LiftSub-typed} \AgdaInductiveConstructor{x₀} \AgdaSymbol{=} \AgdaInductiveConstructor{var} \AgdaInductiveConstructor{x₀}\<%
\\
\>[0]\AgdaIndent{2}{}\<[2]%
\>[2]\AgdaFunction{pre-LiftSub-typed} \AgdaSymbol{(}\AgdaInductiveConstructor{↑} \AgdaBound{x}\AgdaSymbol{)} \AgdaSymbol{=} \AgdaFunction{Weakening} \AgdaSymbol{(}\AgdaBound{σ∶Γ⇒Δ} \AgdaBound{x}\AgdaSymbol{)} \AgdaSymbol{(}\AgdaFunction{↑-typed} \AgdaSymbol{\{}\AgdaArgument{φ} \AgdaSymbol{=} \AgdaBound{φ}\AgdaSymbol{\})}\<%
\end{code}
}

\begin{code}%
\>\AgdaFunction{botSub-typed} \AgdaSymbol{:} \AgdaSymbol{∀} \AgdaSymbol{\{}\AgdaBound{P}\AgdaSymbol{\}} \AgdaSymbol{\{}\AgdaBound{Γ} \AgdaSymbol{:} \AgdaDatatype{Context} \AgdaBound{P}\AgdaSymbol{\}} \AgdaSymbol{\{}\AgdaBound{φ} \AgdaSymbol{:} \AgdaDatatype{Prop}\AgdaSymbol{\}} \AgdaSymbol{\{}\AgdaBound{δ}\AgdaSymbol{\}} \AgdaSymbol{→}\<%
\\
\>[0]\AgdaIndent{2}{}\<[2]%
\>[2]\AgdaBound{Γ} \AgdaDatatype{⊢} \AgdaBound{δ} \AgdaDatatype{∶} \AgdaBound{φ} \AgdaSymbol{→} \AgdaFunction{x₀:=} \AgdaBound{δ} \AgdaFunction{∶} \AgdaSymbol{(}\AgdaBound{Γ} \AgdaFunction{,P} \AgdaBound{φ}\AgdaSymbol{)} \AgdaFunction{⇒} \AgdaBound{Γ}\<%
\end{code}

\AgdaHide{
\begin{code}%
\>\AgdaFunction{botSub-typed} \AgdaSymbol{\{}\AgdaBound{P}\AgdaSymbol{\}} \AgdaSymbol{\{}\AgdaBound{Γ}\AgdaSymbol{\}} \AgdaSymbol{\{}\AgdaBound{φ}\AgdaSymbol{\}} \AgdaSymbol{\{}\AgdaBound{δ}\AgdaSymbol{\}} \AgdaBound{Γ⊢δ:φ} \AgdaBound{x} \AgdaSymbol{=} \<[39]%
\>[39]\<%
\\
\>[0]\AgdaIndent{2}{}\<[2]%
\>[2]\AgdaFunction{change-type} \AgdaSymbol{(}\AgdaFunction{sym} \AgdaSymbol{(}\AgdaFunction{unprp-rep} \AgdaSymbol{(}\AgdaFunction{pretypeof} \AgdaBound{x} \AgdaSymbol{(}\AgdaBound{Γ} \AgdaFunction{,P} \AgdaBound{φ}\AgdaSymbol{))} \AgdaFunction{upRep}\AgdaSymbol{))} \AgdaSymbol{(}\AgdaFunction{pre-botSub-typed} \AgdaBound{x}\AgdaSymbol{)} \AgdaKeyword{where}\<%
\\
\>[0]\AgdaIndent{2}{}\<[2]%
\>[2]\AgdaFunction{pre-botSub-typed} \AgdaSymbol{:} \AgdaSymbol{∀} \AgdaBound{x} \AgdaSymbol{→} \AgdaBound{Γ} \AgdaDatatype{⊢} \AgdaSymbol{(}\AgdaFunction{x₀:=} \AgdaBound{δ}\AgdaSymbol{)} \AgdaInductiveConstructor{-proof} \AgdaBound{x} \AgdaDatatype{∶} \AgdaFunction{unprp} \AgdaSymbol{(}\AgdaFunction{pretypeof} \AgdaBound{x} \AgdaSymbol{(}\AgdaBound{Γ} \AgdaFunction{,P} \AgdaBound{φ}\AgdaSymbol{))}\<%
\\
\>[0]\AgdaIndent{2}{}\<[2]%
\>[2]\AgdaFunction{pre-botSub-typed} \AgdaInductiveConstructor{x₀} \AgdaSymbol{=} \AgdaBound{Γ⊢δ:φ}\<%
\\
\>[0]\AgdaIndent{2}{}\<[2]%
\>[2]\AgdaFunction{pre-botSub-typed} \AgdaSymbol{(}\AgdaInductiveConstructor{↑} \AgdaBound{x}\AgdaSymbol{)} \AgdaSymbol{=} \AgdaInductiveConstructor{var} \AgdaBound{x}\<%
\end{code}
}

\begin{code}%
\>\AgdaFunction{substitution} \AgdaSymbol{:} \AgdaSymbol{∀} \AgdaSymbol{\{}\AgdaBound{P}\AgdaSymbol{\}} \AgdaSymbol{\{}\AgdaBound{Q}\AgdaSymbol{\}}\<%
\\
\>[0]\AgdaIndent{2}{}\<[2]%
\>[2]\AgdaSymbol{\{}\AgdaBound{Γ} \AgdaSymbol{:} \AgdaDatatype{Context} \AgdaBound{P}\AgdaSymbol{\}} \AgdaSymbol{\{}\AgdaBound{Δ} \AgdaSymbol{:} \AgdaDatatype{Context} \AgdaBound{Q}\AgdaSymbol{\}} \AgdaSymbol{\{}\AgdaBound{δ}\AgdaSymbol{\}} \AgdaSymbol{\{}\AgdaBound{φ}\AgdaSymbol{\}} \AgdaSymbol{\{}\AgdaBound{σ}\AgdaSymbol{\}} \AgdaSymbol{→} \<[48]%
\>[48]\<%
\\
\>[0]\AgdaIndent{2}{}\<[2]%
\>[2]\AgdaBound{Γ} \AgdaDatatype{⊢} \AgdaBound{δ} \AgdaDatatype{∶} \AgdaBound{φ} \AgdaSymbol{→} \AgdaBound{σ} \AgdaFunction{∶} \AgdaBound{Γ} \AgdaFunction{⇒} \AgdaBound{Δ} \AgdaSymbol{→} \AgdaBound{Δ} \AgdaDatatype{⊢} \AgdaBound{δ} \AgdaFunction{⟦} \AgdaBound{σ} \AgdaFunction{⟧} \AgdaDatatype{∶} \AgdaBound{φ}\<%
\end{code}

\AgdaHide{
\begin{code}%
\>\AgdaFunction{substitution} \AgdaSymbol{(}\AgdaInductiveConstructor{var} \AgdaSymbol{\_)} \AgdaBound{σ∶Γ→Δ} \AgdaSymbol{=} \AgdaBound{σ∶Γ→Δ} \AgdaSymbol{\_}\<%
\\
\>\AgdaFunction{substitution} \AgdaSymbol{(}\AgdaInductiveConstructor{app} \AgdaBound{Γ⊢δ∶φ→ψ} \AgdaBound{Γ⊢ε∶φ}\AgdaSymbol{)} \AgdaBound{σ∶Γ→Δ} \AgdaSymbol{=} \AgdaInductiveConstructor{app} \AgdaSymbol{(}\AgdaFunction{substitution} \AgdaBound{Γ⊢δ∶φ→ψ} \AgdaBound{σ∶Γ→Δ}\AgdaSymbol{)} \AgdaSymbol{(}\AgdaFunction{substitution} \AgdaBound{Γ⊢ε∶φ} \AgdaBound{σ∶Γ→Δ}\AgdaSymbol{)}\<%
\\
\>\AgdaFunction{substitution} \AgdaSymbol{\{}\AgdaArgument{Q} \AgdaSymbol{=} \AgdaBound{Q}\AgdaSymbol{\}} \AgdaSymbol{\{}\AgdaArgument{Δ} \AgdaSymbol{=} \AgdaBound{Δ}\AgdaSymbol{\}} \AgdaSymbol{\{}\AgdaArgument{σ} \AgdaSymbol{=} \AgdaBound{σ}\AgdaSymbol{\}} \AgdaSymbol{(}\AgdaInductiveConstructor{Λ} \AgdaSymbol{\{}\AgdaBound{P}\AgdaSymbol{\}} \AgdaSymbol{\{}\AgdaBound{Γ}\AgdaSymbol{\}} \AgdaSymbol{\{}\AgdaBound{φ}\AgdaSymbol{\}} \AgdaSymbol{\{}\AgdaBound{δ}\AgdaSymbol{\}} \AgdaSymbol{\{}\AgdaBound{ψ}\AgdaSymbol{\}} \AgdaBound{Γ,φ⊢δ∶ψ}\AgdaSymbol{)} \AgdaBound{σ∶Γ→Δ} \AgdaSymbol{=} \AgdaInductiveConstructor{Λ} \<[79]%
\>[79]\<%
\\
\>[0]\AgdaIndent{2}{}\<[2]%
\>[2]\AgdaSymbol{(}\AgdaFunction{substitution} \AgdaBound{Γ,φ⊢δ∶ψ} \AgdaSymbol{(}\AgdaFunction{liftSub-typed} \AgdaBound{σ∶Γ→Δ}\AgdaSymbol{))}\<%
\end{code}
}

\begin{lemma}[Subject Reduction]
If $\Gamma \vdash \delta : \phi$ and $\delta \rightarrow_\beta \epsilon$ then $\Gamma \vdash \epsilon : \phi$.
\end{lemma}

\begin{code}%
\>\AgdaFunction{subject-reduction} \AgdaSymbol{:} \AgdaSymbol{∀} \AgdaSymbol{\{}\AgdaBound{P}\AgdaSymbol{\}} \AgdaSymbol{\{}\AgdaBound{Γ} \AgdaSymbol{:} \AgdaDatatype{Context} \AgdaBound{P}\AgdaSymbol{\}} \AgdaSymbol{\{}\AgdaBound{δ} \AgdaBound{ε} \AgdaSymbol{:} \AgdaFunction{Proof} \AgdaSymbol{(} \AgdaBound{P}\AgdaSymbol{)\}} \AgdaSymbol{\{}\AgdaBound{φ}\AgdaSymbol{\}} \AgdaSymbol{→} \<[67]%
\>[67]\<%
\\
\>[0]\AgdaIndent{2}{}\<[2]%
\>[2]\AgdaBound{Γ} \AgdaDatatype{⊢} \AgdaBound{δ} \AgdaDatatype{∶} \AgdaBound{φ} \AgdaSymbol{→} \AgdaBound{δ} \AgdaDatatype{⇒} \AgdaBound{ε} \AgdaSymbol{→} \AgdaBound{Γ} \AgdaDatatype{⊢} \AgdaBound{ε} \AgdaDatatype{∶} \AgdaBound{φ}\<%
\end{code}

\AgdaHide{
\begin{code}%
\>\AgdaFunction{subject-reduction} \AgdaSymbol{(}\AgdaInductiveConstructor{var} \AgdaSymbol{\_)} \AgdaSymbol{()}\<%
\\
\>\AgdaFunction{subject-reduction} \AgdaSymbol{(}\AgdaInductiveConstructor{app} \AgdaSymbol{\{}\AgdaArgument{ε} \AgdaSymbol{=} \AgdaBound{ε}\AgdaSymbol{\}} \AgdaSymbol{(}\AgdaInductiveConstructor{Λ} \AgdaSymbol{\{}\AgdaBound{P}\AgdaSymbol{\}} \AgdaSymbol{\{}\AgdaBound{Γ}\AgdaSymbol{\}} \AgdaSymbol{\{}\AgdaBound{φ}\AgdaSymbol{\}} \AgdaSymbol{\{}\AgdaBound{δ}\AgdaSymbol{\}} \AgdaSymbol{\{}\AgdaBound{ψ}\AgdaSymbol{\}} \AgdaBound{Γ,φ⊢δ∶ψ}\AgdaSymbol{)} \AgdaBound{Γ⊢ε∶φ}\AgdaSymbol{)} \AgdaSymbol{(}\AgdaInductiveConstructor{redex} \AgdaInductiveConstructor{βI}\AgdaSymbol{)} \AgdaSymbol{=} \<[83]%
\>[83]\<%
\\
\>[0]\AgdaIndent{2}{}\<[2]%
\>[2]\AgdaFunction{substitution} \AgdaBound{Γ,φ⊢δ∶ψ} \AgdaSymbol{(}\AgdaFunction{botSub-typed} \AgdaBound{Γ⊢ε∶φ}\AgdaSymbol{)}\<%
\\
\>\AgdaFunction{subject-reduction} \AgdaSymbol{(}\AgdaInductiveConstructor{app} \AgdaBound{Γ⊢δ∶φ→ψ} \AgdaBound{Γ⊢ε∶φ}\AgdaSymbol{)} \AgdaSymbol{(}\AgdaInductiveConstructor{app} \AgdaSymbol{(}\AgdaInductiveConstructor{appl} \AgdaBound{δ→δ'}\AgdaSymbol{))} \AgdaSymbol{=} \AgdaInductiveConstructor{app} \AgdaSymbol{(}\AgdaFunction{subject-reduction} \AgdaBound{Γ⊢δ∶φ→ψ} \AgdaBound{δ→δ'}\AgdaSymbol{)} \AgdaBound{Γ⊢ε∶φ}\<%
\\
\>\AgdaFunction{subject-reduction} \AgdaSymbol{(}\AgdaInductiveConstructor{app} \AgdaBound{Γ⊢δ∶φ→ψ} \AgdaBound{Γ⊢ε∶φ}\AgdaSymbol{)} \AgdaSymbol{(}\AgdaInductiveConstructor{app} \AgdaSymbol{(}\AgdaInductiveConstructor{appr} \AgdaSymbol{(}\AgdaInductiveConstructor{appl} \AgdaBound{ε→ε'}\AgdaSymbol{)))} \AgdaSymbol{=} \AgdaInductiveConstructor{app} \AgdaBound{Γ⊢δ∶φ→ψ} \AgdaSymbol{(}\AgdaFunction{subject-reduction} \AgdaBound{Γ⊢ε∶φ} \AgdaBound{ε→ε'}\AgdaSymbol{)}\<%
\\
\>\AgdaFunction{subject-reduction} \AgdaSymbol{(}\AgdaInductiveConstructor{app} \AgdaBound{Γ⊢δ∶φ→ψ} \AgdaBound{Γ⊢ε∶φ}\AgdaSymbol{)} \AgdaSymbol{(}\AgdaInductiveConstructor{app} \AgdaSymbol{(}\AgdaInductiveConstructor{appr} \AgdaSymbol{(}\AgdaInductiveConstructor{appr} \AgdaSymbol{())))}\<%
\\
\>\AgdaFunction{subject-reduction} \AgdaSymbol{(}\AgdaInductiveConstructor{Λ} \AgdaSymbol{\_)} \AgdaSymbol{(}\AgdaInductiveConstructor{redex} \AgdaSymbol{())}\<%
\\
\>\AgdaFunction{subject-reduction} \AgdaSymbol{(}\AgdaInductiveConstructor{Λ} \AgdaBound{Γ,φ⊢δ∶ψ}\AgdaSymbol{)} \AgdaSymbol{(}\AgdaInductiveConstructor{app} \AgdaSymbol{(}\AgdaInductiveConstructor{appl} \AgdaBound{δ⇒ε}\AgdaSymbol{))} \AgdaSymbol{=} \AgdaInductiveConstructor{Λ} \AgdaSymbol{(}\AgdaFunction{subject-reduction} \AgdaBound{Γ,φ⊢δ∶ψ} \AgdaBound{δ⇒ε}\AgdaSymbol{)}\<%
\\
\>\AgdaFunction{subject-reduction} \AgdaSymbol{(}\AgdaInductiveConstructor{Λ} \AgdaBound{Γ⊢δ∶φ}\AgdaSymbol{)} \AgdaSymbol{(}\AgdaInductiveConstructor{app} \AgdaSymbol{(}\AgdaInductiveConstructor{appr} \AgdaSymbol{()))}\<%
\end{code}
}


\AgdaHide{
\begin{code}%
\>\AgdaKeyword{module} \AgdaModule{PHOPL.Computable} \AgdaKeyword{where}\<%
\\
\>\AgdaKeyword{open} \AgdaKeyword{import} \AgdaModule{Data.Empty} \AgdaKeyword{renaming} \AgdaSymbol{(}\AgdaDatatype{⊥} \AgdaSymbol{to} \AgdaDatatype{Empty}\AgdaSymbol{)}\<%
\\
\>\AgdaKeyword{open} \AgdaKeyword{import} \AgdaModule{Data.Product} \AgdaKeyword{renaming} \AgdaSymbol{(}\AgdaInductiveConstructor{\_,\_} \AgdaSymbol{to} \AgdaInductiveConstructor{\_,p\_}\AgdaSymbol{)}\<%
\\
\>\AgdaKeyword{open} \AgdaKeyword{import} \AgdaModule{Prelims}\<%
\\
\>\AgdaKeyword{open} \AgdaKeyword{import} \AgdaModule{PHOPL.Grammar}\<%
\\
\>\AgdaKeyword{open} \AgdaKeyword{import} \AgdaModule{PHOPL.PathSub}\<%
\\
\>\AgdaKeyword{open} \AgdaKeyword{import} \AgdaModule{PHOPL.Red}\<%
\\
\>\AgdaKeyword{open} \AgdaKeyword{import} \AgdaModule{PHOPL.SN}\<%
\\
\>\AgdaKeyword{open} \AgdaKeyword{import} \AgdaModule{PHOPL.Rules}\<%
\\
\>\AgdaKeyword{open} \AgdaKeyword{import} \AgdaModule{PHOPL.Meta}\<%
\\
\>\AgdaKeyword{open} \AgdaKeyword{import} \AgdaModule{PHOPL.KeyRedex}\<%
\\
\>\AgdaKeyword{open} \AgdaKeyword{import} \AgdaModule{PHOPL.Neutral}\<%
\end{code}
}

We define a model of the type theory with types as sets of terms.  For every type (proposition, equation) $A$ in context $\Gamma$, define
the set of \emph{computable} terms (proofs, paths) $E_\Gamma(A)$.

\begin{definition}[Computable Expressions]
\begin{align*}
E_\Gamma(\Omega) \eqdef & \{ M \mid \Gamma \vdash M : \Omega, M \in \SN \} \\
E_\Gamma(A \rightarrow B) \eqdef & \{ M \mid \Gamma \vdash M : A \rightarrow B, \\
& \quad \forall (\Delta \supseteq \Gamma) (N \in E_\Delta(A)). MN \in E_\Delta(B), \\
& \quad \forall (\Delta \supseteq \Gamma) (N, N' \in E_\Delta(A)) (P \in E_\Delta(N =_A N')). \\
& \quad \quad \reff{M}_{N N'} P \in E_\Gamma(MN =_B MN') \} \\
\\
E_\Gamma(\bot) & \eqdef \{ \delta \mid \Gamma \vdash \delta : \bot, \delta \in \SN \} \\
E_\Gamma(\phi \rightarrow \psi) & \eqdef \{ \delta \mid \Gamma \vdash \delta : \phi \rightarrow \psi, \\
& \forall (\Delta \supseteq \Gamma)(\epsilon \in E_\Delta(\phi)). \delta \epsilon \in E_\Gamma(\psi) \} \\
E_\Gamma(\phi) & \eqdef \{ \delta \mid \Gamma \vdash \delta : \phi, \delta \in \SN \} \\
& \qquad (\phi \text{ neutral}) \\
E_\Gamma(\phi) & \eqdef E_\Gamma(nf(\phi)) \\
& \qquad (\phi \mbox{ a normalizable term of type $\Omega$}) \\
\\
E_\Gamma(\phi =_\Omega \psi) & \eqdef \{ P \mid \Gamma \vdash P : \phi =_\Omega \psi, \\
& P^+ \in E_\Gamma(\phi \rightarrow \psi), P^- \in E_\Gamma(\psi \rightarrow \phi) \} \\
\\
E_\Gamma(M =_{A \rightarrow B} M') & \eqdef \{ P \mid \Gamma \vdash P : M =_{A \rightarrow B} M', \\
& \forall (\Delta \supseteq \Gamma) (N, N' \in E_\Delta(A)) (Q \in E_\Delta(N =_A N')). \\
& P_{NN'}Q \in E_\Delta(MN =_B M'N') \}
\end{align*}
\end{definition}

\AgdaHide{
\begin{code}%
\>\AgdaKeyword{data} \AgdaDatatype{Shape} \AgdaSymbol{:} \AgdaPrimitiveType{Set} \AgdaKeyword{where}\<%
\\
\>[0]\AgdaIndent{2}{}\<[2]%
\>[2]\AgdaInductiveConstructor{neutral} \AgdaSymbol{:} \AgdaDatatype{Shape}\<%
\\
\>[0]\AgdaIndent{2}{}\<[2]%
\>[2]\AgdaInductiveConstructor{bot} \AgdaSymbol{:} \AgdaDatatype{Shape}\<%
\\
\>[0]\AgdaIndent{2}{}\<[2]%
\>[2]\AgdaInductiveConstructor{imp} \AgdaSymbol{:} \AgdaDatatype{Shape} \AgdaSymbol{→} \AgdaDatatype{Shape} \AgdaSymbol{→} \AgdaDatatype{Shape}\<%
\\
%
\\
\>\AgdaKeyword{data} \AgdaDatatype{Leaves} \AgdaSymbol{(}\AgdaBound{V} \AgdaSymbol{:} \AgdaDatatype{Alphabet}\AgdaSymbol{)} \AgdaSymbol{:} \AgdaDatatype{Shape} \AgdaSymbol{→} \AgdaPrimitiveType{Set} \AgdaKeyword{where}\<%
\\
\>[0]\AgdaIndent{2}{}\<[2]%
\>[2]\AgdaInductiveConstructor{neutral} \AgdaSymbol{:} \AgdaDatatype{Neutral} \AgdaBound{V} \AgdaSymbol{→} \AgdaDatatype{Leaves} \AgdaBound{V} \AgdaInductiveConstructor{neutral}\<%
\\
\>[0]\AgdaIndent{2}{}\<[2]%
\>[2]\AgdaInductiveConstructor{bot} \AgdaSymbol{:} \AgdaDatatype{Leaves} \AgdaBound{V} \AgdaInductiveConstructor{bot}\<%
\\
\>[0]\AgdaIndent{2}{}\<[2]%
\>[2]\AgdaInductiveConstructor{imp} \AgdaSymbol{:} \AgdaSymbol{∀} \AgdaSymbol{\{}\AgdaBound{S}\AgdaSymbol{\}} \AgdaSymbol{\{}\AgdaBound{T}\AgdaSymbol{\}} \AgdaSymbol{→} \AgdaDatatype{Leaves} \AgdaBound{V} \AgdaBound{S} \AgdaSymbol{→} \AgdaDatatype{Leaves} \AgdaBound{V} \AgdaBound{T} \AgdaSymbol{→} \AgdaDatatype{Leaves} \AgdaBound{V} \AgdaSymbol{(}\AgdaInductiveConstructor{imp} \AgdaBound{S} \AgdaBound{T}\AgdaSymbol{)}\<%
\\
%
\\
\>\AgdaFunction{lrep} \AgdaSymbol{:} \AgdaSymbol{∀} \AgdaSymbol{\{}\AgdaBound{U}\AgdaSymbol{\}} \AgdaSymbol{\{}\AgdaBound{V}\AgdaSymbol{\}} \AgdaSymbol{\{}\AgdaBound{S}\AgdaSymbol{\}} \AgdaSymbol{→} \AgdaFunction{Rep} \AgdaBound{U} \AgdaBound{V} \AgdaSymbol{→} \AgdaDatatype{Leaves} \AgdaBound{U} \AgdaBound{S} \AgdaSymbol{→} \AgdaDatatype{Leaves} \AgdaBound{V} \AgdaBound{S}\<%
\\
\>\AgdaFunction{lrep} \AgdaBound{ρ} \AgdaSymbol{(}\AgdaInductiveConstructor{neutral} \AgdaBound{N}\AgdaSymbol{)} \AgdaSymbol{=} \AgdaInductiveConstructor{neutral} \AgdaSymbol{(}\AgdaFunction{nrep} \AgdaBound{ρ} \AgdaBound{N}\AgdaSymbol{)}\<%
\\
\>\AgdaFunction{lrep} \AgdaBound{ρ} \AgdaInductiveConstructor{bot} \AgdaSymbol{=} \AgdaInductiveConstructor{bot}\<%
\\
\>\AgdaFunction{lrep} \AgdaBound{ρ} \AgdaSymbol{(}\AgdaInductiveConstructor{imp} \AgdaBound{φ} \AgdaBound{ψ}\AgdaSymbol{)} \AgdaSymbol{=} \AgdaInductiveConstructor{imp} \AgdaSymbol{(}\AgdaFunction{lrep} \AgdaBound{ρ} \AgdaBound{φ}\AgdaSymbol{)} \AgdaSymbol{(}\AgdaFunction{lrep} \AgdaBound{ρ} \AgdaBound{ψ}\AgdaSymbol{)}\<%
\\
%
\\
\>\AgdaFunction{decode-Prop} \AgdaSymbol{:} \AgdaSymbol{∀} \AgdaSymbol{\{}\AgdaBound{V}\AgdaSymbol{\}} \AgdaSymbol{\{}\AgdaBound{S}\AgdaSymbol{\}} \AgdaSymbol{→} \AgdaDatatype{Leaves} \AgdaBound{V} \AgdaBound{S} \AgdaSymbol{→} \AgdaFunction{Term} \AgdaBound{V}\<%
\\
\>\AgdaFunction{decode-Prop} \AgdaSymbol{(}\AgdaInductiveConstructor{neutral} \AgdaBound{N}\AgdaSymbol{)} \AgdaSymbol{=} \AgdaFunction{decode-Neutral} \AgdaBound{N}\<%
\\
\>\AgdaFunction{decode-Prop} \AgdaInductiveConstructor{bot} \AgdaSymbol{=} \AgdaFunction{⊥}\<%
\\
\>\AgdaFunction{decode-Prop} \AgdaSymbol{(}\AgdaInductiveConstructor{imp} \AgdaBound{φ} \AgdaBound{ψ}\AgdaSymbol{)} \AgdaSymbol{=} \AgdaFunction{decode-Prop} \AgdaBound{φ} \AgdaFunction{⊃} \AgdaFunction{decode-Prop} \AgdaBound{ψ}\<%
\\
%
\\
\>\AgdaFunction{leaves-red} \AgdaSymbol{:} \AgdaSymbol{∀} \AgdaSymbol{\{}\AgdaBound{V}\AgdaSymbol{\}} \AgdaSymbol{\{}\AgdaBound{S}\AgdaSymbol{\}} \AgdaSymbol{\{}\AgdaBound{L} \AgdaSymbol{:} \AgdaDatatype{Leaves} \AgdaBound{V} \AgdaBound{S}\AgdaSymbol{\}} \AgdaSymbol{\{}\AgdaBound{φ} \AgdaSymbol{:} \AgdaFunction{Term} \AgdaBound{V}\AgdaSymbol{\}} \AgdaSymbol{→}\<%
\\
\>[0]\AgdaIndent{2}{}\<[2]%
\>[2]\AgdaFunction{decode-Prop} \AgdaBound{L} \AgdaDatatype{↠} \AgdaBound{φ} \AgdaSymbol{→}\<%
\\
\>[0]\AgdaIndent{2}{}\<[2]%
\>[2]\AgdaFunction{Σ[} \AgdaBound{L'} \AgdaFunction{∈} \AgdaDatatype{Leaves} \AgdaBound{V} \AgdaBound{S} \AgdaFunction{]} \AgdaFunction{decode-Prop} \AgdaBound{L'} \AgdaDatatype{≡} \AgdaBound{φ}\<%
\\
\>\AgdaFunction{leaves-red} \AgdaSymbol{\{}\AgdaArgument{S} \AgdaSymbol{=} \AgdaInductiveConstructor{neutral}\AgdaSymbol{\}} \AgdaSymbol{\{}\AgdaArgument{L} \AgdaSymbol{=} \AgdaInductiveConstructor{neutral} \AgdaBound{N}\AgdaSymbol{\}} \AgdaBound{L↠φ} \AgdaSymbol{=} \<[47]%
\>[47]\<%
\\
\>[0]\AgdaIndent{2}{}\<[2]%
\>[2]\AgdaKeyword{let} \AgdaSymbol{(}\AgdaBound{N} \AgdaInductiveConstructor{,p} \AgdaBound{N≡φ}\AgdaSymbol{)} \AgdaSymbol{=} \AgdaFunction{neutral-red} \AgdaSymbol{\{}\AgdaArgument{N} \AgdaSymbol{=} \AgdaBound{N}\AgdaSymbol{\}} \AgdaBound{L↠φ} \AgdaKeyword{in} \AgdaInductiveConstructor{neutral} \AgdaBound{N} \AgdaInductiveConstructor{,p} \AgdaBound{N≡φ}\<%
\\
\>\AgdaFunction{leaves-red} \AgdaSymbol{\{}\AgdaArgument{S} \AgdaSymbol{=} \AgdaInductiveConstructor{bot}\AgdaSymbol{\}} \AgdaSymbol{\{}\AgdaArgument{L} \AgdaSymbol{=} \AgdaInductiveConstructor{bot}\AgdaSymbol{\}} \AgdaBound{L↠φ} \AgdaSymbol{=} \AgdaInductiveConstructor{bot} \AgdaInductiveConstructor{,p} \AgdaFunction{sym} \AgdaSymbol{(}\AgdaFunction{bot-red} \AgdaBound{L↠φ}\AgdaSymbol{)}\<%
\\
\>\AgdaFunction{leaves-red} \AgdaSymbol{\{}\AgdaArgument{S} \AgdaSymbol{=} \AgdaInductiveConstructor{imp} \AgdaBound{S} \AgdaBound{T}\AgdaSymbol{\}} \AgdaSymbol{\{}\AgdaArgument{L} \AgdaSymbol{=} \AgdaInductiveConstructor{imp} \AgdaBound{φ} \AgdaBound{ψ}\AgdaSymbol{\}} \AgdaBound{φ⊃ψ↠χ} \AgdaSymbol{=} \<[47]%
\>[47]\<%
\\
\>[0]\AgdaIndent{2}{}\<[2]%
\>[2]\AgdaKeyword{let} \AgdaSymbol{(}\AgdaBound{φ'} \AgdaInductiveConstructor{,p} \AgdaBound{ψ'} \AgdaInductiveConstructor{,p} \AgdaBound{φ↠φ'} \AgdaInductiveConstructor{,p} \AgdaBound{ψ↠ψ'} \AgdaInductiveConstructor{,p} \AgdaBound{χ≡φ'⊃ψ'}\AgdaSymbol{)} \AgdaSymbol{=} \AgdaFunction{imp-red} \AgdaBound{φ⊃ψ↠χ} \AgdaKeyword{in} \<[63]%
\>[63]\<%
\\
\>[0]\AgdaIndent{2}{}\<[2]%
\>[2]\AgdaKeyword{let} \AgdaSymbol{(}\AgdaBound{L₁} \AgdaInductiveConstructor{,p} \AgdaBound{L₁≡φ'}\AgdaSymbol{)} \AgdaSymbol{=} \AgdaFunction{leaves-red} \AgdaSymbol{\{}\AgdaArgument{L} \AgdaSymbol{=} \AgdaBound{φ}\AgdaSymbol{\}} \AgdaBound{φ↠φ'} \AgdaKeyword{in} \<[49]%
\>[49]\<%
\\
\>[0]\AgdaIndent{2}{}\<[2]%
\>[2]\AgdaKeyword{let} \AgdaSymbol{(}\AgdaBound{L₂} \AgdaInductiveConstructor{,p} \AgdaBound{L₂≡ψ'}\AgdaSymbol{)} \AgdaSymbol{=} \AgdaFunction{leaves-red} \AgdaSymbol{\{}\AgdaArgument{L} \AgdaSymbol{=} \AgdaBound{ψ}\AgdaSymbol{\}} \AgdaBound{ψ↠ψ'} \AgdaKeyword{in} \<[49]%
\>[49]\<%
\\
\>[0]\AgdaIndent{2}{}\<[2]%
\>[2]\AgdaSymbol{(}\AgdaInductiveConstructor{imp} \AgdaBound{L₁} \AgdaBound{L₂}\AgdaSymbol{)} \AgdaInductiveConstructor{,p} \AgdaSymbol{(}\AgdaFunction{trans} \AgdaSymbol{(}\AgdaFunction{cong₂} \AgdaFunction{\_⊃\_} \AgdaBound{L₁≡φ'} \AgdaBound{L₂≡ψ'}\AgdaSymbol{)} \AgdaSymbol{(}\AgdaFunction{sym} \AgdaBound{χ≡φ'⊃ψ'}\AgdaSymbol{))}\<%
\\
%
\\
\>\AgdaFunction{computeP} \AgdaSymbol{:} \AgdaSymbol{∀} \AgdaSymbol{\{}\AgdaBound{V}\AgdaSymbol{\}} \AgdaSymbol{\{}\AgdaBound{S}\AgdaSymbol{\}} \AgdaSymbol{→} \AgdaDatatype{Context} \AgdaBound{V} \AgdaSymbol{→} \AgdaDatatype{Leaves} \AgdaBound{V} \AgdaBound{S} \AgdaSymbol{→} \AgdaFunction{Proof} \AgdaBound{V} \AgdaSymbol{→} \AgdaPrimitiveType{Set}\<%
\\
\>\AgdaFunction{computeP} \AgdaSymbol{\{}\AgdaArgument{S} \AgdaSymbol{=} \AgdaInductiveConstructor{neutral}\AgdaSymbol{\}} \AgdaBound{Γ} \AgdaSymbol{(}\AgdaInductiveConstructor{neutral} \AgdaSymbol{\_)} \AgdaBound{δ} \AgdaSymbol{=} \AgdaDatatype{SN} \AgdaBound{δ}\<%
\\
\>\AgdaFunction{computeP} \AgdaSymbol{\{}\AgdaArgument{S} \AgdaSymbol{=} \AgdaInductiveConstructor{bot}\AgdaSymbol{\}} \AgdaBound{Γ} \AgdaInductiveConstructor{bot} \AgdaBound{δ} \AgdaSymbol{=} \AgdaDatatype{SN} \AgdaBound{δ}\<%
\\
\>\AgdaFunction{computeP} \AgdaSymbol{\{}\AgdaArgument{S} \AgdaSymbol{=} \AgdaInductiveConstructor{imp} \AgdaBound{S} \AgdaBound{T}\AgdaSymbol{\}} \AgdaBound{Γ} \AgdaSymbol{(}\AgdaInductiveConstructor{imp} \AgdaBound{φ} \AgdaBound{ψ}\AgdaSymbol{)} \AgdaBound{δ} \AgdaSymbol{=} \<[39]%
\>[39]\<%
\\
\>[0]\AgdaIndent{2}{}\<[2]%
\>[2]\AgdaSymbol{∀} \AgdaSymbol{\{}\AgdaBound{W}\AgdaSymbol{\}} \AgdaSymbol{(}\AgdaBound{Δ} \AgdaSymbol{:} \AgdaDatatype{Context} \AgdaBound{W}\AgdaSymbol{)} \AgdaSymbol{\{}\AgdaBound{ρ}\AgdaSymbol{\}} \AgdaSymbol{\{}\AgdaBound{ε}\AgdaSymbol{\}}\<%
\\
\>[0]\AgdaIndent{2}{}\<[2]%
\>[2]\AgdaSymbol{(}\AgdaBound{ρ∶Γ⇒RΔ} \AgdaSymbol{:} \AgdaBound{ρ} \AgdaPostulate{∶} \AgdaBound{Γ} \AgdaPostulate{⇒R} \AgdaBound{Δ}\AgdaSymbol{)} \AgdaSymbol{(}\AgdaBound{Δ⊢ε∶φ} \AgdaSymbol{:} \AgdaBound{Δ} \AgdaDatatype{⊢} \AgdaBound{ε} \AgdaDatatype{∶} \AgdaSymbol{(}\AgdaFunction{decode-Prop} \AgdaSymbol{(}\AgdaFunction{lrep} \AgdaBound{ρ} \AgdaBound{φ}\AgdaSymbol{)))}\<%
\\
\>[0]\AgdaIndent{2}{}\<[2]%
\>[2]\AgdaSymbol{(}\AgdaBound{computeε} \AgdaSymbol{:} \AgdaFunction{computeP} \AgdaSymbol{\{}\AgdaArgument{S} \AgdaSymbol{=} \AgdaBound{S}\AgdaSymbol{\}} \AgdaBound{Δ} \AgdaSymbol{(}\AgdaFunction{lrep} \AgdaBound{ρ} \AgdaBound{φ}\AgdaSymbol{)} \AgdaBound{ε}\AgdaSymbol{)} \AgdaSymbol{→} \<[49]%
\>[49]\<%
\\
\>[0]\AgdaIndent{2}{}\<[2]%
\>[2]\AgdaFunction{computeP} \AgdaSymbol{\{}\AgdaArgument{S} \AgdaSymbol{=} \AgdaBound{T}\AgdaSymbol{\}} \AgdaBound{Δ} \AgdaSymbol{(}\AgdaFunction{lrep} \AgdaBound{ρ} \AgdaBound{ψ}\AgdaSymbol{)} \AgdaSymbol{(}\AgdaFunction{appP} \AgdaSymbol{(}\AgdaBound{δ} \AgdaFunction{〈} \AgdaBound{ρ} \AgdaFunction{〉}\AgdaSymbol{)} \AgdaBound{ε}\AgdaSymbol{)}\<%
\\
%
\\
\>\AgdaFunction{computeT} \AgdaSymbol{:} \AgdaSymbol{∀} \AgdaSymbol{\{}\AgdaBound{V}\AgdaSymbol{\}} \AgdaSymbol{→} \AgdaDatatype{Context} \AgdaBound{V} \AgdaSymbol{→} \AgdaDatatype{Type} \AgdaSymbol{→} \AgdaFunction{Term} \AgdaBound{V} \AgdaSymbol{→} \AgdaPrimitiveType{Set}\<%
\\
\>\AgdaFunction{computeE} \AgdaSymbol{:} \AgdaSymbol{∀} \AgdaSymbol{\{}\AgdaBound{V}\AgdaSymbol{\}} \AgdaSymbol{→} \AgdaDatatype{Context} \AgdaBound{V} \AgdaSymbol{→} \AgdaFunction{Term} \AgdaBound{V} \AgdaSymbol{→} \AgdaDatatype{Type} \AgdaSymbol{→} \AgdaFunction{Term} \AgdaBound{V} \AgdaSymbol{→} \AgdaFunction{Path} \AgdaBound{V} \AgdaSymbol{→} \AgdaPrimitiveType{Set}\<%
\\
%
\\
\>\AgdaFunction{computeT} \AgdaBound{Γ} \AgdaInductiveConstructor{Ω} \AgdaBound{M} \AgdaSymbol{=} \AgdaDatatype{SN} \AgdaBound{M}\<%
\\
\>\AgdaFunction{computeT} \AgdaBound{Γ} \AgdaSymbol{(}\AgdaBound{A} \AgdaInductiveConstructor{⇛} \AgdaBound{B}\AgdaSymbol{)} \AgdaBound{M} \AgdaSymbol{=} \<[23]%
\>[23]\<%
\\
\>[0]\AgdaIndent{2}{}\<[2]%
\>[2]\AgdaSymbol{(∀} \AgdaSymbol{\{}\AgdaBound{W}\AgdaSymbol{\}} \AgdaSymbol{(}\AgdaBound{Δ} \AgdaSymbol{:} \AgdaDatatype{Context} \AgdaBound{W}\AgdaSymbol{)} \AgdaSymbol{\{}\AgdaBound{ρ}\AgdaSymbol{\}} \AgdaSymbol{\{}\AgdaBound{N}\AgdaSymbol{\}} \AgdaSymbol{(}\AgdaBound{ρ∶Γ⇒Δ} \AgdaSymbol{:} \AgdaBound{ρ} \AgdaPostulate{∶} \AgdaBound{Γ} \AgdaPostulate{⇒R} \AgdaBound{Δ}\AgdaSymbol{)} \AgdaSymbol{(}\AgdaBound{Δ⊢N∶A} \AgdaSymbol{:} \AgdaBound{Δ} \AgdaDatatype{⊢} \AgdaBound{N} \AgdaDatatype{∶} \AgdaFunction{ty} \AgdaBound{A}\AgdaSymbol{)} \AgdaSymbol{(}\AgdaBound{computeN} \AgdaSymbol{:} \AgdaFunction{computeT} \AgdaBound{Δ} \AgdaBound{A} \AgdaBound{N}\AgdaSymbol{)} \AgdaSymbol{→}\<%
\\
\>[2]\AgdaIndent{4}{}\<[4]%
\>[4]\AgdaFunction{computeT} \AgdaBound{Δ} \AgdaBound{B} \AgdaSymbol{(}\AgdaFunction{appT} \AgdaSymbol{(}\AgdaBound{M} \AgdaFunction{〈} \AgdaBound{ρ} \AgdaFunction{〉}\AgdaSymbol{)} \AgdaBound{N}\AgdaSymbol{))} \AgdaFunction{×}\<%
\\
\>[0]\AgdaIndent{2}{}\<[2]%
\>[2]\AgdaSymbol{(∀} \AgdaSymbol{\{}\AgdaBound{W}\AgdaSymbol{\}} \AgdaSymbol{(}\AgdaBound{Δ} \AgdaSymbol{:} \AgdaDatatype{Context} \AgdaBound{W}\AgdaSymbol{)} \AgdaSymbol{\{}\AgdaBound{ρ}\AgdaSymbol{\}} \AgdaSymbol{\{}\AgdaBound{N}\AgdaSymbol{\}} \AgdaSymbol{\{}\AgdaBound{N'}\AgdaSymbol{\}} \AgdaSymbol{\{}\AgdaBound{P}\AgdaSymbol{\}} \<[42]%
\>[42]\<%
\\
\>[2]\AgdaIndent{4}{}\<[4]%
\>[4]\AgdaSymbol{(}\AgdaBound{ρ∶Γ⇒Δ} \AgdaSymbol{:} \AgdaBound{ρ} \AgdaPostulate{∶} \AgdaBound{Γ} \AgdaPostulate{⇒R} \AgdaBound{Δ}\AgdaSymbol{)} \AgdaSymbol{(}\AgdaBound{Δ⊢P∶N≡N'} \AgdaSymbol{:} \AgdaBound{Δ} \AgdaDatatype{⊢} \AgdaBound{P} \AgdaDatatype{∶} \AgdaBound{N} \AgdaFunction{≡〈} \AgdaBound{A} \AgdaFunction{〉} \AgdaBound{N'}\AgdaSymbol{)} \<[58]%
\>[58]\<%
\\
\>[2]\AgdaIndent{4}{}\<[4]%
\>[4]\AgdaSymbol{(}\AgdaBound{computeN} \AgdaSymbol{:} \AgdaFunction{computeT} \AgdaBound{Δ} \AgdaBound{A} \AgdaBound{N}\AgdaSymbol{)} \AgdaSymbol{(}\AgdaBound{computeN'} \AgdaSymbol{:} \AgdaFunction{computeT} \AgdaBound{Δ} \AgdaBound{A} \AgdaBound{N'}\AgdaSymbol{)} \AgdaSymbol{(}\AgdaBound{computeP} \AgdaSymbol{:} \AgdaFunction{computeE} \AgdaBound{Δ} \AgdaBound{N} \AgdaBound{A} \AgdaBound{N'} \AgdaBound{P}\AgdaSymbol{)} \AgdaSymbol{→}\<%
\\
\>[2]\AgdaIndent{4}{}\<[4]%
\>[4]\AgdaFunction{computeE} \AgdaBound{Δ} \AgdaSymbol{(}\AgdaFunction{appT} \AgdaSymbol{(}\AgdaBound{M} \AgdaFunction{〈} \AgdaBound{ρ} \AgdaFunction{〉}\AgdaSymbol{)} \AgdaBound{N}\AgdaSymbol{)} \AgdaBound{B} \AgdaSymbol{(}\AgdaFunction{appT} \AgdaSymbol{(}\AgdaBound{M} \AgdaFunction{〈} \AgdaBound{ρ} \AgdaFunction{〉}\AgdaSymbol{)} \AgdaBound{N'}\AgdaSymbol{)} \AgdaSymbol{(}\AgdaBound{M} \AgdaFunction{〈} \AgdaBound{ρ} \AgdaFunction{〉} \AgdaFunction{⋆[} \AgdaBound{P} \AgdaFunction{∶} \AgdaBound{N} \AgdaFunction{∼} \AgdaBound{N'} \AgdaFunction{]}\AgdaSymbol{))}\<%
\\
%
\\
\>\AgdaFunction{computeE} \AgdaSymbol{\{}\AgdaBound{V}\AgdaSymbol{\}} \AgdaBound{Γ} \AgdaBound{M} \AgdaInductiveConstructor{Ω} \AgdaBound{N} \AgdaBound{P} \AgdaSymbol{=} \AgdaFunction{Σ[} \AgdaBound{S} \AgdaFunction{∈} \AgdaDatatype{Shape} \AgdaFunction{]} \AgdaFunction{Σ[} \AgdaBound{T} \AgdaFunction{∈} \AgdaDatatype{Shape} \AgdaFunction{]} \AgdaFunction{Σ[} \AgdaBound{L} \AgdaFunction{∈} \AgdaDatatype{Leaves} \AgdaBound{V} \AgdaBound{S} \AgdaFunction{]} \AgdaFunction{Σ[} \AgdaBound{L'} \AgdaFunction{∈} \AgdaDatatype{Leaves} \AgdaBound{V} \AgdaBound{T} \AgdaFunction{]} \AgdaBound{M} \AgdaDatatype{↠} \AgdaFunction{decode-Prop} \AgdaBound{L} \AgdaFunction{×} \AgdaBound{N} \AgdaDatatype{↠} \AgdaFunction{decode-Prop} \AgdaBound{L'} \AgdaFunction{×} \AgdaFunction{computeP} \AgdaBound{Γ} \AgdaSymbol{(}\AgdaInductiveConstructor{imp} \AgdaBound{L} \AgdaBound{L'}\AgdaSymbol{)} \AgdaSymbol{(}\AgdaFunction{plus} \AgdaBound{P}\AgdaSymbol{)} \AgdaFunction{×} \AgdaFunction{computeP} \AgdaBound{Γ} \AgdaSymbol{(}\AgdaInductiveConstructor{imp} \AgdaBound{L'} \AgdaBound{L}\AgdaSymbol{)} \AgdaSymbol{(}\AgdaFunction{minus} \AgdaBound{P}\AgdaSymbol{)}\<%
\\
\>\AgdaFunction{computeE} \AgdaBound{Γ} \AgdaBound{M} \AgdaSymbol{(}\AgdaBound{A} \AgdaInductiveConstructor{⇛} \AgdaBound{B}\AgdaSymbol{)} \AgdaBound{M'} \AgdaBound{P} \AgdaSymbol{=}\<%
\\
\>[0]\AgdaIndent{2}{}\<[2]%
\>[2]\AgdaSymbol{∀} \AgdaSymbol{\{}\AgdaBound{W}\AgdaSymbol{\}} \AgdaSymbol{(}\AgdaBound{Δ} \AgdaSymbol{:} \AgdaDatatype{Context} \AgdaBound{W}\AgdaSymbol{)} \AgdaSymbol{\{}\AgdaBound{ρ}\AgdaSymbol{\}} \AgdaSymbol{\{}\AgdaBound{N}\AgdaSymbol{\}} \AgdaSymbol{\{}\AgdaBound{N'}\AgdaSymbol{\}} \AgdaSymbol{\{}\AgdaBound{Q}\AgdaSymbol{\}} \AgdaSymbol{(}\AgdaBound{ρ∶Γ⇒RΔ} \AgdaSymbol{:} \AgdaBound{ρ} \AgdaPostulate{∶} \AgdaBound{Γ} \AgdaPostulate{⇒R} \AgdaBound{Δ}\AgdaSymbol{)} \AgdaSymbol{(}\AgdaBound{Δ⊢Q∶N≡N'} \AgdaSymbol{:} \AgdaBound{Δ} \AgdaDatatype{⊢} \AgdaBound{Q} \AgdaDatatype{∶} \AgdaBound{N} \AgdaFunction{≡〈} \AgdaBound{A} \AgdaFunction{〉} \AgdaBound{N'}\AgdaSymbol{)}\<%
\\
\>[0]\AgdaIndent{2}{}\<[2]%
\>[2]\AgdaSymbol{(}\AgdaBound{computeQ} \AgdaSymbol{:} \AgdaFunction{computeE} \AgdaBound{Δ} \AgdaBound{N} \AgdaBound{A} \AgdaBound{N'} \AgdaBound{Q}\AgdaSymbol{)} \AgdaSymbol{→} \AgdaFunction{computeE} \AgdaBound{Δ} \AgdaSymbol{(}\AgdaFunction{appT} \AgdaSymbol{(}\AgdaBound{M} \AgdaFunction{〈} \AgdaBound{ρ} \AgdaFunction{〉}\AgdaSymbol{)} \AgdaBound{N}\AgdaSymbol{)} \AgdaBound{B} \AgdaSymbol{(}\AgdaFunction{appT} \AgdaSymbol{(}\AgdaBound{M'} \AgdaFunction{〈} \AgdaBound{ρ} \AgdaFunction{〉}\AgdaSymbol{)} \<[87]%
\>[87]\AgdaBound{N'}\AgdaSymbol{)} \<[91]%
\>[91]\<%
\\
\>[2]\AgdaIndent{4}{}\<[4]%
\>[4]\AgdaSymbol{(}\AgdaFunction{app*} \AgdaBound{N} \AgdaBound{N'} \AgdaSymbol{(}\AgdaBound{P} \AgdaFunction{〈} \AgdaBound{ρ} \AgdaFunction{〉}\AgdaSymbol{)} \AgdaBound{Q}\AgdaSymbol{)}\<%
\\
%
\\
\>\AgdaKeyword{postulate} \AgdaPostulate{decode-rep} \AgdaSymbol{:} \AgdaSymbol{∀} \AgdaSymbol{\{}\AgdaBound{U}\AgdaSymbol{\}} \AgdaSymbol{\{}\AgdaBound{V}\AgdaSymbol{\}} \AgdaSymbol{\{}\AgdaBound{S}\AgdaSymbol{\}} \AgdaSymbol{(}\AgdaBound{L} \AgdaSymbol{:} \AgdaDatatype{Leaves} \AgdaBound{U} \AgdaBound{S}\AgdaSymbol{)} \AgdaSymbol{\{}\AgdaBound{ρ} \AgdaSymbol{:} \AgdaFunction{Rep} \AgdaBound{U} \AgdaBound{V}\AgdaSymbol{\}} \AgdaSymbol{→}\<%
\\
\>[4]\AgdaIndent{21}{}\<[21]%
\>[21]\AgdaFunction{decode-Prop} \AgdaSymbol{(}\AgdaFunction{lrep} \AgdaBound{ρ} \AgdaBound{L}\AgdaSymbol{)} \AgdaDatatype{≡} \AgdaFunction{decode-Prop} \AgdaBound{L} \AgdaFunction{〈} \AgdaBound{ρ} \AgdaFunction{〉}\<%
\\
%
\\
\>\AgdaKeyword{postulate} \AgdaPostulate{conv-computeP} \AgdaSymbol{:} \AgdaSymbol{∀} \AgdaSymbol{\{}\AgdaBound{V}\AgdaSymbol{\}} \AgdaSymbol{\{}\AgdaBound{Γ} \AgdaSymbol{:} \AgdaDatatype{Context} \AgdaBound{V}\AgdaSymbol{\}} \AgdaSymbol{\{}\AgdaBound{S}\AgdaSymbol{\}} \AgdaSymbol{\{}\AgdaBound{L} \AgdaBound{M} \AgdaSymbol{:} \AgdaDatatype{Leaves} \AgdaBound{V} \AgdaBound{S}\AgdaSymbol{\}} \AgdaSymbol{\{}\AgdaBound{δ}\AgdaSymbol{\}} \AgdaSymbol{→}\<%
\\
\>[21]\AgdaIndent{24}{}\<[24]%
\>[24]\AgdaFunction{computeP} \AgdaBound{Γ} \AgdaBound{L} \AgdaBound{δ} \AgdaSymbol{→} \AgdaFunction{decode-Prop} \AgdaBound{L} \AgdaDatatype{≃} \AgdaFunction{decode-Prop} \AgdaBound{M} \AgdaSymbol{→}\<%
\\
\>[21]\AgdaIndent{24}{}\<[24]%
\>[24]\AgdaBound{Γ} \AgdaDatatype{⊢} \AgdaFunction{decode-Prop} \AgdaBound{M} \AgdaDatatype{∶} \AgdaFunction{ty} \AgdaInductiveConstructor{Ω} \AgdaSymbol{→} \AgdaFunction{computeP} \AgdaBound{Γ} \AgdaBound{M} \AgdaBound{δ}\<%
\\
%
\\
\>\AgdaFunction{conv-computeE} \AgdaSymbol{:} \AgdaSymbol{∀} \AgdaSymbol{\{}\AgdaBound{V}\AgdaSymbol{\}} \AgdaSymbol{\{}\AgdaBound{Γ} \AgdaSymbol{:} \AgdaDatatype{Context} \AgdaBound{V}\AgdaSymbol{\}} \AgdaSymbol{\{}\AgdaBound{M}\AgdaSymbol{\}} \AgdaSymbol{\{}\AgdaBound{M'}\AgdaSymbol{\}} \AgdaSymbol{\{}\AgdaBound{A}\AgdaSymbol{\}} \AgdaSymbol{\{}\AgdaBound{N}\AgdaSymbol{\}} \AgdaSymbol{\{}\AgdaBound{N'}\AgdaSymbol{\}} \AgdaSymbol{\{}\AgdaBound{P}\AgdaSymbol{\}} \AgdaSymbol{→}\<%
\\
\>[0]\AgdaIndent{2}{}\<[2]%
\>[2]\AgdaFunction{computeE} \AgdaBound{Γ} \AgdaBound{M} \AgdaBound{A} \AgdaBound{N} \AgdaBound{P} \AgdaSymbol{→} \<[23]%
\>[23]\<%
\\
\>[0]\AgdaIndent{2}{}\<[2]%
\>[2]\AgdaBound{Γ} \AgdaDatatype{⊢} \AgdaBound{M} \AgdaDatatype{∶} \AgdaFunction{ty} \AgdaBound{A} \AgdaSymbol{→} \AgdaBound{Γ} \AgdaDatatype{⊢} \AgdaBound{N} \AgdaDatatype{∶} \AgdaFunction{ty} \AgdaBound{A} \AgdaSymbol{→} \AgdaBound{Γ} \AgdaDatatype{⊢} \AgdaBound{M'} \AgdaDatatype{∶} \AgdaFunction{ty} \AgdaBound{A} \AgdaSymbol{→} \AgdaBound{Γ} \AgdaDatatype{⊢} \AgdaBound{N'} \AgdaDatatype{∶} \AgdaFunction{ty} \AgdaBound{A} \AgdaSymbol{→} \AgdaBound{M} \AgdaDatatype{≃} \AgdaBound{M'} \AgdaSymbol{→} \AgdaBound{N} \AgdaDatatype{≃} \AgdaBound{N'} \AgdaSymbol{→}\<%
\\
\>[0]\AgdaIndent{2}{}\<[2]%
\>[2]\AgdaFunction{computeE} \AgdaBound{Γ} \AgdaBound{M'} \AgdaBound{A} \AgdaBound{N'} \AgdaBound{P}\<%
\\
\>\AgdaFunction{conv-computeE} \AgdaSymbol{\{}\AgdaArgument{Γ} \AgdaSymbol{=} \AgdaBound{Γ}\AgdaSymbol{\}} \AgdaSymbol{\{}\AgdaArgument{M} \AgdaSymbol{=} \AgdaBound{M}\AgdaSymbol{\}} \AgdaSymbol{\{}\AgdaArgument{M'} \AgdaSymbol{=} \AgdaBound{M'}\AgdaSymbol{\}} \AgdaSymbol{\{}\AgdaArgument{A} \AgdaSymbol{=} \AgdaInductiveConstructor{Ω}\AgdaSymbol{\}} \AgdaSymbol{\{}\AgdaArgument{N'} \AgdaSymbol{=} \AgdaBound{N'}\AgdaSymbol{\}} \AgdaSymbol{\{}\AgdaBound{P}\AgdaSymbol{\}} \AgdaSymbol{(}\AgdaBound{S} \AgdaInductiveConstructor{,p} \AgdaBound{T} \AgdaInductiveConstructor{,p} \AgdaBound{φ} \AgdaInductiveConstructor{,p} \AgdaBound{ψ} \AgdaInductiveConstructor{,p} \AgdaBound{M↠φ} \AgdaInductiveConstructor{,p} \AgdaBound{N↠ψ} \AgdaInductiveConstructor{,p} \AgdaBound{computeP+} \AgdaInductiveConstructor{,p} \AgdaBound{computeP-}\AgdaSymbol{)} \<[121]%
\>[121]\<%
\\
\>[0]\AgdaIndent{2}{}\<[2]%
\>[2]\AgdaBound{Γ⊢M∶A} \AgdaBound{Γ⊢N∶A} \AgdaBound{Γ⊢M'∶A} \AgdaBound{Γ⊢N'∶A} \AgdaBound{M≃M'} \AgdaBound{N≃N'} \AgdaSymbol{=} \<[40]%
\>[40]\<%
\\
\>[2]\AgdaIndent{4}{}\<[4]%
\>[4]\AgdaKeyword{let} \AgdaSymbol{(}\AgdaBound{Q} \AgdaInductiveConstructor{,p} \AgdaBound{φ↠Q} \AgdaInductiveConstructor{,p} \AgdaBound{M'↠Q}\AgdaSymbol{)} \AgdaSymbol{=} \AgdaPostulate{confluenceT} \AgdaSymbol{(}\AgdaInductiveConstructor{trans-conv} \AgdaSymbol{(}\AgdaInductiveConstructor{sym-conv} \AgdaSymbol{(}\AgdaFunction{red-conv} \AgdaBound{M↠φ}\AgdaSymbol{))} \AgdaBound{M≃M'}\AgdaSymbol{)} \AgdaKeyword{in}\<%
\\
\>[2]\AgdaIndent{4}{}\<[4]%
\>[4]\AgdaKeyword{let} \AgdaSymbol{(}\AgdaBound{φ'} \AgdaInductiveConstructor{,p} \AgdaBound{φ'≡Q}\AgdaSymbol{)} \AgdaSymbol{=} \AgdaFunction{leaves-red} \AgdaSymbol{\{}\AgdaArgument{L} \AgdaSymbol{=} \AgdaBound{φ}\AgdaSymbol{\}} \AgdaBound{φ↠Q} \AgdaKeyword{in}\<%
\\
\>[2]\AgdaIndent{4}{}\<[4]%
\>[4]\AgdaKeyword{let} \AgdaSymbol{(}\AgdaBound{R} \AgdaInductiveConstructor{,p} \AgdaBound{ψ↠R} \AgdaInductiveConstructor{,p} \AgdaBound{N'↠R}\AgdaSymbol{)} \AgdaSymbol{=} \AgdaPostulate{confluenceT} \AgdaSymbol{(}\AgdaInductiveConstructor{trans-conv} \AgdaSymbol{(}\AgdaInductiveConstructor{sym-conv} \AgdaSymbol{(}\AgdaFunction{red-conv} \AgdaBound{N↠ψ}\AgdaSymbol{))} \AgdaBound{N≃N'}\AgdaSymbol{)} \AgdaKeyword{in}\<%
\\
\>[2]\AgdaIndent{4}{}\<[4]%
\>[4]\AgdaKeyword{let} \AgdaSymbol{(}\AgdaBound{ψ'} \AgdaInductiveConstructor{,p} \AgdaBound{ψ'≡R}\AgdaSymbol{)} \AgdaSymbol{=} \AgdaFunction{leaves-red} \AgdaSymbol{\{}\AgdaArgument{L} \AgdaSymbol{=} \AgdaBound{ψ}\AgdaSymbol{\}} \AgdaBound{ψ↠R} \AgdaKeyword{in}\<%
\\
\>[2]\AgdaIndent{4}{}\<[4]%
\>[4]\AgdaBound{S} \AgdaInductiveConstructor{,p} \AgdaBound{T} \AgdaInductiveConstructor{,p} \AgdaBound{φ'} \AgdaInductiveConstructor{,p} \AgdaBound{ψ'} \AgdaInductiveConstructor{,p} \AgdaFunction{subst} \AgdaSymbol{(}\AgdaDatatype{\_↠\_} \AgdaBound{M'}\AgdaSymbol{)} \AgdaSymbol{(}\AgdaFunction{sym} \AgdaBound{φ'≡Q}\AgdaSymbol{)} \AgdaBound{M'↠Q} \AgdaInductiveConstructor{,p} \<[60]%
\>[60]\<%
\\
\>[2]\AgdaIndent{4}{}\<[4]%
\>[4]\AgdaFunction{subst} \AgdaSymbol{(}\AgdaDatatype{\_↠\_} \AgdaBound{N'}\AgdaSymbol{)} \AgdaSymbol{(}\AgdaFunction{sym} \AgdaBound{ψ'≡R}\AgdaSymbol{)} \AgdaBound{N'↠R} \AgdaInductiveConstructor{,p} \<[38]%
\>[38]\<%
\\
\>[2]\AgdaIndent{4}{}\<[4]%
\>[4]\AgdaSymbol{(λ} \AgdaBound{Δ} \AgdaSymbol{\{}\AgdaBound{ρ}\AgdaSymbol{\}} \AgdaSymbol{\{}\AgdaBound{ε}\AgdaSymbol{\}} \AgdaBound{ρ∶Γ⇒RΔ} \AgdaBound{Δ⊢ε∶φ'ρ} \AgdaBound{computeε} \AgdaSymbol{→} \<[43]%
\>[43]\<%
\\
\>[4]\AgdaIndent{6}{}\<[6]%
\>[6]\AgdaKeyword{let} \AgdaBound{step1} \AgdaSymbol{:} \AgdaBound{Δ} \AgdaDatatype{⊢} \AgdaFunction{decode-Prop} \AgdaSymbol{(}\AgdaFunction{lrep} \AgdaBound{ρ} \AgdaBound{φ}\AgdaSymbol{)} \AgdaDatatype{∶} \AgdaFunction{ty} \AgdaInductiveConstructor{Ω}\<%
\\
\>[6]\AgdaIndent{10}{}\<[10]%
\>[10]\AgdaBound{step1} \AgdaSymbol{=} \AgdaFunction{subst} \AgdaSymbol{(λ} \AgdaBound{x} \AgdaSymbol{→} \AgdaBound{Δ} \AgdaDatatype{⊢} \AgdaBound{x} \AgdaDatatype{∶} \AgdaFunction{ty} \AgdaInductiveConstructor{Ω}\AgdaSymbol{)} \<[45]%
\>[45]\<%
\\
\>[10]\AgdaIndent{12}{}\<[12]%
\>[12]\AgdaSymbol{(}\AgdaFunction{sym} \AgdaSymbol{(}\AgdaPostulate{decode-rep} \AgdaBound{φ}\AgdaSymbol{))} \<[33]%
\>[33]\<%
\\
\>[10]\AgdaIndent{12}{}\<[12]%
\>[12]\AgdaSymbol{(}\AgdaPostulate{weakening} \<[23]%
\>[23]\<%
\\
\>[12]\AgdaIndent{14}{}\<[14]%
\>[14]\AgdaSymbol{(}\AgdaPostulate{Subject-Reduction} \<[33]%
\>[33]\<%
\\
\>[14]\AgdaIndent{16}{}\<[16]%
\>[16]\AgdaBound{Γ⊢M∶A} \AgdaBound{M↠φ}\AgdaSymbol{)} \AgdaSymbol{(}\AgdaFunction{context-validity} \AgdaBound{Δ⊢ε∶φ'ρ}\AgdaSymbol{)} \AgdaBound{ρ∶Γ⇒RΔ}\AgdaSymbol{)} \AgdaKeyword{in}\<%
\\
\>[0]\AgdaIndent{6}{}\<[6]%
\>[6]\AgdaKeyword{let} \AgdaBound{step1a} \AgdaSymbol{:} \AgdaFunction{decode-Prop} \AgdaSymbol{(}\AgdaFunction{lrep} \AgdaBound{ρ} \AgdaBound{φ'}\AgdaSymbol{)} \AgdaDatatype{≃} \AgdaFunction{decode-Prop} \AgdaSymbol{(}\AgdaFunction{lrep} \AgdaBound{ρ} \AgdaBound{φ}\AgdaSymbol{)}\<%
\\
\>[6]\AgdaIndent{10}{}\<[10]%
\>[10]\AgdaBound{step1a} \AgdaSymbol{=} \AgdaFunction{subst₂} \AgdaDatatype{\_≃\_} \AgdaSymbol{(}\AgdaFunction{sym} \AgdaSymbol{(}\AgdaFunction{trans} \AgdaSymbol{(}\AgdaPostulate{decode-rep} \AgdaBound{φ'}\AgdaSymbol{)} \AgdaSymbol{(}\AgdaFunction{rep-congl} \AgdaBound{φ'≡Q}\AgdaSymbol{)))} \AgdaSymbol{(}\AgdaFunction{sym} \AgdaSymbol{(}\AgdaPostulate{decode-rep} \AgdaBound{φ}\AgdaSymbol{))} \AgdaSymbol{(}\AgdaPostulate{conv-rep} \AgdaSymbol{\{}\AgdaArgument{M} \AgdaSymbol{=} \AgdaBound{Q}\AgdaSymbol{\}} \AgdaSymbol{\{}\AgdaArgument{N} \AgdaSymbol{=} \AgdaFunction{decode-Prop} \AgdaBound{φ}\AgdaSymbol{\}} \<[136]%
\>[136]\<%
\\
\>[10]\AgdaIndent{12}{}\<[12]%
\>[12]\AgdaSymbol{(}\AgdaInductiveConstructor{sym-conv} \AgdaSymbol{(}\AgdaFunction{red-conv} \AgdaBound{φ↠Q}\AgdaSymbol{)))} \AgdaKeyword{in} \<[42]%
\>[42]\<%
\\
\>[0]\AgdaIndent{6}{}\<[6]%
\>[6]\AgdaKeyword{let} \AgdaBound{step2} \AgdaSymbol{:} \AgdaBound{Δ} \AgdaDatatype{⊢} \AgdaBound{ε} \AgdaDatatype{∶} \AgdaFunction{decode-Prop} \AgdaSymbol{(}\AgdaFunction{lrep} \AgdaBound{ρ} \AgdaBound{φ}\AgdaSymbol{)}\<%
\\
\>[6]\AgdaIndent{10}{}\<[10]%
\>[10]\AgdaBound{step2} \AgdaSymbol{=} \AgdaInductiveConstructor{convR} \AgdaBound{Δ⊢ε∶φ'ρ} \AgdaBound{step1} \AgdaBound{step1a} \AgdaKeyword{in}\<%
\\
\>[0]\AgdaIndent{6}{}\<[6]%
\>[6]\AgdaKeyword{let} \AgdaBound{step3} \AgdaSymbol{:} \AgdaFunction{computeP} \AgdaBound{Δ} \AgdaSymbol{(}\AgdaFunction{lrep} \AgdaBound{ρ} \AgdaBound{φ}\AgdaSymbol{)} \AgdaBound{ε}\<%
\\
\>[6]\AgdaIndent{10}{}\<[10]%
\>[10]\AgdaBound{step3} \AgdaSymbol{=} \AgdaPostulate{conv-computeP} \AgdaSymbol{\{}\AgdaArgument{L} \AgdaSymbol{=} \AgdaFunction{lrep} \AgdaBound{ρ} \AgdaBound{φ'}\AgdaSymbol{\}} \AgdaSymbol{\{}\AgdaArgument{M} \AgdaSymbol{=} \AgdaFunction{lrep} \AgdaBound{ρ} \AgdaBound{φ}\AgdaSymbol{\}} \AgdaBound{computeε} \AgdaBound{step1a} \AgdaBound{step1} \AgdaKeyword{in}\<%
\\
\>[0]\AgdaIndent{6}{}\<[6]%
\>[6]\AgdaKeyword{let} \AgdaBound{step4} \AgdaSymbol{:} \AgdaFunction{computeP} \AgdaBound{Δ} \AgdaSymbol{(}\AgdaFunction{lrep} \AgdaBound{ρ} \AgdaBound{ψ}\AgdaSymbol{)} \AgdaSymbol{(}\AgdaFunction{appP} \AgdaSymbol{(}\AgdaFunction{plus} \AgdaBound{P} \AgdaFunction{〈} \AgdaBound{ρ} \AgdaFunction{〉}\AgdaSymbol{)} \AgdaBound{ε}\AgdaSymbol{)}\<%
\\
\>[6]\AgdaIndent{10}{}\<[10]%
\>[10]\AgdaBound{step4} \AgdaSymbol{=} \AgdaBound{computeP+} \AgdaBound{Δ} \AgdaBound{ρ∶Γ⇒RΔ} \AgdaBound{step2} \AgdaBound{step3} \AgdaKeyword{in} \<[52]%
\>[52]\<%
\\
\>[0]\AgdaIndent{6}{}\<[6]%
\>[6]\AgdaKeyword{let} \AgdaBound{step5} \AgdaSymbol{:} \AgdaFunction{decode-Prop} \AgdaSymbol{(}\AgdaFunction{lrep} \AgdaBound{ρ} \AgdaBound{ψ'}\AgdaSymbol{)} \AgdaDatatype{≃} \AgdaFunction{decode-Prop} \AgdaSymbol{(}\AgdaFunction{lrep} \AgdaBound{ρ} \AgdaBound{ψ}\AgdaSymbol{)}\<%
\\
\>[6]\AgdaIndent{10}{}\<[10]%
\>[10]\AgdaBound{step5} \AgdaSymbol{=} \AgdaFunction{subst₂} \AgdaDatatype{\_≃\_} \AgdaSymbol{(}\AgdaFunction{sym} \AgdaSymbol{(}\AgdaFunction{trans} \AgdaSymbol{(}\AgdaPostulate{decode-rep} \AgdaBound{ψ'}\AgdaSymbol{)} \AgdaSymbol{(}\AgdaFunction{rep-congl} \AgdaBound{ψ'≡R}\AgdaSymbol{)))} \AgdaSymbol{(}\AgdaFunction{sym} \AgdaSymbol{(}\AgdaPostulate{decode-rep} \AgdaBound{ψ}\AgdaSymbol{))} \AgdaSymbol{(}\AgdaPostulate{conv-rep} \AgdaSymbol{\{}\AgdaArgument{M} \AgdaSymbol{=} \AgdaBound{R}\AgdaSymbol{\}} \AgdaSymbol{\{}\AgdaArgument{N} \AgdaSymbol{=} \AgdaFunction{decode-Prop} \AgdaBound{ψ}\AgdaSymbol{\}} \<[135]%
\>[135]\<%
\\
\>[10]\AgdaIndent{12}{}\<[12]%
\>[12]\AgdaSymbol{(}\AgdaInductiveConstructor{sym-conv} \AgdaSymbol{(}\AgdaFunction{red-conv} \AgdaBound{ψ↠R}\AgdaSymbol{)))} \AgdaKeyword{in}\<%
\\
\>[0]\AgdaIndent{6}{}\<[6]%
\>[6]\AgdaKeyword{let} \AgdaBound{step6} \AgdaSymbol{:} \AgdaBound{Δ} \AgdaDatatype{⊢} \AgdaFunction{decode-Prop} \AgdaSymbol{(}\AgdaFunction{lrep} \AgdaBound{ρ} \AgdaBound{ψ'}\AgdaSymbol{)} \AgdaDatatype{∶} \AgdaFunction{ty} \AgdaInductiveConstructor{Ω}\<%
\\
\>[6]\AgdaIndent{10}{}\<[10]%
\>[10]\AgdaBound{step6} \AgdaSymbol{=} \AgdaFunction{subst} \AgdaSymbol{(λ} \AgdaBound{x} \AgdaSymbol{→} \AgdaBound{Δ} \AgdaDatatype{⊢} \AgdaBound{x} \AgdaDatatype{∶} \AgdaFunction{ty} \AgdaInductiveConstructor{Ω}\AgdaSymbol{)} \AgdaSymbol{(}\AgdaFunction{sym} \AgdaSymbol{(}\AgdaPostulate{decode-rep} \AgdaBound{ψ'}\AgdaSymbol{))} \<[67]%
\>[67]\<%
\\
\>[10]\AgdaIndent{16}{}\<[16]%
\>[16]\AgdaSymbol{(}\AgdaPostulate{weakening} \<[27]%
\>[27]\<%
\\
\>[16]\AgdaIndent{18}{}\<[18]%
\>[18]\AgdaSymbol{(}\AgdaFunction{subst} \AgdaSymbol{(λ} \AgdaBound{x} \AgdaSymbol{→} \AgdaBound{Γ} \AgdaDatatype{⊢} \AgdaBound{x} \AgdaDatatype{∶} \AgdaFunction{ty} \AgdaInductiveConstructor{Ω}\AgdaSymbol{)} \AgdaSymbol{(}\AgdaFunction{sym} \AgdaBound{ψ'≡R}\AgdaSymbol{)} \<[57]%
\>[57]\<%
\\
\>[16]\AgdaIndent{18}{}\<[18]%
\>[18]\AgdaSymbol{(}\AgdaPostulate{Subject-Reduction} \AgdaBound{Γ⊢N'∶A} \AgdaBound{N'↠R}\AgdaSymbol{))} \<[51]%
\>[51]\<%
\\
\>[0]\AgdaIndent{16}{}\<[16]%
\>[16]\AgdaSymbol{(}\AgdaFunction{context-validity} \AgdaBound{Δ⊢ε∶φ'ρ}\AgdaSymbol{)} \<[43]%
\>[43]\<%
\\
\>[0]\AgdaIndent{16}{}\<[16]%
\>[16]\AgdaBound{ρ∶Γ⇒RΔ}\AgdaSymbol{)} \AgdaKeyword{in}\<%
\\
\>[0]\AgdaIndent{6}{}\<[6]%
\>[6]\AgdaPostulate{conv-computeP} \AgdaSymbol{\{}\AgdaArgument{L} \AgdaSymbol{=} \AgdaFunction{lrep} \AgdaBound{ρ} \AgdaBound{ψ}\AgdaSymbol{\}} \AgdaSymbol{\{}\AgdaArgument{M} \AgdaSymbol{=} \AgdaFunction{lrep} \AgdaBound{ρ} \AgdaBound{ψ'}\AgdaSymbol{\}} \AgdaBound{step4} \AgdaSymbol{(}\AgdaInductiveConstructor{sym-conv} \AgdaBound{step5}\AgdaSymbol{)} \AgdaBound{step6}\AgdaSymbol{)} \AgdaInductiveConstructor{,p} \<[84]%
\>[84]\<%
\\
\>[0]\AgdaIndent{4}{}\<[4]%
\>[4]\AgdaSymbol{(} \<[9]%
\>[9]\AgdaSymbol{(λ} \AgdaBound{Δ} \AgdaSymbol{\{}\AgdaBound{ρ}\AgdaSymbol{\}} \AgdaSymbol{\{}\AgdaBound{ε}\AgdaSymbol{\}} \AgdaBound{ρ∶Γ⇒RΔ} \AgdaBound{Δ⊢ε∶ψ'ρ} \AgdaBound{computeε} \AgdaSymbol{→} \<[48]%
\>[48]\<%
\\
\>[4]\AgdaIndent{6}{}\<[6]%
\>[6]\AgdaKeyword{let} \AgdaBound{step1} \AgdaSymbol{:} \AgdaBound{Δ} \AgdaDatatype{⊢} \AgdaFunction{decode-Prop} \AgdaSymbol{(}\AgdaFunction{lrep} \AgdaBound{ρ} \AgdaBound{ψ}\AgdaSymbol{)} \AgdaDatatype{∶} \AgdaFunction{ty} \AgdaInductiveConstructor{Ω}\<%
\\
\>[6]\AgdaIndent{10}{}\<[10]%
\>[10]\AgdaBound{step1} \AgdaSymbol{=} \AgdaFunction{subst} \AgdaSymbol{(λ} \AgdaBound{x} \AgdaSymbol{→} \AgdaBound{Δ} \AgdaDatatype{⊢} \AgdaBound{x} \AgdaDatatype{∶} \AgdaFunction{ty} \AgdaInductiveConstructor{Ω}\AgdaSymbol{)} \<[45]%
\>[45]\<%
\\
\>[10]\AgdaIndent{12}{}\<[12]%
\>[12]\AgdaSymbol{(}\AgdaFunction{sym} \AgdaSymbol{(}\AgdaPostulate{decode-rep} \AgdaBound{ψ}\AgdaSymbol{))} \<[33]%
\>[33]\<%
\\
\>[10]\AgdaIndent{12}{}\<[12]%
\>[12]\AgdaSymbol{(}\AgdaPostulate{weakening} \<[23]%
\>[23]\<%
\\
\>[12]\AgdaIndent{14}{}\<[14]%
\>[14]\AgdaSymbol{(}\AgdaPostulate{Subject-Reduction} \<[33]%
\>[33]\<%
\\
\>[14]\AgdaIndent{16}{}\<[16]%
\>[16]\AgdaBound{Γ⊢N∶A} \AgdaBound{N↠ψ}\AgdaSymbol{)} \AgdaSymbol{(}\AgdaFunction{context-validity} \AgdaBound{Δ⊢ε∶ψ'ρ}\AgdaSymbol{)} \AgdaBound{ρ∶Γ⇒RΔ}\AgdaSymbol{)} \AgdaKeyword{in}\<%
\\
\>[0]\AgdaIndent{6}{}\<[6]%
\>[6]\AgdaKeyword{let} \AgdaBound{step1a} \AgdaSymbol{:} \AgdaFunction{decode-Prop} \AgdaSymbol{(}\AgdaFunction{lrep} \AgdaBound{ρ} \AgdaBound{ψ'}\AgdaSymbol{)} \AgdaDatatype{≃} \AgdaFunction{decode-Prop} \AgdaSymbol{(}\AgdaFunction{lrep} \AgdaBound{ρ} \AgdaBound{ψ}\AgdaSymbol{)}\<%
\\
\>[6]\AgdaIndent{10}{}\<[10]%
\>[10]\AgdaBound{step1a} \AgdaSymbol{=} \AgdaFunction{subst₂} \AgdaDatatype{\_≃\_} \AgdaSymbol{(}\AgdaFunction{sym} \AgdaSymbol{(}\AgdaFunction{trans} \AgdaSymbol{(}\AgdaPostulate{decode-rep} \AgdaBound{ψ'}\AgdaSymbol{)} \AgdaSymbol{(}\AgdaFunction{rep-congl} \AgdaBound{ψ'≡R}\AgdaSymbol{)))} \AgdaSymbol{(}\AgdaFunction{sym} \AgdaSymbol{(}\AgdaPostulate{decode-rep} \AgdaBound{ψ}\AgdaSymbol{))} \AgdaSymbol{(}\AgdaPostulate{conv-rep} \AgdaSymbol{\{}\AgdaArgument{M} \AgdaSymbol{=} \AgdaBound{R}\AgdaSymbol{\}} \AgdaSymbol{\{}\AgdaArgument{N} \AgdaSymbol{=} \AgdaFunction{decode-Prop} \AgdaBound{ψ}\AgdaSymbol{\}} \<[136]%
\>[136]\<%
\\
\>[10]\AgdaIndent{12}{}\<[12]%
\>[12]\AgdaSymbol{(}\AgdaInductiveConstructor{sym-conv} \AgdaSymbol{(}\AgdaFunction{red-conv} \AgdaBound{ψ↠R}\AgdaSymbol{)))} \AgdaKeyword{in} \<[42]%
\>[42]\<%
\\
\>[0]\AgdaIndent{6}{}\<[6]%
\>[6]\AgdaKeyword{let} \AgdaBound{step2} \AgdaSymbol{:} \AgdaBound{Δ} \AgdaDatatype{⊢} \AgdaBound{ε} \AgdaDatatype{∶} \AgdaFunction{decode-Prop} \AgdaSymbol{(}\AgdaFunction{lrep} \AgdaBound{ρ} \AgdaBound{ψ}\AgdaSymbol{)}\<%
\\
\>[6]\AgdaIndent{10}{}\<[10]%
\>[10]\AgdaBound{step2} \AgdaSymbol{=} \AgdaInductiveConstructor{convR} \AgdaBound{Δ⊢ε∶ψ'ρ} \AgdaBound{step1} \AgdaBound{step1a} \AgdaKeyword{in}\<%
\\
\>[0]\AgdaIndent{6}{}\<[6]%
\>[6]\AgdaKeyword{let} \AgdaBound{step3} \AgdaSymbol{:} \AgdaFunction{computeP} \AgdaBound{Δ} \AgdaSymbol{(}\AgdaFunction{lrep} \AgdaBound{ρ} \AgdaBound{ψ}\AgdaSymbol{)} \AgdaBound{ε}\<%
\\
\>[6]\AgdaIndent{10}{}\<[10]%
\>[10]\AgdaBound{step3} \AgdaSymbol{=} \AgdaPostulate{conv-computeP} \AgdaSymbol{\{}\AgdaArgument{L} \AgdaSymbol{=} \AgdaFunction{lrep} \AgdaBound{ρ} \AgdaBound{ψ'}\AgdaSymbol{\}} \AgdaSymbol{\{}\AgdaArgument{M} \AgdaSymbol{=} \AgdaFunction{lrep} \AgdaBound{ρ} \AgdaBound{ψ}\AgdaSymbol{\}} \AgdaBound{computeε} \AgdaBound{step1a} \AgdaBound{step1} \AgdaKeyword{in}\<%
\\
\>[0]\AgdaIndent{6}{}\<[6]%
\>[6]\AgdaKeyword{let} \AgdaBound{step4} \AgdaSymbol{:} \AgdaFunction{computeP} \AgdaBound{Δ} \AgdaSymbol{(}\AgdaFunction{lrep} \AgdaBound{ρ} \AgdaBound{φ}\AgdaSymbol{)} \AgdaSymbol{(}\AgdaFunction{appP} \AgdaSymbol{(}\AgdaFunction{minus} \AgdaBound{P} \AgdaFunction{〈} \AgdaBound{ρ} \AgdaFunction{〉}\AgdaSymbol{)} \AgdaBound{ε}\AgdaSymbol{)}\<%
\\
\>[6]\AgdaIndent{10}{}\<[10]%
\>[10]\AgdaBound{step4} \AgdaSymbol{=} \AgdaBound{computeP-} \AgdaBound{Δ} \AgdaBound{ρ∶Γ⇒RΔ} \AgdaBound{step2} \AgdaBound{step3} \AgdaKeyword{in} \<[52]%
\>[52]\<%
\\
\>[0]\AgdaIndent{6}{}\<[6]%
\>[6]\AgdaKeyword{let} \AgdaBound{step5} \AgdaSymbol{:} \AgdaFunction{decode-Prop} \AgdaSymbol{(}\AgdaFunction{lrep} \AgdaBound{ρ} \AgdaBound{φ'}\AgdaSymbol{)} \AgdaDatatype{≃} \AgdaFunction{decode-Prop} \AgdaSymbol{(}\AgdaFunction{lrep} \AgdaBound{ρ} \AgdaBound{φ}\AgdaSymbol{)}\<%
\\
\>[6]\AgdaIndent{10}{}\<[10]%
\>[10]\AgdaBound{step5} \AgdaSymbol{=} \AgdaFunction{subst₂} \AgdaDatatype{\_≃\_} \AgdaSymbol{(}\AgdaFunction{sym} \AgdaSymbol{(}\AgdaFunction{trans} \AgdaSymbol{(}\AgdaPostulate{decode-rep} \AgdaBound{φ'}\AgdaSymbol{)} \AgdaSymbol{(}\AgdaFunction{rep-congl} \AgdaBound{φ'≡Q}\AgdaSymbol{)))} \AgdaSymbol{(}\AgdaFunction{sym} \AgdaSymbol{(}\AgdaPostulate{decode-rep} \AgdaBound{φ}\AgdaSymbol{))} \AgdaSymbol{(}\AgdaPostulate{conv-rep} \AgdaSymbol{\{}\AgdaArgument{M} \AgdaSymbol{=} \AgdaBound{Q}\AgdaSymbol{\}} \AgdaSymbol{\{}\AgdaArgument{N} \AgdaSymbol{=} \AgdaFunction{decode-Prop} \AgdaBound{φ}\AgdaSymbol{\}} \<[135]%
\>[135]\<%
\\
\>[10]\AgdaIndent{12}{}\<[12]%
\>[12]\AgdaSymbol{(}\AgdaInductiveConstructor{sym-conv} \AgdaSymbol{(}\AgdaFunction{red-conv} \AgdaBound{φ↠Q}\AgdaSymbol{)))} \AgdaKeyword{in}\<%
\\
\>[0]\AgdaIndent{6}{}\<[6]%
\>[6]\AgdaKeyword{let} \AgdaBound{step6} \AgdaSymbol{:} \AgdaBound{Δ} \AgdaDatatype{⊢} \AgdaFunction{decode-Prop} \AgdaSymbol{(}\AgdaFunction{lrep} \AgdaBound{ρ} \AgdaBound{φ'}\AgdaSymbol{)} \AgdaDatatype{∶} \AgdaFunction{ty} \AgdaInductiveConstructor{Ω}\<%
\\
\>[6]\AgdaIndent{10}{}\<[10]%
\>[10]\AgdaBound{step6} \AgdaSymbol{=} \AgdaFunction{subst} \AgdaSymbol{(λ} \AgdaBound{x} \AgdaSymbol{→} \AgdaBound{Δ} \AgdaDatatype{⊢} \AgdaBound{x} \AgdaDatatype{∶} \AgdaFunction{ty} \AgdaInductiveConstructor{Ω}\AgdaSymbol{)} \AgdaSymbol{(}\AgdaFunction{sym} \AgdaSymbol{(}\AgdaPostulate{decode-rep} \AgdaBound{φ'}\AgdaSymbol{))} \<[67]%
\>[67]\<%
\\
\>[10]\AgdaIndent{16}{}\<[16]%
\>[16]\AgdaSymbol{(}\AgdaPostulate{weakening} \<[27]%
\>[27]\<%
\\
\>[16]\AgdaIndent{18}{}\<[18]%
\>[18]\AgdaSymbol{(}\AgdaFunction{subst} \AgdaSymbol{(λ} \AgdaBound{x} \AgdaSymbol{→} \AgdaBound{Γ} \AgdaDatatype{⊢} \AgdaBound{x} \AgdaDatatype{∶} \AgdaFunction{ty} \AgdaInductiveConstructor{Ω}\AgdaSymbol{)} \AgdaSymbol{(}\AgdaFunction{sym} \AgdaBound{φ'≡Q}\AgdaSymbol{)} \<[57]%
\>[57]\<%
\\
\>[16]\AgdaIndent{18}{}\<[18]%
\>[18]\AgdaSymbol{(}\AgdaPostulate{Subject-Reduction} \AgdaBound{Γ⊢M'∶A} \AgdaBound{M'↠Q}\AgdaSymbol{))} \<[51]%
\>[51]\<%
\\
\>[0]\AgdaIndent{16}{}\<[16]%
\>[16]\AgdaSymbol{(}\AgdaFunction{context-validity} \AgdaBound{Δ⊢ε∶ψ'ρ}\AgdaSymbol{)} \<[43]%
\>[43]\<%
\\
\>[0]\AgdaIndent{16}{}\<[16]%
\>[16]\AgdaBound{ρ∶Γ⇒RΔ}\AgdaSymbol{)} \AgdaKeyword{in}\<%
\\
\>[0]\AgdaIndent{6}{}\<[6]%
\>[6]\AgdaPostulate{conv-computeP} \AgdaSymbol{\{}\AgdaArgument{L} \AgdaSymbol{=} \AgdaFunction{lrep} \AgdaBound{ρ} \AgdaBound{φ}\AgdaSymbol{\}} \AgdaSymbol{\{}\AgdaArgument{M} \AgdaSymbol{=} \AgdaFunction{lrep} \AgdaBound{ρ} \AgdaBound{φ'}\AgdaSymbol{\}} \AgdaBound{step4} \AgdaSymbol{(}\AgdaInductiveConstructor{sym-conv} \AgdaBound{step5}\AgdaSymbol{)} \AgdaBound{step6}\AgdaSymbol{))}\<%
\\
\>\AgdaFunction{conv-computeE} \AgdaSymbol{\{}\AgdaArgument{A} \AgdaSymbol{=} \AgdaBound{A} \AgdaInductiveConstructor{⇛} \AgdaBound{B}\AgdaSymbol{\}} \AgdaBound{computeP} \AgdaBound{Γ⊢M∶A} \AgdaBound{Γ⊢N∶A} \AgdaBound{Γ⊢M'∶A} \AgdaBound{Γ⊢N'∶A} \AgdaBound{M≃M'} \AgdaBound{N≃N'} \AgdaBound{Δ} \AgdaBound{ρ∶Γ⇒RΔ} \AgdaBound{Δ⊢Q∶N≡N'} \AgdaBound{computeQ} \AgdaSymbol{=} \<[100]%
\>[100]\<%
\\
\>[0]\AgdaIndent{2}{}\<[2]%
\>[2]\AgdaFunction{conv-computeE} \AgdaSymbol{\{}\AgdaArgument{A} \AgdaSymbol{=} \AgdaBound{B}\AgdaSymbol{\}} \<[24]%
\>[24]\<%
\\
\>[0]\AgdaIndent{2}{}\<[2]%
\>[2]\AgdaSymbol{(}\AgdaBound{computeP} \AgdaBound{Δ} \AgdaBound{ρ∶Γ⇒RΔ} \AgdaBound{Δ⊢Q∶N≡N'} \AgdaBound{computeQ}\AgdaSymbol{)} \<[40]%
\>[40]\<%
\\
\>[2]\AgdaIndent{4}{}\<[4]%
\>[4]\AgdaSymbol{(}\AgdaInductiveConstructor{appR} \AgdaSymbol{(}\AgdaPostulate{weakening} \AgdaBound{Γ⊢M∶A} \AgdaSymbol{(}\AgdaFunction{context-validity} \AgdaBound{Δ⊢Q∶N≡N'}\AgdaSymbol{)} \AgdaBound{ρ∶Γ⇒RΔ}\AgdaSymbol{)} \<[63]%
\>[63]\<%
\\
\>[4]\AgdaIndent{6}{}\<[6]%
\>[6]\AgdaSymbol{(}\AgdaPostulate{Equation-Validity₁} \AgdaBound{Δ⊢Q∶N≡N'}\AgdaSymbol{))} \<[37]%
\>[37]\<%
\\
\>[0]\AgdaIndent{4}{}\<[4]%
\>[4]\AgdaSymbol{(}\AgdaInductiveConstructor{appR} \AgdaSymbol{(}\AgdaPostulate{weakening} \AgdaBound{Γ⊢N∶A} \AgdaSymbol{(}\AgdaFunction{context-validity} \AgdaBound{Δ⊢Q∶N≡N'}\AgdaSymbol{)} \AgdaBound{ρ∶Γ⇒RΔ}\AgdaSymbol{)} \<[63]%
\>[63]\<%
\\
\>[4]\AgdaIndent{6}{}\<[6]%
\>[6]\AgdaSymbol{(}\AgdaPostulate{Equation-Validity₂} \AgdaBound{Δ⊢Q∶N≡N'}\AgdaSymbol{))}\<%
\\
\>[0]\AgdaIndent{4}{}\<[4]%
\>[4]\AgdaSymbol{(}\AgdaInductiveConstructor{appR} \AgdaSymbol{(}\AgdaPostulate{weakening} \AgdaBound{Γ⊢M'∶A} \AgdaSymbol{(}\AgdaFunction{context-validity} \AgdaBound{Δ⊢Q∶N≡N'}\AgdaSymbol{)} \AgdaBound{ρ∶Γ⇒RΔ}\AgdaSymbol{)} \AgdaSymbol{(}\AgdaPostulate{Equation-Validity₁} \AgdaBound{Δ⊢Q∶N≡N'}\AgdaSymbol{))} \<[95]%
\>[95]\<%
\\
\>[0]\AgdaIndent{4}{}\<[4]%
\>[4]\AgdaSymbol{(}\AgdaInductiveConstructor{appR} \AgdaSymbol{(}\AgdaPostulate{weakening} \AgdaBound{Γ⊢N'∶A} \AgdaSymbol{(}\AgdaFunction{context-validity} \AgdaBound{Δ⊢Q∶N≡N'}\AgdaSymbol{)} \AgdaBound{ρ∶Γ⇒RΔ}\AgdaSymbol{)} \AgdaSymbol{(}\AgdaPostulate{Equation-Validity₂} \AgdaBound{Δ⊢Q∶N≡N'}\AgdaSymbol{))} \<[95]%
\>[95]\<%
\\
\>[0]\AgdaIndent{4}{}\<[4]%
\>[4]\AgdaSymbol{(}\AgdaPostulate{appT-convl} \AgdaSymbol{(}\AgdaPostulate{conv-rep} \AgdaBound{M≃M'}\AgdaSymbol{))} \AgdaSymbol{(}\AgdaPostulate{appT-convl} \AgdaSymbol{(}\AgdaPostulate{conv-rep} \AgdaBound{N≃N'}\AgdaSymbol{))}\<%
\\
\>\AgdaComment{--TODO Common pattern}\<%
\\
%
\\
\>\AgdaKeyword{postulate} \AgdaPostulate{expand-computeE} \AgdaSymbol{:} \AgdaSymbol{∀} \AgdaSymbol{\{}\AgdaBound{V}\AgdaSymbol{\}} \AgdaSymbol{\{}\AgdaBound{Γ} \AgdaSymbol{:} \AgdaDatatype{Context} \AgdaBound{V}\AgdaSymbol{\}} \AgdaSymbol{\{}\AgdaBound{M}\AgdaSymbol{\}} \AgdaSymbol{\{}\AgdaBound{A}\AgdaSymbol{\}} \AgdaSymbol{\{}\AgdaBound{N}\AgdaSymbol{\}} \AgdaSymbol{\{}\AgdaBound{P}\AgdaSymbol{\}} \AgdaSymbol{\{}\AgdaBound{Q}\AgdaSymbol{\}} \AgdaSymbol{→}\<%
\\
\>[4]\AgdaIndent{26}{}\<[26]%
\>[26]\AgdaFunction{computeE} \AgdaBound{Γ} \AgdaBound{M} \AgdaBound{A} \AgdaBound{N} \AgdaBound{Q} \AgdaSymbol{→} \AgdaBound{Γ} \AgdaDatatype{⊢} \AgdaBound{P} \AgdaDatatype{∶} \AgdaBound{M} \AgdaFunction{≡〈} \AgdaBound{A} \AgdaFunction{〉} \AgdaBound{N} \AgdaSymbol{→} \AgdaDatatype{key-redex} \AgdaBound{P} \AgdaBound{Q} \AgdaSymbol{→} \AgdaFunction{computeE} \AgdaBound{Γ} \AgdaBound{M} \AgdaBound{A} \AgdaBound{N} \AgdaBound{P}\<%
\\
%
\\
\>\AgdaFunction{expand-computeT} \AgdaSymbol{:} \AgdaSymbol{∀} \AgdaSymbol{\{}\AgdaBound{V}\AgdaSymbol{\}} \AgdaSymbol{\{}\AgdaBound{Γ} \AgdaSymbol{:} \AgdaDatatype{Context} \AgdaBound{V}\AgdaSymbol{\}} \AgdaSymbol{\{}\AgdaBound{A}\AgdaSymbol{\}} \AgdaSymbol{\{}\AgdaBound{M}\AgdaSymbol{\}} \AgdaSymbol{\{}\AgdaBound{N}\AgdaSymbol{\}} \AgdaSymbol{→} \AgdaFunction{computeT} \AgdaBound{Γ} \AgdaBound{A} \AgdaBound{N} \AgdaSymbol{→} \AgdaBound{Γ} \AgdaDatatype{⊢} \AgdaBound{M} \AgdaDatatype{∶} \AgdaFunction{ty} \AgdaBound{A} \AgdaSymbol{→} \AgdaDatatype{key-redex} \AgdaBound{M} \AgdaBound{N} \AgdaSymbol{→} \AgdaFunction{computeT} \AgdaBound{Γ} \AgdaBound{A} \AgdaBound{M}\<%
\\
\>\AgdaFunction{expand-computeT} \AgdaSymbol{\{}\AgdaArgument{A} \AgdaSymbol{=} \AgdaInductiveConstructor{Ω}\AgdaSymbol{\}} \AgdaBound{computeψ} \AgdaSymbol{\_} \AgdaBound{φ▷ψ} \AgdaSymbol{=} \AgdaFunction{key-redex-SN} \AgdaBound{computeψ} \AgdaBound{φ▷ψ}\<%
\\
\>\AgdaFunction{expand-computeT} \AgdaSymbol{\{}\AgdaArgument{A} \AgdaSymbol{=} \AgdaBound{A} \AgdaInductiveConstructor{⇛} \AgdaBound{B}\AgdaSymbol{\}} \AgdaSymbol{\{}\AgdaBound{M}\AgdaSymbol{\}} \AgdaSymbol{\{}\AgdaBound{M'}\AgdaSymbol{\}} \AgdaSymbol{(}\AgdaBound{computeM'app} \AgdaInductiveConstructor{,p} \AgdaBound{computeM'eq}\AgdaSymbol{)} \AgdaBound{Γ⊢M∶A⇛B} \AgdaBound{M▷M'} \AgdaSymbol{=} \<[82]%
\>[82]\<%
\\
\>[0]\AgdaIndent{2}{}\<[2]%
\>[2]\AgdaSymbol{(λ} \AgdaBound{Δ} \AgdaSymbol{\{}\AgdaBound{ρ}\AgdaSymbol{\}} \AgdaSymbol{\{}\AgdaBound{N}\AgdaSymbol{\}} \AgdaBound{ρ∶Γ⇒Δ} \AgdaBound{Δ⊢N∶A} \AgdaBound{computeN} \AgdaSymbol{→} \<[38]%
\>[38]\<%
\\
\>[2]\AgdaIndent{4}{}\<[4]%
\>[4]\AgdaKeyword{let} \AgdaBound{computeM'N} \AgdaSymbol{:} \AgdaFunction{computeT} \AgdaBound{Δ} \AgdaBound{B} \AgdaSymbol{(}\AgdaFunction{appT} \AgdaSymbol{(}\AgdaBound{M'} \AgdaFunction{〈} \AgdaBound{ρ} \AgdaFunction{〉}\AgdaSymbol{)} \AgdaBound{N}\AgdaSymbol{)}\<%
\\
\>[4]\AgdaIndent{8}{}\<[8]%
\>[8]\AgdaBound{computeM'N} \AgdaSymbol{=} \AgdaBound{computeM'app} \AgdaBound{Δ} \AgdaBound{ρ∶Γ⇒Δ} \AgdaBound{Δ⊢N∶A} \AgdaBound{computeN}\<%
\\
\>[0]\AgdaIndent{4}{}\<[4]%
\>[4]\AgdaKeyword{in} \AgdaFunction{expand-computeT} \AgdaBound{computeM'N} \<[34]%
\>[34]\<%
\\
\>[4]\AgdaIndent{7}{}\<[7]%
\>[7]\AgdaSymbol{(}\AgdaInductiveConstructor{appR} \AgdaSymbol{(}\AgdaPostulate{weakening} \AgdaBound{Γ⊢M∶A⇛B} \AgdaSymbol{(}\AgdaFunction{context-validity} \AgdaBound{Δ⊢N∶A}\AgdaSymbol{)} \AgdaBound{ρ∶Γ⇒Δ}\AgdaSymbol{)} \AgdaBound{Δ⊢N∶A}\AgdaSymbol{)} \<[71]%
\>[71]\<%
\\
\>[7]\AgdaIndent{13}{}\<[13]%
\>[13]\AgdaSymbol{(}\AgdaInductiveConstructor{appTkr} \AgdaSymbol{(}\AgdaFunction{key-redex-rep} \AgdaBound{M▷M'}\AgdaSymbol{)))} \AgdaInductiveConstructor{,p} \<[47]%
\>[47]\<%
\\
\>[0]\AgdaIndent{2}{}\<[2]%
\>[2]\AgdaSymbol{(λ} \AgdaBound{Δ} \AgdaBound{ρ∶Γ⇒Δ} \AgdaBound{Δ⊢P∶N≡N'} \AgdaBound{computeN} \AgdaBound{computeN'} \AgdaBound{computeP₁} \AgdaSymbol{→} \<[53]%
\>[53]\<%
\\
\>[2]\AgdaIndent{4}{}\<[4]%
\>[4]\AgdaPostulate{expand-computeE} \<[20]%
\>[20]\<%
\\
\>[4]\AgdaIndent{6}{}\<[6]%
\>[6]\AgdaSymbol{(}\AgdaFunction{conv-computeE} \<[21]%
\>[21]\<%
\\
\>[6]\AgdaIndent{8}{}\<[8]%
\>[8]\AgdaSymbol{(}\AgdaBound{computeM'eq} \AgdaBound{Δ} \AgdaBound{ρ∶Γ⇒Δ} \AgdaBound{Δ⊢P∶N≡N'} \AgdaBound{computeN} \AgdaBound{computeN'} \AgdaBound{computeP₁}\AgdaSymbol{)} \<[68]%
\>[68]\<%
\\
\>[6]\AgdaIndent{8}{}\<[8]%
\>[8]\AgdaSymbol{(}\AgdaInductiveConstructor{appR} \AgdaSymbol{(}\AgdaPostulate{weakening} \AgdaSymbol{(}\AgdaPostulate{Subject-Reduction} \AgdaBound{Γ⊢M∶A⇛B} \AgdaSymbol{(}\AgdaPostulate{key-redex-red} \AgdaBound{M▷M'}\AgdaSymbol{))} \<[74]%
\>[74]\<%
\\
\>[8]\AgdaIndent{25}{}\<[25]%
\>[25]\AgdaSymbol{(}\AgdaFunction{context-validity} \AgdaBound{Δ⊢P∶N≡N'}\AgdaSymbol{)} \AgdaBound{ρ∶Γ⇒Δ}\AgdaSymbol{)} \<[60]%
\>[60]\<%
\\
\>[0]\AgdaIndent{14}{}\<[14]%
\>[14]\AgdaSymbol{(}\AgdaPostulate{Equation-Validity₁} \AgdaBound{Δ⊢P∶N≡N'}\AgdaSymbol{))} \<[45]%
\>[45]\<%
\\
\>[0]\AgdaIndent{8}{}\<[8]%
\>[8]\AgdaSymbol{(}\AgdaInductiveConstructor{appR} \AgdaSymbol{(}\AgdaPostulate{weakening} \AgdaSymbol{(}\AgdaPostulate{Subject-Reduction} \AgdaBound{Γ⊢M∶A⇛B} \AgdaSymbol{(}\AgdaPostulate{key-redex-red} \AgdaBound{M▷M'}\AgdaSymbol{))} \<[74]%
\>[74]\<%
\\
\>[8]\AgdaIndent{25}{}\<[25]%
\>[25]\AgdaSymbol{(}\AgdaFunction{context-validity} \AgdaBound{Δ⊢P∶N≡N'}\AgdaSymbol{)} \AgdaBound{ρ∶Γ⇒Δ}\AgdaSymbol{)} \<[60]%
\>[60]\<%
\\
\>[0]\AgdaIndent{14}{}\<[14]%
\>[14]\AgdaSymbol{(}\AgdaPostulate{Equation-Validity₂} \AgdaBound{Δ⊢P∶N≡N'}\AgdaSymbol{))} \<[45]%
\>[45]\<%
\\
\>[0]\AgdaIndent{8}{}\<[8]%
\>[8]\AgdaSymbol{(}\AgdaInductiveConstructor{appR} \AgdaSymbol{(}\AgdaPostulate{weakening} \AgdaBound{Γ⊢M∶A⇛B} \AgdaSymbol{(}\AgdaFunction{context-validity} \AgdaBound{Δ⊢P∶N≡N'}\AgdaSymbol{)} \AgdaBound{ρ∶Γ⇒Δ}\AgdaSymbol{)} \<[68]%
\>[68]\<%
\\
\>[8]\AgdaIndent{14}{}\<[14]%
\>[14]\AgdaSymbol{(}\AgdaPostulate{Equation-Validity₁} \AgdaBound{Δ⊢P∶N≡N'}\AgdaSymbol{))} \<[45]%
\>[45]\<%
\\
\>[0]\AgdaIndent{8}{}\<[8]%
\>[8]\AgdaSymbol{(}\AgdaInductiveConstructor{appR} \AgdaSymbol{(}\AgdaPostulate{weakening} \AgdaBound{Γ⊢M∶A⇛B} \AgdaSymbol{(}\AgdaFunction{context-validity} \AgdaBound{Δ⊢P∶N≡N'}\AgdaSymbol{)} \AgdaBound{ρ∶Γ⇒Δ}\AgdaSymbol{)} \<[68]%
\>[68]\<%
\\
\>[8]\AgdaIndent{14}{}\<[14]%
\>[14]\AgdaSymbol{(}\AgdaPostulate{Equation-Validity₂} \AgdaBound{Δ⊢P∶N≡N'}\AgdaSymbol{))} \<[45]%
\>[45]\<%
\\
\>[0]\AgdaIndent{8}{}\<[8]%
\>[8]\AgdaSymbol{(}\AgdaInductiveConstructor{sym-conv} \AgdaSymbol{(}\AgdaPostulate{appT-convl} \AgdaSymbol{(}\AgdaFunction{red-conv} \AgdaSymbol{(}\AgdaPostulate{red-rep} \AgdaSymbol{(}\AgdaPostulate{key-redex-red} \AgdaBound{M▷M'}\AgdaSymbol{)))))} \<[74]%
\>[74]\<%
\\
\>[0]\AgdaIndent{8}{}\<[8]%
\>[8]\AgdaSymbol{(}\AgdaInductiveConstructor{sym-conv} \AgdaSymbol{(}\AgdaPostulate{appT-convl} \AgdaSymbol{(}\AgdaFunction{red-conv} \AgdaSymbol{(}\AgdaPostulate{red-rep} \AgdaSymbol{(}\AgdaPostulate{key-redex-red} \AgdaBound{M▷M'}\AgdaSymbol{))))))} \<[75]%
\>[75]\<%
\\
\>[0]\AgdaIndent{6}{}\<[6]%
\>[6]\AgdaSymbol{(}\AgdaPostulate{⋆-typed} \AgdaSymbol{(}\AgdaPostulate{weakening} \AgdaBound{Γ⊢M∶A⇛B} \AgdaSymbol{(}\AgdaFunction{context-validity} \AgdaBound{Δ⊢P∶N≡N'}\AgdaSymbol{)} \AgdaBound{ρ∶Γ⇒Δ}\AgdaSymbol{)} \<[69]%
\>[69]\<%
\\
\>[6]\AgdaIndent{8}{}\<[8]%
\>[8]\AgdaBound{Δ⊢P∶N≡N'}\AgdaSymbol{)} \<[18]%
\>[18]\<%
\\
\>[0]\AgdaIndent{6}{}\<[6]%
\>[6]\AgdaSymbol{(}\AgdaPostulate{key-redex-⋆} \AgdaSymbol{(}\AgdaFunction{key-redex-rep} \AgdaBound{M▷M'}\AgdaSymbol{)))}\<%
\\
%
\\
\>\AgdaFunction{compute} \AgdaSymbol{:} \AgdaSymbol{∀} \AgdaSymbol{\{}\AgdaBound{V}\AgdaSymbol{\}} \AgdaSymbol{\{}\AgdaBound{K}\AgdaSymbol{\}} \AgdaSymbol{→} \AgdaDatatype{Context} \AgdaBound{V} \AgdaSymbol{→} \AgdaFunction{Expression} \AgdaBound{V} \AgdaSymbol{(}\AgdaFunction{parent} \AgdaBound{K}\AgdaSymbol{)} \AgdaSymbol{→} \AgdaFunction{Expression} \AgdaBound{V} \AgdaSymbol{(}\AgdaInductiveConstructor{varKind} \AgdaBound{K}\AgdaSymbol{)} \AgdaSymbol{→} \AgdaPrimitiveType{Set}\<%
\\
\>\AgdaFunction{compute} \AgdaSymbol{\{}\AgdaArgument{K} \AgdaSymbol{=} \AgdaInductiveConstructor{-Term}\AgdaSymbol{\}} \AgdaBound{Γ} \AgdaSymbol{(}\AgdaInductiveConstructor{app} \AgdaSymbol{(}\AgdaInductiveConstructor{-ty} \AgdaBound{A}\AgdaSymbol{)} \AgdaInductiveConstructor{out}\AgdaSymbol{)} \AgdaBound{M} \AgdaSymbol{=} \AgdaFunction{computeT} \AgdaBound{Γ} \AgdaBound{A} \AgdaBound{M}\<%
\\
\>\AgdaFunction{compute} \AgdaSymbol{\{}\AgdaBound{V}\AgdaSymbol{\}} \AgdaSymbol{\{}\AgdaArgument{K} \AgdaSymbol{=} \AgdaInductiveConstructor{-Proof}\AgdaSymbol{\}} \AgdaBound{Γ} \AgdaBound{φ} \AgdaBound{δ} \AgdaSymbol{=} \AgdaFunction{Σ[} \AgdaBound{S} \AgdaFunction{∈} \AgdaDatatype{Shape} \AgdaFunction{]} \AgdaFunction{Σ[} \AgdaBound{L} \AgdaFunction{∈} \AgdaDatatype{Leaves} \AgdaBound{V} \AgdaBound{S} \AgdaFunction{]} \AgdaBound{φ} \AgdaDatatype{↠} \AgdaFunction{decode-Prop} \AgdaBound{L} \AgdaFunction{×} \AgdaFunction{computeP} \AgdaBound{Γ} \AgdaBound{L} \AgdaBound{δ}\<%
\\
\>\AgdaFunction{compute} \AgdaSymbol{\{}\AgdaArgument{K} \AgdaSymbol{=} \AgdaInductiveConstructor{-Path}\AgdaSymbol{\}} \AgdaBound{Γ} \AgdaSymbol{(}\AgdaInductiveConstructor{app} \AgdaSymbol{(}\AgdaInductiveConstructor{-eq} \AgdaBound{A}\AgdaSymbol{)} \AgdaSymbol{(}\AgdaBound{M} \AgdaInductiveConstructor{,,} \AgdaBound{N} \AgdaInductiveConstructor{,,} \AgdaInductiveConstructor{out}\AgdaSymbol{))} \AgdaBound{P} \AgdaSymbol{=} \AgdaFunction{computeE} \AgdaBound{Γ} \AgdaBound{M} \AgdaBound{A} \AgdaBound{N} \AgdaBound{P}\<%
\\
%
\\
\>\AgdaKeyword{postulate} \AgdaPostulate{expand-computeP} \AgdaSymbol{:} \AgdaSymbol{∀} \AgdaSymbol{\{}\AgdaBound{V}\AgdaSymbol{\}} \AgdaSymbol{\{}\AgdaBound{Γ} \AgdaSymbol{:} \AgdaDatatype{Context} \AgdaBound{V}\AgdaSymbol{\}} \AgdaSymbol{\{}\AgdaBound{S}\AgdaSymbol{\}} \AgdaSymbol{\{}\AgdaBound{L} \AgdaSymbol{:} \AgdaDatatype{Leaves} \AgdaBound{V} \AgdaBound{S}\AgdaSymbol{\}} \AgdaSymbol{\{}\AgdaBound{δ} \AgdaBound{ε}\AgdaSymbol{\}} \AgdaSymbol{→}\<%
\\
\>[6]\AgdaIndent{26}{}\<[26]%
\>[26]\AgdaFunction{computeP} \AgdaBound{Γ} \AgdaBound{L} \AgdaBound{ε} \AgdaSymbol{→} \AgdaBound{Γ} \AgdaDatatype{⊢} \AgdaBound{δ} \AgdaDatatype{∶} \AgdaFunction{decode-Prop} \AgdaBound{L} \AgdaSymbol{→} \AgdaDatatype{key-redex} \AgdaBound{δ} \AgdaBound{ε} \AgdaSymbol{→} \AgdaFunction{computeP} \AgdaBound{Γ} \AgdaBound{L} \AgdaBound{δ}\<%
\end{code}
}

\begin{lm}
\label{lm:expand-compute}
Suppose $P[x:=N, y:=N', e:=Q] \in E_\Gamma(M =_A M')$.  Suppose also $\Gamma \vdash (\triplelambda e:x=_By.P)_{N N'} Q : M =_A M'$,
and $N$, $N'$ and $Q$ are all strongly normalizing.  Then $(\triplelambda e:x=_By.P)_{N N'} Q \in E_\Gamma(M =_A M')$.
\end{lm}

\begin{code}%
\>\AgdaFunction{expand-compute} \AgdaSymbol{:} \AgdaSymbol{∀} \AgdaSymbol{\{}\AgdaBound{V}\AgdaSymbol{\}} \AgdaSymbol{\{}\AgdaBound{K}\AgdaSymbol{\}} \AgdaSymbol{\{}\AgdaBound{Γ} \AgdaSymbol{:} \AgdaDatatype{Context} \AgdaBound{V}\AgdaSymbol{\}} \AgdaSymbol{\{}\AgdaBound{A} \AgdaSymbol{:} \AgdaFunction{Expression} \AgdaBound{V} \AgdaSymbol{(}\AgdaFunction{parent} \AgdaBound{K}\AgdaSymbol{)\}} \AgdaSymbol{\{}\AgdaBound{M} \AgdaBound{N} \AgdaSymbol{:} \AgdaFunction{Expression} \AgdaBound{V} \AgdaSymbol{(}\AgdaInductiveConstructor{varKind} \AgdaBound{K}\AgdaSymbol{)\}} \AgdaSymbol{→}\<%
\\
\>[0]\AgdaIndent{2}{}\<[2]%
\>[2]\AgdaFunction{compute} \AgdaBound{Γ} \AgdaBound{A} \AgdaBound{N} \AgdaSymbol{→} \AgdaBound{Γ} \AgdaDatatype{⊢} \AgdaBound{M} \AgdaDatatype{∶} \AgdaBound{A} \AgdaSymbol{→} \AgdaDatatype{key-redex} \AgdaBound{M} \AgdaBound{N} \AgdaSymbol{→} \AgdaFunction{compute} \AgdaBound{Γ} \AgdaBound{A} \AgdaBound{M}\<%
\end{code}

\AgdaHide{
\begin{code}%
\>\AgdaFunction{expand-compute} \AgdaSymbol{\{}\AgdaArgument{K} \AgdaSymbol{=} \AgdaInductiveConstructor{-Term}\AgdaSymbol{\}} \AgdaSymbol{\{}\AgdaArgument{A} \AgdaSymbol{=} \AgdaInductiveConstructor{app} \AgdaSymbol{(}\AgdaInductiveConstructor{-ty} \AgdaBound{A}\AgdaSymbol{)} \AgdaInductiveConstructor{out}\AgdaSymbol{\}} \AgdaSymbol{=} \AgdaFunction{expand-computeT} \AgdaSymbol{\{}\AgdaArgument{A} \AgdaSymbol{=} \AgdaBound{A}\AgdaSymbol{\}}\<%
\\
\>\AgdaFunction{expand-compute} \AgdaSymbol{\{}\AgdaArgument{K} \AgdaSymbol{=} \AgdaInductiveConstructor{-Proof}\AgdaSymbol{\}} \AgdaSymbol{(}\AgdaBound{S} \AgdaInductiveConstructor{,p} \AgdaBound{ψ} \AgdaInductiveConstructor{,p} \AgdaBound{φ↠ψ} \AgdaInductiveConstructor{,p} \AgdaBound{computeε}\AgdaSymbol{)} \AgdaBound{Γ⊢δ∶φ} \AgdaBound{δ▷ε} \AgdaSymbol{=} \AgdaSymbol{(}\AgdaBound{S} \AgdaInductiveConstructor{,p} \AgdaBound{ψ} \AgdaInductiveConstructor{,p} \AgdaBound{φ↠ψ} \AgdaInductiveConstructor{,p} \AgdaPostulate{expand-computeP} \AgdaSymbol{\{}\AgdaArgument{S} \AgdaSymbol{=} \AgdaBound{S}\AgdaSymbol{\}} \AgdaBound{computeε} \AgdaSymbol{(}\AgdaPostulate{Type-Reduction} \AgdaBound{Γ⊢δ∶φ} \AgdaBound{φ↠ψ}\AgdaSymbol{)} \AgdaBound{δ▷ε}\AgdaSymbol{)}\<%
\\
\>\AgdaFunction{expand-compute} \AgdaSymbol{\{}\AgdaArgument{K} \AgdaSymbol{=} \AgdaInductiveConstructor{-Path}\AgdaSymbol{\}} \AgdaSymbol{\{}\AgdaArgument{A} \AgdaSymbol{=} \AgdaInductiveConstructor{app} \AgdaSymbol{(}\AgdaInductiveConstructor{-eq} \AgdaBound{A}\AgdaSymbol{)} \AgdaSymbol{(}\AgdaBound{M} \AgdaInductiveConstructor{,,} \AgdaBound{N} \AgdaInductiveConstructor{,,} \AgdaInductiveConstructor{out}\AgdaSymbol{)\}} \AgdaBound{computeQ} \AgdaBound{Γ⊢P∶M≡N} \AgdaBound{P▷Q} \AgdaSymbol{=} \AgdaPostulate{expand-computeE} \AgdaBound{computeQ} \AgdaBound{Γ⊢P∶M≡N} \AgdaBound{P▷Q}\<%
\\
%
\\
\>\AgdaKeyword{record} \AgdaRecord{E'} \AgdaSymbol{\{}\AgdaBound{V}\AgdaSymbol{\}} \AgdaSymbol{\{}\AgdaBound{K}\AgdaSymbol{\}} \AgdaSymbol{(}\AgdaBound{Γ} \AgdaSymbol{:} \AgdaDatatype{Context} \AgdaBound{V}\AgdaSymbol{)} \AgdaSymbol{(}\AgdaBound{A} \AgdaSymbol{:} \AgdaFunction{Expression} \AgdaBound{V} \AgdaSymbol{(}\AgdaFunction{parent} \AgdaBound{K}\AgdaSymbol{))} \AgdaSymbol{(}\AgdaBound{E} \AgdaSymbol{:} \AgdaFunction{Expression} \AgdaBound{V} \AgdaSymbol{(}\AgdaInductiveConstructor{varKind} \AgdaBound{K}\AgdaSymbol{))} \AgdaSymbol{:} \AgdaPrimitiveType{Set} \AgdaKeyword{where}\<%
\\
\>[0]\AgdaIndent{2}{}\<[2]%
\>[2]\AgdaKeyword{constructor} \AgdaInductiveConstructor{E'I}\<%
\\
\>[0]\AgdaIndent{2}{}\<[2]%
\>[2]\AgdaKeyword{field}\<%
\\
\>[2]\AgdaIndent{4}{}\<[4]%
\>[4]\AgdaField{typed} \AgdaSymbol{:} \AgdaBound{Γ} \AgdaDatatype{⊢} \AgdaBound{E} \AgdaDatatype{∶} \AgdaBound{A}\<%
\\
\>[2]\AgdaIndent{4}{}\<[4]%
\>[4]\AgdaField{computable} \AgdaSymbol{:} \AgdaFunction{compute} \AgdaBound{Γ} \AgdaBound{A} \AgdaBound{E}\<%
\\
%
\\
\>\AgdaComment{--TODO Inline the following?}\<%
\\
\>\AgdaFunction{E} \AgdaSymbol{:} \AgdaSymbol{∀} \AgdaSymbol{\{}\AgdaBound{V}\AgdaSymbol{\}} \AgdaSymbol{→} \AgdaDatatype{Context} \AgdaBound{V} \AgdaSymbol{→} \AgdaDatatype{Type} \AgdaSymbol{→} \AgdaFunction{Term} \AgdaBound{V} \AgdaSymbol{→} \AgdaPrimitiveType{Set}\<%
\\
\>\AgdaFunction{E} \AgdaBound{Γ} \AgdaBound{A} \AgdaBound{M} \AgdaSymbol{=} \AgdaRecord{E'} \AgdaBound{Γ} \AgdaSymbol{(}\AgdaFunction{ty} \AgdaBound{A}\AgdaSymbol{)} \AgdaBound{M}\<%
\\
%
\\
\>\AgdaFunction{EP} \AgdaSymbol{:} \AgdaSymbol{∀} \AgdaSymbol{\{}\AgdaBound{V}\AgdaSymbol{\}} \AgdaSymbol{→} \AgdaDatatype{Context} \AgdaBound{V} \AgdaSymbol{→} \AgdaFunction{Term} \AgdaBound{V} \AgdaSymbol{→} \AgdaFunction{Proof} \AgdaBound{V} \AgdaSymbol{→} \AgdaPrimitiveType{Set}\<%
\\
\>\AgdaFunction{EP} \AgdaBound{Γ} \AgdaBound{φ} \AgdaBound{δ} \AgdaSymbol{=} \AgdaRecord{E'} \AgdaBound{Γ} \AgdaBound{φ} \AgdaBound{δ}\<%
\\
%
\\
\>\AgdaFunction{EE} \AgdaSymbol{:} \AgdaSymbol{∀} \AgdaSymbol{\{}\AgdaBound{V}\AgdaSymbol{\}} \AgdaSymbol{→} \AgdaDatatype{Context} \AgdaBound{V} \AgdaSymbol{→} \AgdaFunction{Equation} \AgdaBound{V} \AgdaSymbol{→} \AgdaFunction{Path} \AgdaBound{V} \AgdaSymbol{→} \AgdaPrimitiveType{Set}\<%
\\
\>\AgdaFunction{EE} \AgdaBound{Γ} \AgdaBound{E} \AgdaBound{P} \AgdaSymbol{=} \AgdaRecord{E'} \AgdaBound{Γ} \AgdaBound{E} \AgdaBound{P}\<%
\\
%
\\
\>\AgdaFunction{E'-typed} \AgdaSymbol{:} \AgdaSymbol{∀} \AgdaSymbol{\{}\AgdaBound{V}\AgdaSymbol{\}} \AgdaSymbol{\{}\AgdaBound{K}\AgdaSymbol{\}} \AgdaSymbol{\{}\AgdaBound{Γ} \AgdaSymbol{:} \AgdaDatatype{Context} \AgdaBound{V}\AgdaSymbol{\}} \AgdaSymbol{\{}\AgdaBound{A}\AgdaSymbol{\}} \AgdaSymbol{\{}\AgdaBound{M} \AgdaSymbol{:} \AgdaFunction{Expression} \AgdaBound{V} \AgdaSymbol{(}\AgdaInductiveConstructor{varKind} \AgdaBound{K}\AgdaSymbol{)\}} \AgdaSymbol{→} \<[74]%
\>[74]\<%
\\
\>[4]\AgdaIndent{11}{}\<[11]%
\>[11]\AgdaRecord{E'} \AgdaBound{Γ} \AgdaBound{A} \AgdaBound{M} \AgdaSymbol{→} \AgdaBound{Γ} \AgdaDatatype{⊢} \AgdaBound{M} \AgdaDatatype{∶} \AgdaBound{A}\<%
\\
\>\AgdaFunction{E'-typed} \AgdaSymbol{=} \AgdaField{E'.typed}\<%
\\
%
\\
\>\AgdaFunction{expand-E'} \AgdaSymbol{:} \AgdaSymbol{∀} \AgdaSymbol{\{}\AgdaBound{V}\AgdaSymbol{\}} \AgdaSymbol{\{}\AgdaBound{K}\AgdaSymbol{\}} \AgdaSymbol{\{}\AgdaBound{Γ}\AgdaSymbol{\}} \AgdaSymbol{\{}\AgdaBound{A}\AgdaSymbol{\}} \AgdaSymbol{\{}\AgdaBound{M} \AgdaBound{N} \AgdaSymbol{:} \AgdaFunction{Expression} \AgdaBound{V} \AgdaSymbol{(}\AgdaInductiveConstructor{varKind} \AgdaBound{K}\AgdaSymbol{)\}} \AgdaSymbol{→}\<%
\\
\>[11]\AgdaIndent{12}{}\<[12]%
\>[12]\AgdaRecord{E'} \AgdaBound{Γ} \AgdaBound{A} \AgdaBound{N} \AgdaSymbol{→} \AgdaBound{Γ} \AgdaDatatype{⊢} \AgdaBound{M} \AgdaDatatype{∶} \AgdaBound{A} \AgdaSymbol{→} \AgdaDatatype{key-redex} \AgdaBound{M} \AgdaBound{N} \AgdaSymbol{→} \AgdaRecord{E'} \AgdaBound{Γ} \AgdaBound{A} \AgdaBound{M}\<%
\\
\>\AgdaFunction{expand-E'} \AgdaBound{N∈EΓA} \AgdaBound{Γ⊢M∶A} \AgdaBound{M▷N} \AgdaSymbol{=} \AgdaInductiveConstructor{E'I} \AgdaBound{Γ⊢M∶A} \AgdaSymbol{(}\AgdaFunction{expand-compute} \AgdaSymbol{(}\AgdaField{E'.computable} \AgdaBound{N∈EΓA}\AgdaSymbol{)} \AgdaBound{Γ⊢M∶A} \AgdaBound{M▷N}\AgdaSymbol{)}\<%
\\
%
\\
\>\AgdaKeyword{postulate} \AgdaPostulate{expand-E} \AgdaSymbol{:} \AgdaSymbol{∀} \AgdaSymbol{\{}\AgdaBound{V}\AgdaSymbol{\}} \AgdaSymbol{\{}\AgdaBound{Γ} \AgdaSymbol{:} \AgdaDatatype{Context} \AgdaBound{V}\AgdaSymbol{\}} \AgdaSymbol{\{}\AgdaBound{A} \AgdaSymbol{:} \AgdaDatatype{Type}\AgdaSymbol{\}} \AgdaSymbol{\{}\AgdaBound{M} \AgdaBound{N} \AgdaSymbol{:} \AgdaFunction{Term} \AgdaBound{V}\AgdaSymbol{\}} \AgdaSymbol{→}\<%
\\
\>[12]\AgdaIndent{19}{}\<[19]%
\>[19]\AgdaFunction{E} \AgdaBound{Γ} \AgdaBound{A} \AgdaBound{N} \AgdaSymbol{→} \AgdaBound{Γ} \AgdaDatatype{⊢} \AgdaBound{M} \AgdaDatatype{∶} \AgdaFunction{ty} \AgdaBound{A} \AgdaSymbol{→} \AgdaDatatype{key-redex} \AgdaBound{M} \AgdaBound{N} \AgdaSymbol{→} \AgdaFunction{E} \AgdaBound{Γ} \AgdaBound{A} \AgdaBound{M}\<%
\\
%
\\
\>\AgdaKeyword{postulate} \AgdaPostulate{func-E} \AgdaSymbol{:} \AgdaSymbol{∀} \AgdaSymbol{\{}\AgdaBound{U}\AgdaSymbol{\}} \AgdaSymbol{\{}\AgdaBound{Γ} \AgdaSymbol{:} \AgdaDatatype{Context} \AgdaBound{U}\AgdaSymbol{\}} \AgdaSymbol{\{}\AgdaBound{M} \AgdaSymbol{:} \AgdaFunction{Term} \AgdaBound{U}\AgdaSymbol{\}} \AgdaSymbol{\{}\AgdaBound{A}\AgdaSymbol{\}} \AgdaSymbol{\{}\AgdaBound{B}\AgdaSymbol{\}} \AgdaSymbol{→}\<%
\\
\>[12]\AgdaIndent{19}{}\<[19]%
\>[19]\AgdaSymbol{(∀} \AgdaBound{V} \AgdaBound{Δ} \AgdaSymbol{(}\AgdaBound{ρ} \AgdaSymbol{:} \AgdaFunction{Rep} \AgdaBound{U} \AgdaBound{V}\AgdaSymbol{)} \AgdaSymbol{(}\AgdaBound{N} \AgdaSymbol{:} \AgdaFunction{Term} \AgdaBound{V}\AgdaSymbol{)} \AgdaSymbol{→} \AgdaDatatype{valid} \AgdaBound{Δ} \AgdaSymbol{→} \AgdaBound{ρ} \AgdaPostulate{∶} \AgdaBound{Γ} \AgdaPostulate{⇒R} \AgdaBound{Δ} \AgdaSymbol{→} \AgdaFunction{E} \AgdaBound{Δ} \AgdaBound{A} \AgdaBound{N} \AgdaSymbol{→} \AgdaFunction{E} \AgdaBound{Δ} \AgdaBound{B} \AgdaSymbol{(}\AgdaFunction{appT} \AgdaSymbol{(}\AgdaBound{M} \AgdaFunction{〈} \AgdaBound{ρ} \AgdaFunction{〉}\AgdaSymbol{)} \AgdaBound{N}\AgdaSymbol{))} \AgdaSymbol{→}\<%
\\
\>[12]\AgdaIndent{19}{}\<[19]%
\>[19]\AgdaFunction{E} \AgdaBound{Γ} \AgdaSymbol{(}\AgdaBound{A} \AgdaInductiveConstructor{⇛} \AgdaBound{B}\AgdaSymbol{)} \AgdaBound{M}\<%
\end{code}
}

\begin{lm}$ $
\label{lm:conv-compute}
\begin{enumerate}
\item
If $\delta \in E_\Gamma(\phi)$, $\Gamma \vdash \psi : \Omega$ and $\phi \simeq \psi$, then $\delta \in E_\Gamma(\psi)$.
\item
If $P \in E_\Gamma(M =_A N)$, $\Gamma \vdash M' : A$, $\Gamma \vdash N' : A$, $M \simeq M'$ and $N \simeq N'$,
then $P \in E_\Gamma(M' =_A N')$.
\end{enumerate}
\end{lm}

\begin{code}%
\>\AgdaKeyword{postulate} \AgdaPostulate{conv-E'} \AgdaSymbol{:} \AgdaSymbol{∀} \AgdaSymbol{\{}\AgdaBound{V}\AgdaSymbol{\}} \AgdaSymbol{\{}\AgdaBound{K}\AgdaSymbol{\}} \AgdaSymbol{\{}\AgdaBound{Γ}\AgdaSymbol{\}} \AgdaSymbol{\{}\AgdaBound{A}\AgdaSymbol{\}} \AgdaSymbol{\{}\AgdaBound{B}\AgdaSymbol{\}} \AgdaSymbol{\{}\AgdaBound{M} \AgdaSymbol{:} \AgdaFunction{Expression} \AgdaBound{V} \AgdaSymbol{(}\AgdaInductiveConstructor{varKind} \AgdaBound{K}\AgdaSymbol{)\}} \AgdaSymbol{→}\<%
\\
\>[0]\AgdaIndent{18}{}\<[18]%
\>[18]\AgdaBound{A} \AgdaDatatype{≃} \AgdaBound{B} \AgdaSymbol{→} \AgdaRecord{E'} \AgdaBound{Γ} \AgdaBound{A} \AgdaBound{M} \AgdaSymbol{→} \AgdaDatatype{valid} \AgdaSymbol{(}\AgdaInductiveConstructor{\_,\_} \AgdaSymbol{\{}\AgdaArgument{K} \AgdaSymbol{=} \AgdaBound{K}\AgdaSymbol{\}} \AgdaBound{Γ} \AgdaBound{B}\AgdaSymbol{)} \AgdaSymbol{→} \AgdaRecord{E'} \AgdaBound{Γ} \AgdaBound{B} \AgdaBound{M}\<%
\end{code}

\AgdaHide{
\begin{code}%
\>\AgdaKeyword{postulate} \AgdaPostulate{E'-SN} \AgdaSymbol{:} \AgdaSymbol{∀} \AgdaSymbol{\{}\AgdaBound{V}\AgdaSymbol{\}} \AgdaSymbol{\{}\AgdaBound{K}\AgdaSymbol{\}} \AgdaSymbol{\{}\AgdaBound{Γ}\AgdaSymbol{\}} \AgdaSymbol{\{}\AgdaBound{A}\AgdaSymbol{\}} \AgdaSymbol{\{}\AgdaBound{M} \AgdaSymbol{:} \AgdaFunction{Expression} \AgdaBound{V} \AgdaSymbol{(}\AgdaInductiveConstructor{varKind} \AgdaBound{K}\AgdaSymbol{)\}} \AgdaSymbol{→} \AgdaRecord{E'} \AgdaBound{Γ} \AgdaBound{A} \AgdaBound{M} \AgdaSymbol{→} \AgdaDatatype{SN} \AgdaBound{M}\<%
\end{code}
}

\begin{lm}
\label{lm:var-compute}
Variables are computable.  That is, if $x : A \in \Gamma$ and $\Gamma \vald$, then $x \in E_\Gamma(A)$; and similarly
for proof variables and path variables.
\end{lm}

\begin{code}%
\>\AgdaKeyword{postulate} \AgdaPostulate{var-E'} \AgdaSymbol{:} \AgdaSymbol{∀} \AgdaSymbol{\{}\AgdaBound{V}\AgdaSymbol{\}} \AgdaSymbol{\{}\AgdaBound{K}\AgdaSymbol{\}} \AgdaSymbol{\{}\AgdaBound{x} \AgdaSymbol{:} \AgdaDatatype{Var} \AgdaBound{V} \AgdaBound{K}\AgdaSymbol{\}} \AgdaSymbol{\{}\AgdaBound{Γ} \AgdaSymbol{:} \AgdaDatatype{Context} \AgdaBound{V}\AgdaSymbol{\}} \AgdaSymbol{→} \AgdaRecord{E'} \AgdaBound{Γ} \AgdaSymbol{(}\AgdaFunction{typeof} \AgdaBound{x} \AgdaBound{Γ}\AgdaSymbol{)} \AgdaSymbol{(}\AgdaInductiveConstructor{var} \AgdaBound{x}\AgdaSymbol{)}\<%
\end{code}

\AgdaHide{
\begin{code}%
\>\AgdaKeyword{postulate} \AgdaPostulate{⊥-E} \AgdaSymbol{:} \AgdaSymbol{∀} \AgdaSymbol{\{}\AgdaBound{V}\AgdaSymbol{\}} \AgdaSymbol{\{}\AgdaBound{Γ} \AgdaSymbol{:} \AgdaDatatype{Context} \AgdaBound{V}\AgdaSymbol{\}} \AgdaSymbol{→} \AgdaDatatype{valid} \AgdaBound{Γ} \AgdaSymbol{→} \AgdaRecord{E'} \AgdaBound{Γ} \AgdaSymbol{(}\AgdaFunction{ty} \AgdaInductiveConstructor{Ω}\AgdaSymbol{)} \AgdaFunction{⊥}\<%
\\
%
\\
\>\AgdaKeyword{postulate} \AgdaPostulate{⊃-E} \AgdaSymbol{:} \AgdaSymbol{∀} \AgdaSymbol{\{}\AgdaBound{V}\AgdaSymbol{\}} \AgdaSymbol{\{}\AgdaBound{Γ} \AgdaSymbol{:} \AgdaDatatype{Context} \AgdaBound{V}\AgdaSymbol{\}} \AgdaSymbol{\{}\AgdaBound{φ}\AgdaSymbol{\}} \AgdaSymbol{\{}\AgdaBound{ψ}\AgdaSymbol{\}} \AgdaSymbol{→} \AgdaFunction{E} \AgdaBound{Γ} \AgdaInductiveConstructor{Ω} \AgdaBound{φ} \AgdaSymbol{→} \AgdaFunction{E} \AgdaBound{Γ} \AgdaInductiveConstructor{Ω} \AgdaBound{ψ} \AgdaSymbol{→} \AgdaFunction{E} \AgdaBound{Γ} \AgdaInductiveConstructor{Ω} \AgdaSymbol{(}\AgdaBound{φ} \AgdaFunction{⊃} \AgdaBound{ψ}\AgdaSymbol{)}\<%
\\
%
\\
\>\AgdaKeyword{postulate} \AgdaPostulate{appT-E} \AgdaSymbol{:} \AgdaSymbol{∀} \AgdaSymbol{\{}\AgdaBound{V}\AgdaSymbol{\}} \AgdaSymbol{\{}\AgdaBound{Γ} \AgdaSymbol{:} \AgdaDatatype{Context} \AgdaBound{V}\AgdaSymbol{\}} \AgdaSymbol{\{}\AgdaBound{M} \AgdaBound{N} \AgdaSymbol{:} \AgdaFunction{Term} \AgdaBound{V}\AgdaSymbol{\}} \AgdaSymbol{\{}\AgdaBound{A}\AgdaSymbol{\}} \AgdaSymbol{\{}\AgdaBound{B}\AgdaSymbol{\}} \AgdaSymbol{→}\<%
\\
\>[0]\AgdaIndent{17}{}\<[17]%
\>[17]\AgdaDatatype{valid} \AgdaBound{Γ} \AgdaSymbol{→} \AgdaFunction{E} \AgdaBound{Γ} \AgdaSymbol{(}\AgdaBound{A} \AgdaInductiveConstructor{⇛} \AgdaBound{B}\AgdaSymbol{)} \AgdaBound{M} \AgdaSymbol{→} \AgdaFunction{E} \AgdaBound{Γ} \AgdaBound{A} \AgdaBound{N} \AgdaSymbol{→} \AgdaFunction{E} \AgdaBound{Γ} \AgdaBound{B} \AgdaSymbol{(}\AgdaFunction{appT} \AgdaBound{M} \AgdaBound{N}\AgdaSymbol{)}\<%
\\
%
\\
\>\AgdaFunction{E-typed} \AgdaSymbol{:} \AgdaSymbol{∀} \AgdaSymbol{\{}\AgdaBound{V}\AgdaSymbol{\}} \AgdaSymbol{\{}\AgdaBound{Γ} \AgdaSymbol{:} \AgdaDatatype{Context} \AgdaBound{V}\AgdaSymbol{\}} \AgdaSymbol{\{}\AgdaBound{A}\AgdaSymbol{\}} \AgdaSymbol{\{}\AgdaBound{M}\AgdaSymbol{\}} \AgdaSymbol{→} \AgdaFunction{E} \AgdaBound{Γ} \AgdaBound{A} \AgdaBound{M} \AgdaSymbol{→} \AgdaBound{Γ} \AgdaDatatype{⊢} \AgdaBound{M} \AgdaDatatype{∶} \AgdaFunction{ty} \AgdaBound{A}\<%
\\
\>\AgdaFunction{E-typed} \AgdaSymbol{=} \AgdaFunction{E'-typed}\<%
\\
%
\\
\>\AgdaFunction{E-SN} \AgdaSymbol{:} \AgdaSymbol{∀} \AgdaSymbol{\{}\AgdaBound{V}\AgdaSymbol{\}} \AgdaSymbol{\{}\AgdaBound{Γ} \AgdaSymbol{:} \AgdaDatatype{Context} \AgdaBound{V}\AgdaSymbol{\}} \AgdaBound{A} \AgdaSymbol{\{}\AgdaBound{M}\AgdaSymbol{\}} \AgdaSymbol{→} \AgdaFunction{E} \AgdaBound{Γ} \AgdaBound{A} \AgdaBound{M} \AgdaSymbol{→} \AgdaDatatype{SN} \AgdaBound{M}\<%
\\
\>\AgdaFunction{E-SN} \AgdaSymbol{\_} \AgdaSymbol{=} \AgdaPostulate{E'-SN}\<%
\\
\>\AgdaComment{--TODO Inline}\<%
\\
%
\\
\>\AgdaKeyword{postulate} \AgdaPostulate{appP-EP} \AgdaSymbol{:} \AgdaSymbol{∀} \AgdaSymbol{\{}\AgdaBound{V}\AgdaSymbol{\}} \AgdaSymbol{\{}\AgdaBound{Γ} \AgdaSymbol{:} \AgdaDatatype{Context} \AgdaBound{V}\AgdaSymbol{\}} \AgdaSymbol{\{}\AgdaBound{δ} \AgdaBound{ε} \AgdaSymbol{:} \AgdaFunction{Proof} \AgdaBound{V}\AgdaSymbol{\}} \AgdaSymbol{\{}\AgdaBound{φ}\AgdaSymbol{\}} \AgdaSymbol{\{}\AgdaBound{ψ}\AgdaSymbol{\}} \AgdaSymbol{→}\<%
\\
\>[17]\AgdaIndent{18}{}\<[18]%
\>[18]\AgdaFunction{EP} \AgdaBound{Γ} \AgdaSymbol{(}\AgdaBound{φ} \AgdaFunction{⊃} \AgdaBound{ψ}\AgdaSymbol{)} \AgdaBound{δ} \AgdaSymbol{→} \AgdaFunction{EP} \AgdaBound{Γ} \AgdaBound{φ} \AgdaBound{ε} \AgdaSymbol{→} \AgdaFunction{EP} \AgdaBound{Γ} \AgdaBound{ψ} \AgdaSymbol{(}\AgdaFunction{appP} \AgdaBound{δ} \AgdaBound{ε}\AgdaSymbol{)}\<%
\\
\>\AgdaKeyword{postulate} \AgdaPostulate{plus-EP} \AgdaSymbol{:} \AgdaSymbol{∀} \AgdaSymbol{\{}\AgdaBound{V}\AgdaSymbol{\}} \AgdaSymbol{\{}\AgdaBound{Γ} \AgdaSymbol{:} \AgdaDatatype{Context} \AgdaBound{V}\AgdaSymbol{\}} \AgdaSymbol{\{}\AgdaBound{P} \AgdaSymbol{:} \AgdaFunction{Path} \AgdaBound{V}\AgdaSymbol{\}} \AgdaSymbol{\{}\AgdaBound{φ} \AgdaBound{ψ} \AgdaSymbol{:} \AgdaFunction{Term} \AgdaBound{V}\AgdaSymbol{\}} \AgdaSymbol{→}\<%
\\
\>[17]\AgdaIndent{18}{}\<[18]%
\>[18]\AgdaFunction{EE} \AgdaBound{Γ} \AgdaSymbol{(}\AgdaBound{φ} \AgdaFunction{≡〈} \AgdaInductiveConstructor{Ω} \AgdaFunction{〉} \AgdaBound{ψ}\AgdaSymbol{)} \AgdaBound{P} \AgdaSymbol{→} \AgdaFunction{EP} \AgdaBound{Γ} \AgdaSymbol{(}\AgdaBound{φ} \AgdaFunction{⊃} \AgdaBound{ψ}\AgdaSymbol{)} \AgdaSymbol{(}\AgdaFunction{plus} \AgdaBound{P}\AgdaSymbol{)}\<%
\\
\>\AgdaKeyword{postulate} \AgdaPostulate{minus-EP} \AgdaSymbol{:} \AgdaSymbol{∀} \AgdaSymbol{\{}\AgdaBound{V}\AgdaSymbol{\}} \AgdaSymbol{\{}\AgdaBound{Γ} \AgdaSymbol{:} \AgdaDatatype{Context} \AgdaBound{V}\AgdaSymbol{\}} \AgdaSymbol{\{}\AgdaBound{P} \AgdaSymbol{:} \AgdaFunction{Path} \AgdaBound{V}\AgdaSymbol{\}} \AgdaSymbol{\{}\AgdaBound{φ} \AgdaBound{ψ} \AgdaSymbol{:} \AgdaFunction{Term} \AgdaBound{V}\AgdaSymbol{\}} \AgdaSymbol{→}\<%
\\
\>[18]\AgdaIndent{19}{}\<[19]%
\>[19]\AgdaFunction{EE} \AgdaBound{Γ} \AgdaSymbol{(}\AgdaBound{φ} \AgdaFunction{≡〈} \AgdaInductiveConstructor{Ω} \AgdaFunction{〉} \AgdaBound{ψ}\AgdaSymbol{)} \AgdaBound{P} \AgdaSymbol{→} \AgdaFunction{EP} \AgdaBound{Γ} \AgdaSymbol{(}\AgdaBound{ψ} \AgdaFunction{⊃} \AgdaBound{φ}\AgdaSymbol{)} \AgdaSymbol{(}\AgdaFunction{minus} \AgdaBound{P}\AgdaSymbol{)}\<%
\\
%
\\
\>\AgdaFunction{expand-EP} \AgdaSymbol{:} \AgdaSymbol{∀} \AgdaSymbol{\{}\AgdaBound{V}\AgdaSymbol{\}} \AgdaSymbol{\{}\AgdaBound{Γ} \AgdaSymbol{:} \AgdaDatatype{Context} \AgdaBound{V}\AgdaSymbol{\}} \AgdaSymbol{\{}\AgdaBound{φ} \AgdaSymbol{:} \AgdaFunction{Term} \AgdaBound{V}\AgdaSymbol{\}} \AgdaSymbol{\{}\AgdaBound{δ} \AgdaBound{ε} \AgdaSymbol{:} \AgdaFunction{Proof} \AgdaBound{V}\AgdaSymbol{\}} \AgdaSymbol{→}\<%
\\
\>[0]\AgdaIndent{12}{}\<[12]%
\>[12]\AgdaFunction{EP} \AgdaBound{Γ} \AgdaBound{φ} \AgdaBound{ε} \AgdaSymbol{→} \AgdaBound{Γ} \AgdaDatatype{⊢} \AgdaBound{δ} \AgdaDatatype{∶} \AgdaBound{φ} \AgdaSymbol{→} \AgdaDatatype{key-redex} \AgdaBound{δ} \AgdaBound{ε} \AgdaSymbol{→} \AgdaFunction{EP} \AgdaBound{Γ} \AgdaBound{φ} \AgdaBound{δ}\<%
\\
\>\AgdaFunction{expand-EP} \AgdaSymbol{=} \AgdaFunction{expand-E'}\<%
\\
%
\\
\>\AgdaKeyword{postulate} \AgdaPostulate{func-EP} \AgdaSymbol{:} \AgdaSymbol{∀} \AgdaSymbol{\{}\AgdaBound{U}\AgdaSymbol{\}} \AgdaSymbol{\{}\AgdaBound{Γ} \AgdaSymbol{:} \AgdaDatatype{Context} \AgdaBound{U}\AgdaSymbol{\}} \AgdaSymbol{\{}\AgdaBound{δ} \AgdaSymbol{:} \AgdaFunction{Proof} \AgdaBound{U}\AgdaSymbol{\}} \AgdaSymbol{\{}\AgdaBound{φ}\AgdaSymbol{\}} \AgdaSymbol{\{}\AgdaBound{ψ}\AgdaSymbol{\}} \AgdaSymbol{→}\<%
\\
\>[12]\AgdaIndent{18}{}\<[18]%
\>[18]\AgdaSymbol{(∀} \AgdaBound{V} \AgdaBound{Δ} \AgdaSymbol{(}\AgdaBound{ρ} \AgdaSymbol{:} \AgdaFunction{Rep} \AgdaBound{U} \AgdaBound{V}\AgdaSymbol{)} \AgdaSymbol{(}\AgdaBound{ε} \AgdaSymbol{:} \AgdaFunction{Proof} \AgdaBound{V}\AgdaSymbol{)} \AgdaSymbol{→} \AgdaBound{ρ} \AgdaPostulate{∶} \AgdaBound{Γ} \AgdaPostulate{⇒R} \AgdaBound{Δ} \AgdaSymbol{→} \AgdaFunction{EP} \AgdaBound{Δ} \AgdaSymbol{(}\AgdaBound{φ} \AgdaFunction{〈} \AgdaBound{ρ} \AgdaFunction{〉}\AgdaSymbol{)} \AgdaBound{ε} \AgdaSymbol{→} \AgdaFunction{EP} \AgdaBound{Δ} \AgdaSymbol{(}\AgdaBound{ψ} \AgdaFunction{〈} \AgdaBound{ρ} \AgdaFunction{〉}\AgdaSymbol{)} \AgdaSymbol{(}\AgdaFunction{appP} \AgdaSymbol{(}\AgdaBound{δ} \AgdaFunction{〈} \AgdaBound{ρ} \AgdaFunction{〉}\AgdaSymbol{)} \AgdaBound{ε}\AgdaSymbol{))} \AgdaSymbol{→} \<[124]%
\>[124]\<%
\\
\>[12]\AgdaIndent{18}{}\<[18]%
\>[18]\AgdaBound{Γ} \AgdaDatatype{⊢} \AgdaBound{δ} \AgdaDatatype{∶} \AgdaBound{φ} \AgdaFunction{⊃} \AgdaBound{ψ} \AgdaSymbol{→} \AgdaFunction{EP} \AgdaBound{Γ} \AgdaSymbol{(}\AgdaBound{φ} \AgdaFunction{⊃} \AgdaBound{ψ}\AgdaSymbol{)} \AgdaBound{δ}\<%
\\
%
\\
\>\AgdaFunction{conv-EP} \AgdaSymbol{:} \AgdaSymbol{∀} \AgdaSymbol{\{}\AgdaBound{V}\AgdaSymbol{\}} \AgdaSymbol{\{}\AgdaBound{Γ} \AgdaSymbol{:} \AgdaDatatype{Context} \AgdaBound{V}\AgdaSymbol{\}} \AgdaSymbol{\{}\AgdaBound{φ} \AgdaBound{ψ} \AgdaSymbol{:} \AgdaFunction{Term} \AgdaBound{V}\AgdaSymbol{\}} \AgdaSymbol{\{}\AgdaBound{δ} \AgdaSymbol{:} \AgdaFunction{Proof} \AgdaBound{V}\AgdaSymbol{\}} \AgdaSymbol{→}\<%
\\
\>[0]\AgdaIndent{10}{}\<[10]%
\>[10]\AgdaBound{φ} \AgdaDatatype{≃} \AgdaBound{ψ} \AgdaSymbol{→} \AgdaFunction{EP} \AgdaBound{Γ} \AgdaBound{φ} \AgdaBound{δ} \AgdaSymbol{→} \AgdaBound{Γ} \AgdaDatatype{⊢} \AgdaBound{ψ} \AgdaDatatype{∶} \AgdaFunction{ty} \AgdaInductiveConstructor{Ω} \AgdaSymbol{→} \AgdaFunction{EP} \AgdaBound{Γ} \AgdaBound{ψ} \AgdaBound{δ}\<%
\\
\>\AgdaFunction{conv-EP} \AgdaBound{φ≃ψ} \AgdaBound{δ∈EΓφ} \AgdaBound{Γ⊢ψ∶Ω} \AgdaSymbol{=} \AgdaPostulate{conv-E'} \AgdaBound{φ≃ψ} \AgdaBound{δ∈EΓφ} \AgdaSymbol{(}\AgdaInductiveConstructor{ctxPR} \AgdaBound{Γ⊢ψ∶Ω}\AgdaSymbol{)}\<%
\\
%
\\
\>\AgdaFunction{EP-typed} \AgdaSymbol{:} \AgdaSymbol{∀} \AgdaSymbol{\{}\AgdaBound{V}\AgdaSymbol{\}} \AgdaSymbol{\{}\AgdaBound{Γ} \AgdaSymbol{:} \AgdaDatatype{Context} \AgdaBound{V}\AgdaSymbol{\}} \AgdaSymbol{\{}\AgdaBound{δ} \AgdaSymbol{:} \AgdaFunction{Proof} \AgdaBound{V}\AgdaSymbol{\}} \AgdaSymbol{\{}\AgdaBound{φ} \AgdaSymbol{:} \AgdaFunction{Term} \AgdaBound{V}\AgdaSymbol{\}} \AgdaSymbol{→}\<%
\\
\>[0]\AgdaIndent{9}{}\<[9]%
\>[9]\AgdaFunction{EP} \AgdaBound{Γ} \AgdaBound{φ} \AgdaBound{δ} \AgdaSymbol{→} \AgdaBound{Γ} \AgdaDatatype{⊢} \AgdaBound{δ} \AgdaDatatype{∶} \AgdaBound{φ}\<%
\\
\>\AgdaFunction{EP-typed} \AgdaSymbol{=} \AgdaFunction{E'-typed}\<%
\\
%
\\
\>\AgdaFunction{EP-SN} \AgdaSymbol{:} \AgdaSymbol{∀} \AgdaSymbol{\{}\AgdaBound{V}\AgdaSymbol{\}} \AgdaSymbol{\{}\AgdaBound{Γ} \AgdaSymbol{:} \AgdaDatatype{Context} \AgdaBound{V}\AgdaSymbol{\}} \AgdaSymbol{\{}\AgdaBound{δ}\AgdaSymbol{\}} \AgdaSymbol{\{}\AgdaBound{φ}\AgdaSymbol{\}} \AgdaSymbol{→} \AgdaFunction{EP} \AgdaBound{Γ} \AgdaBound{φ} \AgdaBound{δ} \AgdaSymbol{→} \AgdaDatatype{SN} \AgdaBound{δ}\<%
\\
\>\AgdaFunction{EP-SN} \AgdaSymbol{=} \AgdaPostulate{E'-SN}\<%
\\
%
\\
\>\AgdaKeyword{postulate} \AgdaPostulate{ref-EE} \AgdaSymbol{:} \AgdaSymbol{∀} \AgdaSymbol{\{}\AgdaBound{V}\AgdaSymbol{\}} \AgdaSymbol{\{}\AgdaBound{Γ} \AgdaSymbol{:} \AgdaDatatype{Context} \AgdaBound{V}\AgdaSymbol{\}} \AgdaSymbol{\{}\AgdaBound{M} \AgdaSymbol{:} \AgdaFunction{Term} \AgdaBound{V}\AgdaSymbol{\}} \AgdaSymbol{\{}\AgdaBound{A} \AgdaSymbol{:} \AgdaDatatype{Type}\AgdaSymbol{\}} \AgdaSymbol{→} \AgdaFunction{E} \AgdaBound{Γ} \AgdaBound{A} \AgdaBound{M} \AgdaSymbol{→} \AgdaFunction{EE} \AgdaBound{Γ} \AgdaSymbol{(}\AgdaBound{M} \AgdaFunction{≡〈} \AgdaBound{A} \AgdaFunction{〉} \AgdaBound{M}\AgdaSymbol{)} \AgdaSymbol{(}\AgdaFunction{reff} \AgdaBound{M}\AgdaSymbol{)}\<%
\\
\>\AgdaKeyword{postulate} \AgdaPostulate{⊃*-EE} \AgdaSymbol{:} \AgdaSymbol{∀} \AgdaSymbol{\{}\AgdaBound{V}\AgdaSymbol{\}} \AgdaSymbol{\{}\AgdaBound{Γ} \AgdaSymbol{:} \AgdaDatatype{Context} \AgdaBound{V}\AgdaSymbol{\}} \AgdaSymbol{\{}\AgdaBound{φ} \AgdaBound{φ'} \AgdaBound{ψ} \AgdaBound{ψ'} \AgdaSymbol{:} \AgdaFunction{Term} \AgdaBound{V}\AgdaSymbol{\}} \AgdaSymbol{\{}\AgdaBound{P} \AgdaBound{Q} \AgdaSymbol{:} \AgdaFunction{Path} \AgdaBound{V}\AgdaSymbol{\}} \AgdaSymbol{→}\<%
\\
\>[9]\AgdaIndent{18}{}\<[18]%
\>[18]\AgdaFunction{EE} \AgdaBound{Γ} \AgdaSymbol{(}\AgdaBound{φ} \AgdaFunction{≡〈} \AgdaInductiveConstructor{Ω} \AgdaFunction{〉} \AgdaBound{φ'}\AgdaSymbol{)} \AgdaBound{P} \AgdaSymbol{→} \AgdaFunction{EE} \AgdaBound{Γ} \AgdaSymbol{(}\AgdaBound{ψ} \AgdaFunction{≡〈} \AgdaInductiveConstructor{Ω} \AgdaFunction{〉} \AgdaBound{ψ'}\AgdaSymbol{)} \AgdaBound{Q} \AgdaSymbol{→} \AgdaFunction{EE} \AgdaBound{Γ} \AgdaSymbol{(}\AgdaBound{φ} \AgdaFunction{⊃} \AgdaBound{ψ} \AgdaFunction{≡〈} \AgdaInductiveConstructor{Ω} \AgdaFunction{〉} \AgdaBound{φ'} \AgdaFunction{⊃} \AgdaBound{ψ'}\AgdaSymbol{)} \AgdaSymbol{(}\AgdaBound{P} \AgdaFunction{⊃*} \AgdaBound{Q}\AgdaSymbol{)}\<%
\\
\>\AgdaKeyword{postulate} \AgdaPostulate{univ-EE} \AgdaSymbol{:} \AgdaSymbol{∀} \AgdaSymbol{\{}\AgdaBound{V}\AgdaSymbol{\}} \AgdaSymbol{\{}\AgdaBound{Γ} \AgdaSymbol{:} \AgdaDatatype{Context} \AgdaBound{V}\AgdaSymbol{\}} \AgdaSymbol{\{}\AgdaBound{φ} \AgdaBound{ψ} \AgdaSymbol{:} \AgdaFunction{Term} \AgdaBound{V}\AgdaSymbol{\}} \AgdaSymbol{\{}\AgdaBound{δ} \AgdaBound{ε} \AgdaSymbol{:} \AgdaFunction{Proof} \AgdaBound{V}\AgdaSymbol{\}} \AgdaSymbol{→}\<%
\\
\>[9]\AgdaIndent{18}{}\<[18]%
\>[18]\AgdaFunction{EP} \AgdaBound{Γ} \AgdaSymbol{(}\AgdaBound{φ} \AgdaFunction{⊃} \AgdaBound{ψ}\AgdaSymbol{)} \AgdaBound{δ} \AgdaSymbol{→} \AgdaFunction{EP} \AgdaBound{Γ} \AgdaSymbol{(}\AgdaBound{ψ} \AgdaFunction{⊃} \AgdaBound{φ}\AgdaSymbol{)} \AgdaBound{ε} \AgdaSymbol{→} \AgdaFunction{EE} \AgdaBound{Γ} \AgdaSymbol{(}\AgdaBound{φ} \AgdaFunction{≡〈} \AgdaInductiveConstructor{Ω} \AgdaFunction{〉} \AgdaBound{ψ}\AgdaSymbol{)} \AgdaSymbol{(}\AgdaFunction{univ} \AgdaBound{φ} \AgdaBound{ψ} \AgdaBound{δ} \AgdaBound{ε}\AgdaSymbol{)}\<%
\\
\>\AgdaKeyword{postulate} \AgdaPostulate{app*-EE} \AgdaSymbol{:} \AgdaSymbol{∀} \AgdaSymbol{\{}\AgdaBound{V}\AgdaSymbol{\}} \AgdaSymbol{\{}\AgdaBound{Γ} \AgdaSymbol{:} \AgdaDatatype{Context} \AgdaBound{V}\AgdaSymbol{\}} \AgdaSymbol{\{}\AgdaBound{M}\AgdaSymbol{\}} \AgdaSymbol{\{}\AgdaBound{M'}\AgdaSymbol{\}} \AgdaSymbol{\{}\AgdaBound{N}\AgdaSymbol{\}} \AgdaSymbol{\{}\AgdaBound{N'}\AgdaSymbol{\}} \AgdaSymbol{\{}\AgdaBound{A}\AgdaSymbol{\}} \AgdaSymbol{\{}\AgdaBound{B}\AgdaSymbol{\}} \AgdaSymbol{\{}\AgdaBound{P}\AgdaSymbol{\}} \AgdaSymbol{\{}\AgdaBound{Q}\AgdaSymbol{\}} \AgdaSymbol{→}\<%
\\
\>[9]\AgdaIndent{18}{}\<[18]%
\>[18]\AgdaFunction{EE} \AgdaBound{Γ} \AgdaSymbol{(}\AgdaBound{M} \AgdaFunction{≡〈} \AgdaBound{A} \AgdaInductiveConstructor{⇛} \AgdaBound{B} \AgdaFunction{〉} \AgdaBound{M'}\AgdaSymbol{)} \AgdaBound{P} \AgdaSymbol{→} \AgdaFunction{EE} \AgdaBound{Γ} \AgdaSymbol{(}\AgdaBound{N} \AgdaFunction{≡〈} \AgdaBound{A} \AgdaFunction{〉} \AgdaBound{N'}\AgdaSymbol{)} \AgdaBound{Q} \AgdaSymbol{→}\<%
\\
\>[9]\AgdaIndent{18}{}\<[18]%
\>[18]\AgdaFunction{E} \AgdaBound{Γ} \AgdaBound{A} \AgdaBound{N} \AgdaSymbol{→} \AgdaFunction{E} \AgdaBound{Γ} \AgdaBound{A} \AgdaBound{N'} \AgdaSymbol{→}\<%
\\
\>[9]\AgdaIndent{18}{}\<[18]%
\>[18]\AgdaFunction{EE} \AgdaBound{Γ} \AgdaSymbol{(}\AgdaFunction{appT} \AgdaBound{M} \AgdaBound{N} \AgdaFunction{≡〈} \AgdaBound{B} \AgdaFunction{〉} \AgdaFunction{appT} \AgdaBound{M'} \AgdaBound{N'}\AgdaSymbol{)} \AgdaSymbol{(}\AgdaFunction{app*} \AgdaBound{N} \AgdaBound{N'} \AgdaBound{P} \AgdaBound{Q}\AgdaSymbol{)}\<%
\\
%
\\
\>\AgdaKeyword{postulate} \AgdaPostulate{expand-EE} \AgdaSymbol{:} \AgdaSymbol{∀} \AgdaSymbol{\{}\AgdaBound{V}\AgdaSymbol{\}} \AgdaSymbol{\{}\AgdaBound{Γ} \AgdaSymbol{:} \AgdaDatatype{Context} \AgdaBound{V}\AgdaSymbol{\}} \AgdaSymbol{\{}\AgdaBound{A}\AgdaSymbol{\}} \AgdaSymbol{\{}\AgdaBound{M} \AgdaBound{N} \AgdaSymbol{:} \AgdaFunction{Term} \AgdaBound{V}\AgdaSymbol{\}} \AgdaSymbol{\{}\AgdaBound{P} \AgdaBound{Q}\AgdaSymbol{\}} \AgdaSymbol{→}\<%
\\
\>[18]\AgdaIndent{20}{}\<[20]%
\>[20]\AgdaFunction{EE} \AgdaBound{Γ} \AgdaSymbol{(}\AgdaBound{M} \AgdaFunction{≡〈} \AgdaBound{A} \AgdaFunction{〉} \AgdaBound{N}\AgdaSymbol{)} \AgdaBound{Q} \AgdaSymbol{→} \AgdaBound{Γ} \AgdaDatatype{⊢} \AgdaBound{P} \AgdaDatatype{∶} \AgdaBound{M} \AgdaFunction{≡〈} \AgdaBound{A} \AgdaFunction{〉} \AgdaBound{N} \AgdaSymbol{→} \AgdaDatatype{key-redex} \AgdaBound{P} \AgdaBound{Q} \AgdaSymbol{→} \AgdaFunction{EE} \AgdaBound{Γ} \AgdaSymbol{(}\AgdaBound{M} \AgdaFunction{≡〈} \AgdaBound{A} \AgdaFunction{〉} \AgdaBound{N}\AgdaSymbol{)} \AgdaBound{P}\<%
\\
\>\AgdaKeyword{postulate} \AgdaPostulate{func-EE} \AgdaSymbol{:} \AgdaSymbol{∀} \AgdaSymbol{\{}\AgdaBound{U}\AgdaSymbol{\}} \AgdaSymbol{\{}\AgdaBound{Γ} \AgdaSymbol{:} \AgdaDatatype{Context} \AgdaBound{U}\AgdaSymbol{\}} \AgdaSymbol{\{}\AgdaBound{A}\AgdaSymbol{\}} \AgdaSymbol{\{}\AgdaBound{B}\AgdaSymbol{\}} \AgdaSymbol{\{}\AgdaBound{M}\AgdaSymbol{\}} \AgdaSymbol{\{}\AgdaBound{M'}\AgdaSymbol{\}} \AgdaSymbol{\{}\AgdaBound{P}\AgdaSymbol{\}} \AgdaSymbol{→}\<%
\\
\>[0]\AgdaIndent{18}{}\<[18]%
\>[18]\AgdaBound{Γ} \AgdaDatatype{⊢} \AgdaBound{P} \AgdaDatatype{∶} \AgdaBound{M} \AgdaFunction{≡〈} \AgdaBound{A} \AgdaInductiveConstructor{⇛} \AgdaBound{B} \AgdaFunction{〉} \AgdaBound{M'} \AgdaSymbol{→}\<%
\\
\>[0]\AgdaIndent{18}{}\<[18]%
\>[18]\AgdaSymbol{(∀} \AgdaBound{V} \AgdaSymbol{(}\AgdaBound{Δ} \AgdaSymbol{:} \AgdaDatatype{Context} \AgdaBound{V}\AgdaSymbol{)} \AgdaSymbol{(}\AgdaBound{N} \AgdaBound{N'} \AgdaSymbol{:} \AgdaFunction{Term} \AgdaBound{V}\AgdaSymbol{)} \AgdaBound{Q} \AgdaBound{ρ} \AgdaSymbol{→} \AgdaBound{ρ} \AgdaPostulate{∶} \AgdaBound{Γ} \AgdaPostulate{⇒R} \AgdaBound{Δ} \AgdaSymbol{→} \<[74]%
\>[74]\<%
\\
\>[0]\AgdaIndent{18}{}\<[18]%
\>[18]\AgdaFunction{E} \AgdaBound{Δ} \AgdaBound{A} \AgdaBound{N} \AgdaSymbol{→} \AgdaFunction{E} \AgdaBound{Δ} \AgdaBound{A} \AgdaBound{N'} \AgdaSymbol{→} \AgdaFunction{EE} \AgdaBound{Δ} \AgdaSymbol{(}\AgdaBound{N} \AgdaFunction{≡〈} \AgdaBound{A} \AgdaFunction{〉} \AgdaBound{N'}\AgdaSymbol{)} \AgdaBound{Q} \AgdaSymbol{→}\<%
\\
\>[0]\AgdaIndent{18}{}\<[18]%
\>[18]\AgdaFunction{EE} \AgdaBound{Δ} \AgdaSymbol{(}\AgdaFunction{appT} \AgdaSymbol{(}\AgdaBound{M} \AgdaFunction{〈} \AgdaBound{ρ} \AgdaFunction{〉}\AgdaSymbol{)} \AgdaBound{N} \AgdaFunction{≡〈} \AgdaBound{B} \AgdaFunction{〉} \AgdaFunction{appT} \AgdaSymbol{(}\AgdaBound{M'} \AgdaFunction{〈} \AgdaBound{ρ} \AgdaFunction{〉}\AgdaSymbol{)} \AgdaBound{N'}\AgdaSymbol{)} \AgdaSymbol{(}\AgdaFunction{app*} \AgdaBound{N} \AgdaBound{N'} \AgdaSymbol{(}\AgdaBound{P} \AgdaFunction{〈} \AgdaBound{ρ} \AgdaFunction{〉}\AgdaSymbol{)} \AgdaBound{Q}\AgdaSymbol{))} \AgdaSymbol{→}\<%
\\
\>[0]\AgdaIndent{18}{}\<[18]%
\>[18]\AgdaFunction{EE} \AgdaBound{Γ} \AgdaSymbol{(}\AgdaBound{M} \AgdaFunction{≡〈} \AgdaBound{A} \AgdaInductiveConstructor{⇛} \AgdaBound{B} \AgdaFunction{〉} \AgdaBound{M'}\AgdaSymbol{)} \AgdaBound{P}\<%
\\
%
\\
\>\AgdaFunction{conv-EE} \AgdaSymbol{:} \AgdaSymbol{∀} \AgdaSymbol{\{}\AgdaBound{V}\AgdaSymbol{\}} \AgdaSymbol{\{}\AgdaBound{Γ} \AgdaSymbol{:} \AgdaDatatype{Context} \AgdaBound{V}\AgdaSymbol{\}} \AgdaSymbol{\{}\AgdaBound{M}\AgdaSymbol{\}} \AgdaSymbol{\{}\AgdaBound{N}\AgdaSymbol{\}} \AgdaSymbol{\{}\AgdaBound{M'}\AgdaSymbol{\}} \AgdaSymbol{\{}\AgdaBound{N'}\AgdaSymbol{\}} \AgdaSymbol{\{}\AgdaBound{A}\AgdaSymbol{\}} \AgdaSymbol{\{}\AgdaBound{P}\AgdaSymbol{\}} \AgdaSymbol{→}\<%
\\
\>[0]\AgdaIndent{10}{}\<[10]%
\>[10]\AgdaFunction{EE} \AgdaBound{Γ} \AgdaSymbol{(}\AgdaBound{M} \AgdaFunction{≡〈} \AgdaBound{A} \AgdaFunction{〉} \AgdaBound{N}\AgdaSymbol{)} \AgdaBound{P} \AgdaSymbol{→} \AgdaBound{M} \AgdaDatatype{≃} \AgdaBound{M'} \AgdaSymbol{→} \AgdaBound{N} \AgdaDatatype{≃} \AgdaBound{N'} \AgdaSymbol{→} \AgdaBound{Γ} \AgdaDatatype{⊢} \AgdaBound{M'} \AgdaDatatype{∶} \AgdaFunction{ty} \AgdaBound{A} \AgdaSymbol{→} \AgdaBound{Γ} \AgdaDatatype{⊢} \AgdaBound{N'} \AgdaDatatype{∶} \AgdaFunction{ty} \AgdaBound{A} \AgdaSymbol{→} \<[82]%
\>[82]\<%
\\
\>[0]\AgdaIndent{10}{}\<[10]%
\>[10]\AgdaFunction{EE} \AgdaBound{Γ} \AgdaSymbol{(}\AgdaBound{M'} \AgdaFunction{≡〈} \AgdaBound{A} \AgdaFunction{〉} \AgdaBound{N'}\AgdaSymbol{)} \AgdaBound{P}\<%
\\
\>\AgdaFunction{conv-EE} \AgdaBound{P∈EΓM≡N} \AgdaBound{M≃M'} \AgdaBound{N≃N'} \AgdaBound{Γ⊢M'∶A} \AgdaBound{Γ⊢N'∶A} \AgdaSymbol{=} \<[42]%
\>[42]\<%
\\
\>[0]\AgdaIndent{2}{}\<[2]%
\>[2]\AgdaPostulate{conv-E'} \AgdaSymbol{(}\AgdaPostulate{eq-resp-conv} \<[25]%
\>[25]\AgdaBound{M≃M'} \AgdaBound{N≃N'}\AgdaSymbol{)} \AgdaBound{P∈EΓM≡N} \AgdaSymbol{(}\AgdaInductiveConstructor{ctxER} \AgdaBound{Γ⊢M'∶A} \AgdaBound{Γ⊢N'∶A}\AgdaSymbol{)}\<%
\\
%
\\
\>\AgdaFunction{EE-typed} \AgdaSymbol{:} \AgdaSymbol{∀} \AgdaSymbol{\{}\AgdaBound{V}\AgdaSymbol{\}} \AgdaSymbol{\{}\AgdaBound{Γ} \AgdaSymbol{:} \AgdaDatatype{Context} \AgdaBound{V}\AgdaSymbol{\}} \AgdaSymbol{\{}\AgdaBound{E}\AgdaSymbol{\}} \AgdaSymbol{\{}\AgdaBound{P}\AgdaSymbol{\}} \AgdaSymbol{→} \AgdaFunction{EE} \AgdaBound{Γ} \AgdaBound{E} \AgdaBound{P} \AgdaSymbol{→} \AgdaBound{Γ} \AgdaDatatype{⊢} \AgdaBound{P} \AgdaDatatype{∶} \AgdaBound{E}\<%
\\
\>\AgdaFunction{EE-typed} \AgdaSymbol{=} \AgdaFunction{E'-typed}\<%
\\
%
\\
\>\AgdaFunction{EE-SN} \AgdaSymbol{:} \AgdaSymbol{∀} \AgdaSymbol{\{}\AgdaBound{V}\AgdaSymbol{\}} \AgdaSymbol{\{}\AgdaBound{Γ} \AgdaSymbol{:} \AgdaDatatype{Context} \AgdaBound{V}\AgdaSymbol{\}} \AgdaBound{E} \AgdaSymbol{\{}\AgdaBound{P}\AgdaSymbol{\}} \AgdaSymbol{→} \AgdaFunction{EE} \AgdaBound{Γ} \AgdaBound{E} \AgdaBound{P} \AgdaSymbol{→} \AgdaDatatype{SN} \AgdaBound{P}\<%
\\
\>\AgdaFunction{EE-SN} \AgdaSymbol{\_} \AgdaSymbol{=} \AgdaPostulate{E'-SN}\<%
\\
%
\\
\>\AgdaComment{\{-\<\\
\>postulate Neutral-computeE : ∀ \{V\} \{Γ : Context V\} \{M\} \{A\} \{N\} \{P : NeutralP V\} →\<\\
\>                           Γ ⊢ decode P ∶ M ≡〈 A 〉 N → computeE Γ M A N (decode P)\<\\
\>\<\\
\>postulate compute-SN : ∀ \{V\} \{Γ : Context V\} \{A\} \{δ\} → compute Γ A δ → valid Γ → SN δ\<\\
\>\<\\
\>postulate NF : ∀ \{V\} \{Γ\} \{φ : Term V\} → Γ ⊢ φ ∶ ty Ω → closed-prop\<\\
\>\<\\
\>postulate red-NF : ∀ \{V\} \{Γ\} \{φ : Term V\} (Γ⊢φ∶Ω : Γ ⊢ φ ∶ ty Ω) → φ ↠ cp2term (NF Γ⊢φ∶Ω)\<\\
\>\<\\
\>postulate closed-rep : ∀ \{U\} \{V\} \{ρ : Rep U V\} (A : closed-prop) → (cp2term A) 〈 ρ 〉 ≡ cp2term A\<\\
\>\<\\
\>postulate red-conv : ∀ \{V\} \{C\} \{K\} \{E F : Subexpression V C K\} → E ↠ F → E ≃ F\<\\
\>\<\\
\>postulate confluent : ∀ \{V\} \{φ : Term V\} \{ψ ψ' : closed-prop\} → φ ↠ cp2term ψ → φ ↠ cp2term ψ' → ψ ≡ ψ'\<\\
\>\<\\
\>func-EP \{δ = δ\} \{φ = φ\} \{ψ = ψ\} hyp Γ⊢δ∶φ⊃ψ = let Γ⊢φ⊃ψ∶Ω = Prop-Validity Γ⊢δ∶φ⊃ψ in\<\\
\>                      let Γ⊢φ∶Ω = ⊃-gen₁ Γ⊢φ⊃ψ∶Ω in\<\\
\>                      let Γ⊢ψ∶Ω = ⊃-gen₂ Γ⊢φ⊃ψ∶Ω in\<\\
\>                      let φ' = NF Γ⊢φ∶Ω in\<\\
\>                      Γ⊢δ∶φ⊃ψ ,p NF Γ⊢φ∶Ω ⊃C NF Γ⊢ψ∶Ω ,p \<\\
\>                      trans-red (respects-red \{f = λ x → x ⊃ ψ\} (λ x → app (appl x)) (red-NF Γ⊢φ∶Ω)) \<\\
\>                                (respects-red \{f = λ x → cp2term (NF Γ⊢φ∶Ω) ⊃ x\} (λ x → app (appr (appl x))) (red-NF Γ⊢ψ∶Ω)) ,p  --TODO Extract lemma for reduction\<\\
\>                      (λ W Δ ρ ε ρ∶Γ⇒Δ Δ⊢ε∶φ computeε →\<\\
\>                      let φρ↠φ' : φ 〈 ρ 〉 ↠ cp2term φ'\<\\
\>                          φρ↠φ' = subst (λ x → (φ 〈 ρ 〉) ↠ x) (closed-rep φ') (respects-red (respects-osr replacement β-respects-rep) (red-NF Γ⊢φ∶Ω)) in\<\\
\>                      let ε∈EΔψ = hyp W Δ ρ ε (context-validity Δ⊢ε∶φ) ρ∶Γ⇒Δ        \<\\
\>                                  ((convR Δ⊢ε∶φ (weakening Γ⊢φ∶Ω (context-validity Δ⊢ε∶φ) ρ∶Γ⇒Δ) (sym-conv (red-conv φρ↠φ')) ) ,p φ' ,p φρ↠φ' ,p computeε ) in \<\\
\>                      let ψ' = proj₁ (proj₂ ε∈EΔψ) in \<\\
\>                      let ψρ↠ψ' : ψ 〈 ρ 〉 ↠ cp2term ψ'\<\\
\>                          ψρ↠ψ' = proj₁ (proj₂ (proj₂ ε∈EΔψ)) in \<\\
\>                      subst (λ a → compute Δ a (appP (δ 〈 ρ 〉) ε)) (confluent ψρ↠ψ' \<\\
\>                        (subst (λ x → (ψ 〈 ρ 〉) ↠ x) (closed-rep (NF Γ⊢ψ∶Ω)) (respects-red (respects-osr replacement β-respects-rep) (red-NF Γ⊢ψ∶Ω)))) \<\\
\>                        (proj₂ (proj₂ (proj₂ ε∈EΔψ))))\<\\
\>\<\\
\>  plus-univ : ∀ \{V\} \{φ ψ : Term V\} \{δ ε\} → key-redex (plus (univ φ ψ δ ε)) δ\<\\
\>  minus-univ : ∀ \{V\} \{φ ψ : Term V\} \{δ ε\} → key-redex (minus (univ φ ψ δ ε)) ε\<\\
\>  imp*-plus : ∀ \{V\} \{P Q : Path V\} \{δ ε\} → key-redex (appP (appP (plus (P ⊃* Q)) δ) ε) (appP (plus Q) (appP δ (appP (minus P) ε)))\<\\
\>  imp*-minus : ∀ \{V\} \{P Q : Path V\} \{δ ε\} → key-redex (appP (appP (minus (P ⊃* Q)) δ) ε) (appP (minus Q) (appP δ (appP (plus P) ε)))\<\\
\>  appPkr : ∀ \{V\} \{δ ε χ : Proof V\} → key-redex δ ε → key-redex (appP δ χ) (appP ε χ)\<\\
\>  pluskr : ∀ \{V\} \{P Q : Path V\} → key-redex P Q → key-redex (plus P) (plus Q)\<\\
\>  minuskr : ∀ \{V\} \{P Q : Path V\} → key-redex P Q → key-redex (minus P) (minus Q)\<\\
\>  app*kr : ∀ \{V\} \{N N' : Term V\} \{P\} \{P'\} \{Q\} → key-redex P P' → key-redex (app* N N' P Q) (app* N N' P' Q)\<\\
\>  plus-ref : ∀ \{V\} \{φ : Term V\} \{δ\} → key-redex (appP (plus (reff φ)) δ) δ\<\\
\>  minus-ref : ∀ \{V\} \{φ : Term V\} \{δ\} → key-redex (appP (minus (reff φ)) δ) δ\<\\
\>  app*-ref : ∀ \{V\} \{M N N' : Term V\} \{Q\} → key-redex (app* N N' (reff M) Q) (M ⋆[ Q ∶ N ∼ N' ])\<\\
\>\<\\
\>postulate key-redex-SN : ∀ \{V\} \{K\} \{E F : Expression V K\} → key-redex E F → SN F → SN E\<\\
\>\<\\
\>postulate key-redex-rep : ∀ \{U\} \{V\} \{K\} \{ρ : Rep U V\} \{E F : Expression U K\} → key-redex E F → key-redex (E 〈 ρ 〉) (F 〈 ρ 〉)\<\\
\>\<\\
\>expand-compute : ∀ \{V\} \{Γ : Context V\} \{A : closed-prop\} \{δ ε : Proof V\} →\<\\
\>                compute Γ A ε → valid Γ → key-redex δ ε → compute Γ A δ\<\\
\>expand-compute \{A = ⊥C\} computeε validΓ δ▷ε = key-redex-SN δ▷ε (compute-SN computeε validΓ)\<\\
\>expand-compute \{A = A ⊃C B\} computeε validΓ δ▷ε W Δ ρ χ ρ∶Γ⇒RΔ Δ⊢χ∶A computeχ = \<\\
\>  expand-compute (computeε W Δ ρ χ ρ∶Γ⇒RΔ Δ⊢χ∶A computeχ) (context-validity Δ⊢χ∶A)\<\\
\>      (appPkr (key-redex-rep δ▷ε)) \<\\
\>\<\\
\>expand-EP (Γ⊢ε∶φ ,p φ' ,p φ↠φ' ,p computeε) Γ⊢δ∶φ δ▷ε = Γ⊢δ∶φ ,p φ' ,p φ↠φ' ,p expand-compute computeε (context-validity Γ⊢δ∶φ) δ▷ε\<\\
\>\<\\
\>expand-computeE : ∀ \{V\} \{Γ : Context V\} \{A\} \{M\} \{N\} \{P\} \{Q\} →\<\\
\>  computeE Γ M A N Q → Γ ⊢ P ∶ M ≡〈 A 〉 N → key-redex P Q → computeE Γ M A N P\<\\
\>expand-computeE \{A = Ω\} ((\_ ,p M⊃Nnf ,p M⊃N↠M⊃Nnf ,p computeQ+) ,p (\_ ,p N⊃Mnf ,p N⊃M↠N⊃Mnf ,p computeQ-)) Γ⊢P∶M≡N P▷Q = \<\\
\>  ((plusR Γ⊢P∶M≡N) ,p M⊃Nnf ,p M⊃N↠M⊃Nnf ,p expand-compute computeQ+ \<\\
\>    (context-validity Γ⊢P∶M≡N) (pluskr P▷Q)) ,p \<\\
\>  (minusR Γ⊢P∶M≡N) ,p N⊃Mnf ,p N⊃M↠N⊃Mnf ,p expand-compute computeQ- \<\\
\>    (context-validity Γ⊢P∶M≡N) (minuskr P▷Q)\<\\
\>expand-computeE \{A = A ⇛ B\} \{M\} \{M'\} computeQ Γ⊢P∶M≡M' P▷Q = \<\\
\>  λ W Δ ρ N N' R ρ∶Γ⇒Δ Δ⊢R∶N≡N' N∈EΔA N'∈EΔA computeR → \<\\
\>  expand-computeE (computeQ W Δ ρ N N' R ρ∶Γ⇒Δ Δ⊢R∶N≡N' N∈EΔA N'∈EΔA computeR) \<\\
\>  (app*R (E-typed N∈EΔA) (E-typed N'∈EΔA) \<\\
\>    (weakening Γ⊢P∶M≡M' (context-validity Δ⊢R∶N≡N') ρ∶Γ⇒Δ)\<\\
\>    Δ⊢R∶N≡N')\<\\
\>  (app*kr (key-redex-rep P▷Q))\<\\
\>\<\\
\>ref-compute : ∀ \{V\} \{Γ : Context V\} \{M : Term V\} \{A : Type\} → E Γ A M → computeE Γ M A M (reff M)\<\\
\>ref-compute \{Γ = Γ\} \{M = φ\} \{A = Ω\} φ∈EΓΩ = \<\\
\>  let Γ⊢φ∶Ω : Γ ⊢ φ ∶ ty Ω\<\\
\>      Γ⊢φ∶Ω = E-typed φ∈EΓΩ in\<\\
\>  (func-EP (λ V Δ ρ ε validΔ ρ∶Γ⇒Δ ε∈EΔφ → expand-EP ε∈EΔφ (appPR (plusR (refR (weakening Γ⊢φ∶Ω validΔ ρ∶Γ⇒Δ))) (EP-typed ε∈EΔφ)) plus-ref) \<\\
\>  (plusR (refR Γ⊢φ∶Ω))) ,p \<\\
\>  func-EP (λ V Δ ρ ε validΔ ρ∶Γ⇒Δ ε∈EΔφ → expand-EP ε∈EΔφ (appPR (minusR (refR (weakening Γ⊢φ∶Ω validΔ ρ∶Γ⇒Δ))) (EP-typed ε∈EΔφ)) minus-ref) \<\\
\>  (minusR (refR Γ⊢φ∶Ω))\<\\
\>ref-compute \{A = A ⇛ B\} (Γ⊢M∶A⇛B ,p computeM ,p compute-eqM) = λ W Δ ρ N N' Q ρ∶Γ⇒Δ Δ⊢Q∶N≡N' N∈EΔA N'∈EΔA computeQ → \<\\
\>  expand-computeE (compute-eqM W Δ ρ N N' Q ρ∶Γ⇒Δ N∈EΔA N'∈EΔA computeQ Δ⊢Q∶N≡N') \<\\
\>    (app*R (E-typed N∈EΔA) (E-typed N'∈EΔA) (refR (weakening Γ⊢M∶A⇛B (context-validity Δ⊢Q∶N≡N') ρ∶Γ⇒Δ)) \<\\
\>      Δ⊢Q∶N≡N') app*-ref\<\\
\>\<\\
\>postulate Neutral-E : ∀ \{V\} \{Γ : Context V\} \{A\} \{M\} → Neutral M → Γ ⊢ M ∶ ty A → E Γ A M\<\\
\>\<\\
\>var-E' : ∀ \{V\} \{A\} (Γ : Context V) (x : Var V -Term) → \<\\
\>  valid Γ → typeof x Γ ≡ ty A → E Γ A (var x)\<\\
\>var-E : ∀ \{V\} (Γ : Context V) (x : Var V -Term) → \<\\
\>        valid Γ → E Γ (typeof' x Γ) (var x)\<\\
\>computeE-SN : ∀ \{V\} \{Γ : Context V\} \{M\} \{N\} \{A\} \{P\} → computeE Γ M A N P → valid Γ → SN P\<\\
\>\<\\
\>computeE-SN \{A = Ω\} \{P\} (P+∈EΓM⊃N ,p \_) \_ = \<\\
\>  let SNplusP : SN (plus P)\<\\
\>      SNplusP = EP-SN P+∈EΓM⊃N \<\\
\>  in SNsubbodyl (SNsubexp SNplusP)\<\\
\>computeE-SN \{V\} \{Γ\} \{A = A ⇛ B\} \{P\} computeP validΓ =\<\\
\>  let x₀∈EΓ,AA : E (Γ ,T A) A (var x₀)\<\\
\>      x₀∈EΓ,AA = var-E' \{A = A\} (Γ ,T A) x₀ (ctxTR validΓ) refl in\<\\
\>  let SNapp*xxPref : SN (app* (var x₀) (var x₀) (P ⇑) (reff (var x₀)))\<\\
\>      SNapp*xxPref = computeE-SN \{A = B\} (computeP (V , -Term) (Γ ,T A ) upRep \<\\
\>          (var x₀) (var x₀) (app -ref (var x₀ ,, out)) upRep-typed \<\\
\>          (refR (varR x₀ (ctxTR validΓ)) )\<\\
\>          x₀∈EΓ,AA x₀∈EΓ,AA (ref-compute x₀∈EΓ,AA)) \<\\
\>          (ctxTR validΓ)\<\\
\>  in SNap' \{Ops = replacement\} \{σ = upRep\} R-respects-replacement (SNsubbodyl (SNsubbodyr (SNsubbodyr (SNsubexp SNapp*xxPref))))\<\\
\>\<\\
\>\<\\
\>E-SN (Ω) = EΩ.sn\<\\
\>E-SN \{V\} \{Γ\} (A ⇛ B) \{M\} (Γ⊢M∶A⇛B ,p computeM ,p computeMpath) =\<\\
\>  let SNMx : SN (appT (M ⇑) (var x₀))\<\\
\>      SNMx = E-SN B \<\\
\>             (computeM (V , -Term) (Γ ,T A) upRep (var x₀) upRep-typed \<\\
\>             (var-E' \{A = A\} (Γ ,T A) x₀ (ctxTR (context-validity Γ⊢M∶A⇛B)) refl)) \<\\
\>  in SNap' \{Ops = replacement\} \{σ = upRep\} R-respects-replacement (SNsubbodyl (SNsubexp SNMx)) \<\\
\>\<\\
\>\{- Neutral-E \{A = Ω\} neutralM Γ⊢M∶A = record \{ \<\\
\>  typed = Γ⊢M∶A ; \<\\
\>  sn = Neutral-SN neutralM \}\<\\
\>Neutral-E \{A = A ⇛ B\} \{M\} neutralM Γ⊢M∶A⇛B = \<\\
\>  Γ⊢M∶A⇛B ,p \<\\
\>  (λ W Δ ρ N ρ∶Γ⇒Δ N∈EΔA → Neutral-E \{A = B\} (app (Neutral-rep M ρ neutralM) (E-SN A N∈EΔA)) \<\\
\>    (appR (weakening Γ⊢M∶A⇛B (context-validity (E-typed N∈EΔA)) ρ∶Γ⇒Δ) (E-typed N∈EΔA))) ,p \<\\
\>  (λ W Δ ρ N N' P ρ∶Γ⇒Δ N∈EΔA N'∈EΔA computeP Δ⊢P∶N≡N' → \<\\
\>    let validΔ = context-validity (E-typed N∈EΔA) in\<\\
\>    Neutral-computeE (Neutral-⋆ (Neutral-rep M ρ neutralM) (computeE-SN computeP validΔ) (E-SN A N∈EΔA) (E-SN A N'∈EΔA)) \<\\
\>    (⋆-typed (weakening Γ⊢M∶A⇛B validΔ ρ∶Γ⇒Δ) Δ⊢P∶N≡N')) -\}\<\\
\>\<\\
\>var-E' \{A = A\} Γ x validΓ x∶A∈Γ = Neutral-E (var x) (change-type (varR x validΓ) x∶A∈Γ)\<\\
\>\<\\
\>var-E Γ x validΓ = var-E' \{A = typeof' x Γ\} Γ x validΓ typeof-typeof'\<\\
\>\<\\
\>⊥-E validΓ = record \{ typed = ⊥R validΓ ; sn = ⊥SN \}\<\\
\>\<\\
\>⊃-E φ∈EΓΩ ψ∈EΓΩ = record \{ typed = ⊃R (E-typed φ∈EΓΩ) (E-typed ψ∈EΓΩ) ; \<\\
\>  sn = ⊃SN (E-SN Ω φ∈EΓΩ) (E-SN Ω ψ∈EΓΩ) \}\<\\
\>\<\\
\>appT-E \{V\} \{Γ\} \{M\} \{N\} \{A\} \{B\} validΓ (Γ⊢M∶A⇛B ,p computeM ,p computeMpath) N∈EΓA = \<\\
\>  subst (λ a → E Γ B (appT a N)) rep-idOp (computeM V Γ (idRep V) N idRep-typed N∈EΓA)\<\\
\>\<\\
\>postulate cp-typed : ∀ \{V\} \{Γ : Context V\} A → valid Γ → Γ ⊢ cp2term A ∶ ty Ω\<\\
\>\<\\
\>postulate ⊃-not-⊥ : ∀ \{V\} \{φ ψ : Term V\} → φ ⊃ ψ ↠ ⊥ → Empty\<\\
\>\<\\
\>postulate ⊃-inj₁ : ∀ \{V\} \{φ φ' ψ ψ' : Term V\} → φ ⊃ ψ ↠ φ' ⊃ ψ' → φ ↠ φ'\<\\
\>\<\\
\>postulate ⊃-inj₂ : ∀ \{V\} \{φ φ' ψ ψ' : Term V\} → φ ⊃ ψ ↠ φ' ⊃ ψ' → ψ ↠ ψ'\<\\
\>\<\\
\>postulate confluent₂ : ∀ \{V\} \{φ ψ : Term V\} \{χ : closed-prop\} → φ ≃ ψ → φ ↠ cp2term χ → ψ ↠ cp2term χ\<\\
\>\<\\
\>appP-EP (\_ ,p ⊥C ,p φ⊃ψ↠⊥ ,p \_) \_ = ⊥-elim (⊃-not-⊥ φ⊃ψ↠⊥)\<\\
\>appP-EP \{V\} \{Γ\} \{ε = ε\} \{φ\} \{ψ = ψ\} (Γ⊢δ∶φ⊃ψ ,p (φ' ⊃C ψ') ,p φ⊃ψ↠φ'⊃ψ' ,p computeδ) (Γ⊢ε∶φ ,p φ'' ,p φ↠φ'' ,p computeε) = \<\\
\>  (appPR Γ⊢δ∶φ⊃ψ Γ⊢ε∶φ) ,p ψ' ,p ⊃-inj₂ φ⊃ψ↠φ'⊃ψ' ,p \<\\
\>  subst (λ x → compute Γ ψ' (appP x ε)) rep-idOp \<\\
\>  (computeδ V Γ (idRep V) ε idRep-typed \<\\
\>    (convR Γ⊢ε∶φ (cp-typed φ' (context-validity Γ⊢ε∶φ)) (red-conv (⊃-inj₁ φ⊃ψ↠φ'⊃ψ')))\<\\
\>  (subst (λ x → compute Γ x ε) (confluent φ↠φ'' (⊃-inj₁ φ⊃ψ↠φ'⊃ψ')) computeε))\<\\
\>\<\\
\>conv-EP φ≃ψ (Γ⊢δ∶φ ,p φ' ,p φ↠φ' ,p computeδ) Γ⊢ψ∶Ω = convR Γ⊢δ∶φ Γ⊢ψ∶Ω φ≃ψ ,p φ' ,p confluent₂ \{χ = φ'\} φ≃ψ φ↠φ' ,p computeδ\<\\
\>\<\\
\>\<\\
\>postulate rep-EP : ∀ \{U\} \{V\} \{Γ\} \{Δ\} \{ρ : Rep U V\} \{φ\} \{δ\} →\<\\
\>                 EP Γ φ δ → ρ ∶ Γ ⇒R Δ → EP Δ (φ 〈 ρ 〉) (δ 〈 ρ 〉)\<\\
\>\<\\
\>univ-EE \{V\} \{Γ\} \{φ\} \{ψ\} \{δ\} \{ε\} δ∈EΓφ⊃ψ ε∈EΓψ⊃φ = \<\\
\>  let Γ⊢univ∶φ≡ψ : Γ ⊢ univ φ ψ δ ε ∶ φ ≡〈 Ω 〉 ψ\<\\
\>      Γ⊢univ∶φ≡ψ = (univR (EP-typed δ∈EΓφ⊃ψ) (EP-typed ε∈EΓψ⊃φ)) in\<\\
\>      (Γ⊢univ∶φ≡ψ ,p \<\\
\>      expand-EP δ∈EΓφ⊃ψ (plusR Γ⊢univ∶φ≡ψ) plus-univ ,p \<\\
\>      expand-EP ε∈EΓψ⊃φ (minusR Γ⊢univ∶φ≡ψ) minus-univ)\<\\
\>\<\\
\>postulate rep-EE : ∀ \{U\} \{V\} \{Γ\} \{Δ\} \{ρ : Rep U V\} \{E\} \{P\} →\<\\
\>                 EE Γ E P → ρ ∶ Γ ⇒R Δ → EE Δ (E 〈 ρ 〉) (P 〈 ρ 〉)\<\\
\>\<\\
\>imp*-EE \{Γ = Γ\} \{φ\} \{φ'\} \{ψ = ψ\} \{ψ'\} \{P\} \{Q = Q\} P∈EΓφ≡φ' Q∈EΓψ≡ψ' = (⊃*R (EE-typed P∈EΓφ≡φ') (EE-typed Q∈EΓψ≡ψ')) ,p \<\\
\>  func-EP (λ V Δ ρ ε validΔ ρ∶Γ⇒RΔ ε∈EΔφ⊃ψ →\<\\
\>    let Pρ : EE Δ (φ 〈 ρ 〉 ≡〈 Ω 〉 φ' 〈 ρ 〉) (P 〈 ρ 〉)\<\\
\>        Pρ = rep-EE P∈EΓφ≡φ' ρ∶Γ⇒RΔ in\<\\
\>    let Qρ : EE Δ (ψ 〈 ρ 〉 ≡〈 Ω 〉 ψ' 〈 ρ 〉) (Q 〈 ρ 〉)\<\\
\>        Qρ = rep-EE Q∈EΓψ≡ψ' ρ∶Γ⇒RΔ in\<\\
\>    func-EP (λ W Θ σ χ validΘ σ∶Δ⇒RΘ χ∈EΘφ' → \<\\
\>    let Pρσ : EE Θ (φ 〈 ρ 〉 〈 σ 〉 ≡〈 Ω 〉 φ' 〈 ρ 〉 〈 σ 〉) (P 〈 ρ 〉 〈 σ 〉)\<\\
\>        Pρσ = rep-EE Pρ σ∶Δ⇒RΘ in\<\\
\>    let Pρσ- : EP Θ (φ' 〈 ρ 〉 〈 σ 〉 ⊃ φ 〈 ρ 〉 〈 σ 〉) (minus (P 〈 ρ 〉 〈 σ 〉))\<\\
\>        Pρσ- = minus-EP Pρσ in\<\\
\>    let Qρσ : EE Θ (ψ 〈 ρ 〉 〈 σ 〉 ≡〈 Ω 〉 ψ' 〈 ρ 〉 〈 σ 〉) (Q 〈 ρ 〉 〈 σ 〉)\<\\
\>        Qρσ = rep-EE Qρ σ∶Δ⇒RΘ in\<\\
\>    let Qρσ+ : EP Θ (ψ 〈 ρ 〉 〈 σ 〉 ⊃ ψ' 〈 ρ 〉 〈 σ 〉) (plus (Q 〈 ρ 〉 〈 σ 〉))\<\\
\>        Qρσ+ = plus-EP Qρσ in\<\\
\>    let εσ : EP Θ (φ 〈 ρ 〉 〈 σ 〉 ⊃ ψ 〈 ρ 〉 〈 σ 〉) (ε 〈 σ 〉)\<\\
\>        εσ = rep-EP ε∈EΔφ⊃ψ σ∶Δ⇒RΘ in\<\\
\>    expand-EP \<\\
\>    (appP-EP Qρσ+ (appP-EP εσ (appP-EP Pρσ- χ∈EΘφ')))\<\\
\>    (appPR (appPR (plusR (⊃*R (EE-typed Pρσ) (EE-typed Qρσ))) (EP-typed εσ)) (EP-typed χ∈EΘφ')) \<\\
\>    imp*-plus) \<\\
\>    (appPR (plusR (⊃*R (EE-typed Pρ) (EE-typed Qρ))) (EP-typed ε∈EΔφ⊃ψ))) \<\\
\>  (plusR (⊃*R (EE-typed P∈EΓφ≡φ') (EE-typed Q∈EΓψ≡ψ'))) ,p \<\\
\>  func-EP (λ V Δ ρ ε validΔ ρ∶Γ⇒RΔ ε∈EΔφ'⊃ψ' →\<\\
\>    let Pρ : EE Δ (φ 〈 ρ 〉 ≡〈 Ω 〉 φ' 〈 ρ 〉) (P 〈 ρ 〉)\<\\
\>        Pρ = rep-EE P∈EΓφ≡φ' ρ∶Γ⇒RΔ in\<\\
\>    let Qρ : EE Δ (ψ 〈 ρ 〉 ≡〈 Ω 〉 ψ' 〈 ρ 〉) (Q 〈 ρ 〉)\<\\
\>        Qρ = rep-EE Q∈EΓψ≡ψ' ρ∶Γ⇒RΔ in\<\\
\>    func-EP (λ W Θ σ χ validΘ σ∶Δ⇒RΘ χ∈EΘφ' → \<\\
\>      let Pρσ : EE Θ (φ 〈 ρ 〉 〈 σ 〉 ≡〈 Ω 〉 φ' 〈 ρ 〉 〈 σ 〉) (P 〈 ρ 〉 〈 σ 〉)\<\\
\>          Pρσ = rep-EE Pρ σ∶Δ⇒RΘ in\<\\
\>      let Pρσ+ : EP Θ (φ 〈 ρ 〉 〈 σ 〉 ⊃ φ' 〈 ρ 〉 〈 σ 〉) (plus (P 〈 ρ 〉 〈 σ 〉))\<\\
\>          Pρσ+ = plus-EP Pρσ in\<\\
\>      let Qρσ : EE Θ (ψ 〈 ρ 〉 〈 σ 〉 ≡〈 Ω 〉 ψ' 〈 ρ 〉 〈 σ 〉) (Q 〈 ρ 〉 〈 σ 〉)\<\\
\>          Qρσ = rep-EE Qρ σ∶Δ⇒RΘ in\<\\
\>      let Qρσ- : EP Θ (ψ' 〈 ρ 〉 〈 σ 〉 ⊃ ψ 〈 ρ 〉 〈 σ 〉) (minus (Q 〈 ρ 〉 〈 σ 〉))\<\\
\>          Qρσ- = minus-EP Qρσ in\<\\
\>      let εσ : EP Θ (φ' 〈 ρ 〉 〈 σ 〉 ⊃ ψ' 〈 ρ 〉 〈 σ 〉) (ε 〈 σ 〉)\<\\
\>          εσ = rep-EP ε∈EΔφ'⊃ψ' σ∶Δ⇒RΘ in \<\\
\>      expand-EP \<\\
\>        (appP-EP Qρσ- (appP-EP εσ (appP-EP Pρσ+ χ∈EΘφ'))) \<\\
\>          (appPR (appPR (minusR (⊃*R (EE-typed Pρσ) (EE-typed Qρσ))) (EP-typed εσ)) (EP-typed χ∈EΘφ')) \<\\
\>        imp*-minus)\<\\
\>    (appPR (minusR (⊃*R (EE-typed Pρ) (EE-typed Qρ))) (EP-typed ε∈EΔφ'⊃ψ'))) \<\\
\>  (minusR (⊃*R (EE-typed P∈EΓφ≡φ') (EE-typed Q∈EΓψ≡ψ')))\<\\
\>\<\\
\>app*-EE \{V\} \{Γ\} \{M\} \{M'\} \{N\} \{N'\} \{A\} \{B\} \{P\} \{Q\} (Γ⊢P∶M≡M' ,p computeP) (Γ⊢Q∶N≡N' ,p computeQ) N∈EΓA N'∈EΓA = (app*R (E-typed N∈EΓA) (E-typed N'∈EΓA) Γ⊢P∶M≡M' Γ⊢Q∶N≡N') ,p \<\\
\>  subst₃\<\\
\>    (λ a b c →\<\\
\>       computeE Γ (appT a N) B (appT b N') (app* N N' c Q))\<\\
\>    rep-idOp rep-idOp rep-idOp \<\\
\>    (computeP V Γ (idRep V) N N' Q idRep-typed Γ⊢Q∶N≡N' \<\\
\>      N∈EΓA N'∈EΓA computeQ)\<\\
\>\<\\
\>func-EE \{U\} \{Γ\} \{A\} \{B\} \{M\} \{M'\} \{P\} Γ⊢P∶M≡M' hyp = Γ⊢P∶M≡M' ,p (λ W Δ ρ N N' Q ρ∶Γ⇒Δ Δ⊢Q∶N≡N' N∈EΔA N'∈EΔA computeQ → \<\\
\>  proj₂ (hyp W Δ N N' Q ρ ρ∶Γ⇒Δ N∈EΔA N'∈EΔA (Δ⊢Q∶N≡N' ,p computeQ)))\<\\
\>\<\\
\>ref-EE \{V\} \{Γ\} \{M\} \{A\} M∈EΓA = refR (E-typed M∈EΓA) ,p ref-compute M∈EΓA\<\\
\>\<\\
\>expand-EE \{V\} \{Γ\} \{A\} \{M\} \{N\} \{P\} \{Q\} (Γ⊢Q∶M≡N ,p computeQ) Γ⊢P∶M≡N P▷Q = Γ⊢P∶M≡N ,p expand-computeE computeQ Γ⊢P∶M≡N P▷Q\<\\
\>\<\\
\>postulate ⊃-respects-conv : ∀ \{V\} \{φ\} \{φ'\} \{ψ\} \{ψ' : Term V\} → φ ≃ φ' → ψ ≃ ψ' →\<\\
\>                          φ ⊃ ψ ≃ φ' ⊃ ψ'\<\\
\>\<\\
\>postulate appT-respects-convl : ∀ \{V\} \{M M' N : Term V\} → M ≃ M' → appT M N ≃ appT M' N\<\\
\>\<\\
\>conv-computeE : ∀ \{V\} \{Γ : Context V\} \{M\} \{M'\} \{N\} \{N'\} \{A\} \{P\} →\<\\
\>             computeE Γ M A N P → M ≃ M' → N ≃ N' → \<\\
\>             Γ ⊢ M' ∶ ty A  → Γ ⊢ N' ∶ ty A  →\<\\
\>             computeE Γ M' A N' P\<\\
\>conv-computeE \{M = M\} \{A = Ω\} \<\\
\>  (EPΓM⊃NP+ ,p EPΓN⊃MP-) M≃M' N≃N' Γ⊢M'∶Ω Γ⊢N'∶Ω = \<\\
\>  (conv-EP (⊃-respects-conv M≃M' N≃N')\<\\
\>    EPΓM⊃NP+ (⊃R Γ⊢M'∶Ω Γ⊢N'∶Ω)) ,p \<\\
\>  conv-EP (⊃-respects-conv N≃N' M≃M') EPΓN⊃MP- (⊃R Γ⊢N'∶Ω Γ⊢M'∶Ω)\<\\
\>conv-computeE \{M = M\} \{M'\} \{N\} \{N'\} \{A = A ⇛ B\} computeP M≃M' N≃N' Γ⊢M'∶A⇛B Γ⊢N'∶A⇛B =\<\\
\>  λ W Δ ρ L L' Q ρ∶Γ⇒RΔ Δ⊢Q∶L≡L' L∈EΔA L'∈EΔA computeQ → conv-computeE \{A = B\} \<\\
\>  (computeP W Δ ρ L L' Q ρ∶Γ⇒RΔ Δ⊢Q∶L≡L' L∈EΔA L'∈EΔA computeQ) \<\\
\>  (appT-respects-convl (respects-conv (respects-osr replacement β-respects-rep) M≃M')) \<\\
\>  (appT-respects-convl (respects-conv (respects-osr replacement β-respects-rep) N≃N')) \<\\
\>  (appR (weakening Γ⊢M'∶A⇛B (context-validity Δ⊢Q∶L≡L') ρ∶Γ⇒RΔ) (E-typed \{W\} \{Γ = Δ\} \{A = A\} \{L\} L∈EΔA)) \<\\
\>  (appR (weakening Γ⊢N'∶A⇛B (context-validity Δ⊢Q∶L≡L') ρ∶Γ⇒RΔ) (E-typed L'∈EΔA)) \<\\
\>--REFACTOR Duplication\<\\
\>\<\\
\>conv-EE (Γ⊢P∶M≡N ,p computeP) M≃M' N≃N' Γ⊢M'∶A Γ⊢N'∶A = convER Γ⊢P∶M≡N Γ⊢M'∶A Γ⊢N'∶A M≃M' N≃N' ,p conv-computeE computeP M≃M' N≃N' Γ⊢M'∶A Γ⊢N'∶A\<\\
\>--REFACTOR Duplication                      \<\\
\>                 \<\\
\>EE-SN (app (-eq \_) (\_ ,, \_ ,, out)) (Γ⊢P∶E ,p computeP) = computeE-SN computeP (context-validity Γ⊢P∶E) -\}}\<%
\end{code}
}


\section{Predicative Higher-Order Propositional Logic}

\AgdaHide{
\begin{code}%
\>\AgdaKeyword{module} \AgdaModule{PHOPL.Grammar} \AgdaKeyword{where}\<%
\\
%
\\
\>\AgdaKeyword{open} \AgdaKeyword{import} \AgdaModule{Data.Nat}\<%
\\
\>\AgdaKeyword{open} \AgdaKeyword{import} \AgdaModule{Data.Empty} \AgdaKeyword{renaming} \AgdaSymbol{(}\AgdaDatatype{⊥} \AgdaSymbol{to} \AgdaDatatype{Empty}\AgdaSymbol{)}\<%
\\
\>\AgdaKeyword{open} \AgdaKeyword{import} \AgdaModule{Data.List} \AgdaKeyword{hiding} \AgdaSymbol{(}\AgdaFunction{replicate}\AgdaSymbol{)}\<%
\\
\>\AgdaKeyword{open} \AgdaKeyword{import} \AgdaModule{Data.Vec} \AgdaKeyword{hiding} \AgdaSymbol{(}\AgdaFunction{replicate}\AgdaSymbol{)}\<%
\\
\>\AgdaKeyword{open} \AgdaKeyword{import} \AgdaModule{Prelims}\<%
\\
\>\AgdaKeyword{open} \AgdaKeyword{import} \AgdaModule{Grammar.Taxonomy}\<%
\\
\>\AgdaKeyword{open} \AgdaKeyword{import} \AgdaModule{Grammar.Base}\<%
\end{code}
}

\subsection{Syntax}

Fix three disjoint, infinite sets of variables, which we shall call \emph{term variables}, \emph{proof variables}
and \emph{path variables}.  We shall use $x$ and $y$ as term variables, $p$ and $q$ as proof variables,
$e$ as a path variable, and $z$ for a variable that may come from any of these three sets.

The syntax of $\lambda o e$ is given by the grammar:

\[
\begin{array}{lrcl}
\text{Type} & A,B,C & ::= & \Omega \mid A \rightarrow B \\
\text{Term} & L,M,N, \phi,\psi,\chi & ::= & x \mid \bot \mid \phi \supset \psi \mid \lambda x:A.M \mid MN \\
\text{Proof} & \delta, \epsilon & ::= & p \mid \lambda p:\phi.\delta \mid \delta \epsilon \mid P^+ \mid P^- \\
\text{Path} & P, Q & ::= & e \mid \reff{M} \mid P \supset^* Q \mid \univ{\phi}{\psi}{P}{Q} \mid \\
& & & \triplelambda e : x =_A y. P \mid P_{MN} Q \\
\text{Context} & \Gamma, \Delta, \Theta & ::= & \langle \rangle \mid \Gamma, x : A \mid \Gamma, p : \phi \mid \Gamma, e : M =_A N \\
\text{Judgement} & \mathbf{J} & ::= & \Gamma \vald \mid \Gamma \vdash M : A \mid \Gamma \vdash \delta : \phi \mid \\
& & & \Gamma \vdash P : M =_A N
\end{array}
\]

In the path $\triplelambda e : x =_A y . P$, the term variables $x$ and $y$ must be distinct.  (We also have $x \not\equiv e \not\equiv y$, thanks to our
stipulation that term variables and path variables are disjoint.)  The term variable $x$ is bound within $M$ in the term $\lambda x:A.M$,
and the proof variable $p$ is bound within $\delta$ in $\lambda p:\phi.\delta$.  The three variables $e$, $x$ and $y$ are bound within $P$ in the path
$\triplelambda e:x =_A y.P$.  We identify terms, proofs and paths up to $\alpha$-conversion.

We shall use the word 'expression' to mean either a type, term, proof, path, or equation (an equation having the form $M =_A N$).  We shall use $r$, $s$, $t$, $S$ and $T$ as metavariables that range over expressions.

Note that we use both Roman letters $M$, $N$ and Greek letters $\phi$, $\psi$, $\chi$ to range over terms.  Intuitively, a term is understood as either a proposition or a function,
and we shall use Greek letters for terms that are intended to be propositions.  Formally, there is no significance to which letter we choose.

Note also that the types of $\lambda o e$ are just the simple types over $\Omega$; therefore, no variable can occur in a type.

The intuition behind the new expressions is as follows (see also the rules of deduction in Figure \ref{fig:lambdaoe}).  For any object $M : A$, there is the trivial path $\reff{M} : M =_A M$.  The constructor $\supset^*$ ensures congruence for $\supset$ --- if $P : \phi =_\Omega \phi'$ and $Q : \psi =_\Omega \psi'$ then $P \supset^* Q : \phi \supset \psi =_\Omega \phi' \supset \psi'$.  The constructor $\mathsf{univ}$ gives univalence for our propositions: if $\delta : \phi \supset \psi$ and $\epsilon : \psi \supset \phi$, then $\univ{\phi}{\psi}{\delta}{\epsilon}$ is a path of type $\phi =_\Omega \psi$.  The constructors $^+$ and $^-$ are the converses: if $P$ is a path of type $\phi =_\Omega \psi$, then $P^+$ is a proof of $\phi \supset \psi$, and $P^-$ is a proof of $\psi \supset \phi$.

The constructor $\triplelambda$ gives functional extensionality.  Let $F$ and $G$ be functions of type $A \rightarrow B$.  If $F x =_B G y$ whenever $x =_A y$, then $F =_{A \rightarrow B} G$.  More formally, if $P$ is a path of type $Fx =_B Gy$ that depends on $x : A$, $y : A$ and $e : x =_A y$, then $\triplelambda e : x =_A y . P$ is a path of type $F =_{A \rightarrow B} G$.

Finally, if $P$ is a path of type $F =_{A \rightarrow B} G$, and $Q$ is a path $M =_A N$, then $P_{MN} Q$ is a path $FM =_B G N$.

\begin{code}%
\>\AgdaKeyword{data} \AgdaDatatype{PHOPLVarKind} \AgdaSymbol{:} \AgdaPrimitiveType{Set} \AgdaKeyword{where}\<%
\\
\>[0]\AgdaIndent{2}{}\<[2]%
\>[2]\AgdaInductiveConstructor{-Proof} \AgdaSymbol{:} \AgdaDatatype{PHOPLVarKind}\<%
\\
\>[0]\AgdaIndent{2}{}\<[2]%
\>[2]\AgdaInductiveConstructor{-Term} \AgdaSymbol{:} \AgdaDatatype{PHOPLVarKind}\<%
\\
\>[0]\AgdaIndent{2}{}\<[2]%
\>[2]\AgdaInductiveConstructor{-Path} \AgdaSymbol{:} \AgdaDatatype{PHOPLVarKind}\<%
\\
%
\\
\>\AgdaKeyword{data} \AgdaDatatype{PHOPLNonVarKind} \AgdaSymbol{:} \AgdaPrimitiveType{Set} \AgdaKeyword{where}\<%
\\
\>[0]\AgdaIndent{2}{}\<[2]%
\>[2]\AgdaInductiveConstructor{-Type} \AgdaSymbol{:} \AgdaDatatype{PHOPLNonVarKind}\<%
\\
\>[0]\AgdaIndent{2}{}\<[2]%
\>[2]\AgdaInductiveConstructor{-Equation} \AgdaSymbol{:} \AgdaDatatype{PHOPLNonVarKind}\<%
\\
%
\\
\>\AgdaFunction{PHOPLTaxonomy} \AgdaSymbol{:} \AgdaRecord{Taxonomy}\<%
\\
\>\AgdaFunction{PHOPLTaxonomy} \AgdaSymbol{=} \AgdaKeyword{record} \AgdaSymbol{\{} \<[25]%
\>[25]\<%
\\
\>[0]\AgdaIndent{2}{}\<[2]%
\>[2]\AgdaField{VarKind} \AgdaSymbol{=} \AgdaDatatype{PHOPLVarKind}\AgdaSymbol{;} \<[26]%
\>[26]\<%
\\
\>[0]\AgdaIndent{2}{}\<[2]%
\>[2]\AgdaField{NonVarKind} \AgdaSymbol{=} \AgdaDatatype{PHOPLNonVarKind} \AgdaSymbol{\}}\<%
\\
%
\\
\>\AgdaKeyword{module} \AgdaModule{PHOPLgrammar} \AgdaKeyword{where}\<%
\\
\>[0]\AgdaIndent{2}{}\<[2]%
\>[2]\AgdaKeyword{open} \AgdaModule{Taxonomy} \AgdaFunction{PHOPLTaxonomy}\<%
\\
%
\\
\>[0]\AgdaIndent{2}{}\<[2]%
\>[2]\AgdaFunction{-vProof} \AgdaSymbol{:} \AgdaDatatype{ExpKind}\<%
\\
\>[0]\AgdaIndent{2}{}\<[2]%
\>[2]\AgdaFunction{-vProof} \AgdaSymbol{=} \AgdaInductiveConstructor{varKind} \AgdaInductiveConstructor{-Proof}\<%
\\
%
\\
\>[0]\AgdaIndent{2}{}\<[2]%
\>[2]\AgdaFunction{-vTerm} \AgdaSymbol{:} \AgdaDatatype{ExpKind}\<%
\\
\>[0]\AgdaIndent{2}{}\<[2]%
\>[2]\AgdaFunction{-vTerm} \AgdaSymbol{=} \AgdaInductiveConstructor{varKind} \AgdaInductiveConstructor{-Term}\<%
\\
%
\\
\>[0]\AgdaIndent{2}{}\<[2]%
\>[2]\AgdaFunction{-vPath} \AgdaSymbol{:} \AgdaDatatype{ExpKind}\<%
\\
\>[0]\AgdaIndent{2}{}\<[2]%
\>[2]\AgdaFunction{-vPath} \AgdaSymbol{=} \AgdaInductiveConstructor{varKind} \AgdaInductiveConstructor{-Path}\<%
\\
%
\\
\>[0]\AgdaIndent{2}{}\<[2]%
\>[2]\AgdaFunction{-nvType} \AgdaSymbol{:} \AgdaDatatype{ExpKind}\<%
\\
\>[0]\AgdaIndent{2}{}\<[2]%
\>[2]\AgdaFunction{-nvType} \AgdaSymbol{=} \AgdaInductiveConstructor{nonVarKind} \AgdaInductiveConstructor{-Type}\<%
\\
%
\\
\>[0]\AgdaIndent{2}{}\<[2]%
\>[2]\AgdaFunction{-nvEq} \AgdaSymbol{:} \AgdaDatatype{ExpKind}\<%
\\
\>[0]\AgdaIndent{2}{}\<[2]%
\>[2]\AgdaFunction{-nvEq} \AgdaSymbol{=} \AgdaInductiveConstructor{nonVarKind} \AgdaInductiveConstructor{-Equation}\<%
\\
%
\\
\>[0]\AgdaIndent{2}{}\<[2]%
\>[2]\AgdaKeyword{data} \AgdaDatatype{Type} \AgdaSymbol{:} \AgdaPrimitiveType{Set} \AgdaKeyword{where}\<%
\\
\>[2]\AgdaIndent{4}{}\<[4]%
\>[4]\AgdaInductiveConstructor{Ω} \AgdaSymbol{:} \AgdaDatatype{Type}\<%
\\
\>[2]\AgdaIndent{4}{}\<[4]%
\>[4]\AgdaInductiveConstructor{\_⇛\_} \AgdaSymbol{:} \AgdaDatatype{Type} \AgdaSymbol{→} \AgdaDatatype{Type} \AgdaSymbol{→} \AgdaDatatype{Type}\<%
\\
%
\\
\>[0]\AgdaIndent{2}{}\<[2]%
\>[2]\AgdaKeyword{data} \AgdaDatatype{Dir} \AgdaSymbol{:} \AgdaPrimitiveType{Set} \AgdaKeyword{where}\<%
\\
\>[2]\AgdaIndent{4}{}\<[4]%
\>[4]\AgdaInductiveConstructor{-plus} \AgdaSymbol{:} \AgdaDatatype{Dir}\<%
\\
\>[2]\AgdaIndent{4}{}\<[4]%
\>[4]\AgdaInductiveConstructor{-minus} \AgdaSymbol{:} \AgdaDatatype{Dir}\<%
\\
%
\\
\>[0]\AgdaIndent{2}{}\<[2]%
\>[2]\AgdaFunction{pathDom} \AgdaSymbol{:} \AgdaDatatype{List} \AgdaFunction{VarKind}\<%
\\
\>[0]\AgdaIndent{2}{}\<[2]%
\>[2]\AgdaFunction{pathDom} \AgdaSymbol{=} \AgdaInductiveConstructor{-Term} \AgdaInductiveConstructor{∷} \AgdaInductiveConstructor{-Term} \AgdaInductiveConstructor{∷} \AgdaInductiveConstructor{-Path} \AgdaInductiveConstructor{∷} \AgdaInductiveConstructor{[]}\<%
\\
%
\\
\>[0]\AgdaIndent{2}{}\<[2]%
\>[2]\AgdaKeyword{data} \AgdaDatatype{PHOPLcon} \AgdaSymbol{:} \AgdaFunction{ConKind} \AgdaSymbol{→} \AgdaPrimitiveType{Set} \AgdaKeyword{where}\<%
\\
\>[2]\AgdaIndent{4}{}\<[4]%
\>[4]\AgdaInductiveConstructor{-ty} \AgdaSymbol{:} \AgdaDatatype{Type} \AgdaSymbol{→} \AgdaDatatype{PHOPLcon} \AgdaSymbol{(}\AgdaFunction{-nvType} \AgdaFunction{✧}\AgdaSymbol{)}\<%
\\
\>[2]\AgdaIndent{4}{}\<[4]%
\>[4]\AgdaInductiveConstructor{-bot} \AgdaSymbol{:} \AgdaDatatype{PHOPLcon} \AgdaSymbol{(}\AgdaFunction{-vTerm} \AgdaFunction{✧}\AgdaSymbol{)}\<%
\\
\>[2]\AgdaIndent{4}{}\<[4]%
\>[4]\AgdaInductiveConstructor{-imp} \AgdaSymbol{:} \AgdaDatatype{PHOPLcon} \AgdaSymbol{(}\AgdaFunction{-vTerm} \AgdaFunction{✧} \AgdaFunction{⟶} \AgdaFunction{-vTerm} \AgdaFunction{✧} \AgdaFunction{⟶} \AgdaFunction{-vTerm} \AgdaFunction{✧}\AgdaSymbol{)}\<%
\\
\>[2]\AgdaIndent{4}{}\<[4]%
\>[4]\AgdaInductiveConstructor{-lamTerm} \AgdaSymbol{:} \AgdaDatatype{Type} \AgdaSymbol{→} \AgdaDatatype{PHOPLcon} \AgdaSymbol{((}\AgdaInductiveConstructor{-Term} \AgdaFunction{⟶} \AgdaFunction{-vTerm} \AgdaFunction{✧}\AgdaSymbol{)} \AgdaFunction{⟶} \AgdaFunction{-vTerm} \AgdaFunction{✧}\AgdaSymbol{)}\<%
\\
\>[2]\AgdaIndent{4}{}\<[4]%
\>[4]\AgdaInductiveConstructor{-appTerm} \AgdaSymbol{:} \AgdaDatatype{PHOPLcon} \AgdaSymbol{(}\AgdaFunction{-vTerm} \AgdaFunction{✧} \AgdaFunction{⟶} \AgdaFunction{-vTerm} \AgdaFunction{✧} \AgdaFunction{⟶} \AgdaFunction{-vTerm} \AgdaFunction{✧}\AgdaSymbol{)}\<%
\\
\>[2]\AgdaIndent{4}{}\<[4]%
\>[4]\AgdaInductiveConstructor{-lamProof} \AgdaSymbol{:} \AgdaDatatype{PHOPLcon} \AgdaSymbol{(}\AgdaFunction{-vTerm} \AgdaFunction{✧} \AgdaFunction{⟶} \AgdaSymbol{(}\AgdaInductiveConstructor{-Proof} \AgdaFunction{⟶} \AgdaFunction{-vProof} \AgdaFunction{✧}\AgdaSymbol{)} \AgdaFunction{⟶} \AgdaFunction{-vProof} \AgdaFunction{✧}\AgdaSymbol{)}\<%
\\
\>[2]\AgdaIndent{4}{}\<[4]%
\>[4]\AgdaInductiveConstructor{-appProof} \AgdaSymbol{:} \AgdaDatatype{PHOPLcon} \AgdaSymbol{(}\AgdaFunction{-vProof} \AgdaFunction{✧} \AgdaFunction{⟶} \AgdaFunction{-vProof} \AgdaFunction{✧} \AgdaFunction{⟶} \AgdaFunction{-vProof} \AgdaFunction{✧}\AgdaSymbol{)}\<%
\\
\>[2]\AgdaIndent{4}{}\<[4]%
\>[4]\AgdaInductiveConstructor{-dir} \AgdaSymbol{:} \AgdaDatatype{Dir} \AgdaSymbol{→} \AgdaDatatype{PHOPLcon} \AgdaSymbol{(}\AgdaFunction{-vPath} \AgdaFunction{✧} \AgdaFunction{⟶} \AgdaFunction{-vProof} \AgdaFunction{✧}\AgdaSymbol{)}\<%
\\
\>[2]\AgdaIndent{4}{}\<[4]%
\>[4]\AgdaInductiveConstructor{-ref} \AgdaSymbol{:} \AgdaDatatype{PHOPLcon} \AgdaSymbol{(}\AgdaFunction{-vTerm} \AgdaFunction{✧} \AgdaFunction{⟶} \AgdaFunction{-vPath} \AgdaFunction{✧}\AgdaSymbol{)}\<%
\\
\>[2]\AgdaIndent{4}{}\<[4]%
\>[4]\AgdaInductiveConstructor{-imp*} \AgdaSymbol{:} \AgdaDatatype{PHOPLcon} \AgdaSymbol{(}\AgdaFunction{-vPath} \AgdaFunction{✧} \AgdaFunction{⟶} \AgdaFunction{-vPath} \AgdaFunction{✧} \AgdaFunction{⟶} \AgdaFunction{-vPath} \AgdaFunction{✧}\AgdaSymbol{)}\<%
\\
\>[2]\AgdaIndent{4}{}\<[4]%
\>[4]\AgdaInductiveConstructor{-univ} \AgdaSymbol{:} \AgdaDatatype{PHOPLcon} \AgdaSymbol{(}\AgdaFunction{-vTerm} \AgdaFunction{✧} \AgdaFunction{⟶} \AgdaFunction{-vTerm} \AgdaFunction{✧} \AgdaFunction{⟶} \AgdaFunction{-vProof} \AgdaFunction{✧} \AgdaFunction{⟶} \AgdaFunction{-vProof} \AgdaFunction{✧} \AgdaFunction{⟶} \AgdaFunction{-vPath} \AgdaFunction{✧}\AgdaSymbol{)}\<%
\\
\>[2]\AgdaIndent{4}{}\<[4]%
\>[4]\AgdaInductiveConstructor{-lll} \AgdaSymbol{:} \AgdaDatatype{Type} \AgdaSymbol{→} \AgdaDatatype{PHOPLcon} \AgdaSymbol{(}\AgdaInductiveConstructor{SK} \AgdaFunction{pathDom} \AgdaFunction{-vPath} \AgdaFunction{⟶} \AgdaFunction{-vPath} \AgdaFunction{✧}\AgdaSymbol{)}\<%
\\
\>[2]\AgdaIndent{4}{}\<[4]%
\>[4]\AgdaInductiveConstructor{-app*} \AgdaSymbol{:} \AgdaDatatype{PHOPLcon} \AgdaSymbol{(}\AgdaFunction{-vTerm} \AgdaFunction{✧} \AgdaFunction{⟶} \AgdaFunction{-vTerm} \AgdaFunction{✧} \AgdaFunction{⟶} \AgdaFunction{-vPath} \AgdaFunction{✧} \AgdaFunction{⟶} \AgdaFunction{-vPath} \AgdaFunction{✧} \AgdaFunction{⟶} \AgdaFunction{-vPath} \AgdaFunction{✧}\AgdaSymbol{)}\<%
\\
\>[2]\AgdaIndent{4}{}\<[4]%
\>[4]\AgdaInductiveConstructor{-eq} \AgdaSymbol{:} \AgdaDatatype{Type} \AgdaSymbol{→} \AgdaDatatype{PHOPLcon} \AgdaSymbol{(}\AgdaFunction{-vTerm} \AgdaFunction{✧} \AgdaFunction{⟶} \AgdaFunction{-vTerm} \AgdaFunction{✧} \AgdaFunction{⟶} \AgdaFunction{-nvEq} \AgdaFunction{✧}\AgdaSymbol{)}\<%
\\
%
\\
\>[0]\AgdaIndent{2}{}\<[2]%
\>[2]\AgdaFunction{PHOPLparent} \AgdaSymbol{:} \AgdaDatatype{PHOPLVarKind} \AgdaSymbol{→} \AgdaDatatype{ExpKind}\<%
\\
\>[0]\AgdaIndent{2}{}\<[2]%
\>[2]\AgdaFunction{PHOPLparent} \AgdaInductiveConstructor{-Proof} \AgdaSymbol{=} \AgdaFunction{-vTerm}\<%
\\
\>[0]\AgdaIndent{2}{}\<[2]%
\>[2]\AgdaFunction{PHOPLparent} \AgdaInductiveConstructor{-Term} \AgdaSymbol{=} \AgdaFunction{-nvType}\<%
\\
\>[0]\AgdaIndent{2}{}\<[2]%
\>[2]\AgdaFunction{PHOPLparent} \AgdaInductiveConstructor{-Path} \AgdaSymbol{=} \AgdaFunction{-nvEq}\<%
\\
%
\\
\>[0]\AgdaIndent{2}{}\<[2]%
\>[2]\AgdaFunction{PHOPL} \AgdaSymbol{:} \AgdaRecord{Grammar}\<%
\\
\>[0]\AgdaIndent{2}{}\<[2]%
\>[2]\AgdaFunction{PHOPL} \AgdaSymbol{=} \AgdaKeyword{record} \AgdaSymbol{\{} \<[19]%
\>[19]\<%
\\
\>[2]\AgdaIndent{4}{}\<[4]%
\>[4]\AgdaField{taxonomy} \AgdaSymbol{=} \AgdaFunction{PHOPLTaxonomy}\AgdaSymbol{;}\<%
\\
\>[2]\AgdaIndent{4}{}\<[4]%
\>[4]\AgdaField{isGrammar} \AgdaSymbol{=} \AgdaKeyword{record} \AgdaSymbol{\{} \<[25]%
\>[25]\<%
\\
\>[4]\AgdaIndent{6}{}\<[6]%
\>[6]\AgdaField{Con} \AgdaSymbol{=} \AgdaDatatype{PHOPLcon}\AgdaSymbol{;} \<[22]%
\>[22]\<%
\\
\>[4]\AgdaIndent{6}{}\<[6]%
\>[6]\AgdaField{parent} \AgdaSymbol{=} \AgdaFunction{PHOPLparent} \AgdaSymbol{\}} \AgdaSymbol{\}}\<%
\end{code}

\AgdaHide{
\begin{code}%
\>\AgdaKeyword{open} \AgdaModule{PHOPLgrammar} \AgdaKeyword{public}\<%
\\
\>\AgdaKeyword{open} \AgdaKeyword{import} \AgdaModule{Grammar} \AgdaFunction{PHOPL} \AgdaKeyword{public}\<%
\\
%
\\
\>\AgdaFunction{Proof} \AgdaSymbol{:} \AgdaDatatype{Alphabet} \AgdaSymbol{→} \AgdaPrimitiveType{Set}\<%
\\
\>\AgdaFunction{Proof} \AgdaBound{V} \AgdaSymbol{=} \AgdaFunction{Expression} \AgdaBound{V} \AgdaFunction{-vProof}\<%
\\
%
\\
\>\AgdaFunction{Term} \AgdaSymbol{:} \AgdaDatatype{Alphabet} \AgdaSymbol{→} \AgdaPrimitiveType{Set}\<%
\\
\>\AgdaFunction{Term} \AgdaBound{V} \AgdaSymbol{=} \AgdaFunction{Expression} \AgdaBound{V} \AgdaFunction{-vTerm}\<%
\\
%
\\
\>\AgdaFunction{Path} \AgdaSymbol{:} \AgdaDatatype{Alphabet} \AgdaSymbol{→} \AgdaPrimitiveType{Set}\<%
\\
\>\AgdaFunction{Path} \AgdaBound{V} \AgdaSymbol{=} \AgdaFunction{Expression} \AgdaBound{V} \AgdaFunction{-vPath}\<%
\\
%
\\
\>\AgdaFunction{Equation} \AgdaSymbol{:} \AgdaDatatype{Alphabet} \AgdaSymbol{→} \AgdaPrimitiveType{Set}\<%
\\
\>\AgdaFunction{Equation} \AgdaBound{V} \AgdaSymbol{=} \AgdaFunction{Expression} \AgdaBound{V} \AgdaFunction{-nvEq}\<%
\\
%
\\
\>\AgdaFunction{ty} \AgdaSymbol{:} \AgdaSymbol{∀} \AgdaSymbol{\{}\AgdaBound{V}\AgdaSymbol{\}} \AgdaSymbol{→} \AgdaDatatype{Type} \AgdaSymbol{→} \AgdaFunction{Expression} \AgdaBound{V} \AgdaSymbol{(}\AgdaInductiveConstructor{nonVarKind} \AgdaInductiveConstructor{-Type}\AgdaSymbol{)}\<%
\\
\>\AgdaFunction{ty} \AgdaBound{A} \AgdaSymbol{=} \AgdaInductiveConstructor{app} \AgdaSymbol{(}\AgdaInductiveConstructor{-ty} \AgdaBound{A}\AgdaSymbol{)} \AgdaInductiveConstructor{[]}\<%
\\
%
\\
\>\AgdaFunction{⊥} \AgdaSymbol{:} \AgdaSymbol{∀} \AgdaSymbol{\{}\AgdaBound{V}\AgdaSymbol{\}} \AgdaSymbol{→} \AgdaFunction{Term} \AgdaBound{V}\<%
\\
\>\AgdaFunction{⊥} \AgdaSymbol{=} \AgdaInductiveConstructor{app} \AgdaInductiveConstructor{-bot} \AgdaInductiveConstructor{[]}\<%
\\
%
\\
\>\AgdaKeyword{infix} \AgdaNumber{65} \AgdaFixityOp{\_⊃\_}\<%
\\
\>\AgdaFunction{\_⊃\_} \AgdaSymbol{:} \AgdaSymbol{∀} \AgdaSymbol{\{}\AgdaBound{V}\AgdaSymbol{\}} \AgdaSymbol{→} \AgdaFunction{Term} \AgdaBound{V} \AgdaSymbol{→} \AgdaFunction{Term} \AgdaBound{V} \AgdaSymbol{→} \AgdaFunction{Term} \AgdaBound{V}\<%
\\
\>\AgdaBound{φ} \AgdaFunction{⊃} \AgdaBound{ψ} \AgdaSymbol{=} \AgdaInductiveConstructor{app} \AgdaInductiveConstructor{-imp} \AgdaSymbol{(}\AgdaBound{φ} \AgdaInductiveConstructor{∷} \AgdaBound{ψ} \AgdaInductiveConstructor{∷} \AgdaInductiveConstructor{[]}\AgdaSymbol{)}\<%
\\
%
\\
\>\AgdaFunction{ΛT} \AgdaSymbol{:} \AgdaSymbol{∀} \AgdaSymbol{\{}\AgdaBound{V}\AgdaSymbol{\}} \AgdaSymbol{→} \AgdaDatatype{Type} \AgdaSymbol{→} \AgdaFunction{Term} \AgdaSymbol{(}\AgdaBound{V} \AgdaInductiveConstructor{,} \AgdaInductiveConstructor{-Term}\AgdaSymbol{)} \AgdaSymbol{→} \AgdaFunction{Term} \AgdaBound{V}\<%
\\
\>\AgdaFunction{ΛT} \AgdaBound{A} \AgdaBound{M} \AgdaSymbol{=} \AgdaInductiveConstructor{app} \AgdaSymbol{(}\AgdaInductiveConstructor{-lamTerm} \AgdaBound{A}\AgdaSymbol{)} \AgdaSymbol{(}\AgdaBound{M} \AgdaInductiveConstructor{∷} \AgdaInductiveConstructor{[]}\AgdaSymbol{)}\<%
\\
%
\\
\>\AgdaFunction{appT} \AgdaSymbol{:} \AgdaSymbol{∀} \AgdaSymbol{\{}\AgdaBound{V}\AgdaSymbol{\}} \AgdaSymbol{→} \AgdaFunction{Term} \AgdaBound{V} \AgdaSymbol{→} \AgdaFunction{Term} \AgdaBound{V} \AgdaSymbol{→} \AgdaFunction{Term} \AgdaBound{V}\<%
\\
\>\AgdaFunction{appT} \AgdaBound{M} \AgdaBound{N} \AgdaSymbol{=} \AgdaInductiveConstructor{app} \AgdaInductiveConstructor{-appTerm} \AgdaSymbol{(}\AgdaBound{M} \AgdaInductiveConstructor{∷} \AgdaBound{N} \AgdaInductiveConstructor{∷} \AgdaInductiveConstructor{[]}\AgdaSymbol{)}\<%
\\
%
\\
\>\AgdaFunction{ΛP} \AgdaSymbol{:} \AgdaSymbol{∀} \AgdaSymbol{\{}\AgdaBound{V}\AgdaSymbol{\}} \AgdaSymbol{→} \AgdaFunction{Term} \AgdaBound{V} \AgdaSymbol{→} \AgdaFunction{Proof} \AgdaSymbol{(}\AgdaBound{V} \AgdaInductiveConstructor{,} \AgdaInductiveConstructor{-Proof}\AgdaSymbol{)} \AgdaSymbol{→} \AgdaFunction{Proof} \AgdaBound{V}\<%
\\
\>\AgdaFunction{ΛP} \AgdaBound{φ} \AgdaBound{δ} \AgdaSymbol{=} \AgdaInductiveConstructor{app} \AgdaInductiveConstructor{-lamProof} \AgdaSymbol{(}\AgdaBound{φ} \AgdaInductiveConstructor{∷} \AgdaBound{δ} \AgdaInductiveConstructor{∷} \AgdaInductiveConstructor{[]}\AgdaSymbol{)}\<%
\\
%
\\
\>\AgdaFunction{appP} \AgdaSymbol{:} \AgdaSymbol{∀} \AgdaSymbol{\{}\AgdaBound{V}\AgdaSymbol{\}} \AgdaSymbol{→} \AgdaFunction{Proof} \AgdaBound{V} \AgdaSymbol{→} \AgdaFunction{Proof} \AgdaBound{V} \AgdaSymbol{→} \AgdaFunction{Proof} \AgdaBound{V}\<%
\\
\>\AgdaFunction{appP} \AgdaBound{δ} \AgdaBound{ε} \AgdaSymbol{=} \AgdaInductiveConstructor{app} \AgdaInductiveConstructor{-appProof} \AgdaSymbol{(}\AgdaBound{δ} \AgdaInductiveConstructor{∷} \AgdaBound{ε} \AgdaInductiveConstructor{∷} \AgdaInductiveConstructor{[]}\AgdaSymbol{)}\<%
\\
%
\\
\>\AgdaFunction{dir} \AgdaSymbol{:} \AgdaSymbol{∀} \AgdaSymbol{\{}\AgdaBound{V}\AgdaSymbol{\}} \AgdaSymbol{→} \AgdaDatatype{Dir} \AgdaSymbol{→} \AgdaFunction{Path} \AgdaBound{V} \AgdaSymbol{→} \AgdaFunction{Proof} \AgdaBound{V}\<%
\\
\>\AgdaFunction{dir} \AgdaBound{d} \AgdaBound{P} \AgdaSymbol{=} \AgdaInductiveConstructor{app} \AgdaSymbol{(}\AgdaInductiveConstructor{-dir} \AgdaBound{d}\AgdaSymbol{)} \AgdaSymbol{(}\AgdaBound{P} \AgdaInductiveConstructor{∷} \AgdaInductiveConstructor{[]}\AgdaSymbol{)}\<%
\\
%
\\
\>\AgdaFunction{plus} \AgdaSymbol{:} \AgdaSymbol{∀} \AgdaSymbol{\{}\AgdaBound{V}\AgdaSymbol{\}} \AgdaSymbol{→} \AgdaFunction{Path} \AgdaBound{V} \AgdaSymbol{→} \AgdaFunction{Proof} \AgdaBound{V}\<%
\\
\>\AgdaFunction{plus} \AgdaBound{P} \AgdaSymbol{=} \AgdaFunction{dir} \AgdaInductiveConstructor{-plus} \AgdaBound{P}\<%
\\
%
\\
\>\AgdaFunction{minus} \AgdaSymbol{:} \AgdaSymbol{∀} \AgdaSymbol{\{}\AgdaBound{V}\AgdaSymbol{\}} \AgdaSymbol{→} \AgdaFunction{Path} \AgdaBound{V} \AgdaSymbol{→} \AgdaFunction{Proof} \AgdaBound{V}\<%
\\
\>\AgdaFunction{minus} \AgdaBound{P} \AgdaSymbol{=} \AgdaFunction{dir} \AgdaInductiveConstructor{-minus} \AgdaBound{P}\<%
\\
%
\\
\>\AgdaFunction{reff} \AgdaSymbol{:} \AgdaSymbol{∀} \AgdaSymbol{\{}\AgdaBound{V}\AgdaSymbol{\}} \AgdaSymbol{→} \AgdaFunction{Term} \AgdaBound{V} \AgdaSymbol{→} \AgdaFunction{Path} \AgdaBound{V}\<%
\\
\>\AgdaFunction{reff} \AgdaBound{M} \AgdaSymbol{=} \AgdaInductiveConstructor{app} \AgdaInductiveConstructor{-ref} \AgdaSymbol{(}\AgdaBound{M} \AgdaInductiveConstructor{∷} \AgdaInductiveConstructor{[]}\AgdaSymbol{)}\<%
\\
%
\\
\>\AgdaKeyword{infix} \AgdaNumber{15} \AgdaFixityOp{\_⊃*\_}\<%
\\
\>\AgdaFunction{\_⊃*\_} \AgdaSymbol{:} \AgdaSymbol{∀} \AgdaSymbol{\{}\AgdaBound{V}\AgdaSymbol{\}} \AgdaSymbol{→} \AgdaFunction{Path} \AgdaBound{V} \AgdaSymbol{→} \AgdaFunction{Path} \AgdaBound{V} \AgdaSymbol{→} \AgdaFunction{Path} \AgdaBound{V}\<%
\\
\>\AgdaBound{P} \AgdaFunction{⊃*} \AgdaBound{Q} \AgdaSymbol{=} \AgdaInductiveConstructor{app} \AgdaInductiveConstructor{-imp*} \AgdaSymbol{(}\AgdaBound{P} \AgdaInductiveConstructor{∷} \AgdaBound{Q} \AgdaInductiveConstructor{∷} \AgdaInductiveConstructor{[]}\AgdaSymbol{)}\<%
\\
%
\\
\>\AgdaFunction{univ} \AgdaSymbol{:} \AgdaSymbol{∀} \AgdaSymbol{\{}\AgdaBound{V}\AgdaSymbol{\}} \AgdaSymbol{→} \AgdaFunction{Term} \AgdaBound{V} \AgdaSymbol{→} \AgdaFunction{Term} \AgdaBound{V} \AgdaSymbol{→} \AgdaFunction{Proof} \AgdaBound{V} \AgdaSymbol{→} \AgdaFunction{Proof} \AgdaBound{V} \AgdaSymbol{→} \AgdaFunction{Path} \AgdaBound{V}\<%
\\
\>\AgdaFunction{univ} \AgdaBound{φ} \AgdaBound{ψ} \AgdaBound{P} \AgdaBound{Q} \AgdaSymbol{=} \AgdaInductiveConstructor{app} \AgdaInductiveConstructor{-univ} \AgdaSymbol{(}\AgdaBound{φ} \AgdaInductiveConstructor{∷} \AgdaBound{ψ} \AgdaInductiveConstructor{∷} \AgdaBound{P} \AgdaInductiveConstructor{∷} \AgdaBound{Q} \AgdaInductiveConstructor{∷} \AgdaInductiveConstructor{[]}\AgdaSymbol{)}\<%
\\
%
\\
\>\AgdaFunction{λλλ} \AgdaSymbol{:} \AgdaSymbol{∀} \AgdaSymbol{\{}\AgdaBound{V}\AgdaSymbol{\}} \AgdaSymbol{→} \AgdaDatatype{Type} \AgdaSymbol{→} \AgdaFunction{Path} \AgdaSymbol{(}\AgdaBound{V} \AgdaInductiveConstructor{,} \AgdaInductiveConstructor{-Term} \AgdaInductiveConstructor{,} \AgdaInductiveConstructor{-Term} \AgdaInductiveConstructor{,} \AgdaInductiveConstructor{-Path}\AgdaSymbol{)} \AgdaSymbol{→} \AgdaFunction{Path} \AgdaBound{V}\<%
\\
\>\AgdaFunction{λλλ} \AgdaBound{A} \AgdaBound{P} \AgdaSymbol{=} \AgdaInductiveConstructor{app} \AgdaSymbol{(}\AgdaInductiveConstructor{-lll} \AgdaBound{A}\AgdaSymbol{)} \AgdaSymbol{(}\AgdaBound{P} \AgdaInductiveConstructor{∷} \AgdaInductiveConstructor{[]}\AgdaSymbol{)}\<%
\\
%
\\
\>\AgdaFunction{app*} \AgdaSymbol{:} \AgdaSymbol{∀} \AgdaSymbol{\{}\AgdaBound{V}\AgdaSymbol{\}} \AgdaSymbol{→} \AgdaFunction{Term} \AgdaBound{V} \AgdaSymbol{→} \AgdaFunction{Term} \AgdaBound{V} \AgdaSymbol{→} \AgdaFunction{Path} \AgdaBound{V} \AgdaSymbol{→} \AgdaFunction{Path} \AgdaBound{V} \AgdaSymbol{→} \AgdaFunction{Path} \AgdaBound{V}\<%
\\
\>\AgdaFunction{app*} \AgdaBound{M} \AgdaBound{N} \AgdaBound{P} \AgdaBound{Q} \AgdaSymbol{=} \AgdaInductiveConstructor{app} \AgdaInductiveConstructor{-app*} \AgdaSymbol{(}\AgdaBound{M} \AgdaInductiveConstructor{∷} \AgdaBound{N} \AgdaInductiveConstructor{∷} \AgdaBound{P} \AgdaInductiveConstructor{∷} \AgdaBound{Q} \AgdaInductiveConstructor{∷} \AgdaInductiveConstructor{[]}\AgdaSymbol{)}\<%
\\
%
\\
\>\AgdaKeyword{infix} \AgdaNumber{60} \AgdaFixityOp{\_≡〈\_〉\_}\<%
\\
\>\AgdaFunction{\_≡〈\_〉\_} \AgdaSymbol{:} \AgdaSymbol{∀} \AgdaSymbol{\{}\AgdaBound{V}\AgdaSymbol{\}} \AgdaSymbol{→} \AgdaFunction{Term} \AgdaBound{V} \AgdaSymbol{→} \AgdaDatatype{Type} \AgdaSymbol{→} \AgdaFunction{Term} \AgdaBound{V} \AgdaSymbol{→} \AgdaFunction{Equation} \AgdaBound{V}\<%
\\
\>\AgdaBound{M} \AgdaFunction{≡〈} \AgdaBound{A} \AgdaFunction{〉} \AgdaBound{N} \AgdaSymbol{=} \AgdaInductiveConstructor{app} \AgdaSymbol{(}\AgdaInductiveConstructor{-eq} \AgdaBound{A}\AgdaSymbol{)} \AgdaSymbol{(}\AgdaBound{M} \AgdaInductiveConstructor{∷} \AgdaBound{N} \AgdaInductiveConstructor{∷} \AgdaInductiveConstructor{[]}\AgdaSymbol{)}\<%
\\
%
\\
\>\AgdaKeyword{infixl} \AgdaNumber{59} \AgdaFixityOp{\_,T\_}\<%
\\
\>\AgdaFunction{\_,T\_} \AgdaSymbol{:} \AgdaSymbol{∀} \AgdaSymbol{\{}\AgdaBound{V}\AgdaSymbol{\}} \AgdaSymbol{→} \AgdaDatatype{Context} \AgdaBound{V} \AgdaSymbol{→} \AgdaDatatype{Type} \AgdaSymbol{→} \AgdaDatatype{Context} \AgdaSymbol{(}\AgdaBound{V} \AgdaInductiveConstructor{,} \AgdaInductiveConstructor{-Term}\AgdaSymbol{)}\<%
\\
\>\AgdaBound{Γ} \AgdaFunction{,T} \AgdaBound{A} \AgdaSymbol{=} \AgdaBound{Γ} \AgdaInductiveConstructor{,} \AgdaFunction{ty} \AgdaBound{A}\<%
\\
%
\\
\>\AgdaKeyword{infixl} \AgdaNumber{59} \AgdaFixityOp{\_,P\_}\<%
\\
\>\AgdaFunction{\_,P\_} \AgdaSymbol{:} \AgdaSymbol{∀} \AgdaSymbol{\{}\AgdaBound{V}\AgdaSymbol{\}} \AgdaSymbol{→} \AgdaDatatype{Context} \AgdaBound{V} \AgdaSymbol{→} \AgdaFunction{Term} \AgdaBound{V} \AgdaSymbol{→} \AgdaDatatype{Context} \AgdaSymbol{(}\AgdaBound{V} \AgdaInductiveConstructor{,} \AgdaInductiveConstructor{-Proof}\AgdaSymbol{)}\<%
\\
\>\AgdaFunction{\_,P\_} \AgdaSymbol{=} \AgdaInductiveConstructor{\_,\_}\<%
\\
%
\\
\>\AgdaKeyword{infixl} \AgdaNumber{59} \AgdaFixityOp{\_,E\_}\<%
\\
\>\AgdaFunction{\_,E\_} \AgdaSymbol{:} \AgdaSymbol{∀} \AgdaSymbol{\{}\AgdaBound{V}\AgdaSymbol{\}} \AgdaSymbol{→} \AgdaDatatype{Context} \AgdaBound{V} \AgdaSymbol{→} \AgdaFunction{Equation} \AgdaBound{V} \AgdaSymbol{→} \AgdaDatatype{Context} \AgdaSymbol{(}\AgdaBound{V} \AgdaInductiveConstructor{,} \AgdaInductiveConstructor{-Path}\AgdaSymbol{)}\<%
\\
\>\AgdaFunction{\_,E\_} \AgdaSymbol{=} \AgdaInductiveConstructor{\_,\_}\<%
\\
%
\\
\>\AgdaFunction{yt} \AgdaSymbol{:} \AgdaSymbol{∀} \AgdaSymbol{\{}\AgdaBound{V}\AgdaSymbol{\}} \AgdaSymbol{→} \AgdaFunction{Expression} \AgdaBound{V} \AgdaSymbol{(}\AgdaInductiveConstructor{nonVarKind} \AgdaInductiveConstructor{-Type}\AgdaSymbol{)} \AgdaSymbol{→} \AgdaDatatype{Type}\<%
\\
\>\AgdaFunction{yt} \AgdaSymbol{(}\AgdaInductiveConstructor{app} \AgdaSymbol{(}\AgdaInductiveConstructor{-ty} \AgdaBound{A}\AgdaSymbol{)} \AgdaInductiveConstructor{[]}\AgdaSymbol{)} \AgdaSymbol{=} \AgdaBound{A}\<%
\\
%
\\
\>\AgdaFunction{ty-yt} \AgdaSymbol{:} \AgdaSymbol{∀} \AgdaSymbol{\{}\AgdaBound{V}\AgdaSymbol{\}} \AgdaSymbol{\{}\AgdaBound{A} \AgdaSymbol{:} \AgdaFunction{Expression} \AgdaBound{V} \AgdaSymbol{(}\AgdaInductiveConstructor{nonVarKind} \AgdaInductiveConstructor{-Type}\AgdaSymbol{)\}} \AgdaSymbol{→} \AgdaFunction{ty} \AgdaSymbol{(}\AgdaFunction{yt} \AgdaBound{A}\AgdaSymbol{)} \AgdaDatatype{≡} \AgdaBound{A}\<%
\\
\>\AgdaFunction{ty-yt} \AgdaSymbol{\{}\AgdaArgument{A} \AgdaSymbol{=} \AgdaInductiveConstructor{app} \AgdaSymbol{(}\AgdaInductiveConstructor{-ty} \AgdaSymbol{\_)} \AgdaInductiveConstructor{[]}\AgdaSymbol{\}} \AgdaSymbol{=} \AgdaInductiveConstructor{refl} \AgdaComment{--TODO Remove?}\<%
\\
%
\\
\>\AgdaFunction{appT-injl} \AgdaSymbol{:} \AgdaSymbol{∀} \AgdaSymbol{\{}\AgdaBound{V}\AgdaSymbol{\}} \AgdaSymbol{\{}\AgdaBound{M} \AgdaBound{M'} \AgdaBound{N} \AgdaBound{N'} \AgdaSymbol{:} \AgdaFunction{Term} \AgdaBound{V}\AgdaSymbol{\}} \AgdaSymbol{→} \AgdaFunction{appT} \AgdaBound{M} \AgdaBound{N} \AgdaDatatype{≡} \AgdaFunction{appT} \AgdaBound{M'} \AgdaBound{N'} \AgdaSymbol{→} \AgdaBound{M} \AgdaDatatype{≡} \AgdaBound{M'}\<%
\\
\>\AgdaFunction{appT-injl} \AgdaInductiveConstructor{refl} \AgdaSymbol{=} \AgdaInductiveConstructor{refl}\<%
\\
%
\\
\>\AgdaFunction{Pi} \AgdaSymbol{:} \AgdaSymbol{∀} \AgdaSymbol{\{}\AgdaBound{n}\AgdaSymbol{\}} \AgdaSymbol{→} \AgdaDatatype{snocVec} \AgdaDatatype{Type} \AgdaBound{n} \AgdaSymbol{→} \AgdaDatatype{Type} \AgdaSymbol{→} \AgdaDatatype{Type}\<%
\\
\>\AgdaFunction{Pi} \AgdaInductiveConstructor{[]} \AgdaBound{B} \AgdaSymbol{=} \AgdaBound{B}\<%
\\
\>\AgdaFunction{Pi} \AgdaSymbol{(}\AgdaBound{AA} \AgdaInductiveConstructor{snoc} \AgdaBound{A}\AgdaSymbol{)} \AgdaBound{B} \AgdaSymbol{=} \AgdaFunction{Pi} \AgdaBound{AA} \AgdaSymbol{(}\AgdaBound{A} \AgdaInductiveConstructor{⇛} \AgdaBound{B}\AgdaSymbol{)}\<%
\\
%
\\
\>\AgdaFunction{APP} \AgdaSymbol{:} \AgdaSymbol{∀} \AgdaSymbol{\{}\AgdaBound{V} \AgdaBound{n}\AgdaSymbol{\}} \AgdaSymbol{→} \AgdaFunction{Term} \AgdaBound{V} \AgdaSymbol{→} \AgdaDatatype{snocVec} \AgdaSymbol{(}\AgdaFunction{Term} \AgdaBound{V}\AgdaSymbol{)} \AgdaBound{n} \AgdaSymbol{→} \AgdaFunction{Term} \AgdaBound{V}\<%
\\
\>\AgdaFunction{APP} \AgdaBound{M} \AgdaInductiveConstructor{[]} \AgdaSymbol{=} \AgdaBound{M}\<%
\\
\>\AgdaFunction{APP} \AgdaBound{M} \AgdaSymbol{(}\AgdaBound{NN} \AgdaInductiveConstructor{snoc} \AgdaBound{N}\AgdaSymbol{)} \AgdaSymbol{=} \AgdaFunction{appT} \AgdaSymbol{(}\AgdaFunction{APP} \AgdaBound{M} \AgdaBound{NN}\AgdaSymbol{)} \AgdaBound{N}\<%
\\
%
\\
\>\AgdaKeyword{postulate} \AgdaPostulate{APP-rep} \AgdaSymbol{:} \AgdaSymbol{∀} \AgdaSymbol{\{}\AgdaBound{U} \AgdaBound{V} \AgdaBound{n} \AgdaBound{M}\AgdaSymbol{\}} \AgdaSymbol{(}\AgdaBound{NN} \AgdaSymbol{:} \AgdaDatatype{snocVec} \AgdaSymbol{(}\AgdaFunction{Term} \AgdaBound{U}\AgdaSymbol{)} \AgdaBound{n}\AgdaSymbol{)} \AgdaSymbol{\{}\AgdaBound{ρ} \AgdaSymbol{:} \AgdaFunction{Rep} \AgdaBound{U} \AgdaBound{V}\AgdaSymbol{\}} \AgdaSymbol{→}\<%
\\
\>[6]\AgdaIndent{18}{}\<[18]%
\>[18]\AgdaSymbol{(}\AgdaFunction{APP} \AgdaBound{M} \AgdaBound{NN}\AgdaSymbol{)} \AgdaFunction{〈} \AgdaBound{ρ} \AgdaFunction{〉} \AgdaDatatype{≡} \AgdaFunction{APP} \AgdaSymbol{(}\AgdaBound{M} \AgdaFunction{〈} \AgdaBound{ρ} \AgdaFunction{〉}\AgdaSymbol{)} \AgdaSymbol{(}\AgdaFunction{snocVec-rep} \AgdaBound{NN} \AgdaBound{ρ}\AgdaSymbol{)}\<%
\\
%
\\
\>\AgdaFunction{APPP} \AgdaSymbol{:} \AgdaSymbol{∀} \AgdaSymbol{\{}\AgdaBound{V}\AgdaSymbol{\}} \AgdaSymbol{\{}\AgdaBound{n}\AgdaSymbol{\}} \AgdaSymbol{→} \AgdaFunction{Proof} \AgdaBound{V} \AgdaSymbol{→} \AgdaDatatype{snocVec} \AgdaSymbol{(}\AgdaFunction{Proof} \AgdaBound{V}\AgdaSymbol{)} \AgdaBound{n} \AgdaSymbol{→} \AgdaFunction{Proof} \AgdaBound{V}\<%
\\
\>\AgdaFunction{APPP} \AgdaBound{δ} \AgdaInductiveConstructor{[]} \AgdaSymbol{=} \AgdaBound{δ}\<%
\\
\>\AgdaFunction{APPP} \AgdaBound{δ} \AgdaSymbol{(}\AgdaBound{εε} \AgdaInductiveConstructor{snoc} \AgdaBound{ε}\AgdaSymbol{)} \AgdaSymbol{=} \AgdaFunction{appP} \AgdaSymbol{(}\AgdaFunction{APPP} \AgdaBound{δ} \AgdaBound{εε}\AgdaSymbol{)} \AgdaBound{ε}\<%
\\
%
\\
\>\AgdaFunction{APPP-rep} \AgdaSymbol{:} \AgdaSymbol{∀} \AgdaSymbol{\{}\AgdaBound{U} \AgdaBound{V} \AgdaBound{n} \AgdaBound{δ}\AgdaSymbol{\}} \AgdaSymbol{(}\AgdaBound{εε} \AgdaSymbol{:} \AgdaDatatype{snocVec} \AgdaSymbol{(}\AgdaFunction{Proof} \AgdaBound{U}\AgdaSymbol{)} \AgdaBound{n}\AgdaSymbol{)} \AgdaSymbol{\{}\AgdaBound{ρ} \AgdaSymbol{:} \AgdaFunction{Rep} \AgdaBound{U} \AgdaBound{V}\AgdaSymbol{\}} \AgdaSymbol{→}\<%
\\
\>[0]\AgdaIndent{2}{}\<[2]%
\>[2]\AgdaSymbol{(}\AgdaFunction{APPP} \AgdaBound{δ} \AgdaBound{εε}\AgdaSymbol{)} \AgdaFunction{〈} \AgdaBound{ρ} \AgdaFunction{〉} \AgdaDatatype{≡} \AgdaFunction{APPP} \AgdaSymbol{(}\AgdaBound{δ} \AgdaFunction{〈} \AgdaBound{ρ} \AgdaFunction{〉}\AgdaSymbol{)} \AgdaSymbol{(}\AgdaFunction{snocVec-rep} \AgdaBound{εε} \AgdaBound{ρ}\AgdaSymbol{)}\<%
\\
\>\AgdaFunction{APPP-rep} \AgdaInductiveConstructor{[]} \AgdaSymbol{=} \AgdaInductiveConstructor{refl}\<%
\\
\>\AgdaFunction{APPP-rep} \AgdaSymbol{(}\AgdaBound{εε} \AgdaInductiveConstructor{snoc} \AgdaBound{ε}\AgdaSymbol{)} \AgdaSymbol{\{}\AgdaBound{ρ}\AgdaSymbol{\}} \AgdaSymbol{=} \AgdaFunction{cong} \AgdaSymbol{(λ} \AgdaBound{x} \AgdaSymbol{→} \AgdaFunction{appP} \AgdaBound{x} \AgdaSymbol{(}\AgdaBound{ε} \AgdaFunction{〈} \AgdaBound{ρ} \AgdaFunction{〉}\AgdaSymbol{))} \AgdaSymbol{(}\AgdaFunction{APPP-rep} \AgdaBound{εε}\AgdaSymbol{)}\<%
\\
%
\\
\>\AgdaFunction{APP*} \AgdaSymbol{:} \AgdaSymbol{∀} \AgdaSymbol{\{}\AgdaBound{V} \AgdaBound{n}\AgdaSymbol{\}} \AgdaSymbol{→} \AgdaDatatype{snocVec} \AgdaSymbol{(}\AgdaFunction{Term} \AgdaBound{V}\AgdaSymbol{)} \AgdaBound{n} \AgdaSymbol{→} \AgdaDatatype{snocVec} \AgdaSymbol{(}\AgdaFunction{Term} \AgdaBound{V}\AgdaSymbol{)} \AgdaBound{n} \AgdaSymbol{→} \AgdaFunction{Path} \AgdaBound{V} \AgdaSymbol{→} \AgdaDatatype{snocVec} \AgdaSymbol{(}\AgdaFunction{Path} \AgdaBound{V}\AgdaSymbol{)} \AgdaBound{n} \AgdaSymbol{→} \AgdaFunction{Path} \AgdaBound{V}\<%
\\
\>\AgdaFunction{APP*} \AgdaInductiveConstructor{[]} \AgdaInductiveConstructor{[]} \AgdaBound{P} \AgdaInductiveConstructor{[]} \AgdaSymbol{=} \AgdaBound{P}\<%
\\
\>\AgdaFunction{APP*} \AgdaSymbol{(}\AgdaBound{MM} \AgdaInductiveConstructor{snoc} \AgdaBound{M}\AgdaSymbol{)} \AgdaSymbol{(}\AgdaBound{NN} \AgdaInductiveConstructor{snoc} \AgdaBound{N}\AgdaSymbol{)} \AgdaBound{P} \AgdaSymbol{(}\AgdaBound{QQ} \AgdaInductiveConstructor{snoc} \AgdaBound{Q}\AgdaSymbol{)} \AgdaSymbol{=} \AgdaFunction{app*} \AgdaBound{M} \AgdaBound{N} \AgdaSymbol{(}\AgdaFunction{APP*} \AgdaBound{MM} \AgdaBound{NN} \AgdaBound{P} \AgdaBound{QQ}\AgdaSymbol{)} \AgdaBound{Q}\<%
\\
%
\\
\>\AgdaFunction{APP*-rep} \AgdaSymbol{:} \AgdaSymbol{∀} \AgdaSymbol{\{}\AgdaBound{U} \AgdaBound{V} \AgdaBound{n}\AgdaSymbol{\}} \AgdaBound{MM} \AgdaSymbol{\{}\AgdaBound{NN} \AgdaSymbol{:} \AgdaDatatype{snocVec} \AgdaSymbol{(}\AgdaFunction{Term} \AgdaBound{U}\AgdaSymbol{)} \AgdaBound{n}\AgdaSymbol{\}} \AgdaSymbol{\{}\AgdaBound{P} \AgdaBound{QQ}\AgdaSymbol{\}} \AgdaSymbol{\{}\AgdaBound{ρ} \AgdaSymbol{:} \AgdaFunction{Rep} \AgdaBound{U} \AgdaBound{V}\AgdaSymbol{\}} \AgdaSymbol{→}\<%
\\
\>[0]\AgdaIndent{2}{}\<[2]%
\>[2]\AgdaSymbol{(}\AgdaFunction{APP*} \AgdaBound{MM} \AgdaBound{NN} \AgdaBound{P} \AgdaBound{QQ}\AgdaSymbol{)} \AgdaFunction{〈} \AgdaBound{ρ} \AgdaFunction{〉} \AgdaDatatype{≡} \AgdaFunction{APP*} \AgdaSymbol{(}\AgdaFunction{snocVec-rep} \AgdaBound{MM} \AgdaBound{ρ}\AgdaSymbol{)} \AgdaSymbol{(}\AgdaFunction{snocVec-rep} \AgdaBound{NN} \AgdaBound{ρ}\AgdaSymbol{)} \AgdaSymbol{(}\AgdaBound{P} \AgdaFunction{〈} \AgdaBound{ρ} \AgdaFunction{〉}\AgdaSymbol{)} \AgdaSymbol{(}\AgdaFunction{snocVec-rep} \AgdaBound{QQ} \AgdaBound{ρ}\AgdaSymbol{)}\<%
\\
\>\AgdaFunction{APP*-rep} \AgdaInductiveConstructor{[]} \AgdaSymbol{\{}\AgdaInductiveConstructor{[]}\AgdaSymbol{\}} \AgdaSymbol{\{}\AgdaArgument{QQ} \AgdaSymbol{=} \AgdaInductiveConstructor{[]}\AgdaSymbol{\}} \AgdaSymbol{=} \AgdaInductiveConstructor{refl}\<%
\\
\>\AgdaFunction{APP*-rep} \AgdaSymbol{(}\AgdaBound{MM} \AgdaInductiveConstructor{snoc} \AgdaBound{M}\AgdaSymbol{)} \AgdaSymbol{\{}\AgdaBound{NN} \AgdaInductiveConstructor{snoc} \AgdaBound{N}\AgdaSymbol{\}} \AgdaSymbol{\{}\AgdaArgument{QQ} \AgdaSymbol{=} \AgdaBound{QQ} \AgdaInductiveConstructor{snoc} \AgdaBound{Q}\AgdaSymbol{\}} \AgdaSymbol{\{}\AgdaArgument{ρ} \AgdaSymbol{=} \AgdaBound{ρ}\AgdaSymbol{\}} \AgdaSymbol{=} \<[60]%
\>[60]\<%
\\
\>[0]\AgdaIndent{2}{}\<[2]%
\>[2]\AgdaFunction{cong} \AgdaSymbol{(λ} \AgdaBound{x} \AgdaSymbol{→} \AgdaFunction{app*} \AgdaSymbol{(}\AgdaBound{M} \AgdaFunction{〈} \AgdaBound{ρ} \AgdaFunction{〉}\AgdaSymbol{)} \AgdaSymbol{(}\AgdaBound{N} \AgdaFunction{〈} \AgdaBound{ρ} \AgdaFunction{〉}\AgdaSymbol{)} \AgdaBound{x} \AgdaSymbol{(}\AgdaBound{Q} \AgdaFunction{〈} \AgdaBound{ρ} \AgdaFunction{〉}\AgdaSymbol{))} \AgdaSymbol{(}\AgdaFunction{APP*-rep} \AgdaBound{MM}\AgdaSymbol{)}\<%
\\
%
\\
\>\AgdaFunction{typeof'} \AgdaSymbol{:} \AgdaSymbol{∀} \AgdaSymbol{\{}\AgdaBound{V}\AgdaSymbol{\}} \AgdaSymbol{→} \AgdaDatatype{Var} \AgdaBound{V} \AgdaInductiveConstructor{-Term} \AgdaSymbol{→} \AgdaDatatype{Context} \AgdaBound{V} \AgdaSymbol{→} \AgdaDatatype{Type}\<%
\\
\>\AgdaFunction{typeof'} \AgdaBound{x} \AgdaBound{Γ} \<[13]%
\>[13]\AgdaSymbol{=} \AgdaFunction{yt} \AgdaSymbol{(}\AgdaFunction{typeof} \AgdaBound{x} \AgdaBound{Γ}\AgdaSymbol{)}\<%
\\
%
\\
\>\AgdaFunction{typeof-typeof'} \AgdaSymbol{:} \AgdaSymbol{∀} \AgdaSymbol{\{}\AgdaBound{V}\AgdaSymbol{\}} \AgdaSymbol{\{}\AgdaBound{x} \AgdaSymbol{:} \AgdaDatatype{Var} \AgdaBound{V} \AgdaInductiveConstructor{-Term}\AgdaSymbol{\}} \AgdaSymbol{\{}\AgdaBound{Γ}\AgdaSymbol{\}} \AgdaSymbol{→} \AgdaFunction{typeof} \AgdaBound{x} \AgdaBound{Γ} \AgdaDatatype{≡} \AgdaFunction{ty} \AgdaSymbol{(}\AgdaFunction{typeof'} \AgdaBound{x} \AgdaBound{Γ}\AgdaSymbol{)}\<%
\\
\>\AgdaFunction{typeof-typeof'} \AgdaSymbol{=} \AgdaFunction{sym} \AgdaFunction{ty-yt} \AgdaComment{-- TODO Remove?}\<%
\\
%
\\
\>\AgdaFunction{addpath} \AgdaSymbol{:} \AgdaSymbol{∀} \AgdaSymbol{\{}\AgdaBound{V}\AgdaSymbol{\}} \AgdaSymbol{→} \AgdaDatatype{Context} \AgdaBound{V} \AgdaSymbol{→} \AgdaDatatype{Type} \AgdaSymbol{→} \AgdaDatatype{Context} \AgdaSymbol{(}\AgdaBound{V} \AgdaInductiveConstructor{,} \AgdaInductiveConstructor{-Term} \AgdaInductiveConstructor{,} \AgdaInductiveConstructor{-Term} \AgdaInductiveConstructor{,} \AgdaInductiveConstructor{-Path}\AgdaSymbol{)}\<%
\\
\>\AgdaFunction{addpath} \AgdaBound{Γ} \AgdaBound{A} \AgdaSymbol{=} \AgdaBound{Γ} \AgdaFunction{,T} \AgdaBound{A} \AgdaFunction{,T} \AgdaBound{A} \AgdaFunction{,E} \AgdaInductiveConstructor{var} \AgdaFunction{x₁} \AgdaFunction{≡〈} \AgdaBound{A} \AgdaFunction{〉} \AgdaInductiveConstructor{var} \AgdaInductiveConstructor{x₀}\<%
\\
%
\\
\>\AgdaFunction{sub↖} \AgdaSymbol{:} \AgdaSymbol{∀} \AgdaSymbol{\{}\AgdaBound{U}\AgdaSymbol{\}} \AgdaSymbol{\{}\AgdaBound{V}\AgdaSymbol{\}} \AgdaSymbol{→} \AgdaFunction{Sub} \AgdaBound{U} \AgdaBound{V} \AgdaSymbol{→} \AgdaFunction{Sub} \AgdaSymbol{(}\AgdaBound{U} \AgdaInductiveConstructor{,} \AgdaInductiveConstructor{-Term}\AgdaSymbol{)} \AgdaSymbol{(}\AgdaBound{V} \AgdaInductiveConstructor{,} \AgdaInductiveConstructor{-Term} \AgdaInductiveConstructor{,} \AgdaInductiveConstructor{-Term} \AgdaInductiveConstructor{,} \AgdaInductiveConstructor{-Path}\AgdaSymbol{)}\<%
\\
\>\AgdaFunction{sub↖} \AgdaBound{σ} \AgdaSymbol{\_} \AgdaInductiveConstructor{x₀} \AgdaSymbol{=} \AgdaInductiveConstructor{var} \AgdaFunction{x₂}\<%
\\
\>\AgdaFunction{sub↖} \AgdaBound{σ} \AgdaSymbol{\_} \AgdaSymbol{(}\AgdaInductiveConstructor{↑} \AgdaBound{x}\AgdaSymbol{)} \AgdaSymbol{=} \AgdaBound{σ} \AgdaSymbol{\_} \AgdaBound{x} \AgdaFunction{⇑} \AgdaFunction{⇑} \AgdaFunction{⇑}\<%
\\
%
\\
\>\AgdaKeyword{postulate} \AgdaPostulate{sub↖-cong} \AgdaSymbol{:} \AgdaSymbol{∀} \AgdaSymbol{\{}\AgdaBound{U}\AgdaSymbol{\}} \AgdaSymbol{\{}\AgdaBound{V}\AgdaSymbol{\}} \AgdaSymbol{\{}\AgdaBound{ρ} \AgdaBound{σ} \AgdaSymbol{:} \AgdaFunction{Sub} \AgdaBound{U} \AgdaBound{V}\AgdaSymbol{\}} \AgdaSymbol{→} \AgdaBound{ρ} \AgdaFunction{∼} \AgdaBound{σ} \AgdaSymbol{→} \AgdaFunction{sub↖} \AgdaBound{ρ} \AgdaFunction{∼} \AgdaFunction{sub↖} \AgdaBound{σ}\<%
\\
%
\\
\>\AgdaKeyword{postulate} \AgdaPostulate{sub↖-comp₁} \AgdaSymbol{:} \AgdaSymbol{∀} \AgdaSymbol{\{}\AgdaBound{U}\AgdaSymbol{\}} \AgdaSymbol{\{}\AgdaBound{V}\AgdaSymbol{\}} \AgdaSymbol{\{}\AgdaBound{W}\AgdaSymbol{\}} \AgdaSymbol{\{}\AgdaBound{ρ} \AgdaSymbol{:} \AgdaFunction{Rep} \AgdaBound{V} \AgdaBound{W}\AgdaSymbol{\}} \AgdaSymbol{\{}\AgdaBound{σ} \AgdaSymbol{:} \AgdaFunction{Sub} \AgdaBound{U} \AgdaBound{V}\AgdaSymbol{\}} \AgdaSymbol{→}\<%
\\
\>[2]\AgdaIndent{21}{}\<[21]%
\>[21]\AgdaFunction{sub↖} \AgdaSymbol{(}\AgdaBound{ρ} \AgdaFunction{•RS} \AgdaBound{σ}\AgdaSymbol{)} \AgdaFunction{∼} \AgdaFunction{liftRep} \AgdaInductiveConstructor{-Path} \AgdaSymbol{(}\AgdaFunction{liftRep} \AgdaInductiveConstructor{-Term} \AgdaSymbol{(}\AgdaFunction{liftRep} \AgdaInductiveConstructor{-Term} \AgdaBound{ρ}\AgdaSymbol{))} \AgdaFunction{•RS} \AgdaFunction{sub↖} \AgdaBound{σ}\<%
\\
%
\\
\>\AgdaFunction{sub↗} \AgdaSymbol{:} \AgdaSymbol{∀} \AgdaSymbol{\{}\AgdaBound{U}\AgdaSymbol{\}} \AgdaSymbol{\{}\AgdaBound{V}\AgdaSymbol{\}} \AgdaSymbol{→} \AgdaFunction{Sub} \AgdaBound{U} \AgdaBound{V} \AgdaSymbol{→} \AgdaFunction{Sub} \AgdaSymbol{(}\AgdaBound{U} \AgdaInductiveConstructor{,} \AgdaInductiveConstructor{-Term}\AgdaSymbol{)} \AgdaSymbol{(}\AgdaBound{V} \AgdaInductiveConstructor{,} \AgdaInductiveConstructor{-Term} \AgdaInductiveConstructor{,} \AgdaInductiveConstructor{-Term} \AgdaInductiveConstructor{,} \AgdaInductiveConstructor{-Path}\AgdaSymbol{)}\<%
\\
\>\AgdaFunction{sub↗} \AgdaBound{σ} \AgdaSymbol{\_} \AgdaInductiveConstructor{x₀} \AgdaSymbol{=} \AgdaInductiveConstructor{var} \AgdaFunction{x₁}\<%
\\
\>\AgdaFunction{sub↗} \AgdaBound{σ} \AgdaSymbol{\_} \AgdaSymbol{(}\AgdaInductiveConstructor{↑} \AgdaBound{x}\AgdaSymbol{)} \AgdaSymbol{=} \AgdaBound{σ} \AgdaSymbol{\_} \AgdaBound{x} \AgdaFunction{⇑} \AgdaFunction{⇑} \AgdaFunction{⇑}\<%
\\
%
\\
\>\AgdaKeyword{postulate} \AgdaPostulate{sub↗-cong} \AgdaSymbol{:} \AgdaSymbol{∀} \AgdaSymbol{\{}\AgdaBound{U}\AgdaSymbol{\}} \AgdaSymbol{\{}\AgdaBound{V}\AgdaSymbol{\}} \AgdaSymbol{\{}\AgdaBound{ρ} \AgdaBound{σ} \AgdaSymbol{:} \AgdaFunction{Sub} \AgdaBound{U} \AgdaBound{V}\AgdaSymbol{\}} \AgdaSymbol{→} \AgdaBound{ρ} \AgdaFunction{∼} \AgdaBound{σ} \AgdaSymbol{→} \AgdaFunction{sub↗} \AgdaBound{ρ} \AgdaFunction{∼} \AgdaFunction{sub↗} \AgdaBound{σ}\<%
\\
%
\\
\>\AgdaKeyword{postulate} \AgdaPostulate{sub↗-comp₁} \AgdaSymbol{:} \AgdaSymbol{∀} \AgdaSymbol{\{}\AgdaBound{U}\AgdaSymbol{\}} \AgdaSymbol{\{}\AgdaBound{V}\AgdaSymbol{\}} \AgdaSymbol{\{}\AgdaBound{W}\AgdaSymbol{\}} \AgdaSymbol{\{}\AgdaBound{ρ} \AgdaSymbol{:} \AgdaFunction{Rep} \AgdaBound{V} \AgdaBound{W}\AgdaSymbol{\}} \AgdaSymbol{\{}\AgdaBound{σ} \AgdaSymbol{:} \AgdaFunction{Sub} \AgdaBound{U} \AgdaBound{V}\AgdaSymbol{\}} \AgdaSymbol{→}\<%
\\
\>[2]\AgdaIndent{21}{}\<[21]%
\>[21]\AgdaFunction{sub↗} \AgdaSymbol{(}\AgdaBound{ρ} \AgdaFunction{•RS} \AgdaBound{σ}\AgdaSymbol{)} \AgdaFunction{∼} \AgdaFunction{liftRep} \AgdaInductiveConstructor{-Path} \AgdaSymbol{(}\AgdaFunction{liftRep} \AgdaInductiveConstructor{-Term} \AgdaSymbol{(}\AgdaFunction{liftRep} \AgdaInductiveConstructor{-Term} \AgdaBound{ρ}\AgdaSymbol{))} \AgdaFunction{•RS} \AgdaFunction{sub↗} \AgdaBound{σ}\<%
\\
%
\\
\>\AgdaComment{--REFACTOR Duplication}\<%
\\
%
\\
\>\AgdaFunction{var-not-Λ} \AgdaSymbol{:} \AgdaSymbol{∀} \AgdaSymbol{\{}\AgdaBound{V}\AgdaSymbol{\}} \AgdaSymbol{\{}\AgdaBound{x} \AgdaSymbol{:} \AgdaDatatype{Var} \AgdaBound{V} \AgdaInductiveConstructor{-Term}\AgdaSymbol{\}} \AgdaSymbol{\{}\AgdaBound{A}\AgdaSymbol{\}} \AgdaSymbol{\{}\AgdaBound{M} \AgdaSymbol{:} \AgdaFunction{Term} \AgdaSymbol{(}\AgdaBound{V} \AgdaInductiveConstructor{,} \AgdaInductiveConstructor{-Term}\AgdaSymbol{)\}} \AgdaSymbol{→} \AgdaInductiveConstructor{var} \AgdaBound{x} \AgdaDatatype{≡} \AgdaFunction{ΛT} \AgdaBound{A} \AgdaBound{M} \AgdaSymbol{→} \AgdaDatatype{Empty}\<%
\\
\>\AgdaFunction{var-not-Λ} \AgdaSymbol{()}\<%
\\
%
\\
\>\AgdaFunction{app-not-Λ} \AgdaSymbol{:} \AgdaSymbol{∀} \AgdaSymbol{\{}\AgdaBound{V}\AgdaSymbol{\}} \AgdaSymbol{\{}\AgdaBound{M} \AgdaBound{N} \AgdaSymbol{:} \AgdaFunction{Term} \AgdaBound{V}\AgdaSymbol{\}} \AgdaSymbol{\{}\AgdaBound{A}\AgdaSymbol{\}} \AgdaSymbol{\{}\AgdaBound{P} \AgdaSymbol{:} \AgdaFunction{Term} \AgdaSymbol{(}\AgdaBound{V} \AgdaInductiveConstructor{,} \AgdaInductiveConstructor{-Term}\AgdaSymbol{)\}} \AgdaSymbol{→} \AgdaFunction{appT} \AgdaBound{M} \AgdaBound{N} \AgdaDatatype{≡} \AgdaFunction{ΛT} \AgdaBound{A} \AgdaBound{P} \AgdaSymbol{→} \AgdaDatatype{Empty}\<%
\\
\>\AgdaFunction{app-not-Λ} \AgdaSymbol{()}\<%
\\
%
\\
\>\AgdaFunction{appT-injr} \AgdaSymbol{:} \AgdaSymbol{∀} \AgdaSymbol{\{}\AgdaBound{V}\AgdaSymbol{\}} \AgdaSymbol{\{}\AgdaBound{M} \AgdaBound{N} \AgdaBound{P} \AgdaBound{Q} \AgdaSymbol{:} \AgdaFunction{Term} \AgdaBound{V}\AgdaSymbol{\}} \AgdaSymbol{→} \AgdaFunction{appT} \AgdaBound{M} \AgdaBound{N} \AgdaDatatype{≡} \AgdaFunction{appT} \AgdaBound{P} \AgdaBound{Q} \AgdaSymbol{→} \AgdaBound{N} \AgdaDatatype{≡} \AgdaBound{Q}\<%
\\
\>\AgdaFunction{appT-injr} \AgdaInductiveConstructor{refl} \AgdaSymbol{=} \AgdaInductiveConstructor{refl}\<%
\\
%
\\
\>\AgdaFunction{imp-injl} \AgdaSymbol{:} \AgdaSymbol{∀} \AgdaSymbol{\{}\AgdaBound{V}\AgdaSymbol{\}} \AgdaSymbol{\{}\AgdaBound{φ} \AgdaBound{φ'} \AgdaBound{ψ} \AgdaBound{ψ'} \AgdaSymbol{:} \AgdaFunction{Term} \AgdaBound{V}\AgdaSymbol{\}} \AgdaSymbol{→} \AgdaBound{φ} \AgdaFunction{⊃} \AgdaBound{ψ} \AgdaDatatype{≡} \AgdaBound{φ'} \AgdaFunction{⊃} \AgdaBound{ψ'} \AgdaSymbol{→} \AgdaBound{φ} \AgdaDatatype{≡} \AgdaBound{φ'}\<%
\\
\>\AgdaFunction{imp-injl} \AgdaInductiveConstructor{refl} \AgdaSymbol{=} \AgdaInductiveConstructor{refl}\<%
\\
%
\\
\>\AgdaFunction{imp-injr} \AgdaSymbol{:} \AgdaSymbol{∀} \AgdaSymbol{\{}\AgdaBound{V}\AgdaSymbol{\}} \AgdaSymbol{\{}\AgdaBound{φ} \AgdaBound{φ'} \AgdaBound{ψ} \AgdaBound{ψ'} \AgdaSymbol{:} \AgdaFunction{Term} \AgdaBound{V}\AgdaSymbol{\}} \AgdaSymbol{→} \AgdaBound{φ} \AgdaFunction{⊃} \AgdaBound{ψ} \AgdaDatatype{≡} \AgdaBound{φ'} \AgdaFunction{⊃} \AgdaBound{ψ'} \AgdaSymbol{→} \AgdaBound{ψ} \AgdaDatatype{≡} \AgdaBound{ψ'}\<%
\\
\>\AgdaFunction{imp-injr} \AgdaInductiveConstructor{refl} \AgdaSymbol{=} \AgdaInductiveConstructor{refl}\<%
\\
\>\AgdaComment{--REFACTOR General pattern}\<%
\\
%
\\
\>\AgdaFunction{toSnocTypes} \AgdaSymbol{:} \AgdaSymbol{∀} \AgdaSymbol{\{}\AgdaBound{V} \AgdaBound{n}\AgdaSymbol{\}} \AgdaSymbol{→} \AgdaDatatype{snocVec} \AgdaDatatype{Type} \AgdaBound{n} \AgdaSymbol{→} \AgdaDatatype{snocTypes} \AgdaBound{V} \AgdaSymbol{(}\AgdaFunction{replicate} \AgdaBound{n} \AgdaInductiveConstructor{-Term}\AgdaSymbol{)}\<%
\\
\>\AgdaFunction{toSnocTypes} \AgdaInductiveConstructor{[]} \AgdaSymbol{=} \AgdaInductiveConstructor{[]}\<%
\\
\>\AgdaFunction{toSnocTypes} \AgdaSymbol{(}\AgdaBound{AA} \AgdaInductiveConstructor{snoc} \AgdaBound{A}\AgdaSymbol{)} \AgdaSymbol{=} \AgdaFunction{toSnocTypes} \AgdaBound{AA} \AgdaInductiveConstructor{snoc} \AgdaFunction{ty} \AgdaBound{A}\<%
\\
%
\\
\>\AgdaFunction{toSnocTypes-rep} \AgdaSymbol{:} \AgdaSymbol{∀} \AgdaSymbol{\{}\AgdaBound{U} \AgdaBound{V} \AgdaBound{n}\AgdaSymbol{\}} \AgdaSymbol{\{}\AgdaBound{AA} \AgdaSymbol{:} \AgdaDatatype{snocVec} \AgdaDatatype{Type} \AgdaBound{n}\AgdaSymbol{\}} \AgdaSymbol{\{}\AgdaBound{ρ} \AgdaSymbol{:} \AgdaFunction{Rep} \AgdaBound{U} \AgdaBound{V}\AgdaSymbol{\}} \AgdaSymbol{→} \AgdaFunction{snocTypes-rep} \AgdaSymbol{(}\AgdaFunction{toSnocTypes} \AgdaBound{AA}\AgdaSymbol{)} \AgdaBound{ρ} \AgdaDatatype{≡} \AgdaFunction{toSnocTypes} \AgdaBound{AA}\<%
\\
\>\AgdaFunction{toSnocTypes-rep} \AgdaSymbol{\{}\AgdaArgument{AA} \AgdaSymbol{=} \AgdaInductiveConstructor{[]}\AgdaSymbol{\}} \AgdaSymbol{=} \AgdaInductiveConstructor{refl}\<%
\\
\>\AgdaFunction{toSnocTypes-rep} \AgdaSymbol{\{}\AgdaArgument{AA} \AgdaSymbol{=} \AgdaBound{AA} \AgdaInductiveConstructor{snoc} \AgdaBound{A}\AgdaSymbol{\}} \AgdaSymbol{=} \AgdaFunction{cong} \AgdaSymbol{(λ} \AgdaBound{x} \AgdaSymbol{→} \AgdaBound{x} \AgdaInductiveConstructor{snoc} \AgdaFunction{ty} \AgdaBound{A}\AgdaSymbol{)} \AgdaFunction{toSnocTypes-rep}\<%
\\
%
\\
\>\AgdaFunction{eqmult} \AgdaSymbol{:} \AgdaSymbol{∀} \AgdaSymbol{\{}\AgdaBound{V} \AgdaBound{n}\AgdaSymbol{\}} \AgdaSymbol{→} \AgdaDatatype{snocVec} \AgdaSymbol{(}\AgdaFunction{Term} \AgdaBound{V}\AgdaSymbol{)} \AgdaBound{n} \AgdaSymbol{→} \AgdaDatatype{snocVec} \AgdaDatatype{Type} \AgdaBound{n} \AgdaSymbol{→} \AgdaDatatype{snocVec} \AgdaSymbol{(}\AgdaFunction{Term} \AgdaBound{V}\AgdaSymbol{)} \AgdaBound{n} \AgdaSymbol{→} \AgdaDatatype{snocTypes} \AgdaBound{V} \AgdaSymbol{(}\AgdaFunction{Prelims.replicate} \AgdaBound{n} \AgdaInductiveConstructor{-Path}\AgdaSymbol{)}\<%
\\
\>\AgdaFunction{eqmult} \AgdaInductiveConstructor{[]} \AgdaInductiveConstructor{[]} \AgdaInductiveConstructor{[]} \AgdaSymbol{=} \AgdaInductiveConstructor{[]}\<%
\\
\>\AgdaFunction{eqmult} \AgdaSymbol{\{}\AgdaArgument{n} \AgdaSymbol{=} \AgdaInductiveConstructor{suc} \AgdaBound{n}\AgdaSymbol{\}} \AgdaSymbol{(}\AgdaBound{MM} \AgdaInductiveConstructor{snoc} \AgdaBound{M}\AgdaSymbol{)} \AgdaSymbol{(}\AgdaBound{AA} \AgdaInductiveConstructor{snoc} \AgdaBound{A}\AgdaSymbol{)} \AgdaSymbol{(}\AgdaBound{NN} \AgdaInductiveConstructor{snoc} \AgdaBound{N}\AgdaSymbol{)} \AgdaSymbol{=} \AgdaFunction{eqmult} \AgdaBound{MM} \AgdaBound{AA} \AgdaBound{NN} \AgdaInductiveConstructor{snoc} \AgdaSymbol{(}\AgdaFunction{\_⇑⇑} \AgdaSymbol{\{}\AgdaArgument{KK} \AgdaSymbol{=} \AgdaFunction{Prelims.replicate} \AgdaBound{n} \AgdaInductiveConstructor{-Path}\AgdaSymbol{\}} \AgdaBound{M}\AgdaSymbol{)} \AgdaFunction{≡〈} \AgdaBound{A} \AgdaFunction{〉} \AgdaSymbol{(}\AgdaFunction{\_⇑⇑} \AgdaSymbol{\{}\AgdaArgument{KK} \AgdaSymbol{=} \AgdaFunction{Prelims.replicate} \AgdaBound{n} \AgdaInductiveConstructor{-Path}\AgdaSymbol{\}} \AgdaBound{N}\AgdaSymbol{)}\<%
\\
%
\\
\>\AgdaFunction{eqmult-rep} \AgdaSymbol{:} \AgdaSymbol{∀} \AgdaSymbol{\{}\AgdaBound{U} \AgdaBound{V} \AgdaBound{n}\AgdaSymbol{\}} \AgdaSymbol{\{}\AgdaBound{MM} \AgdaSymbol{:} \AgdaDatatype{snocVec} \AgdaSymbol{(}\AgdaFunction{Term} \AgdaBound{U}\AgdaSymbol{)} \AgdaBound{n}\AgdaSymbol{\}} \AgdaSymbol{\{}\AgdaBound{AA} \AgdaBound{NN}\AgdaSymbol{\}} \AgdaSymbol{\{}\AgdaBound{ρ} \AgdaSymbol{:} \AgdaFunction{Rep} \AgdaBound{U} \AgdaBound{V}\AgdaSymbol{\}} \AgdaSymbol{→}\<%
\\
\>[0]\AgdaIndent{2}{}\<[2]%
\>[2]\AgdaFunction{snocTypes-rep} \AgdaSymbol{(}\AgdaFunction{eqmult} \AgdaBound{MM} \AgdaBound{AA} \AgdaBound{NN}\AgdaSymbol{)} \AgdaBound{ρ} \AgdaDatatype{≡} \AgdaFunction{eqmult} \AgdaSymbol{(}\AgdaFunction{snocVec-rep} \AgdaBound{MM} \AgdaBound{ρ}\AgdaSymbol{)} \AgdaBound{AA} \AgdaSymbol{(}\AgdaFunction{snocVec-rep} \AgdaBound{NN} \AgdaBound{ρ}\AgdaSymbol{)}\<%
\\
\>\AgdaFunction{eqmult-rep} \AgdaSymbol{\{}\AgdaArgument{MM} \AgdaSymbol{=} \AgdaInductiveConstructor{[]}\AgdaSymbol{\}} \AgdaSymbol{\{}\AgdaInductiveConstructor{[]}\AgdaSymbol{\}} \AgdaSymbol{\{}\AgdaInductiveConstructor{[]}\AgdaSymbol{\}} \AgdaSymbol{=} \AgdaInductiveConstructor{refl}\<%
\\
\>\AgdaFunction{eqmult-rep} \AgdaSymbol{\{}\AgdaArgument{n} \AgdaSymbol{=} \AgdaInductiveConstructor{suc} \AgdaBound{n}\AgdaSymbol{\}} \AgdaSymbol{\{}\AgdaArgument{MM} \AgdaSymbol{=} \AgdaBound{MM} \AgdaInductiveConstructor{snoc} \AgdaBound{M}\AgdaSymbol{\}} \AgdaSymbol{\{}\AgdaBound{AA} \AgdaInductiveConstructor{snoc} \AgdaBound{A}\AgdaSymbol{\}} \AgdaSymbol{\{}\AgdaBound{NN} \AgdaInductiveConstructor{snoc} \AgdaBound{N}\AgdaSymbol{\}} \AgdaSymbol{=} \AgdaFunction{cong₃} \AgdaSymbol{(λ} \AgdaBound{a} \AgdaBound{b} \AgdaBound{c} \AgdaSymbol{→} \AgdaBound{a} \AgdaInductiveConstructor{snoc} \AgdaBound{b} \AgdaFunction{≡〈} \AgdaBound{A} \AgdaFunction{〉} \AgdaBound{c}\AgdaSymbol{)} \<[102]%
\>[102]\<%
\\
\>[0]\AgdaIndent{2}{}\<[2]%
\>[2]\AgdaFunction{eqmult-rep} \<[13]%
\>[13]\<%
\\
\>[0]\AgdaIndent{2}{}\<[2]%
\>[2]\AgdaSymbol{(}\AgdaFunction{liftsnocRep-ups} \AgdaSymbol{(}\AgdaFunction{Prelims.replicate} \AgdaBound{n} \AgdaInductiveConstructor{-Path}\AgdaSymbol{)} \AgdaBound{M}\AgdaSymbol{)} \AgdaSymbol{(}\AgdaFunction{liftsnocRep-ups} \AgdaSymbol{(}\AgdaFunction{Prelims.replicate} \AgdaBound{n} \AgdaInductiveConstructor{-Path}\AgdaSymbol{)} \AgdaBound{N}\AgdaSymbol{)}\<%
\\
%
\\
\>\AgdaFunction{toSnocListExp} \AgdaSymbol{:} \AgdaSymbol{∀} \AgdaSymbol{\{}\AgdaBound{V} \AgdaBound{K} \AgdaBound{n}\AgdaSymbol{\}} \AgdaSymbol{→} \AgdaDatatype{snocVec} \AgdaSymbol{(}\AgdaFunction{Expression} \AgdaBound{V} \AgdaSymbol{(}\AgdaInductiveConstructor{varKind} \AgdaBound{K}\AgdaSymbol{))} \AgdaBound{n} \AgdaSymbol{→} \AgdaDatatype{snocListExp} \AgdaBound{V} \AgdaSymbol{(}\AgdaFunction{replicate} \AgdaBound{n} \AgdaBound{K}\AgdaSymbol{)}\<%
\\
\>\AgdaFunction{toSnocListExp} \AgdaInductiveConstructor{[]} \AgdaSymbol{=} \AgdaInductiveConstructor{[]}\<%
\\
\>\AgdaFunction{toSnocListExp} \AgdaSymbol{(}\AgdaBound{MM} \AgdaInductiveConstructor{snoc} \AgdaBound{M}\AgdaSymbol{)} \AgdaSymbol{=} \AgdaFunction{toSnocListExp} \AgdaBound{MM} \AgdaInductiveConstructor{snoc} \AgdaBound{M}\<%
\\
%
\\
\>\AgdaFunction{toSnocListExp-rep} \AgdaSymbol{:} \AgdaSymbol{∀} \AgdaSymbol{\{}\AgdaBound{U} \AgdaBound{V} \AgdaBound{K} \AgdaBound{n}\AgdaSymbol{\}} \AgdaSymbol{\{}\AgdaBound{MM} \AgdaSymbol{:} \AgdaDatatype{snocVec} \AgdaSymbol{(}\AgdaFunction{Expression} \AgdaBound{U} \AgdaSymbol{(}\AgdaInductiveConstructor{varKind} \AgdaBound{K}\AgdaSymbol{))} \AgdaBound{n}\AgdaSymbol{\}} \AgdaSymbol{\{}\AgdaBound{ρ} \AgdaSymbol{:} \AgdaFunction{Rep} \AgdaBound{U} \AgdaBound{V}\AgdaSymbol{\}} \AgdaSymbol{→}\<%
\\
\>[0]\AgdaIndent{2}{}\<[2]%
\>[2]\AgdaFunction{snocListExp-rep} \AgdaSymbol{(}\AgdaFunction{toSnocListExp} \AgdaBound{MM}\AgdaSymbol{)} \AgdaBound{ρ} \AgdaDatatype{≡} \AgdaFunction{toSnocListExp} \AgdaSymbol{(}\AgdaFunction{snocVec-rep} \AgdaBound{MM} \AgdaBound{ρ}\AgdaSymbol{)}\<%
\\
\>\AgdaFunction{toSnocListExp-rep} \AgdaSymbol{\{}\AgdaArgument{MM} \AgdaSymbol{=} \AgdaInductiveConstructor{[]}\AgdaSymbol{\}} \AgdaSymbol{=} \AgdaInductiveConstructor{refl}\<%
\\
\>\AgdaFunction{toSnocListExp-rep} \AgdaSymbol{\{}\AgdaArgument{MM} \AgdaSymbol{=} \AgdaBound{MM} \AgdaInductiveConstructor{snoc} \AgdaBound{M}\AgdaSymbol{\}} \AgdaSymbol{\{}\AgdaBound{ρ}\AgdaSymbol{\}} \AgdaSymbol{=} \AgdaFunction{cong} \AgdaSymbol{(λ} \AgdaBound{x} \AgdaSymbol{→} \AgdaBound{x} \AgdaInductiveConstructor{snoc} \AgdaBound{M} \AgdaFunction{〈} \AgdaBound{ρ} \AgdaFunction{〉}\AgdaSymbol{)} \AgdaFunction{toSnocListExp-rep}\<%
\end{code}
}

\paragraph{Substitution}

We write $t[z:=s]$ for the result of substituting $s$ for $z$ in $t$,
renaming bound variables to avoid capture.  We write $s[z_1 := t_1, \ldots, z_n := t_n]$
or $s[\vec{z} := \vec{t}]$ for the result of simultaneously substituting
each $t_i$ for $z_i$ in $s$.

A \emph{substitution} $\sigma$ is a function whose domain is a finite set of variables, and
which maps term variables to terms, proof variables to proofs, and path variables to paths.
Given a substitution $\sigma$ and an expression $t$, we write $t[\sigma]$ for the result
of simultaneously substituting $\sigma(z)$ for $z$ within $t$, for each variable $z$ in the domain of $\sigma$.

Given two substitutions $\sigma$ and $\rho$, we define their \emph{composition} $\sigma \circ \rho$ to
be the substitution with the same domain an $\rho$, such that
\[ (\sigma \circ \rho)(x) \eqdef \rho(x)[\sigma] \enspace . \]
An easy induction on $t$ shows that we have $t [\sigma \circ \rho] \equiv t [ \rho ] [ \sigma ]$.

\mode<all>{\AgdaHide{
\begin{code}%
\>\AgdaKeyword{module} \AgdaModule{PL.Rules} \AgdaKeyword{where}\<%
\\
\>\AgdaKeyword{open} \AgdaKeyword{import} \AgdaModule{Data.Empty}\<%
\\
\>\AgdaKeyword{open} \AgdaKeyword{import} \AgdaModule{Prelims}\<%
\\
\>\AgdaKeyword{open} \AgdaKeyword{import} \AgdaModule{PL.Grammar}\<%
\\
\>\AgdaKeyword{open} \AgdaModule{PLgrammar}\<%
\\
\>\AgdaKeyword{open} \AgdaKeyword{import} \AgdaModule{Grammar} \AgdaFunction{Propositional-Logic}\<%
\\
\>\AgdaKeyword{open} \AgdaKeyword{import} \AgdaModule{Reduction} \AgdaFunction{Propositional-Logic} \AgdaDatatype{β}\<%
\end{code}
}

\subsection{Rules of Deduction}

The rules of deduction of the system are as follows.

\[ \infer[(p : \phi \in \Gamma)]{\Gamma \vdash p : \phi}{\Gamma \vald} \]

\[ \infer{\Gamma \vdash \delta \epsilon : \psi}{\Gamma \vdash \delta : \phi \rightarrow \psi}{\Gamma \vdash \epsilon : \phi} \]

\[ \infer{\Gamma \vdash \lambda p : \phi . \delta : \phi \rightarrow \psi}{\Gamma, p : \phi \vdash \delta : \psi} \]

\begin{code}%
\>\AgdaKeyword{infix} \AgdaNumber{10} \AgdaFixityOp{\_⊢\_∶\_}\<%
\\
\>\AgdaKeyword{data} \AgdaDatatype{\_⊢\_∶\_} \AgdaSymbol{:} \AgdaSymbol{∀} \AgdaSymbol{\{}\AgdaBound{P}\AgdaSymbol{\}} \AgdaSymbol{→} \AgdaDatatype{Context} \AgdaBound{P} \AgdaSymbol{→} \AgdaFunction{Proof} \AgdaBound{P} \AgdaSymbol{→} \AgdaDatatype{Prop} \AgdaSymbol{→} \AgdaPrimitiveType{Set} \AgdaKeyword{where}\<%
\\
\>[0]\AgdaIndent{2}{}\<[2]%
\>[2]\AgdaInductiveConstructor{var} \AgdaSymbol{:} \AgdaSymbol{∀} \AgdaSymbol{\{}\AgdaBound{P}\AgdaSymbol{\}} \AgdaSymbol{\{}\AgdaBound{Γ} \AgdaSymbol{:} \AgdaDatatype{Context} \AgdaBound{P}\AgdaSymbol{\}} \AgdaSymbol{(}\AgdaBound{p} \AgdaSymbol{:} \AgdaDatatype{Var} \AgdaBound{P} \AgdaInductiveConstructor{-proof}\AgdaSymbol{)} \AgdaSymbol{→} \<[51]%
\>[51]\<%
\\
\>[2]\AgdaIndent{4}{}\<[4]%
\>[4]\AgdaBound{Γ} \AgdaDatatype{⊢} \AgdaInductiveConstructor{var} \AgdaBound{p} \AgdaDatatype{∶} \AgdaFunction{unprp} \AgdaSymbol{(}\AgdaFunction{typeof} \AgdaBound{p} \AgdaBound{Γ}\AgdaSymbol{)}\<%
\\
\>[0]\AgdaIndent{2}{}\<[2]%
\>[2]\AgdaInductiveConstructor{app} \AgdaSymbol{:} \AgdaSymbol{∀} \AgdaSymbol{\{}\AgdaBound{P}\AgdaSymbol{\}} \AgdaSymbol{\{}\AgdaBound{Γ} \AgdaSymbol{:} \AgdaDatatype{Context} \AgdaBound{P}\AgdaSymbol{\}} \AgdaSymbol{\{}\AgdaBound{δ}\AgdaSymbol{\}} \AgdaSymbol{\{}\AgdaBound{ε}\AgdaSymbol{\}} \AgdaSymbol{\{}\AgdaBound{φ}\AgdaSymbol{\}} \AgdaSymbol{\{}\AgdaBound{ψ}\AgdaSymbol{\}} \AgdaSymbol{→} \<[48]%
\>[48]\<%
\\
\>[2]\AgdaIndent{4}{}\<[4]%
\>[4]\AgdaBound{Γ} \AgdaDatatype{⊢} \AgdaBound{δ} \AgdaDatatype{∶} \AgdaBound{φ} \AgdaInductiveConstructor{⇛} \AgdaBound{ψ} \AgdaSymbol{→} \AgdaBound{Γ} \AgdaDatatype{⊢} \AgdaBound{ε} \AgdaDatatype{∶} \AgdaBound{φ} \AgdaSymbol{→} \AgdaBound{Γ} \AgdaDatatype{⊢} \AgdaFunction{appP} \AgdaBound{δ} \AgdaBound{ε} \AgdaDatatype{∶} \AgdaBound{ψ}\<%
\\
\>[0]\AgdaIndent{2}{}\<[2]%
\>[2]\AgdaInductiveConstructor{Λ} \AgdaSymbol{:} \AgdaSymbol{∀} \AgdaSymbol{\{}\AgdaBound{P}\AgdaSymbol{\}} \AgdaSymbol{\{}\AgdaBound{Γ} \AgdaSymbol{:} \AgdaDatatype{Context} \AgdaBound{P}\AgdaSymbol{\}} \AgdaSymbol{\{}\AgdaBound{φ}\AgdaSymbol{\}} \AgdaSymbol{\{}\AgdaBound{δ}\AgdaSymbol{\}} \AgdaSymbol{\{}\AgdaBound{ψ}\AgdaSymbol{\}} \AgdaSymbol{→} \<[42]%
\>[42]\<%
\\
\>[2]\AgdaIndent{4}{}\<[4]%
\>[4]\AgdaBound{Γ} \AgdaFunction{,P} \AgdaBound{φ} \AgdaDatatype{⊢} \AgdaBound{δ} \AgdaDatatype{∶} \AgdaBound{ψ} \AgdaSymbol{→} \AgdaBound{Γ} \AgdaDatatype{⊢} \AgdaFunction{ΛP} \AgdaBound{φ} \AgdaBound{δ} \AgdaDatatype{∶} \AgdaBound{φ} \AgdaInductiveConstructor{⇛} \AgdaBound{ψ}\<%
\end{code}

\AgdaHide{
\begin{code}%
\>\AgdaFunction{change-type} \AgdaSymbol{:} \AgdaSymbol{∀} \AgdaSymbol{\{}\AgdaBound{P}\AgdaSymbol{\}} \AgdaSymbol{\{}\AgdaBound{Γ} \AgdaSymbol{:} \AgdaDatatype{Context} \AgdaBound{P}\AgdaSymbol{\}} \AgdaSymbol{\{}\AgdaBound{δ} \AgdaBound{φ} \AgdaBound{ψ}\AgdaSymbol{\}} \AgdaSymbol{→}\<%
\\
\>[0]\AgdaIndent{2}{}\<[2]%
\>[2]\AgdaBound{φ} \AgdaDatatype{≡} \AgdaBound{ψ} \AgdaSymbol{→} \AgdaBound{Γ} \AgdaDatatype{⊢} \AgdaBound{δ} \AgdaDatatype{∶} \AgdaBound{φ} \AgdaSymbol{→} \AgdaBound{Γ} \AgdaDatatype{⊢} \AgdaBound{δ} \AgdaDatatype{∶} \AgdaBound{ψ}\<%
\\
\>\AgdaFunction{change-type} \AgdaSymbol{=} \AgdaFunction{subst} \AgdaSymbol{(λ} \AgdaBound{A} \AgdaSymbol{→} \AgdaSymbol{\_} \AgdaDatatype{⊢} \AgdaSymbol{\_} \AgdaDatatype{∶} \AgdaBound{A}\AgdaSymbol{)}\<%
\end{code}
}

Let $\rho$ be a replacement.  We say $\rho$ is a replacement from $\Gamma$ to $\Delta$, $\rho : \Gamma \rightarrow \Delta$,
iff for all $x : \phi \in \Gamma$ we have $\rho(x) : \phi \in \Delta$.

\begin{code}%
\>\AgdaFunction{\_∶\_⇒R\_} \AgdaSymbol{:} \AgdaSymbol{∀} \AgdaSymbol{\{}\AgdaBound{P}\AgdaSymbol{\}} \AgdaSymbol{\{}\AgdaBound{Q}\AgdaSymbol{\}} \AgdaSymbol{→} \AgdaFunction{Rep} \AgdaBound{P} \AgdaBound{Q} \AgdaSymbol{→} \AgdaDatatype{Context} \AgdaBound{P} \AgdaSymbol{→} \AgdaDatatype{Context} \AgdaBound{Q} \AgdaSymbol{→} \AgdaPrimitiveType{Set}\<%
\\
\>\AgdaBound{ρ} \AgdaFunction{∶} \AgdaBound{Γ} \AgdaFunction{⇒R} \AgdaBound{Δ} \AgdaSymbol{=} \AgdaSymbol{∀} \AgdaBound{x} \AgdaSymbol{→} \AgdaFunction{unprp} \AgdaSymbol{(}\AgdaFunction{typeof} \AgdaSymbol{\{}\AgdaArgument{K} \AgdaSymbol{=} \AgdaInductiveConstructor{-proof}\AgdaSymbol{\}} \AgdaSymbol{(}\AgdaBound{ρ} \AgdaSymbol{\_} \AgdaBound{x}\AgdaSymbol{)} \AgdaBound{Δ}\AgdaSymbol{)} \AgdaDatatype{≡} \AgdaFunction{unprp} \AgdaSymbol{(}\AgdaFunction{typeof} \AgdaBound{x} \AgdaBound{Γ} \AgdaSymbol{)}\<%
\end{code}

\begin{lemma}$ $
\begin{enumerate}
\item
$\id{P}$ is a replacement $\Gamma \rightarrow \Gamma$.
\item
$\uparrow$ is a replacement $\Gamma \rightarrow \Gamma , \phi$.
\item
If $\rho : \Gamma \rightarrow \Delta$ then $(\rho , \mathrm{Proof}) : (\Gamma , x : \phi) \rightarrow (\Delta , x : \phi)$.
\item
If $\rho : \Gamma \rightarrow \Delta$ and $\sigma : \Delta \rightarrow \Theta$ then $\sigma \circ \rho : \Gamma \rightarrow \Delta$.
\item
(\textbf{Weakening})
If $\rho : \Gamma \rightarrow \Delta$ and $\Gamma \vdash \delta : \phi$ then $\Delta \vdash \delta \langle \rho \rangle : \phi$.
\end{enumerate}
\end{lemma}

\begin{code}%
\>\AgdaFunction{idRep-typed} \AgdaSymbol{:} \AgdaSymbol{∀} \AgdaSymbol{\{}\AgdaBound{P}\AgdaSymbol{\}} \AgdaSymbol{\{}\AgdaBound{Γ} \AgdaSymbol{:} \AgdaDatatype{Context} \AgdaBound{P}\AgdaSymbol{\}} \AgdaSymbol{→} \AgdaFunction{idRep} \AgdaBound{P} \AgdaFunction{∶} \AgdaBound{Γ} \AgdaFunction{⇒R} \AgdaBound{Γ}\<%
\end{code}

\AgdaHide{
\begin{code}%
\>\AgdaFunction{idRep-typed} \AgdaSymbol{\{}\AgdaBound{P}\AgdaSymbol{\}} \AgdaSymbol{\{}\AgdaBound{Γ}\AgdaSymbol{\}} \AgdaBound{x} \AgdaSymbol{=} \AgdaInductiveConstructor{refl}\<%
\end{code}
}

\begin{code}%
\>\AgdaFunction{unprp-rep} \AgdaSymbol{:} \AgdaSymbol{∀} \AgdaSymbol{\{}\AgdaBound{U} \AgdaBound{V}\AgdaSymbol{\}} \AgdaBound{φ} \AgdaSymbol{(}\AgdaBound{ρ} \AgdaSymbol{:} \AgdaFunction{Rep} \AgdaBound{U} \AgdaBound{V}\AgdaSymbol{)} \AgdaSymbol{→} \AgdaFunction{unprp} \AgdaSymbol{(}\AgdaBound{φ} \AgdaFunction{〈} \AgdaBound{ρ} \AgdaFunction{〉}\AgdaSymbol{)} \AgdaDatatype{≡} \AgdaFunction{unprp} \AgdaBound{φ}\<%
\\
\>\AgdaFunction{unprp-rep} \AgdaSymbol{(}\AgdaInductiveConstructor{app} \AgdaSymbol{(}\AgdaInductiveConstructor{-prp} \AgdaSymbol{\_)} \AgdaInductiveConstructor{[]}\AgdaSymbol{)} \AgdaSymbol{\_} \AgdaSymbol{=} \AgdaInductiveConstructor{refl}\<%
\\
%
\\
\>\AgdaFunction{↑-typed} \AgdaSymbol{:} \AgdaSymbol{∀} \AgdaSymbol{\{}\AgdaBound{P}\AgdaSymbol{\}} \AgdaSymbol{\{}\AgdaBound{Γ} \AgdaSymbol{:} \AgdaDatatype{Context} \AgdaBound{P}\AgdaSymbol{\}} \AgdaSymbol{\{}\AgdaBound{φ} \AgdaSymbol{:} \AgdaDatatype{Prop}\AgdaSymbol{\}} \AgdaSymbol{→} \AgdaFunction{upRep} \AgdaFunction{∶} \AgdaBound{Γ} \AgdaFunction{⇒R} \AgdaSymbol{(}\AgdaBound{Γ} \AgdaFunction{,P} \AgdaBound{φ}\AgdaSymbol{)}\<%
\end{code}

\AgdaHide{
\begin{code}%
\>\AgdaFunction{↑-typed} \AgdaSymbol{\{}\AgdaBound{P}\AgdaSymbol{\}} \AgdaSymbol{\{}\AgdaBound{Γ}\AgdaSymbol{\}} \AgdaSymbol{\{}\AgdaBound{φ}\AgdaSymbol{\}} \AgdaBound{x} \AgdaSymbol{=} \AgdaFunction{unprp-rep} \AgdaSymbol{(}\AgdaFunction{typeof} \AgdaBound{x} \AgdaBound{Γ}\AgdaSymbol{)} \AgdaFunction{upRep}\<%
\end{code}
}

\begin{code}%
\>\AgdaFunction{liftRep-typed} \AgdaSymbol{:} \AgdaSymbol{∀} \AgdaSymbol{\{}\AgdaBound{P}\AgdaSymbol{\}} \AgdaSymbol{\{}\AgdaBound{Q}\AgdaSymbol{\}} \AgdaSymbol{\{}\AgdaBound{ρ}\AgdaSymbol{\}} \AgdaSymbol{\{}\AgdaBound{Γ} \AgdaSymbol{:} \AgdaDatatype{Context} \AgdaBound{P}\AgdaSymbol{\}} \AgdaSymbol{\{}\AgdaBound{Δ} \AgdaSymbol{:} \AgdaDatatype{Context} \AgdaBound{Q}\AgdaSymbol{\}} \AgdaSymbol{\{}\AgdaBound{φ} \AgdaSymbol{:} \AgdaDatatype{Prop}\AgdaSymbol{\}} \AgdaSymbol{→} \<[75]%
\>[75]\<%
\\
\>[0]\AgdaIndent{2}{}\<[2]%
\>[2]\AgdaBound{ρ} \AgdaFunction{∶} \AgdaBound{Γ} \AgdaFunction{⇒R} \AgdaBound{Δ} \AgdaSymbol{→} \AgdaFunction{liftRep} \AgdaInductiveConstructor{-proof} \AgdaBound{ρ} \AgdaFunction{∶} \AgdaSymbol{(}\AgdaBound{Γ} \AgdaFunction{,P} \AgdaBound{φ}\AgdaSymbol{)} \AgdaFunction{⇒R} \AgdaSymbol{(}\AgdaBound{Δ} \AgdaFunction{,P} \AgdaBound{φ}\AgdaSymbol{)}\<%
\end{code}

\AgdaHide{
\begin{code}%
\>\AgdaFunction{liftRep-typed} \AgdaSymbol{\{}\AgdaBound{P}\AgdaSymbol{\}} \AgdaSymbol{\{}\AgdaArgument{Q} \AgdaSymbol{=} \AgdaBound{Q}\AgdaSymbol{\}} \AgdaSymbol{\{}\AgdaArgument{ρ} \AgdaSymbol{=} \AgdaBound{ρ}\AgdaSymbol{\}} \AgdaSymbol{\{}\AgdaBound{Γ}\AgdaSymbol{\}} \AgdaSymbol{\{}\AgdaArgument{Δ} \AgdaSymbol{=} \AgdaBound{Δ}\AgdaSymbol{\}} \AgdaSymbol{\{}\AgdaArgument{φ} \AgdaSymbol{=} \AgdaBound{φ}\AgdaSymbol{\}} \AgdaBound{ρ∶Γ→Δ} \AgdaInductiveConstructor{x₀} \AgdaSymbol{=} \AgdaInductiveConstructor{refl}\<%
\\
\>\AgdaFunction{liftRep-typed} \AgdaSymbol{\{}\AgdaArgument{Q} \AgdaSymbol{=} \AgdaBound{Q}\AgdaSymbol{\}} \AgdaSymbol{\{}\AgdaArgument{ρ} \AgdaSymbol{=} \AgdaBound{ρ}\AgdaSymbol{\}} \AgdaSymbol{\{}\AgdaArgument{Γ} \AgdaSymbol{=} \AgdaBound{Γ}\AgdaSymbol{\}} \AgdaSymbol{\{}\AgdaArgument{Δ} \AgdaSymbol{=} \AgdaBound{Δ}\AgdaSymbol{\}} \AgdaSymbol{\{}\AgdaBound{φ}\AgdaSymbol{\}} \AgdaBound{ρ∶Γ→Δ} \AgdaSymbol{(}\AgdaInductiveConstructor{↑} \AgdaBound{x}\AgdaSymbol{)} \AgdaSymbol{=} \<[64]%
\>[64]\<%
\\
\>[0]\AgdaIndent{2}{}\<[2]%
\>[2]\AgdaKeyword{let} \AgdaKeyword{open} \AgdaModule{≡-Reasoning} \AgdaKeyword{in} \<[26]%
\>[26]\<%
\\
\>[0]\AgdaIndent{2}{}\<[2]%
\>[2]\AgdaFunction{begin}\<%
\\
\>[2]\AgdaIndent{4}{}\<[4]%
\>[4]\AgdaFunction{unprp} \AgdaSymbol{(}\AgdaFunction{typeof} \AgdaSymbol{(}\AgdaFunction{liftRep} \AgdaInductiveConstructor{-proof} \AgdaBound{ρ} \AgdaInductiveConstructor{-proof} \AgdaSymbol{(}\AgdaInductiveConstructor{↑} \AgdaBound{x}\AgdaSymbol{))} \AgdaSymbol{(}\AgdaBound{Δ} \AgdaFunction{,P} \AgdaBound{φ}\AgdaSymbol{))}\<%
\\
\>[0]\AgdaIndent{2}{}\<[2]%
\>[2]\AgdaFunction{≡⟨⟩}\<%
\\
\>[2]\AgdaIndent{4}{}\<[4]%
\>[4]\AgdaFunction{unprp} \AgdaSymbol{(}\AgdaFunction{typeof} \AgdaSymbol{(}\AgdaInductiveConstructor{↑} \AgdaSymbol{(}\AgdaBound{ρ} \AgdaInductiveConstructor{-proof} \AgdaBound{x}\AgdaSymbol{))} \AgdaSymbol{(}\AgdaBound{Δ} \AgdaFunction{,P} \AgdaBound{φ}\AgdaSymbol{))}\<%
\\
\>[0]\AgdaIndent{2}{}\<[2]%
\>[2]\AgdaFunction{≡⟨⟩}\<%
\\
\>[2]\AgdaIndent{4}{}\<[4]%
\>[4]\AgdaFunction{unprp} \AgdaSymbol{(}\AgdaFunction{typeof} \AgdaSymbol{(}\AgdaBound{ρ} \AgdaInductiveConstructor{-proof} \AgdaBound{x}\AgdaSymbol{)} \AgdaBound{Δ} \AgdaFunction{〈} \AgdaFunction{upRep} \AgdaFunction{〉}\AgdaSymbol{)}\<%
\\
\>[0]\AgdaIndent{2}{}\<[2]%
\>[2]\AgdaFunction{≡⟨} \AgdaFunction{unprp-rep} \AgdaSymbol{(}\AgdaFunction{typeof} \AgdaSymbol{(}\AgdaBound{ρ} \AgdaInductiveConstructor{-proof} \AgdaBound{x}\AgdaSymbol{)} \AgdaBound{Δ}\AgdaSymbol{)} \AgdaFunction{upRep} \AgdaFunction{⟩}\<%
\\
\>[2]\AgdaIndent{4}{}\<[4]%
\>[4]\AgdaFunction{unprp} \AgdaSymbol{(}\AgdaFunction{typeof} \AgdaSymbol{(}\AgdaBound{ρ} \AgdaInductiveConstructor{-proof} \AgdaBound{x}\AgdaSymbol{)} \AgdaBound{Δ}\AgdaSymbol{)}\<%
\\
\>[0]\AgdaIndent{2}{}\<[2]%
\>[2]\AgdaFunction{≡⟨} \AgdaBound{ρ∶Γ→Δ} \AgdaBound{x} \AgdaFunction{⟩}\<%
\\
\>[2]\AgdaIndent{4}{}\<[4]%
\>[4]\AgdaFunction{unprp} \AgdaSymbol{(}\AgdaFunction{typeof} \AgdaBound{x} \AgdaBound{Γ}\AgdaSymbol{)}\<%
\\
\>[0]\AgdaIndent{2}{}\<[2]%
\>[2]\AgdaFunction{≡⟨⟨} \AgdaFunction{unprp-rep} \AgdaSymbol{(}\AgdaFunction{typeof} \AgdaBound{x} \AgdaBound{Γ}\AgdaSymbol{)} \AgdaFunction{upRep} \AgdaFunction{⟩⟩}\<%
\\
\>[2]\AgdaIndent{4}{}\<[4]%
\>[4]\AgdaFunction{unprp} \AgdaSymbol{(}\AgdaFunction{typeof} \AgdaBound{x} \AgdaBound{Γ} \AgdaFunction{〈} \AgdaFunction{upRep} \AgdaFunction{〉}\AgdaSymbol{)}\<%
\\
\>[0]\AgdaIndent{2}{}\<[2]%
\>[2]\AgdaFunction{≡⟨⟩}\<%
\\
\>[2]\AgdaIndent{4}{}\<[4]%
\>[4]\AgdaFunction{unprp} \AgdaSymbol{(}\AgdaFunction{typeof} \AgdaSymbol{(}\AgdaInductiveConstructor{↑} \AgdaBound{x}\AgdaSymbol{)} \AgdaSymbol{(}\AgdaBound{Γ} \AgdaFunction{,P} \AgdaBound{φ}\AgdaSymbol{))}\<%
\\
\>[0]\AgdaIndent{2}{}\<[2]%
\>[2]\AgdaFunction{∎}\<%
\end{code}
}

\begin{code}%
\>\AgdaFunction{•R-typed} \AgdaSymbol{:} \AgdaSymbol{∀} \AgdaSymbol{\{}\AgdaBound{P}\AgdaSymbol{\}} \AgdaSymbol{\{}\AgdaBound{Q}\AgdaSymbol{\}} \AgdaSymbol{\{}\AgdaBound{R}\AgdaSymbol{\}} \AgdaSymbol{\{}\AgdaBound{σ} \AgdaSymbol{:} \AgdaFunction{Rep} \AgdaBound{Q} \AgdaBound{R}\AgdaSymbol{\}} \AgdaSymbol{\{}\AgdaBound{ρ} \AgdaSymbol{:} \AgdaFunction{Rep} \AgdaBound{P} \AgdaBound{Q}\AgdaSymbol{\}} \AgdaSymbol{\{}\AgdaBound{Γ}\AgdaSymbol{\}} \AgdaSymbol{\{}\AgdaBound{Δ}\AgdaSymbol{\}} \AgdaSymbol{\{}\AgdaBound{Θ}\AgdaSymbol{\}} \AgdaSymbol{→} \<[67]%
\>[67]\<%
\\
\>[0]\AgdaIndent{2}{}\<[2]%
\>[2]\AgdaBound{ρ} \AgdaFunction{∶} \AgdaBound{Γ} \AgdaFunction{⇒R} \AgdaBound{Δ} \AgdaSymbol{→} \AgdaBound{σ} \AgdaFunction{∶} \AgdaBound{Δ} \AgdaFunction{⇒R} \AgdaBound{Θ} \AgdaSymbol{→} \AgdaSymbol{(}\AgdaBound{σ} \AgdaFunction{•R} \AgdaBound{ρ}\AgdaSymbol{)} \AgdaFunction{∶} \AgdaBound{Γ} \AgdaFunction{⇒R} \AgdaBound{Θ}\<%
\end{code}

\AgdaHide{
\begin{code}%
\>\AgdaFunction{•R-typed} \AgdaSymbol{\{}\AgdaArgument{R} \AgdaSymbol{=} \AgdaBound{R}\AgdaSymbol{\}} \AgdaSymbol{\{}\AgdaBound{σ}\AgdaSymbol{\}} \AgdaSymbol{\{}\AgdaBound{ρ}\AgdaSymbol{\}} \AgdaSymbol{\{}\AgdaBound{Γ}\AgdaSymbol{\}} \AgdaSymbol{\{}\AgdaBound{Δ}\AgdaSymbol{\}} \AgdaSymbol{\{}\AgdaBound{Θ}\AgdaSymbol{\}} \AgdaBound{ρ∶Γ→Δ} \AgdaBound{σ∶Δ→Θ} \AgdaBound{x} \AgdaSymbol{=} \AgdaKeyword{let} \AgdaKeyword{open} \AgdaModule{≡-Reasoning} \AgdaKeyword{in} \<[77]%
\>[77]\<%
\\
\>[0]\AgdaIndent{2}{}\<[2]%
\>[2]\AgdaFunction{begin} \<[8]%
\>[8]\<%
\\
\>[2]\AgdaIndent{4}{}\<[4]%
\>[4]\AgdaFunction{unprp} \AgdaSymbol{(}\AgdaFunction{typeof} \AgdaSymbol{(}\AgdaBound{σ} \AgdaInductiveConstructor{-proof} \AgdaSymbol{(}\AgdaBound{ρ} \AgdaInductiveConstructor{-proof} \AgdaBound{x}\AgdaSymbol{))} \AgdaBound{Θ}\AgdaSymbol{)}\<%
\\
\>[0]\AgdaIndent{2}{}\<[2]%
\>[2]\AgdaFunction{≡⟨} \AgdaBound{σ∶Δ→Θ} \AgdaSymbol{(}\AgdaBound{ρ} \AgdaInductiveConstructor{-proof} \AgdaBound{x}\AgdaSymbol{)} \AgdaFunction{⟩}\<%
\\
\>[2]\AgdaIndent{4}{}\<[4]%
\>[4]\AgdaFunction{unprp} \AgdaSymbol{(}\AgdaFunction{typeof} \AgdaSymbol{(}\AgdaBound{ρ} \AgdaInductiveConstructor{-proof} \AgdaBound{x}\AgdaSymbol{)} \AgdaBound{Δ}\AgdaSymbol{)}\<%
\\
\>[0]\AgdaIndent{2}{}\<[2]%
\>[2]\AgdaFunction{≡⟨} \AgdaBound{ρ∶Γ→Δ} \AgdaBound{x} \AgdaFunction{⟩}\<%
\\
\>[2]\AgdaIndent{4}{}\<[4]%
\>[4]\AgdaFunction{unprp} \AgdaSymbol{(}\AgdaFunction{typeof} \AgdaBound{x} \AgdaBound{Γ}\AgdaSymbol{)}\<%
\\
\>[0]\AgdaIndent{2}{}\<[2]%
\>[2]\AgdaFunction{∎}\<%
\end{code}
}

\begin{code}%
\>\AgdaFunction{Weakening} \AgdaSymbol{:} \AgdaSymbol{∀} \AgdaSymbol{\{}\AgdaBound{P}\AgdaSymbol{\}} \AgdaSymbol{\{}\AgdaBound{Q}\AgdaSymbol{\}} \AgdaSymbol{\{}\AgdaBound{Γ} \AgdaSymbol{:} \AgdaDatatype{Context} \AgdaBound{P}\AgdaSymbol{\}} \AgdaSymbol{\{}\AgdaBound{Δ} \AgdaSymbol{:} \AgdaDatatype{Context} \AgdaBound{Q}\AgdaSymbol{\}} \AgdaSymbol{\{}\AgdaBound{ρ}\AgdaSymbol{\}} \AgdaSymbol{\{}\AgdaBound{δ}\AgdaSymbol{\}} \AgdaSymbol{\{}\AgdaBound{φ}\AgdaSymbol{\}} \AgdaSymbol{→} \<[68]%
\>[68]\<%
\\
\>[0]\AgdaIndent{2}{}\<[2]%
\>[2]\AgdaBound{Γ} \AgdaDatatype{⊢} \AgdaBound{δ} \AgdaDatatype{∶} \AgdaBound{φ} \AgdaSymbol{→} \AgdaBound{ρ} \AgdaFunction{∶} \AgdaBound{Γ} \AgdaFunction{⇒R} \AgdaBound{Δ} \AgdaSymbol{→} \AgdaBound{Δ} \AgdaDatatype{⊢} \AgdaBound{δ} \AgdaFunction{〈} \AgdaBound{ρ} \AgdaFunction{〉} \AgdaDatatype{∶} \AgdaBound{φ}\<%
\end{code}

\AgdaHide{
\begin{code}%
\>\AgdaFunction{Weakening} \AgdaSymbol{\{}\AgdaBound{P}\AgdaSymbol{\}} \AgdaSymbol{\{}\AgdaBound{Q}\AgdaSymbol{\}} \AgdaSymbol{\{}\AgdaBound{Γ}\AgdaSymbol{\}} \AgdaSymbol{\{}\AgdaBound{Δ}\AgdaSymbol{\}} \AgdaSymbol{\{}\AgdaBound{ρ}\AgdaSymbol{\}} \AgdaSymbol{(}\AgdaInductiveConstructor{var} \AgdaBound{p}\AgdaSymbol{)} \AgdaBound{ρ∶Γ→Δ} \AgdaSymbol{=} \AgdaFunction{change-type} \AgdaSymbol{(}\AgdaBound{ρ∶Γ→Δ} \AgdaBound{p}\AgdaSymbol{)} \AgdaSymbol{(}\AgdaInductiveConstructor{var} \AgdaSymbol{(}\AgdaBound{ρ} \AgdaSymbol{\_} \AgdaBound{p}\AgdaSymbol{))}\<%
\\
\>\AgdaFunction{Weakening} \AgdaSymbol{(}\AgdaInductiveConstructor{app} \AgdaBound{Γ⊢δ∶φ→ψ} \AgdaBound{Γ⊢ε∶φ}\AgdaSymbol{)} \AgdaBound{ρ∶Γ→Δ} \AgdaSymbol{=} \AgdaInductiveConstructor{app} \AgdaSymbol{(}\AgdaFunction{Weakening} \AgdaBound{Γ⊢δ∶φ→ψ} \AgdaBound{ρ∶Γ→Δ}\AgdaSymbol{)} \AgdaSymbol{(}\AgdaFunction{Weakening} \AgdaBound{Γ⊢ε∶φ} \AgdaBound{ρ∶Γ→Δ}\AgdaSymbol{)}\<%
\\
\>\AgdaFunction{Weakening} \AgdaSymbol{.\{}\AgdaBound{P}\AgdaSymbol{\}} \AgdaSymbol{\{}\AgdaBound{Q}\AgdaSymbol{\}} \AgdaSymbol{.\{}\AgdaBound{Γ}\AgdaSymbol{\}} \AgdaSymbol{\{}\AgdaBound{Δ}\AgdaSymbol{\}} \AgdaSymbol{\{}\AgdaBound{ρ}\AgdaSymbol{\}} \AgdaSymbol{(}\AgdaInductiveConstructor{Λ} \AgdaSymbol{\{}\AgdaBound{P}\AgdaSymbol{\}} \AgdaSymbol{\{}\AgdaBound{Γ}\AgdaSymbol{\}} \AgdaSymbol{\{}\AgdaBound{φ}\AgdaSymbol{\}} \AgdaSymbol{\{}\AgdaBound{δ}\AgdaSymbol{\}} \AgdaSymbol{\{}\AgdaBound{ψ}\AgdaSymbol{\}} \AgdaBound{Γ,φ⊢δ∶ψ}\AgdaSymbol{)} \AgdaBound{ρ∶Γ→Δ} \AgdaSymbol{=} \AgdaInductiveConstructor{Λ} \<[74]%
\>[74]\<%
\\
\>[0]\AgdaIndent{2}{}\<[2]%
\>[2]\AgdaSymbol{(}\AgdaFunction{Weakening} \AgdaSymbol{\{}\AgdaBound{P} \AgdaInductiveConstructor{,} \AgdaInductiveConstructor{-proof}\AgdaSymbol{\}} \AgdaSymbol{\{}\AgdaBound{Q} \AgdaInductiveConstructor{,} \AgdaInductiveConstructor{-proof}\AgdaSymbol{\}} \AgdaSymbol{\{}\AgdaBound{Γ} \AgdaFunction{,P} \AgdaBound{φ}\AgdaSymbol{\}} \AgdaSymbol{\{}\AgdaBound{Δ} \AgdaFunction{,P} \AgdaBound{φ}\AgdaSymbol{\}} \AgdaSymbol{\{}\AgdaFunction{liftRep} \AgdaInductiveConstructor{-proof} \AgdaBound{ρ}\AgdaSymbol{\}} \AgdaSymbol{\{}\AgdaBound{δ}\AgdaSymbol{\}} \AgdaSymbol{\{}\AgdaBound{ψ}\AgdaSymbol{\}} \<[84]%
\>[84]\<%
\\
\>[2]\AgdaIndent{4}{}\<[4]%
\>[4]\AgdaBound{Γ,φ⊢δ∶ψ} \AgdaSymbol{(}\AgdaFunction{liftRep-typed} \AgdaBound{ρ∶Γ→Δ}\AgdaSymbol{))}\<%
\end{code}
}
A \emph{substitution} $\sigma$ from a context $\Gamma$ to a context $\Delta$, $\sigma : \Gamma \rightarrow \Delta$,  is a substitution $\sigma$ such that
for every $x : \phi$ in $\Gamma$, we have $\Delta \vdash \sigma(x) : \phi$.

\begin{code}%
\>\AgdaFunction{\_∶\_⇒\_} \AgdaSymbol{:} \AgdaSymbol{∀} \AgdaSymbol{\{}\AgdaBound{P}\AgdaSymbol{\}} \AgdaSymbol{\{}\AgdaBound{Q}\AgdaSymbol{\}} \AgdaSymbol{→} \AgdaFunction{Sub} \AgdaBound{P} \AgdaBound{Q} \AgdaSymbol{→} \AgdaDatatype{Context} \AgdaBound{P} \AgdaSymbol{→} \AgdaDatatype{Context} \AgdaBound{Q} \AgdaSymbol{→} \AgdaPrimitiveType{Set}\<%
\\
\>\AgdaBound{σ} \AgdaFunction{∶} \AgdaBound{Γ} \AgdaFunction{⇒} \AgdaBound{Δ} \AgdaSymbol{=} \AgdaSymbol{∀} \AgdaBound{x} \AgdaSymbol{→} \AgdaBound{Δ} \AgdaDatatype{⊢} \AgdaBound{σ} \AgdaSymbol{\_} \AgdaBound{x} \AgdaDatatype{∶} \AgdaFunction{unprp} \AgdaSymbol{(}\AgdaFunction{typeof} \AgdaBound{x} \AgdaBound{Γ}\AgdaSymbol{)}\<%
\end{code}

\begin{lemma}$ $
\begin{enumerate}
\item
If $\sigma : \Gamma \rightarrow \Delta$ then $(\sigma , \mathrm{Proof}) : (\Gamma , p : \phi) \rightarrow (\Delta , p : \phi [ \sigma ])$.
\item
If $\Gamma \vdash \delta : \phi$ then $(p := \delta) : (\Gamma, p : \phi) \rightarrow \Gamma$.
\item
(\textbf{substitution Lemma})

If $\Gamma \vdash \delta : \phi$ and $\sigma : \Gamma \rightarrow \Delta$ then $\Delta \vdash \delta [ \sigma ] : \phi [ \sigma ]$.
\end{enumerate}
\end{lemma}

\begin{code}%
\>\AgdaFunction{liftSub-typed} \AgdaSymbol{:} \AgdaSymbol{∀} \AgdaSymbol{\{}\AgdaBound{P}\AgdaSymbol{\}} \AgdaSymbol{\{}\AgdaBound{Q}\AgdaSymbol{\}} \AgdaSymbol{\{}\AgdaBound{σ}\AgdaSymbol{\}} \<[30]%
\>[30]\<%
\\
\>[0]\AgdaIndent{2}{}\<[2]%
\>[2]\AgdaSymbol{\{}\AgdaBound{Γ} \AgdaSymbol{:} \AgdaDatatype{Context} \AgdaBound{P}\AgdaSymbol{\}} \AgdaSymbol{\{}\AgdaBound{Δ} \AgdaSymbol{:} \AgdaDatatype{Context} \AgdaBound{Q}\AgdaSymbol{\}} \AgdaSymbol{\{}\AgdaBound{φ} \AgdaSymbol{:} \AgdaDatatype{Prop}\AgdaSymbol{\}} \AgdaSymbol{→} \<[47]%
\>[47]\<%
\\
\>[0]\AgdaIndent{2}{}\<[2]%
\>[2]\AgdaBound{σ} \AgdaFunction{∶} \AgdaBound{Γ} \AgdaFunction{⇒} \AgdaBound{Δ} \AgdaSymbol{→} \AgdaFunction{liftSub} \AgdaInductiveConstructor{-proof} \AgdaBound{σ} \AgdaFunction{∶} \AgdaSymbol{(}\AgdaBound{Γ} \AgdaFunction{,P} \AgdaBound{φ}\AgdaSymbol{)} \AgdaFunction{⇒} \AgdaSymbol{(}\AgdaBound{Δ} \AgdaFunction{,P} \AgdaBound{φ}\AgdaSymbol{)}\<%
\end{code}

\AgdaHide{
\begin{code}%
\>\AgdaFunction{liftSub-typed} \AgdaSymbol{\{}\AgdaArgument{σ} \AgdaSymbol{=} \AgdaBound{σ}\AgdaSymbol{\}} \AgdaSymbol{\{}\AgdaBound{Γ}\AgdaSymbol{\}} \AgdaSymbol{\{}\AgdaBound{Δ}\AgdaSymbol{\}} \AgdaSymbol{\{}\AgdaBound{φ}\AgdaSymbol{\}} \AgdaBound{σ∶Γ⇒Δ} \AgdaBound{x} \AgdaSymbol{=}\<%
\\
\>[0]\AgdaIndent{2}{}\<[2]%
\>[2]\AgdaFunction{change-type} \AgdaSymbol{(}\AgdaFunction{sym} \AgdaSymbol{(}\AgdaFunction{unprp-rep} \AgdaSymbol{(}\AgdaFunction{pretypeof} \AgdaBound{x} \AgdaSymbol{(}\AgdaBound{Γ} \AgdaFunction{,P} \AgdaBound{φ}\AgdaSymbol{))} \AgdaFunction{upRep}\AgdaSymbol{))} \AgdaSymbol{(}\AgdaFunction{pre-LiftSub-typed} \AgdaBound{x}\AgdaSymbol{)} \AgdaKeyword{where}\<%
\\
\>[0]\AgdaIndent{2}{}\<[2]%
\>[2]\AgdaFunction{pre-LiftSub-typed} \AgdaSymbol{:} \AgdaSymbol{∀} \AgdaBound{x} \AgdaSymbol{→} \AgdaBound{Δ} \AgdaFunction{,P} \AgdaBound{φ} \AgdaDatatype{⊢} \AgdaFunction{liftSub} \AgdaInductiveConstructor{-proof} \AgdaBound{σ} \AgdaInductiveConstructor{-proof} \AgdaBound{x} \AgdaDatatype{∶} \AgdaFunction{unprp} \AgdaSymbol{(}\AgdaFunction{pretypeof} \AgdaBound{x} \AgdaSymbol{(}\AgdaBound{Γ} \AgdaFunction{,P} \AgdaBound{φ}\AgdaSymbol{))}\<%
\\
\>[0]\AgdaIndent{2}{}\<[2]%
\>[2]\AgdaFunction{pre-LiftSub-typed} \AgdaInductiveConstructor{x₀} \AgdaSymbol{=} \AgdaInductiveConstructor{var} \AgdaInductiveConstructor{x₀}\<%
\\
\>[0]\AgdaIndent{2}{}\<[2]%
\>[2]\AgdaFunction{pre-LiftSub-typed} \AgdaSymbol{(}\AgdaInductiveConstructor{↑} \AgdaBound{x}\AgdaSymbol{)} \AgdaSymbol{=} \AgdaFunction{Weakening} \AgdaSymbol{(}\AgdaBound{σ∶Γ⇒Δ} \AgdaBound{x}\AgdaSymbol{)} \AgdaSymbol{(}\AgdaFunction{↑-typed} \AgdaSymbol{\{}\AgdaArgument{φ} \AgdaSymbol{=} \AgdaBound{φ}\AgdaSymbol{\})}\<%
\end{code}
}

\begin{code}%
\>\AgdaFunction{botSub-typed} \AgdaSymbol{:} \AgdaSymbol{∀} \AgdaSymbol{\{}\AgdaBound{P}\AgdaSymbol{\}} \AgdaSymbol{\{}\AgdaBound{Γ} \AgdaSymbol{:} \AgdaDatatype{Context} \AgdaBound{P}\AgdaSymbol{\}} \AgdaSymbol{\{}\AgdaBound{φ} \AgdaSymbol{:} \AgdaDatatype{Prop}\AgdaSymbol{\}} \AgdaSymbol{\{}\AgdaBound{δ}\AgdaSymbol{\}} \AgdaSymbol{→}\<%
\\
\>[0]\AgdaIndent{2}{}\<[2]%
\>[2]\AgdaBound{Γ} \AgdaDatatype{⊢} \AgdaBound{δ} \AgdaDatatype{∶} \AgdaBound{φ} \AgdaSymbol{→} \AgdaFunction{x₀:=} \AgdaBound{δ} \AgdaFunction{∶} \AgdaSymbol{(}\AgdaBound{Γ} \AgdaFunction{,P} \AgdaBound{φ}\AgdaSymbol{)} \AgdaFunction{⇒} \AgdaBound{Γ}\<%
\end{code}

\AgdaHide{
\begin{code}%
\>\AgdaFunction{botSub-typed} \AgdaSymbol{\{}\AgdaBound{P}\AgdaSymbol{\}} \AgdaSymbol{\{}\AgdaBound{Γ}\AgdaSymbol{\}} \AgdaSymbol{\{}\AgdaBound{φ}\AgdaSymbol{\}} \AgdaSymbol{\{}\AgdaBound{δ}\AgdaSymbol{\}} \AgdaBound{Γ⊢δ:φ} \AgdaBound{x} \AgdaSymbol{=} \<[39]%
\>[39]\<%
\\
\>[0]\AgdaIndent{2}{}\<[2]%
\>[2]\AgdaFunction{change-type} \AgdaSymbol{(}\AgdaFunction{sym} \AgdaSymbol{(}\AgdaFunction{unprp-rep} \AgdaSymbol{(}\AgdaFunction{pretypeof} \AgdaBound{x} \AgdaSymbol{(}\AgdaBound{Γ} \AgdaFunction{,P} \AgdaBound{φ}\AgdaSymbol{))} \AgdaFunction{upRep}\AgdaSymbol{))} \AgdaSymbol{(}\AgdaFunction{pre-botSub-typed} \AgdaBound{x}\AgdaSymbol{)} \AgdaKeyword{where}\<%
\\
\>[0]\AgdaIndent{2}{}\<[2]%
\>[2]\AgdaFunction{pre-botSub-typed} \AgdaSymbol{:} \AgdaSymbol{∀} \AgdaBound{x} \AgdaSymbol{→} \AgdaBound{Γ} \AgdaDatatype{⊢} \AgdaSymbol{(}\AgdaFunction{x₀:=} \AgdaBound{δ}\AgdaSymbol{)} \AgdaInductiveConstructor{-proof} \AgdaBound{x} \AgdaDatatype{∶} \AgdaFunction{unprp} \AgdaSymbol{(}\AgdaFunction{pretypeof} \AgdaBound{x} \AgdaSymbol{(}\AgdaBound{Γ} \AgdaFunction{,P} \AgdaBound{φ}\AgdaSymbol{))}\<%
\\
\>[0]\AgdaIndent{2}{}\<[2]%
\>[2]\AgdaFunction{pre-botSub-typed} \AgdaInductiveConstructor{x₀} \AgdaSymbol{=} \AgdaBound{Γ⊢δ:φ}\<%
\\
\>[0]\AgdaIndent{2}{}\<[2]%
\>[2]\AgdaFunction{pre-botSub-typed} \AgdaSymbol{(}\AgdaInductiveConstructor{↑} \AgdaBound{x}\AgdaSymbol{)} \AgdaSymbol{=} \AgdaInductiveConstructor{var} \AgdaBound{x}\<%
\end{code}
}

\begin{code}%
\>\AgdaFunction{substitution} \AgdaSymbol{:} \AgdaSymbol{∀} \AgdaSymbol{\{}\AgdaBound{P}\AgdaSymbol{\}} \AgdaSymbol{\{}\AgdaBound{Q}\AgdaSymbol{\}}\<%
\\
\>[0]\AgdaIndent{2}{}\<[2]%
\>[2]\AgdaSymbol{\{}\AgdaBound{Γ} \AgdaSymbol{:} \AgdaDatatype{Context} \AgdaBound{P}\AgdaSymbol{\}} \AgdaSymbol{\{}\AgdaBound{Δ} \AgdaSymbol{:} \AgdaDatatype{Context} \AgdaBound{Q}\AgdaSymbol{\}} \AgdaSymbol{\{}\AgdaBound{δ}\AgdaSymbol{\}} \AgdaSymbol{\{}\AgdaBound{φ}\AgdaSymbol{\}} \AgdaSymbol{\{}\AgdaBound{σ}\AgdaSymbol{\}} \AgdaSymbol{→} \<[48]%
\>[48]\<%
\\
\>[0]\AgdaIndent{2}{}\<[2]%
\>[2]\AgdaBound{Γ} \AgdaDatatype{⊢} \AgdaBound{δ} \AgdaDatatype{∶} \AgdaBound{φ} \AgdaSymbol{→} \AgdaBound{σ} \AgdaFunction{∶} \AgdaBound{Γ} \AgdaFunction{⇒} \AgdaBound{Δ} \AgdaSymbol{→} \AgdaBound{Δ} \AgdaDatatype{⊢} \AgdaBound{δ} \AgdaFunction{⟦} \AgdaBound{σ} \AgdaFunction{⟧} \AgdaDatatype{∶} \AgdaBound{φ}\<%
\end{code}

\AgdaHide{
\begin{code}%
\>\AgdaFunction{substitution} \AgdaSymbol{(}\AgdaInductiveConstructor{var} \AgdaSymbol{\_)} \AgdaBound{σ∶Γ→Δ} \AgdaSymbol{=} \AgdaBound{σ∶Γ→Δ} \AgdaSymbol{\_}\<%
\\
\>\AgdaFunction{substitution} \AgdaSymbol{(}\AgdaInductiveConstructor{app} \AgdaBound{Γ⊢δ∶φ→ψ} \AgdaBound{Γ⊢ε∶φ}\AgdaSymbol{)} \AgdaBound{σ∶Γ→Δ} \AgdaSymbol{=} \AgdaInductiveConstructor{app} \AgdaSymbol{(}\AgdaFunction{substitution} \AgdaBound{Γ⊢δ∶φ→ψ} \AgdaBound{σ∶Γ→Δ}\AgdaSymbol{)} \AgdaSymbol{(}\AgdaFunction{substitution} \AgdaBound{Γ⊢ε∶φ} \AgdaBound{σ∶Γ→Δ}\AgdaSymbol{)}\<%
\\
\>\AgdaFunction{substitution} \AgdaSymbol{\{}\AgdaArgument{Q} \AgdaSymbol{=} \AgdaBound{Q}\AgdaSymbol{\}} \AgdaSymbol{\{}\AgdaArgument{Δ} \AgdaSymbol{=} \AgdaBound{Δ}\AgdaSymbol{\}} \AgdaSymbol{\{}\AgdaArgument{σ} \AgdaSymbol{=} \AgdaBound{σ}\AgdaSymbol{\}} \AgdaSymbol{(}\AgdaInductiveConstructor{Λ} \AgdaSymbol{\{}\AgdaBound{P}\AgdaSymbol{\}} \AgdaSymbol{\{}\AgdaBound{Γ}\AgdaSymbol{\}} \AgdaSymbol{\{}\AgdaBound{φ}\AgdaSymbol{\}} \AgdaSymbol{\{}\AgdaBound{δ}\AgdaSymbol{\}} \AgdaSymbol{\{}\AgdaBound{ψ}\AgdaSymbol{\}} \AgdaBound{Γ,φ⊢δ∶ψ}\AgdaSymbol{)} \AgdaBound{σ∶Γ→Δ} \AgdaSymbol{=} \AgdaInductiveConstructor{Λ} \<[79]%
\>[79]\<%
\\
\>[0]\AgdaIndent{2}{}\<[2]%
\>[2]\AgdaSymbol{(}\AgdaFunction{substitution} \AgdaBound{Γ,φ⊢δ∶ψ} \AgdaSymbol{(}\AgdaFunction{liftSub-typed} \AgdaBound{σ∶Γ→Δ}\AgdaSymbol{))}\<%
\end{code}
}

\begin{lemma}[Subject Reduction]
If $\Gamma \vdash \delta : \phi$ and $\delta \rightarrow_\beta \epsilon$ then $\Gamma \vdash \epsilon : \phi$.
\end{lemma}

\begin{code}%
\>\AgdaFunction{subject-reduction} \AgdaSymbol{:} \AgdaSymbol{∀} \AgdaSymbol{\{}\AgdaBound{P}\AgdaSymbol{\}} \AgdaSymbol{\{}\AgdaBound{Γ} \AgdaSymbol{:} \AgdaDatatype{Context} \AgdaBound{P}\AgdaSymbol{\}} \AgdaSymbol{\{}\AgdaBound{δ} \AgdaBound{ε} \AgdaSymbol{:} \AgdaFunction{Proof} \AgdaSymbol{(} \AgdaBound{P}\AgdaSymbol{)\}} \AgdaSymbol{\{}\AgdaBound{φ}\AgdaSymbol{\}} \AgdaSymbol{→} \<[67]%
\>[67]\<%
\\
\>[0]\AgdaIndent{2}{}\<[2]%
\>[2]\AgdaBound{Γ} \AgdaDatatype{⊢} \AgdaBound{δ} \AgdaDatatype{∶} \AgdaBound{φ} \AgdaSymbol{→} \AgdaBound{δ} \AgdaDatatype{⇒} \AgdaBound{ε} \AgdaSymbol{→} \AgdaBound{Γ} \AgdaDatatype{⊢} \AgdaBound{ε} \AgdaDatatype{∶} \AgdaBound{φ}\<%
\end{code}

\AgdaHide{
\begin{code}%
\>\AgdaFunction{subject-reduction} \AgdaSymbol{(}\AgdaInductiveConstructor{var} \AgdaSymbol{\_)} \AgdaSymbol{()}\<%
\\
\>\AgdaFunction{subject-reduction} \AgdaSymbol{(}\AgdaInductiveConstructor{app} \AgdaSymbol{\{}\AgdaArgument{ε} \AgdaSymbol{=} \AgdaBound{ε}\AgdaSymbol{\}} \AgdaSymbol{(}\AgdaInductiveConstructor{Λ} \AgdaSymbol{\{}\AgdaBound{P}\AgdaSymbol{\}} \AgdaSymbol{\{}\AgdaBound{Γ}\AgdaSymbol{\}} \AgdaSymbol{\{}\AgdaBound{φ}\AgdaSymbol{\}} \AgdaSymbol{\{}\AgdaBound{δ}\AgdaSymbol{\}} \AgdaSymbol{\{}\AgdaBound{ψ}\AgdaSymbol{\}} \AgdaBound{Γ,φ⊢δ∶ψ}\AgdaSymbol{)} \AgdaBound{Γ⊢ε∶φ}\AgdaSymbol{)} \AgdaSymbol{(}\AgdaInductiveConstructor{redex} \AgdaInductiveConstructor{βI}\AgdaSymbol{)} \AgdaSymbol{=} \<[83]%
\>[83]\<%
\\
\>[0]\AgdaIndent{2}{}\<[2]%
\>[2]\AgdaFunction{substitution} \AgdaBound{Γ,φ⊢δ∶ψ} \AgdaSymbol{(}\AgdaFunction{botSub-typed} \AgdaBound{Γ⊢ε∶φ}\AgdaSymbol{)}\<%
\\
\>\AgdaFunction{subject-reduction} \AgdaSymbol{(}\AgdaInductiveConstructor{app} \AgdaBound{Γ⊢δ∶φ→ψ} \AgdaBound{Γ⊢ε∶φ}\AgdaSymbol{)} \AgdaSymbol{(}\AgdaInductiveConstructor{app} \AgdaSymbol{(}\AgdaInductiveConstructor{appl} \AgdaBound{δ→δ'}\AgdaSymbol{))} \AgdaSymbol{=} \AgdaInductiveConstructor{app} \AgdaSymbol{(}\AgdaFunction{subject-reduction} \AgdaBound{Γ⊢δ∶φ→ψ} \AgdaBound{δ→δ'}\AgdaSymbol{)} \AgdaBound{Γ⊢ε∶φ}\<%
\\
\>\AgdaFunction{subject-reduction} \AgdaSymbol{(}\AgdaInductiveConstructor{app} \AgdaBound{Γ⊢δ∶φ→ψ} \AgdaBound{Γ⊢ε∶φ}\AgdaSymbol{)} \AgdaSymbol{(}\AgdaInductiveConstructor{app} \AgdaSymbol{(}\AgdaInductiveConstructor{appr} \AgdaSymbol{(}\AgdaInductiveConstructor{appl} \AgdaBound{ε→ε'}\AgdaSymbol{)))} \AgdaSymbol{=} \AgdaInductiveConstructor{app} \AgdaBound{Γ⊢δ∶φ→ψ} \AgdaSymbol{(}\AgdaFunction{subject-reduction} \AgdaBound{Γ⊢ε∶φ} \AgdaBound{ε→ε'}\AgdaSymbol{)}\<%
\\
\>\AgdaFunction{subject-reduction} \AgdaSymbol{(}\AgdaInductiveConstructor{app} \AgdaBound{Γ⊢δ∶φ→ψ} \AgdaBound{Γ⊢ε∶φ}\AgdaSymbol{)} \AgdaSymbol{(}\AgdaInductiveConstructor{app} \AgdaSymbol{(}\AgdaInductiveConstructor{appr} \AgdaSymbol{(}\AgdaInductiveConstructor{appr} \AgdaSymbol{())))}\<%
\\
\>\AgdaFunction{subject-reduction} \AgdaSymbol{(}\AgdaInductiveConstructor{Λ} \AgdaSymbol{\_)} \AgdaSymbol{(}\AgdaInductiveConstructor{redex} \AgdaSymbol{())}\<%
\\
\>\AgdaFunction{subject-reduction} \AgdaSymbol{(}\AgdaInductiveConstructor{Λ} \AgdaBound{Γ,φ⊢δ∶ψ}\AgdaSymbol{)} \AgdaSymbol{(}\AgdaInductiveConstructor{app} \AgdaSymbol{(}\AgdaInductiveConstructor{appl} \AgdaBound{δ⇒ε}\AgdaSymbol{))} \AgdaSymbol{=} \AgdaInductiveConstructor{Λ} \AgdaSymbol{(}\AgdaFunction{subject-reduction} \AgdaBound{Γ,φ⊢δ∶ψ} \AgdaBound{δ⇒ε}\AgdaSymbol{)}\<%
\\
\>\AgdaFunction{subject-reduction} \AgdaSymbol{(}\AgdaInductiveConstructor{Λ} \AgdaBound{Γ⊢δ∶φ}\AgdaSymbol{)} \AgdaSymbol{(}\AgdaInductiveConstructor{app} \AgdaSymbol{(}\AgdaInductiveConstructor{appr} \AgdaSymbol{()))}\<%
\end{code}
}

}

\subsection{The Reduction Relation}

\mode<all>{\AgdaHide{
\begin{code}%
\>\AgdaKeyword{module} \AgdaModule{PHOPL.Red} \AgdaKeyword{where}\<%
\\
\>\AgdaKeyword{open} \AgdaKeyword{import} \AgdaModule{Data.Unit}\<%
\\
\>\AgdaKeyword{open} \AgdaKeyword{import} \AgdaModule{Data.Product} \AgdaKeyword{renaming} \AgdaSymbol{(}\AgdaInductiveConstructor{\_,\_} \AgdaSymbol{to} \AgdaInductiveConstructor{\_,p\_}\AgdaSymbol{)}\<%
\\
\>\AgdaKeyword{open} \AgdaKeyword{import} \AgdaModule{Data.List}\<%
\\
\>\AgdaKeyword{open} \AgdaKeyword{import} \AgdaModule{Prelims}\<%
\\
\>\AgdaKeyword{open} \AgdaKeyword{import} \AgdaModule{PHOPL.Grammar}\<%
\\
\>\AgdaKeyword{open} \AgdaKeyword{import} \AgdaModule{PHOPL.PathSub}\<%
\end{code}
}

\subsection{The Reduction Relation}

We make the following definitions simultaneously:
\begin{enumerate}
\item
Let \emph{contraction} $\rhd$ be the relation consisting of the following pairs, where $M$, $N$, $N'$, $N_1$, $N_2$, $P$, $Q$, $\phi$, $\phi'$, $\psi$, $\psi'$, $\chi$, $\delta$, $\delta'$, $\epsilon$, $\epsilon'$ are closed expressions in normal form:
\begin{align*}
(\lambda x:A.M)N & \rhd M[x:=N] & (\lambda p:\phi.\delta)\epsilon & \rhd \delta[p:=\epsilon] \\
 \reff{\phi}^+ & \rhd \lambda p:\phi.p & \reff{\phi}^- & \rhd \lambda p:\phi.p \\
\univ{\phi}{\psi}{\delta}{\epsilon}^+ & \rhd \delta & \univ{\phi}{\psi}{\delta}{\epsilon}^- & \rhd \epsilon
\end{align*}
\begin{align*}
& \reff \phi \supset^* \univ{\psi}{\chi}{\delta}{\epsilon} \\
& \quad \rhd \mathsf{univ}_{\phi \supset \psi,\phi \supset \chi}(\lambda p : \phi \supset \psi. \lambda q : \phi . \delta (p q), 
\lambda p : \phi \supset \chi. \lambda q : \phi . \epsilon (p q)) \\
& \univ{\phi}{\psi}{\delta}{\epsilon} \supset^* \reff{\chi} \\
& \quad \rhd \univ{\phi \supset \chi}{\psi \supset \chi}{\lambda p : \phi \supset \chi . \lambda q : \psi .p (\epsilon q)}{\lambda p : \psi \supset \chi . \lambda q : \phi .p (\delta q)} \\
& \univ{\phi}{\psi}{\delta}{\epsilon} \supset^* \univ{\phi'}{\psi'}{\delta'}{\epsilon'} \\
& \quad \rhd \univ{\phi \supset \phi'}{\psi \supset \psi'}
{\lambda p : \phi \supset \phi' . \lambda q : \psi . \delta' (p (\epsilon q))}{\lambda p : \psi \supset \psi'. \lambda q : \phi . \epsilon' (p (\delta q))}
\end{align*}
\begin{align*}
\reff{\phi} \supset^* \reff{\psi} & \rhd \reff{\phi \supset \psi} \\
\reff{M}_{N_1N_2} \reff{N} & \rhd \reff{MN} \\
(\triplelambda e:x =_A y. P)_{MN}Q & \rhd P[x:=M, y:=N, e:=Q] \\
\text{If $P$ does not have the form } \reff{-}, \text{ then } \\ \reff{\lambda x:A.M}_{N,N'} P & \rhd M \{ x := P : N ∼ N' \}
\end{align*}
\item
{One-step reduction} $\rightarrow$ is the congruence generated by $\rhd$.  That is, the expression $E$ \emph{reduces in one step} to the expression $F$, $E \rightarrow F$, iff $F$ is formed from $E$ by replacing a subterm $G$ with a subterm $H$, where $G \rhd H$.
\item
An expression $E$ is in \emph{normal form} iff there is no expression $F$ such that $E \rightarrow F$.
\end{enumerate}

\todo{Conjecture: We can remove the restriction on $P$ in the last clause if we add reduction rules:
$\triplelambda e:y=_Ay'.M\{x:=e:y\sim y'\} \rhd \reff{\lambda x:A.M}$}

\begin{code}%
\>\AgdaKeyword{data} \AgdaDatatype{R} \AgdaSymbol{:} \AgdaFunction{Reduction} \AgdaKeyword{where}\<%
\\
\>[0]\AgdaIndent{2}{}\<[2]%
\>[2]\AgdaInductiveConstructor{βT} \AgdaSymbol{:} \AgdaSymbol{∀} \AgdaSymbol{\{}\AgdaBound{V}\AgdaSymbol{\}} \AgdaSymbol{\{}\AgdaBound{A}\AgdaSymbol{\}} \AgdaSymbol{\{}\AgdaBound{M}\AgdaSymbol{\}} \AgdaSymbol{\{}\AgdaBound{N}\AgdaSymbol{\}} \AgdaSymbol{→} \AgdaDatatype{R} \AgdaSymbol{\{}\AgdaBound{V}\AgdaSymbol{\}} \AgdaInductiveConstructor{-appTerm} \AgdaSymbol{(}\AgdaFunction{ΛT} \AgdaBound{A} \AgdaBound{M} \AgdaInductiveConstructor{,,} \AgdaBound{N} \AgdaInductiveConstructor{,,} \AgdaInductiveConstructor{out}\AgdaSymbol{)} \AgdaSymbol{(}\AgdaBound{M} \AgdaFunction{⟦} \AgdaFunction{x₀:=} \AgdaBound{N} \AgdaFunction{⟧}\AgdaSymbol{)}\<%
\\
\>[0]\AgdaIndent{2}{}\<[2]%
\>[2]\AgdaInductiveConstructor{βR} \AgdaSymbol{:} \AgdaSymbol{∀} \AgdaSymbol{\{}\AgdaBound{V}\AgdaSymbol{\}} \AgdaSymbol{\{}\AgdaBound{φ}\AgdaSymbol{\}} \AgdaSymbol{\{}\AgdaBound{δ}\AgdaSymbol{\}} \AgdaSymbol{\{}\AgdaBound{ε}\AgdaSymbol{\}} \AgdaSymbol{→} \AgdaDatatype{R} \AgdaSymbol{\{}\AgdaBound{V}\AgdaSymbol{\}} \AgdaInductiveConstructor{-appProof} \AgdaSymbol{(}\AgdaFunction{ΛP} \AgdaBound{φ} \AgdaBound{δ} \AgdaInductiveConstructor{,,} \AgdaBound{ε} \AgdaInductiveConstructor{,,} \AgdaInductiveConstructor{out}\AgdaSymbol{)} \AgdaSymbol{(}\AgdaBound{δ} \AgdaFunction{⟦} \AgdaFunction{x₀:=} \AgdaBound{ε} \AgdaFunction{⟧}\AgdaSymbol{)}\<%
\\
\>[0]\AgdaIndent{2}{}\<[2]%
\>[2]\AgdaInductiveConstructor{plus-ref} \AgdaSymbol{:} \AgdaSymbol{∀} \AgdaSymbol{\{}\AgdaBound{V}\AgdaSymbol{\}} \AgdaSymbol{\{}\AgdaBound{φ}\AgdaSymbol{\}} \AgdaSymbol{→} \AgdaDatatype{R} \AgdaSymbol{\{}\AgdaBound{V}\AgdaSymbol{\}} \AgdaInductiveConstructor{-plus} \AgdaSymbol{(}\AgdaFunction{reff} \AgdaBound{φ} \AgdaInductiveConstructor{,,} \AgdaInductiveConstructor{out}\AgdaSymbol{)} \AgdaSymbol{(}\AgdaFunction{ΛP} \AgdaBound{φ} \AgdaSymbol{(}\AgdaInductiveConstructor{var} \AgdaInductiveConstructor{x₀}\AgdaSymbol{))}\<%
\\
\>[0]\AgdaIndent{2}{}\<[2]%
\>[2]\AgdaInductiveConstructor{minus-ref} \AgdaSymbol{:} \AgdaSymbol{∀} \AgdaSymbol{\{}\AgdaBound{V}\AgdaSymbol{\}} \AgdaSymbol{\{}\AgdaBound{φ}\AgdaSymbol{\}} \AgdaSymbol{→} \AgdaDatatype{R} \AgdaSymbol{\{}\AgdaBound{V}\AgdaSymbol{\}} \AgdaInductiveConstructor{-minus} \AgdaSymbol{(}\AgdaFunction{reff} \AgdaBound{φ} \AgdaInductiveConstructor{,,} \AgdaInductiveConstructor{out}\AgdaSymbol{)} \AgdaSymbol{(}\AgdaFunction{ΛP} \AgdaBound{φ} \AgdaSymbol{(}\AgdaInductiveConstructor{var} \AgdaInductiveConstructor{x₀}\AgdaSymbol{))}\<%
\\
\>[0]\AgdaIndent{2}{}\<[2]%
\>[2]\AgdaInductiveConstructor{plus-univ} \AgdaSymbol{:} \AgdaSymbol{∀} \AgdaSymbol{\{}\AgdaBound{V}\AgdaSymbol{\}} \AgdaSymbol{\{}\AgdaBound{φ}\AgdaSymbol{\}} \AgdaSymbol{\{}\AgdaBound{ψ}\AgdaSymbol{\}} \AgdaSymbol{\{}\AgdaBound{δ}\AgdaSymbol{\}} \AgdaSymbol{\{}\AgdaBound{ε}\AgdaSymbol{\}} \AgdaSymbol{→} \AgdaDatatype{R} \AgdaSymbol{\{}\AgdaBound{V}\AgdaSymbol{\}} \AgdaInductiveConstructor{-plus} \AgdaSymbol{(}\AgdaFunction{univ} \AgdaBound{φ} \AgdaBound{ψ} \AgdaBound{δ} \AgdaBound{ε} \AgdaInductiveConstructor{,,} \AgdaInductiveConstructor{out}\AgdaSymbol{)} \AgdaBound{δ} \<[74]%
\>[74]\<%
\\
\>[0]\AgdaIndent{2}{}\<[2]%
\>[2]\AgdaInductiveConstructor{minus-univ} \AgdaSymbol{:} \AgdaSymbol{∀} \AgdaSymbol{\{}\AgdaBound{V}\AgdaSymbol{\}} \AgdaSymbol{\{}\AgdaBound{φ}\AgdaSymbol{\}} \AgdaSymbol{\{}\AgdaBound{ψ}\AgdaSymbol{\}} \AgdaSymbol{\{}\AgdaBound{δ}\AgdaSymbol{\}} \AgdaSymbol{\{}\AgdaBound{ε}\AgdaSymbol{\}} \AgdaSymbol{→} \AgdaDatatype{R} \AgdaSymbol{\{}\AgdaBound{V}\AgdaSymbol{\}} \AgdaInductiveConstructor{-minus} \AgdaSymbol{(}\AgdaFunction{univ} \AgdaBound{φ} \AgdaBound{ψ} \AgdaBound{δ} \AgdaBound{ε} \AgdaInductiveConstructor{,,} \AgdaInductiveConstructor{out}\AgdaSymbol{)} \AgdaBound{ε}\<%
\\
\>[0]\AgdaIndent{2}{}\<[2]%
\>[2]\AgdaInductiveConstructor{ref⊃*univ} \AgdaSymbol{:} \AgdaSymbol{∀} \AgdaSymbol{\{}\AgdaBound{V}\AgdaSymbol{\}} \AgdaSymbol{\{}\AgdaBound{φ}\AgdaSymbol{\}} \AgdaSymbol{\{}\AgdaBound{ψ}\AgdaSymbol{\}} \AgdaSymbol{\{}\AgdaBound{χ}\AgdaSymbol{\}} \AgdaSymbol{\{}\AgdaBound{δ}\AgdaSymbol{\}} \AgdaSymbol{\{}\AgdaBound{ε}\AgdaSymbol{\}} \AgdaSymbol{→} \AgdaDatatype{R} \AgdaSymbol{\{}\AgdaBound{V}\AgdaSymbol{\}} \AgdaInductiveConstructor{-imp*} \AgdaSymbol{(}\AgdaFunction{reff} \AgdaBound{φ} \AgdaInductiveConstructor{,,} \AgdaFunction{univ} \AgdaBound{ψ} \AgdaBound{χ} \AgdaBound{δ} \AgdaBound{ε} \AgdaInductiveConstructor{,,} \AgdaInductiveConstructor{out}\AgdaSymbol{)} \AgdaSymbol{(}\AgdaFunction{univ} \AgdaSymbol{(}\AgdaBound{φ} \AgdaFunction{⊃} \AgdaBound{ψ}\AgdaSymbol{)} \AgdaSymbol{(}\AgdaBound{φ} \AgdaFunction{⊃} \AgdaBound{χ}\AgdaSymbol{)} \AgdaSymbol{(}\AgdaFunction{ΛP} \AgdaSymbol{(}\AgdaBound{φ} \AgdaFunction{⊃} \AgdaBound{ψ}\AgdaSymbol{)} \AgdaSymbol{(}\AgdaFunction{ΛP} \AgdaSymbol{(}\AgdaBound{φ} \AgdaFunction{⇑}\AgdaSymbol{)} \AgdaSymbol{(}\AgdaFunction{appP} \AgdaSymbol{(}\AgdaBound{δ} \AgdaFunction{⇑} \AgdaFunction{⇑}\AgdaSymbol{)} \AgdaSymbol{(}\AgdaFunction{appP} \AgdaSymbol{(}\AgdaInductiveConstructor{var} \AgdaFunction{x₁}\AgdaSymbol{)} \AgdaSymbol{(}\AgdaInductiveConstructor{var} \AgdaInductiveConstructor{x₀}\AgdaSymbol{)))))} \AgdaSymbol{(}\AgdaFunction{ΛP} \AgdaSymbol{(}\AgdaBound{φ} \AgdaFunction{⊃} \AgdaBound{χ}\AgdaSymbol{)} \AgdaSymbol{(}\AgdaFunction{ΛP} \AgdaSymbol{(}\AgdaBound{φ} \AgdaFunction{⇑}\AgdaSymbol{)} \AgdaSymbol{(}\AgdaFunction{appP} \AgdaSymbol{(}\AgdaBound{ε} \AgdaFunction{⇑} \AgdaFunction{⇑}\AgdaSymbol{)} \AgdaSymbol{(}\AgdaFunction{appP} \AgdaSymbol{(}\AgdaInductiveConstructor{var} \AgdaFunction{x₁}\AgdaSymbol{)} \AgdaSymbol{(}\AgdaInductiveConstructor{var} \AgdaInductiveConstructor{x₀}\AgdaSymbol{))))))}\<%
\\
\>[0]\AgdaIndent{2}{}\<[2]%
\>[2]\AgdaInductiveConstructor{univ⊃*ref} \AgdaSymbol{:} \AgdaSymbol{∀} \AgdaSymbol{\{}\AgdaBound{V}\AgdaSymbol{\}} \AgdaSymbol{\{}\AgdaBound{φ}\AgdaSymbol{\}} \AgdaSymbol{\{}\AgdaBound{ψ}\AgdaSymbol{\}} \AgdaSymbol{\{}\AgdaBound{χ}\AgdaSymbol{\}} \AgdaSymbol{\{}\AgdaBound{δ}\AgdaSymbol{\}} \AgdaSymbol{\{}\AgdaBound{ε}\AgdaSymbol{\}} \AgdaSymbol{→} \AgdaDatatype{R} \AgdaSymbol{\{}\AgdaBound{V}\AgdaSymbol{\}} \AgdaInductiveConstructor{-imp*} \AgdaSymbol{(}\AgdaFunction{univ} \AgdaBound{φ} \AgdaBound{ψ} \AgdaBound{δ} \AgdaBound{ε} \AgdaInductiveConstructor{,,} \AgdaFunction{reff} \AgdaBound{χ} \AgdaInductiveConstructor{,,} \AgdaInductiveConstructor{out}\AgdaSymbol{)} \AgdaSymbol{(}\AgdaFunction{univ} \AgdaSymbol{(}\AgdaBound{φ} \AgdaFunction{⊃} \AgdaBound{χ}\AgdaSymbol{)} \AgdaSymbol{(}\AgdaBound{ψ} \AgdaFunction{⊃} \AgdaBound{χ}\AgdaSymbol{)} \AgdaSymbol{(}\AgdaFunction{ΛP} \AgdaSymbol{(}\AgdaBound{φ} \AgdaFunction{⊃} \AgdaBound{χ}\AgdaSymbol{)} \AgdaSymbol{(}\AgdaFunction{ΛP} \AgdaSymbol{(}\AgdaBound{ψ} \AgdaFunction{⇑}\AgdaSymbol{)} \AgdaSymbol{(}\AgdaFunction{appP} \AgdaSymbol{(}\AgdaInductiveConstructor{var} \AgdaFunction{x₁}\AgdaSymbol{)} \AgdaSymbol{(}\AgdaFunction{appP} \AgdaSymbol{(}\AgdaBound{ε} \AgdaFunction{⇑} \AgdaFunction{⇑}\AgdaSymbol{)} \AgdaSymbol{(}\AgdaInductiveConstructor{var} \AgdaInductiveConstructor{x₀}\AgdaSymbol{)))))} \AgdaSymbol{(}\AgdaFunction{ΛP} \AgdaSymbol{(}\AgdaBound{ψ} \AgdaFunction{⊃} \AgdaBound{χ}\AgdaSymbol{)} \AgdaSymbol{(}\AgdaFunction{ΛP} \AgdaSymbol{(}\AgdaBound{φ} \AgdaFunction{⇑}\AgdaSymbol{)} \AgdaSymbol{(}\AgdaFunction{appP} \AgdaSymbol{(}\AgdaInductiveConstructor{var} \AgdaFunction{x₁}\AgdaSymbol{)} \AgdaSymbol{(}\AgdaFunction{appP} \AgdaSymbol{(}\AgdaBound{δ} \AgdaFunction{⇑} \AgdaFunction{⇑}\AgdaSymbol{)} \AgdaSymbol{(}\AgdaInductiveConstructor{var} \AgdaInductiveConstructor{x₀}\AgdaSymbol{))))))}\<%
\\
\>[0]\AgdaIndent{2}{}\<[2]%
\>[2]\AgdaInductiveConstructor{univ⊃*univ} \AgdaSymbol{:} \AgdaSymbol{∀} \AgdaSymbol{\{}\AgdaBound{V}\AgdaSymbol{\}} \AgdaSymbol{\{}\AgdaBound{φ}\AgdaSymbol{\}} \AgdaSymbol{\{}\AgdaBound{φ'}\AgdaSymbol{\}} \AgdaSymbol{\{}\AgdaBound{ψ}\AgdaSymbol{\}} \AgdaSymbol{\{}\AgdaBound{ψ'}\AgdaSymbol{\}} \AgdaSymbol{\{}\AgdaBound{δ}\AgdaSymbol{\}} \AgdaSymbol{\{}\AgdaBound{δ'}\AgdaSymbol{\}} \AgdaSymbol{\{}\AgdaBound{ε}\AgdaSymbol{\}} \AgdaSymbol{\{}\AgdaBound{ε'}\AgdaSymbol{\}} \AgdaSymbol{→}\<%
\\
\>[2]\AgdaIndent{4}{}\<[4]%
\>[4]\AgdaDatatype{R} \AgdaSymbol{\{}\AgdaBound{V}\AgdaSymbol{\}} \AgdaInductiveConstructor{-imp*} \AgdaSymbol{(}\AgdaFunction{univ} \AgdaBound{φ} \AgdaBound{ψ} \AgdaBound{δ} \AgdaBound{ε} \AgdaInductiveConstructor{,,} \AgdaFunction{univ} \AgdaBound{φ'} \AgdaBound{ψ'} \AgdaBound{δ'} \AgdaBound{ε'} \AgdaInductiveConstructor{,,} \AgdaInductiveConstructor{out}\AgdaSymbol{)}\<%
\\
\>[2]\AgdaIndent{4}{}\<[4]%
\>[4]\AgdaSymbol{(}\AgdaFunction{univ} \AgdaSymbol{(}\AgdaBound{φ} \AgdaFunction{⊃} \AgdaBound{φ'}\AgdaSymbol{)} \AgdaSymbol{(}\AgdaBound{ψ} \AgdaFunction{⊃} \AgdaBound{ψ'}\AgdaSymbol{)} \AgdaSymbol{(}\AgdaFunction{ΛP} \AgdaSymbol{(}\AgdaBound{φ} \AgdaFunction{⊃} \AgdaBound{φ'}\AgdaSymbol{)} \AgdaSymbol{(}\AgdaFunction{ΛP} \AgdaSymbol{(}\AgdaBound{ψ} \AgdaFunction{⇑}\AgdaSymbol{)} \AgdaSymbol{(}\AgdaFunction{appP} \AgdaSymbol{(}\AgdaBound{δ'} \AgdaFunction{⇑} \AgdaFunction{⇑}\AgdaSymbol{)} \AgdaSymbol{(}\AgdaFunction{appP} \AgdaSymbol{(}\AgdaInductiveConstructor{var} \AgdaFunction{x₁}\AgdaSymbol{)} \AgdaSymbol{(}\AgdaFunction{appP} \AgdaSymbol{(}\AgdaBound{ε} \AgdaFunction{⇑} \AgdaFunction{⇑}\AgdaSymbol{)} \AgdaSymbol{(}\AgdaInductiveConstructor{var} \AgdaInductiveConstructor{x₀}\AgdaSymbol{))))))}\<%
\\
\>[4]\AgdaIndent{6}{}\<[6]%
\>[6]\AgdaSymbol{(}\AgdaFunction{ΛP} \AgdaSymbol{(}\AgdaBound{ψ} \AgdaFunction{⊃} \AgdaBound{ψ'}\AgdaSymbol{)} \AgdaSymbol{(}\AgdaFunction{ΛP} \AgdaSymbol{(}\AgdaBound{φ} \AgdaFunction{⇑}\AgdaSymbol{)} \AgdaSymbol{(}\AgdaFunction{appP} \AgdaSymbol{(}\AgdaBound{ε'} \AgdaFunction{⇑} \AgdaFunction{⇑}\AgdaSymbol{)} \AgdaSymbol{(}\AgdaFunction{appP} \AgdaSymbol{(}\AgdaInductiveConstructor{var} \AgdaFunction{x₁}\AgdaSymbol{)} \AgdaSymbol{(}\AgdaFunction{appP} \AgdaSymbol{(}\AgdaBound{δ} \AgdaFunction{⇑} \AgdaFunction{⇑}\AgdaSymbol{)} \AgdaSymbol{(}\AgdaInductiveConstructor{var} \AgdaInductiveConstructor{x₀}\AgdaSymbol{)))))))}\<%
\\
\>[0]\AgdaIndent{2}{}\<[2]%
\>[2]\AgdaInductiveConstructor{ref⊃*ref} \AgdaSymbol{:} \AgdaSymbol{∀} \AgdaSymbol{\{}\AgdaBound{V}\AgdaSymbol{\}} \AgdaSymbol{\{}\AgdaBound{φ}\AgdaSymbol{\}} \AgdaSymbol{\{}\AgdaBound{ψ}\AgdaSymbol{\}} \AgdaSymbol{→} \AgdaDatatype{R} \AgdaSymbol{\{}\AgdaBound{V}\AgdaSymbol{\}} \AgdaInductiveConstructor{-imp*} \AgdaSymbol{(}\AgdaFunction{reff} \AgdaBound{φ} \AgdaInductiveConstructor{,,} \AgdaFunction{reff} \AgdaBound{ψ} \AgdaInductiveConstructor{,,} \AgdaInductiveConstructor{out}\AgdaSymbol{)} \AgdaSymbol{(}\AgdaFunction{reff} \AgdaSymbol{(}\AgdaBound{φ} \AgdaFunction{⊃} \AgdaBound{ψ}\AgdaSymbol{))}\<%
\\
\>[0]\AgdaIndent{2}{}\<[2]%
\>[2]\AgdaInductiveConstructor{refref} \AgdaSymbol{:} \AgdaSymbol{∀} \AgdaSymbol{\{}\AgdaBound{V}\AgdaSymbol{\}} \AgdaSymbol{\{}\AgdaBound{M}\AgdaSymbol{\}} \AgdaSymbol{\{}\AgdaBound{N}\AgdaSymbol{\}} \AgdaSymbol{→} \AgdaDatatype{R} \AgdaSymbol{\{}\AgdaBound{V}\AgdaSymbol{\}} \AgdaInductiveConstructor{-app*} \AgdaSymbol{(}\AgdaBound{N} \AgdaInductiveConstructor{,,} \AgdaBound{N} \AgdaInductiveConstructor{,,} \AgdaFunction{reff} \AgdaBound{M} \AgdaInductiveConstructor{,,} \AgdaFunction{reff} \AgdaBound{N} \AgdaInductiveConstructor{,,} \AgdaInductiveConstructor{out}\AgdaSymbol{)} \AgdaSymbol{(}\AgdaFunction{reff} \AgdaSymbol{(}\AgdaFunction{appT} \AgdaBound{M} \AgdaBound{N}\AgdaSymbol{))}\<%
\\
\>[0]\AgdaIndent{2}{}\<[2]%
\>[2]\AgdaInductiveConstructor{βE} \AgdaSymbol{:} \AgdaSymbol{∀} \AgdaSymbol{\{}\AgdaBound{V}\AgdaSymbol{\}} \AgdaSymbol{\{}\AgdaBound{M}\AgdaSymbol{\}} \AgdaSymbol{\{}\AgdaBound{N}\AgdaSymbol{\}} \AgdaSymbol{\{}\AgdaBound{A}\AgdaSymbol{\}} \AgdaSymbol{\{}\AgdaBound{P}\AgdaSymbol{\}} \AgdaSymbol{\{}\AgdaBound{Q}\AgdaSymbol{\}} \AgdaSymbol{→} \AgdaDatatype{R} \AgdaSymbol{\{}\AgdaBound{V}\AgdaSymbol{\}} \AgdaInductiveConstructor{-app*} \AgdaSymbol{(}\AgdaBound{M} \AgdaInductiveConstructor{,,} \AgdaBound{N} \AgdaInductiveConstructor{,,} \AgdaFunction{λλλ} \AgdaBound{A} \AgdaBound{P} \AgdaInductiveConstructor{,,} \AgdaBound{Q} \AgdaInductiveConstructor{,,} \AgdaInductiveConstructor{out}\AgdaSymbol{)} \<[79]%
\>[79]\<%
\\
\>[2]\AgdaIndent{4}{}\<[4]%
\>[4]\AgdaSymbol{(}\AgdaBound{P} \AgdaFunction{⟦} \AgdaFunction{x₂:=} \AgdaBound{M} \AgdaFunction{,x₁:=} \AgdaBound{N} \AgdaFunction{,x₀:=} \AgdaBound{Q} \AgdaFunction{⟧}\AgdaSymbol{)}\<%
\\
\>[0]\AgdaIndent{2}{}\<[2]%
\>[2]\AgdaInductiveConstructor{reflamvar} \AgdaSymbol{:} \AgdaSymbol{∀} \AgdaSymbol{\{}\AgdaBound{V}\AgdaSymbol{\}} \AgdaSymbol{\{}\AgdaBound{N}\AgdaSymbol{\}} \AgdaSymbol{\{}\AgdaBound{N'}\AgdaSymbol{\}} \AgdaSymbol{\{}\AgdaBound{A}\AgdaSymbol{\}} \AgdaSymbol{\{}\AgdaBound{M}\AgdaSymbol{\}} \AgdaSymbol{\{}\AgdaBound{e}\AgdaSymbol{\}} \AgdaSymbol{→} \AgdaDatatype{R} \AgdaSymbol{\{}\AgdaBound{V}\AgdaSymbol{\}} \AgdaInductiveConstructor{-app*} \AgdaSymbol{(}\AgdaBound{N} \AgdaInductiveConstructor{,,} \AgdaBound{N'} \AgdaInductiveConstructor{,,} \AgdaFunction{reff} \AgdaSymbol{(}\AgdaFunction{ΛT} \AgdaBound{A} \AgdaBound{M}\AgdaSymbol{)} \AgdaInductiveConstructor{,,} \AgdaInductiveConstructor{var} \AgdaBound{e} \AgdaInductiveConstructor{,,} \AgdaInductiveConstructor{out}\AgdaSymbol{)} \AgdaSymbol{(}\AgdaBound{M} \AgdaFunction{⟦⟦} \AgdaFunction{x₀::=} \AgdaSymbol{(}\AgdaInductiveConstructor{var} \AgdaBound{e}\AgdaSymbol{)} \AgdaFunction{∶} \AgdaFunction{x₀:=} \AgdaBound{N} \AgdaFunction{∼} \AgdaFunction{x₀:=} \AgdaBound{N'} \AgdaFunction{⟧⟧}\AgdaSymbol{)}\<%
\\
\>[0]\AgdaIndent{2}{}\<[2]%
\>[2]\AgdaInductiveConstructor{reflam⊃*} \AgdaSymbol{:} \AgdaSymbol{∀} \AgdaSymbol{\{}\AgdaBound{V}\AgdaSymbol{\}} \AgdaSymbol{\{}\AgdaBound{N}\AgdaSymbol{\}} \AgdaSymbol{\{}\AgdaBound{N'}\AgdaSymbol{\}} \AgdaSymbol{\{}\AgdaBound{A}\AgdaSymbol{\}} \AgdaSymbol{\{}\AgdaBound{M}\AgdaSymbol{\}} \AgdaSymbol{\{}\AgdaBound{P}\AgdaSymbol{\}} \AgdaSymbol{\{}\AgdaBound{Q}\AgdaSymbol{\}} \AgdaSymbol{→} \AgdaDatatype{R} \AgdaSymbol{\{}\AgdaBound{V}\AgdaSymbol{\}} \AgdaInductiveConstructor{-app*} \AgdaSymbol{(}\AgdaBound{N} \AgdaInductiveConstructor{,,} \AgdaBound{N'} \AgdaInductiveConstructor{,,} \AgdaFunction{reff} \AgdaSymbol{(}\AgdaFunction{ΛT} \AgdaBound{A} \AgdaBound{M}\AgdaSymbol{)} \AgdaInductiveConstructor{,,} \AgdaSymbol{(}\AgdaBound{P} \AgdaFunction{⊃*} \AgdaBound{Q}\AgdaSymbol{)} \AgdaInductiveConstructor{,,} \AgdaInductiveConstructor{out}\AgdaSymbol{)} \AgdaSymbol{(}\AgdaBound{M} \AgdaFunction{⟦⟦} \AgdaFunction{x₀::=} \AgdaSymbol{(}\AgdaBound{P} \AgdaFunction{⊃*} \AgdaBound{Q}\AgdaSymbol{)} \AgdaFunction{∶} \AgdaFunction{x₀:=} \AgdaBound{N} \AgdaFunction{∼} \AgdaFunction{x₀:=} \AgdaBound{N'} \AgdaFunction{⟧⟧}\AgdaSymbol{)}\<%
\\
\>[0]\AgdaIndent{2}{}\<[2]%
\>[2]\AgdaInductiveConstructor{reflamuniv} \AgdaSymbol{:} \AgdaSymbol{∀} \AgdaSymbol{\{}\AgdaBound{V}\AgdaSymbol{\}} \AgdaSymbol{\{}\AgdaBound{N}\AgdaSymbol{\}} \AgdaSymbol{\{}\AgdaBound{N'}\AgdaSymbol{\}} \AgdaSymbol{\{}\AgdaBound{A}\AgdaSymbol{\}} \AgdaSymbol{\{}\AgdaBound{M}\AgdaSymbol{\}} \AgdaSymbol{\{}\AgdaBound{φ}\AgdaSymbol{\}} \AgdaSymbol{\{}\AgdaBound{ψ}\AgdaSymbol{\}} \AgdaSymbol{\{}\AgdaBound{δ}\AgdaSymbol{\}} \AgdaSymbol{\{}\AgdaBound{ε}\AgdaSymbol{\}} \AgdaSymbol{→} \AgdaDatatype{R} \AgdaSymbol{\{}\AgdaBound{V}\AgdaSymbol{\}} \AgdaInductiveConstructor{-app*} \AgdaSymbol{(}\AgdaBound{N} \AgdaInductiveConstructor{,,} \AgdaBound{N'} \AgdaInductiveConstructor{,,} \AgdaFunction{reff} \AgdaSymbol{(}\AgdaFunction{ΛT} \AgdaBound{A} \AgdaBound{M}\AgdaSymbol{)} \AgdaInductiveConstructor{,,} \AgdaFunction{univ} \AgdaBound{φ} \AgdaBound{ψ} \AgdaBound{δ} \AgdaBound{ε} \AgdaInductiveConstructor{,,} \AgdaInductiveConstructor{out}\AgdaSymbol{)} \AgdaSymbol{(}\AgdaBound{M} \AgdaFunction{⟦⟦} \AgdaFunction{x₀::=} \AgdaSymbol{(}\AgdaFunction{univ} \AgdaBound{φ} \AgdaBound{ψ} \AgdaBound{δ} \AgdaBound{ε}\AgdaSymbol{)} \AgdaFunction{∶} \AgdaFunction{x₀:=} \AgdaBound{N} \AgdaFunction{∼} \AgdaFunction{x₀:=} \AgdaBound{N'} \AgdaFunction{⟧⟧}\AgdaSymbol{)}\<%
\\
\>[0]\AgdaIndent{2}{}\<[2]%
\>[2]\AgdaInductiveConstructor{reflamλλλ} \AgdaSymbol{:} \AgdaSymbol{∀} \AgdaSymbol{\{}\AgdaBound{V}\AgdaSymbol{\}} \AgdaSymbol{\{}\AgdaBound{N}\AgdaSymbol{\}} \AgdaSymbol{\{}\AgdaBound{N'}\AgdaSymbol{\}} \AgdaSymbol{\{}\AgdaBound{A}\AgdaSymbol{\}} \AgdaSymbol{\{}\AgdaBound{M}\AgdaSymbol{\}} \AgdaSymbol{\{}\AgdaBound{B}\AgdaSymbol{\}} \AgdaSymbol{\{}\AgdaBound{P}\AgdaSymbol{\}} \AgdaSymbol{→} \AgdaDatatype{R} \AgdaSymbol{\{}\AgdaBound{V}\AgdaSymbol{\}} \AgdaInductiveConstructor{-app*} \AgdaSymbol{(}\AgdaBound{N} \AgdaInductiveConstructor{,,} \AgdaBound{N'} \AgdaInductiveConstructor{,,} \AgdaFunction{reff} \AgdaSymbol{(}\AgdaFunction{ΛT} \AgdaBound{A} \AgdaBound{M}\AgdaSymbol{)} \AgdaInductiveConstructor{,,} \AgdaFunction{λλλ} \AgdaBound{B} \AgdaBound{P} \AgdaInductiveConstructor{,,} \AgdaInductiveConstructor{out}\AgdaSymbol{)} \AgdaSymbol{(}\AgdaBound{M} \AgdaFunction{⟦⟦} \AgdaFunction{x₀::=} \AgdaSymbol{(}\AgdaFunction{λλλ} \AgdaBound{B} \AgdaBound{P}\AgdaSymbol{)} \AgdaFunction{∶} \AgdaFunction{x₀:=} \AgdaBound{N} \AgdaFunction{∼} \AgdaFunction{x₀:=} \AgdaBound{N'} \AgdaFunction{⟧⟧}\AgdaSymbol{)}\<%
\end{code}

Let \emph{reduction} $\twoheadrightarrow$ be the reflexive, transitive closure of $\rightarrow$, and \emph{conversion} $\simeq$ the equivalence relation generated by $\rightarrow$.

\AgdaHide{
\begin{code}%
\>\AgdaKeyword{open} \AgdaKeyword{import} \AgdaModule{Reduction} \AgdaFunction{PHOPL} \AgdaDatatype{R} \AgdaKeyword{public} \<[37]%
\>[37]\<%
\\
%
\\
\>\AgdaKeyword{postulate} \AgdaPostulate{eq-resp-conv} \AgdaSymbol{:} \AgdaSymbol{∀} \AgdaSymbol{\{}\AgdaBound{V}\AgdaSymbol{\}} \AgdaSymbol{\{}\AgdaBound{M} \AgdaBound{M'} \AgdaBound{N} \AgdaBound{N'} \AgdaSymbol{:} \AgdaFunction{Term} \AgdaBound{V}\AgdaSymbol{\}} \AgdaSymbol{\{}\AgdaBound{A} \AgdaSymbol{:} \AgdaDatatype{Type}\AgdaSymbol{\}} \AgdaSymbol{→}\<%
\\
\>[2]\AgdaIndent{23}{}\<[23]%
\>[23]\AgdaBound{M} \AgdaDatatype{≃} \AgdaBound{M'} \AgdaSymbol{→} \AgdaBound{N} \AgdaDatatype{≃} \AgdaBound{N'} \AgdaSymbol{→} \AgdaBound{M} \AgdaFunction{≡〈} \AgdaBound{A} \AgdaFunction{〉} \AgdaBound{N} \AgdaDatatype{≃} \AgdaBound{M'} \AgdaFunction{≡〈} \AgdaBound{A} \AgdaFunction{〉} \AgdaBound{N'}\<%
\\
%
\\
\>\AgdaKeyword{postulate} \AgdaPostulate{R-creates-rep} \AgdaSymbol{:} \AgdaFunction{creates'} \AgdaFunction{replacement}\<%
\\
%
\\
\>\AgdaKeyword{postulate} \AgdaPostulate{R-respects-replacement} \AgdaSymbol{:} \AgdaFunction{respects'} \AgdaFunction{replacement}\<%
\\
%
\\
\>\AgdaKeyword{postulate} \AgdaPostulate{R-creates-replacement} \AgdaSymbol{:} \AgdaFunction{creates'} \AgdaFunction{replacement}\<%
\\
%
\\
\>\AgdaKeyword{postulate} \AgdaPostulate{R-respects-sub} \AgdaSymbol{:} \AgdaFunction{respects'} \AgdaFunction{SUB}\<%
\\
%
\\
\>\AgdaFunction{red-subl} \AgdaSymbol{:} \AgdaSymbol{∀} \AgdaSymbol{\{}\AgdaBound{U}\AgdaSymbol{\}} \AgdaSymbol{\{}\AgdaBound{V}\AgdaSymbol{\}} \AgdaSymbol{\{}\AgdaBound{C}\AgdaSymbol{\}} \AgdaSymbol{\{}\AgdaBound{K}\AgdaSymbol{\}} \AgdaSymbol{\{}\AgdaBound{E} \AgdaBound{F} \AgdaSymbol{:} \AgdaDatatype{Subexpression} \AgdaBound{U} \AgdaBound{C} \AgdaBound{K}\AgdaSymbol{\}} \AgdaSymbol{\{}\AgdaBound{σ} \AgdaSymbol{:} \AgdaFunction{Sub} \AgdaBound{U} \AgdaBound{V}\AgdaSymbol{\}} \AgdaSymbol{→} \AgdaBound{E} \AgdaDatatype{↠} \AgdaBound{F} \AgdaSymbol{→} \AgdaBound{E} \AgdaFunction{⟦} \AgdaBound{σ} \AgdaFunction{⟧} \AgdaDatatype{↠} \AgdaBound{F} \AgdaFunction{⟦} \AgdaBound{σ} \AgdaFunction{⟧}\<%
\\
\>\AgdaFunction{red-subl} \AgdaBound{E↠F} \AgdaSymbol{=} \AgdaFunction{respects-red} \AgdaSymbol{(}\AgdaFunction{respects-osr} \AgdaFunction{SUB} \AgdaPostulate{R-respects-sub}\AgdaSymbol{)} \AgdaBound{E↠F}\<%
\\
%
\\
\>\AgdaKeyword{postulate} \AgdaPostulate{red-subr} \AgdaSymbol{:} \AgdaSymbol{∀} \AgdaSymbol{\{}\AgdaBound{U}\AgdaSymbol{\}} \AgdaSymbol{\{}\AgdaBound{V}\AgdaSymbol{\}} \AgdaSymbol{\{}\AgdaBound{C}\AgdaSymbol{\}} \AgdaSymbol{\{}\AgdaBound{K}\AgdaSymbol{\}} \AgdaSymbol{(}\AgdaBound{E} \AgdaSymbol{:} \AgdaDatatype{Subexpression} \AgdaBound{U} \AgdaBound{C} \AgdaBound{K}\AgdaSymbol{)} \AgdaSymbol{\{}\AgdaBound{ρ} \AgdaBound{σ} \AgdaSymbol{:} \AgdaFunction{Sub} \AgdaBound{U} \AgdaBound{V}\AgdaSymbol{\}} \AgdaSymbol{→} \AgdaFunction{\_↠s\_} \AgdaFunction{SUB} \AgdaBound{ρ} \AgdaBound{σ} \AgdaSymbol{→} \AgdaBound{E} \AgdaFunction{⟦} \AgdaBound{ρ} \AgdaFunction{⟧} \AgdaDatatype{↠} \AgdaBound{E} \AgdaFunction{⟦} \AgdaBound{σ} \AgdaFunction{⟧}\<%
\\
%
\\
\>\AgdaKeyword{postulate} \AgdaPostulate{⊥SN} \AgdaSymbol{:} \AgdaSymbol{∀} \AgdaSymbol{\{}\AgdaBound{V}\AgdaSymbol{\}} \AgdaSymbol{→} \AgdaDatatype{SN} \AgdaSymbol{\{}\AgdaBound{V}\AgdaSymbol{\}} \AgdaFunction{⊥}\<%
\\
%
\\
\>\AgdaKeyword{postulate} \AgdaPostulate{⊃SN} \AgdaSymbol{:} \AgdaSymbol{∀} \AgdaSymbol{\{}\AgdaBound{V}\AgdaSymbol{\}} \AgdaSymbol{\{}\AgdaBound{φ} \AgdaBound{ψ} \AgdaSymbol{:} \AgdaFunction{Term} \AgdaBound{V}\AgdaSymbol{\}} \AgdaSymbol{→} \AgdaDatatype{SN} \AgdaBound{φ} \AgdaSymbol{→} \AgdaDatatype{SN} \AgdaBound{ψ} \AgdaSymbol{→} \AgdaDatatype{SN} \AgdaSymbol{(}\AgdaBound{φ} \AgdaFunction{⊃} \AgdaBound{ψ}\AgdaSymbol{)}\<%
\\
%
\\
\>\AgdaKeyword{postulate} \AgdaPostulate{SN-βexp} \AgdaSymbol{:} \AgdaSymbol{∀} \AgdaSymbol{\{}\AgdaBound{V}\AgdaSymbol{\}} \AgdaSymbol{\{}\AgdaBound{φ} \AgdaSymbol{:} \AgdaFunction{Term} \AgdaBound{V}\AgdaSymbol{\}} \AgdaSymbol{\{}\AgdaBound{δ} \AgdaSymbol{:} \AgdaFunction{Proof} \AgdaSymbol{(}\AgdaBound{V} \AgdaInductiveConstructor{,} \AgdaInductiveConstructor{-Proof}\AgdaSymbol{)\}} \AgdaSymbol{\{}\AgdaBound{ε} \AgdaSymbol{:} \AgdaFunction{Proof} \AgdaBound{V}\AgdaSymbol{\}} \AgdaSymbol{→}\<%
\\
\>[0]\AgdaIndent{18}{}\<[18]%
\>[18]\AgdaDatatype{SN} \AgdaBound{ε} \AgdaSymbol{→} \AgdaDatatype{SN} \AgdaSymbol{(}\AgdaBound{δ} \AgdaFunction{⟦} \AgdaFunction{x₀:=} \AgdaBound{ε} \AgdaFunction{⟧}\AgdaSymbol{)} \AgdaSymbol{→} \AgdaDatatype{SN} \AgdaSymbol{(}\AgdaFunction{appP} \AgdaSymbol{(}\AgdaFunction{ΛP} \AgdaBound{φ} \AgdaBound{δ}\AgdaSymbol{)} \AgdaBound{ε}\AgdaSymbol{)} \<[66]%
\>[66]\<%
\\
%
\\
\>\AgdaKeyword{postulate} \AgdaPostulate{univ-red} \AgdaSymbol{:} \AgdaSymbol{∀} \AgdaSymbol{\{}\AgdaBound{V}\AgdaSymbol{\}} \AgdaSymbol{\{}\AgdaBound{φ} \AgdaBound{φ'} \AgdaBound{ψ} \AgdaBound{ψ'} \AgdaSymbol{:} \AgdaFunction{Term} \AgdaBound{V}\AgdaSymbol{\}} \AgdaSymbol{\{}\AgdaBound{δ}\AgdaSymbol{\}} \AgdaSymbol{\{}\AgdaBound{δ'}\AgdaSymbol{\}} \AgdaSymbol{\{}\AgdaBound{ε}\AgdaSymbol{\}} \AgdaSymbol{\{}\AgdaBound{ε'}\AgdaSymbol{\}} \AgdaSymbol{→} \<[68]%
\>[68]\<%
\\
\>[18]\AgdaIndent{19}{}\<[19]%
\>[19]\AgdaBound{φ} \AgdaDatatype{↠} \AgdaBound{φ'} \AgdaSymbol{→} \AgdaBound{ψ} \AgdaDatatype{↠} \AgdaBound{ψ'} \AgdaSymbol{→} \AgdaBound{δ} \AgdaDatatype{↠} \AgdaBound{δ'} \AgdaSymbol{→} \AgdaBound{ε} \AgdaDatatype{↠} \AgdaBound{ε'} \AgdaSymbol{→} \AgdaFunction{univ} \AgdaBound{φ} \AgdaBound{ψ} \AgdaBound{δ} \AgdaBound{ε} \AgdaDatatype{↠} \AgdaFunction{univ} \AgdaBound{φ'} \AgdaBound{ψ'} \AgdaBound{δ'} \AgdaBound{ε'}\<%
\\
%
\\
\>\AgdaKeyword{postulate} \AgdaPostulate{ΛP-red} \AgdaSymbol{:} \AgdaSymbol{∀} \AgdaSymbol{\{}\AgdaBound{V}\AgdaSymbol{\}} \AgdaSymbol{\{}\AgdaBound{φ} \AgdaBound{φ'} \AgdaSymbol{:} \AgdaFunction{Term} \AgdaBound{V}\AgdaSymbol{\}} \AgdaSymbol{\{}\AgdaBound{δ}\AgdaSymbol{\}} \AgdaSymbol{\{}\AgdaBound{δ'}\AgdaSymbol{\}} \AgdaSymbol{→} \AgdaBound{φ} \AgdaDatatype{↠} \AgdaBound{φ'} \AgdaSymbol{→} \AgdaBound{δ} \AgdaDatatype{↠} \AgdaBound{δ'} \AgdaSymbol{→} \AgdaFunction{ΛP} \AgdaBound{φ} \AgdaBound{δ} \AgdaDatatype{↠} \AgdaFunction{ΛP} \AgdaBound{φ'} \AgdaBound{δ'}\<%
\\
%
\\
\>\AgdaKeyword{postulate} \AgdaPostulate{pre-Confluent} \AgdaSymbol{:} \AgdaSymbol{∀} \AgdaSymbol{\{}\AgdaBound{V}\AgdaSymbol{\}} \AgdaSymbol{\{}\AgdaBound{K}\AgdaSymbol{\}} \AgdaSymbol{\{}\AgdaBound{C}\AgdaSymbol{\}} \AgdaSymbol{\{}\AgdaBound{c} \AgdaSymbol{:} \AgdaFunction{Constructor} \AgdaBound{C}\AgdaSymbol{\}} \AgdaSymbol{\{}\AgdaBound{E} \AgdaBound{E'} \AgdaSymbol{:} \AgdaFunction{Body} \AgdaBound{V} \AgdaBound{C}\AgdaSymbol{\}} \AgdaSymbol{\{}\AgdaBound{F}\AgdaSymbol{\}} \AgdaSymbol{→}\<%
\\
\>[19]\AgdaIndent{24}{}\<[24]%
\>[24]\AgdaDatatype{R} \AgdaBound{c} \AgdaBound{E} \AgdaBound{F} \AgdaSymbol{→} \AgdaBound{E} \AgdaDatatype{⇒} \AgdaBound{E'} \AgdaSymbol{→} \AgdaFunction{Σ[} \AgdaBound{F'} \AgdaFunction{∈} \AgdaFunction{Expression} \AgdaBound{V} \AgdaBound{K} \AgdaFunction{]} \AgdaDatatype{R} \AgdaBound{c} \AgdaBound{E'} \AgdaBound{F'} \AgdaFunction{×} \AgdaBound{F} \AgdaDatatype{↠} \AgdaBound{F'}\<%
\\
\>\AgdaComment{\{-pre-Confluent βT (appl (redex ()))\<\\
\>pre-Confluent βT (appl (app (appl M⇒M'))) = \_ ,p βT ,p red-subl (osr-red M⇒M')\<\\
\>pre-Confluent βT (appl (app (appr ())))\<\\
\>pre-Confluent (βT \{M = M\}) (appr (appl N⇒N')) = \_ ,p βT ,p red-subr M (botsub-red N⇒N')\<\\
\>pre-Confluent βT (appr (appr ()))\<\\
\>pre-Confluent βR (appl (redex ()))\<\\
\>pre-Confluent βR (appl (app (appl \_))) = \_ ,p βR ,p ref\<\\
\>pre-Confluent βR (appl (app (appr (appl δ⇒δ')))) = \_ ,p βR ,p red-subl (osr-red δ⇒δ')\<\\
\>pre-Confluent βR (appl (app (appr (appr ()))))\<\\
\>pre-Confluent (βR \{δ = δ\}) (appr (appl ε⇒ε')) = \_ ,p βR ,p red-subr δ (botsub-red ε⇒ε')\<\\
\>pre-Confluent βR (appr (appr ()))\<\\
\>pre-Confluent plus-ref (appl (redex ()))\<\\
\>pre-Confluent plus-ref (appl (app (appl φ⇒φ'))) = \_ ,p plus-ref ,p osr-red (app (appl φ⇒φ'))\<\\
\>pre-Confluent plus-ref (appl (app (appr ())))\<\\
\>pre-Confluent plus-ref (appr ())\<\\
\>pre-Confluent minus-ref (appl (redex ()))\<\\
\>pre-Confluent minus-ref (appl (app (appl φ⇒φ'))) = \_ ,p minus-ref ,p osr-red (app (appl φ⇒φ'))\<\\
\>pre-Confluent minus-ref (appl (app (appr ())))\<\\
\>pre-Confluent minus-ref (appr ())\<\\
\>pre-Confluent plus-univ (appl (redex ()))\<\\
\>pre-Confluent plus-univ (appl (app (appl \_))) = \_ ,p plus-univ ,p ref\<\\
\>pre-Confluent plus-univ (appl (app (appr (appl \_)))) = \_ ,p plus-univ ,p ref\<\\
\>pre-Confluent plus-univ (appl (app (appr (appr (appl δ⇒δ'))))) = \_ ,p plus-univ ,p osr-red δ⇒δ'\<\\
\>pre-Confluent plus-univ (appl (app (appr (appr (appr (appl \_)))))) = \_ ,p plus-univ ,p ref\<\\
\>pre-Confluent plus-univ (appl (app (appr (appr (appr (appr ()))))))\<\\
\>pre-Confluent plus-univ (appr ())\<\\
\>pre-Confluent minus-univ (appl (redex ()))\<\\
\>pre-Confluent minus-univ (appl (app (appl \_))) = \_ ,p minus-univ ,p ref\<\\
\>pre-Confluent minus-univ (appl (app (appr (appl \_)))) = \_ ,p minus-univ ,p ref\<\\
\>pre-Confluent minus-univ (appl (app (appr (appr (appl \_))))) = \_ ,p minus-univ ,p ref\<\\
\>pre-Confluent minus-univ (appl (app (appr (appr (appr (appl ε⇒ε')))))) = \_ ,p minus-univ ,p osr-red ε⇒ε'\<\\
\>pre-Confluent minus-univ (appl (app (appr (appr (appr (appr ()))))))\<\\
\>pre-Confluent minus-univ (appr ())\<\\
\>pre-Confluent ref⊃*univ (appl (redex ()))\<\\
\>pre-Confluent ref⊃*univ (appl (app (appl \{E = φ\} \{E' = φ'\} φ⇒φ'))) = \<\\
\>  let φ⊃ψ↠φ'⊃ψ : ∀ x → (φ ⊃ x) ↠ (φ' ⊃ x)\<\\
\>      φ⊃ψ↠φ'⊃ψ = λ \_ → osr-red (app (appl φ⇒φ')) in\<\\
\>  \_ ,p ref⊃*univ ,p univ-red (φ⊃ψ↠φ'⊃ψ \_) (φ⊃ψ↠φ'⊃ψ \_) (ΛP-red (φ⊃ψ↠φ'⊃ψ \_) (osr-red (app (appl (respects-osr replacement R-respects-replacement φ⇒φ'))))) \<\\
\>  (ΛP-red (φ⊃ψ↠φ'⊃ψ \_) (osr-red (app (appl (respects-osr replacement R-respects-replacement φ⇒φ')))))\<\\
\>pre-Confluent ref⊃*univ (appl (app (appr ())))\<\\
\>pre-Confluent ref⊃*univ (appr (appl (redex ())))\<\\
\>pre-Confluent ref⊃*univ (appr (appl (app (appl ψ⇒ψ')))) = \_ ,p ref⊃*univ ,p (univ-red (osr-red (app (appr (appl ψ⇒ψ')))) ref \<\\
\>  (ΛP-red (osr-red (app (appr (appl ψ⇒ψ')))) ref) ref)\<\\
\>pre-Confluent ref⊃*univ (appr (appl (app (appr (appl χ⇒χ'))))) = \_ ,p (ref⊃*univ ,p (univ-red ref (osr-red (app (appr (appl χ⇒χ')))) \<\\
\>  ref (osr-red (app (appl (app (appr (appl χ⇒χ'))))))))\<\\
\>pre-Confluent ref⊃*univ (appr (appl (app (appr (appr (appl δ⇒δ')))))) = \_ ,p ref⊃*univ ,p osr-red (app (appr (appr (appl (app (appr (appl (app (appr (appl \<\\
\>  (app (appl (respects-osr replacement R-respects-replacement (respects-osr replacement R-respects-replacement δ⇒δ'))))))))))))))\<\\
\>pre-Confluent ref⊃*univ (appr (appl (app (appr (appr (appr (appl ε⇒ε'))))))) = \_ ,p ref⊃*univ ,p osr-red (app (appr (appr (appr (appl (app (appr (appl (app (appr \<\\
\>  (appl (app (appl (respects-osr replacement R-respects-replacement (respects-osr replacement R-respects-replacement ε⇒ε')))))))))))))))\<\\
\>pre-Confluent ref⊃*univ (appr (appl (app (appr (appr (appr (appr ())))))))\<\\
\>pre-Confluent ref⊃*univ (appr (appr ()))\<\\
\>pre-Confluent univ⊃*ref E⇒E' = \{!!\}\<\\
\>pre-Confluent univ⊃*univ E⇒E' = \{!!\}\<\\
\>pre-Confluent ref⊃*ref E⇒E' = \{!!\}\<\\
\>pre-Confluent refref E⇒E' = \{!!\}\<\\
\>pre-Confluent βE E⇒E' = \{!!\}\<\\
\>pre-Confluent reflamvar E⇒E' = \{!!\}\<\\
\>pre-Confluent reflam⊃* E⇒E' = \{!!\}\<\\
\>pre-Confluent reflamuniv E⇒E' = \{!!\}\<\\
\>pre-Confluent reflamλλλ E⇒E' = \{!!\} -\}}\<%
\\
%
\\
\>\AgdaFunction{Critical-Pairs} \AgdaSymbol{:} \AgdaSymbol{∀} \AgdaSymbol{\{}\AgdaBound{V}\AgdaSymbol{\}} \AgdaSymbol{\{}\AgdaBound{K}\AgdaSymbol{\}} \AgdaSymbol{\{}\AgdaBound{C}\AgdaSymbol{\}} \AgdaSymbol{\{}\AgdaBound{c} \AgdaSymbol{:} \AgdaFunction{Constructor} \AgdaBound{C}\AgdaSymbol{\}} \AgdaSymbol{\{}\AgdaBound{E} \AgdaSymbol{:} \AgdaFunction{Body} \AgdaBound{V} \AgdaBound{C}\AgdaSymbol{\}} \AgdaSymbol{\{}\AgdaBound{F}\AgdaSymbol{\}} \AgdaSymbol{\{}\AgdaBound{G}\AgdaSymbol{\}} \AgdaSymbol{→} \AgdaDatatype{R} \AgdaBound{c} \AgdaBound{E} \AgdaBound{F} \AgdaSymbol{→} \AgdaDatatype{R} \AgdaBound{c} \AgdaBound{E} \AgdaBound{G} \AgdaSymbol{→} \AgdaFunction{Σ[} \AgdaBound{H} \AgdaFunction{∈} \AgdaFunction{Expression} \AgdaBound{V} \AgdaBound{K} \AgdaFunction{]} \AgdaBound{F} \AgdaDatatype{↠} \AgdaBound{H} \AgdaFunction{×} \AgdaBound{G} \AgdaDatatype{↠} \AgdaBound{H}\<%
\\
\>\AgdaFunction{Critical-Pairs} \AgdaInductiveConstructor{βT} \AgdaInductiveConstructor{βT} \AgdaSymbol{=} \AgdaSymbol{\_} \AgdaInductiveConstructor{,p} \AgdaInductiveConstructor{ref} \AgdaInductiveConstructor{,p} \AgdaInductiveConstructor{ref}\<%
\\
\>\AgdaFunction{Critical-Pairs} \AgdaInductiveConstructor{βR} \AgdaInductiveConstructor{βR} \AgdaSymbol{=} \AgdaSymbol{\_} \AgdaInductiveConstructor{,p} \AgdaInductiveConstructor{ref} \AgdaInductiveConstructor{,p} \AgdaInductiveConstructor{ref}\<%
\\
\>\AgdaFunction{Critical-Pairs} \AgdaInductiveConstructor{plus-ref} \AgdaInductiveConstructor{plus-ref} \AgdaSymbol{=} \AgdaSymbol{\_} \AgdaInductiveConstructor{,p} \AgdaInductiveConstructor{ref} \AgdaInductiveConstructor{,p} \AgdaInductiveConstructor{ref}\<%
\\
\>\AgdaFunction{Critical-Pairs} \AgdaInductiveConstructor{minus-ref} \AgdaInductiveConstructor{minus-ref} \AgdaSymbol{=} \AgdaSymbol{\_} \AgdaInductiveConstructor{,p} \AgdaInductiveConstructor{ref} \AgdaInductiveConstructor{,p} \AgdaInductiveConstructor{ref}\<%
\\
\>\AgdaFunction{Critical-Pairs} \AgdaInductiveConstructor{plus-univ} \AgdaInductiveConstructor{plus-univ} \AgdaSymbol{=} \AgdaSymbol{\_} \AgdaInductiveConstructor{,p} \AgdaInductiveConstructor{ref} \AgdaInductiveConstructor{,p} \AgdaInductiveConstructor{ref}\<%
\\
\>\AgdaFunction{Critical-Pairs} \AgdaInductiveConstructor{minus-univ} \AgdaInductiveConstructor{minus-univ} \AgdaSymbol{=} \AgdaSymbol{\_} \AgdaInductiveConstructor{,p} \AgdaInductiveConstructor{ref} \AgdaInductiveConstructor{,p} \AgdaInductiveConstructor{ref}\<%
\\
\>\AgdaFunction{Critical-Pairs} \AgdaInductiveConstructor{ref⊃*univ} \AgdaInductiveConstructor{ref⊃*univ} \AgdaSymbol{=} \AgdaSymbol{\_} \AgdaInductiveConstructor{,p} \AgdaInductiveConstructor{ref} \AgdaInductiveConstructor{,p} \AgdaInductiveConstructor{ref}\<%
\\
\>\AgdaFunction{Critical-Pairs} \AgdaInductiveConstructor{univ⊃*ref} \AgdaInductiveConstructor{univ⊃*ref} \AgdaSymbol{=} \AgdaSymbol{\_} \AgdaInductiveConstructor{,p} \AgdaInductiveConstructor{ref} \AgdaInductiveConstructor{,p} \AgdaInductiveConstructor{ref}\<%
\\
\>\AgdaFunction{Critical-Pairs} \AgdaInductiveConstructor{univ⊃*univ} \AgdaInductiveConstructor{univ⊃*univ} \AgdaSymbol{=} \AgdaSymbol{\_} \AgdaInductiveConstructor{,p} \AgdaInductiveConstructor{ref} \AgdaInductiveConstructor{,p} \AgdaInductiveConstructor{ref}\<%
\\
\>\AgdaFunction{Critical-Pairs} \AgdaInductiveConstructor{ref⊃*ref} \AgdaInductiveConstructor{ref⊃*ref} \AgdaSymbol{=} \AgdaSymbol{\_} \AgdaInductiveConstructor{,p} \AgdaInductiveConstructor{ref} \AgdaInductiveConstructor{,p} \AgdaInductiveConstructor{ref}\<%
\\
\>\AgdaFunction{Critical-Pairs} \AgdaInductiveConstructor{refref} \AgdaInductiveConstructor{refref} \AgdaSymbol{=} \AgdaSymbol{\_} \AgdaInductiveConstructor{,p} \AgdaInductiveConstructor{ref} \AgdaInductiveConstructor{,p} \AgdaInductiveConstructor{ref}\<%
\\
\>\AgdaFunction{Critical-Pairs} \AgdaInductiveConstructor{βE} \AgdaInductiveConstructor{βE} \AgdaSymbol{=} \AgdaSymbol{\_} \AgdaInductiveConstructor{,p} \AgdaInductiveConstructor{ref} \AgdaInductiveConstructor{,p} \AgdaInductiveConstructor{ref}\<%
\\
\>\AgdaFunction{Critical-Pairs} \AgdaInductiveConstructor{reflamvar} \AgdaInductiveConstructor{reflamvar} \AgdaSymbol{=} \AgdaSymbol{\_} \AgdaInductiveConstructor{,p} \AgdaInductiveConstructor{ref} \AgdaInductiveConstructor{,p} \AgdaInductiveConstructor{ref}\<%
\\
\>\AgdaFunction{Critical-Pairs} \AgdaInductiveConstructor{reflam⊃*} \AgdaInductiveConstructor{reflam⊃*} \AgdaSymbol{=} \AgdaSymbol{\_} \AgdaInductiveConstructor{,p} \AgdaInductiveConstructor{ref} \AgdaInductiveConstructor{,p} \AgdaInductiveConstructor{ref}\<%
\\
\>\AgdaFunction{Critical-Pairs} \AgdaInductiveConstructor{reflamuniv} \AgdaInductiveConstructor{reflamuniv} \AgdaSymbol{=} \AgdaSymbol{\_} \AgdaInductiveConstructor{,p} \AgdaInductiveConstructor{ref} \AgdaInductiveConstructor{,p} \AgdaInductiveConstructor{ref}\<%
\\
\>\AgdaFunction{Critical-Pairs} \AgdaInductiveConstructor{reflamλλλ} \AgdaInductiveConstructor{reflamλλλ} \AgdaSymbol{=} \AgdaSymbol{\_} \AgdaInductiveConstructor{,p} \AgdaInductiveConstructor{ref} \AgdaInductiveConstructor{,p} \AgdaInductiveConstructor{ref}\<%
\end{code}
}

\paragraph{Note}
\begin{enumerate}
\item
Contraction is a relation between closed terms only: if $E \rhd F$ then $E$ and $F$ are closed.  This is not true for $\rightarrow$, $\twoheadrightarrow$ or $\simeq$, however.  For example, we have $\reff{\bot}^+ x \rightarrow (\lambda p:\bot.p)x$.
\item
This relation does not play any part in the rules of deduction of $\lambda o e$.
The relation $\simeq_\beta$ that appears in the rules of $\lambda o e$ is the usual $\beta$-convertibility relation.  In particular, we \emph{do} allow $\beta$-contraction of open terms there: thus, the judgement $x : \Omega, p : (\lambda y : \Omega . y) x \vdash p : x$ is derivable in $\lambda o e$.  We shall always use the subscript $\beta$ when referring to this relation.
\end{enumerate}

\begin{lm}
$ $
\begin{enumerate}
\item
If $t \rightarrow t'$ then $t[x:=s] \rightarrow t'[x:=s]$.
\item
If $t \rightarrow t'$ then $s[x:=t] \twoheadrightarrow s[x:=t']$.
\item
If $M \rightarrow M'$ then $M \{ x:=P : N \sim N' \} \twoheadrightarrow M' \{ x:=P : N \sim N' \}$.
\item
If $P \rightarrow P'$ then $M \{ x:= P : N \sim N' \} \twoheadrightarrow M \{ x:=P' : N \sim N' \}$.
\end{enumerate}
\end{lm}

\begin{proof}
A straightforward induction in each case.  We give the details for one case in part 3, where $M \equiv (\lambda y:A.L)L'$ and $M' \equiv L[y:=L']$.  We have
\begin{align*}
((\lambda y:A.L)L') \{ x:=P \} & \eqdef (\triplelambda e : y =_A y' . L \{ x := P, y := e \})_{L'[x:=N] L'[x:=N']} L'\{ x:= P \} \\
& \rightarrow L \{ x:= P : N \sim N' , y := e : y \sim y' \} [ y := L' [x:=N], y' := L'[x:=Nl], e := L' \{ x:=P \} ] \\
& \equiv L \{ x := P : N \sim N' , y := L' \{ x := P \} : L' [ x := N ] \sim L' [ x := N' ] \} \\
& \equiv L [ y := L' ] \{ x := P \}
\end{align*}
\end{proof}

\begin{prop}
Reduction satisfies the diamond property: if $E \rightarrow F$ and $E \rightarrow G$ then there exists $H$ such that $F \rightarrow H$ and $G \rightarrow H$.
\end{prop}

\begin{proof}
Case analysis on $E \rightarrow F$ and $E \rightarrow G$.  There are no critical pairs thanks to our restriction that, if $E \rhd F$, then all proper subterms of $E$ are normal forms.
\end{proof}

\begin{cor}[Confluence]
$ $
\begin{enumerate}
\item
The reduction relation is confluent: if $E \twoheadrightarrow F$ and $E \twoheadrightarrow G$, then there exists $H$ such that $F \twoheadrightarrow H$ and $G \twoheadrightarrow H$.
\item
If $E \simeq F$, then there exists $G$ such that $E \twoheadrightarrow G$ and $F \twoheadrightarrow G$.
\end{enumerate}
\end{cor}

%<*Local-Confluent>
\begin{code}%
\>\AgdaFunction{Local-Confluent} \AgdaSymbol{:} \AgdaSymbol{∀} \AgdaSymbol{\{}\AgdaBound{V}\AgdaSymbol{\}} \AgdaSymbol{\{}\AgdaBound{C}\AgdaSymbol{\}} \AgdaSymbol{\{}\AgdaBound{K}\AgdaSymbol{\}} \<[32]%
\>[32]\<%
\\
\>[0]\AgdaIndent{2}{}\<[2]%
\>[2]\AgdaSymbol{\{}\AgdaBound{E} \AgdaBound{F} \AgdaBound{G} \AgdaSymbol{:} \AgdaDatatype{Subexpression} \AgdaBound{V} \AgdaBound{C} \AgdaBound{K}\AgdaSymbol{\}} \AgdaSymbol{→} \AgdaBound{E} \AgdaDatatype{⇒} \AgdaBound{F} \AgdaSymbol{→} \AgdaBound{E} \AgdaDatatype{⇒} \AgdaBound{G} \AgdaSymbol{→} \<[50]%
\>[50]\<%
\\
\>[0]\AgdaIndent{2}{}\<[2]%
\>[2]\AgdaFunction{Σ[} \AgdaBound{H} \AgdaFunction{∈} \AgdaDatatype{Subexpression} \AgdaBound{V} \AgdaBound{C} \AgdaBound{K} \AgdaFunction{]} \AgdaSymbol{(}\AgdaBound{F} \AgdaDatatype{↠} \AgdaBound{H} \AgdaFunction{×} \AgdaBound{G} \AgdaDatatype{↠} \AgdaBound{H}\AgdaSymbol{)}\<%
\end{code}
%</Local-Confluent>

\AgdaHide{
\begin{code}%
\>\AgdaFunction{Local-Confluent} \AgdaSymbol{(}\AgdaInductiveConstructor{redex} \AgdaBound{x}\AgdaSymbol{)} \AgdaSymbol{(}\AgdaInductiveConstructor{redex} \AgdaBound{y}\AgdaSymbol{)} \AgdaSymbol{=} \AgdaFunction{Critical-Pairs} \AgdaBound{x} \AgdaBound{y}\<%
\\
\>\AgdaFunction{Local-Confluent} \AgdaSymbol{(}\AgdaInductiveConstructor{redex} \AgdaBound{r}\AgdaSymbol{)} \AgdaSymbol{(}\AgdaInductiveConstructor{app} \AgdaBound{E⇒G}\AgdaSymbol{)} \AgdaSymbol{=} \AgdaKeyword{let} \AgdaSymbol{(}\AgdaBound{H} \AgdaInductiveConstructor{,p} \AgdaBound{r⇒H} \AgdaInductiveConstructor{,p} \AgdaBound{G↠H}\AgdaSymbol{)} \AgdaSymbol{=} \AgdaPostulate{pre-Confluent} \AgdaBound{r} \AgdaBound{E⇒G} \AgdaKeyword{in} \<[85]%
\>[85]\<%
\\
\>[0]\AgdaIndent{2}{}\<[2]%
\>[2]\AgdaBound{H} \AgdaInductiveConstructor{,p} \AgdaBound{G↠H} \AgdaInductiveConstructor{,p} \AgdaInductiveConstructor{osr-red} \AgdaSymbol{(}\AgdaInductiveConstructor{redex} \AgdaBound{r⇒H}\AgdaSymbol{)}\<%
\\
\>\AgdaFunction{Local-Confluent} \AgdaSymbol{(}\AgdaInductiveConstructor{app} \AgdaBound{E⇒F}\AgdaSymbol{)} \AgdaSymbol{(}\AgdaInductiveConstructor{redex} \AgdaBound{r}\AgdaSymbol{)} \AgdaSymbol{=} \AgdaKeyword{let} \AgdaSymbol{(}\AgdaBound{H} \AgdaInductiveConstructor{,p} \AgdaBound{r⇒H} \AgdaInductiveConstructor{,p} \AgdaBound{G↠H}\AgdaSymbol{)} \AgdaSymbol{=} \AgdaPostulate{pre-Confluent} \AgdaBound{r} \AgdaBound{E⇒F} \AgdaKeyword{in} \<[85]%
\>[85]\<%
\\
\>[0]\AgdaIndent{2}{}\<[2]%
\>[2]\AgdaBound{H} \AgdaInductiveConstructor{,p} \AgdaInductiveConstructor{osr-red} \AgdaSymbol{(}\AgdaInductiveConstructor{redex} \AgdaBound{r⇒H}\AgdaSymbol{)} \AgdaInductiveConstructor{,p} \AgdaBound{G↠H}\<%
\\
\>\AgdaFunction{Local-Confluent} \AgdaSymbol{(}\AgdaInductiveConstructor{app} \AgdaBound{E⇒F}\AgdaSymbol{)} \AgdaSymbol{(}\AgdaInductiveConstructor{app} \AgdaBound{E⇒G}\AgdaSymbol{)} \AgdaSymbol{=} \AgdaKeyword{let} \AgdaSymbol{(}\AgdaBound{H} \AgdaInductiveConstructor{,p} \AgdaBound{F↠H} \AgdaInductiveConstructor{,p} \AgdaBound{G↠H}\AgdaSymbol{)} \AgdaSymbol{=} \AgdaFunction{Local-Confluent} \AgdaBound{E⇒F} \AgdaBound{E⇒G} \AgdaKeyword{in} \<[89]%
\>[89]\<%
\\
\>[0]\AgdaIndent{2}{}\<[2]%
\>[2]\AgdaInductiveConstructor{app} \AgdaSymbol{\_} \AgdaBound{H} \AgdaInductiveConstructor{,p} \AgdaFunction{respects-red} \AgdaInductiveConstructor{app} \AgdaBound{F↠H} \AgdaInductiveConstructor{,p} \AgdaFunction{respects-red} \AgdaInductiveConstructor{app} \AgdaBound{G↠H}\<%
\\
\>\AgdaFunction{Local-Confluent} \AgdaSymbol{(}\AgdaInductiveConstructor{appl} \AgdaBound{E⇒F}\AgdaSymbol{)} \AgdaSymbol{(}\AgdaInductiveConstructor{appl} \AgdaBound{E⇒G}\AgdaSymbol{)} \AgdaSymbol{=} \AgdaKeyword{let} \AgdaSymbol{(}\AgdaBound{H} \AgdaInductiveConstructor{,p} \AgdaBound{F↠H} \AgdaInductiveConstructor{,p} \AgdaBound{G↠H}\AgdaSymbol{)} \AgdaSymbol{=} \AgdaFunction{Local-Confluent} \AgdaBound{E⇒F} \AgdaBound{E⇒G} \AgdaKeyword{in} \<[91]%
\>[91]\<%
\\
\>[0]\AgdaIndent{2}{}\<[2]%
\>[2]\AgdaSymbol{(}\AgdaBound{H} \AgdaInductiveConstructor{,,} \AgdaSymbol{\_)} \AgdaInductiveConstructor{,p} \AgdaFunction{respects-red} \AgdaInductiveConstructor{appl} \AgdaBound{F↠H} \AgdaInductiveConstructor{,p} \AgdaFunction{respects-red} \AgdaInductiveConstructor{appl} \AgdaBound{G↠H}\<%
\\
\>\AgdaFunction{Local-Confluent} \AgdaSymbol{(}\AgdaInductiveConstructor{appl} \AgdaSymbol{\{}\AgdaArgument{E'} \AgdaSymbol{=} \AgdaBound{E'}\AgdaSymbol{\}} \AgdaBound{E⇒F}\AgdaSymbol{)} \AgdaSymbol{(}\AgdaInductiveConstructor{appr} \AgdaSymbol{\{}\AgdaArgument{F'} \AgdaSymbol{=} \AgdaBound{F'}\AgdaSymbol{\}} \AgdaBound{E⇒G}\AgdaSymbol{)} \AgdaSymbol{=} \AgdaSymbol{(}\AgdaBound{E'} \AgdaInductiveConstructor{,,} \AgdaBound{F'}\AgdaSymbol{)} \AgdaInductiveConstructor{,p} \AgdaInductiveConstructor{osr-red} \AgdaSymbol{(}\AgdaInductiveConstructor{appr} \AgdaBound{E⇒G}\AgdaSymbol{)} \AgdaInductiveConstructor{,p} \AgdaInductiveConstructor{osr-red} \AgdaSymbol{(}\AgdaInductiveConstructor{appl} \AgdaBound{E⇒F}\AgdaSymbol{)}\<%
\\
\>\AgdaFunction{Local-Confluent} \AgdaSymbol{(}\AgdaInductiveConstructor{appr} \AgdaSymbol{\{}\AgdaArgument{F'} \AgdaSymbol{=} \AgdaBound{F'}\AgdaSymbol{\}} \AgdaBound{E⇒F}\AgdaSymbol{)} \AgdaSymbol{(}\AgdaInductiveConstructor{appl} \AgdaSymbol{\{}\AgdaArgument{E'} \AgdaSymbol{=} \AgdaBound{E'}\AgdaSymbol{\}} \AgdaBound{E⇒G}\AgdaSymbol{)} \AgdaSymbol{=} \AgdaBound{E'} \AgdaInductiveConstructor{,,} \AgdaBound{F'} \AgdaInductiveConstructor{,p} \AgdaSymbol{(}\AgdaInductiveConstructor{osr-red} \AgdaSymbol{(}\AgdaInductiveConstructor{appl} \AgdaBound{E⇒G}\AgdaSymbol{))} \AgdaInductiveConstructor{,p} \AgdaSymbol{(}\AgdaInductiveConstructor{osr-red} \AgdaSymbol{(}\AgdaInductiveConstructor{appr} \AgdaBound{E⇒F}\AgdaSymbol{))}\<%
\\
\>\AgdaFunction{Local-Confluent} \AgdaSymbol{(}\AgdaInductiveConstructor{appr} \AgdaBound{E⇒F}\AgdaSymbol{)} \AgdaSymbol{(}\AgdaInductiveConstructor{appr} \AgdaBound{E⇒G}\AgdaSymbol{)} \AgdaSymbol{=} \AgdaKeyword{let} \AgdaSymbol{(}\AgdaBound{H} \AgdaInductiveConstructor{,p} \AgdaBound{F↠H} \AgdaInductiveConstructor{,p} \AgdaBound{G↠H}\AgdaSymbol{)} \AgdaSymbol{=} \AgdaFunction{Local-Confluent} \AgdaBound{E⇒F} \AgdaBound{E⇒G} \AgdaKeyword{in}\<%
\\
\>[0]\AgdaIndent{2}{}\<[2]%
\>[2]\AgdaSymbol{(\_} \AgdaInductiveConstructor{,,} \AgdaBound{H}\AgdaSymbol{)} \AgdaInductiveConstructor{,p} \AgdaFunction{respects-red} \AgdaInductiveConstructor{appr} \AgdaBound{F↠H} \AgdaInductiveConstructor{,p} \AgdaFunction{respects-red} \AgdaInductiveConstructor{appr} \AgdaBound{G↠H}\<%
\\
\>\AgdaComment{\{-Local-Confluent (redex βT) (redex βT) = \_ ,p ref ,p ref\<\\
\>Local-Confluent (redex βT) (app (appl (redex ())))\<\\
\>Local-Confluent (redex (βT \{M = M\} \{N\})) (app (appl (app (appl \{E' = M'\} M⇒M')))) = \<\\
\>  M' ⟦ x₀:= N ⟧ ,p (red-subl (osr-red M⇒M')) ,p (osr-red (redex βT))\<\\
\>Local-Confluent (redex βT) (app (appl (app (appr ()))))\<\\
\>Local-Confluent (redex (βT \{M = M\})) (app (appr (appl \{E' = N'\} N⇒N'))) =\<\\
\>  M ⟦ x₀:= N' ⟧ ,p red-subr M (botsub-red N⇒N') ,p \<\\
\>  (osr-red (redex βT))\<\\
\>Local-Confluent (redex βT) (app (appr (appr ())))\<\\
\>Local-Confluent (redex βR) (redex βR) = \_ ,p ref ,p ref\<\\
\>Local-Confluent (redex βR) (app (appl (redex ())))\<\\
\>Local-Confluent (redex βR) (app (appl (app (appl φ⇒φ')))) = \_ ,p ref ,p (osr-red (redex βR))\<\\
\>Local-Confluent (redex (βR \{δ = δ\} \{ε = ε\})) \<\\
\>  (app (appl (app (appr (appl \{E' = δ'\} δ⇒δ'))))) = \<\\
\>  δ' ⟦ x₀:= ε ⟧ ,p red-subl (osr-red δ⇒δ') ,p osr-red (redex βR)\<\\
\>Local-Confluent (redex βR) (app (appl (app (appr (appr ())))))\<\\
\>Local-Confluent (redex (βR \{δ = δ\})) (app (appr (appl \{E' = ε'\} ε⇒ε'))) = \<\\
\>  δ ⟦ x₀:= ε' ⟧ ,p red-subr δ (botsub-red ε⇒ε') ,p osr-red (redex βR)\<\\
\>Local-Confluent (redex βR) (app (appr (appr ())))\<\\
\>Local-Confluent (redex βE) (redex βE) = \_ ,p ref ,p ref\<\\
\>Local-Confluent (redex (βE \{N = N\} \{A = A\} \{P = P\} \{Q = Q\})) (app (appl \{E' = M'\} M⇒M')) = \<\\
\>  P ⟦ x₂:= M' ,x₁:= N ,x₀:= Q ⟧ ,p red-subr P (botsub₃-red (osr-red M⇒M') ref ref) ,p osr-red (redex βE)\<\\
\>Local-Confluent (redex (βE \{M = M\} \{P = P\} \{Q = Q\})) (app (appr (appl \{E' = N'\} N⇒N'))) = \<\\
\>  P ⟦ x₂:= M ,x₁:= N' ,x₀:= Q ⟧ ,p (red-subr P (botsub₃-red ref (osr-red N⇒N') ref)) ,p \<\\
\>  osr-red (redex βE)\<\\
\>Local-Confluent (redex βE) (app (appr (appr (appl (redex ())))))\<\\
\>Local-Confluent (redex (βE \{M = M\} \{N = N\} \{Q = Q\})) (app (appr (appr (appl (app (appl \{E' = P'\} P⇒P')))))) = P' ⟦ x₂:= M ,x₁:= N ,x₀:= Q ⟧ ,p (red-subl (osr-red P⇒P')) ,p osr-red (redex βE)\<\\
\>Local-Confluent (redex βE) (app (appr (appr (appl (app (appr ()))))))\<\\
\>Local-Confluent (redex (βE \{M = M\} \{N = N\} \{P = P\})) (app (appr (appr (appr (appl \{E' = Q'\} Q⇒Q'))))) = P ⟦ x₂:= M ,x₁:= N ,x₀:= Q' ⟧ ,p (red-subr P (botsub₃-red ref ref (osr-red Q⇒Q'))) ,p (osr-red (redex βE))\<\\
\>Local-Confluent (redex βE) (app (appr (appr (appr (appr ())))))\<\\
\>Local-Confluent (redex plus-ref) (redex plus-ref) = \_ ,p (ref ,p ref)\<\\
\>Local-Confluent (redex plus-ref) (app (appl (redex ())))\<\\
\>Local-Confluent (redex plus-ref) (app (appl (app (appl \{E' = φ'\} φ⇒φ')))) = ΛP φ' (var x₀)  ,p (osr-red (app (appl φ⇒φ'))) ,p (osr-red (redex plus-ref))\<\\
\>Local-Confluent (redex plus-ref) (app (appl (app (appr ()))))\<\\
\>Local-Confluent (redex plus-ref) (app (appr ()))\<\\
\>Local-Confluent (redex minus-ref) (redex minus-ref) = \_ ,p ref ,p ref\<\\
\>Local-Confluent (redex minus-ref) (app (appl (redex ())))\<\\
\>Local-Confluent (redex minus-ref) (app (appl (app (appl \{E' = φ'\} φ⇒φ')))) = ΛP φ' (var x₀) ,p (osr-red (app (appl φ⇒φ'))) ,p (osr-red (redex minus-ref))\<\\
\>Local-Confluent (redex minus-ref) (app (appl (app (appr ()))))\<\\
\>Local-Confluent (redex minus-ref) (app (appr ()))\<\\
\>Local-Confluent (redex plus-univ) (redex plus-univ) = \_ ,p ref ,p ref\<\\
\>Local-Confluent (redex plus-univ) (app (appl (redex ())))\<\\
\>Local-Confluent (redex plus-univ) (app (appl (app (appl \_)))) = \<\\
\>  \_ ,p ref ,p osr-red (redex plus-univ)\<\\
\>Local-Confluent (redex plus-univ) (app (appl (app (appr (appl \_))))) = \<\\
\>  \_ ,p ref ,p (osr-red (redex plus-univ))\<\\
\>Local-Confluent (redex plus-univ) (app (appl (app (appr (appr (appl δ⇒δ')))))) = \<\\
\>  \_ ,p osr-red δ⇒δ' ,p osr-red (redex plus-univ)\<\\
\>Local-Confluent (redex plus-univ) (app (appl (app (appr (appr (appr (appl E⇒G))))))) = \<\\
\>  \_ ,p ref ,p osr-red (redex plus-univ)\<\\
\>Local-Confluent (redex plus-univ) (app (appl (app (appr (appr (appr (appr ())))))))\<\\
\>Local-Confluent (redex plus-univ) (app (appr ()))\<\\
\>Local-Confluent (redex minus-univ) (redex minus-univ) = \_ ,p ref ,p ref\<\\
\>Local-Confluent (redex minus-univ) (app (appl (redex ())))\<\\
\>Local-Confluent (redex minus-univ) (app (appl (app (appl \_)))) = \<\\
\>  \_ ,p ref ,p osr-red (redex minus-univ)\<\\
\>Local-Confluent (redex minus-univ) (app (appl (app (appr (appl \_))))) = \<\\
\>  \_ ,p ref ,p (osr-red (redex minus-univ))\<\\
\>Local-Confluent (redex minus-univ) (app (appl (app (appr (appr (appl \_)))))) = \<\\
\>  \_ ,p ref ,p osr-red (redex minus-univ)\<\\
\>Local-Confluent (redex minus-univ) (app (appl (app (appr (appr (appr (appl ε⇒ε'))))))) = \<\\
\>  \_ ,p osr-red ε⇒ε' ,p osr-red (redex minus-univ)\<\\
\>Local-Confluent (redex minus-univ) (app (appl (app (appr (appr (appr (appr ())))))))\<\\
\>Local-Confluent (redex minus-univ) (app (appr ()))\<\\
\>Local-Confluent (redex ref⊃*univ) (redex ref⊃*univ) = \<\\
\>  \_ ,p ref ,p ref\<\\
\>Local-Confluent (redex ref⊃*univ) (app (appl (redex ())))\<\\
\>Local-Confluent (redex ref⊃*univ) (app (appl (app (appl φ⇒φ')))) = \<\\
\>  \_ ,p univ-red (osr-red (app (appl φ⇒φ'))) (osr-red (app (appl φ⇒φ'))) \<\\
\>    (ΛP-red (osr-red (app (appl φ⇒φ'))) (ΛP-red (apredr replacement R-respects-replacement (osr-red φ⇒φ')) ref)) (ΛP-red (osr-red (app (appl φ⇒φ'))) (ΛP-red (apredr replacement R-respects-replacement (osr-red φ⇒φ')) ref)) ,p osr-red (redex ref⊃*univ)\<\\
\>Local-Confluent (redex ref⊃*univ) (app (appl (app (appr ()))))\<\\
\>Local-Confluent (redex ref⊃*univ) (app (appr (appl E⇒G))) = \{!!\}\<\\
\>Local-Confluent (redex ref⊃*univ) (app (appr (appr E⇒G))) = \{!!\}\<\\
\>Local-Confluent (redex univ⊃*ref) E⇒G = \{!!\}\<\\
\>Local-Confluent (redex univ⊃*univ) E⇒G = \{!!\}\<\\
\>Local-Confluent (redex ref⊃*ref) E⇒G = \{!!\}\<\\
\>Local-Confluent (redex refref) E⇒G = \{!!\}\<\\
\>Local-Confluent (redex reflamvar) E⇒G = \{!!\}\<\\
\>Local-Confluent (redex reflam⊃*) E⇒G = \{!!\}\<\\
\>Local-Confluent (redex reflamuniv) E⇒G = \{!!\}\<\\
\>Local-Confluent (redex reflamλλλ) E⇒G = \{!!\}\<\\
\>Local-Confluent (app E⇒F) E⇒G = \{!!\} -\}}\<%
\\
\>\AgdaComment{--TODO General theory of reduction}\<%
\end{code}
}

\begin{code}%
\>\AgdaKeyword{postulate} \AgdaPostulate{Newmans} \AgdaSymbol{:} \AgdaSymbol{∀} \AgdaSymbol{\{}\AgdaBound{V}\AgdaSymbol{\}} \AgdaSymbol{\{}\AgdaBound{C}\AgdaSymbol{\}} \AgdaSymbol{\{}\AgdaBound{K}\AgdaSymbol{\}} \AgdaSymbol{\{}\AgdaBound{E} \AgdaBound{F} \AgdaBound{G} \AgdaSymbol{:} \AgdaDatatype{Subexpression} \AgdaBound{V} \AgdaBound{C} \AgdaBound{K}\AgdaSymbol{\}} \AgdaSymbol{→} \<[66]%
\>[66]\<%
\\
\>[2]\AgdaIndent{18}{}\<[18]%
\>[18]\AgdaDatatype{SN} \AgdaBound{E} \AgdaSymbol{→} \AgdaSymbol{(}\AgdaBound{E} \AgdaDatatype{↠} \AgdaBound{F}\AgdaSymbol{)} \AgdaSymbol{→} \AgdaSymbol{(}\AgdaBound{E} \AgdaDatatype{↠} \AgdaBound{G}\AgdaSymbol{)} \AgdaSymbol{→}\<%
\\
\>[2]\AgdaIndent{18}{}\<[18]%
\>[18]\AgdaFunction{Σ[} \AgdaBound{H} \AgdaFunction{∈} \AgdaDatatype{Subexpression} \AgdaBound{V} \AgdaBound{C} \AgdaBound{K} \AgdaFunction{]} \AgdaSymbol{(}\AgdaBound{F} \AgdaDatatype{↠} \AgdaBound{H} \AgdaFunction{×} \AgdaBound{G} \AgdaDatatype{↠} \AgdaBound{H}\AgdaSymbol{)}\<%
\\
%
\\
\>\AgdaKeyword{postulate} \AgdaPostulate{ChurchRosserT} \AgdaSymbol{:} \AgdaSymbol{∀} \AgdaSymbol{\{}\AgdaBound{V}\AgdaSymbol{\}} \AgdaSymbol{\{}\AgdaBound{M} \AgdaBound{N} \AgdaBound{P} \AgdaSymbol{:} \AgdaFunction{Term} \AgdaBound{V}\AgdaSymbol{\}} \AgdaSymbol{→} \AgdaBound{M} \AgdaDatatype{↠} \AgdaBound{N} \AgdaSymbol{→} \AgdaBound{M} \AgdaDatatype{↠} \AgdaBound{P} \AgdaSymbol{→}\<%
\\
\>[18]\AgdaIndent{24}{}\<[24]%
\>[24]\AgdaFunction{Σ[} \AgdaBound{Q} \AgdaFunction{∈} \AgdaFunction{Term} \AgdaBound{V} \AgdaFunction{]} \AgdaBound{N} \AgdaDatatype{↠} \AgdaBound{Q} \AgdaFunction{×} \AgdaBound{P} \AgdaDatatype{↠} \AgdaBound{Q}\<%
\\
%
\\
\>\AgdaKeyword{postulate} \AgdaPostulate{confluenceT} \AgdaSymbol{:} \AgdaSymbol{∀} \AgdaSymbol{\{}\AgdaBound{V}\AgdaSymbol{\}} \AgdaSymbol{\{}\AgdaBound{M} \AgdaBound{N} \AgdaSymbol{:} \AgdaFunction{Term} \AgdaBound{V}\AgdaSymbol{\}} \AgdaSymbol{→} \AgdaBound{M} \AgdaDatatype{≃} \AgdaBound{N} \AgdaSymbol{→}\<%
\\
\>[18]\AgdaIndent{24}{}\<[24]%
\>[24]\AgdaFunction{Σ[} \AgdaBound{Q} \AgdaFunction{∈} \AgdaFunction{Term} \AgdaBound{V} \AgdaFunction{]} \AgdaBound{M} \AgdaDatatype{↠} \AgdaBound{Q} \AgdaFunction{×} \AgdaBound{N} \AgdaDatatype{↠} \AgdaBound{Q}\<%
\\
%
\\
\>\AgdaKeyword{postulate} \AgdaPostulate{SNE} \AgdaSymbol{:} \AgdaSymbol{∀} \AgdaSymbol{\{}\AgdaBound{V}\AgdaSymbol{\}} \AgdaSymbol{\{}\AgdaBound{C}\AgdaSymbol{\}} \AgdaSymbol{\{}\AgdaBound{K}\AgdaSymbol{\}} \AgdaSymbol{(}\AgdaBound{P} \AgdaSymbol{:} \AgdaDatatype{Subexpression} \AgdaBound{V} \AgdaBound{C} \AgdaBound{K} \AgdaSymbol{→} \AgdaPrimitiveType{Set}\AgdaSymbol{)} \AgdaSymbol{→}\<%
\\
\>[0]\AgdaIndent{14}{}\<[14]%
\>[14]\AgdaSymbol{(∀} \AgdaSymbol{\{}\AgdaBound{M} \AgdaSymbol{:} \AgdaDatatype{Subexpression} \AgdaBound{V} \AgdaBound{C} \AgdaBound{K}\AgdaSymbol{\}} \AgdaSymbol{→} \AgdaDatatype{SN} \AgdaBound{M} \AgdaSymbol{→} \AgdaSymbol{(∀} \AgdaBound{N} \AgdaSymbol{→} \AgdaBound{M} \AgdaDatatype{↠⁺} \AgdaBound{N} \AgdaSymbol{→} \AgdaBound{P} \AgdaBound{N}\AgdaSymbol{)} \AgdaSymbol{→} \AgdaBound{P} \AgdaBound{M}\AgdaSymbol{)} \AgdaSymbol{→}\<%
\\
\>[0]\AgdaIndent{14}{}\<[14]%
\>[14]\AgdaSymbol{∀} \AgdaSymbol{\{}\AgdaBound{M} \AgdaSymbol{:} \AgdaDatatype{Subexpression} \AgdaBound{V} \AgdaBound{C} \AgdaBound{K}\AgdaSymbol{\}} \AgdaSymbol{→} \AgdaDatatype{SN} \AgdaBound{M} \AgdaSymbol{→} \AgdaBound{P} \AgdaBound{M}\<%
\\
%
\\
\>\AgdaKeyword{private} \AgdaFunction{var-red'} \AgdaSymbol{:} \AgdaSymbol{∀} \AgdaSymbol{\{}\AgdaBound{V}\AgdaSymbol{\}} \AgdaSymbol{\{}\AgdaBound{K}\AgdaSymbol{\}} \AgdaSymbol{\{}\AgdaBound{x} \AgdaSymbol{:} \AgdaDatatype{Var} \AgdaBound{V} \AgdaBound{K}\AgdaSymbol{\}} \AgdaSymbol{\{}\AgdaBound{M}\AgdaSymbol{\}} \AgdaSymbol{\{}\AgdaBound{N}\AgdaSymbol{\}} \AgdaSymbol{→} \AgdaBound{M} \AgdaDatatype{↠} \AgdaBound{N} \AgdaSymbol{→} \AgdaBound{M} \AgdaDatatype{≡} \AgdaInductiveConstructor{var} \AgdaBound{x} \AgdaSymbol{→} \AgdaBound{N} \AgdaDatatype{≡} \AgdaInductiveConstructor{var} \AgdaBound{x}\<%
\\
\>\AgdaFunction{var-red'} \AgdaSymbol{(}\AgdaInductiveConstructor{osr-red} \AgdaSymbol{(}\AgdaInductiveConstructor{redex} \AgdaSymbol{\_))} \AgdaSymbol{()}\<%
\\
\>\AgdaFunction{var-red'} \AgdaSymbol{(}\AgdaInductiveConstructor{osr-red} \AgdaSymbol{(}\AgdaInductiveConstructor{app} \AgdaSymbol{\_))} \AgdaSymbol{()}\<%
\\
\>\AgdaFunction{var-red'} \AgdaInductiveConstructor{ref} \AgdaBound{M≡x} \AgdaSymbol{=} \AgdaBound{M≡x}\<%
\\
\>\AgdaFunction{var-red'} \AgdaSymbol{(}\AgdaInductiveConstructor{trans-red} \AgdaBound{M↠N} \AgdaBound{N↠P}\AgdaSymbol{)} \AgdaBound{M≡x} \AgdaSymbol{=} \AgdaFunction{var-red'} \AgdaBound{N↠P} \AgdaSymbol{(}\AgdaFunction{var-red'} \AgdaBound{M↠N} \AgdaBound{M≡x}\AgdaSymbol{)}\<%
\\
%
\\
\>\AgdaFunction{var-red} \AgdaSymbol{:} \AgdaSymbol{∀} \AgdaSymbol{\{}\AgdaBound{V}\AgdaSymbol{\}} \AgdaSymbol{\{}\AgdaBound{K}\AgdaSymbol{\}} \AgdaSymbol{\{}\AgdaBound{x} \AgdaSymbol{:} \AgdaDatatype{Var} \AgdaBound{V} \AgdaBound{K}\AgdaSymbol{\}} \AgdaSymbol{\{}\AgdaBound{M}\AgdaSymbol{\}} \AgdaSymbol{→} \AgdaInductiveConstructor{var} \AgdaBound{x} \AgdaDatatype{↠} \AgdaBound{M} \AgdaSymbol{→} \AgdaBound{M} \AgdaDatatype{≡} \AgdaInductiveConstructor{var} \AgdaBound{x}\<%
\\
\>\AgdaFunction{var-red} \AgdaBound{x↠M} \AgdaSymbol{=} \AgdaFunction{var-red'} \AgdaBound{x↠M} \AgdaInductiveConstructor{refl}\<%
\\
%
\\
\>\AgdaKeyword{private} \AgdaFunction{bot-red'} \AgdaSymbol{:} \AgdaSymbol{∀} \AgdaSymbol{\{}\AgdaBound{V}\AgdaSymbol{\}} \AgdaSymbol{\{}\AgdaBound{φ} \AgdaBound{ψ} \AgdaSymbol{:} \AgdaFunction{Term} \AgdaBound{V}\AgdaSymbol{\}} \AgdaSymbol{→} \AgdaBound{φ} \AgdaDatatype{↠} \AgdaBound{ψ} \AgdaSymbol{→} \AgdaBound{φ} \AgdaDatatype{≡} \AgdaFunction{⊥} \AgdaSymbol{→} \AgdaBound{ψ} \AgdaDatatype{≡} \AgdaFunction{⊥}\<%
\\
\>\AgdaFunction{bot-red'} \AgdaSymbol{(}\AgdaInductiveConstructor{osr-red} \AgdaSymbol{(}\AgdaInductiveConstructor{redex} \AgdaInductiveConstructor{βT}\AgdaSymbol{))} \AgdaSymbol{()}\<%
\\
\>\AgdaFunction{bot-red'} \AgdaSymbol{(}\AgdaInductiveConstructor{osr-red} \AgdaSymbol{(}\AgdaInductiveConstructor{app} \AgdaSymbol{\{}\AgdaArgument{c} \AgdaSymbol{=} \AgdaInductiveConstructor{-bot}\AgdaSymbol{\}} \AgdaSymbol{\{}\AgdaArgument{F} \AgdaSymbol{=} \AgdaInductiveConstructor{out}\AgdaSymbol{\}} \AgdaBound{x}\AgdaSymbol{))} \AgdaSymbol{\_} \AgdaSymbol{=} \AgdaInductiveConstructor{refl}\<%
\\
\>\AgdaFunction{bot-red'} \AgdaSymbol{(}\AgdaInductiveConstructor{osr-red} \AgdaSymbol{(}\AgdaInductiveConstructor{app} \AgdaSymbol{\{}\AgdaArgument{c} \AgdaSymbol{=} \AgdaInductiveConstructor{-imp}\AgdaSymbol{\}} \AgdaSymbol{\_))} \AgdaSymbol{()}\<%
\\
\>\AgdaFunction{bot-red'} \AgdaSymbol{(}\AgdaInductiveConstructor{osr-red} \AgdaSymbol{(}\AgdaInductiveConstructor{app} \AgdaSymbol{\{}\AgdaArgument{c} \AgdaSymbol{=} \AgdaInductiveConstructor{-appTerm}\AgdaSymbol{\}} \AgdaSymbol{\_))} \AgdaSymbol{()}\<%
\\
\>\AgdaFunction{bot-red'} \AgdaSymbol{(}\AgdaInductiveConstructor{osr-red} \AgdaSymbol{(}\AgdaInductiveConstructor{app} \AgdaSymbol{\{}\AgdaArgument{c} \AgdaSymbol{=} \AgdaInductiveConstructor{-lamTerm} \AgdaSymbol{\_\}} \AgdaSymbol{\_))} \AgdaSymbol{()}\<%
\\
\>\AgdaFunction{bot-red'} \AgdaInductiveConstructor{ref} \AgdaBound{φ≡⊥} \AgdaSymbol{=} \AgdaBound{φ≡⊥}\<%
\\
\>\AgdaFunction{bot-red'} \AgdaSymbol{(}\AgdaInductiveConstructor{trans-red} \AgdaBound{φ↠ψ} \AgdaBound{ψ↠χ}\AgdaSymbol{)} \AgdaBound{φ≡⊥} \AgdaSymbol{=} \AgdaFunction{bot-red'} \AgdaBound{ψ↠χ} \AgdaSymbol{(}\AgdaFunction{bot-red'} \AgdaBound{φ↠ψ} \AgdaBound{φ≡⊥}\AgdaSymbol{)}\<%
\\
%
\\
\>\AgdaFunction{bot-red} \AgdaSymbol{:} \AgdaSymbol{∀} \AgdaSymbol{\{}\AgdaBound{V}\AgdaSymbol{\}} \AgdaSymbol{\{}\AgdaBound{φ} \AgdaSymbol{:} \AgdaFunction{Term} \AgdaBound{V}\AgdaSymbol{\}} \AgdaSymbol{→} \AgdaFunction{⊥} \AgdaDatatype{↠} \AgdaBound{φ} \AgdaSymbol{→} \AgdaBound{φ} \AgdaDatatype{≡} \AgdaFunction{⊥}\<%
\\
\>\AgdaFunction{bot-red} \AgdaBound{⊥↠φ} \AgdaSymbol{=} \AgdaFunction{bot-red'} \AgdaBound{⊥↠φ} \AgdaInductiveConstructor{refl}\<%
\\
%
\\
\>\AgdaFunction{imp-red'} \AgdaSymbol{:} \AgdaSymbol{∀} \AgdaSymbol{\{}\AgdaBound{V}\AgdaSymbol{\}} \AgdaSymbol{\{}\AgdaBound{φ} \AgdaBound{ψ} \AgdaBound{χ} \AgdaBound{θ} \AgdaSymbol{:} \AgdaFunction{Term} \AgdaBound{V}\AgdaSymbol{\}} \AgdaSymbol{→} \AgdaBound{φ} \AgdaDatatype{↠} \AgdaBound{ψ} \AgdaSymbol{→} \AgdaBound{φ} \AgdaDatatype{≡} \AgdaBound{χ} \AgdaFunction{⊃} \AgdaBound{θ} \AgdaSymbol{→}\<%
\\
\>[0]\AgdaIndent{2}{}\<[2]%
\>[2]\AgdaFunction{Σ[} \AgdaBound{χ'} \AgdaFunction{∈} \AgdaFunction{Term} \AgdaBound{V} \AgdaFunction{]} \AgdaFunction{Σ[} \AgdaBound{θ'} \AgdaFunction{∈} \AgdaFunction{Term} \AgdaBound{V} \AgdaFunction{]} \AgdaBound{χ} \AgdaDatatype{↠} \AgdaBound{χ'} \AgdaFunction{×} \AgdaBound{θ} \AgdaDatatype{↠} \AgdaBound{θ'} \AgdaFunction{×} \AgdaBound{ψ} \AgdaDatatype{≡} \AgdaBound{χ'} \AgdaFunction{⊃} \AgdaBound{θ'}\<%
\\
\>\AgdaFunction{imp-red'} \AgdaSymbol{(}\AgdaInductiveConstructor{osr-red} \AgdaSymbol{(}\AgdaInductiveConstructor{redex} \AgdaInductiveConstructor{βT}\AgdaSymbol{))} \AgdaSymbol{()}\<%
\\
\>\AgdaFunction{imp-red'} \AgdaSymbol{(}\AgdaInductiveConstructor{osr-red} \AgdaSymbol{(}\AgdaInductiveConstructor{app} \AgdaSymbol{\{}\AgdaArgument{c} \AgdaSymbol{=} \AgdaInductiveConstructor{-bot}\AgdaSymbol{\}} \AgdaSymbol{\_))} \AgdaSymbol{()}\<%
\\
\>\AgdaFunction{imp-red'} \AgdaSymbol{\{}\AgdaArgument{θ} \AgdaSymbol{=} \AgdaBound{θ}\AgdaSymbol{\}} \AgdaSymbol{(}\AgdaInductiveConstructor{osr-red} \AgdaSymbol{(}\AgdaInductiveConstructor{app} \AgdaSymbol{\{}\AgdaArgument{c} \AgdaSymbol{=} \AgdaInductiveConstructor{-imp}\AgdaSymbol{\}} \AgdaSymbol{(}\AgdaInductiveConstructor{appl} \AgdaSymbol{\{}\AgdaArgument{E'} \AgdaSymbol{=} \AgdaBound{χ'}\AgdaSymbol{\}} \AgdaSymbol{\{}\AgdaArgument{F} \AgdaSymbol{=} \AgdaSymbol{\_} \AgdaInductiveConstructor{,,} \AgdaInductiveConstructor{out}\AgdaSymbol{\}} \AgdaBound{χ⇒χ'}\AgdaSymbol{)))} \AgdaBound{φ≡χ⊃θ} \AgdaSymbol{=} \<[89]%
\>[89]\<%
\\
\>[0]\AgdaIndent{2}{}\<[2]%
\>[2]\AgdaBound{χ'} \AgdaInductiveConstructor{,p} \AgdaBound{θ} \AgdaInductiveConstructor{,p} \AgdaFunction{subst} \AgdaSymbol{(λ} \AgdaBound{x} \AgdaSymbol{→} \AgdaBound{x} \AgdaDatatype{↠} \AgdaBound{χ'}\AgdaSymbol{)} \AgdaSymbol{(}\AgdaFunction{imp-injl} \AgdaBound{φ≡χ⊃θ}\AgdaSymbol{)} \AgdaSymbol{(}\AgdaInductiveConstructor{osr-red} \AgdaBound{χ⇒χ'}\AgdaSymbol{)} \AgdaInductiveConstructor{,p} \<[69]%
\>[69]\<%
\\
\>[0]\AgdaIndent{2}{}\<[2]%
\>[2]\AgdaInductiveConstructor{ref} \AgdaInductiveConstructor{,p} \AgdaSymbol{(}\AgdaFunction{cong} \AgdaSymbol{(λ} \AgdaBound{x} \AgdaSymbol{→} \AgdaBound{χ'} \AgdaFunction{⊃} \AgdaBound{x}\AgdaSymbol{)} \AgdaSymbol{(}\AgdaFunction{imp-injr} \AgdaBound{φ≡χ⊃θ}\AgdaSymbol{))}\<%
\\
\>\AgdaFunction{imp-red'} \AgdaSymbol{\{}\AgdaArgument{χ} \AgdaSymbol{=} \AgdaBound{χ}\AgdaSymbol{\}} \AgdaSymbol{(}\AgdaInductiveConstructor{osr-red} \AgdaSymbol{(}\AgdaInductiveConstructor{app} \AgdaSymbol{\{}\AgdaArgument{c} \AgdaSymbol{=} \AgdaInductiveConstructor{-imp}\AgdaSymbol{\}} \AgdaSymbol{(}\AgdaInductiveConstructor{appr} \AgdaSymbol{(}\AgdaInductiveConstructor{appl} \AgdaSymbol{\{}\AgdaArgument{E'} \AgdaSymbol{=} \AgdaBound{θ'}\AgdaSymbol{\}} \AgdaSymbol{\{}\AgdaArgument{F} \AgdaSymbol{=} \AgdaInductiveConstructor{out}\AgdaSymbol{\}} \AgdaBound{θ⇒θ'}\AgdaSymbol{))))} \AgdaBound{φ≡χ⊃θ} \AgdaSymbol{=} \<[91]%
\>[91]\<%
\\
\>[0]\AgdaIndent{2}{}\<[2]%
\>[2]\AgdaBound{χ} \AgdaInductiveConstructor{,p} \AgdaBound{θ'} \AgdaInductiveConstructor{,p} \AgdaInductiveConstructor{ref} \AgdaInductiveConstructor{,p} \AgdaSymbol{(}\AgdaFunction{subst} \AgdaSymbol{(λ} \AgdaBound{x} \AgdaSymbol{→} \AgdaBound{x} \AgdaDatatype{↠} \AgdaBound{θ'}\AgdaSymbol{)} \AgdaSymbol{(}\AgdaFunction{imp-injr} \AgdaBound{φ≡χ⊃θ}\AgdaSymbol{)} \AgdaSymbol{(}\AgdaInductiveConstructor{osr-red} \AgdaBound{θ⇒θ'}\AgdaSymbol{))} \AgdaInductiveConstructor{,p} \<[78]%
\>[78]\<%
\\
\>[0]\AgdaIndent{2}{}\<[2]%
\>[2]\AgdaFunction{cong} \AgdaSymbol{(λ} \AgdaBound{x} \AgdaSymbol{→} \AgdaBound{x} \AgdaFunction{⊃} \AgdaBound{θ'}\AgdaSymbol{)} \AgdaSymbol{(}\AgdaFunction{imp-injl} \AgdaBound{φ≡χ⊃θ}\AgdaSymbol{)}\<%
\\
\>\AgdaFunction{imp-red'} \AgdaSymbol{(}\AgdaInductiveConstructor{osr-red} \AgdaSymbol{(}\AgdaInductiveConstructor{app} \AgdaSymbol{\{}\AgdaArgument{c} \AgdaSymbol{=} \AgdaInductiveConstructor{-imp}\AgdaSymbol{\}} \AgdaSymbol{(}\AgdaInductiveConstructor{appr} \AgdaSymbol{(}\AgdaInductiveConstructor{appr} \AgdaSymbol{()))))} \AgdaSymbol{\_}\<%
\\
\>\AgdaFunction{imp-red'} \AgdaSymbol{(}\AgdaInductiveConstructor{osr-red} \AgdaSymbol{(}\AgdaInductiveConstructor{app} \AgdaSymbol{\{}\AgdaArgument{c} \AgdaSymbol{=} \AgdaInductiveConstructor{-appTerm}\AgdaSymbol{\}} \AgdaSymbol{\_))} \AgdaSymbol{()}\<%
\\
\>\AgdaFunction{imp-red'} \AgdaSymbol{(}\AgdaInductiveConstructor{osr-red} \AgdaSymbol{(}\AgdaInductiveConstructor{app} \AgdaSymbol{\{}\AgdaArgument{c} \AgdaSymbol{=} \AgdaInductiveConstructor{-lamTerm} \AgdaSymbol{\_\}} \AgdaSymbol{\_))} \AgdaSymbol{()}\<%
\\
\>\AgdaFunction{imp-red'} \AgdaSymbol{\{}\AgdaArgument{χ} \AgdaSymbol{=} \AgdaBound{χ}\AgdaSymbol{\}} \AgdaSymbol{\{}\AgdaBound{θ}\AgdaSymbol{\}} \AgdaInductiveConstructor{ref} \AgdaBound{φ≡χ⊃θ} \AgdaSymbol{=} \AgdaBound{χ} \AgdaInductiveConstructor{,p} \AgdaBound{θ} \AgdaInductiveConstructor{,p} \AgdaInductiveConstructor{ref} \AgdaInductiveConstructor{,p} \AgdaInductiveConstructor{ref} \AgdaInductiveConstructor{,p} \AgdaBound{φ≡χ⊃θ}\<%
\\
\>\AgdaFunction{imp-red'} \AgdaSymbol{(}\AgdaInductiveConstructor{trans-red} \AgdaBound{φ↠ψ} \AgdaBound{ψ↠ψ'}\AgdaSymbol{)} \AgdaBound{φ≡χ⊃θ} \AgdaSymbol{=} \<[38]%
\>[38]\<%
\\
\>[0]\AgdaIndent{2}{}\<[2]%
\>[2]\AgdaKeyword{let} \AgdaSymbol{(}\AgdaBound{χ'} \AgdaInductiveConstructor{,p} \AgdaBound{θ'} \AgdaInductiveConstructor{,p} \AgdaBound{χ↠χ'} \AgdaInductiveConstructor{,p} \AgdaBound{θ↠θ'} \AgdaInductiveConstructor{,p} \AgdaBound{ψ≡χ'⊃θ'}\AgdaSymbol{)} \AgdaSymbol{=} \AgdaFunction{imp-red'} \AgdaBound{φ↠ψ} \AgdaBound{φ≡χ⊃θ} \AgdaKeyword{in} \<[68]%
\>[68]\<%
\\
\>[0]\AgdaIndent{2}{}\<[2]%
\>[2]\AgdaKeyword{let} \AgdaSymbol{(}\AgdaBound{χ''} \AgdaInductiveConstructor{,p} \AgdaBound{θ''} \AgdaInductiveConstructor{,p} \AgdaBound{χ'↠χ''} \AgdaInductiveConstructor{,p} \AgdaBound{θ'↠θ''} \AgdaInductiveConstructor{,p} \AgdaBound{ψ'≡χ''⊃θ''}\AgdaSymbol{)} \AgdaSymbol{=} \AgdaFunction{imp-red'} \AgdaBound{ψ↠ψ'} \AgdaBound{ψ≡χ'⊃θ'} \AgdaKeyword{in} \<[80]%
\>[80]\<%
\\
\>[0]\AgdaIndent{2}{}\<[2]%
\>[2]\AgdaBound{χ''} \AgdaInductiveConstructor{,p} \AgdaBound{θ''} \AgdaInductiveConstructor{,p} \AgdaInductiveConstructor{trans-red} \AgdaBound{χ↠χ'} \AgdaBound{χ'↠χ''} \AgdaInductiveConstructor{,p} \AgdaInductiveConstructor{trans-red} \AgdaBound{θ↠θ'} \AgdaBound{θ'↠θ''} \AgdaInductiveConstructor{,p} \AgdaBound{ψ'≡χ''⊃θ''}\<%
\\
%
\\
\>\AgdaFunction{imp-red} \AgdaSymbol{:} \AgdaSymbol{∀} \AgdaSymbol{\{}\AgdaBound{V}\AgdaSymbol{\}} \AgdaSymbol{\{}\AgdaBound{χ} \AgdaBound{θ} \AgdaBound{ψ} \AgdaSymbol{:} \AgdaFunction{Term} \AgdaBound{V}\AgdaSymbol{\}} \AgdaSymbol{→} \AgdaBound{χ} \AgdaFunction{⊃} \AgdaBound{θ} \AgdaDatatype{↠} \AgdaBound{ψ} \AgdaSymbol{→}\<%
\\
\>[0]\AgdaIndent{2}{}\<[2]%
\>[2]\AgdaFunction{Σ[} \AgdaBound{χ'} \AgdaFunction{∈} \AgdaFunction{Term} \AgdaBound{V} \AgdaFunction{]} \AgdaFunction{Σ[} \AgdaBound{θ'} \AgdaFunction{∈} \AgdaFunction{Term} \AgdaBound{V} \AgdaFunction{]} \AgdaBound{χ} \AgdaDatatype{↠} \AgdaBound{χ'} \AgdaFunction{×} \AgdaBound{θ} \AgdaDatatype{↠} \AgdaBound{θ'} \AgdaFunction{×} \AgdaBound{ψ} \AgdaDatatype{≡} \AgdaBound{χ'} \AgdaFunction{⊃} \AgdaBound{θ'}\<%
\\
\>\AgdaFunction{imp-red} \AgdaBound{χ⊃θ↠ψ} \AgdaSymbol{=} \AgdaFunction{imp-red'} \AgdaBound{χ⊃θ↠ψ} \AgdaInductiveConstructor{refl}\<%
\\
%
\\
\>\AgdaKeyword{postulate} \AgdaPostulate{red-rep} \AgdaSymbol{:} \AgdaSymbol{∀} \AgdaSymbol{\{}\AgdaBound{U}\AgdaSymbol{\}} \AgdaSymbol{\{}\AgdaBound{V}\AgdaSymbol{\}} \AgdaSymbol{\{}\AgdaBound{C}\AgdaSymbol{\}} \AgdaSymbol{\{}\AgdaBound{K}\AgdaSymbol{\}} \AgdaSymbol{\{}\AgdaBound{ρ} \AgdaSymbol{:} \AgdaFunction{Rep} \AgdaBound{U} \AgdaBound{V}\AgdaSymbol{\}} \AgdaSymbol{\{}\AgdaBound{M} \AgdaBound{N} \AgdaSymbol{:} \AgdaDatatype{Subexpression} \AgdaBound{U} \AgdaBound{C} \AgdaBound{K}\AgdaSymbol{\}} \AgdaSymbol{→} \AgdaBound{M} \AgdaDatatype{↠} \AgdaBound{N} \AgdaSymbol{→} \AgdaBound{M} \AgdaFunction{〈} \AgdaBound{ρ} \AgdaFunction{〉} \AgdaDatatype{↠} \AgdaBound{N} \AgdaFunction{〈} \AgdaBound{ρ} \AgdaFunction{〉}\<%
\\
%
\\
\>\AgdaKeyword{postulate} \AgdaPostulate{conv-rep} \AgdaSymbol{:} \AgdaSymbol{∀} \AgdaSymbol{\{}\AgdaBound{U}\AgdaSymbol{\}} \AgdaSymbol{\{}\AgdaBound{V}\AgdaSymbol{\}} \AgdaSymbol{\{}\AgdaBound{C}\AgdaSymbol{\}} \AgdaSymbol{\{}\AgdaBound{K}\AgdaSymbol{\}} \AgdaSymbol{\{}\AgdaBound{ρ} \AgdaSymbol{:} \AgdaFunction{Rep} \AgdaBound{U} \AgdaBound{V}\AgdaSymbol{\}} \AgdaSymbol{\{}\AgdaBound{M} \AgdaBound{N} \AgdaSymbol{:} \AgdaDatatype{Subexpression} \AgdaBound{U} \AgdaBound{C} \AgdaBound{K}\AgdaSymbol{\}} \AgdaSymbol{→} \AgdaBound{M} \AgdaDatatype{≃} \AgdaBound{N} \AgdaSymbol{→} \AgdaBound{M} \AgdaFunction{〈} \AgdaBound{ρ} \AgdaFunction{〉} \AgdaDatatype{≃} \AgdaBound{N} \AgdaFunction{〈} \AgdaBound{ρ} \AgdaFunction{〉}\<%
\\
%
\\
\>\AgdaKeyword{postulate} \AgdaPostulate{conv-sub} \AgdaSymbol{:} \AgdaSymbol{∀} \AgdaSymbol{\{}\AgdaBound{U}\AgdaSymbol{\}} \AgdaSymbol{\{}\AgdaBound{V}\AgdaSymbol{\}} \AgdaSymbol{\{}\AgdaBound{C}\AgdaSymbol{\}} \AgdaSymbol{\{}\AgdaBound{K}\AgdaSymbol{\}} \AgdaSymbol{\{}\AgdaBound{σ} \AgdaSymbol{:} \AgdaFunction{Sub} \AgdaBound{U} \AgdaBound{V}\AgdaSymbol{\}} \AgdaSymbol{\{}\AgdaBound{M} \AgdaBound{N} \AgdaSymbol{:} \AgdaDatatype{Subexpression} \AgdaBound{U} \AgdaBound{C} \AgdaBound{K}\AgdaSymbol{\}} \AgdaSymbol{→} \AgdaBound{M} \AgdaDatatype{≃} \AgdaBound{N} \AgdaSymbol{→} \AgdaBound{M} \AgdaFunction{⟦} \AgdaBound{σ} \AgdaFunction{⟧} \AgdaDatatype{≃} \AgdaBound{N} \AgdaFunction{⟦} \AgdaBound{σ} \AgdaFunction{⟧}\<%
\\
%
\\
\>\AgdaKeyword{postulate} \AgdaPostulate{appT-convl} \AgdaSymbol{:} \AgdaSymbol{∀} \AgdaSymbol{\{}\AgdaBound{V}\AgdaSymbol{\}} \AgdaSymbol{\{}\AgdaBound{M} \AgdaBound{M'} \AgdaBound{N} \AgdaSymbol{:} \AgdaFunction{Term} \AgdaBound{V}\AgdaSymbol{\}} \AgdaSymbol{→} \AgdaBound{M} \AgdaDatatype{≃} \AgdaBound{M'} \AgdaSymbol{→} \AgdaFunction{appT} \AgdaBound{M} \AgdaBound{N} \AgdaDatatype{≃} \AgdaFunction{appT} \AgdaBound{M'} \AgdaBound{N}\<%
\end{code}
}

\subsection{Strong Normalization Proof}

\mode<all>{\AgdaHide{
\begin{code}%
\>\AgdaKeyword{module} \AgdaModule{PHOPL.Computable} \AgdaKeyword{where}\<%
\\
\>\AgdaKeyword{open} \AgdaKeyword{import} \AgdaModule{Data.Empty} \AgdaKeyword{renaming} \AgdaSymbol{(}\AgdaDatatype{⊥} \AgdaSymbol{to} \AgdaDatatype{Empty}\AgdaSymbol{)}\<%
\\
\>\AgdaKeyword{open} \AgdaKeyword{import} \AgdaModule{Data.Product} \AgdaKeyword{renaming} \AgdaSymbol{(}\AgdaInductiveConstructor{\_,\_} \AgdaSymbol{to} \AgdaInductiveConstructor{\_,p\_}\AgdaSymbol{)}\<%
\\
\>\AgdaKeyword{open} \AgdaKeyword{import} \AgdaModule{Prelims}\<%
\\
\>\AgdaKeyword{open} \AgdaKeyword{import} \AgdaModule{PHOPL.Grammar}\<%
\\
\>\AgdaKeyword{open} \AgdaKeyword{import} \AgdaModule{PHOPL.PathSub}\<%
\\
\>\AgdaKeyword{open} \AgdaKeyword{import} \AgdaModule{PHOPL.Red}\<%
\\
\>\AgdaKeyword{open} \AgdaKeyword{import} \AgdaModule{PHOPL.SN}\<%
\\
\>\AgdaKeyword{open} \AgdaKeyword{import} \AgdaModule{PHOPL.Rules}\<%
\\
\>\AgdaKeyword{open} \AgdaKeyword{import} \AgdaModule{PHOPL.Meta}\<%
\\
\>\AgdaKeyword{open} \AgdaKeyword{import} \AgdaModule{PHOPL.KeyRedex}\<%
\\
\>\AgdaKeyword{open} \AgdaKeyword{import} \AgdaModule{PHOPL.Neutral}\<%
\end{code}
}

We define a model of the type theory with types as sets of terms.  For every type (proposition, equation) $A$ in context $\Gamma$, define
the set of \emph{computable} terms (proofs, paths) $E_\Gamma(A)$.

\begin{definition}[Computable Expressions]
\begin{align*}
E_\Gamma(\Omega) \eqdef & \{ M \mid \Gamma \vdash M : \Omega, M \in \SN \} \\
E_\Gamma(A \rightarrow B) \eqdef & \{ M \mid \Gamma \vdash M : A \rightarrow B, \\
& \quad \forall (\Delta \supseteq \Gamma) (N \in E_\Delta(A)). MN \in E_\Delta(B), \\
& \quad \forall (\Delta \supseteq \Gamma) (N, N' \in E_\Delta(A)) (P \in E_\Delta(N =_A N')). \\
& \quad \quad \reff{M}_{N N'} P \in E_\Gamma(MN =_B MN') \} \\
\\
E_\Gamma(\bot) & \eqdef \{ \delta \mid \Gamma \vdash \delta : \bot, \delta \in \SN \} \\
E_\Gamma(\phi \rightarrow \psi) & \eqdef \{ \delta \mid \Gamma \vdash \delta : \phi \rightarrow \psi, \\
& \forall (\Delta \supseteq \Gamma)(\epsilon \in E_\Delta(\phi)). \delta \epsilon \in E_\Gamma(\psi) \} \\
E_\Gamma(\phi) & \eqdef \{ \delta \mid \Gamma \vdash \delta : \phi, \delta \in \SN \} \\
& \qquad (\phi \text{ neutral}) \\
E_\Gamma(\phi) & \eqdef E_\Gamma(nf(\phi)) \\
& \qquad (\phi \mbox{ a normalizable term of type $\Omega$}) \\
\\
E_\Gamma(\phi =_\Omega \psi) & \eqdef \{ P \mid \Gamma \vdash P : \phi =_\Omega \psi, \\
& P^+ \in E_\Gamma(\phi \rightarrow \psi), P^- \in E_\Gamma(\psi \rightarrow \phi) \} \\
\\
E_\Gamma(M =_{A \rightarrow B} M') & \eqdef \{ P \mid \Gamma \vdash P : M =_{A \rightarrow B} M', \\
& \forall (\Delta \supseteq \Gamma) (N, N' \in E_\Delta(A)) (Q \in E_\Delta(N =_A N')). \\
& P_{NN'}Q \in E_\Delta(MN =_B M'N') \}
\end{align*}
\end{definition}

\AgdaHide{
\begin{code}%
\>\AgdaKeyword{data} \AgdaDatatype{Shape} \AgdaSymbol{:} \AgdaPrimitiveType{Set} \AgdaKeyword{where}\<%
\\
\>[0]\AgdaIndent{2}{}\<[2]%
\>[2]\AgdaInductiveConstructor{neutral} \AgdaSymbol{:} \AgdaDatatype{Shape}\<%
\\
\>[0]\AgdaIndent{2}{}\<[2]%
\>[2]\AgdaInductiveConstructor{bot} \AgdaSymbol{:} \AgdaDatatype{Shape}\<%
\\
\>[0]\AgdaIndent{2}{}\<[2]%
\>[2]\AgdaInductiveConstructor{imp} \AgdaSymbol{:} \AgdaDatatype{Shape} \AgdaSymbol{→} \AgdaDatatype{Shape} \AgdaSymbol{→} \AgdaDatatype{Shape}\<%
\\
%
\\
\>\AgdaKeyword{data} \AgdaDatatype{Leaves} \AgdaSymbol{(}\AgdaBound{V} \AgdaSymbol{:} \AgdaDatatype{Alphabet}\AgdaSymbol{)} \AgdaSymbol{:} \AgdaDatatype{Shape} \AgdaSymbol{→} \AgdaPrimitiveType{Set} \AgdaKeyword{where}\<%
\\
\>[0]\AgdaIndent{2}{}\<[2]%
\>[2]\AgdaInductiveConstructor{neutral} \AgdaSymbol{:} \AgdaDatatype{Neutral} \AgdaBound{V} \AgdaSymbol{→} \AgdaDatatype{Leaves} \AgdaBound{V} \AgdaInductiveConstructor{neutral}\<%
\\
\>[0]\AgdaIndent{2}{}\<[2]%
\>[2]\AgdaInductiveConstructor{bot} \AgdaSymbol{:} \AgdaDatatype{Leaves} \AgdaBound{V} \AgdaInductiveConstructor{bot}\<%
\\
\>[0]\AgdaIndent{2}{}\<[2]%
\>[2]\AgdaInductiveConstructor{imp} \AgdaSymbol{:} \AgdaSymbol{∀} \AgdaSymbol{\{}\AgdaBound{S}\AgdaSymbol{\}} \AgdaSymbol{\{}\AgdaBound{T}\AgdaSymbol{\}} \AgdaSymbol{→} \AgdaDatatype{Leaves} \AgdaBound{V} \AgdaBound{S} \AgdaSymbol{→} \AgdaDatatype{Leaves} \AgdaBound{V} \AgdaBound{T} \AgdaSymbol{→} \AgdaDatatype{Leaves} \AgdaBound{V} \AgdaSymbol{(}\AgdaInductiveConstructor{imp} \AgdaBound{S} \AgdaBound{T}\AgdaSymbol{)}\<%
\\
%
\\
\>\AgdaFunction{lrep} \AgdaSymbol{:} \AgdaSymbol{∀} \AgdaSymbol{\{}\AgdaBound{U}\AgdaSymbol{\}} \AgdaSymbol{\{}\AgdaBound{V}\AgdaSymbol{\}} \AgdaSymbol{\{}\AgdaBound{S}\AgdaSymbol{\}} \AgdaSymbol{→} \AgdaFunction{Rep} \AgdaBound{U} \AgdaBound{V} \AgdaSymbol{→} \AgdaDatatype{Leaves} \AgdaBound{U} \AgdaBound{S} \AgdaSymbol{→} \AgdaDatatype{Leaves} \AgdaBound{V} \AgdaBound{S}\<%
\\
\>\AgdaFunction{lrep} \AgdaBound{ρ} \AgdaSymbol{(}\AgdaInductiveConstructor{neutral} \AgdaBound{N}\AgdaSymbol{)} \AgdaSymbol{=} \AgdaInductiveConstructor{neutral} \AgdaSymbol{(}\AgdaFunction{nrep} \AgdaBound{ρ} \AgdaBound{N}\AgdaSymbol{)}\<%
\\
\>\AgdaFunction{lrep} \AgdaBound{ρ} \AgdaInductiveConstructor{bot} \AgdaSymbol{=} \AgdaInductiveConstructor{bot}\<%
\\
\>\AgdaFunction{lrep} \AgdaBound{ρ} \AgdaSymbol{(}\AgdaInductiveConstructor{imp} \AgdaBound{φ} \AgdaBound{ψ}\AgdaSymbol{)} \AgdaSymbol{=} \AgdaInductiveConstructor{imp} \AgdaSymbol{(}\AgdaFunction{lrep} \AgdaBound{ρ} \AgdaBound{φ}\AgdaSymbol{)} \AgdaSymbol{(}\AgdaFunction{lrep} \AgdaBound{ρ} \AgdaBound{ψ}\AgdaSymbol{)}\<%
\\
%
\\
\>\AgdaFunction{decode-Prop} \AgdaSymbol{:} \AgdaSymbol{∀} \AgdaSymbol{\{}\AgdaBound{V}\AgdaSymbol{\}} \AgdaSymbol{\{}\AgdaBound{S}\AgdaSymbol{\}} \AgdaSymbol{→} \AgdaDatatype{Leaves} \AgdaBound{V} \AgdaBound{S} \AgdaSymbol{→} \AgdaFunction{Term} \AgdaBound{V}\<%
\\
\>\AgdaFunction{decode-Prop} \AgdaSymbol{(}\AgdaInductiveConstructor{neutral} \AgdaBound{N}\AgdaSymbol{)} \AgdaSymbol{=} \AgdaFunction{decode-Neutral} \AgdaBound{N}\<%
\\
\>\AgdaFunction{decode-Prop} \AgdaInductiveConstructor{bot} \AgdaSymbol{=} \AgdaFunction{⊥}\<%
\\
\>\AgdaFunction{decode-Prop} \AgdaSymbol{(}\AgdaInductiveConstructor{imp} \AgdaBound{φ} \AgdaBound{ψ}\AgdaSymbol{)} \AgdaSymbol{=} \AgdaFunction{decode-Prop} \AgdaBound{φ} \AgdaFunction{⊃} \AgdaFunction{decode-Prop} \AgdaBound{ψ}\<%
\\
%
\\
\>\AgdaFunction{leaves-red} \AgdaSymbol{:} \AgdaSymbol{∀} \AgdaSymbol{\{}\AgdaBound{V}\AgdaSymbol{\}} \AgdaSymbol{\{}\AgdaBound{S}\AgdaSymbol{\}} \AgdaSymbol{\{}\AgdaBound{L} \AgdaSymbol{:} \AgdaDatatype{Leaves} \AgdaBound{V} \AgdaBound{S}\AgdaSymbol{\}} \AgdaSymbol{\{}\AgdaBound{φ} \AgdaSymbol{:} \AgdaFunction{Term} \AgdaBound{V}\AgdaSymbol{\}} \AgdaSymbol{→}\<%
\\
\>[0]\AgdaIndent{2}{}\<[2]%
\>[2]\AgdaFunction{decode-Prop} \AgdaBound{L} \AgdaDatatype{↠} \AgdaBound{φ} \AgdaSymbol{→}\<%
\\
\>[0]\AgdaIndent{2}{}\<[2]%
\>[2]\AgdaFunction{Σ[} \AgdaBound{L'} \AgdaFunction{∈} \AgdaDatatype{Leaves} \AgdaBound{V} \AgdaBound{S} \AgdaFunction{]} \AgdaFunction{decode-Prop} \AgdaBound{L'} \AgdaDatatype{≡} \AgdaBound{φ}\<%
\\
\>\AgdaFunction{leaves-red} \AgdaSymbol{\{}\AgdaArgument{S} \AgdaSymbol{=} \AgdaInductiveConstructor{neutral}\AgdaSymbol{\}} \AgdaSymbol{\{}\AgdaArgument{L} \AgdaSymbol{=} \AgdaInductiveConstructor{neutral} \AgdaBound{N}\AgdaSymbol{\}} \AgdaBound{L↠φ} \AgdaSymbol{=} \<[47]%
\>[47]\<%
\\
\>[0]\AgdaIndent{2}{}\<[2]%
\>[2]\AgdaKeyword{let} \AgdaSymbol{(}\AgdaBound{N} \AgdaInductiveConstructor{,p} \AgdaBound{N≡φ}\AgdaSymbol{)} \AgdaSymbol{=} \AgdaFunction{neutral-red} \AgdaSymbol{\{}\AgdaArgument{N} \AgdaSymbol{=} \AgdaBound{N}\AgdaSymbol{\}} \AgdaBound{L↠φ} \AgdaKeyword{in} \AgdaInductiveConstructor{neutral} \AgdaBound{N} \AgdaInductiveConstructor{,p} \AgdaBound{N≡φ}\<%
\\
\>\AgdaFunction{leaves-red} \AgdaSymbol{\{}\AgdaArgument{S} \AgdaSymbol{=} \AgdaInductiveConstructor{bot}\AgdaSymbol{\}} \AgdaSymbol{\{}\AgdaArgument{L} \AgdaSymbol{=} \AgdaInductiveConstructor{bot}\AgdaSymbol{\}} \AgdaBound{L↠φ} \AgdaSymbol{=} \AgdaInductiveConstructor{bot} \AgdaInductiveConstructor{,p} \AgdaFunction{sym} \AgdaSymbol{(}\AgdaFunction{bot-red} \AgdaBound{L↠φ}\AgdaSymbol{)}\<%
\\
\>\AgdaFunction{leaves-red} \AgdaSymbol{\{}\AgdaArgument{S} \AgdaSymbol{=} \AgdaInductiveConstructor{imp} \AgdaBound{S} \AgdaBound{T}\AgdaSymbol{\}} \AgdaSymbol{\{}\AgdaArgument{L} \AgdaSymbol{=} \AgdaInductiveConstructor{imp} \AgdaBound{φ} \AgdaBound{ψ}\AgdaSymbol{\}} \AgdaBound{φ⊃ψ↠χ} \AgdaSymbol{=} \<[47]%
\>[47]\<%
\\
\>[0]\AgdaIndent{2}{}\<[2]%
\>[2]\AgdaKeyword{let} \AgdaSymbol{(}\AgdaBound{φ'} \AgdaInductiveConstructor{,p} \AgdaBound{ψ'} \AgdaInductiveConstructor{,p} \AgdaBound{φ↠φ'} \AgdaInductiveConstructor{,p} \AgdaBound{ψ↠ψ'} \AgdaInductiveConstructor{,p} \AgdaBound{χ≡φ'⊃ψ'}\AgdaSymbol{)} \AgdaSymbol{=} \AgdaFunction{imp-red} \AgdaBound{φ⊃ψ↠χ} \AgdaKeyword{in} \<[63]%
\>[63]\<%
\\
\>[0]\AgdaIndent{2}{}\<[2]%
\>[2]\AgdaKeyword{let} \AgdaSymbol{(}\AgdaBound{L₁} \AgdaInductiveConstructor{,p} \AgdaBound{L₁≡φ'}\AgdaSymbol{)} \AgdaSymbol{=} \AgdaFunction{leaves-red} \AgdaSymbol{\{}\AgdaArgument{L} \AgdaSymbol{=} \AgdaBound{φ}\AgdaSymbol{\}} \AgdaBound{φ↠φ'} \AgdaKeyword{in} \<[49]%
\>[49]\<%
\\
\>[0]\AgdaIndent{2}{}\<[2]%
\>[2]\AgdaKeyword{let} \AgdaSymbol{(}\AgdaBound{L₂} \AgdaInductiveConstructor{,p} \AgdaBound{L₂≡ψ'}\AgdaSymbol{)} \AgdaSymbol{=} \AgdaFunction{leaves-red} \AgdaSymbol{\{}\AgdaArgument{L} \AgdaSymbol{=} \AgdaBound{ψ}\AgdaSymbol{\}} \AgdaBound{ψ↠ψ'} \AgdaKeyword{in} \<[49]%
\>[49]\<%
\\
\>[0]\AgdaIndent{2}{}\<[2]%
\>[2]\AgdaSymbol{(}\AgdaInductiveConstructor{imp} \AgdaBound{L₁} \AgdaBound{L₂}\AgdaSymbol{)} \AgdaInductiveConstructor{,p} \AgdaSymbol{(}\AgdaFunction{trans} \AgdaSymbol{(}\AgdaFunction{cong₂} \AgdaFunction{\_⊃\_} \AgdaBound{L₁≡φ'} \AgdaBound{L₂≡ψ'}\AgdaSymbol{)} \AgdaSymbol{(}\AgdaFunction{sym} \AgdaBound{χ≡φ'⊃ψ'}\AgdaSymbol{))}\<%
\\
%
\\
\>\AgdaFunction{computeP} \AgdaSymbol{:} \AgdaSymbol{∀} \AgdaSymbol{\{}\AgdaBound{V}\AgdaSymbol{\}} \AgdaSymbol{\{}\AgdaBound{S}\AgdaSymbol{\}} \AgdaSymbol{→} \AgdaDatatype{Context} \AgdaBound{V} \AgdaSymbol{→} \AgdaDatatype{Leaves} \AgdaBound{V} \AgdaBound{S} \AgdaSymbol{→} \AgdaFunction{Proof} \AgdaBound{V} \AgdaSymbol{→} \AgdaPrimitiveType{Set}\<%
\\
\>\AgdaFunction{computeP} \AgdaSymbol{\{}\AgdaArgument{S} \AgdaSymbol{=} \AgdaInductiveConstructor{neutral}\AgdaSymbol{\}} \AgdaBound{Γ} \AgdaSymbol{(}\AgdaInductiveConstructor{neutral} \AgdaSymbol{\_)} \AgdaBound{δ} \AgdaSymbol{=} \AgdaDatatype{SN} \AgdaBound{δ}\<%
\\
\>\AgdaFunction{computeP} \AgdaSymbol{\{}\AgdaArgument{S} \AgdaSymbol{=} \AgdaInductiveConstructor{bot}\AgdaSymbol{\}} \AgdaBound{Γ} \AgdaInductiveConstructor{bot} \AgdaBound{δ} \AgdaSymbol{=} \AgdaDatatype{SN} \AgdaBound{δ}\<%
\\
\>\AgdaFunction{computeP} \AgdaSymbol{\{}\AgdaArgument{S} \AgdaSymbol{=} \AgdaInductiveConstructor{imp} \AgdaBound{S} \AgdaBound{T}\AgdaSymbol{\}} \AgdaBound{Γ} \AgdaSymbol{(}\AgdaInductiveConstructor{imp} \AgdaBound{φ} \AgdaBound{ψ}\AgdaSymbol{)} \AgdaBound{δ} \AgdaSymbol{=} \<[39]%
\>[39]\<%
\\
\>[0]\AgdaIndent{2}{}\<[2]%
\>[2]\AgdaSymbol{∀} \AgdaSymbol{\{}\AgdaBound{W}\AgdaSymbol{\}} \AgdaSymbol{(}\AgdaBound{Δ} \AgdaSymbol{:} \AgdaDatatype{Context} \AgdaBound{W}\AgdaSymbol{)} \AgdaSymbol{\{}\AgdaBound{ρ}\AgdaSymbol{\}} \AgdaSymbol{\{}\AgdaBound{ε}\AgdaSymbol{\}}\<%
\\
\>[0]\AgdaIndent{2}{}\<[2]%
\>[2]\AgdaSymbol{(}\AgdaBound{ρ∶Γ⇒RΔ} \AgdaSymbol{:} \AgdaBound{ρ} \AgdaPostulate{∶} \AgdaBound{Γ} \AgdaPostulate{⇒R} \AgdaBound{Δ}\AgdaSymbol{)} \AgdaSymbol{(}\AgdaBound{Δ⊢ε∶φ} \AgdaSymbol{:} \AgdaBound{Δ} \AgdaDatatype{⊢} \AgdaBound{ε} \AgdaDatatype{∶} \AgdaSymbol{(}\AgdaFunction{decode-Prop} \AgdaSymbol{(}\AgdaFunction{lrep} \AgdaBound{ρ} \AgdaBound{φ}\AgdaSymbol{)))}\<%
\\
\>[0]\AgdaIndent{2}{}\<[2]%
\>[2]\AgdaSymbol{(}\AgdaBound{computeε} \AgdaSymbol{:} \AgdaFunction{computeP} \AgdaSymbol{\{}\AgdaArgument{S} \AgdaSymbol{=} \AgdaBound{S}\AgdaSymbol{\}} \AgdaBound{Δ} \AgdaSymbol{(}\AgdaFunction{lrep} \AgdaBound{ρ} \AgdaBound{φ}\AgdaSymbol{)} \AgdaBound{ε}\AgdaSymbol{)} \AgdaSymbol{→} \<[49]%
\>[49]\<%
\\
\>[0]\AgdaIndent{2}{}\<[2]%
\>[2]\AgdaFunction{computeP} \AgdaSymbol{\{}\AgdaArgument{S} \AgdaSymbol{=} \AgdaBound{T}\AgdaSymbol{\}} \AgdaBound{Δ} \AgdaSymbol{(}\AgdaFunction{lrep} \AgdaBound{ρ} \AgdaBound{ψ}\AgdaSymbol{)} \AgdaSymbol{(}\AgdaFunction{appP} \AgdaSymbol{(}\AgdaBound{δ} \AgdaFunction{〈} \AgdaBound{ρ} \AgdaFunction{〉}\AgdaSymbol{)} \AgdaBound{ε}\AgdaSymbol{)}\<%
\\
%
\\
\>\AgdaFunction{computeT} \AgdaSymbol{:} \AgdaSymbol{∀} \AgdaSymbol{\{}\AgdaBound{V}\AgdaSymbol{\}} \AgdaSymbol{→} \AgdaDatatype{Context} \AgdaBound{V} \AgdaSymbol{→} \AgdaDatatype{Type} \AgdaSymbol{→} \AgdaFunction{Term} \AgdaBound{V} \AgdaSymbol{→} \AgdaPrimitiveType{Set}\<%
\\
\>\AgdaFunction{computeE} \AgdaSymbol{:} \AgdaSymbol{∀} \AgdaSymbol{\{}\AgdaBound{V}\AgdaSymbol{\}} \AgdaSymbol{→} \AgdaDatatype{Context} \AgdaBound{V} \AgdaSymbol{→} \AgdaFunction{Term} \AgdaBound{V} \AgdaSymbol{→} \AgdaDatatype{Type} \AgdaSymbol{→} \AgdaFunction{Term} \AgdaBound{V} \AgdaSymbol{→} \AgdaFunction{Path} \AgdaBound{V} \AgdaSymbol{→} \AgdaPrimitiveType{Set}\<%
\\
%
\\
\>\AgdaFunction{computeT} \AgdaBound{Γ} \AgdaInductiveConstructor{Ω} \AgdaBound{M} \AgdaSymbol{=} \AgdaDatatype{SN} \AgdaBound{M}\<%
\\
\>\AgdaFunction{computeT} \AgdaBound{Γ} \AgdaSymbol{(}\AgdaBound{A} \AgdaInductiveConstructor{⇛} \AgdaBound{B}\AgdaSymbol{)} \AgdaBound{M} \AgdaSymbol{=} \<[23]%
\>[23]\<%
\\
\>[0]\AgdaIndent{2}{}\<[2]%
\>[2]\AgdaSymbol{(∀} \AgdaSymbol{\{}\AgdaBound{W}\AgdaSymbol{\}} \AgdaSymbol{(}\AgdaBound{Δ} \AgdaSymbol{:} \AgdaDatatype{Context} \AgdaBound{W}\AgdaSymbol{)} \AgdaSymbol{\{}\AgdaBound{ρ}\AgdaSymbol{\}} \AgdaSymbol{\{}\AgdaBound{N}\AgdaSymbol{\}} \AgdaSymbol{(}\AgdaBound{ρ∶Γ⇒Δ} \AgdaSymbol{:} \AgdaBound{ρ} \AgdaPostulate{∶} \AgdaBound{Γ} \AgdaPostulate{⇒R} \AgdaBound{Δ}\AgdaSymbol{)} \AgdaSymbol{(}\AgdaBound{Δ⊢N∶A} \AgdaSymbol{:} \AgdaBound{Δ} \AgdaDatatype{⊢} \AgdaBound{N} \AgdaDatatype{∶} \AgdaFunction{ty} \AgdaBound{A}\AgdaSymbol{)} \AgdaSymbol{(}\AgdaBound{computeN} \AgdaSymbol{:} \AgdaFunction{computeT} \AgdaBound{Δ} \AgdaBound{A} \AgdaBound{N}\AgdaSymbol{)} \AgdaSymbol{→}\<%
\\
\>[2]\AgdaIndent{4}{}\<[4]%
\>[4]\AgdaFunction{computeT} \AgdaBound{Δ} \AgdaBound{B} \AgdaSymbol{(}\AgdaFunction{appT} \AgdaSymbol{(}\AgdaBound{M} \AgdaFunction{〈} \AgdaBound{ρ} \AgdaFunction{〉}\AgdaSymbol{)} \AgdaBound{N}\AgdaSymbol{))} \AgdaFunction{×}\<%
\\
\>[0]\AgdaIndent{2}{}\<[2]%
\>[2]\AgdaSymbol{(∀} \AgdaSymbol{\{}\AgdaBound{W}\AgdaSymbol{\}} \AgdaSymbol{(}\AgdaBound{Δ} \AgdaSymbol{:} \AgdaDatatype{Context} \AgdaBound{W}\AgdaSymbol{)} \AgdaSymbol{\{}\AgdaBound{ρ}\AgdaSymbol{\}} \AgdaSymbol{\{}\AgdaBound{N}\AgdaSymbol{\}} \AgdaSymbol{\{}\AgdaBound{N'}\AgdaSymbol{\}} \AgdaSymbol{\{}\AgdaBound{P}\AgdaSymbol{\}} \<[42]%
\>[42]\<%
\\
\>[2]\AgdaIndent{4}{}\<[4]%
\>[4]\AgdaSymbol{(}\AgdaBound{ρ∶Γ⇒Δ} \AgdaSymbol{:} \AgdaBound{ρ} \AgdaPostulate{∶} \AgdaBound{Γ} \AgdaPostulate{⇒R} \AgdaBound{Δ}\AgdaSymbol{)} \AgdaSymbol{(}\AgdaBound{Δ⊢P∶N≡N'} \AgdaSymbol{:} \AgdaBound{Δ} \AgdaDatatype{⊢} \AgdaBound{P} \AgdaDatatype{∶} \AgdaBound{N} \AgdaFunction{≡〈} \AgdaBound{A} \AgdaFunction{〉} \AgdaBound{N'}\AgdaSymbol{)} \<[58]%
\>[58]\<%
\\
\>[2]\AgdaIndent{4}{}\<[4]%
\>[4]\AgdaSymbol{(}\AgdaBound{computeN} \AgdaSymbol{:} \AgdaFunction{computeT} \AgdaBound{Δ} \AgdaBound{A} \AgdaBound{N}\AgdaSymbol{)} \AgdaSymbol{(}\AgdaBound{computeN'} \AgdaSymbol{:} \AgdaFunction{computeT} \AgdaBound{Δ} \AgdaBound{A} \AgdaBound{N'}\AgdaSymbol{)} \AgdaSymbol{(}\AgdaBound{computeP} \AgdaSymbol{:} \AgdaFunction{computeE} \AgdaBound{Δ} \AgdaBound{N} \AgdaBound{A} \AgdaBound{N'} \AgdaBound{P}\AgdaSymbol{)} \AgdaSymbol{→}\<%
\\
\>[2]\AgdaIndent{4}{}\<[4]%
\>[4]\AgdaFunction{computeE} \AgdaBound{Δ} \AgdaSymbol{(}\AgdaFunction{appT} \AgdaSymbol{(}\AgdaBound{M} \AgdaFunction{〈} \AgdaBound{ρ} \AgdaFunction{〉}\AgdaSymbol{)} \AgdaBound{N}\AgdaSymbol{)} \AgdaBound{B} \AgdaSymbol{(}\AgdaFunction{appT} \AgdaSymbol{(}\AgdaBound{M} \AgdaFunction{〈} \AgdaBound{ρ} \AgdaFunction{〉}\AgdaSymbol{)} \AgdaBound{N'}\AgdaSymbol{)} \AgdaSymbol{(}\AgdaBound{M} \AgdaFunction{〈} \AgdaBound{ρ} \AgdaFunction{〉} \AgdaFunction{⋆[} \AgdaBound{P} \AgdaFunction{∶} \AgdaBound{N} \AgdaFunction{∼} \AgdaBound{N'} \AgdaFunction{]}\AgdaSymbol{))}\<%
\\
%
\\
\>\AgdaFunction{computeE} \AgdaSymbol{\{}\AgdaBound{V}\AgdaSymbol{\}} \AgdaBound{Γ} \AgdaBound{M} \AgdaInductiveConstructor{Ω} \AgdaBound{N} \AgdaBound{P} \AgdaSymbol{=} \AgdaFunction{Σ[} \AgdaBound{S} \AgdaFunction{∈} \AgdaDatatype{Shape} \AgdaFunction{]} \AgdaFunction{Σ[} \AgdaBound{T} \AgdaFunction{∈} \AgdaDatatype{Shape} \AgdaFunction{]} \AgdaFunction{Σ[} \AgdaBound{L} \AgdaFunction{∈} \AgdaDatatype{Leaves} \AgdaBound{V} \AgdaBound{S} \AgdaFunction{]} \AgdaFunction{Σ[} \AgdaBound{L'} \AgdaFunction{∈} \AgdaDatatype{Leaves} \AgdaBound{V} \AgdaBound{T} \AgdaFunction{]} \AgdaBound{M} \AgdaDatatype{↠} \AgdaFunction{decode-Prop} \AgdaBound{L} \AgdaFunction{×} \AgdaBound{N} \AgdaDatatype{↠} \AgdaFunction{decode-Prop} \AgdaBound{L'} \AgdaFunction{×} \AgdaFunction{computeP} \AgdaBound{Γ} \AgdaSymbol{(}\AgdaInductiveConstructor{imp} \AgdaBound{L} \AgdaBound{L'}\AgdaSymbol{)} \AgdaSymbol{(}\AgdaFunction{plus} \AgdaBound{P}\AgdaSymbol{)} \AgdaFunction{×} \AgdaFunction{computeP} \AgdaBound{Γ} \AgdaSymbol{(}\AgdaInductiveConstructor{imp} \AgdaBound{L'} \AgdaBound{L}\AgdaSymbol{)} \AgdaSymbol{(}\AgdaFunction{minus} \AgdaBound{P}\AgdaSymbol{)}\<%
\\
\>\AgdaFunction{computeE} \AgdaBound{Γ} \AgdaBound{M} \AgdaSymbol{(}\AgdaBound{A} \AgdaInductiveConstructor{⇛} \AgdaBound{B}\AgdaSymbol{)} \AgdaBound{M'} \AgdaBound{P} \AgdaSymbol{=}\<%
\\
\>[0]\AgdaIndent{2}{}\<[2]%
\>[2]\AgdaSymbol{∀} \AgdaSymbol{\{}\AgdaBound{W}\AgdaSymbol{\}} \AgdaSymbol{(}\AgdaBound{Δ} \AgdaSymbol{:} \AgdaDatatype{Context} \AgdaBound{W}\AgdaSymbol{)} \AgdaSymbol{\{}\AgdaBound{ρ}\AgdaSymbol{\}} \AgdaSymbol{\{}\AgdaBound{N}\AgdaSymbol{\}} \AgdaSymbol{\{}\AgdaBound{N'}\AgdaSymbol{\}} \AgdaSymbol{\{}\AgdaBound{Q}\AgdaSymbol{\}} \AgdaSymbol{(}\AgdaBound{ρ∶Γ⇒RΔ} \AgdaSymbol{:} \AgdaBound{ρ} \AgdaPostulate{∶} \AgdaBound{Γ} \AgdaPostulate{⇒R} \AgdaBound{Δ}\AgdaSymbol{)} \AgdaSymbol{(}\AgdaBound{Δ⊢Q∶N≡N'} \AgdaSymbol{:} \AgdaBound{Δ} \AgdaDatatype{⊢} \AgdaBound{Q} \AgdaDatatype{∶} \AgdaBound{N} \AgdaFunction{≡〈} \AgdaBound{A} \AgdaFunction{〉} \AgdaBound{N'}\AgdaSymbol{)}\<%
\\
\>[0]\AgdaIndent{2}{}\<[2]%
\>[2]\AgdaSymbol{(}\AgdaBound{computeQ} \AgdaSymbol{:} \AgdaFunction{computeE} \AgdaBound{Δ} \AgdaBound{N} \AgdaBound{A} \AgdaBound{N'} \AgdaBound{Q}\AgdaSymbol{)} \AgdaSymbol{→} \AgdaFunction{computeE} \AgdaBound{Δ} \AgdaSymbol{(}\AgdaFunction{appT} \AgdaSymbol{(}\AgdaBound{M} \AgdaFunction{〈} \AgdaBound{ρ} \AgdaFunction{〉}\AgdaSymbol{)} \AgdaBound{N}\AgdaSymbol{)} \AgdaBound{B} \AgdaSymbol{(}\AgdaFunction{appT} \AgdaSymbol{(}\AgdaBound{M'} \AgdaFunction{〈} \AgdaBound{ρ} \AgdaFunction{〉}\AgdaSymbol{)} \<[87]%
\>[87]\AgdaBound{N'}\AgdaSymbol{)} \<[91]%
\>[91]\<%
\\
\>[2]\AgdaIndent{4}{}\<[4]%
\>[4]\AgdaSymbol{(}\AgdaFunction{app*} \AgdaBound{N} \AgdaBound{N'} \AgdaSymbol{(}\AgdaBound{P} \AgdaFunction{〈} \AgdaBound{ρ} \AgdaFunction{〉}\AgdaSymbol{)} \AgdaBound{Q}\AgdaSymbol{)}\<%
\\
%
\\
\>\AgdaKeyword{postulate} \AgdaPostulate{decode-rep} \AgdaSymbol{:} \AgdaSymbol{∀} \AgdaSymbol{\{}\AgdaBound{U}\AgdaSymbol{\}} \AgdaSymbol{\{}\AgdaBound{V}\AgdaSymbol{\}} \AgdaSymbol{\{}\AgdaBound{S}\AgdaSymbol{\}} \AgdaSymbol{(}\AgdaBound{L} \AgdaSymbol{:} \AgdaDatatype{Leaves} \AgdaBound{U} \AgdaBound{S}\AgdaSymbol{)} \AgdaSymbol{\{}\AgdaBound{ρ} \AgdaSymbol{:} \AgdaFunction{Rep} \AgdaBound{U} \AgdaBound{V}\AgdaSymbol{\}} \AgdaSymbol{→}\<%
\\
\>[4]\AgdaIndent{21}{}\<[21]%
\>[21]\AgdaFunction{decode-Prop} \AgdaSymbol{(}\AgdaFunction{lrep} \AgdaBound{ρ} \AgdaBound{L}\AgdaSymbol{)} \AgdaDatatype{≡} \AgdaFunction{decode-Prop} \AgdaBound{L} \AgdaFunction{〈} \AgdaBound{ρ} \AgdaFunction{〉}\<%
\\
%
\\
\>\AgdaKeyword{postulate} \AgdaPostulate{conv-computeP} \AgdaSymbol{:} \AgdaSymbol{∀} \AgdaSymbol{\{}\AgdaBound{V}\AgdaSymbol{\}} \AgdaSymbol{\{}\AgdaBound{Γ} \AgdaSymbol{:} \AgdaDatatype{Context} \AgdaBound{V}\AgdaSymbol{\}} \AgdaSymbol{\{}\AgdaBound{S}\AgdaSymbol{\}} \AgdaSymbol{\{}\AgdaBound{L} \AgdaBound{M} \AgdaSymbol{:} \AgdaDatatype{Leaves} \AgdaBound{V} \AgdaBound{S}\AgdaSymbol{\}} \AgdaSymbol{\{}\AgdaBound{δ}\AgdaSymbol{\}} \AgdaSymbol{→}\<%
\\
\>[21]\AgdaIndent{24}{}\<[24]%
\>[24]\AgdaFunction{computeP} \AgdaBound{Γ} \AgdaBound{L} \AgdaBound{δ} \AgdaSymbol{→} \AgdaFunction{decode-Prop} \AgdaBound{L} \AgdaDatatype{≃} \AgdaFunction{decode-Prop} \AgdaBound{M} \AgdaSymbol{→}\<%
\\
\>[21]\AgdaIndent{24}{}\<[24]%
\>[24]\AgdaBound{Γ} \AgdaDatatype{⊢} \AgdaFunction{decode-Prop} \AgdaBound{M} \AgdaDatatype{∶} \AgdaFunction{ty} \AgdaInductiveConstructor{Ω} \AgdaSymbol{→} \AgdaFunction{computeP} \AgdaBound{Γ} \AgdaBound{M} \AgdaBound{δ}\<%
\\
%
\\
\>\AgdaFunction{conv-computeE} \AgdaSymbol{:} \AgdaSymbol{∀} \AgdaSymbol{\{}\AgdaBound{V}\AgdaSymbol{\}} \AgdaSymbol{\{}\AgdaBound{Γ} \AgdaSymbol{:} \AgdaDatatype{Context} \AgdaBound{V}\AgdaSymbol{\}} \AgdaSymbol{\{}\AgdaBound{M}\AgdaSymbol{\}} \AgdaSymbol{\{}\AgdaBound{M'}\AgdaSymbol{\}} \AgdaSymbol{\{}\AgdaBound{A}\AgdaSymbol{\}} \AgdaSymbol{\{}\AgdaBound{N}\AgdaSymbol{\}} \AgdaSymbol{\{}\AgdaBound{N'}\AgdaSymbol{\}} \AgdaSymbol{\{}\AgdaBound{P}\AgdaSymbol{\}} \AgdaSymbol{→}\<%
\\
\>[0]\AgdaIndent{2}{}\<[2]%
\>[2]\AgdaFunction{computeE} \AgdaBound{Γ} \AgdaBound{M} \AgdaBound{A} \AgdaBound{N} \AgdaBound{P} \AgdaSymbol{→} \<[23]%
\>[23]\<%
\\
\>[0]\AgdaIndent{2}{}\<[2]%
\>[2]\AgdaBound{Γ} \AgdaDatatype{⊢} \AgdaBound{M} \AgdaDatatype{∶} \AgdaFunction{ty} \AgdaBound{A} \AgdaSymbol{→} \AgdaBound{Γ} \AgdaDatatype{⊢} \AgdaBound{N} \AgdaDatatype{∶} \AgdaFunction{ty} \AgdaBound{A} \AgdaSymbol{→} \AgdaBound{Γ} \AgdaDatatype{⊢} \AgdaBound{M'} \AgdaDatatype{∶} \AgdaFunction{ty} \AgdaBound{A} \AgdaSymbol{→} \AgdaBound{Γ} \AgdaDatatype{⊢} \AgdaBound{N'} \AgdaDatatype{∶} \AgdaFunction{ty} \AgdaBound{A} \AgdaSymbol{→} \AgdaBound{M} \AgdaDatatype{≃} \AgdaBound{M'} \AgdaSymbol{→} \AgdaBound{N} \AgdaDatatype{≃} \AgdaBound{N'} \AgdaSymbol{→}\<%
\\
\>[0]\AgdaIndent{2}{}\<[2]%
\>[2]\AgdaFunction{computeE} \AgdaBound{Γ} \AgdaBound{M'} \AgdaBound{A} \AgdaBound{N'} \AgdaBound{P}\<%
\\
\>\AgdaFunction{conv-computeE} \AgdaSymbol{\{}\AgdaArgument{Γ} \AgdaSymbol{=} \AgdaBound{Γ}\AgdaSymbol{\}} \AgdaSymbol{\{}\AgdaArgument{M} \AgdaSymbol{=} \AgdaBound{M}\AgdaSymbol{\}} \AgdaSymbol{\{}\AgdaArgument{M'} \AgdaSymbol{=} \AgdaBound{M'}\AgdaSymbol{\}} \AgdaSymbol{\{}\AgdaArgument{A} \AgdaSymbol{=} \AgdaInductiveConstructor{Ω}\AgdaSymbol{\}} \AgdaSymbol{\{}\AgdaArgument{N'} \AgdaSymbol{=} \AgdaBound{N'}\AgdaSymbol{\}} \AgdaSymbol{\{}\AgdaBound{P}\AgdaSymbol{\}} \AgdaSymbol{(}\AgdaBound{S} \AgdaInductiveConstructor{,p} \AgdaBound{T} \AgdaInductiveConstructor{,p} \AgdaBound{φ} \AgdaInductiveConstructor{,p} \AgdaBound{ψ} \AgdaInductiveConstructor{,p} \AgdaBound{M↠φ} \AgdaInductiveConstructor{,p} \AgdaBound{N↠ψ} \AgdaInductiveConstructor{,p} \AgdaBound{computeP+} \AgdaInductiveConstructor{,p} \AgdaBound{computeP-}\AgdaSymbol{)} \<[121]%
\>[121]\<%
\\
\>[0]\AgdaIndent{2}{}\<[2]%
\>[2]\AgdaBound{Γ⊢M∶A} \AgdaBound{Γ⊢N∶A} \AgdaBound{Γ⊢M'∶A} \AgdaBound{Γ⊢N'∶A} \AgdaBound{M≃M'} \AgdaBound{N≃N'} \AgdaSymbol{=} \<[40]%
\>[40]\<%
\\
\>[2]\AgdaIndent{4}{}\<[4]%
\>[4]\AgdaKeyword{let} \AgdaSymbol{(}\AgdaBound{Q} \AgdaInductiveConstructor{,p} \AgdaBound{φ↠Q} \AgdaInductiveConstructor{,p} \AgdaBound{M'↠Q}\AgdaSymbol{)} \AgdaSymbol{=} \AgdaPostulate{confluenceT} \AgdaSymbol{(}\AgdaInductiveConstructor{trans-conv} \AgdaSymbol{(}\AgdaInductiveConstructor{sym-conv} \AgdaSymbol{(}\AgdaFunction{red-conv} \AgdaBound{M↠φ}\AgdaSymbol{))} \AgdaBound{M≃M'}\AgdaSymbol{)} \AgdaKeyword{in}\<%
\\
\>[2]\AgdaIndent{4}{}\<[4]%
\>[4]\AgdaKeyword{let} \AgdaSymbol{(}\AgdaBound{φ'} \AgdaInductiveConstructor{,p} \AgdaBound{φ'≡Q}\AgdaSymbol{)} \AgdaSymbol{=} \AgdaFunction{leaves-red} \AgdaSymbol{\{}\AgdaArgument{L} \AgdaSymbol{=} \AgdaBound{φ}\AgdaSymbol{\}} \AgdaBound{φ↠Q} \AgdaKeyword{in}\<%
\\
\>[2]\AgdaIndent{4}{}\<[4]%
\>[4]\AgdaKeyword{let} \AgdaSymbol{(}\AgdaBound{R} \AgdaInductiveConstructor{,p} \AgdaBound{ψ↠R} \AgdaInductiveConstructor{,p} \AgdaBound{N'↠R}\AgdaSymbol{)} \AgdaSymbol{=} \AgdaPostulate{confluenceT} \AgdaSymbol{(}\AgdaInductiveConstructor{trans-conv} \AgdaSymbol{(}\AgdaInductiveConstructor{sym-conv} \AgdaSymbol{(}\AgdaFunction{red-conv} \AgdaBound{N↠ψ}\AgdaSymbol{))} \AgdaBound{N≃N'}\AgdaSymbol{)} \AgdaKeyword{in}\<%
\\
\>[2]\AgdaIndent{4}{}\<[4]%
\>[4]\AgdaKeyword{let} \AgdaSymbol{(}\AgdaBound{ψ'} \AgdaInductiveConstructor{,p} \AgdaBound{ψ'≡R}\AgdaSymbol{)} \AgdaSymbol{=} \AgdaFunction{leaves-red} \AgdaSymbol{\{}\AgdaArgument{L} \AgdaSymbol{=} \AgdaBound{ψ}\AgdaSymbol{\}} \AgdaBound{ψ↠R} \AgdaKeyword{in}\<%
\\
\>[2]\AgdaIndent{4}{}\<[4]%
\>[4]\AgdaBound{S} \AgdaInductiveConstructor{,p} \AgdaBound{T} \AgdaInductiveConstructor{,p} \AgdaBound{φ'} \AgdaInductiveConstructor{,p} \AgdaBound{ψ'} \AgdaInductiveConstructor{,p} \AgdaFunction{subst} \AgdaSymbol{(}\AgdaDatatype{\_↠\_} \AgdaBound{M'}\AgdaSymbol{)} \AgdaSymbol{(}\AgdaFunction{sym} \AgdaBound{φ'≡Q}\AgdaSymbol{)} \AgdaBound{M'↠Q} \AgdaInductiveConstructor{,p} \<[60]%
\>[60]\<%
\\
\>[2]\AgdaIndent{4}{}\<[4]%
\>[4]\AgdaFunction{subst} \AgdaSymbol{(}\AgdaDatatype{\_↠\_} \AgdaBound{N'}\AgdaSymbol{)} \AgdaSymbol{(}\AgdaFunction{sym} \AgdaBound{ψ'≡R}\AgdaSymbol{)} \AgdaBound{N'↠R} \AgdaInductiveConstructor{,p} \<[38]%
\>[38]\<%
\\
\>[2]\AgdaIndent{4}{}\<[4]%
\>[4]\AgdaSymbol{(λ} \AgdaBound{Δ} \AgdaSymbol{\{}\AgdaBound{ρ}\AgdaSymbol{\}} \AgdaSymbol{\{}\AgdaBound{ε}\AgdaSymbol{\}} \AgdaBound{ρ∶Γ⇒RΔ} \AgdaBound{Δ⊢ε∶φ'ρ} \AgdaBound{computeε} \AgdaSymbol{→} \<[43]%
\>[43]\<%
\\
\>[4]\AgdaIndent{6}{}\<[6]%
\>[6]\AgdaKeyword{let} \AgdaBound{step1} \AgdaSymbol{:} \AgdaBound{Δ} \AgdaDatatype{⊢} \AgdaFunction{decode-Prop} \AgdaSymbol{(}\AgdaFunction{lrep} \AgdaBound{ρ} \AgdaBound{φ}\AgdaSymbol{)} \AgdaDatatype{∶} \AgdaFunction{ty} \AgdaInductiveConstructor{Ω}\<%
\\
\>[6]\AgdaIndent{10}{}\<[10]%
\>[10]\AgdaBound{step1} \AgdaSymbol{=} \AgdaFunction{subst} \AgdaSymbol{(λ} \AgdaBound{x} \AgdaSymbol{→} \AgdaBound{Δ} \AgdaDatatype{⊢} \AgdaBound{x} \AgdaDatatype{∶} \AgdaFunction{ty} \AgdaInductiveConstructor{Ω}\AgdaSymbol{)} \<[45]%
\>[45]\<%
\\
\>[10]\AgdaIndent{12}{}\<[12]%
\>[12]\AgdaSymbol{(}\AgdaFunction{sym} \AgdaSymbol{(}\AgdaPostulate{decode-rep} \AgdaBound{φ}\AgdaSymbol{))} \<[33]%
\>[33]\<%
\\
\>[10]\AgdaIndent{12}{}\<[12]%
\>[12]\AgdaSymbol{(}\AgdaPostulate{weakening} \<[23]%
\>[23]\<%
\\
\>[12]\AgdaIndent{14}{}\<[14]%
\>[14]\AgdaSymbol{(}\AgdaPostulate{Subject-Reduction} \<[33]%
\>[33]\<%
\\
\>[14]\AgdaIndent{16}{}\<[16]%
\>[16]\AgdaBound{Γ⊢M∶A} \AgdaBound{M↠φ}\AgdaSymbol{)} \AgdaSymbol{(}\AgdaFunction{context-validity} \AgdaBound{Δ⊢ε∶φ'ρ}\AgdaSymbol{)} \AgdaBound{ρ∶Γ⇒RΔ}\AgdaSymbol{)} \AgdaKeyword{in}\<%
\\
\>[0]\AgdaIndent{6}{}\<[6]%
\>[6]\AgdaKeyword{let} \AgdaBound{step1a} \AgdaSymbol{:} \AgdaFunction{decode-Prop} \AgdaSymbol{(}\AgdaFunction{lrep} \AgdaBound{ρ} \AgdaBound{φ'}\AgdaSymbol{)} \AgdaDatatype{≃} \AgdaFunction{decode-Prop} \AgdaSymbol{(}\AgdaFunction{lrep} \AgdaBound{ρ} \AgdaBound{φ}\AgdaSymbol{)}\<%
\\
\>[6]\AgdaIndent{10}{}\<[10]%
\>[10]\AgdaBound{step1a} \AgdaSymbol{=} \AgdaFunction{subst₂} \AgdaDatatype{\_≃\_} \AgdaSymbol{(}\AgdaFunction{sym} \AgdaSymbol{(}\AgdaFunction{trans} \AgdaSymbol{(}\AgdaPostulate{decode-rep} \AgdaBound{φ'}\AgdaSymbol{)} \AgdaSymbol{(}\AgdaFunction{rep-congl} \AgdaBound{φ'≡Q}\AgdaSymbol{)))} \AgdaSymbol{(}\AgdaFunction{sym} \AgdaSymbol{(}\AgdaPostulate{decode-rep} \AgdaBound{φ}\AgdaSymbol{))} \AgdaSymbol{(}\AgdaPostulate{conv-rep} \AgdaSymbol{\{}\AgdaArgument{M} \AgdaSymbol{=} \AgdaBound{Q}\AgdaSymbol{\}} \AgdaSymbol{\{}\AgdaArgument{N} \AgdaSymbol{=} \AgdaFunction{decode-Prop} \AgdaBound{φ}\AgdaSymbol{\}} \<[136]%
\>[136]\<%
\\
\>[10]\AgdaIndent{12}{}\<[12]%
\>[12]\AgdaSymbol{(}\AgdaInductiveConstructor{sym-conv} \AgdaSymbol{(}\AgdaFunction{red-conv} \AgdaBound{φ↠Q}\AgdaSymbol{)))} \AgdaKeyword{in} \<[42]%
\>[42]\<%
\\
\>[0]\AgdaIndent{6}{}\<[6]%
\>[6]\AgdaKeyword{let} \AgdaBound{step2} \AgdaSymbol{:} \AgdaBound{Δ} \AgdaDatatype{⊢} \AgdaBound{ε} \AgdaDatatype{∶} \AgdaFunction{decode-Prop} \AgdaSymbol{(}\AgdaFunction{lrep} \AgdaBound{ρ} \AgdaBound{φ}\AgdaSymbol{)}\<%
\\
\>[6]\AgdaIndent{10}{}\<[10]%
\>[10]\AgdaBound{step2} \AgdaSymbol{=} \AgdaInductiveConstructor{convR} \AgdaBound{Δ⊢ε∶φ'ρ} \AgdaBound{step1} \AgdaBound{step1a} \AgdaKeyword{in}\<%
\\
\>[0]\AgdaIndent{6}{}\<[6]%
\>[6]\AgdaKeyword{let} \AgdaBound{step3} \AgdaSymbol{:} \AgdaFunction{computeP} \AgdaBound{Δ} \AgdaSymbol{(}\AgdaFunction{lrep} \AgdaBound{ρ} \AgdaBound{φ}\AgdaSymbol{)} \AgdaBound{ε}\<%
\\
\>[6]\AgdaIndent{10}{}\<[10]%
\>[10]\AgdaBound{step3} \AgdaSymbol{=} \AgdaPostulate{conv-computeP} \AgdaSymbol{\{}\AgdaArgument{L} \AgdaSymbol{=} \AgdaFunction{lrep} \AgdaBound{ρ} \AgdaBound{φ'}\AgdaSymbol{\}} \AgdaSymbol{\{}\AgdaArgument{M} \AgdaSymbol{=} \AgdaFunction{lrep} \AgdaBound{ρ} \AgdaBound{φ}\AgdaSymbol{\}} \AgdaBound{computeε} \AgdaBound{step1a} \AgdaBound{step1} \AgdaKeyword{in}\<%
\\
\>[0]\AgdaIndent{6}{}\<[6]%
\>[6]\AgdaKeyword{let} \AgdaBound{step4} \AgdaSymbol{:} \AgdaFunction{computeP} \AgdaBound{Δ} \AgdaSymbol{(}\AgdaFunction{lrep} \AgdaBound{ρ} \AgdaBound{ψ}\AgdaSymbol{)} \AgdaSymbol{(}\AgdaFunction{appP} \AgdaSymbol{(}\AgdaFunction{plus} \AgdaBound{P} \AgdaFunction{〈} \AgdaBound{ρ} \AgdaFunction{〉}\AgdaSymbol{)} \AgdaBound{ε}\AgdaSymbol{)}\<%
\\
\>[6]\AgdaIndent{10}{}\<[10]%
\>[10]\AgdaBound{step4} \AgdaSymbol{=} \AgdaBound{computeP+} \AgdaBound{Δ} \AgdaBound{ρ∶Γ⇒RΔ} \AgdaBound{step2} \AgdaBound{step3} \AgdaKeyword{in} \<[52]%
\>[52]\<%
\\
\>[0]\AgdaIndent{6}{}\<[6]%
\>[6]\AgdaKeyword{let} \AgdaBound{step5} \AgdaSymbol{:} \AgdaFunction{decode-Prop} \AgdaSymbol{(}\AgdaFunction{lrep} \AgdaBound{ρ} \AgdaBound{ψ'}\AgdaSymbol{)} \AgdaDatatype{≃} \AgdaFunction{decode-Prop} \AgdaSymbol{(}\AgdaFunction{lrep} \AgdaBound{ρ} \AgdaBound{ψ}\AgdaSymbol{)}\<%
\\
\>[6]\AgdaIndent{10}{}\<[10]%
\>[10]\AgdaBound{step5} \AgdaSymbol{=} \AgdaFunction{subst₂} \AgdaDatatype{\_≃\_} \AgdaSymbol{(}\AgdaFunction{sym} \AgdaSymbol{(}\AgdaFunction{trans} \AgdaSymbol{(}\AgdaPostulate{decode-rep} \AgdaBound{ψ'}\AgdaSymbol{)} \AgdaSymbol{(}\AgdaFunction{rep-congl} \AgdaBound{ψ'≡R}\AgdaSymbol{)))} \AgdaSymbol{(}\AgdaFunction{sym} \AgdaSymbol{(}\AgdaPostulate{decode-rep} \AgdaBound{ψ}\AgdaSymbol{))} \AgdaSymbol{(}\AgdaPostulate{conv-rep} \AgdaSymbol{\{}\AgdaArgument{M} \AgdaSymbol{=} \AgdaBound{R}\AgdaSymbol{\}} \AgdaSymbol{\{}\AgdaArgument{N} \AgdaSymbol{=} \AgdaFunction{decode-Prop} \AgdaBound{ψ}\AgdaSymbol{\}} \<[135]%
\>[135]\<%
\\
\>[10]\AgdaIndent{12}{}\<[12]%
\>[12]\AgdaSymbol{(}\AgdaInductiveConstructor{sym-conv} \AgdaSymbol{(}\AgdaFunction{red-conv} \AgdaBound{ψ↠R}\AgdaSymbol{)))} \AgdaKeyword{in}\<%
\\
\>[0]\AgdaIndent{6}{}\<[6]%
\>[6]\AgdaKeyword{let} \AgdaBound{step6} \AgdaSymbol{:} \AgdaBound{Δ} \AgdaDatatype{⊢} \AgdaFunction{decode-Prop} \AgdaSymbol{(}\AgdaFunction{lrep} \AgdaBound{ρ} \AgdaBound{ψ'}\AgdaSymbol{)} \AgdaDatatype{∶} \AgdaFunction{ty} \AgdaInductiveConstructor{Ω}\<%
\\
\>[6]\AgdaIndent{10}{}\<[10]%
\>[10]\AgdaBound{step6} \AgdaSymbol{=} \AgdaFunction{subst} \AgdaSymbol{(λ} \AgdaBound{x} \AgdaSymbol{→} \AgdaBound{Δ} \AgdaDatatype{⊢} \AgdaBound{x} \AgdaDatatype{∶} \AgdaFunction{ty} \AgdaInductiveConstructor{Ω}\AgdaSymbol{)} \AgdaSymbol{(}\AgdaFunction{sym} \AgdaSymbol{(}\AgdaPostulate{decode-rep} \AgdaBound{ψ'}\AgdaSymbol{))} \<[67]%
\>[67]\<%
\\
\>[10]\AgdaIndent{16}{}\<[16]%
\>[16]\AgdaSymbol{(}\AgdaPostulate{weakening} \<[27]%
\>[27]\<%
\\
\>[16]\AgdaIndent{18}{}\<[18]%
\>[18]\AgdaSymbol{(}\AgdaFunction{subst} \AgdaSymbol{(λ} \AgdaBound{x} \AgdaSymbol{→} \AgdaBound{Γ} \AgdaDatatype{⊢} \AgdaBound{x} \AgdaDatatype{∶} \AgdaFunction{ty} \AgdaInductiveConstructor{Ω}\AgdaSymbol{)} \AgdaSymbol{(}\AgdaFunction{sym} \AgdaBound{ψ'≡R}\AgdaSymbol{)} \<[57]%
\>[57]\<%
\\
\>[16]\AgdaIndent{18}{}\<[18]%
\>[18]\AgdaSymbol{(}\AgdaPostulate{Subject-Reduction} \AgdaBound{Γ⊢N'∶A} \AgdaBound{N'↠R}\AgdaSymbol{))} \<[51]%
\>[51]\<%
\\
\>[0]\AgdaIndent{16}{}\<[16]%
\>[16]\AgdaSymbol{(}\AgdaFunction{context-validity} \AgdaBound{Δ⊢ε∶φ'ρ}\AgdaSymbol{)} \<[43]%
\>[43]\<%
\\
\>[0]\AgdaIndent{16}{}\<[16]%
\>[16]\AgdaBound{ρ∶Γ⇒RΔ}\AgdaSymbol{)} \AgdaKeyword{in}\<%
\\
\>[0]\AgdaIndent{6}{}\<[6]%
\>[6]\AgdaPostulate{conv-computeP} \AgdaSymbol{\{}\AgdaArgument{L} \AgdaSymbol{=} \AgdaFunction{lrep} \AgdaBound{ρ} \AgdaBound{ψ}\AgdaSymbol{\}} \AgdaSymbol{\{}\AgdaArgument{M} \AgdaSymbol{=} \AgdaFunction{lrep} \AgdaBound{ρ} \AgdaBound{ψ'}\AgdaSymbol{\}} \AgdaBound{step4} \AgdaSymbol{(}\AgdaInductiveConstructor{sym-conv} \AgdaBound{step5}\AgdaSymbol{)} \AgdaBound{step6}\AgdaSymbol{)} \AgdaInductiveConstructor{,p} \<[84]%
\>[84]\<%
\\
\>[0]\AgdaIndent{4}{}\<[4]%
\>[4]\AgdaSymbol{(} \<[9]%
\>[9]\AgdaSymbol{(λ} \AgdaBound{Δ} \AgdaSymbol{\{}\AgdaBound{ρ}\AgdaSymbol{\}} \AgdaSymbol{\{}\AgdaBound{ε}\AgdaSymbol{\}} \AgdaBound{ρ∶Γ⇒RΔ} \AgdaBound{Δ⊢ε∶ψ'ρ} \AgdaBound{computeε} \AgdaSymbol{→} \<[48]%
\>[48]\<%
\\
\>[4]\AgdaIndent{6}{}\<[6]%
\>[6]\AgdaKeyword{let} \AgdaBound{step1} \AgdaSymbol{:} \AgdaBound{Δ} \AgdaDatatype{⊢} \AgdaFunction{decode-Prop} \AgdaSymbol{(}\AgdaFunction{lrep} \AgdaBound{ρ} \AgdaBound{ψ}\AgdaSymbol{)} \AgdaDatatype{∶} \AgdaFunction{ty} \AgdaInductiveConstructor{Ω}\<%
\\
\>[6]\AgdaIndent{10}{}\<[10]%
\>[10]\AgdaBound{step1} \AgdaSymbol{=} \AgdaFunction{subst} \AgdaSymbol{(λ} \AgdaBound{x} \AgdaSymbol{→} \AgdaBound{Δ} \AgdaDatatype{⊢} \AgdaBound{x} \AgdaDatatype{∶} \AgdaFunction{ty} \AgdaInductiveConstructor{Ω}\AgdaSymbol{)} \<[45]%
\>[45]\<%
\\
\>[10]\AgdaIndent{12}{}\<[12]%
\>[12]\AgdaSymbol{(}\AgdaFunction{sym} \AgdaSymbol{(}\AgdaPostulate{decode-rep} \AgdaBound{ψ}\AgdaSymbol{))} \<[33]%
\>[33]\<%
\\
\>[10]\AgdaIndent{12}{}\<[12]%
\>[12]\AgdaSymbol{(}\AgdaPostulate{weakening} \<[23]%
\>[23]\<%
\\
\>[12]\AgdaIndent{14}{}\<[14]%
\>[14]\AgdaSymbol{(}\AgdaPostulate{Subject-Reduction} \<[33]%
\>[33]\<%
\\
\>[14]\AgdaIndent{16}{}\<[16]%
\>[16]\AgdaBound{Γ⊢N∶A} \AgdaBound{N↠ψ}\AgdaSymbol{)} \AgdaSymbol{(}\AgdaFunction{context-validity} \AgdaBound{Δ⊢ε∶ψ'ρ}\AgdaSymbol{)} \AgdaBound{ρ∶Γ⇒RΔ}\AgdaSymbol{)} \AgdaKeyword{in}\<%
\\
\>[0]\AgdaIndent{6}{}\<[6]%
\>[6]\AgdaKeyword{let} \AgdaBound{step1a} \AgdaSymbol{:} \AgdaFunction{decode-Prop} \AgdaSymbol{(}\AgdaFunction{lrep} \AgdaBound{ρ} \AgdaBound{ψ'}\AgdaSymbol{)} \AgdaDatatype{≃} \AgdaFunction{decode-Prop} \AgdaSymbol{(}\AgdaFunction{lrep} \AgdaBound{ρ} \AgdaBound{ψ}\AgdaSymbol{)}\<%
\\
\>[6]\AgdaIndent{10}{}\<[10]%
\>[10]\AgdaBound{step1a} \AgdaSymbol{=} \AgdaFunction{subst₂} \AgdaDatatype{\_≃\_} \AgdaSymbol{(}\AgdaFunction{sym} \AgdaSymbol{(}\AgdaFunction{trans} \AgdaSymbol{(}\AgdaPostulate{decode-rep} \AgdaBound{ψ'}\AgdaSymbol{)} \AgdaSymbol{(}\AgdaFunction{rep-congl} \AgdaBound{ψ'≡R}\AgdaSymbol{)))} \AgdaSymbol{(}\AgdaFunction{sym} \AgdaSymbol{(}\AgdaPostulate{decode-rep} \AgdaBound{ψ}\AgdaSymbol{))} \AgdaSymbol{(}\AgdaPostulate{conv-rep} \AgdaSymbol{\{}\AgdaArgument{M} \AgdaSymbol{=} \AgdaBound{R}\AgdaSymbol{\}} \AgdaSymbol{\{}\AgdaArgument{N} \AgdaSymbol{=} \AgdaFunction{decode-Prop} \AgdaBound{ψ}\AgdaSymbol{\}} \<[136]%
\>[136]\<%
\\
\>[10]\AgdaIndent{12}{}\<[12]%
\>[12]\AgdaSymbol{(}\AgdaInductiveConstructor{sym-conv} \AgdaSymbol{(}\AgdaFunction{red-conv} \AgdaBound{ψ↠R}\AgdaSymbol{)))} \AgdaKeyword{in} \<[42]%
\>[42]\<%
\\
\>[0]\AgdaIndent{6}{}\<[6]%
\>[6]\AgdaKeyword{let} \AgdaBound{step2} \AgdaSymbol{:} \AgdaBound{Δ} \AgdaDatatype{⊢} \AgdaBound{ε} \AgdaDatatype{∶} \AgdaFunction{decode-Prop} \AgdaSymbol{(}\AgdaFunction{lrep} \AgdaBound{ρ} \AgdaBound{ψ}\AgdaSymbol{)}\<%
\\
\>[6]\AgdaIndent{10}{}\<[10]%
\>[10]\AgdaBound{step2} \AgdaSymbol{=} \AgdaInductiveConstructor{convR} \AgdaBound{Δ⊢ε∶ψ'ρ} \AgdaBound{step1} \AgdaBound{step1a} \AgdaKeyword{in}\<%
\\
\>[0]\AgdaIndent{6}{}\<[6]%
\>[6]\AgdaKeyword{let} \AgdaBound{step3} \AgdaSymbol{:} \AgdaFunction{computeP} \AgdaBound{Δ} \AgdaSymbol{(}\AgdaFunction{lrep} \AgdaBound{ρ} \AgdaBound{ψ}\AgdaSymbol{)} \AgdaBound{ε}\<%
\\
\>[6]\AgdaIndent{10}{}\<[10]%
\>[10]\AgdaBound{step3} \AgdaSymbol{=} \AgdaPostulate{conv-computeP} \AgdaSymbol{\{}\AgdaArgument{L} \AgdaSymbol{=} \AgdaFunction{lrep} \AgdaBound{ρ} \AgdaBound{ψ'}\AgdaSymbol{\}} \AgdaSymbol{\{}\AgdaArgument{M} \AgdaSymbol{=} \AgdaFunction{lrep} \AgdaBound{ρ} \AgdaBound{ψ}\AgdaSymbol{\}} \AgdaBound{computeε} \AgdaBound{step1a} \AgdaBound{step1} \AgdaKeyword{in}\<%
\\
\>[0]\AgdaIndent{6}{}\<[6]%
\>[6]\AgdaKeyword{let} \AgdaBound{step4} \AgdaSymbol{:} \AgdaFunction{computeP} \AgdaBound{Δ} \AgdaSymbol{(}\AgdaFunction{lrep} \AgdaBound{ρ} \AgdaBound{φ}\AgdaSymbol{)} \AgdaSymbol{(}\AgdaFunction{appP} \AgdaSymbol{(}\AgdaFunction{minus} \AgdaBound{P} \AgdaFunction{〈} \AgdaBound{ρ} \AgdaFunction{〉}\AgdaSymbol{)} \AgdaBound{ε}\AgdaSymbol{)}\<%
\\
\>[6]\AgdaIndent{10}{}\<[10]%
\>[10]\AgdaBound{step4} \AgdaSymbol{=} \AgdaBound{computeP-} \AgdaBound{Δ} \AgdaBound{ρ∶Γ⇒RΔ} \AgdaBound{step2} \AgdaBound{step3} \AgdaKeyword{in} \<[52]%
\>[52]\<%
\\
\>[0]\AgdaIndent{6}{}\<[6]%
\>[6]\AgdaKeyword{let} \AgdaBound{step5} \AgdaSymbol{:} \AgdaFunction{decode-Prop} \AgdaSymbol{(}\AgdaFunction{lrep} \AgdaBound{ρ} \AgdaBound{φ'}\AgdaSymbol{)} \AgdaDatatype{≃} \AgdaFunction{decode-Prop} \AgdaSymbol{(}\AgdaFunction{lrep} \AgdaBound{ρ} \AgdaBound{φ}\AgdaSymbol{)}\<%
\\
\>[6]\AgdaIndent{10}{}\<[10]%
\>[10]\AgdaBound{step5} \AgdaSymbol{=} \AgdaFunction{subst₂} \AgdaDatatype{\_≃\_} \AgdaSymbol{(}\AgdaFunction{sym} \AgdaSymbol{(}\AgdaFunction{trans} \AgdaSymbol{(}\AgdaPostulate{decode-rep} \AgdaBound{φ'}\AgdaSymbol{)} \AgdaSymbol{(}\AgdaFunction{rep-congl} \AgdaBound{φ'≡Q}\AgdaSymbol{)))} \AgdaSymbol{(}\AgdaFunction{sym} \AgdaSymbol{(}\AgdaPostulate{decode-rep} \AgdaBound{φ}\AgdaSymbol{))} \AgdaSymbol{(}\AgdaPostulate{conv-rep} \AgdaSymbol{\{}\AgdaArgument{M} \AgdaSymbol{=} \AgdaBound{Q}\AgdaSymbol{\}} \AgdaSymbol{\{}\AgdaArgument{N} \AgdaSymbol{=} \AgdaFunction{decode-Prop} \AgdaBound{φ}\AgdaSymbol{\}} \<[135]%
\>[135]\<%
\\
\>[10]\AgdaIndent{12}{}\<[12]%
\>[12]\AgdaSymbol{(}\AgdaInductiveConstructor{sym-conv} \AgdaSymbol{(}\AgdaFunction{red-conv} \AgdaBound{φ↠Q}\AgdaSymbol{)))} \AgdaKeyword{in}\<%
\\
\>[0]\AgdaIndent{6}{}\<[6]%
\>[6]\AgdaKeyword{let} \AgdaBound{step6} \AgdaSymbol{:} \AgdaBound{Δ} \AgdaDatatype{⊢} \AgdaFunction{decode-Prop} \AgdaSymbol{(}\AgdaFunction{lrep} \AgdaBound{ρ} \AgdaBound{φ'}\AgdaSymbol{)} \AgdaDatatype{∶} \AgdaFunction{ty} \AgdaInductiveConstructor{Ω}\<%
\\
\>[6]\AgdaIndent{10}{}\<[10]%
\>[10]\AgdaBound{step6} \AgdaSymbol{=} \AgdaFunction{subst} \AgdaSymbol{(λ} \AgdaBound{x} \AgdaSymbol{→} \AgdaBound{Δ} \AgdaDatatype{⊢} \AgdaBound{x} \AgdaDatatype{∶} \AgdaFunction{ty} \AgdaInductiveConstructor{Ω}\AgdaSymbol{)} \AgdaSymbol{(}\AgdaFunction{sym} \AgdaSymbol{(}\AgdaPostulate{decode-rep} \AgdaBound{φ'}\AgdaSymbol{))} \<[67]%
\>[67]\<%
\\
\>[10]\AgdaIndent{16}{}\<[16]%
\>[16]\AgdaSymbol{(}\AgdaPostulate{weakening} \<[27]%
\>[27]\<%
\\
\>[16]\AgdaIndent{18}{}\<[18]%
\>[18]\AgdaSymbol{(}\AgdaFunction{subst} \AgdaSymbol{(λ} \AgdaBound{x} \AgdaSymbol{→} \AgdaBound{Γ} \AgdaDatatype{⊢} \AgdaBound{x} \AgdaDatatype{∶} \AgdaFunction{ty} \AgdaInductiveConstructor{Ω}\AgdaSymbol{)} \AgdaSymbol{(}\AgdaFunction{sym} \AgdaBound{φ'≡Q}\AgdaSymbol{)} \<[57]%
\>[57]\<%
\\
\>[16]\AgdaIndent{18}{}\<[18]%
\>[18]\AgdaSymbol{(}\AgdaPostulate{Subject-Reduction} \AgdaBound{Γ⊢M'∶A} \AgdaBound{M'↠Q}\AgdaSymbol{))} \<[51]%
\>[51]\<%
\\
\>[0]\AgdaIndent{16}{}\<[16]%
\>[16]\AgdaSymbol{(}\AgdaFunction{context-validity} \AgdaBound{Δ⊢ε∶ψ'ρ}\AgdaSymbol{)} \<[43]%
\>[43]\<%
\\
\>[0]\AgdaIndent{16}{}\<[16]%
\>[16]\AgdaBound{ρ∶Γ⇒RΔ}\AgdaSymbol{)} \AgdaKeyword{in}\<%
\\
\>[0]\AgdaIndent{6}{}\<[6]%
\>[6]\AgdaPostulate{conv-computeP} \AgdaSymbol{\{}\AgdaArgument{L} \AgdaSymbol{=} \AgdaFunction{lrep} \AgdaBound{ρ} \AgdaBound{φ}\AgdaSymbol{\}} \AgdaSymbol{\{}\AgdaArgument{M} \AgdaSymbol{=} \AgdaFunction{lrep} \AgdaBound{ρ} \AgdaBound{φ'}\AgdaSymbol{\}} \AgdaBound{step4} \AgdaSymbol{(}\AgdaInductiveConstructor{sym-conv} \AgdaBound{step5}\AgdaSymbol{)} \AgdaBound{step6}\AgdaSymbol{))}\<%
\\
\>\AgdaFunction{conv-computeE} \AgdaSymbol{\{}\AgdaArgument{A} \AgdaSymbol{=} \AgdaBound{A} \AgdaInductiveConstructor{⇛} \AgdaBound{B}\AgdaSymbol{\}} \AgdaBound{computeP} \AgdaBound{Γ⊢M∶A} \AgdaBound{Γ⊢N∶A} \AgdaBound{Γ⊢M'∶A} \AgdaBound{Γ⊢N'∶A} \AgdaBound{M≃M'} \AgdaBound{N≃N'} \AgdaBound{Δ} \AgdaBound{ρ∶Γ⇒RΔ} \AgdaBound{Δ⊢Q∶N≡N'} \AgdaBound{computeQ} \AgdaSymbol{=} \<[100]%
\>[100]\<%
\\
\>[0]\AgdaIndent{2}{}\<[2]%
\>[2]\AgdaFunction{conv-computeE} \AgdaSymbol{\{}\AgdaArgument{A} \AgdaSymbol{=} \AgdaBound{B}\AgdaSymbol{\}} \<[24]%
\>[24]\<%
\\
\>[0]\AgdaIndent{2}{}\<[2]%
\>[2]\AgdaSymbol{(}\AgdaBound{computeP} \AgdaBound{Δ} \AgdaBound{ρ∶Γ⇒RΔ} \AgdaBound{Δ⊢Q∶N≡N'} \AgdaBound{computeQ}\AgdaSymbol{)} \<[40]%
\>[40]\<%
\\
\>[2]\AgdaIndent{4}{}\<[4]%
\>[4]\AgdaSymbol{(}\AgdaInductiveConstructor{appR} \AgdaSymbol{(}\AgdaPostulate{weakening} \AgdaBound{Γ⊢M∶A} \AgdaSymbol{(}\AgdaFunction{context-validity} \AgdaBound{Δ⊢Q∶N≡N'}\AgdaSymbol{)} \AgdaBound{ρ∶Γ⇒RΔ}\AgdaSymbol{)} \<[63]%
\>[63]\<%
\\
\>[4]\AgdaIndent{6}{}\<[6]%
\>[6]\AgdaSymbol{(}\AgdaPostulate{Equation-Validity₁} \AgdaBound{Δ⊢Q∶N≡N'}\AgdaSymbol{))} \<[37]%
\>[37]\<%
\\
\>[0]\AgdaIndent{4}{}\<[4]%
\>[4]\AgdaSymbol{(}\AgdaInductiveConstructor{appR} \AgdaSymbol{(}\AgdaPostulate{weakening} \AgdaBound{Γ⊢N∶A} \AgdaSymbol{(}\AgdaFunction{context-validity} \AgdaBound{Δ⊢Q∶N≡N'}\AgdaSymbol{)} \AgdaBound{ρ∶Γ⇒RΔ}\AgdaSymbol{)} \<[63]%
\>[63]\<%
\\
\>[4]\AgdaIndent{6}{}\<[6]%
\>[6]\AgdaSymbol{(}\AgdaPostulate{Equation-Validity₂} \AgdaBound{Δ⊢Q∶N≡N'}\AgdaSymbol{))}\<%
\\
\>[0]\AgdaIndent{4}{}\<[4]%
\>[4]\AgdaSymbol{(}\AgdaInductiveConstructor{appR} \AgdaSymbol{(}\AgdaPostulate{weakening} \AgdaBound{Γ⊢M'∶A} \AgdaSymbol{(}\AgdaFunction{context-validity} \AgdaBound{Δ⊢Q∶N≡N'}\AgdaSymbol{)} \AgdaBound{ρ∶Γ⇒RΔ}\AgdaSymbol{)} \AgdaSymbol{(}\AgdaPostulate{Equation-Validity₁} \AgdaBound{Δ⊢Q∶N≡N'}\AgdaSymbol{))} \<[95]%
\>[95]\<%
\\
\>[0]\AgdaIndent{4}{}\<[4]%
\>[4]\AgdaSymbol{(}\AgdaInductiveConstructor{appR} \AgdaSymbol{(}\AgdaPostulate{weakening} \AgdaBound{Γ⊢N'∶A} \AgdaSymbol{(}\AgdaFunction{context-validity} \AgdaBound{Δ⊢Q∶N≡N'}\AgdaSymbol{)} \AgdaBound{ρ∶Γ⇒RΔ}\AgdaSymbol{)} \AgdaSymbol{(}\AgdaPostulate{Equation-Validity₂} \AgdaBound{Δ⊢Q∶N≡N'}\AgdaSymbol{))} \<[95]%
\>[95]\<%
\\
\>[0]\AgdaIndent{4}{}\<[4]%
\>[4]\AgdaSymbol{(}\AgdaPostulate{appT-convl} \AgdaSymbol{(}\AgdaPostulate{conv-rep} \AgdaBound{M≃M'}\AgdaSymbol{))} \AgdaSymbol{(}\AgdaPostulate{appT-convl} \AgdaSymbol{(}\AgdaPostulate{conv-rep} \AgdaBound{N≃N'}\AgdaSymbol{))}\<%
\\
\>\AgdaComment{--TODO Common pattern}\<%
\\
%
\\
\>\AgdaKeyword{postulate} \AgdaPostulate{expand-computeE} \AgdaSymbol{:} \AgdaSymbol{∀} \AgdaSymbol{\{}\AgdaBound{V}\AgdaSymbol{\}} \AgdaSymbol{\{}\AgdaBound{Γ} \AgdaSymbol{:} \AgdaDatatype{Context} \AgdaBound{V}\AgdaSymbol{\}} \AgdaSymbol{\{}\AgdaBound{M}\AgdaSymbol{\}} \AgdaSymbol{\{}\AgdaBound{A}\AgdaSymbol{\}} \AgdaSymbol{\{}\AgdaBound{N}\AgdaSymbol{\}} \AgdaSymbol{\{}\AgdaBound{P}\AgdaSymbol{\}} \AgdaSymbol{\{}\AgdaBound{Q}\AgdaSymbol{\}} \AgdaSymbol{→}\<%
\\
\>[4]\AgdaIndent{26}{}\<[26]%
\>[26]\AgdaFunction{computeE} \AgdaBound{Γ} \AgdaBound{M} \AgdaBound{A} \AgdaBound{N} \AgdaBound{Q} \AgdaSymbol{→} \AgdaBound{Γ} \AgdaDatatype{⊢} \AgdaBound{P} \AgdaDatatype{∶} \AgdaBound{M} \AgdaFunction{≡〈} \AgdaBound{A} \AgdaFunction{〉} \AgdaBound{N} \AgdaSymbol{→} \AgdaDatatype{key-redex} \AgdaBound{P} \AgdaBound{Q} \AgdaSymbol{→} \AgdaFunction{computeE} \AgdaBound{Γ} \AgdaBound{M} \AgdaBound{A} \AgdaBound{N} \AgdaBound{P}\<%
\\
%
\\
\>\AgdaFunction{expand-computeT} \AgdaSymbol{:} \AgdaSymbol{∀} \AgdaSymbol{\{}\AgdaBound{V}\AgdaSymbol{\}} \AgdaSymbol{\{}\AgdaBound{Γ} \AgdaSymbol{:} \AgdaDatatype{Context} \AgdaBound{V}\AgdaSymbol{\}} \AgdaSymbol{\{}\AgdaBound{A}\AgdaSymbol{\}} \AgdaSymbol{\{}\AgdaBound{M}\AgdaSymbol{\}} \AgdaSymbol{\{}\AgdaBound{N}\AgdaSymbol{\}} \AgdaSymbol{→} \AgdaFunction{computeT} \AgdaBound{Γ} \AgdaBound{A} \AgdaBound{N} \AgdaSymbol{→} \AgdaBound{Γ} \AgdaDatatype{⊢} \AgdaBound{M} \AgdaDatatype{∶} \AgdaFunction{ty} \AgdaBound{A} \AgdaSymbol{→} \AgdaDatatype{key-redex} \AgdaBound{M} \AgdaBound{N} \AgdaSymbol{→} \AgdaFunction{computeT} \AgdaBound{Γ} \AgdaBound{A} \AgdaBound{M}\<%
\\
\>\AgdaFunction{expand-computeT} \AgdaSymbol{\{}\AgdaArgument{A} \AgdaSymbol{=} \AgdaInductiveConstructor{Ω}\AgdaSymbol{\}} \AgdaBound{computeψ} \AgdaSymbol{\_} \AgdaBound{φ▷ψ} \AgdaSymbol{=} \AgdaFunction{key-redex-SN} \AgdaBound{computeψ} \AgdaBound{φ▷ψ}\<%
\\
\>\AgdaFunction{expand-computeT} \AgdaSymbol{\{}\AgdaArgument{A} \AgdaSymbol{=} \AgdaBound{A} \AgdaInductiveConstructor{⇛} \AgdaBound{B}\AgdaSymbol{\}} \AgdaSymbol{\{}\AgdaBound{M}\AgdaSymbol{\}} \AgdaSymbol{\{}\AgdaBound{M'}\AgdaSymbol{\}} \AgdaSymbol{(}\AgdaBound{computeM'app} \AgdaInductiveConstructor{,p} \AgdaBound{computeM'eq}\AgdaSymbol{)} \AgdaBound{Γ⊢M∶A⇛B} \AgdaBound{M▷M'} \AgdaSymbol{=} \<[82]%
\>[82]\<%
\\
\>[0]\AgdaIndent{2}{}\<[2]%
\>[2]\AgdaSymbol{(λ} \AgdaBound{Δ} \AgdaSymbol{\{}\AgdaBound{ρ}\AgdaSymbol{\}} \AgdaSymbol{\{}\AgdaBound{N}\AgdaSymbol{\}} \AgdaBound{ρ∶Γ⇒Δ} \AgdaBound{Δ⊢N∶A} \AgdaBound{computeN} \AgdaSymbol{→} \<[38]%
\>[38]\<%
\\
\>[2]\AgdaIndent{4}{}\<[4]%
\>[4]\AgdaKeyword{let} \AgdaBound{computeM'N} \AgdaSymbol{:} \AgdaFunction{computeT} \AgdaBound{Δ} \AgdaBound{B} \AgdaSymbol{(}\AgdaFunction{appT} \AgdaSymbol{(}\AgdaBound{M'} \AgdaFunction{〈} \AgdaBound{ρ} \AgdaFunction{〉}\AgdaSymbol{)} \AgdaBound{N}\AgdaSymbol{)}\<%
\\
\>[4]\AgdaIndent{8}{}\<[8]%
\>[8]\AgdaBound{computeM'N} \AgdaSymbol{=} \AgdaBound{computeM'app} \AgdaBound{Δ} \AgdaBound{ρ∶Γ⇒Δ} \AgdaBound{Δ⊢N∶A} \AgdaBound{computeN}\<%
\\
\>[0]\AgdaIndent{4}{}\<[4]%
\>[4]\AgdaKeyword{in} \AgdaFunction{expand-computeT} \AgdaBound{computeM'N} \<[34]%
\>[34]\<%
\\
\>[4]\AgdaIndent{7}{}\<[7]%
\>[7]\AgdaSymbol{(}\AgdaInductiveConstructor{appR} \AgdaSymbol{(}\AgdaPostulate{weakening} \AgdaBound{Γ⊢M∶A⇛B} \AgdaSymbol{(}\AgdaFunction{context-validity} \AgdaBound{Δ⊢N∶A}\AgdaSymbol{)} \AgdaBound{ρ∶Γ⇒Δ}\AgdaSymbol{)} \AgdaBound{Δ⊢N∶A}\AgdaSymbol{)} \<[71]%
\>[71]\<%
\\
\>[7]\AgdaIndent{13}{}\<[13]%
\>[13]\AgdaSymbol{(}\AgdaInductiveConstructor{appTkr} \AgdaSymbol{(}\AgdaFunction{key-redex-rep} \AgdaBound{M▷M'}\AgdaSymbol{)))} \AgdaInductiveConstructor{,p} \<[47]%
\>[47]\<%
\\
\>[0]\AgdaIndent{2}{}\<[2]%
\>[2]\AgdaSymbol{(λ} \AgdaBound{Δ} \AgdaBound{ρ∶Γ⇒Δ} \AgdaBound{Δ⊢P∶N≡N'} \AgdaBound{computeN} \AgdaBound{computeN'} \AgdaBound{computeP₁} \AgdaSymbol{→} \<[53]%
\>[53]\<%
\\
\>[2]\AgdaIndent{4}{}\<[4]%
\>[4]\AgdaPostulate{expand-computeE} \<[20]%
\>[20]\<%
\\
\>[4]\AgdaIndent{6}{}\<[6]%
\>[6]\AgdaSymbol{(}\AgdaFunction{conv-computeE} \<[21]%
\>[21]\<%
\\
\>[6]\AgdaIndent{8}{}\<[8]%
\>[8]\AgdaSymbol{(}\AgdaBound{computeM'eq} \AgdaBound{Δ} \AgdaBound{ρ∶Γ⇒Δ} \AgdaBound{Δ⊢P∶N≡N'} \AgdaBound{computeN} \AgdaBound{computeN'} \AgdaBound{computeP₁}\AgdaSymbol{)} \<[68]%
\>[68]\<%
\\
\>[6]\AgdaIndent{8}{}\<[8]%
\>[8]\AgdaSymbol{(}\AgdaInductiveConstructor{appR} \AgdaSymbol{(}\AgdaPostulate{weakening} \AgdaSymbol{(}\AgdaPostulate{Subject-Reduction} \AgdaBound{Γ⊢M∶A⇛B} \AgdaSymbol{(}\AgdaPostulate{key-redex-red} \AgdaBound{M▷M'}\AgdaSymbol{))} \<[74]%
\>[74]\<%
\\
\>[8]\AgdaIndent{25}{}\<[25]%
\>[25]\AgdaSymbol{(}\AgdaFunction{context-validity} \AgdaBound{Δ⊢P∶N≡N'}\AgdaSymbol{)} \AgdaBound{ρ∶Γ⇒Δ}\AgdaSymbol{)} \<[60]%
\>[60]\<%
\\
\>[0]\AgdaIndent{14}{}\<[14]%
\>[14]\AgdaSymbol{(}\AgdaPostulate{Equation-Validity₁} \AgdaBound{Δ⊢P∶N≡N'}\AgdaSymbol{))} \<[45]%
\>[45]\<%
\\
\>[0]\AgdaIndent{8}{}\<[8]%
\>[8]\AgdaSymbol{(}\AgdaInductiveConstructor{appR} \AgdaSymbol{(}\AgdaPostulate{weakening} \AgdaSymbol{(}\AgdaPostulate{Subject-Reduction} \AgdaBound{Γ⊢M∶A⇛B} \AgdaSymbol{(}\AgdaPostulate{key-redex-red} \AgdaBound{M▷M'}\AgdaSymbol{))} \<[74]%
\>[74]\<%
\\
\>[8]\AgdaIndent{25}{}\<[25]%
\>[25]\AgdaSymbol{(}\AgdaFunction{context-validity} \AgdaBound{Δ⊢P∶N≡N'}\AgdaSymbol{)} \AgdaBound{ρ∶Γ⇒Δ}\AgdaSymbol{)} \<[60]%
\>[60]\<%
\\
\>[0]\AgdaIndent{14}{}\<[14]%
\>[14]\AgdaSymbol{(}\AgdaPostulate{Equation-Validity₂} \AgdaBound{Δ⊢P∶N≡N'}\AgdaSymbol{))} \<[45]%
\>[45]\<%
\\
\>[0]\AgdaIndent{8}{}\<[8]%
\>[8]\AgdaSymbol{(}\AgdaInductiveConstructor{appR} \AgdaSymbol{(}\AgdaPostulate{weakening} \AgdaBound{Γ⊢M∶A⇛B} \AgdaSymbol{(}\AgdaFunction{context-validity} \AgdaBound{Δ⊢P∶N≡N'}\AgdaSymbol{)} \AgdaBound{ρ∶Γ⇒Δ}\AgdaSymbol{)} \<[68]%
\>[68]\<%
\\
\>[8]\AgdaIndent{14}{}\<[14]%
\>[14]\AgdaSymbol{(}\AgdaPostulate{Equation-Validity₁} \AgdaBound{Δ⊢P∶N≡N'}\AgdaSymbol{))} \<[45]%
\>[45]\<%
\\
\>[0]\AgdaIndent{8}{}\<[8]%
\>[8]\AgdaSymbol{(}\AgdaInductiveConstructor{appR} \AgdaSymbol{(}\AgdaPostulate{weakening} \AgdaBound{Γ⊢M∶A⇛B} \AgdaSymbol{(}\AgdaFunction{context-validity} \AgdaBound{Δ⊢P∶N≡N'}\AgdaSymbol{)} \AgdaBound{ρ∶Γ⇒Δ}\AgdaSymbol{)} \<[68]%
\>[68]\<%
\\
\>[8]\AgdaIndent{14}{}\<[14]%
\>[14]\AgdaSymbol{(}\AgdaPostulate{Equation-Validity₂} \AgdaBound{Δ⊢P∶N≡N'}\AgdaSymbol{))} \<[45]%
\>[45]\<%
\\
\>[0]\AgdaIndent{8}{}\<[8]%
\>[8]\AgdaSymbol{(}\AgdaInductiveConstructor{sym-conv} \AgdaSymbol{(}\AgdaPostulate{appT-convl} \AgdaSymbol{(}\AgdaFunction{red-conv} \AgdaSymbol{(}\AgdaPostulate{red-rep} \AgdaSymbol{(}\AgdaPostulate{key-redex-red} \AgdaBound{M▷M'}\AgdaSymbol{)))))} \<[74]%
\>[74]\<%
\\
\>[0]\AgdaIndent{8}{}\<[8]%
\>[8]\AgdaSymbol{(}\AgdaInductiveConstructor{sym-conv} \AgdaSymbol{(}\AgdaPostulate{appT-convl} \AgdaSymbol{(}\AgdaFunction{red-conv} \AgdaSymbol{(}\AgdaPostulate{red-rep} \AgdaSymbol{(}\AgdaPostulate{key-redex-red} \AgdaBound{M▷M'}\AgdaSymbol{))))))} \<[75]%
\>[75]\<%
\\
\>[0]\AgdaIndent{6}{}\<[6]%
\>[6]\AgdaSymbol{(}\AgdaPostulate{⋆-typed} \AgdaSymbol{(}\AgdaPostulate{weakening} \AgdaBound{Γ⊢M∶A⇛B} \AgdaSymbol{(}\AgdaFunction{context-validity} \AgdaBound{Δ⊢P∶N≡N'}\AgdaSymbol{)} \AgdaBound{ρ∶Γ⇒Δ}\AgdaSymbol{)} \<[69]%
\>[69]\<%
\\
\>[6]\AgdaIndent{8}{}\<[8]%
\>[8]\AgdaBound{Δ⊢P∶N≡N'}\AgdaSymbol{)} \<[18]%
\>[18]\<%
\\
\>[0]\AgdaIndent{6}{}\<[6]%
\>[6]\AgdaSymbol{(}\AgdaPostulate{key-redex-⋆} \AgdaSymbol{(}\AgdaFunction{key-redex-rep} \AgdaBound{M▷M'}\AgdaSymbol{)))}\<%
\\
%
\\
\>\AgdaFunction{compute} \AgdaSymbol{:} \AgdaSymbol{∀} \AgdaSymbol{\{}\AgdaBound{V}\AgdaSymbol{\}} \AgdaSymbol{\{}\AgdaBound{K}\AgdaSymbol{\}} \AgdaSymbol{→} \AgdaDatatype{Context} \AgdaBound{V} \AgdaSymbol{→} \AgdaFunction{Expression} \AgdaBound{V} \AgdaSymbol{(}\AgdaFunction{parent} \AgdaBound{K}\AgdaSymbol{)} \AgdaSymbol{→} \AgdaFunction{Expression} \AgdaBound{V} \AgdaSymbol{(}\AgdaInductiveConstructor{varKind} \AgdaBound{K}\AgdaSymbol{)} \AgdaSymbol{→} \AgdaPrimitiveType{Set}\<%
\\
\>\AgdaFunction{compute} \AgdaSymbol{\{}\AgdaArgument{K} \AgdaSymbol{=} \AgdaInductiveConstructor{-Term}\AgdaSymbol{\}} \AgdaBound{Γ} \AgdaSymbol{(}\AgdaInductiveConstructor{app} \AgdaSymbol{(}\AgdaInductiveConstructor{-ty} \AgdaBound{A}\AgdaSymbol{)} \AgdaInductiveConstructor{out}\AgdaSymbol{)} \AgdaBound{M} \AgdaSymbol{=} \AgdaFunction{computeT} \AgdaBound{Γ} \AgdaBound{A} \AgdaBound{M}\<%
\\
\>\AgdaFunction{compute} \AgdaSymbol{\{}\AgdaBound{V}\AgdaSymbol{\}} \AgdaSymbol{\{}\AgdaArgument{K} \AgdaSymbol{=} \AgdaInductiveConstructor{-Proof}\AgdaSymbol{\}} \AgdaBound{Γ} \AgdaBound{φ} \AgdaBound{δ} \AgdaSymbol{=} \AgdaFunction{Σ[} \AgdaBound{S} \AgdaFunction{∈} \AgdaDatatype{Shape} \AgdaFunction{]} \AgdaFunction{Σ[} \AgdaBound{L} \AgdaFunction{∈} \AgdaDatatype{Leaves} \AgdaBound{V} \AgdaBound{S} \AgdaFunction{]} \AgdaBound{φ} \AgdaDatatype{↠} \AgdaFunction{decode-Prop} \AgdaBound{L} \AgdaFunction{×} \AgdaFunction{computeP} \AgdaBound{Γ} \AgdaBound{L} \AgdaBound{δ}\<%
\\
\>\AgdaFunction{compute} \AgdaSymbol{\{}\AgdaArgument{K} \AgdaSymbol{=} \AgdaInductiveConstructor{-Path}\AgdaSymbol{\}} \AgdaBound{Γ} \AgdaSymbol{(}\AgdaInductiveConstructor{app} \AgdaSymbol{(}\AgdaInductiveConstructor{-eq} \AgdaBound{A}\AgdaSymbol{)} \AgdaSymbol{(}\AgdaBound{M} \AgdaInductiveConstructor{,,} \AgdaBound{N} \AgdaInductiveConstructor{,,} \AgdaInductiveConstructor{out}\AgdaSymbol{))} \AgdaBound{P} \AgdaSymbol{=} \AgdaFunction{computeE} \AgdaBound{Γ} \AgdaBound{M} \AgdaBound{A} \AgdaBound{N} \AgdaBound{P}\<%
\\
%
\\
\>\AgdaKeyword{postulate} \AgdaPostulate{expand-computeP} \AgdaSymbol{:} \AgdaSymbol{∀} \AgdaSymbol{\{}\AgdaBound{V}\AgdaSymbol{\}} \AgdaSymbol{\{}\AgdaBound{Γ} \AgdaSymbol{:} \AgdaDatatype{Context} \AgdaBound{V}\AgdaSymbol{\}} \AgdaSymbol{\{}\AgdaBound{S}\AgdaSymbol{\}} \AgdaSymbol{\{}\AgdaBound{L} \AgdaSymbol{:} \AgdaDatatype{Leaves} \AgdaBound{V} \AgdaBound{S}\AgdaSymbol{\}} \AgdaSymbol{\{}\AgdaBound{δ} \AgdaBound{ε}\AgdaSymbol{\}} \AgdaSymbol{→}\<%
\\
\>[6]\AgdaIndent{26}{}\<[26]%
\>[26]\AgdaFunction{computeP} \AgdaBound{Γ} \AgdaBound{L} \AgdaBound{ε} \AgdaSymbol{→} \AgdaBound{Γ} \AgdaDatatype{⊢} \AgdaBound{δ} \AgdaDatatype{∶} \AgdaFunction{decode-Prop} \AgdaBound{L} \AgdaSymbol{→} \AgdaDatatype{key-redex} \AgdaBound{δ} \AgdaBound{ε} \AgdaSymbol{→} \AgdaFunction{computeP} \AgdaBound{Γ} \AgdaBound{L} \AgdaBound{δ}\<%
\end{code}
}

\begin{lm}
\label{lm:expand-compute}
Suppose $P[x:=N, y:=N', e:=Q] \in E_\Gamma(M =_A M')$.  Suppose also $\Gamma \vdash (\triplelambda e:x=_By.P)_{N N'} Q : M =_A M'$,
and $N$, $N'$ and $Q$ are all strongly normalizing.  Then $(\triplelambda e:x=_By.P)_{N N'} Q \in E_\Gamma(M =_A M')$.
\end{lm}

\begin{code}%
\>\AgdaFunction{expand-compute} \AgdaSymbol{:} \AgdaSymbol{∀} \AgdaSymbol{\{}\AgdaBound{V}\AgdaSymbol{\}} \AgdaSymbol{\{}\AgdaBound{K}\AgdaSymbol{\}} \AgdaSymbol{\{}\AgdaBound{Γ} \AgdaSymbol{:} \AgdaDatatype{Context} \AgdaBound{V}\AgdaSymbol{\}} \AgdaSymbol{\{}\AgdaBound{A} \AgdaSymbol{:} \AgdaFunction{Expression} \AgdaBound{V} \AgdaSymbol{(}\AgdaFunction{parent} \AgdaBound{K}\AgdaSymbol{)\}} \AgdaSymbol{\{}\AgdaBound{M} \AgdaBound{N} \AgdaSymbol{:} \AgdaFunction{Expression} \AgdaBound{V} \AgdaSymbol{(}\AgdaInductiveConstructor{varKind} \AgdaBound{K}\AgdaSymbol{)\}} \AgdaSymbol{→}\<%
\\
\>[0]\AgdaIndent{2}{}\<[2]%
\>[2]\AgdaFunction{compute} \AgdaBound{Γ} \AgdaBound{A} \AgdaBound{N} \AgdaSymbol{→} \AgdaBound{Γ} \AgdaDatatype{⊢} \AgdaBound{M} \AgdaDatatype{∶} \AgdaBound{A} \AgdaSymbol{→} \AgdaDatatype{key-redex} \AgdaBound{M} \AgdaBound{N} \AgdaSymbol{→} \AgdaFunction{compute} \AgdaBound{Γ} \AgdaBound{A} \AgdaBound{M}\<%
\end{code}

\AgdaHide{
\begin{code}%
\>\AgdaFunction{expand-compute} \AgdaSymbol{\{}\AgdaArgument{K} \AgdaSymbol{=} \AgdaInductiveConstructor{-Term}\AgdaSymbol{\}} \AgdaSymbol{\{}\AgdaArgument{A} \AgdaSymbol{=} \AgdaInductiveConstructor{app} \AgdaSymbol{(}\AgdaInductiveConstructor{-ty} \AgdaBound{A}\AgdaSymbol{)} \AgdaInductiveConstructor{out}\AgdaSymbol{\}} \AgdaSymbol{=} \AgdaFunction{expand-computeT} \AgdaSymbol{\{}\AgdaArgument{A} \AgdaSymbol{=} \AgdaBound{A}\AgdaSymbol{\}}\<%
\\
\>\AgdaFunction{expand-compute} \AgdaSymbol{\{}\AgdaArgument{K} \AgdaSymbol{=} \AgdaInductiveConstructor{-Proof}\AgdaSymbol{\}} \AgdaSymbol{(}\AgdaBound{S} \AgdaInductiveConstructor{,p} \AgdaBound{ψ} \AgdaInductiveConstructor{,p} \AgdaBound{φ↠ψ} \AgdaInductiveConstructor{,p} \AgdaBound{computeε}\AgdaSymbol{)} \AgdaBound{Γ⊢δ∶φ} \AgdaBound{δ▷ε} \AgdaSymbol{=} \AgdaSymbol{(}\AgdaBound{S} \AgdaInductiveConstructor{,p} \AgdaBound{ψ} \AgdaInductiveConstructor{,p} \AgdaBound{φ↠ψ} \AgdaInductiveConstructor{,p} \AgdaPostulate{expand-computeP} \AgdaSymbol{\{}\AgdaArgument{S} \AgdaSymbol{=} \AgdaBound{S}\AgdaSymbol{\}} \AgdaBound{computeε} \AgdaSymbol{(}\AgdaPostulate{Type-Reduction} \AgdaBound{Γ⊢δ∶φ} \AgdaBound{φ↠ψ}\AgdaSymbol{)} \AgdaBound{δ▷ε}\AgdaSymbol{)}\<%
\\
\>\AgdaFunction{expand-compute} \AgdaSymbol{\{}\AgdaArgument{K} \AgdaSymbol{=} \AgdaInductiveConstructor{-Path}\AgdaSymbol{\}} \AgdaSymbol{\{}\AgdaArgument{A} \AgdaSymbol{=} \AgdaInductiveConstructor{app} \AgdaSymbol{(}\AgdaInductiveConstructor{-eq} \AgdaBound{A}\AgdaSymbol{)} \AgdaSymbol{(}\AgdaBound{M} \AgdaInductiveConstructor{,,} \AgdaBound{N} \AgdaInductiveConstructor{,,} \AgdaInductiveConstructor{out}\AgdaSymbol{)\}} \AgdaBound{computeQ} \AgdaBound{Γ⊢P∶M≡N} \AgdaBound{P▷Q} \AgdaSymbol{=} \AgdaPostulate{expand-computeE} \AgdaBound{computeQ} \AgdaBound{Γ⊢P∶M≡N} \AgdaBound{P▷Q}\<%
\\
%
\\
\>\AgdaKeyword{record} \AgdaRecord{E'} \AgdaSymbol{\{}\AgdaBound{V}\AgdaSymbol{\}} \AgdaSymbol{\{}\AgdaBound{K}\AgdaSymbol{\}} \AgdaSymbol{(}\AgdaBound{Γ} \AgdaSymbol{:} \AgdaDatatype{Context} \AgdaBound{V}\AgdaSymbol{)} \AgdaSymbol{(}\AgdaBound{A} \AgdaSymbol{:} \AgdaFunction{Expression} \AgdaBound{V} \AgdaSymbol{(}\AgdaFunction{parent} \AgdaBound{K}\AgdaSymbol{))} \AgdaSymbol{(}\AgdaBound{E} \AgdaSymbol{:} \AgdaFunction{Expression} \AgdaBound{V} \AgdaSymbol{(}\AgdaInductiveConstructor{varKind} \AgdaBound{K}\AgdaSymbol{))} \AgdaSymbol{:} \AgdaPrimitiveType{Set} \AgdaKeyword{where}\<%
\\
\>[0]\AgdaIndent{2}{}\<[2]%
\>[2]\AgdaKeyword{constructor} \AgdaInductiveConstructor{E'I}\<%
\\
\>[0]\AgdaIndent{2}{}\<[2]%
\>[2]\AgdaKeyword{field}\<%
\\
\>[2]\AgdaIndent{4}{}\<[4]%
\>[4]\AgdaField{typed} \AgdaSymbol{:} \AgdaBound{Γ} \AgdaDatatype{⊢} \AgdaBound{E} \AgdaDatatype{∶} \AgdaBound{A}\<%
\\
\>[2]\AgdaIndent{4}{}\<[4]%
\>[4]\AgdaField{computable} \AgdaSymbol{:} \AgdaFunction{compute} \AgdaBound{Γ} \AgdaBound{A} \AgdaBound{E}\<%
\\
%
\\
\>\AgdaComment{--TODO Inline the following?}\<%
\\
\>\AgdaFunction{E} \AgdaSymbol{:} \AgdaSymbol{∀} \AgdaSymbol{\{}\AgdaBound{V}\AgdaSymbol{\}} \AgdaSymbol{→} \AgdaDatatype{Context} \AgdaBound{V} \AgdaSymbol{→} \AgdaDatatype{Type} \AgdaSymbol{→} \AgdaFunction{Term} \AgdaBound{V} \AgdaSymbol{→} \AgdaPrimitiveType{Set}\<%
\\
\>\AgdaFunction{E} \AgdaBound{Γ} \AgdaBound{A} \AgdaBound{M} \AgdaSymbol{=} \AgdaRecord{E'} \AgdaBound{Γ} \AgdaSymbol{(}\AgdaFunction{ty} \AgdaBound{A}\AgdaSymbol{)} \AgdaBound{M}\<%
\\
%
\\
\>\AgdaFunction{EP} \AgdaSymbol{:} \AgdaSymbol{∀} \AgdaSymbol{\{}\AgdaBound{V}\AgdaSymbol{\}} \AgdaSymbol{→} \AgdaDatatype{Context} \AgdaBound{V} \AgdaSymbol{→} \AgdaFunction{Term} \AgdaBound{V} \AgdaSymbol{→} \AgdaFunction{Proof} \AgdaBound{V} \AgdaSymbol{→} \AgdaPrimitiveType{Set}\<%
\\
\>\AgdaFunction{EP} \AgdaBound{Γ} \AgdaBound{φ} \AgdaBound{δ} \AgdaSymbol{=} \AgdaRecord{E'} \AgdaBound{Γ} \AgdaBound{φ} \AgdaBound{δ}\<%
\\
%
\\
\>\AgdaFunction{EE} \AgdaSymbol{:} \AgdaSymbol{∀} \AgdaSymbol{\{}\AgdaBound{V}\AgdaSymbol{\}} \AgdaSymbol{→} \AgdaDatatype{Context} \AgdaBound{V} \AgdaSymbol{→} \AgdaFunction{Equation} \AgdaBound{V} \AgdaSymbol{→} \AgdaFunction{Path} \AgdaBound{V} \AgdaSymbol{→} \AgdaPrimitiveType{Set}\<%
\\
\>\AgdaFunction{EE} \AgdaBound{Γ} \AgdaBound{E} \AgdaBound{P} \AgdaSymbol{=} \AgdaRecord{E'} \AgdaBound{Γ} \AgdaBound{E} \AgdaBound{P}\<%
\\
%
\\
\>\AgdaFunction{E'-typed} \AgdaSymbol{:} \AgdaSymbol{∀} \AgdaSymbol{\{}\AgdaBound{V}\AgdaSymbol{\}} \AgdaSymbol{\{}\AgdaBound{K}\AgdaSymbol{\}} \AgdaSymbol{\{}\AgdaBound{Γ} \AgdaSymbol{:} \AgdaDatatype{Context} \AgdaBound{V}\AgdaSymbol{\}} \AgdaSymbol{\{}\AgdaBound{A}\AgdaSymbol{\}} \AgdaSymbol{\{}\AgdaBound{M} \AgdaSymbol{:} \AgdaFunction{Expression} \AgdaBound{V} \AgdaSymbol{(}\AgdaInductiveConstructor{varKind} \AgdaBound{K}\AgdaSymbol{)\}} \AgdaSymbol{→} \<[74]%
\>[74]\<%
\\
\>[4]\AgdaIndent{11}{}\<[11]%
\>[11]\AgdaRecord{E'} \AgdaBound{Γ} \AgdaBound{A} \AgdaBound{M} \AgdaSymbol{→} \AgdaBound{Γ} \AgdaDatatype{⊢} \AgdaBound{M} \AgdaDatatype{∶} \AgdaBound{A}\<%
\\
\>\AgdaFunction{E'-typed} \AgdaSymbol{=} \AgdaField{E'.typed}\<%
\\
%
\\
\>\AgdaFunction{expand-E'} \AgdaSymbol{:} \AgdaSymbol{∀} \AgdaSymbol{\{}\AgdaBound{V}\AgdaSymbol{\}} \AgdaSymbol{\{}\AgdaBound{K}\AgdaSymbol{\}} \AgdaSymbol{\{}\AgdaBound{Γ}\AgdaSymbol{\}} \AgdaSymbol{\{}\AgdaBound{A}\AgdaSymbol{\}} \AgdaSymbol{\{}\AgdaBound{M} \AgdaBound{N} \AgdaSymbol{:} \AgdaFunction{Expression} \AgdaBound{V} \AgdaSymbol{(}\AgdaInductiveConstructor{varKind} \AgdaBound{K}\AgdaSymbol{)\}} \AgdaSymbol{→}\<%
\\
\>[11]\AgdaIndent{12}{}\<[12]%
\>[12]\AgdaRecord{E'} \AgdaBound{Γ} \AgdaBound{A} \AgdaBound{N} \AgdaSymbol{→} \AgdaBound{Γ} \AgdaDatatype{⊢} \AgdaBound{M} \AgdaDatatype{∶} \AgdaBound{A} \AgdaSymbol{→} \AgdaDatatype{key-redex} \AgdaBound{M} \AgdaBound{N} \AgdaSymbol{→} \AgdaRecord{E'} \AgdaBound{Γ} \AgdaBound{A} \AgdaBound{M}\<%
\\
\>\AgdaFunction{expand-E'} \AgdaBound{N∈EΓA} \AgdaBound{Γ⊢M∶A} \AgdaBound{M▷N} \AgdaSymbol{=} \AgdaInductiveConstructor{E'I} \AgdaBound{Γ⊢M∶A} \AgdaSymbol{(}\AgdaFunction{expand-compute} \AgdaSymbol{(}\AgdaField{E'.computable} \AgdaBound{N∈EΓA}\AgdaSymbol{)} \AgdaBound{Γ⊢M∶A} \AgdaBound{M▷N}\AgdaSymbol{)}\<%
\\
%
\\
\>\AgdaKeyword{postulate} \AgdaPostulate{expand-E} \AgdaSymbol{:} \AgdaSymbol{∀} \AgdaSymbol{\{}\AgdaBound{V}\AgdaSymbol{\}} \AgdaSymbol{\{}\AgdaBound{Γ} \AgdaSymbol{:} \AgdaDatatype{Context} \AgdaBound{V}\AgdaSymbol{\}} \AgdaSymbol{\{}\AgdaBound{A} \AgdaSymbol{:} \AgdaDatatype{Type}\AgdaSymbol{\}} \AgdaSymbol{\{}\AgdaBound{M} \AgdaBound{N} \AgdaSymbol{:} \AgdaFunction{Term} \AgdaBound{V}\AgdaSymbol{\}} \AgdaSymbol{→}\<%
\\
\>[12]\AgdaIndent{19}{}\<[19]%
\>[19]\AgdaFunction{E} \AgdaBound{Γ} \AgdaBound{A} \AgdaBound{N} \AgdaSymbol{→} \AgdaBound{Γ} \AgdaDatatype{⊢} \AgdaBound{M} \AgdaDatatype{∶} \AgdaFunction{ty} \AgdaBound{A} \AgdaSymbol{→} \AgdaDatatype{key-redex} \AgdaBound{M} \AgdaBound{N} \AgdaSymbol{→} \AgdaFunction{E} \AgdaBound{Γ} \AgdaBound{A} \AgdaBound{M}\<%
\\
%
\\
\>\AgdaKeyword{postulate} \AgdaPostulate{func-E} \AgdaSymbol{:} \AgdaSymbol{∀} \AgdaSymbol{\{}\AgdaBound{U}\AgdaSymbol{\}} \AgdaSymbol{\{}\AgdaBound{Γ} \AgdaSymbol{:} \AgdaDatatype{Context} \AgdaBound{U}\AgdaSymbol{\}} \AgdaSymbol{\{}\AgdaBound{M} \AgdaSymbol{:} \AgdaFunction{Term} \AgdaBound{U}\AgdaSymbol{\}} \AgdaSymbol{\{}\AgdaBound{A}\AgdaSymbol{\}} \AgdaSymbol{\{}\AgdaBound{B}\AgdaSymbol{\}} \AgdaSymbol{→}\<%
\\
\>[12]\AgdaIndent{19}{}\<[19]%
\>[19]\AgdaSymbol{(∀} \AgdaBound{V} \AgdaBound{Δ} \AgdaSymbol{(}\AgdaBound{ρ} \AgdaSymbol{:} \AgdaFunction{Rep} \AgdaBound{U} \AgdaBound{V}\AgdaSymbol{)} \AgdaSymbol{(}\AgdaBound{N} \AgdaSymbol{:} \AgdaFunction{Term} \AgdaBound{V}\AgdaSymbol{)} \AgdaSymbol{→} \AgdaDatatype{valid} \AgdaBound{Δ} \AgdaSymbol{→} \AgdaBound{ρ} \AgdaPostulate{∶} \AgdaBound{Γ} \AgdaPostulate{⇒R} \AgdaBound{Δ} \AgdaSymbol{→} \AgdaFunction{E} \AgdaBound{Δ} \AgdaBound{A} \AgdaBound{N} \AgdaSymbol{→} \AgdaFunction{E} \AgdaBound{Δ} \AgdaBound{B} \AgdaSymbol{(}\AgdaFunction{appT} \AgdaSymbol{(}\AgdaBound{M} \AgdaFunction{〈} \AgdaBound{ρ} \AgdaFunction{〉}\AgdaSymbol{)} \AgdaBound{N}\AgdaSymbol{))} \AgdaSymbol{→}\<%
\\
\>[12]\AgdaIndent{19}{}\<[19]%
\>[19]\AgdaFunction{E} \AgdaBound{Γ} \AgdaSymbol{(}\AgdaBound{A} \AgdaInductiveConstructor{⇛} \AgdaBound{B}\AgdaSymbol{)} \AgdaBound{M}\<%
\end{code}
}

\begin{lm}$ $
\label{lm:conv-compute}
\begin{enumerate}
\item
If $\delta \in E_\Gamma(\phi)$, $\Gamma \vdash \psi : \Omega$ and $\phi \simeq \psi$, then $\delta \in E_\Gamma(\psi)$.
\item
If $P \in E_\Gamma(M =_A N)$, $\Gamma \vdash M' : A$, $\Gamma \vdash N' : A$, $M \simeq M'$ and $N \simeq N'$,
then $P \in E_\Gamma(M' =_A N')$.
\end{enumerate}
\end{lm}

\begin{code}%
\>\AgdaKeyword{postulate} \AgdaPostulate{conv-E'} \AgdaSymbol{:} \AgdaSymbol{∀} \AgdaSymbol{\{}\AgdaBound{V}\AgdaSymbol{\}} \AgdaSymbol{\{}\AgdaBound{K}\AgdaSymbol{\}} \AgdaSymbol{\{}\AgdaBound{Γ}\AgdaSymbol{\}} \AgdaSymbol{\{}\AgdaBound{A}\AgdaSymbol{\}} \AgdaSymbol{\{}\AgdaBound{B}\AgdaSymbol{\}} \AgdaSymbol{\{}\AgdaBound{M} \AgdaSymbol{:} \AgdaFunction{Expression} \AgdaBound{V} \AgdaSymbol{(}\AgdaInductiveConstructor{varKind} \AgdaBound{K}\AgdaSymbol{)\}} \AgdaSymbol{→}\<%
\\
\>[0]\AgdaIndent{18}{}\<[18]%
\>[18]\AgdaBound{A} \AgdaDatatype{≃} \AgdaBound{B} \AgdaSymbol{→} \AgdaRecord{E'} \AgdaBound{Γ} \AgdaBound{A} \AgdaBound{M} \AgdaSymbol{→} \AgdaDatatype{valid} \AgdaSymbol{(}\AgdaInductiveConstructor{\_,\_} \AgdaSymbol{\{}\AgdaArgument{K} \AgdaSymbol{=} \AgdaBound{K}\AgdaSymbol{\}} \AgdaBound{Γ} \AgdaBound{B}\AgdaSymbol{)} \AgdaSymbol{→} \AgdaRecord{E'} \AgdaBound{Γ} \AgdaBound{B} \AgdaBound{M}\<%
\end{code}

\AgdaHide{
\begin{code}%
\>\AgdaKeyword{postulate} \AgdaPostulate{E'-SN} \AgdaSymbol{:} \AgdaSymbol{∀} \AgdaSymbol{\{}\AgdaBound{V}\AgdaSymbol{\}} \AgdaSymbol{\{}\AgdaBound{K}\AgdaSymbol{\}} \AgdaSymbol{\{}\AgdaBound{Γ}\AgdaSymbol{\}} \AgdaSymbol{\{}\AgdaBound{A}\AgdaSymbol{\}} \AgdaSymbol{\{}\AgdaBound{M} \AgdaSymbol{:} \AgdaFunction{Expression} \AgdaBound{V} \AgdaSymbol{(}\AgdaInductiveConstructor{varKind} \AgdaBound{K}\AgdaSymbol{)\}} \AgdaSymbol{→} \AgdaRecord{E'} \AgdaBound{Γ} \AgdaBound{A} \AgdaBound{M} \AgdaSymbol{→} \AgdaDatatype{SN} \AgdaBound{M}\<%
\end{code}
}

\begin{lm}
\label{lm:var-compute}
Variables are computable.  That is, if $x : A \in \Gamma$ and $\Gamma \vald$, then $x \in E_\Gamma(A)$; and similarly
for proof variables and path variables.
\end{lm}

\begin{code}%
\>\AgdaKeyword{postulate} \AgdaPostulate{var-E'} \AgdaSymbol{:} \AgdaSymbol{∀} \AgdaSymbol{\{}\AgdaBound{V}\AgdaSymbol{\}} \AgdaSymbol{\{}\AgdaBound{K}\AgdaSymbol{\}} \AgdaSymbol{\{}\AgdaBound{x} \AgdaSymbol{:} \AgdaDatatype{Var} \AgdaBound{V} \AgdaBound{K}\AgdaSymbol{\}} \AgdaSymbol{\{}\AgdaBound{Γ} \AgdaSymbol{:} \AgdaDatatype{Context} \AgdaBound{V}\AgdaSymbol{\}} \AgdaSymbol{→} \AgdaRecord{E'} \AgdaBound{Γ} \AgdaSymbol{(}\AgdaFunction{typeof} \AgdaBound{x} \AgdaBound{Γ}\AgdaSymbol{)} \AgdaSymbol{(}\AgdaInductiveConstructor{var} \AgdaBound{x}\AgdaSymbol{)}\<%
\end{code}

\AgdaHide{
\begin{code}%
\>\AgdaKeyword{postulate} \AgdaPostulate{⊥-E} \AgdaSymbol{:} \AgdaSymbol{∀} \AgdaSymbol{\{}\AgdaBound{V}\AgdaSymbol{\}} \AgdaSymbol{\{}\AgdaBound{Γ} \AgdaSymbol{:} \AgdaDatatype{Context} \AgdaBound{V}\AgdaSymbol{\}} \AgdaSymbol{→} \AgdaDatatype{valid} \AgdaBound{Γ} \AgdaSymbol{→} \AgdaRecord{E'} \AgdaBound{Γ} \AgdaSymbol{(}\AgdaFunction{ty} \AgdaInductiveConstructor{Ω}\AgdaSymbol{)} \AgdaFunction{⊥}\<%
\\
%
\\
\>\AgdaKeyword{postulate} \AgdaPostulate{⊃-E} \AgdaSymbol{:} \AgdaSymbol{∀} \AgdaSymbol{\{}\AgdaBound{V}\AgdaSymbol{\}} \AgdaSymbol{\{}\AgdaBound{Γ} \AgdaSymbol{:} \AgdaDatatype{Context} \AgdaBound{V}\AgdaSymbol{\}} \AgdaSymbol{\{}\AgdaBound{φ}\AgdaSymbol{\}} \AgdaSymbol{\{}\AgdaBound{ψ}\AgdaSymbol{\}} \AgdaSymbol{→} \AgdaFunction{E} \AgdaBound{Γ} \AgdaInductiveConstructor{Ω} \AgdaBound{φ} \AgdaSymbol{→} \AgdaFunction{E} \AgdaBound{Γ} \AgdaInductiveConstructor{Ω} \AgdaBound{ψ} \AgdaSymbol{→} \AgdaFunction{E} \AgdaBound{Γ} \AgdaInductiveConstructor{Ω} \AgdaSymbol{(}\AgdaBound{φ} \AgdaFunction{⊃} \AgdaBound{ψ}\AgdaSymbol{)}\<%
\\
%
\\
\>\AgdaKeyword{postulate} \AgdaPostulate{appT-E} \AgdaSymbol{:} \AgdaSymbol{∀} \AgdaSymbol{\{}\AgdaBound{V}\AgdaSymbol{\}} \AgdaSymbol{\{}\AgdaBound{Γ} \AgdaSymbol{:} \AgdaDatatype{Context} \AgdaBound{V}\AgdaSymbol{\}} \AgdaSymbol{\{}\AgdaBound{M} \AgdaBound{N} \AgdaSymbol{:} \AgdaFunction{Term} \AgdaBound{V}\AgdaSymbol{\}} \AgdaSymbol{\{}\AgdaBound{A}\AgdaSymbol{\}} \AgdaSymbol{\{}\AgdaBound{B}\AgdaSymbol{\}} \AgdaSymbol{→}\<%
\\
\>[0]\AgdaIndent{17}{}\<[17]%
\>[17]\AgdaDatatype{valid} \AgdaBound{Γ} \AgdaSymbol{→} \AgdaFunction{E} \AgdaBound{Γ} \AgdaSymbol{(}\AgdaBound{A} \AgdaInductiveConstructor{⇛} \AgdaBound{B}\AgdaSymbol{)} \AgdaBound{M} \AgdaSymbol{→} \AgdaFunction{E} \AgdaBound{Γ} \AgdaBound{A} \AgdaBound{N} \AgdaSymbol{→} \AgdaFunction{E} \AgdaBound{Γ} \AgdaBound{B} \AgdaSymbol{(}\AgdaFunction{appT} \AgdaBound{M} \AgdaBound{N}\AgdaSymbol{)}\<%
\\
%
\\
\>\AgdaFunction{E-typed} \AgdaSymbol{:} \AgdaSymbol{∀} \AgdaSymbol{\{}\AgdaBound{V}\AgdaSymbol{\}} \AgdaSymbol{\{}\AgdaBound{Γ} \AgdaSymbol{:} \AgdaDatatype{Context} \AgdaBound{V}\AgdaSymbol{\}} \AgdaSymbol{\{}\AgdaBound{A}\AgdaSymbol{\}} \AgdaSymbol{\{}\AgdaBound{M}\AgdaSymbol{\}} \AgdaSymbol{→} \AgdaFunction{E} \AgdaBound{Γ} \AgdaBound{A} \AgdaBound{M} \AgdaSymbol{→} \AgdaBound{Γ} \AgdaDatatype{⊢} \AgdaBound{M} \AgdaDatatype{∶} \AgdaFunction{ty} \AgdaBound{A}\<%
\\
\>\AgdaFunction{E-typed} \AgdaSymbol{=} \AgdaFunction{E'-typed}\<%
\\
%
\\
\>\AgdaFunction{E-SN} \AgdaSymbol{:} \AgdaSymbol{∀} \AgdaSymbol{\{}\AgdaBound{V}\AgdaSymbol{\}} \AgdaSymbol{\{}\AgdaBound{Γ} \AgdaSymbol{:} \AgdaDatatype{Context} \AgdaBound{V}\AgdaSymbol{\}} \AgdaBound{A} \AgdaSymbol{\{}\AgdaBound{M}\AgdaSymbol{\}} \AgdaSymbol{→} \AgdaFunction{E} \AgdaBound{Γ} \AgdaBound{A} \AgdaBound{M} \AgdaSymbol{→} \AgdaDatatype{SN} \AgdaBound{M}\<%
\\
\>\AgdaFunction{E-SN} \AgdaSymbol{\_} \AgdaSymbol{=} \AgdaPostulate{E'-SN}\<%
\\
\>\AgdaComment{--TODO Inline}\<%
\\
%
\\
\>\AgdaKeyword{postulate} \AgdaPostulate{appP-EP} \AgdaSymbol{:} \AgdaSymbol{∀} \AgdaSymbol{\{}\AgdaBound{V}\AgdaSymbol{\}} \AgdaSymbol{\{}\AgdaBound{Γ} \AgdaSymbol{:} \AgdaDatatype{Context} \AgdaBound{V}\AgdaSymbol{\}} \AgdaSymbol{\{}\AgdaBound{δ} \AgdaBound{ε} \AgdaSymbol{:} \AgdaFunction{Proof} \AgdaBound{V}\AgdaSymbol{\}} \AgdaSymbol{\{}\AgdaBound{φ}\AgdaSymbol{\}} \AgdaSymbol{\{}\AgdaBound{ψ}\AgdaSymbol{\}} \AgdaSymbol{→}\<%
\\
\>[17]\AgdaIndent{18}{}\<[18]%
\>[18]\AgdaFunction{EP} \AgdaBound{Γ} \AgdaSymbol{(}\AgdaBound{φ} \AgdaFunction{⊃} \AgdaBound{ψ}\AgdaSymbol{)} \AgdaBound{δ} \AgdaSymbol{→} \AgdaFunction{EP} \AgdaBound{Γ} \AgdaBound{φ} \AgdaBound{ε} \AgdaSymbol{→} \AgdaFunction{EP} \AgdaBound{Γ} \AgdaBound{ψ} \AgdaSymbol{(}\AgdaFunction{appP} \AgdaBound{δ} \AgdaBound{ε}\AgdaSymbol{)}\<%
\\
\>\AgdaKeyword{postulate} \AgdaPostulate{plus-EP} \AgdaSymbol{:} \AgdaSymbol{∀} \AgdaSymbol{\{}\AgdaBound{V}\AgdaSymbol{\}} \AgdaSymbol{\{}\AgdaBound{Γ} \AgdaSymbol{:} \AgdaDatatype{Context} \AgdaBound{V}\AgdaSymbol{\}} \AgdaSymbol{\{}\AgdaBound{P} \AgdaSymbol{:} \AgdaFunction{Path} \AgdaBound{V}\AgdaSymbol{\}} \AgdaSymbol{\{}\AgdaBound{φ} \AgdaBound{ψ} \AgdaSymbol{:} \AgdaFunction{Term} \AgdaBound{V}\AgdaSymbol{\}} \AgdaSymbol{→}\<%
\\
\>[17]\AgdaIndent{18}{}\<[18]%
\>[18]\AgdaFunction{EE} \AgdaBound{Γ} \AgdaSymbol{(}\AgdaBound{φ} \AgdaFunction{≡〈} \AgdaInductiveConstructor{Ω} \AgdaFunction{〉} \AgdaBound{ψ}\AgdaSymbol{)} \AgdaBound{P} \AgdaSymbol{→} \AgdaFunction{EP} \AgdaBound{Γ} \AgdaSymbol{(}\AgdaBound{φ} \AgdaFunction{⊃} \AgdaBound{ψ}\AgdaSymbol{)} \AgdaSymbol{(}\AgdaFunction{plus} \AgdaBound{P}\AgdaSymbol{)}\<%
\\
\>\AgdaKeyword{postulate} \AgdaPostulate{minus-EP} \AgdaSymbol{:} \AgdaSymbol{∀} \AgdaSymbol{\{}\AgdaBound{V}\AgdaSymbol{\}} \AgdaSymbol{\{}\AgdaBound{Γ} \AgdaSymbol{:} \AgdaDatatype{Context} \AgdaBound{V}\AgdaSymbol{\}} \AgdaSymbol{\{}\AgdaBound{P} \AgdaSymbol{:} \AgdaFunction{Path} \AgdaBound{V}\AgdaSymbol{\}} \AgdaSymbol{\{}\AgdaBound{φ} \AgdaBound{ψ} \AgdaSymbol{:} \AgdaFunction{Term} \AgdaBound{V}\AgdaSymbol{\}} \AgdaSymbol{→}\<%
\\
\>[18]\AgdaIndent{19}{}\<[19]%
\>[19]\AgdaFunction{EE} \AgdaBound{Γ} \AgdaSymbol{(}\AgdaBound{φ} \AgdaFunction{≡〈} \AgdaInductiveConstructor{Ω} \AgdaFunction{〉} \AgdaBound{ψ}\AgdaSymbol{)} \AgdaBound{P} \AgdaSymbol{→} \AgdaFunction{EP} \AgdaBound{Γ} \AgdaSymbol{(}\AgdaBound{ψ} \AgdaFunction{⊃} \AgdaBound{φ}\AgdaSymbol{)} \AgdaSymbol{(}\AgdaFunction{minus} \AgdaBound{P}\AgdaSymbol{)}\<%
\\
%
\\
\>\AgdaFunction{expand-EP} \AgdaSymbol{:} \AgdaSymbol{∀} \AgdaSymbol{\{}\AgdaBound{V}\AgdaSymbol{\}} \AgdaSymbol{\{}\AgdaBound{Γ} \AgdaSymbol{:} \AgdaDatatype{Context} \AgdaBound{V}\AgdaSymbol{\}} \AgdaSymbol{\{}\AgdaBound{φ} \AgdaSymbol{:} \AgdaFunction{Term} \AgdaBound{V}\AgdaSymbol{\}} \AgdaSymbol{\{}\AgdaBound{δ} \AgdaBound{ε} \AgdaSymbol{:} \AgdaFunction{Proof} \AgdaBound{V}\AgdaSymbol{\}} \AgdaSymbol{→}\<%
\\
\>[0]\AgdaIndent{12}{}\<[12]%
\>[12]\AgdaFunction{EP} \AgdaBound{Γ} \AgdaBound{φ} \AgdaBound{ε} \AgdaSymbol{→} \AgdaBound{Γ} \AgdaDatatype{⊢} \AgdaBound{δ} \AgdaDatatype{∶} \AgdaBound{φ} \AgdaSymbol{→} \AgdaDatatype{key-redex} \AgdaBound{δ} \AgdaBound{ε} \AgdaSymbol{→} \AgdaFunction{EP} \AgdaBound{Γ} \AgdaBound{φ} \AgdaBound{δ}\<%
\\
\>\AgdaFunction{expand-EP} \AgdaSymbol{=} \AgdaFunction{expand-E'}\<%
\\
%
\\
\>\AgdaKeyword{postulate} \AgdaPostulate{func-EP} \AgdaSymbol{:} \AgdaSymbol{∀} \AgdaSymbol{\{}\AgdaBound{U}\AgdaSymbol{\}} \AgdaSymbol{\{}\AgdaBound{Γ} \AgdaSymbol{:} \AgdaDatatype{Context} \AgdaBound{U}\AgdaSymbol{\}} \AgdaSymbol{\{}\AgdaBound{δ} \AgdaSymbol{:} \AgdaFunction{Proof} \AgdaBound{U}\AgdaSymbol{\}} \AgdaSymbol{\{}\AgdaBound{φ}\AgdaSymbol{\}} \AgdaSymbol{\{}\AgdaBound{ψ}\AgdaSymbol{\}} \AgdaSymbol{→}\<%
\\
\>[12]\AgdaIndent{18}{}\<[18]%
\>[18]\AgdaSymbol{(∀} \AgdaBound{V} \AgdaBound{Δ} \AgdaSymbol{(}\AgdaBound{ρ} \AgdaSymbol{:} \AgdaFunction{Rep} \AgdaBound{U} \AgdaBound{V}\AgdaSymbol{)} \AgdaSymbol{(}\AgdaBound{ε} \AgdaSymbol{:} \AgdaFunction{Proof} \AgdaBound{V}\AgdaSymbol{)} \AgdaSymbol{→} \AgdaBound{ρ} \AgdaPostulate{∶} \AgdaBound{Γ} \AgdaPostulate{⇒R} \AgdaBound{Δ} \AgdaSymbol{→} \AgdaFunction{EP} \AgdaBound{Δ} \AgdaSymbol{(}\AgdaBound{φ} \AgdaFunction{〈} \AgdaBound{ρ} \AgdaFunction{〉}\AgdaSymbol{)} \AgdaBound{ε} \AgdaSymbol{→} \AgdaFunction{EP} \AgdaBound{Δ} \AgdaSymbol{(}\AgdaBound{ψ} \AgdaFunction{〈} \AgdaBound{ρ} \AgdaFunction{〉}\AgdaSymbol{)} \AgdaSymbol{(}\AgdaFunction{appP} \AgdaSymbol{(}\AgdaBound{δ} \AgdaFunction{〈} \AgdaBound{ρ} \AgdaFunction{〉}\AgdaSymbol{)} \AgdaBound{ε}\AgdaSymbol{))} \AgdaSymbol{→} \<[124]%
\>[124]\<%
\\
\>[12]\AgdaIndent{18}{}\<[18]%
\>[18]\AgdaBound{Γ} \AgdaDatatype{⊢} \AgdaBound{δ} \AgdaDatatype{∶} \AgdaBound{φ} \AgdaFunction{⊃} \AgdaBound{ψ} \AgdaSymbol{→} \AgdaFunction{EP} \AgdaBound{Γ} \AgdaSymbol{(}\AgdaBound{φ} \AgdaFunction{⊃} \AgdaBound{ψ}\AgdaSymbol{)} \AgdaBound{δ}\<%
\\
%
\\
\>\AgdaFunction{conv-EP} \AgdaSymbol{:} \AgdaSymbol{∀} \AgdaSymbol{\{}\AgdaBound{V}\AgdaSymbol{\}} \AgdaSymbol{\{}\AgdaBound{Γ} \AgdaSymbol{:} \AgdaDatatype{Context} \AgdaBound{V}\AgdaSymbol{\}} \AgdaSymbol{\{}\AgdaBound{φ} \AgdaBound{ψ} \AgdaSymbol{:} \AgdaFunction{Term} \AgdaBound{V}\AgdaSymbol{\}} \AgdaSymbol{\{}\AgdaBound{δ} \AgdaSymbol{:} \AgdaFunction{Proof} \AgdaBound{V}\AgdaSymbol{\}} \AgdaSymbol{→}\<%
\\
\>[0]\AgdaIndent{10}{}\<[10]%
\>[10]\AgdaBound{φ} \AgdaDatatype{≃} \AgdaBound{ψ} \AgdaSymbol{→} \AgdaFunction{EP} \AgdaBound{Γ} \AgdaBound{φ} \AgdaBound{δ} \AgdaSymbol{→} \AgdaBound{Γ} \AgdaDatatype{⊢} \AgdaBound{ψ} \AgdaDatatype{∶} \AgdaFunction{ty} \AgdaInductiveConstructor{Ω} \AgdaSymbol{→} \AgdaFunction{EP} \AgdaBound{Γ} \AgdaBound{ψ} \AgdaBound{δ}\<%
\\
\>\AgdaFunction{conv-EP} \AgdaBound{φ≃ψ} \AgdaBound{δ∈EΓφ} \AgdaBound{Γ⊢ψ∶Ω} \AgdaSymbol{=} \AgdaPostulate{conv-E'} \AgdaBound{φ≃ψ} \AgdaBound{δ∈EΓφ} \AgdaSymbol{(}\AgdaInductiveConstructor{ctxPR} \AgdaBound{Γ⊢ψ∶Ω}\AgdaSymbol{)}\<%
\\
%
\\
\>\AgdaFunction{EP-typed} \AgdaSymbol{:} \AgdaSymbol{∀} \AgdaSymbol{\{}\AgdaBound{V}\AgdaSymbol{\}} \AgdaSymbol{\{}\AgdaBound{Γ} \AgdaSymbol{:} \AgdaDatatype{Context} \AgdaBound{V}\AgdaSymbol{\}} \AgdaSymbol{\{}\AgdaBound{δ} \AgdaSymbol{:} \AgdaFunction{Proof} \AgdaBound{V}\AgdaSymbol{\}} \AgdaSymbol{\{}\AgdaBound{φ} \AgdaSymbol{:} \AgdaFunction{Term} \AgdaBound{V}\AgdaSymbol{\}} \AgdaSymbol{→}\<%
\\
\>[0]\AgdaIndent{9}{}\<[9]%
\>[9]\AgdaFunction{EP} \AgdaBound{Γ} \AgdaBound{φ} \AgdaBound{δ} \AgdaSymbol{→} \AgdaBound{Γ} \AgdaDatatype{⊢} \AgdaBound{δ} \AgdaDatatype{∶} \AgdaBound{φ}\<%
\\
\>\AgdaFunction{EP-typed} \AgdaSymbol{=} \AgdaFunction{E'-typed}\<%
\\
%
\\
\>\AgdaFunction{EP-SN} \AgdaSymbol{:} \AgdaSymbol{∀} \AgdaSymbol{\{}\AgdaBound{V}\AgdaSymbol{\}} \AgdaSymbol{\{}\AgdaBound{Γ} \AgdaSymbol{:} \AgdaDatatype{Context} \AgdaBound{V}\AgdaSymbol{\}} \AgdaSymbol{\{}\AgdaBound{δ}\AgdaSymbol{\}} \AgdaSymbol{\{}\AgdaBound{φ}\AgdaSymbol{\}} \AgdaSymbol{→} \AgdaFunction{EP} \AgdaBound{Γ} \AgdaBound{φ} \AgdaBound{δ} \AgdaSymbol{→} \AgdaDatatype{SN} \AgdaBound{δ}\<%
\\
\>\AgdaFunction{EP-SN} \AgdaSymbol{=} \AgdaPostulate{E'-SN}\<%
\\
%
\\
\>\AgdaKeyword{postulate} \AgdaPostulate{ref-EE} \AgdaSymbol{:} \AgdaSymbol{∀} \AgdaSymbol{\{}\AgdaBound{V}\AgdaSymbol{\}} \AgdaSymbol{\{}\AgdaBound{Γ} \AgdaSymbol{:} \AgdaDatatype{Context} \AgdaBound{V}\AgdaSymbol{\}} \AgdaSymbol{\{}\AgdaBound{M} \AgdaSymbol{:} \AgdaFunction{Term} \AgdaBound{V}\AgdaSymbol{\}} \AgdaSymbol{\{}\AgdaBound{A} \AgdaSymbol{:} \AgdaDatatype{Type}\AgdaSymbol{\}} \AgdaSymbol{→} \AgdaFunction{E} \AgdaBound{Γ} \AgdaBound{A} \AgdaBound{M} \AgdaSymbol{→} \AgdaFunction{EE} \AgdaBound{Γ} \AgdaSymbol{(}\AgdaBound{M} \AgdaFunction{≡〈} \AgdaBound{A} \AgdaFunction{〉} \AgdaBound{M}\AgdaSymbol{)} \AgdaSymbol{(}\AgdaFunction{reff} \AgdaBound{M}\AgdaSymbol{)}\<%
\\
\>\AgdaKeyword{postulate} \AgdaPostulate{⊃*-EE} \AgdaSymbol{:} \AgdaSymbol{∀} \AgdaSymbol{\{}\AgdaBound{V}\AgdaSymbol{\}} \AgdaSymbol{\{}\AgdaBound{Γ} \AgdaSymbol{:} \AgdaDatatype{Context} \AgdaBound{V}\AgdaSymbol{\}} \AgdaSymbol{\{}\AgdaBound{φ} \AgdaBound{φ'} \AgdaBound{ψ} \AgdaBound{ψ'} \AgdaSymbol{:} \AgdaFunction{Term} \AgdaBound{V}\AgdaSymbol{\}} \AgdaSymbol{\{}\AgdaBound{P} \AgdaBound{Q} \AgdaSymbol{:} \AgdaFunction{Path} \AgdaBound{V}\AgdaSymbol{\}} \AgdaSymbol{→}\<%
\\
\>[9]\AgdaIndent{18}{}\<[18]%
\>[18]\AgdaFunction{EE} \AgdaBound{Γ} \AgdaSymbol{(}\AgdaBound{φ} \AgdaFunction{≡〈} \AgdaInductiveConstructor{Ω} \AgdaFunction{〉} \AgdaBound{φ'}\AgdaSymbol{)} \AgdaBound{P} \AgdaSymbol{→} \AgdaFunction{EE} \AgdaBound{Γ} \AgdaSymbol{(}\AgdaBound{ψ} \AgdaFunction{≡〈} \AgdaInductiveConstructor{Ω} \AgdaFunction{〉} \AgdaBound{ψ'}\AgdaSymbol{)} \AgdaBound{Q} \AgdaSymbol{→} \AgdaFunction{EE} \AgdaBound{Γ} \AgdaSymbol{(}\AgdaBound{φ} \AgdaFunction{⊃} \AgdaBound{ψ} \AgdaFunction{≡〈} \AgdaInductiveConstructor{Ω} \AgdaFunction{〉} \AgdaBound{φ'} \AgdaFunction{⊃} \AgdaBound{ψ'}\AgdaSymbol{)} \AgdaSymbol{(}\AgdaBound{P} \AgdaFunction{⊃*} \AgdaBound{Q}\AgdaSymbol{)}\<%
\\
\>\AgdaKeyword{postulate} \AgdaPostulate{univ-EE} \AgdaSymbol{:} \AgdaSymbol{∀} \AgdaSymbol{\{}\AgdaBound{V}\AgdaSymbol{\}} \AgdaSymbol{\{}\AgdaBound{Γ} \AgdaSymbol{:} \AgdaDatatype{Context} \AgdaBound{V}\AgdaSymbol{\}} \AgdaSymbol{\{}\AgdaBound{φ} \AgdaBound{ψ} \AgdaSymbol{:} \AgdaFunction{Term} \AgdaBound{V}\AgdaSymbol{\}} \AgdaSymbol{\{}\AgdaBound{δ} \AgdaBound{ε} \AgdaSymbol{:} \AgdaFunction{Proof} \AgdaBound{V}\AgdaSymbol{\}} \AgdaSymbol{→}\<%
\\
\>[9]\AgdaIndent{18}{}\<[18]%
\>[18]\AgdaFunction{EP} \AgdaBound{Γ} \AgdaSymbol{(}\AgdaBound{φ} \AgdaFunction{⊃} \AgdaBound{ψ}\AgdaSymbol{)} \AgdaBound{δ} \AgdaSymbol{→} \AgdaFunction{EP} \AgdaBound{Γ} \AgdaSymbol{(}\AgdaBound{ψ} \AgdaFunction{⊃} \AgdaBound{φ}\AgdaSymbol{)} \AgdaBound{ε} \AgdaSymbol{→} \AgdaFunction{EE} \AgdaBound{Γ} \AgdaSymbol{(}\AgdaBound{φ} \AgdaFunction{≡〈} \AgdaInductiveConstructor{Ω} \AgdaFunction{〉} \AgdaBound{ψ}\AgdaSymbol{)} \AgdaSymbol{(}\AgdaFunction{univ} \AgdaBound{φ} \AgdaBound{ψ} \AgdaBound{δ} \AgdaBound{ε}\AgdaSymbol{)}\<%
\\
\>\AgdaKeyword{postulate} \AgdaPostulate{app*-EE} \AgdaSymbol{:} \AgdaSymbol{∀} \AgdaSymbol{\{}\AgdaBound{V}\AgdaSymbol{\}} \AgdaSymbol{\{}\AgdaBound{Γ} \AgdaSymbol{:} \AgdaDatatype{Context} \AgdaBound{V}\AgdaSymbol{\}} \AgdaSymbol{\{}\AgdaBound{M}\AgdaSymbol{\}} \AgdaSymbol{\{}\AgdaBound{M'}\AgdaSymbol{\}} \AgdaSymbol{\{}\AgdaBound{N}\AgdaSymbol{\}} \AgdaSymbol{\{}\AgdaBound{N'}\AgdaSymbol{\}} \AgdaSymbol{\{}\AgdaBound{A}\AgdaSymbol{\}} \AgdaSymbol{\{}\AgdaBound{B}\AgdaSymbol{\}} \AgdaSymbol{\{}\AgdaBound{P}\AgdaSymbol{\}} \AgdaSymbol{\{}\AgdaBound{Q}\AgdaSymbol{\}} \AgdaSymbol{→}\<%
\\
\>[9]\AgdaIndent{18}{}\<[18]%
\>[18]\AgdaFunction{EE} \AgdaBound{Γ} \AgdaSymbol{(}\AgdaBound{M} \AgdaFunction{≡〈} \AgdaBound{A} \AgdaInductiveConstructor{⇛} \AgdaBound{B} \AgdaFunction{〉} \AgdaBound{M'}\AgdaSymbol{)} \AgdaBound{P} \AgdaSymbol{→} \AgdaFunction{EE} \AgdaBound{Γ} \AgdaSymbol{(}\AgdaBound{N} \AgdaFunction{≡〈} \AgdaBound{A} \AgdaFunction{〉} \AgdaBound{N'}\AgdaSymbol{)} \AgdaBound{Q} \AgdaSymbol{→}\<%
\\
\>[9]\AgdaIndent{18}{}\<[18]%
\>[18]\AgdaFunction{E} \AgdaBound{Γ} \AgdaBound{A} \AgdaBound{N} \AgdaSymbol{→} \AgdaFunction{E} \AgdaBound{Γ} \AgdaBound{A} \AgdaBound{N'} \AgdaSymbol{→}\<%
\\
\>[9]\AgdaIndent{18}{}\<[18]%
\>[18]\AgdaFunction{EE} \AgdaBound{Γ} \AgdaSymbol{(}\AgdaFunction{appT} \AgdaBound{M} \AgdaBound{N} \AgdaFunction{≡〈} \AgdaBound{B} \AgdaFunction{〉} \AgdaFunction{appT} \AgdaBound{M'} \AgdaBound{N'}\AgdaSymbol{)} \AgdaSymbol{(}\AgdaFunction{app*} \AgdaBound{N} \AgdaBound{N'} \AgdaBound{P} \AgdaBound{Q}\AgdaSymbol{)}\<%
\\
%
\\
\>\AgdaKeyword{postulate} \AgdaPostulate{expand-EE} \AgdaSymbol{:} \AgdaSymbol{∀} \AgdaSymbol{\{}\AgdaBound{V}\AgdaSymbol{\}} \AgdaSymbol{\{}\AgdaBound{Γ} \AgdaSymbol{:} \AgdaDatatype{Context} \AgdaBound{V}\AgdaSymbol{\}} \AgdaSymbol{\{}\AgdaBound{A}\AgdaSymbol{\}} \AgdaSymbol{\{}\AgdaBound{M} \AgdaBound{N} \AgdaSymbol{:} \AgdaFunction{Term} \AgdaBound{V}\AgdaSymbol{\}} \AgdaSymbol{\{}\AgdaBound{P} \AgdaBound{Q}\AgdaSymbol{\}} \AgdaSymbol{→}\<%
\\
\>[18]\AgdaIndent{20}{}\<[20]%
\>[20]\AgdaFunction{EE} \AgdaBound{Γ} \AgdaSymbol{(}\AgdaBound{M} \AgdaFunction{≡〈} \AgdaBound{A} \AgdaFunction{〉} \AgdaBound{N}\AgdaSymbol{)} \AgdaBound{Q} \AgdaSymbol{→} \AgdaBound{Γ} \AgdaDatatype{⊢} \AgdaBound{P} \AgdaDatatype{∶} \AgdaBound{M} \AgdaFunction{≡〈} \AgdaBound{A} \AgdaFunction{〉} \AgdaBound{N} \AgdaSymbol{→} \AgdaDatatype{key-redex} \AgdaBound{P} \AgdaBound{Q} \AgdaSymbol{→} \AgdaFunction{EE} \AgdaBound{Γ} \AgdaSymbol{(}\AgdaBound{M} \AgdaFunction{≡〈} \AgdaBound{A} \AgdaFunction{〉} \AgdaBound{N}\AgdaSymbol{)} \AgdaBound{P}\<%
\\
\>\AgdaKeyword{postulate} \AgdaPostulate{func-EE} \AgdaSymbol{:} \AgdaSymbol{∀} \AgdaSymbol{\{}\AgdaBound{U}\AgdaSymbol{\}} \AgdaSymbol{\{}\AgdaBound{Γ} \AgdaSymbol{:} \AgdaDatatype{Context} \AgdaBound{U}\AgdaSymbol{\}} \AgdaSymbol{\{}\AgdaBound{A}\AgdaSymbol{\}} \AgdaSymbol{\{}\AgdaBound{B}\AgdaSymbol{\}} \AgdaSymbol{\{}\AgdaBound{M}\AgdaSymbol{\}} \AgdaSymbol{\{}\AgdaBound{M'}\AgdaSymbol{\}} \AgdaSymbol{\{}\AgdaBound{P}\AgdaSymbol{\}} \AgdaSymbol{→}\<%
\\
\>[0]\AgdaIndent{18}{}\<[18]%
\>[18]\AgdaBound{Γ} \AgdaDatatype{⊢} \AgdaBound{P} \AgdaDatatype{∶} \AgdaBound{M} \AgdaFunction{≡〈} \AgdaBound{A} \AgdaInductiveConstructor{⇛} \AgdaBound{B} \AgdaFunction{〉} \AgdaBound{M'} \AgdaSymbol{→}\<%
\\
\>[0]\AgdaIndent{18}{}\<[18]%
\>[18]\AgdaSymbol{(∀} \AgdaBound{V} \AgdaSymbol{(}\AgdaBound{Δ} \AgdaSymbol{:} \AgdaDatatype{Context} \AgdaBound{V}\AgdaSymbol{)} \AgdaSymbol{(}\AgdaBound{N} \AgdaBound{N'} \AgdaSymbol{:} \AgdaFunction{Term} \AgdaBound{V}\AgdaSymbol{)} \AgdaBound{Q} \AgdaBound{ρ} \AgdaSymbol{→} \AgdaBound{ρ} \AgdaPostulate{∶} \AgdaBound{Γ} \AgdaPostulate{⇒R} \AgdaBound{Δ} \AgdaSymbol{→} \<[74]%
\>[74]\<%
\\
\>[0]\AgdaIndent{18}{}\<[18]%
\>[18]\AgdaFunction{E} \AgdaBound{Δ} \AgdaBound{A} \AgdaBound{N} \AgdaSymbol{→} \AgdaFunction{E} \AgdaBound{Δ} \AgdaBound{A} \AgdaBound{N'} \AgdaSymbol{→} \AgdaFunction{EE} \AgdaBound{Δ} \AgdaSymbol{(}\AgdaBound{N} \AgdaFunction{≡〈} \AgdaBound{A} \AgdaFunction{〉} \AgdaBound{N'}\AgdaSymbol{)} \AgdaBound{Q} \AgdaSymbol{→}\<%
\\
\>[0]\AgdaIndent{18}{}\<[18]%
\>[18]\AgdaFunction{EE} \AgdaBound{Δ} \AgdaSymbol{(}\AgdaFunction{appT} \AgdaSymbol{(}\AgdaBound{M} \AgdaFunction{〈} \AgdaBound{ρ} \AgdaFunction{〉}\AgdaSymbol{)} \AgdaBound{N} \AgdaFunction{≡〈} \AgdaBound{B} \AgdaFunction{〉} \AgdaFunction{appT} \AgdaSymbol{(}\AgdaBound{M'} \AgdaFunction{〈} \AgdaBound{ρ} \AgdaFunction{〉}\AgdaSymbol{)} \AgdaBound{N'}\AgdaSymbol{)} \AgdaSymbol{(}\AgdaFunction{app*} \AgdaBound{N} \AgdaBound{N'} \AgdaSymbol{(}\AgdaBound{P} \AgdaFunction{〈} \AgdaBound{ρ} \AgdaFunction{〉}\AgdaSymbol{)} \AgdaBound{Q}\AgdaSymbol{))} \AgdaSymbol{→}\<%
\\
\>[0]\AgdaIndent{18}{}\<[18]%
\>[18]\AgdaFunction{EE} \AgdaBound{Γ} \AgdaSymbol{(}\AgdaBound{M} \AgdaFunction{≡〈} \AgdaBound{A} \AgdaInductiveConstructor{⇛} \AgdaBound{B} \AgdaFunction{〉} \AgdaBound{M'}\AgdaSymbol{)} \AgdaBound{P}\<%
\\
%
\\
\>\AgdaFunction{conv-EE} \AgdaSymbol{:} \AgdaSymbol{∀} \AgdaSymbol{\{}\AgdaBound{V}\AgdaSymbol{\}} \AgdaSymbol{\{}\AgdaBound{Γ} \AgdaSymbol{:} \AgdaDatatype{Context} \AgdaBound{V}\AgdaSymbol{\}} \AgdaSymbol{\{}\AgdaBound{M}\AgdaSymbol{\}} \AgdaSymbol{\{}\AgdaBound{N}\AgdaSymbol{\}} \AgdaSymbol{\{}\AgdaBound{M'}\AgdaSymbol{\}} \AgdaSymbol{\{}\AgdaBound{N'}\AgdaSymbol{\}} \AgdaSymbol{\{}\AgdaBound{A}\AgdaSymbol{\}} \AgdaSymbol{\{}\AgdaBound{P}\AgdaSymbol{\}} \AgdaSymbol{→}\<%
\\
\>[0]\AgdaIndent{10}{}\<[10]%
\>[10]\AgdaFunction{EE} \AgdaBound{Γ} \AgdaSymbol{(}\AgdaBound{M} \AgdaFunction{≡〈} \AgdaBound{A} \AgdaFunction{〉} \AgdaBound{N}\AgdaSymbol{)} \AgdaBound{P} \AgdaSymbol{→} \AgdaBound{M} \AgdaDatatype{≃} \AgdaBound{M'} \AgdaSymbol{→} \AgdaBound{N} \AgdaDatatype{≃} \AgdaBound{N'} \AgdaSymbol{→} \AgdaBound{Γ} \AgdaDatatype{⊢} \AgdaBound{M'} \AgdaDatatype{∶} \AgdaFunction{ty} \AgdaBound{A} \AgdaSymbol{→} \AgdaBound{Γ} \AgdaDatatype{⊢} \AgdaBound{N'} \AgdaDatatype{∶} \AgdaFunction{ty} \AgdaBound{A} \AgdaSymbol{→} \<[82]%
\>[82]\<%
\\
\>[0]\AgdaIndent{10}{}\<[10]%
\>[10]\AgdaFunction{EE} \AgdaBound{Γ} \AgdaSymbol{(}\AgdaBound{M'} \AgdaFunction{≡〈} \AgdaBound{A} \AgdaFunction{〉} \AgdaBound{N'}\AgdaSymbol{)} \AgdaBound{P}\<%
\\
\>\AgdaFunction{conv-EE} \AgdaBound{P∈EΓM≡N} \AgdaBound{M≃M'} \AgdaBound{N≃N'} \AgdaBound{Γ⊢M'∶A} \AgdaBound{Γ⊢N'∶A} \AgdaSymbol{=} \<[42]%
\>[42]\<%
\\
\>[0]\AgdaIndent{2}{}\<[2]%
\>[2]\AgdaPostulate{conv-E'} \AgdaSymbol{(}\AgdaPostulate{eq-resp-conv} \<[25]%
\>[25]\AgdaBound{M≃M'} \AgdaBound{N≃N'}\AgdaSymbol{)} \AgdaBound{P∈EΓM≡N} \AgdaSymbol{(}\AgdaInductiveConstructor{ctxER} \AgdaBound{Γ⊢M'∶A} \AgdaBound{Γ⊢N'∶A}\AgdaSymbol{)}\<%
\\
%
\\
\>\AgdaFunction{EE-typed} \AgdaSymbol{:} \AgdaSymbol{∀} \AgdaSymbol{\{}\AgdaBound{V}\AgdaSymbol{\}} \AgdaSymbol{\{}\AgdaBound{Γ} \AgdaSymbol{:} \AgdaDatatype{Context} \AgdaBound{V}\AgdaSymbol{\}} \AgdaSymbol{\{}\AgdaBound{E}\AgdaSymbol{\}} \AgdaSymbol{\{}\AgdaBound{P}\AgdaSymbol{\}} \AgdaSymbol{→} \AgdaFunction{EE} \AgdaBound{Γ} \AgdaBound{E} \AgdaBound{P} \AgdaSymbol{→} \AgdaBound{Γ} \AgdaDatatype{⊢} \AgdaBound{P} \AgdaDatatype{∶} \AgdaBound{E}\<%
\\
\>\AgdaFunction{EE-typed} \AgdaSymbol{=} \AgdaFunction{E'-typed}\<%
\\
%
\\
\>\AgdaFunction{EE-SN} \AgdaSymbol{:} \AgdaSymbol{∀} \AgdaSymbol{\{}\AgdaBound{V}\AgdaSymbol{\}} \AgdaSymbol{\{}\AgdaBound{Γ} \AgdaSymbol{:} \AgdaDatatype{Context} \AgdaBound{V}\AgdaSymbol{\}} \AgdaBound{E} \AgdaSymbol{\{}\AgdaBound{P}\AgdaSymbol{\}} \AgdaSymbol{→} \AgdaFunction{EE} \AgdaBound{Γ} \AgdaBound{E} \AgdaBound{P} \AgdaSymbol{→} \AgdaDatatype{SN} \AgdaBound{P}\<%
\\
\>\AgdaFunction{EE-SN} \AgdaSymbol{\_} \AgdaSymbol{=} \AgdaPostulate{E'-SN}\<%
\\
%
\\
\>\AgdaComment{\{-\<\\
\>postulate Neutral-computeE : ∀ \{V\} \{Γ : Context V\} \{M\} \{A\} \{N\} \{P : NeutralP V\} →\<\\
\>                           Γ ⊢ decode P ∶ M ≡〈 A 〉 N → computeE Γ M A N (decode P)\<\\
\>\<\\
\>postulate compute-SN : ∀ \{V\} \{Γ : Context V\} \{A\} \{δ\} → compute Γ A δ → valid Γ → SN δ\<\\
\>\<\\
\>postulate NF : ∀ \{V\} \{Γ\} \{φ : Term V\} → Γ ⊢ φ ∶ ty Ω → closed-prop\<\\
\>\<\\
\>postulate red-NF : ∀ \{V\} \{Γ\} \{φ : Term V\} (Γ⊢φ∶Ω : Γ ⊢ φ ∶ ty Ω) → φ ↠ cp2term (NF Γ⊢φ∶Ω)\<\\
\>\<\\
\>postulate closed-rep : ∀ \{U\} \{V\} \{ρ : Rep U V\} (A : closed-prop) → (cp2term A) 〈 ρ 〉 ≡ cp2term A\<\\
\>\<\\
\>postulate red-conv : ∀ \{V\} \{C\} \{K\} \{E F : Subexpression V C K\} → E ↠ F → E ≃ F\<\\
\>\<\\
\>postulate confluent : ∀ \{V\} \{φ : Term V\} \{ψ ψ' : closed-prop\} → φ ↠ cp2term ψ → φ ↠ cp2term ψ' → ψ ≡ ψ'\<\\
\>\<\\
\>func-EP \{δ = δ\} \{φ = φ\} \{ψ = ψ\} hyp Γ⊢δ∶φ⊃ψ = let Γ⊢φ⊃ψ∶Ω = Prop-Validity Γ⊢δ∶φ⊃ψ in\<\\
\>                      let Γ⊢φ∶Ω = ⊃-gen₁ Γ⊢φ⊃ψ∶Ω in\<\\
\>                      let Γ⊢ψ∶Ω = ⊃-gen₂ Γ⊢φ⊃ψ∶Ω in\<\\
\>                      let φ' = NF Γ⊢φ∶Ω in\<\\
\>                      Γ⊢δ∶φ⊃ψ ,p NF Γ⊢φ∶Ω ⊃C NF Γ⊢ψ∶Ω ,p \<\\
\>                      trans-red (respects-red \{f = λ x → x ⊃ ψ\} (λ x → app (appl x)) (red-NF Γ⊢φ∶Ω)) \<\\
\>                                (respects-red \{f = λ x → cp2term (NF Γ⊢φ∶Ω) ⊃ x\} (λ x → app (appr (appl x))) (red-NF Γ⊢ψ∶Ω)) ,p  --TODO Extract lemma for reduction\<\\
\>                      (λ W Δ ρ ε ρ∶Γ⇒Δ Δ⊢ε∶φ computeε →\<\\
\>                      let φρ↠φ' : φ 〈 ρ 〉 ↠ cp2term φ'\<\\
\>                          φρ↠φ' = subst (λ x → (φ 〈 ρ 〉) ↠ x) (closed-rep φ') (respects-red (respects-osr replacement β-respects-rep) (red-NF Γ⊢φ∶Ω)) in\<\\
\>                      let ε∈EΔψ = hyp W Δ ρ ε (context-validity Δ⊢ε∶φ) ρ∶Γ⇒Δ        \<\\
\>                                  ((convR Δ⊢ε∶φ (weakening Γ⊢φ∶Ω (context-validity Δ⊢ε∶φ) ρ∶Γ⇒Δ) (sym-conv (red-conv φρ↠φ')) ) ,p φ' ,p φρ↠φ' ,p computeε ) in \<\\
\>                      let ψ' = proj₁ (proj₂ ε∈EΔψ) in \<\\
\>                      let ψρ↠ψ' : ψ 〈 ρ 〉 ↠ cp2term ψ'\<\\
\>                          ψρ↠ψ' = proj₁ (proj₂ (proj₂ ε∈EΔψ)) in \<\\
\>                      subst (λ a → compute Δ a (appP (δ 〈 ρ 〉) ε)) (confluent ψρ↠ψ' \<\\
\>                        (subst (λ x → (ψ 〈 ρ 〉) ↠ x) (closed-rep (NF Γ⊢ψ∶Ω)) (respects-red (respects-osr replacement β-respects-rep) (red-NF Γ⊢ψ∶Ω)))) \<\\
\>                        (proj₂ (proj₂ (proj₂ ε∈EΔψ))))\<\\
\>\<\\
\>  plus-univ : ∀ \{V\} \{φ ψ : Term V\} \{δ ε\} → key-redex (plus (univ φ ψ δ ε)) δ\<\\
\>  minus-univ : ∀ \{V\} \{φ ψ : Term V\} \{δ ε\} → key-redex (minus (univ φ ψ δ ε)) ε\<\\
\>  imp*-plus : ∀ \{V\} \{P Q : Path V\} \{δ ε\} → key-redex (appP (appP (plus (P ⊃* Q)) δ) ε) (appP (plus Q) (appP δ (appP (minus P) ε)))\<\\
\>  imp*-minus : ∀ \{V\} \{P Q : Path V\} \{δ ε\} → key-redex (appP (appP (minus (P ⊃* Q)) δ) ε) (appP (minus Q) (appP δ (appP (plus P) ε)))\<\\
\>  appPkr : ∀ \{V\} \{δ ε χ : Proof V\} → key-redex δ ε → key-redex (appP δ χ) (appP ε χ)\<\\
\>  pluskr : ∀ \{V\} \{P Q : Path V\} → key-redex P Q → key-redex (plus P) (plus Q)\<\\
\>  minuskr : ∀ \{V\} \{P Q : Path V\} → key-redex P Q → key-redex (minus P) (minus Q)\<\\
\>  app*kr : ∀ \{V\} \{N N' : Term V\} \{P\} \{P'\} \{Q\} → key-redex P P' → key-redex (app* N N' P Q) (app* N N' P' Q)\<\\
\>  plus-ref : ∀ \{V\} \{φ : Term V\} \{δ\} → key-redex (appP (plus (reff φ)) δ) δ\<\\
\>  minus-ref : ∀ \{V\} \{φ : Term V\} \{δ\} → key-redex (appP (minus (reff φ)) δ) δ\<\\
\>  app*-ref : ∀ \{V\} \{M N N' : Term V\} \{Q\} → key-redex (app* N N' (reff M) Q) (M ⋆[ Q ∶ N ∼ N' ])\<\\
\>\<\\
\>postulate key-redex-SN : ∀ \{V\} \{K\} \{E F : Expression V K\} → key-redex E F → SN F → SN E\<\\
\>\<\\
\>postulate key-redex-rep : ∀ \{U\} \{V\} \{K\} \{ρ : Rep U V\} \{E F : Expression U K\} → key-redex E F → key-redex (E 〈 ρ 〉) (F 〈 ρ 〉)\<\\
\>\<\\
\>expand-compute : ∀ \{V\} \{Γ : Context V\} \{A : closed-prop\} \{δ ε : Proof V\} →\<\\
\>                compute Γ A ε → valid Γ → key-redex δ ε → compute Γ A δ\<\\
\>expand-compute \{A = ⊥C\} computeε validΓ δ▷ε = key-redex-SN δ▷ε (compute-SN computeε validΓ)\<\\
\>expand-compute \{A = A ⊃C B\} computeε validΓ δ▷ε W Δ ρ χ ρ∶Γ⇒RΔ Δ⊢χ∶A computeχ = \<\\
\>  expand-compute (computeε W Δ ρ χ ρ∶Γ⇒RΔ Δ⊢χ∶A computeχ) (context-validity Δ⊢χ∶A)\<\\
\>      (appPkr (key-redex-rep δ▷ε)) \<\\
\>\<\\
\>expand-EP (Γ⊢ε∶φ ,p φ' ,p φ↠φ' ,p computeε) Γ⊢δ∶φ δ▷ε = Γ⊢δ∶φ ,p φ' ,p φ↠φ' ,p expand-compute computeε (context-validity Γ⊢δ∶φ) δ▷ε\<\\
\>\<\\
\>expand-computeE : ∀ \{V\} \{Γ : Context V\} \{A\} \{M\} \{N\} \{P\} \{Q\} →\<\\
\>  computeE Γ M A N Q → Γ ⊢ P ∶ M ≡〈 A 〉 N → key-redex P Q → computeE Γ M A N P\<\\
\>expand-computeE \{A = Ω\} ((\_ ,p M⊃Nnf ,p M⊃N↠M⊃Nnf ,p computeQ+) ,p (\_ ,p N⊃Mnf ,p N⊃M↠N⊃Mnf ,p computeQ-)) Γ⊢P∶M≡N P▷Q = \<\\
\>  ((plusR Γ⊢P∶M≡N) ,p M⊃Nnf ,p M⊃N↠M⊃Nnf ,p expand-compute computeQ+ \<\\
\>    (context-validity Γ⊢P∶M≡N) (pluskr P▷Q)) ,p \<\\
\>  (minusR Γ⊢P∶M≡N) ,p N⊃Mnf ,p N⊃M↠N⊃Mnf ,p expand-compute computeQ- \<\\
\>    (context-validity Γ⊢P∶M≡N) (minuskr P▷Q)\<\\
\>expand-computeE \{A = A ⇛ B\} \{M\} \{M'\} computeQ Γ⊢P∶M≡M' P▷Q = \<\\
\>  λ W Δ ρ N N' R ρ∶Γ⇒Δ Δ⊢R∶N≡N' N∈EΔA N'∈EΔA computeR → \<\\
\>  expand-computeE (computeQ W Δ ρ N N' R ρ∶Γ⇒Δ Δ⊢R∶N≡N' N∈EΔA N'∈EΔA computeR) \<\\
\>  (app*R (E-typed N∈EΔA) (E-typed N'∈EΔA) \<\\
\>    (weakening Γ⊢P∶M≡M' (context-validity Δ⊢R∶N≡N') ρ∶Γ⇒Δ)\<\\
\>    Δ⊢R∶N≡N')\<\\
\>  (app*kr (key-redex-rep P▷Q))\<\\
\>\<\\
\>ref-compute : ∀ \{V\} \{Γ : Context V\} \{M : Term V\} \{A : Type\} → E Γ A M → computeE Γ M A M (reff M)\<\\
\>ref-compute \{Γ = Γ\} \{M = φ\} \{A = Ω\} φ∈EΓΩ = \<\\
\>  let Γ⊢φ∶Ω : Γ ⊢ φ ∶ ty Ω\<\\
\>      Γ⊢φ∶Ω = E-typed φ∈EΓΩ in\<\\
\>  (func-EP (λ V Δ ρ ε validΔ ρ∶Γ⇒Δ ε∈EΔφ → expand-EP ε∈EΔφ (appPR (plusR (refR (weakening Γ⊢φ∶Ω validΔ ρ∶Γ⇒Δ))) (EP-typed ε∈EΔφ)) plus-ref) \<\\
\>  (plusR (refR Γ⊢φ∶Ω))) ,p \<\\
\>  func-EP (λ V Δ ρ ε validΔ ρ∶Γ⇒Δ ε∈EΔφ → expand-EP ε∈EΔφ (appPR (minusR (refR (weakening Γ⊢φ∶Ω validΔ ρ∶Γ⇒Δ))) (EP-typed ε∈EΔφ)) minus-ref) \<\\
\>  (minusR (refR Γ⊢φ∶Ω))\<\\
\>ref-compute \{A = A ⇛ B\} (Γ⊢M∶A⇛B ,p computeM ,p compute-eqM) = λ W Δ ρ N N' Q ρ∶Γ⇒Δ Δ⊢Q∶N≡N' N∈EΔA N'∈EΔA computeQ → \<\\
\>  expand-computeE (compute-eqM W Δ ρ N N' Q ρ∶Γ⇒Δ N∈EΔA N'∈EΔA computeQ Δ⊢Q∶N≡N') \<\\
\>    (app*R (E-typed N∈EΔA) (E-typed N'∈EΔA) (refR (weakening Γ⊢M∶A⇛B (context-validity Δ⊢Q∶N≡N') ρ∶Γ⇒Δ)) \<\\
\>      Δ⊢Q∶N≡N') app*-ref\<\\
\>\<\\
\>postulate Neutral-E : ∀ \{V\} \{Γ : Context V\} \{A\} \{M\} → Neutral M → Γ ⊢ M ∶ ty A → E Γ A M\<\\
\>\<\\
\>var-E' : ∀ \{V\} \{A\} (Γ : Context V) (x : Var V -Term) → \<\\
\>  valid Γ → typeof x Γ ≡ ty A → E Γ A (var x)\<\\
\>var-E : ∀ \{V\} (Γ : Context V) (x : Var V -Term) → \<\\
\>        valid Γ → E Γ (typeof' x Γ) (var x)\<\\
\>computeE-SN : ∀ \{V\} \{Γ : Context V\} \{M\} \{N\} \{A\} \{P\} → computeE Γ M A N P → valid Γ → SN P\<\\
\>\<\\
\>computeE-SN \{A = Ω\} \{P\} (P+∈EΓM⊃N ,p \_) \_ = \<\\
\>  let SNplusP : SN (plus P)\<\\
\>      SNplusP = EP-SN P+∈EΓM⊃N \<\\
\>  in SNsubbodyl (SNsubexp SNplusP)\<\\
\>computeE-SN \{V\} \{Γ\} \{A = A ⇛ B\} \{P\} computeP validΓ =\<\\
\>  let x₀∈EΓ,AA : E (Γ ,T A) A (var x₀)\<\\
\>      x₀∈EΓ,AA = var-E' \{A = A\} (Γ ,T A) x₀ (ctxTR validΓ) refl in\<\\
\>  let SNapp*xxPref : SN (app* (var x₀) (var x₀) (P ⇑) (reff (var x₀)))\<\\
\>      SNapp*xxPref = computeE-SN \{A = B\} (computeP (V , -Term) (Γ ,T A ) upRep \<\\
\>          (var x₀) (var x₀) (app -ref (var x₀ ,, out)) upRep-typed \<\\
\>          (refR (varR x₀ (ctxTR validΓ)) )\<\\
\>          x₀∈EΓ,AA x₀∈EΓ,AA (ref-compute x₀∈EΓ,AA)) \<\\
\>          (ctxTR validΓ)\<\\
\>  in SNap' \{Ops = replacement\} \{σ = upRep\} R-respects-replacement (SNsubbodyl (SNsubbodyr (SNsubbodyr (SNsubexp SNapp*xxPref))))\<\\
\>\<\\
\>\<\\
\>E-SN (Ω) = EΩ.sn\<\\
\>E-SN \{V\} \{Γ\} (A ⇛ B) \{M\} (Γ⊢M∶A⇛B ,p computeM ,p computeMpath) =\<\\
\>  let SNMx : SN (appT (M ⇑) (var x₀))\<\\
\>      SNMx = E-SN B \<\\
\>             (computeM (V , -Term) (Γ ,T A) upRep (var x₀) upRep-typed \<\\
\>             (var-E' \{A = A\} (Γ ,T A) x₀ (ctxTR (context-validity Γ⊢M∶A⇛B)) refl)) \<\\
\>  in SNap' \{Ops = replacement\} \{σ = upRep\} R-respects-replacement (SNsubbodyl (SNsubexp SNMx)) \<\\
\>\<\\
\>\{- Neutral-E \{A = Ω\} neutralM Γ⊢M∶A = record \{ \<\\
\>  typed = Γ⊢M∶A ; \<\\
\>  sn = Neutral-SN neutralM \}\<\\
\>Neutral-E \{A = A ⇛ B\} \{M\} neutralM Γ⊢M∶A⇛B = \<\\
\>  Γ⊢M∶A⇛B ,p \<\\
\>  (λ W Δ ρ N ρ∶Γ⇒Δ N∈EΔA → Neutral-E \{A = B\} (app (Neutral-rep M ρ neutralM) (E-SN A N∈EΔA)) \<\\
\>    (appR (weakening Γ⊢M∶A⇛B (context-validity (E-typed N∈EΔA)) ρ∶Γ⇒Δ) (E-typed N∈EΔA))) ,p \<\\
\>  (λ W Δ ρ N N' P ρ∶Γ⇒Δ N∈EΔA N'∈EΔA computeP Δ⊢P∶N≡N' → \<\\
\>    let validΔ = context-validity (E-typed N∈EΔA) in\<\\
\>    Neutral-computeE (Neutral-⋆ (Neutral-rep M ρ neutralM) (computeE-SN computeP validΔ) (E-SN A N∈EΔA) (E-SN A N'∈EΔA)) \<\\
\>    (⋆-typed (weakening Γ⊢M∶A⇛B validΔ ρ∶Γ⇒Δ) Δ⊢P∶N≡N')) -\}\<\\
\>\<\\
\>var-E' \{A = A\} Γ x validΓ x∶A∈Γ = Neutral-E (var x) (change-type (varR x validΓ) x∶A∈Γ)\<\\
\>\<\\
\>var-E Γ x validΓ = var-E' \{A = typeof' x Γ\} Γ x validΓ typeof-typeof'\<\\
\>\<\\
\>⊥-E validΓ = record \{ typed = ⊥R validΓ ; sn = ⊥SN \}\<\\
\>\<\\
\>⊃-E φ∈EΓΩ ψ∈EΓΩ = record \{ typed = ⊃R (E-typed φ∈EΓΩ) (E-typed ψ∈EΓΩ) ; \<\\
\>  sn = ⊃SN (E-SN Ω φ∈EΓΩ) (E-SN Ω ψ∈EΓΩ) \}\<\\
\>\<\\
\>appT-E \{V\} \{Γ\} \{M\} \{N\} \{A\} \{B\} validΓ (Γ⊢M∶A⇛B ,p computeM ,p computeMpath) N∈EΓA = \<\\
\>  subst (λ a → E Γ B (appT a N)) rep-idOp (computeM V Γ (idRep V) N idRep-typed N∈EΓA)\<\\
\>\<\\
\>postulate cp-typed : ∀ \{V\} \{Γ : Context V\} A → valid Γ → Γ ⊢ cp2term A ∶ ty Ω\<\\
\>\<\\
\>postulate ⊃-not-⊥ : ∀ \{V\} \{φ ψ : Term V\} → φ ⊃ ψ ↠ ⊥ → Empty\<\\
\>\<\\
\>postulate ⊃-inj₁ : ∀ \{V\} \{φ φ' ψ ψ' : Term V\} → φ ⊃ ψ ↠ φ' ⊃ ψ' → φ ↠ φ'\<\\
\>\<\\
\>postulate ⊃-inj₂ : ∀ \{V\} \{φ φ' ψ ψ' : Term V\} → φ ⊃ ψ ↠ φ' ⊃ ψ' → ψ ↠ ψ'\<\\
\>\<\\
\>postulate confluent₂ : ∀ \{V\} \{φ ψ : Term V\} \{χ : closed-prop\} → φ ≃ ψ → φ ↠ cp2term χ → ψ ↠ cp2term χ\<\\
\>\<\\
\>appP-EP (\_ ,p ⊥C ,p φ⊃ψ↠⊥ ,p \_) \_ = ⊥-elim (⊃-not-⊥ φ⊃ψ↠⊥)\<\\
\>appP-EP \{V\} \{Γ\} \{ε = ε\} \{φ\} \{ψ = ψ\} (Γ⊢δ∶φ⊃ψ ,p (φ' ⊃C ψ') ,p φ⊃ψ↠φ'⊃ψ' ,p computeδ) (Γ⊢ε∶φ ,p φ'' ,p φ↠φ'' ,p computeε) = \<\\
\>  (appPR Γ⊢δ∶φ⊃ψ Γ⊢ε∶φ) ,p ψ' ,p ⊃-inj₂ φ⊃ψ↠φ'⊃ψ' ,p \<\\
\>  subst (λ x → compute Γ ψ' (appP x ε)) rep-idOp \<\\
\>  (computeδ V Γ (idRep V) ε idRep-typed \<\\
\>    (convR Γ⊢ε∶φ (cp-typed φ' (context-validity Γ⊢ε∶φ)) (red-conv (⊃-inj₁ φ⊃ψ↠φ'⊃ψ')))\<\\
\>  (subst (λ x → compute Γ x ε) (confluent φ↠φ'' (⊃-inj₁ φ⊃ψ↠φ'⊃ψ')) computeε))\<\\
\>\<\\
\>conv-EP φ≃ψ (Γ⊢δ∶φ ,p φ' ,p φ↠φ' ,p computeδ) Γ⊢ψ∶Ω = convR Γ⊢δ∶φ Γ⊢ψ∶Ω φ≃ψ ,p φ' ,p confluent₂ \{χ = φ'\} φ≃ψ φ↠φ' ,p computeδ\<\\
\>\<\\
\>\<\\
\>postulate rep-EP : ∀ \{U\} \{V\} \{Γ\} \{Δ\} \{ρ : Rep U V\} \{φ\} \{δ\} →\<\\
\>                 EP Γ φ δ → ρ ∶ Γ ⇒R Δ → EP Δ (φ 〈 ρ 〉) (δ 〈 ρ 〉)\<\\
\>\<\\
\>univ-EE \{V\} \{Γ\} \{φ\} \{ψ\} \{δ\} \{ε\} δ∈EΓφ⊃ψ ε∈EΓψ⊃φ = \<\\
\>  let Γ⊢univ∶φ≡ψ : Γ ⊢ univ φ ψ δ ε ∶ φ ≡〈 Ω 〉 ψ\<\\
\>      Γ⊢univ∶φ≡ψ = (univR (EP-typed δ∈EΓφ⊃ψ) (EP-typed ε∈EΓψ⊃φ)) in\<\\
\>      (Γ⊢univ∶φ≡ψ ,p \<\\
\>      expand-EP δ∈EΓφ⊃ψ (plusR Γ⊢univ∶φ≡ψ) plus-univ ,p \<\\
\>      expand-EP ε∈EΓψ⊃φ (minusR Γ⊢univ∶φ≡ψ) minus-univ)\<\\
\>\<\\
\>postulate rep-EE : ∀ \{U\} \{V\} \{Γ\} \{Δ\} \{ρ : Rep U V\} \{E\} \{P\} →\<\\
\>                 EE Γ E P → ρ ∶ Γ ⇒R Δ → EE Δ (E 〈 ρ 〉) (P 〈 ρ 〉)\<\\
\>\<\\
\>imp*-EE \{Γ = Γ\} \{φ\} \{φ'\} \{ψ = ψ\} \{ψ'\} \{P\} \{Q = Q\} P∈EΓφ≡φ' Q∈EΓψ≡ψ' = (⊃*R (EE-typed P∈EΓφ≡φ') (EE-typed Q∈EΓψ≡ψ')) ,p \<\\
\>  func-EP (λ V Δ ρ ε validΔ ρ∶Γ⇒RΔ ε∈EΔφ⊃ψ →\<\\
\>    let Pρ : EE Δ (φ 〈 ρ 〉 ≡〈 Ω 〉 φ' 〈 ρ 〉) (P 〈 ρ 〉)\<\\
\>        Pρ = rep-EE P∈EΓφ≡φ' ρ∶Γ⇒RΔ in\<\\
\>    let Qρ : EE Δ (ψ 〈 ρ 〉 ≡〈 Ω 〉 ψ' 〈 ρ 〉) (Q 〈 ρ 〉)\<\\
\>        Qρ = rep-EE Q∈EΓψ≡ψ' ρ∶Γ⇒RΔ in\<\\
\>    func-EP (λ W Θ σ χ validΘ σ∶Δ⇒RΘ χ∈EΘφ' → \<\\
\>    let Pρσ : EE Θ (φ 〈 ρ 〉 〈 σ 〉 ≡〈 Ω 〉 φ' 〈 ρ 〉 〈 σ 〉) (P 〈 ρ 〉 〈 σ 〉)\<\\
\>        Pρσ = rep-EE Pρ σ∶Δ⇒RΘ in\<\\
\>    let Pρσ- : EP Θ (φ' 〈 ρ 〉 〈 σ 〉 ⊃ φ 〈 ρ 〉 〈 σ 〉) (minus (P 〈 ρ 〉 〈 σ 〉))\<\\
\>        Pρσ- = minus-EP Pρσ in\<\\
\>    let Qρσ : EE Θ (ψ 〈 ρ 〉 〈 σ 〉 ≡〈 Ω 〉 ψ' 〈 ρ 〉 〈 σ 〉) (Q 〈 ρ 〉 〈 σ 〉)\<\\
\>        Qρσ = rep-EE Qρ σ∶Δ⇒RΘ in\<\\
\>    let Qρσ+ : EP Θ (ψ 〈 ρ 〉 〈 σ 〉 ⊃ ψ' 〈 ρ 〉 〈 σ 〉) (plus (Q 〈 ρ 〉 〈 σ 〉))\<\\
\>        Qρσ+ = plus-EP Qρσ in\<\\
\>    let εσ : EP Θ (φ 〈 ρ 〉 〈 σ 〉 ⊃ ψ 〈 ρ 〉 〈 σ 〉) (ε 〈 σ 〉)\<\\
\>        εσ = rep-EP ε∈EΔφ⊃ψ σ∶Δ⇒RΘ in\<\\
\>    expand-EP \<\\
\>    (appP-EP Qρσ+ (appP-EP εσ (appP-EP Pρσ- χ∈EΘφ')))\<\\
\>    (appPR (appPR (plusR (⊃*R (EE-typed Pρσ) (EE-typed Qρσ))) (EP-typed εσ)) (EP-typed χ∈EΘφ')) \<\\
\>    imp*-plus) \<\\
\>    (appPR (plusR (⊃*R (EE-typed Pρ) (EE-typed Qρ))) (EP-typed ε∈EΔφ⊃ψ))) \<\\
\>  (plusR (⊃*R (EE-typed P∈EΓφ≡φ') (EE-typed Q∈EΓψ≡ψ'))) ,p \<\\
\>  func-EP (λ V Δ ρ ε validΔ ρ∶Γ⇒RΔ ε∈EΔφ'⊃ψ' →\<\\
\>    let Pρ : EE Δ (φ 〈 ρ 〉 ≡〈 Ω 〉 φ' 〈 ρ 〉) (P 〈 ρ 〉)\<\\
\>        Pρ = rep-EE P∈EΓφ≡φ' ρ∶Γ⇒RΔ in\<\\
\>    let Qρ : EE Δ (ψ 〈 ρ 〉 ≡〈 Ω 〉 ψ' 〈 ρ 〉) (Q 〈 ρ 〉)\<\\
\>        Qρ = rep-EE Q∈EΓψ≡ψ' ρ∶Γ⇒RΔ in\<\\
\>    func-EP (λ W Θ σ χ validΘ σ∶Δ⇒RΘ χ∈EΘφ' → \<\\
\>      let Pρσ : EE Θ (φ 〈 ρ 〉 〈 σ 〉 ≡〈 Ω 〉 φ' 〈 ρ 〉 〈 σ 〉) (P 〈 ρ 〉 〈 σ 〉)\<\\
\>          Pρσ = rep-EE Pρ σ∶Δ⇒RΘ in\<\\
\>      let Pρσ+ : EP Θ (φ 〈 ρ 〉 〈 σ 〉 ⊃ φ' 〈 ρ 〉 〈 σ 〉) (plus (P 〈 ρ 〉 〈 σ 〉))\<\\
\>          Pρσ+ = plus-EP Pρσ in\<\\
\>      let Qρσ : EE Θ (ψ 〈 ρ 〉 〈 σ 〉 ≡〈 Ω 〉 ψ' 〈 ρ 〉 〈 σ 〉) (Q 〈 ρ 〉 〈 σ 〉)\<\\
\>          Qρσ = rep-EE Qρ σ∶Δ⇒RΘ in\<\\
\>      let Qρσ- : EP Θ (ψ' 〈 ρ 〉 〈 σ 〉 ⊃ ψ 〈 ρ 〉 〈 σ 〉) (minus (Q 〈 ρ 〉 〈 σ 〉))\<\\
\>          Qρσ- = minus-EP Qρσ in\<\\
\>      let εσ : EP Θ (φ' 〈 ρ 〉 〈 σ 〉 ⊃ ψ' 〈 ρ 〉 〈 σ 〉) (ε 〈 σ 〉)\<\\
\>          εσ = rep-EP ε∈EΔφ'⊃ψ' σ∶Δ⇒RΘ in \<\\
\>      expand-EP \<\\
\>        (appP-EP Qρσ- (appP-EP εσ (appP-EP Pρσ+ χ∈EΘφ'))) \<\\
\>          (appPR (appPR (minusR (⊃*R (EE-typed Pρσ) (EE-typed Qρσ))) (EP-typed εσ)) (EP-typed χ∈EΘφ')) \<\\
\>        imp*-minus)\<\\
\>    (appPR (minusR (⊃*R (EE-typed Pρ) (EE-typed Qρ))) (EP-typed ε∈EΔφ'⊃ψ'))) \<\\
\>  (minusR (⊃*R (EE-typed P∈EΓφ≡φ') (EE-typed Q∈EΓψ≡ψ')))\<\\
\>\<\\
\>app*-EE \{V\} \{Γ\} \{M\} \{M'\} \{N\} \{N'\} \{A\} \{B\} \{P\} \{Q\} (Γ⊢P∶M≡M' ,p computeP) (Γ⊢Q∶N≡N' ,p computeQ) N∈EΓA N'∈EΓA = (app*R (E-typed N∈EΓA) (E-typed N'∈EΓA) Γ⊢P∶M≡M' Γ⊢Q∶N≡N') ,p \<\\
\>  subst₃\<\\
\>    (λ a b c →\<\\
\>       computeE Γ (appT a N) B (appT b N') (app* N N' c Q))\<\\
\>    rep-idOp rep-idOp rep-idOp \<\\
\>    (computeP V Γ (idRep V) N N' Q idRep-typed Γ⊢Q∶N≡N' \<\\
\>      N∈EΓA N'∈EΓA computeQ)\<\\
\>\<\\
\>func-EE \{U\} \{Γ\} \{A\} \{B\} \{M\} \{M'\} \{P\} Γ⊢P∶M≡M' hyp = Γ⊢P∶M≡M' ,p (λ W Δ ρ N N' Q ρ∶Γ⇒Δ Δ⊢Q∶N≡N' N∈EΔA N'∈EΔA computeQ → \<\\
\>  proj₂ (hyp W Δ N N' Q ρ ρ∶Γ⇒Δ N∈EΔA N'∈EΔA (Δ⊢Q∶N≡N' ,p computeQ)))\<\\
\>\<\\
\>ref-EE \{V\} \{Γ\} \{M\} \{A\} M∈EΓA = refR (E-typed M∈EΓA) ,p ref-compute M∈EΓA\<\\
\>\<\\
\>expand-EE \{V\} \{Γ\} \{A\} \{M\} \{N\} \{P\} \{Q\} (Γ⊢Q∶M≡N ,p computeQ) Γ⊢P∶M≡N P▷Q = Γ⊢P∶M≡N ,p expand-computeE computeQ Γ⊢P∶M≡N P▷Q\<\\
\>\<\\
\>postulate ⊃-respects-conv : ∀ \{V\} \{φ\} \{φ'\} \{ψ\} \{ψ' : Term V\} → φ ≃ φ' → ψ ≃ ψ' →\<\\
\>                          φ ⊃ ψ ≃ φ' ⊃ ψ'\<\\
\>\<\\
\>postulate appT-respects-convl : ∀ \{V\} \{M M' N : Term V\} → M ≃ M' → appT M N ≃ appT M' N\<\\
\>\<\\
\>conv-computeE : ∀ \{V\} \{Γ : Context V\} \{M\} \{M'\} \{N\} \{N'\} \{A\} \{P\} →\<\\
\>             computeE Γ M A N P → M ≃ M' → N ≃ N' → \<\\
\>             Γ ⊢ M' ∶ ty A  → Γ ⊢ N' ∶ ty A  →\<\\
\>             computeE Γ M' A N' P\<\\
\>conv-computeE \{M = M\} \{A = Ω\} \<\\
\>  (EPΓM⊃NP+ ,p EPΓN⊃MP-) M≃M' N≃N' Γ⊢M'∶Ω Γ⊢N'∶Ω = \<\\
\>  (conv-EP (⊃-respects-conv M≃M' N≃N')\<\\
\>    EPΓM⊃NP+ (⊃R Γ⊢M'∶Ω Γ⊢N'∶Ω)) ,p \<\\
\>  conv-EP (⊃-respects-conv N≃N' M≃M') EPΓN⊃MP- (⊃R Γ⊢N'∶Ω Γ⊢M'∶Ω)\<\\
\>conv-computeE \{M = M\} \{M'\} \{N\} \{N'\} \{A = A ⇛ B\} computeP M≃M' N≃N' Γ⊢M'∶A⇛B Γ⊢N'∶A⇛B =\<\\
\>  λ W Δ ρ L L' Q ρ∶Γ⇒RΔ Δ⊢Q∶L≡L' L∈EΔA L'∈EΔA computeQ → conv-computeE \{A = B\} \<\\
\>  (computeP W Δ ρ L L' Q ρ∶Γ⇒RΔ Δ⊢Q∶L≡L' L∈EΔA L'∈EΔA computeQ) \<\\
\>  (appT-respects-convl (respects-conv (respects-osr replacement β-respects-rep) M≃M')) \<\\
\>  (appT-respects-convl (respects-conv (respects-osr replacement β-respects-rep) N≃N')) \<\\
\>  (appR (weakening Γ⊢M'∶A⇛B (context-validity Δ⊢Q∶L≡L') ρ∶Γ⇒RΔ) (E-typed \{W\} \{Γ = Δ\} \{A = A\} \{L\} L∈EΔA)) \<\\
\>  (appR (weakening Γ⊢N'∶A⇛B (context-validity Δ⊢Q∶L≡L') ρ∶Γ⇒RΔ) (E-typed L'∈EΔA)) \<\\
\>--REFACTOR Duplication\<\\
\>\<\\
\>conv-EE (Γ⊢P∶M≡N ,p computeP) M≃M' N≃N' Γ⊢M'∶A Γ⊢N'∶A = convER Γ⊢P∶M≡N Γ⊢M'∶A Γ⊢N'∶A M≃M' N≃N' ,p conv-computeE computeP M≃M' N≃N' Γ⊢M'∶A Γ⊢N'∶A\<\\
\>--REFACTOR Duplication                      \<\\
\>                 \<\\
\>EE-SN (app (-eq \_) (\_ ,, \_ ,, out)) (Γ⊢P∶E ,p computeP) = computeE-SN computeP (context-validity Γ⊢P∶E) -\}}\<%
\end{code}
}
}
\mode<all>{\AgdaHide{
\begin{code}%
\>\AgdaKeyword{module} \AgdaModule{PHOPL.MainProp} \AgdaKeyword{where}\<%
\\
\>\AgdaKeyword{open} \AgdaKeyword{import} \AgdaModule{Data.Empty} \AgdaKeyword{renaming} \AgdaSymbol{(}\AgdaDatatype{⊥} \AgdaSymbol{to} \AgdaDatatype{Empty}\AgdaSymbol{)}\<%
\\
\>\AgdaKeyword{open} \AgdaKeyword{import} \AgdaModule{Data.Unit}\<%
\\
\>\AgdaKeyword{open} \AgdaKeyword{import} \AgdaModule{Data.Product} \AgdaKeyword{renaming} \AgdaSymbol{(}\AgdaInductiveConstructor{\_,\_} \AgdaSymbol{to} \AgdaInductiveConstructor{\_,p\_}\AgdaSymbol{)}\<%
\\
\>\AgdaKeyword{open} \AgdaKeyword{import} \AgdaModule{Data.List}\<%
\\
\>\AgdaKeyword{open} \AgdaKeyword{import} \AgdaModule{Prelims}\<%
\\
\>\AgdaKeyword{open} \AgdaKeyword{import} \AgdaModule{PHOPL.Grammar}\<%
\\
\>\AgdaKeyword{open} \AgdaKeyword{import} \AgdaModule{PHOPL.Rules}\<%
\\
\>\AgdaKeyword{open} \AgdaKeyword{import} \AgdaModule{PHOPL.PathSub}\<%
\\
\>\AgdaKeyword{open} \AgdaKeyword{import} \AgdaModule{PHOPL.Red}\<%
\\
\>\AgdaKeyword{open} \AgdaKeyword{import} \AgdaModule{PHOPL.Meta}\<%
\\
\>\AgdaKeyword{open} \AgdaKeyword{import} \AgdaModule{PHOPL.Computable}\<%
\\
\>\AgdaKeyword{open} \AgdaKeyword{import} \AgdaModule{PHOPL.SubC}\<%
\\
\>\AgdaKeyword{open} \AgdaKeyword{import} \AgdaModule{PHOPL.SN}\<%
\end{code}
}

Our main theorem is as follows.

\begin{theorem}$ $
\label{theorem:mainprop}
\begin{enumerate}
\item
If $\Gamma \vdash t : T$ and $\sigma : \Gamma \Rightarrow \Delta$ is computable, and $\Delta \vald$, then $t[\sigma] \in E_\Delta(T[\sigma])$.

\begin{code}%
\>\AgdaKeyword{postulate} \AgdaPostulate{Computable-Sub} \AgdaSymbol{:} \AgdaSymbol{∀} \AgdaSymbol{\{}\AgdaBound{U} \AgdaBound{V} \AgdaBound{K}\AgdaSymbol{\}} \AgdaSymbol{(}\AgdaBound{σ} \AgdaSymbol{:} \AgdaFunction{Sub} \AgdaBound{U} \AgdaBound{V}\AgdaSymbol{)} \AgdaSymbol{\{}\AgdaBound{Γ} \AgdaBound{Δ}\AgdaSymbol{\}} \<[57]%
\>[57]\<%
\\
\>[0]\AgdaIndent{25}{}\<[25]%
\>[25]\AgdaSymbol{\{}\AgdaBound{M} \AgdaSymbol{:} \AgdaFunction{Expression} \AgdaBound{U} \AgdaSymbol{(}\AgdaInductiveConstructor{varKind} \AgdaBound{K}\AgdaSymbol{)\}} \AgdaSymbol{\{}\AgdaBound{A}\AgdaSymbol{\}} \AgdaSymbol{→}\<%
\\
\>[0]\AgdaIndent{25}{}\<[25]%
\>[25]\AgdaBound{σ} \AgdaFunction{∶} \AgdaBound{Γ} \AgdaFunction{⇒C} \AgdaBound{Δ} \AgdaSymbol{→} \AgdaBound{Γ} \AgdaDatatype{⊢} \AgdaBound{M} \AgdaDatatype{∶} \AgdaBound{A} \AgdaSymbol{→} \AgdaDatatype{valid} \AgdaBound{Δ} \AgdaSymbol{→} \AgdaRecord{E} \AgdaBound{Δ} \AgdaSymbol{(}\AgdaBound{A} \AgdaFunction{⟦} \AgdaBound{σ} \AgdaFunction{⟧}\AgdaSymbol{)} \AgdaSymbol{(}\AgdaBound{M} \AgdaFunction{⟦} \AgdaBound{σ} \AgdaFunction{⟧}\AgdaSymbol{)}\<%
\end{code}

\item
\label{computable-path-sub}
If $\Gamma \vdash M : A$, $\tau : \sigma \sim \rho : \Gamma \Rightarrow \Delta$, and $\tau$, $\sigma$
and $\rho$ are all computable, and $\Delta \vald$, then $M \{ \tau : \sigma \sim \rho \} \in E_\Delta(M [ \sigma ] =_A M [ \rho ])$.

\begin{code}%
\>\AgdaKeyword{postulate} \AgdaPostulate{computable-path-substitution} \AgdaSymbol{:} \AgdaSymbol{∀} \AgdaSymbol{\{}\AgdaBound{U} \AgdaBound{V}\AgdaSymbol{\}} \AgdaSymbol{(}\AgdaBound{τ} \AgdaSymbol{:} \AgdaFunction{PathSub} \AgdaBound{U} \AgdaBound{V}\AgdaSymbol{)} \AgdaSymbol{\{}\AgdaBound{σ} \AgdaBound{σ'} \AgdaBound{Γ} \AgdaBound{Δ} \AgdaBound{M} \AgdaBound{A}\AgdaSymbol{\}} \AgdaSymbol{→} \<[84]%
\>[84]\<%
\\
\>[25]\AgdaIndent{41}{}\<[41]%
\>[41]\AgdaBound{σ} \AgdaFunction{∶} \AgdaBound{Γ} \AgdaFunction{⇒C} \AgdaBound{Δ} \AgdaSymbol{→} \AgdaBound{σ'} \AgdaFunction{∶} \AgdaBound{Γ} \AgdaFunction{⇒C} \AgdaBound{Δ} \AgdaSymbol{→} \AgdaBound{τ} \AgdaFunction{∶} \AgdaBound{σ} \AgdaFunction{∼} \AgdaBound{σ'} \AgdaFunction{∶} \AgdaBound{Γ} \AgdaFunction{⇒C} \AgdaBound{Δ} \AgdaSymbol{→} \AgdaBound{Γ} \AgdaDatatype{⊢} \AgdaBound{M} \AgdaDatatype{∶} \AgdaFunction{ty} \AgdaBound{A} \AgdaSymbol{→} \AgdaDatatype{valid} \AgdaBound{Δ} \AgdaSymbol{→} \<[115]%
\>[115]\<%
\\
\>[25]\AgdaIndent{41}{}\<[41]%
\>[41]\AgdaRecord{E} \AgdaBound{Δ} \AgdaSymbol{(}\AgdaBound{M} \AgdaFunction{⟦} \AgdaBound{σ} \AgdaFunction{⟧} \AgdaFunction{≡〈} \AgdaBound{A} \AgdaFunction{〉} \AgdaBound{M} \AgdaFunction{⟦} \AgdaBound{σ'} \AgdaFunction{⟧}\AgdaSymbol{)} \AgdaSymbol{(}\AgdaBound{M} \AgdaFunction{⟦⟦} \AgdaBound{τ} \AgdaFunction{∶} \AgdaBound{σ} \AgdaFunction{∼} \AgdaBound{σ'} \AgdaFunction{⟧⟧}\AgdaSymbol{)} \<%
\end{code}

\end{enumerate}
\end{theorem}

\begin{proof}
The four parts are proved simultaneously by induction on derivations.

\begin{itemize}
\item
$$ \infer[(x : A \in \Gamma)]{\Gamma \vdash x : A}{\Gamma \vald} $$

We have that $\sigma(x) \in E_\Delta(A)$ and $\tau(x) \in E_\Delta(\rho(x) =_A \sigma(x))$ by hypothesis.
\item
$$ \infer[(p : \phi \in \Gamma)]{\Gamma \vdash p : \phi}{\Gamma \vald}$$

We have that $\sigma(p) \in E_\Delta(\phi[\sigma])$ by hypothesis.
\AgdaHide{
\begin{code}%
\>\AgdaComment{--Computable-Sub \_ σ∶Γ⇒CΔ (varR x \_) \_ = σ∶Γ⇒CΔ x}\<%
\end{code}
}

\item
$$ \infer{\Gamma \vdash \bot : \Omega}{\Gamma \vald} $$

From Lemma \ref{lm:Ebot}, we have $\bot \in E_\Delta(\Omega)$ and therefore
$\bot\{\} \in E_\Delta(\bot =_\Omega \bot)$.

\AgdaHide{
\begin{code}%
\>\<%
\\
\>\AgdaComment{\{- Computable-Sub σ σ∶Γ⇒CΔ (⊥R validΓ) validΔ = \{!!\} -\}}\<%
\end{code}
}

\AgdaHide{
\begin{code}%
\>\AgdaComment{\{- Computable-Sub σ σ∶Γ⇒CΔ (appR Γ⊢M∶A⇛B Γ⊢N∶A) validΔ = appT-E validΔ (Computable-Sub σ σ∶Γ⇒CΔ Γ⊢M∶A⇛B validΔ) (Computable-Sub σ σ∶Γ⇒CΔ Γ⊢N∶A validΔ)\<\\
\>Computable-Sub σ \{Δ = Δ\} σ∶Γ⇒CΔ (ΛR \{A = A\} \{M\} \{B\} Γ,A⊢M∶B) validΔ = \<\\
\>  let Δ,A⊢M⟦σ⟧∶B : Δ ,T A ⊢ M ⟦ liftSub \_ σ ⟧ ∶ ty B\<\\
\>      Δ,A⊢M⟦σ⟧∶B = substitution Γ,A⊢M∶B (ctxTR validΔ) (liftSub-typed (subC-typed σ∶Γ⇒CΔ)) in\<\\
\>  EI \<\\
\>  (ΛR Δ,A⊢M⟦σ⟧∶B) \<\\
\>  ((λ Θ \{ρ\} \{N\} ρ∶Δ⇒RΘ Θ⊢N∶A computeN → \<\\
\>    let validΘ : valid Θ\<\\
\>        validΘ = context-validity Θ⊢N∶A in\<\\
\>    E.computable (wteT (weakening Δ,A⊢M⟦σ⟧∶B (ctxTR validΘ) (liftRep-typed ρ∶Δ⇒RΘ)) (EI Θ⊢N∶A computeN) \<\\
\>    (subst (E Θ (ty B)) \<\\
\>      (let open ≡-Reasoning in \<\\
\>      begin\<\\
\>        M ⟦ x₀:= N • liftSub \_ (ρ •RS σ) ⟧\<\\
\>      ≡⟨ sub-comp M ⟩\<\\
\>        M ⟦ liftSub \_ (ρ •RS σ) ⟧ ⟦ x₀:= N ⟧\<\\
\>      ≡⟨ sub-congl (sub-congr M liftSub-compRS) ⟩\<\\
\>        M ⟦ liftRep \_ ρ •RS liftSub \_ σ ⟧ ⟦ x₀:= N ⟧\<\\
\>      ≡⟨ sub-congl (sub-compRS M) ⟩\<\\
\>        M ⟦ liftSub \_ σ ⟧ 〈 liftRep \_ ρ 〉 ⟦ x₀:= N ⟧\<\\
\>      ∎) \<\\
\>      (Computable-Sub (x₀:= N • liftSub \_ (ρ •RS σ)) \<\\
\>      (extend-subC (subCRS ρ∶Δ⇒RΘ σ∶Γ⇒CΔ validΘ) (EI Θ⊢N∶A computeN)) Γ,A⊢M∶B validΘ)))) ,p \<\\
\>  (λ Θ ρ∶Γ⇒RΔ Θ⊢Q∶N≡N' computeN computeN' computeQ → \{!!\}))\<\\
\>Computable-Sub σ σ∶Γ⇒CΔ (⊃R Γ⊢M∶A Γ⊢M∶A₁) validΔ = \{!!\}\<\\
\>Computable-Sub σ σ∶Γ⇒CΔ (appPR Γ⊢M∶A Γ⊢M∶A₁) validΔ = \{!!\}\<\\
\>Computable-Sub σ σ∶Γ⇒CΔ (ΛPR Γ⊢M∶A Γ⊢M∶A₁) validΔ = \{!!\}\<\\
\>Computable-Sub σ σ∶Γ⇒CΔ (convR Γ⊢M∶A Γ⊢M∶A₁ x) validΔ = \{!!\}\<\\
\>Computable-Sub σ σ∶Γ⇒CΔ (refR Γ⊢M∶A) validΔ = \{!!\}\<\\
\>Computable-Sub σ σ∶Γ⇒CΔ (⊃*R Γ⊢M∶A Γ⊢M∶A₁) validΔ = \{!!\}\<\\
\>Computable-Sub σ σ∶Γ⇒CΔ (univR Γ⊢M∶A Γ⊢M∶A₁) validΔ = \{!!\}\<\\
\>Computable-Sub σ σ∶Γ⇒CΔ (plusR Γ⊢M∶A) validΔ = \{!!\}\<\\
\>Computable-Sub σ σ∶Γ⇒CΔ (minusR Γ⊢M∶A) validΔ = \{!!\}\<\\
\>Computable-Sub σ σ∶Γ⇒CΔ (lllR Γ⊢M∶A) validΔ = \{!!\}\<\\
\>Computable-Sub σ σ∶Γ⇒CΔ (app*R Γ⊢M∶A Γ⊢M∶A₁ Γ⊢M∶A₂ Γ⊢M∶A₃) validΔ = \{!!\}\<\\
\>Computable-Sub σ σ∶Γ⇒CΔ (convER Γ⊢M∶A Γ⊢M∶A₁ Γ⊢M∶A₂ M≃M' N≃N') validΔ = \{!!\}\<\\
\>\<\\
\>computable-path-substitution τ σ∶Γ⇒CΔ σ'∶Γ⇒CΔ τ∶σ∼σ' Γ⊢M∶A validΔ = \{!!\} -\}}\<%
\\
%
\\
\>\AgdaComment{\{- Computable-Sub σ σ∶Γ⇒Δ (varR x validΓ) validΔ \_ = σ∶Γ⇒Δ x\<\\
\>Computable-Sub \{V = V\} σ \{Δ = Δ\} σ∶Γ⇒Δ (appR Γ⊢M∶A⇛B Γ⊢N∶A) validΔ \_ = \<\\
\>  appT-E validΔ (Computable-Sub σ σ∶Γ⇒Δ Γ⊢M∶A⇛B validΔ) (Computable-Sub σ σ∶Γ⇒Δ Γ⊢N∶A validΔ)\<\\
\>Computable-Sub σ σ∶Γ⇒Δ (ΛR \{M = M\} \{B\} Γ,A⊢M∶B) validΔ = \<\\
\>  func-E (λ \_ Θ ρ N validΘ ρ∶Δ⇒Θ N∈EΘA → \<\\
\>    let MN∈EΘB = subst (E Θ B) (subrepsub M)\<\\
\>                 (Computable-Sub (x₀:= N •SR liftRep \_ ρ • liftSub -Term σ) \<\\
\>                 (compC (compSRC (botsubC N∈EΘA) \<\\
\>                        (liftRep-typed ρ∶Δ⇒Θ)) \<\\
\>                 (liftSubC σ∶Γ⇒Δ)) \<\\
\>                 Γ,A⊢M∶B validΘ) in\<\\
\>    expand-E MN∈EΘB\<\\
\>      (appR (ΛR (weakening (substitution Γ,A⊢M∶B (ctxTR validΔ) (liftSub-typed (subC-typed σ∶Γ⇒Δ))) \<\\
\>                                                      (ctxTR validΘ) \<\\
\>                                         (liftRep-typed ρ∶Δ⇒Θ))) \<\\
\>                (E-typed N∈EΘA)) \<\\
\>      (βTkr (SNap' \{Ops = SUB\} R-respects-sub (E-SN B MN∈EΘB))))\<\\
\>Computable-Sub σ σ∶Γ⇒Δ (⊥R \_) validΔ = ⊥-E validΔ\<\\
\>Computable-Sub σ σ∶Γ⇒Δ (⊃R Γ⊢φ∶Ω Γ⊢ψ∶Ω) validΔ = ⊃-E \<\\
\>  (Computable-Sub σ σ∶Γ⇒Δ Γ⊢φ∶Ω validΔ) (Computable-Sub σ σ∶Γ⇒Δ Γ⊢ψ∶Ω validΔ)\<\\
\>Computable-Sub σ σ∶Γ⇒Δ (appPR Γ⊢δ∶φ⊃ψ Γ⊢ε∶φ) validΔ = appP-EP \<\\
\>  (Computable-Sub σ σ∶Γ⇒Δ Γ⊢δ∶φ⊃ψ validΔ) (Computable-Sub σ σ∶Γ⇒Δ Γ⊢ε∶φ validΔ)\<\\
\>Computable-Sub σ \{Γ = Γ\} \{Δ = Δ\} σ∶Γ⇒Δ (ΛPR \{δ = δ\} \{φ\} \{ψ\} Γ⊢φ∶Ω Γ,φ⊢δ∶ψ) validΔ = \<\\
\>  let Δ⊢Λφδσ∶φ⊃ψ : Δ ⊢ (ΛP φ δ) ⟦ σ ⟧ ∶ φ ⟦ σ ⟧ ⊃ ψ ⟦ σ ⟧\<\\
\>      Δ⊢Λφδσ∶φ⊃ψ = substitution \{A = φ ⊃ ψ\} (ΛPR Γ⊢φ∶Ω Γ,φ⊢δ∶ψ) validΔ (subC-typed σ∶Γ⇒Δ) in\<\\
\>  func-EP (λ W Θ ρ ε ρ∶Δ⇒Θ ε∈EΔφ → \<\\
\>    let δε∈EΘψ : EP Θ (ψ ⟦ σ ⟧ 〈 ρ 〉) (δ ⟦ liftSub \_ σ ⟧ 〈 liftRep \_ ρ 〉 ⟦ x₀:= ε ⟧)\<\\
\>        δε∈EΘψ = subst₂ (EP Θ) (subrepbotsub-up ψ) (subrepsub δ) \<\\
\>                 (Computable-Sub (x₀:= ε •SR liftRep \_ ρ • liftSub \_ σ) \<\\
\>                 (compC (compSRC (botsubCP ε∈EΔφ) \<\\
\>                        (liftRep-typed ρ∶Δ⇒Θ)) \<\\
\>                (liftSubC σ∶Γ⇒Δ)) Γ,φ⊢δ∶ψ (context-validity (EP-typed ε∈EΔφ))) in\<\\
\>    expand-EP δε∈EΘψ (appPR (weakening Δ⊢Λφδσ∶φ⊃ψ (context-validity (EP-typed ε∈EΔφ)) ρ∶Δ⇒Θ) (EP-typed ε∈EΔφ)) (βR-exp \{!!\} (EP-SN ε∈EΔφ) (EP-SN δε∈EΘψ))\<\\
\>      (βPkr (SNrep R-creates-rep (E-SN Ω (Computable-Sub σ σ∶Γ⇒Δ Γ⊢φ∶Ω validΔ))) (EP-SN ε∈EΔφ)))\<\\
\>  Δ⊢Λφδσ∶φ⊃ψ\<\\
\>Computable-Sub σ σ∶Γ⇒Δ (convR Γ⊢δ∶φ Γ⊢ψ∶Ω φ≃ψ) validΔ = conv-E (respects-conv (respects-osr SUB R-respects-sub) φ≃ψ) \<\\
\>  (Computable-Sub σ σ∶Γ⇒Δ Γ⊢δ∶φ validΔ) (ctxPR (substitution Γ⊢ψ∶Ω validΔ (subC-typed σ∶Γ⇒Δ)))\<\\
\>Computable-Sub σ σ∶Γ⇒Δ (refR Γ⊢M∶A) validΔ = ref-EE (Computable-Sub σ σ∶Γ⇒Δ Γ⊢M∶A validΔ)\<\\
\>Computable-Sub σ σ∶Γ⇒Δ (⊃*R Γ⊢φ∶Ω Γ⊢ψ∶Ω) validΔ = ⊃*-EE (Computable-Sub σ σ∶Γ⇒Δ Γ⊢φ∶Ω validΔ) (Computable-Sub σ σ∶Γ⇒Δ Γ⊢ψ∶Ω validΔ)\<\\
\>Computable-Sub σ σ∶Γ⇒Δ (univR Γ⊢δ∶φ⊃ψ Γ⊢ε∶ψ⊃φ) validΔ = univ-EE (Computable-Sub σ σ∶Γ⇒Δ Γ⊢δ∶φ⊃ψ validΔ) (Computable-Sub σ σ∶Γ⇒Δ Γ⊢ε∶ψ⊃φ validΔ)\<\\
\>Computable-Sub σ σ∶Γ⇒Δ (plusR Γ⊢P∶φ≡ψ) validΔ = plus-EP (Computable-Sub σ σ∶Γ⇒Δ Γ⊢P∶φ≡ψ validΔ)\<\\
\>Computable-Sub σ σ∶Γ⇒Δ (minusR Γ⊢P∶φ≡ψ) validΔ = minus-EP (Computable-Sub σ σ∶Γ⇒Δ Γ⊢P∶φ≡ψ validΔ)\<\\
\>Computable-Sub σ \{Δ = Δ\} σ∶Γ⇒Δ (lllR \{A = A\} \{B = B\} \{M = M\} \{N = N\} \{P = P\} ΓAAe⊢P∶Mx≡Ny) validΔ = \<\\
\>   let validΔAA : valid (Δ ,T A ,T A)\<\\
\>       validΔAA = ctxTR (ctxTR validΔ) in\<\\
\>   let ΔAAE⊢P∶Mx≡Ny : addpath Δ A ⊢ P ⟦ liftSub -Path (liftSub -Term (liftSub -Term σ)) ⟧ ∶ appT (M ⟦ σ ⟧ ⇑ ⇑ ⇑) (var x₂) ≡〈 B 〉 appT (N ⟦ σ ⟧ ⇑ ⇑ ⇑) (var x₁)\<\\
\>       ΔAAE⊢P∶Mx≡Ny = change-type \<\\
\>                      (substitution ΓAAe⊢P∶Mx≡Ny (valid-addpath validΔ) \<\\
\>                        (liftSub-typed (liftSub-typed (liftSub-typed (subC-typed σ∶Γ⇒Δ))))) \<\\
\>                      (cong₂ (λ x y → appT x (var x₂) ≡〈 B 〉 appT y (var x₁)) (liftSub-upRep₃ M) (liftSub-upRep₃ N)) in\<\\
\>   func-EE \<\\
\>   (lllR ΔAAE⊢P∶Mx≡Ny)\<\\
\>   (λ V Θ L L' Q ρ ρ∶Δ⇒RΘ L∈EΘA L'∈EΘA Q∈EΘL≡L' → \<\\
\>     let validΘ : valid Θ\<\\
\>         validΘ = context-validity (E.typed L∈EΘA) in\<\\
\>     let liftRepρ∶apΔ⇒RapΘ : liftRep \_ (liftRep \_ (liftRep \_ ρ)) ∶ addpath Δ A ⇒R addpath Θ A\<\\
\>         liftRepρ∶apΔ⇒RapΘ = liftRep-typed (liftRep-typed (liftRep-typed ρ∶Δ⇒RΘ)) in --TODO General lemma\<\\
\>     expand-EE \<\\
\>       (subst₂ (EE Θ) \<\\
\>         (cong₂ (λ x y → appT x L ≡〈 B 〉 appT y L') \<\\
\>           (let open ≡-Reasoning in \<\\
\>           begin\<\\
\>             M ⇑ ⇑ ⇑ ⟦ (x₂:= L ,x₁:= L' ,x₀:= Q) •SR liftRep -Path (liftRep -Term (liftRep -Term ρ)) • liftSub -Path (liftSub -Term (liftSub -Term σ)) ⟧\<\\
\>           ≡⟨ sub-comp (M ⇑ ⇑ ⇑) ⟩\<\\
\>             M ⇑ ⇑ ⇑ ⟦ liftSub -Path (liftSub -Term (liftSub -Term σ)) ⟧ ⟦ (x₂:= L ,x₁:= L' ,x₀:= Q) •SR liftRep -Path (liftRep -Term (liftRep -Term ρ)) ⟧\<\\
\>           ≡⟨ sub-congl (liftSub-upRep₃ M) ⟩\<\\
\>             M ⟦ σ ⟧ ⇑ ⇑ ⇑ ⟦ (x₂:= L ,x₁:= L' ,x₀:= Q) •SR liftRep -Path (liftRep -Term (liftRep -Term ρ)) ⟧\<\\
\>           ≡⟨ sub-compSR (M ⟦ σ ⟧ ⇑ ⇑ ⇑) ⟩\<\\
\>             M ⟦ σ ⟧ ⇑ ⇑ ⇑ 〈 liftRep -Path (liftRep -Term (liftRep -Term ρ)) 〉 ⟦ x₂:= L ,x₁:= L' ,x₀:= Q ⟧\<\\
\>           ≡⟨ sub-congl (liftRep-upRep₃ (M ⟦ σ ⟧)) ⟩\<\\
\>             M ⟦ σ ⟧ 〈 ρ 〉 ⇑ ⇑ ⇑ ⟦ x₂:= L ,x₁:= L' ,x₀:= Q ⟧\<\\
\>           ≡⟨ botSub-upRep₃ ⟩\<\\
\>             M ⟦ σ ⟧ 〈 ρ 〉\<\\
\>           ∎) \<\\
\>           (let open ≡-Reasoning in \<\\
\>           begin\<\\
\>             N ⇑ ⇑ ⇑ ⟦ (x₂:= L ,x₁:= L' ,x₀:= Q) •SR liftRep -Path (liftRep -Term (liftRep -Term ρ)) • liftSub -Path (liftSub -Term (liftSub -Term σ)) ⟧\<\\
\>           ≡⟨ sub-comp (N ⇑ ⇑ ⇑) ⟩\<\\
\>             N ⇑ ⇑ ⇑ ⟦ liftSub -Path (liftSub -Term (liftSub -Term σ)) ⟧ ⟦ (x₂:= L ,x₁:= L' ,x₀:= Q) •SR liftRep -Path (liftRep -Term (liftRep -Term ρ)) ⟧\<\\
\>           ≡⟨ sub-congl (liftSub-upRep₃ N) ⟩\<\\
\>             N ⟦ σ ⟧ ⇑ ⇑ ⇑ ⟦ (x₂:= L ,x₁:= L' ,x₀:= Q) •SR liftRep -Path (liftRep -Term (liftRep -Term ρ)) ⟧\<\\
\>           ≡⟨ sub-compSR (N ⟦ σ ⟧ ⇑ ⇑ ⇑) ⟩\<\\
\>             N ⟦ σ ⟧ ⇑ ⇑ ⇑ 〈 liftRep -Path (liftRep -Term (liftRep -Term ρ)) 〉 ⟦ x₂:= L ,x₁:= L' ,x₀:= Q ⟧\<\\
\>           ≡⟨ sub-congl (liftRep-upRep₃ (N ⟦ σ ⟧)) ⟩\<\\
\>             N ⟦ σ ⟧ 〈 ρ 〉 ⇑ ⇑ ⇑ ⟦ x₂:= L ,x₁:= L' ,x₀:= Q ⟧\<\\
\>           ≡⟨ botSub-upRep₃ ⟩\<\\
\>             N ⟦ σ ⟧ 〈 ρ 〉\<\\
\>           ∎)) --TODO Common pattern\<\\
\>           (let open ≡-Reasoning in \<\\
\>           begin\<\\
\>             P ⟦ (x₂:= L ,x₁:= L' ,x₀:= Q) •SR liftRep -Path (liftRep -Term (liftRep -Term ρ)) • liftSub -Path (liftSub -Term (liftSub -Term σ)) ⟧\<\\
\>           ≡⟨ sub-comp P ⟩\<\\
\>             P ⟦ liftSub -Path (liftSub -Term (liftSub -Term σ)) ⟧ ⟦ (x₂:= L ,x₁:= L' ,x₀:= Q) •SR liftRep -Path (liftRep -Term (liftRep -Term ρ)) ⟧\<\\
\>           ≡⟨ sub-compSR (P ⟦ liftSub -Path (liftSub -Term (liftSub -Term σ)) ⟧) ⟩\<\\
\>             P ⟦ liftSub -Path (liftSub -Term (liftSub -Term σ)) ⟧ 〈 liftRep -Path (liftRep -Term (liftRep -Term ρ)) 〉 ⟦ x₂:= L ,x₁:= L' ,x₀:= Q ⟧\<\\
\>           ∎) \<\\
\>         (Computable-Sub \<\\
\>           ((x₂:= L ,x₁:= L' ,x₀:= Q) •SR \<\\
\>             liftRep -Path (liftRep -Term (liftRep -Term ρ)) • \<\\
\>             liftSub -Path (liftSub -Term (liftSub -Term σ))) \<\\
\>           (compC (compSRC \<\\
\>                  (botsub₃C L∈EΘA L'∈EΘA Q∈EΘL≡L') \<\\
\>                  liftRepρ∶apΔ⇒RapΘ)\<\\
\>           (liftSubC (liftSubC (liftSubC σ∶Γ⇒Δ)))) \<\\
\>           ΓAAe⊢P∶Mx≡Ny \<\\
\>           validΘ))\<\\
\>       (app*R (E-typed L∈EΘA) (E-typed L'∈EΘA) \<\\
\>         (lllR (change-type (weakening ΔAAE⊢P∶Mx≡Ny (valid-addpath validΘ) \<\\
\>           liftRepρ∶apΔ⇒RapΘ) \<\\
\>         (cong₂ (λ x y → appT x (var x₂) ≡〈 B 〉 appT y (var x₁)) \<\\
\>           (liftRep-upRep₃ (M ⟦ σ ⟧)) (liftRep-upRep₃ (N ⟦ σ ⟧)))))\<\\
\>         (EE-typed Q∈EΘL≡L')) \<\\
\>       (βEkr (E-SN L∈EΘA) (E-SN L'∈EΘA) (E-SN Q∈EΘL≡L')))\<\\
\>Computable-Sub σ σ∶Γ⇒Δ (app*R Γ⊢N∶A Γ⊢N'∶A Γ⊢P∶M≡M' Γ⊢Q∶N≡N') validΔ = \<\\
\>  app*-EE (Computable-Sub σ σ∶Γ⇒Δ Γ⊢P∶M≡M' validΔ) (Computable-Sub σ σ∶Γ⇒Δ Γ⊢Q∶N≡N' validΔ) \<\\
\>  (Computable-Sub σ σ∶Γ⇒Δ Γ⊢N∶A validΔ) (Computable-Sub σ σ∶Γ⇒Δ Γ⊢N'∶A validΔ)\<\\
\>Computable-Sub σ σ∶Γ⇒Δ (convER Γ⊢P∶M≡N Γ⊢M'∶A Γ⊢N'∶A M≃M' N≃N') validΔ = \<\\
\>  conv-EE (Computable-Sub σ σ∶Γ⇒Δ Γ⊢P∶M≡N validΔ) \<\\
\>    (conv-sub M≃M') (conv-sub N≃N') \<\\
\>    (substitution Γ⊢M'∶A validΔ (subC-typed σ∶Γ⇒Δ)) (substitution Γ⊢N'∶A validΔ (subC-typed σ∶Γ⇒Δ))\<\\
\>--TODO Common pattern\<\\
\>\<\\
\>--TODO Rename\<\\
\>computable-path-substitution U V τ σ σ' Γ Δ .(var x) .\_ σ∶Γ⇒CΔ σ'∶Γ⇒CΔ τ∶σ∼σ' (varR x \_) \_ = \<\\
\>  τ∶σ∼σ' x\<\\
\>computable-path-substitution U V τ σ σ' Γ Δ .(app -bot out) .(ty Ω) σ∶Γ⇒CΔ σ'∶Γ⇒CΔ τ∶σ∼σ' (⊥R x) validΔ = ref-EE (⊥-E validΔ)\<\\
\>computable-path-substitution U V τ σ σ' Γ Δ \_ .(ty Ω) σ∶Γ⇒CΔ σ'∶Γ⇒CΔ τ∶σ∼σ' (⊃R Γ⊢φ∶Ω Γ⊢ψ∶Ω) validΔ = ⊃*-EE \<\\
\>  (computable-path-substitution U V τ σ σ' Γ Δ \_ (ty Ω) σ∶Γ⇒CΔ σ'∶Γ⇒CΔ τ∶σ∼σ' Γ⊢φ∶Ω validΔ) \<\\
\>  (computable-path-substitution U V τ σ σ' Γ Δ \_ (ty Ω) σ∶Γ⇒CΔ σ'∶Γ⇒CΔ τ∶σ∼σ' Γ⊢ψ∶Ω validΔ) \<\\
\>computable-path-substitution U V τ σ σ' Γ Δ \_ .(ty B) σ∶Γ⇒CΔ σ'∶Γ⇒CΔ τ∶σ∼σ' (appR \{N = N\} \{A\} \{B\} Γ⊢M∶A⇒B Γ⊢N∶A) validΔ = \<\\
\>  app*-EE \<\\
\>  (computable-path-substitution U V τ σ σ' Γ Δ \_ \_ σ∶Γ⇒CΔ σ'∶Γ⇒CΔ τ∶σ∼σ' Γ⊢M∶A⇒B validΔ) \<\\
\>  (computable-path-substitution U V τ σ σ' Γ Δ \_ \_ σ∶Γ⇒CΔ σ'∶Γ⇒CΔ τ∶σ∼σ' Γ⊢N∶A validΔ)\<\\
\>  (Computable-Sub σ (pathsubC-valid₁ \{U\} \{V\} \{τ\} \{σ\} \{σ'\} τ∶σ∼σ') Γ⊢N∶A validΔ)\<\\
\>  (Computable-Sub σ' (pathsubC-valid₂ \{τ = τ\} \{σ\} \{σ = σ'\} \{Γ\} \{Δ\} τ∶σ∼σ') Γ⊢N∶A validΔ)\<\\
\>computable-path-substitution U V τ σ σ' Γ Δ .\_ .\_ σ∶Γ⇒CΔ σ'∶Γ⇒CΔ τ∶σ∼σ' (ΛR \{A = A\} \{M = M\} \{B = B\} Γ,A⊢M∶B) validΔ = \<\\
\>  let validΔA : valid (Δ ,T A)\<\\
\>      validΔA = ctxTR validΔ in\<\\
\>  let validΔAA : valid (Δ ,T A ,T A)\<\\
\>      validΔAA = ctxTR validΔA in\<\\
\>  let validapΔ : valid (addpath Δ A)\<\\
\>      validapΔ = valid-addpath validΔ in\<\\
\>  let σ∶Γ⇒Δ : σ ∶ Γ ⇒ Δ\<\\
\>      σ∶Γ⇒Δ = subC-typed σ∶Γ⇒CΔ in\<\\
\>  let σ'∶Γ⇒Δ : σ' ∶ Γ ⇒ Δ\<\\
\>      σ'∶Γ⇒Δ = subC-typed σ'∶Γ⇒CΔ in\<\\
\>  let σ↑∶ΓA⇒ΔA : liftSub -Term σ ∶ Γ ,T A ⇒ Δ ,T A\<\\
\>      σ↑∶ΓA⇒ΔA = liftSub-typed σ∶Γ⇒Δ in\<\\
\>  let σ'↑∶ΓA⇒ΔA : liftSub -Term σ' ∶ Γ ,T A ⇒ Δ ,T A\<\\
\>      σ'↑∶ΓA⇒ΔA = liftSub-typed σ'∶Γ⇒Δ in\<\\
\>  let ΔA⊢Mσ∶B : Δ ,T A ⊢ M ⟦ liftSub -Term σ ⟧ ∶ ty B\<\\
\>      ΔA⊢Mσ∶B = substitution Γ,A⊢M∶B validΔA σ↑∶ΓA⇒ΔA in\<\\
\>  let ΔA⊢Mσ'∶B : Δ ,T A ⊢ M ⟦ liftSub -Term σ' ⟧ ∶ ty B\<\\
\>      ΔA⊢Mσ'∶B = substitution Γ,A⊢M∶B validΔA σ'↑∶ΓA⇒ΔA in\<\\
\>  let τ↑∶σ↖∼σ'↗ : liftPathSub τ ∶ sub↖ σ ∼ sub↗ σ' ∶ Γ ,T A ⇒ addpath Δ A\<\\
\>      τ↑∶σ↖∼σ'↗ = liftPathSub-typed (pathsubC-typed σ σ' τ∶σ∼σ') validΔ in\<\\
\>  let σ↖∶ΓA⇒apΔ : sub↖ σ ∶ Γ ,T A ⇒ addpath Δ A\<\\
\>      σ↖∶ΓA⇒apΔ = sub↖-typed σ∶Γ⇒Δ in\<\\
\>  let σ'↗∶ΓA⇒apΔ : sub↗ σ' ∶ Γ ,T A ⇒ addpath Δ A\<\\
\>      σ'↗∶ΓA⇒apΔ = sub↗-typed σ'∶Γ⇒Δ in\<\\
\>  let Mτ∶Mσ∼Mσ' : addpath Δ A ⊢ M ⟦⟦ liftPathSub τ ∶ sub↖ σ ∼ sub↗ σ' ⟧⟧ ∶ M ⟦ sub↖ σ ⟧ ≡〈 B 〉 M ⟦ sub↗ σ' ⟧\<\\
\>      Mτ∶Mσ∼Mσ' = path-substitution Γ,A⊢M∶B τ↑∶σ↖∼σ'↗ σ↖∶ΓA⇒apΔ σ'↗∶ΓA⇒apΔ (valid-addpath validΔ) in\<\\
\>  func-EE \<\\
\>  (lllR (convER Mτ∶Mσ∼Mσ'\<\\
\>    (appR \<\\
\>      (ΛR (weakening (weakening (weakening ΔA⊢Mσ∶B\<\\
\>                                           validΔAA (liftRep-typed upRep-typed)) \<\\
\>                                (ctxTR validΔAA) (liftRep-typed upRep-typed)) \<\\
\>                     (ctxTR validapΔ) (liftRep-typed upRep-typed))) \<\\
\>      (varR x₂ validapΔ))\<\\
\>    (appR (ΛR (weakening (weakening (weakening ΔA⊢Mσ'∶B\<\\
\>                                               validΔAA (liftRep-typed upRep-typed)) \<\\
\>                                    (ctxTR validΔAA) (liftRep-typed upRep-typed)) \<\\
\>                         (ctxTR validapΔ) (liftRep-typed upRep-typed))) \<\\
\>              (varR x₁ validapΔ)) \<\\
\>    (sym-conv (osr-conv (redex (subst\<\\
\>                                  (R -appTerm\<\\
\>                                   (ΛT A\<\\
\>                                    ((((M ⟦ liftSub -Term σ ⟧) 〈 liftRep -Term upRep 〉) 〈 liftRep -Term upRep 〉)\<\\
\>                                     〈 liftRep -Term upRep 〉)\<\\
\>                                    ,, var x₂ ,, out))\<\\
\>                                  (sub↖-decomp M) βT)))) \<\\
\>    (sym-conv (osr-conv (redex (subst\<\\
\>                                  (R -appTerm\<\\
\>                                   (ΛT A\<\\
\>                                    ((((M ⟦ liftSub -Term σ' ⟧) 〈 liftRep -Term upRep 〉) 〈 liftRep -Term upRep\<\\
\>                                      〉)\<\\
\>                                     〈 liftRep -Term upRep 〉)\<\\
\>                                    ,, var x₁ ,, out))\<\\
\>                                  (sub↗-decomp M) βT)))))) \<\\
\>   (λ W Θ N N' Q ρ ρ∶Δ⇒RΘ N∈EΘA N'∈EΘA Q∈EΘN≡N' → \<\\
\>   let validΘ : valid Θ\<\\
\>       validΘ = context-validity (E.typed N∈EΘA) in\<\\
\>   let validΘA : valid (Θ ,T A)\<\\
\>       validΘA = ctxTR validΘ in\<\\
\>   let validΘAA : valid (Θ ,T A ,T A)\<\\
\>       validΘAA = ctxTR validΘA in\<\\
\>   let validapΘ : valid (addpath Θ A)\<\\
\>       validapΘ = valid-addpath validΘ in\<\\
\>   let ρ↑∶ΔA⇒ΘA : liftRep -Term ρ ∶ Δ ,T A ⇒R Θ ,T A\<\\
\>       ρ↑∶ΔA⇒ΘA = liftRep-typed ρ∶Δ⇒RΘ in\<\\
\>   let Θ⊢ΛM∶A⇛B : Θ ⊢ ΛT A (M ⟦ liftSub -Term σ ⟧ 〈 liftRep -Term ρ 〉) ∶ ty (A ⇛ B)\<\\
\>       Θ⊢ΛM∶A⇛B = ΛR (weakening ΔA⊢Mσ∶B validΘA ρ↑∶ΔA⇒ΘA) in\<\\
\>   let Θ⊢N∶A : Θ ⊢ N ∶ ty A\<\\
\>       Θ⊢N∶A = E-typed N∈EΘA in\<\\
\>   let Θ⊢N'∶A : Θ ⊢ N' ∶ ty A\<\\
\>       Θ⊢N'∶A = E-typed N'∈EΘA in\<\\
\>   let ΘA⊢Mσ'∶B : Θ ,T A ⊢ M ⟦ liftSub -Term σ' ⟧ 〈 liftRep -Term ρ 〉 ∶ ty B\<\\
\>       ΘA⊢Mσ'∶B = weakening ΔA⊢Mσ'∶B validΘA ρ↑∶ΔA⇒ΘA in\<\\
\>   expand-EE (conv-EE\<\\
\>     (computable-path-substitution (U , -Term) W (extendPS (ρ •RP τ) Q) (extendSub (ρ •RS σ) N) (extendSub (ρ •RS σ') N') (Γ ,T A) Θ M (ty B) \<\\
\>     (extendSubC (compRSC ρ∶Δ⇒RΘ σ∶Γ⇒CΔ) N∈EΘA) \<\\
\>     (extendSubC (compRSC ρ∶Δ⇒RΘ σ'∶Γ⇒CΔ) N'∈EΘA) \<\\
\>     (extendPSC (compRPC \{σ = σ\} \{σ' = σ'\} τ∶σ∼σ' ρ∶Δ⇒RΘ) Q∈EΘN≡N') Γ,A⊢M∶B validΘ)\<\\
\>     (sym-conv (osr-conv (redex \<\\
\>       (subst\<\\
\>          (R -appTerm (ΛT A ((M ⟦ liftSub \_ σ ⟧) 〈 liftRep \_ ρ 〉) ,, N ,, out)) \<\\
\>          (let open ≡-Reasoning in \<\\
\>          begin\<\\
\>            M ⟦ liftSub -Term σ ⟧ 〈 liftRep -Term ρ 〉 ⟦ x₀:= N ⟧\<\\
\>          ≡⟨⟨ sub-congl (sub-compRS M) ⟩⟩\<\\
\>            M ⟦ liftRep -Term ρ •RS liftSub -Term σ ⟧ ⟦ x₀:= N ⟧\<\\
\>          ≡⟨⟨ sub-congl (sub-congr liftSub-compRS M) ⟩⟩\<\\
\>            M ⟦ liftSub -Term (ρ •RS σ) ⟧ ⟦ x₀:= N ⟧\<\\
\>          ≡⟨ extendSub-decomp M ⟩\<\\
\>            M ⟦ extendSub (ρ •RS σ) N ⟧\<\\
\>          ∎)\<\\
\>          βT)))) \<\\
\>     (sym-conv (osr-conv (redex \<\\
\>       (subst\<\\
\>          (R -appTerm (ΛT A ((M ⟦ liftSub \_ σ' ⟧) 〈 liftRep \_ ρ 〉) ,, N' ,, out)) \<\\
\>          (let open ≡-Reasoning in \<\\
\>          begin\<\\
\>            M ⟦ liftSub -Term σ' ⟧ 〈 liftRep -Term ρ 〉 ⟦ x₀:= N' ⟧\<\\
\>          ≡⟨⟨ sub-congl (sub-compRS M) ⟩⟩\<\\
\>            M ⟦ liftRep -Term ρ •RS liftSub -Term σ' ⟧ ⟦ x₀:= N' ⟧\<\\
\>          ≡⟨⟨ sub-congl (sub-congr liftSub-compRS M) ⟩⟩\<\\
\>            M ⟦ liftSub -Term (ρ •RS σ') ⟧ ⟦ x₀:= N' ⟧\<\\
\>          ≡⟨ extendSub-decomp M ⟩\<\\
\>            M ⟦ extendSub (ρ •RS σ') N' ⟧\<\\
\>          ∎)\<\\
\>          βT)))) \<\\
\>     (appR Θ⊢ΛM∶A⇛B Θ⊢N∶A) \<\\
\>     (appR (ΛR ΘA⊢Mσ'∶B) Θ⊢N'∶A)) \<\\
\>     (app*R Θ⊢N∶A Θ⊢N'∶A (lllR (convER \<\\
\>       (weakening Mτ∶Mσ∼Mσ' (valid-addpath validΘ) (liftRep-typed (liftRep-typed (liftRep-typed ρ∶Δ⇒RΘ)))) \<\\
\>       (appR (weakening (weakening (weakening Θ⊢ΛM∶A⇛B validΘA upRep-typed) validΘAA upRep-typed) validapΘ upRep-typed) \<\\
\>         (varR x₂ validapΘ)) \<\\
\>       (appR (ΛR (weakening (weakening (weakening ΘA⊢Mσ'∶B validΘAA (liftRep-typed upRep-typed)) (ctxTR validΘAA) (liftRep-typed upRep-typed)) (ctxTR validapΘ) \<\\
\>         (liftRep-typed upRep-typed))) (varR x₁ validapΘ)) \<\\
\>       (sym-conv (subst (\_≃\_ (appT (ΛT A (M ⟦ liftSub -Term σ ⟧ 〈 liftRep -Term ρ 〉) ⇑ ⇑ ⇑) (var x₂))) \<\\
\>         (let open ≡-Reasoning in \<\\
\>         begin\<\\
\>           M ⟦ liftSub -Term σ ⟧ 〈 liftRep -Term ρ 〉 〈 liftRep -Term upRep 〉 〈 liftRep -Term upRep 〉 〈 liftRep -Term upRep 〉 ⟦ x₀:= var x₂ ⟧\<\\
\>         ≡⟨⟨ sub-congl (rep-comp₄ (M ⟦ liftSub -Term σ ⟧)) ⟩⟩\<\\
\>           M ⟦ liftSub -Term σ ⟧ 〈 liftRep -Term upRep •R liftRep -Term upRep •R liftRep -Term upRep •R liftRep -Term ρ 〉 ⟦ x₀:= var x₂ ⟧\<\\
\>         ≡⟨⟨ sub-congl (rep-congr liftRep-comp₄ (M ⟦ liftSub -Term σ ⟧)) ⟩⟩\<\\
\>           M ⟦ liftSub -Term σ ⟧ 〈 liftRep -Term (upRep •R upRep •R upRep •R ρ) 〉 ⟦ x₀:= var x₂ ⟧ \<\\
\>         ≡⟨ sub-congl (rep-congr (liftRep-cong (liftRep-upRep₄' ρ)) (M ⟦ liftSub -Term σ ⟧)) ⟩\<\\
\>           M ⟦ liftSub -Term σ ⟧ 〈 liftRep -Term (liftRep -Path (liftRep -Term (liftRep -Term ρ)) •R upRep •R upRep •R upRep) 〉 ⟦ x₀:= var x₂ ⟧ \<\\
\>         ≡⟨ sub-congl (rep-congr liftRep-comp₄ (M ⟦ liftSub -Term σ ⟧) ) ⟩\<\\
\>           M ⟦ liftSub -Term σ ⟧ 〈 liftRep -Term (liftRep -Path (liftRep -Term (liftRep -Term ρ))) •R liftRep -Term upRep •R liftRep -Term upRep •R liftRep -Term upRep 〉 ⟦ x₀:= var x₂ ⟧ \<\\
\>         ≡⟨ sub-congl (rep-comp₄ (M ⟦ liftSub -Term σ ⟧)) ⟩\<\\
\>           M ⟦ liftSub -Term σ ⟧ 〈 liftRep \_ upRep 〉 〈 liftRep \_ upRep 〉 〈 liftRep \_ upRep 〉 〈 liftRep -Term (liftRep -Path (liftRep -Term (liftRep -Term ρ))) 〉 ⟦ x₀:= var x₂ ⟧\<\\
\>         ≡⟨⟨ compRS-botSub (M ⟦ liftSub -Term σ ⟧ 〈 liftRep \_ upRep 〉 〈 liftRep \_ upRep 〉 〈 liftRep \_ upRep 〉) ⟩⟩\<\\
\>           M ⟦ liftSub -Term σ ⟧ 〈 liftRep \_ upRep 〉 〈 liftRep \_ upRep 〉 〈 liftRep \_ upRep 〉 ⟦ x₀:= var x₂ ⟧ 〈 liftRep -Path (liftRep -Term (liftRep -Term ρ)) 〉\<\\
\>         ≡⟨ rep-congl (sub↖-decomp M) ⟩\<\\
\>           M ⟦ sub↖ σ ⟧ 〈 liftRep -Path (liftRep -Term (liftRep -Term ρ)) 〉\<\\
\>         ∎) \<\\
\>         (osr-conv (redex βT)))) \<\\
\>         (sym-conv (subst (\_≃\_ (appT (ΛT A (M ⟦ liftSub -Term σ' ⟧ 〈 liftRep -Term ρ 〉) ⇑ ⇑ ⇑) (var x₁))) \<\\
\>         (let open ≡-Reasoning in \<\\
\>         begin\<\\
\>           M ⟦ liftSub -Term σ' ⟧ 〈 liftRep -Term ρ 〉 〈 liftRep -Term upRep 〉 〈 liftRep -Term upRep 〉 〈 liftRep -Term upRep 〉 ⟦ x₀:= var x₁ ⟧\<\\
\>         ≡⟨⟨ sub-congl (rep-comp₄ (M ⟦ liftSub -Term σ' ⟧)) ⟩⟩\<\\
\>           M ⟦ liftSub -Term σ' ⟧ 〈 liftRep -Term upRep •R liftRep -Term upRep •R liftRep -Term upRep •R liftRep -Term ρ 〉 ⟦ x₀:= var x₁ ⟧\<\\
\>         ≡⟨⟨ sub-congl (rep-congr liftRep-comp₄ (M ⟦ liftSub -Term σ' ⟧)) ⟩⟩\<\\
\>           M ⟦ liftSub -Term σ' ⟧ 〈 liftRep -Term (upRep •R upRep •R upRep •R ρ) 〉 ⟦ x₀:= var x₁ ⟧ \<\\
\>         ≡⟨ sub-congl (rep-congr (liftRep-cong (liftRep-upRep₄' ρ)) (M ⟦ liftSub -Term σ' ⟧)) ⟩\<\\
\>           M ⟦ liftSub -Term σ' ⟧ 〈 liftRep -Term (liftRep -Path (liftRep -Term (liftRep -Term ρ)) •R upRep •R upRep •R upRep) 〉 ⟦ x₀:= var x₁ ⟧ \<\\
\>         ≡⟨ sub-congl (rep-congr liftRep-comp₄ (M ⟦ liftSub -Term σ' ⟧)) ⟩\<\\
\>           M ⟦ liftSub -Term σ' ⟧ 〈 liftRep -Term (liftRep -Path (liftRep -Term (liftRep -Term ρ))) •R liftRep -Term upRep •R liftRep -Term upRep •R liftRep -Term upRep 〉 ⟦ x₀:= var x₁ ⟧ \<\\
\>         ≡⟨ sub-congl (rep-comp₄ (M ⟦ liftSub -Term σ' ⟧)) ⟩\<\\
\>           M ⟦ liftSub -Term σ' ⟧ 〈 liftRep \_ upRep 〉 〈 liftRep \_ upRep 〉 〈 liftRep \_ upRep 〉 〈 liftRep -Term (liftRep -Path (liftRep -Term (liftRep -Term ρ))) 〉 ⟦ x₀:= var x₁ ⟧\<\\
\>         ≡⟨⟨ compRS-botSub (M ⟦ liftSub -Term σ' ⟧ 〈 liftRep \_ upRep 〉 〈 liftRep \_ upRep 〉 〈 liftRep \_ upRep 〉) ⟩⟩\<\\
\>           M ⟦ liftSub -Term σ' ⟧ 〈 liftRep \_ upRep 〉 〈 liftRep \_ upRep 〉 〈 liftRep \_ upRep 〉 ⟦ x₀:= var x₁ ⟧ 〈 liftRep -Path (liftRep -Term (liftRep -Term ρ)) 〉\<\\
\>         ≡⟨ rep-congl (sub↗-decomp M) ⟩\<\\
\>           M ⟦ sub↗ σ' ⟧ 〈 liftRep -Path (liftRep -Term (liftRep -Term ρ)) 〉\<\\
\>         ∎) \<\\
\>         (osr-conv (redex βT)))))) --TODO Duplication\<\\
\>         (EE-typed Q∈EΘN≡N')) \<\\
\>     (subst (key-redex \_) \<\\
\>     (let open ≡-Reasoning in \<\\
\>     begin\<\\
\>       M ⟦⟦ liftPathSub τ ∶ sub↖ σ ∼ sub↗ σ' ⟧⟧ 〈 liftRep -Path (liftRep -Term (liftRep -Term ρ)) 〉 ⟦ x₂:= N ,x₁:= N' ,x₀:= Q ⟧\<\\
\>     ≡⟨⟨ sub-congl (pathsub-compRP M) ⟩⟩\<\\
\>       M ⟦⟦ liftRep -Path (liftRep -Term (liftRep -Term ρ)) •RP liftPathSub τ ∶ liftRep -Path (liftRep -Term (liftRep -Term ρ)) •RS sub↖ σ ∼ liftRep -Path (liftRep -Term (liftRep -Term ρ)) •RS sub↗ σ' ⟧⟧ ⟦ x₂:= N ,x₁:= N' ,x₀:= Q ⟧\<\\
\>     ≡⟨⟨ sub-congl (pathsub-cong M liftPathSub-compRP sub↖-compRP sub↗-compRP) ⟩⟩\<\\
\>       M ⟦⟦ liftPathSub (ρ •RP τ) ∶ sub↖ (ρ •RS σ) ∼ sub↗ (ρ •RS σ') ⟧⟧ ⟦ x₂:= N ,x₁:= N' ,x₀:= Q ⟧\<\\
\>     ≡⟨⟨ pathsub-extendPS M ⟩⟩\<\\
\>       M ⟦⟦ extendPS (ρ •RP τ) Q ∶ extendSub (ρ •RS σ) N ∼ extendSub (ρ •RS σ') N' ⟧⟧\<\\
\>     ∎) (βEkr (E-SN N∈EΘA) (E-SN N'∈EΘA) (E-SN Q∈EΘN≡N')))) -\}}\<%
\end{code}
}

\item
$$ \infer{\Gamma \vdash \phi \supset \psi : \Omega}{\Gamma \vdash \phi : \Omega \quad \Gamma \vdash \psi : \Omega} $$

By the induction hypothesis, $\phi[\sigma], \psi[\sigma] \in \SN$, hence $\phi[\sigma] \supset \psi[\sigma] \in \SN$.

Also by the induction hypothesis, we have $\phi[\sigma]\{\} \in E_\Delta(\phi[\sigma] =_\Omega \phi[\sigma])$, and
$\psi[\sigma]\{\} \in E_\Delta(\psi[\sigma] =_\Omega \psi[\sigma])$.  Therefore, $\phi[\sigma]\{\} \supset^* \psi[\sigma]\{\}
\in E_\Delta(\phi[\sigma] \supset \psi[\sigma] =_\Omega \phi[\sigma] \supset \psi[\sigma])$ by Lemma \ref{lm:Esupset}.
\item
$$ \infer{\Gamma \vdash M N : B} {\Gamma \vdash M : A \rightarrow B \quad \Gamma \vdash N : A} $$

\begin{enumerate}
\item[1]
We have $M[\sigma] \in E_\Delta(A \rightarrow B)$ and $N[\sigma] \in E_\Delta(A)$, so $M[\sigma] N[\sigma] \in E_\Delta(B)$.
\item[4]
We have $M\{\tau\} \in E_\Delta(M [ \rho ] =_{A \rightarrow B} M [ \sigma ])$ and $N[\rho], N[\sigma] \in E_\Delta(A)$,
$N \{ \tau \} \in E_\Delta(N[ \rho ] =_A N[\sigma])$ by the induction hypothesis (1) and (4).  Therefore,
$M \{ \tau \}_{N[\rho] N[\sigma]} N \{ \tau \} \in E_\Delta(M[\rho] N[\rho] =_B M[\sigma] N[\sigma])$.
\end{enumerate}
\item
$$\infer{\Gamma \vdash \delta \epsilon : \psi} {\Gamma \vdash \delta : \phi \supset \psi \quad \Gamma \vdash \epsilon : \phi}$$

We have $\delta [ \sigma ] \in E_\Delta(\phi [ \sigma ] \supset \psi [ \sigma ])$ and $\epsilon [ \sigma ] \in E_\Delta(\phi [ \sigma ])$,
hence $\delta [ \sigma ] \epsilon [\sigma] \in E_\Delta(\psi [ \sigma ])$.
\item
$$ \infer{\Gamma \vdash \lambda x:A.M : A \rightarrow B}{\Gamma, x : A \vdash M : B}$$

\begin{enumerate}
\item[1]
\begin{itemize}
\item
Let $\Theta \supseteq \Delta$ and $N \in E_\Theta(A)$.  We must show that $(\lambda x:A.M[\sigma])N \in E_\Theta(B)$.

We have that $(\sigma, x:=N) : (\Gamma, x : A) \rightarrow \Theta$ is computable, and so the induction hypothesis gives
$M[\sigma, x:=N] \in E_\Theta(B)$.  The result follows by Lemma \ref{lm:wte}.\ref{lm:wteT}.
\item
We must show that $\triplelambda e:x =_A y. M [ \sigma ] \{ x := e : x \sim y \} \equiv
\triplelambda e:x =_A y. M \{ z_1 := \sigma(z_1) \{ \}, \ldots, z_n := \sigma(z_n)\{\}, x := e \} \in E_\Delta(\lambda x:A.M[\sigma] =_{A \rightarrow B} \lambda x:A.M[\sigma])$.

So let $\Theta \supseteq \Delta$ and $N, N' \in E_\Theta(A)$, $P \in E_\Theta(N =_A N')$.  Then $(z_1 := \sigma(z_1)\{\}, \ldots, z_n := \sigma(z_n)\{\}, x := P) : (\sigma, x:=N) \sim (\sigma, x:=N') : (\Gamma, x:A) \rightarrow \Theta$
is computable, and so the induction hypothesis gives
\[ M \{ z_i := \sigma(z_i) \{\}, x := P \} \in E_\Theta(M [ \sigma, x:=N] =_B M [\sigma, x:=N']) \enspace . \]
Therefore, by Lemma \ref{lm:wte}.\ref{lm:wteE}, we have that $(\triplelambda e:x =_A y. M \{ z_i := \sigma(z_i) \{\}, x := e \})_{N N'} P \in E_\Theta(M[\sigma, x:=N] =_B M[\sigma, x:=N'])$.

Hence Lemma \ref{lm:conv-compute} gives $(\triplelambda e:x =_A y. M \{ z_i := \sigma(z_i) \{\}, x := e \})_{N N'} P \in E_\Theta((\lambda x:A.M[\sigma])N =_B (\lambda x:A.M[\sigma])N')$
as required.
\end{itemize}
\item[4]
Let $\Theta \supseteq \Delta$ and $N, N' \in E_\Theta(A)$, $P \in E_\Theta(N =_A N')$.  Then $(\tau, x:=P) : (\rho, x:=N) \sim (\sigma, x:=N') : (\Gamma, x :A) \rightarrow \Delta$ is computable,
and so the induction hypothesis gives
\[ M \{ \tau, x := P \} \in E_\Theta(M[\rho, x:=N] =_B M[\sigma, x:=N']) \enspace . \]
By Lemma \ref{lm:conv-compute},
\[ M \{ \tau, x := P \} \in E_\Theta((\lambda x:A.M[\rho]) N =_B (\lambda x:A.M[\sigma]) N') \]
and so Lemma \ref{lm:wte}.\ref{lm:wteE} gives
\[ (\triplelambda e:x=_A y.M \{ \tau, x:=e \})_{N N'} P \in E_\Theta((\lambda x:A.M[\rho]) N =_B (\lambda x:A.M[\sigma]) N') \]
as required.
\end{enumerate}
\item
$$\infer{\Gamma \vdash \lambda p : \phi . \delta : \phi \supset \psi}{\Gamma, p : \phi \vdash \delta : \psi}$$

Let $\Theta \supseteq \Delta$ and $\epsilon \in E_\Theta(\phi[\sigma])$.  Then $(\sigma, p:=\epsilon) : (\Gamma, p : \phi) \rightarrow \Theta$
is computable, and so the induction hypothesis gives
\[ \delta[\sigma, p:=\epsilon] \in E_\Theta(\psi[\sigma)) \enspace . \]
Hence by Lemma \ref{lm:wte}.\ref{lm:wteP}, we have $(\lambda p:\phi[\sigma].\delta[\sigma]) \epsilon \in E_\Theta(\psi[\sigma])$, as required.
\item
$$ \infer[(\phi \simeq \psi)]{\Gamma \vdash \delta : \psi}{\Gamma \vdash \delta : \phi \quad \Gamma \vdash \psi : \Omega} $$

We have that $\delta[\sigma] \in E_\Gamma(\phi[\sigma])$ by induction hypothesis, and so $\delta[\sigma] \in E_\Gamma(\psi[\sigma])$ by
Lemma \ref{lm:conv-compute}.
\item
$$ \infer[(e : M =_A N \in \Gamma)]{\Gamma \vdash e : M =_A N}{\Gamma \vald} $$

We have $\sigma(e) \in E_\Gamma(M[\sigma] =_A N[\sigma])$ by hypothesis.
\item
$$ \infer{\Gamma \vdash \reff{M} : M =_A M}{\Gamma \vdash M : A} $$

This case holds by Lemma \ref{lm:Eref}.
\item
$$ \infer{\Gamma \vdash P \supset^* Q : \phi \supset \psi =_\Omega \phi' \supset \psi'}{\Gamma \vdash P : \phi =_\Omega \phi' \quad \Gamma \vdash Q : \psi =_\Omega \psi'} $$

This case holds by Lemma \ref{lm:Esupset}.

\item
$$ \infer{\Gamma \vdash \univ{\phi}{\psi}{\delta}{\epsilon} : \phi =_\Omega \psi}{\Gamma \vdash \delta : \phi \supset \psi \quad \Gamma \vdash \epsilon : \psi \supset \phi} $$

This case holds by Lemma \ref{lm:Euniv}.
\item
$$ \infer{\Gamma \vdash P^+ : \phi \supset \psi}{\Gamma \vdash P : \phi =_\Omega \psi} $$

The induction hypothesis gives $P[\sigma] \in E_\Delta(\phi[\sigma] =_\Omega \psi[\sigma])$, and so immediately $P[\sigma]^+ \in E_\Delta(\phi[\sigma] \supset \psi[\sigma])$.
\item
$$ \infer{\Gamma \vdash P^- : \psi \supset \phi}{\Gamma \vdash P : \phi =_\Omega \psi} $$

The induction hypothesis gives $P[\sigma] \in E_\Delta(\phi[\sigma] =_\Omega \psi[\sigma])$, and so immediately $P[\sigma]^- \in E_\Delta(\psi[\sigma] \supset \phi[\sigma])$.
\item
$$ \infer{\Gamma \vdash \triplelambda e : x =_A y . P : M =_{A \rightarrow B} N}
  {\begin{array}{c}
     \Gamma, x : A, y : A, e : x =_A y \vdash P : M x =_B N y \\
     \Gamma \vdash M : A \rightarrow B \quad
\Gamma \vdash N : A \rightarrow B
     \end{array}} $$

Let $\Theta \supseteq \Delta$, $L, L' \in E_\Theta(A)$, and $Q \in E_\Theta(L =_A L')$.  We must show that
\[ (\triplelambda e:x =_A y. P [ \sigma ])_{L L'} Q \in E_\Theta(ML =_B NL') \enspace . \]
We have that $(\sigma, x:=L, y:=L', e:=Q) : (\Gamma, x : A, y : A, e : x =_A y) \rightarrow \Theta$ is computable, and so
the induction hypothesis gives
\[ P [ \sigma, x := L, y := L', e := Q ] \in E_\Theta(ML =_B NL') \enspace . \]
The result follow by Lemma \ref{lm:wte}.\ref{lm:wteE}.
\item
$$ \infer{\Gamma \vdash P_{NN'}Q : MN =_B M' N'}{\Gamma \vdash P : M =_{A \rightarrow B} M' \quad \Gamma \vdash Q : N =_A N' \quad \Gamma \vdash N : A \quad \Gamma \vdash N' : A}$$

The induction hypothesis gives $P[\sigma] \in E_\Delta(M[\sigma] =_{A \rightarrow B} M'[\sigma])$ and $N[\sigma] \in E_\Delta(A)$, $N'[\sigma] \in E_\Delta(A)$, $Q[\sigma] \in E_\Delta(N =_A N')$.
It follows immediately that $(P_{NN'}Q)[\sigma] \in E_\Delta(M[\sigma] N[\sigma] =_B M'[\sigma] N'[\sigma])$.
\item
$$ \infer[(M \simeq M', N \simeq N')]{\Gamma \vdash P : M' =_A N'}{\Gamma \vdash P : M =_A N \quad \Gamma \vdash M' : A \quad \Gamma \vdash N' : A} $$

The induction hypothesis gives $P[\sigma] \in E_\Delta(M[\sigma] =_A N[\sigma])$, hence $P[\sigma] \in E_\Delta(M'[\sigma] =_A N'[\sigma])$ by Lemma
\ref{lm:conv-compute}.
\end{itemize}
\end{proof}

\AgdaHide{
\begin{code}%
\>\AgdaComment{\{-computable-path-substitution .U V τ σ σ' .Γ Δ \_ \_ σ∶Γ⇒CΔ σ'∶Γ⇒CΔ τ∶σ∼σ' (ΛR \{U\} \{Γ\} \{A\} \{M\} \{B\} Γ,A⊢M∶B) validΔ = \<\\
\>  let validΔAA : valid (Δ ,T A ,T A)\<\\
\>      validΔAA = ctxTR (ctxTR validΔ) in\<\\
\>  let validΔAAE : valid (addpath Δ A)\<\\
\>      validΔAAE = ctxER (varR x₁ validΔAA) (varR x₀ validΔAA) in\<\\
\>  let σ∶Γ⇒Δ = subC-typed σ∶Γ⇒CΔ in\<\\
\>  let σ'∶Γ⇒Δ = subC-typed σ'∶Γ⇒CΔ in\<\\
\>  let sub↖σ-typed : sub↖ σ ∶ Γ ,T A ⇒ addpath Δ A\<\\
\>      sub↖σ-typed = sub↖-typed σ∶Γ⇒Δ in\<\\
\>  let sub↗σ'-typed : sub↗ σ' ∶ Γ ,T A ⇒ addpath Δ A\<\\
\>      sub↗σ'-typed = sub↗-typed σ'∶Γ⇒Δ in\<\\
\>  func-EE (lllR (convER (Path-substitution Γ,A⊢M∶B\<\\
\>                             (liftPathSub-typed (pathsubC-typed \{τ = τ\} \{σ\} \{σ = σ'\} \{Γ\} \{Δ\} τ∶σ∼σ')) sub↖σ-typed sub↗σ'-typed\<\\
\>                             validΔAAE)\<\\
\>                             (appR (ΛR (weakening \{ρ = liftRep \_ upRep\}\<\\
\>                                           \{M = ((M ⟦ liftSub \_ σ ⟧) 〈 liftRep \_ upRep 〉) 〈 liftRep \_ upRep 〉\} \<\\
\>                                        (weakening \{ρ = liftRep \_ upRep\}\<\\
\>                                           \{M = (M ⟦ liftSub \_ σ ⟧) 〈 liftRep \_ upRep 〉\} \<\\
\>                                        (weakening \{ρ = liftRep \_ upRep\} \{M = M ⟦ liftSub \_ σ ⟧\} \<\\
\>                                        (substitution \{σ = liftSub -Term σ\} \{M = M\} Γ,A⊢M∶B (ctxTR validΔ) \<\\
\>                                          (liftSub-typed (subC-typed σ∶Γ⇒CΔ))) (ctxTR (ctxTR validΔ)) (liftRep-typed upRep-typed)) \<\\
\>                                        (ctxTR (ctxTR (ctxTR validΔ))) \<\\
\>                                        (liftRep-typed upRep-typed)) \<\\
\>                           (ctxTR (ctxER (varR (↑ x₀) (ctxTR (ctxTR validΔ))) (varR x₀ (ctxTR (ctxTR validΔ))))) \<\\
\>                           (liftRep-typed upRep-typed))) \<\\
\>                           (varR x₂ (ctxER ((varR (↑ x₀) (ctxTR (ctxTR validΔ)))) (varR x₀ (ctxTR (ctxTR validΔ))))))  \<\\
\>                              (let stepA : addpath Δ A ,T A ⊢ M ⟦ liftSub \_ σ' ⟧ 〈 liftRep \_ upRep 〉 〈 liftRep \_ upRep 〉 〈 liftRep \_ upRep 〉 ∶ ty B\<\\
\>                                   stepA = weakening \{U = V , -Term , -Term , -Term\} \{V = V , -Term , -Term , -Path , -Term\} \{ρ = liftRep \_ upRep\} \{Γ = Δ , ty A , ty A , ty A\} \{M = M ⟦ liftSub \_ σ' ⟧ 〈 liftRep \_ upRep 〉 〈 liftRep \_ upRep 〉\} \<\\
\>                                      (weakening \{ρ = liftRep \_ upRep\} \{Γ = Δ , ty A , ty A\}\<\\
\>                                         \{M = (M ⟦ liftSub \_ σ' ⟧) 〈 liftRep \_ upRep 〉\} \<\\
\>                                      (weakening \{ρ = liftRep \_ upRep\} \{M = M ⟦ liftSub \_ σ' ⟧\} \<\\
\>                                      (substitution \{σ = liftSub \_ σ'\} \{M = M\} \<\\
\>                                      Γ,A⊢M∶B \<\\
\>                                      (ctxTR validΔ) \<\\
\>                                      (liftSub-typed σ'∶Γ⇒Δ))\<\\
\>                                      validΔAA \<\\
\>                                      (liftRep-typed upRep-typed)) \<\\
\>                                      (ctxTR validΔAA) \<\\
\>                                      (liftRep-typed upRep-typed))\<\\
\>                                      (ctxTR validΔAAE)\<\\
\>                                      (liftRep-typed upRep-typed) in\<\\
\>                              let stepB : addpath Δ A ⊢ (ΛT A M) ⟦ σ' ⟧ ⇑ ⇑ ⇑ ∶ ty (A ⇛ B)\<\\
\>                                  stepB = ΛR stepA in\<\\
\>                              let stepC : addpath Δ A ⊢ var x₁ ∶ ty A\<\\
\>                                  stepC = varR x₁ validΔAAE in\<\\
\>                              let stepD : addpath Δ A ⊢ appT ((ΛT A M) ⟦ σ' ⟧ ⇑ ⇑ ⇑) (var x₁) ∶ ty B\<\\
\>                                  stepD = appR stepB stepC in\<\\
\>                              stepD)\<\\
\>                        (sym-conv (osr-conv (subst (λ a → appT ((ΛT A M ⟦ σ ⟧) ⇑ ⇑ ⇑) (var x₂) ⇒ a) (let open ≡-Reasoning in\<\\
\>                           M ⟦ liftSub \_ σ ⟧ 〈 liftRep \_ upRep 〉 〈 liftRep \_ upRep 〉 〈 liftRep \_ upRep 〉 ⟦ x₀:= (var x₂) ⟧\<\\
\>                         ≡⟨⟨ sub-compSR (M ⟦ liftSub \_ σ ⟧ 〈 liftRep \_ upRep 〉 〈 liftRep \_ upRep 〉) ⟩⟩\<\\
\>                           M ⟦ liftSub \_ σ ⟧ 〈 liftRep \_ upRep 〉 〈 liftRep \_ upRep 〉 ⟦ x₀:= (var x₂) •SR liftRep \_ upRep ⟧\<\\
\>                         ≡⟨⟨ sub-compSR (M ⟦ liftSub \_ σ ⟧ 〈 liftRep \_ upRep 〉) ⟩⟩\<\\
\>                           M ⟦ liftSub \_ σ ⟧ 〈 liftRep \_ upRep 〉 ⟦ x₀:= (var x₂) •SR liftRep \_ upRep •SR liftRep \_ upRep ⟧\<\\
\>                         ≡⟨⟨ sub-compSR (M ⟦ liftSub \_ σ ⟧) ⟩⟩\<\\
\>                           M ⟦ liftSub \_ σ ⟧ ⟦ x₀:= (var x₂) •SR liftRep \_ upRep •SR liftRep \_ upRep •SR liftRep \_ upRep ⟧\<\\
\>                         ≡⟨⟨ sub-comp M ⟩⟩\<\\
\>                           M ⟦ x₀:= (var x₂) •SR liftRep \_ upRep •SR liftRep \_ upRep •SR liftRep \_ upRep • liftSub \_ σ ⟧\<\\
\>                         ≡⟨ sub-congr M aux₃ ⟩\<\\
\>                           M ⟦ sub↖ σ ⟧\<\\
\>                           ∎) (redex ?)))) \<\\
\>                         (sym-conv (osr-conv (subst (λ a → appT ((ΛT A M ⟦ σ' ⟧) ⇑ ⇑ ⇑) (var x₁) ⇒ a) \<\\
\>                         (let open ≡-Reasoning in\<\\
\>                           M ⟦ liftSub \_ σ' ⟧ 〈 liftRep \_ upRep 〉 〈 liftRep \_ upRep 〉 〈 liftRep \_ upRep 〉 ⟦ x₀:= (var x₁) ⟧\<\\
\>                         ≡⟨⟨ sub-compSR (M ⟦ liftSub \_ σ' ⟧ 〈 liftRep \_ upRep 〉 〈 liftRep \_ upRep 〉) ⟩⟩\<\\
\>                           M ⟦ liftSub \_ σ' ⟧ 〈 liftRep \_ upRep 〉 〈 liftRep \_ upRep 〉 ⟦ x₀:= (var x₁) •SR liftRep \_ upRep ⟧\<\\
\>                         ≡⟨⟨ sub-compSR (M ⟦ liftSub \_ σ' ⟧ 〈 liftRep \_ upRep 〉) ⟩⟩\<\\
\>                           M ⟦ liftSub \_ σ' ⟧ 〈 liftRep \_ upRep 〉 ⟦ x₀:= (var x₁) •SR liftRep \_ upRep •SR liftRep \_ upRep ⟧\<\\
\>                         ≡⟨⟨ sub-compSR (M ⟦ liftSub \_ σ' ⟧) ⟩⟩\<\\
\>                           M ⟦ liftSub \_ σ' ⟧ ⟦ x₀:= (var x₁) •SR liftRep \_ upRep •SR liftRep \_ upRep •SR liftRep \_ upRep ⟧\<\\
\>                         ≡⟨⟨ sub-comp M ⟩⟩\<\\
\>                           M ⟦ x₀:= (var x₁) •SR liftRep \_ upRep •SR liftRep \_ upRep •SR liftRep \_ upRep • liftSub \_ σ' ⟧\<\\
\>                         ≡⟨ sub-congr M aux₄ ⟩\<\\
\>                           M ⟦ sub↗ σ' ⟧\<\\
\>                           ∎) \<\\
\>                         (redex ?))))))\<\\
\>    (λ W Θ N N' Q ρ ρ∶Δ⇒Θ N∈EΘA N'∈EΘA Q∈EΘN≡N' → \<\\
\>    let validΘ : valid Θ\<\\
\>        validΘ = context-validity (E-typed N∈EΘA) in\<\\
\>    let σ₁ : Sub (U , -Term) W\<\\
\>        σ₁ = x₀:= N •SR liftRep -Term ρ • liftSub -Term σ in\<\\
\>    let σ₁∶Γ,A⇒Θ : σ₁ ∶ Γ ,T A ⇒C Θ\<\\
\>        σ₁∶Γ,A⇒Θ = compC (compSRC (botsubC N∈EΘA) (liftRep-typed ρ∶Δ⇒Θ)) (liftSubC σ∶Γ⇒CΔ) in\<\\
\>    let σ₂ : Sub (U , -Term) W\<\\
\>        σ₂ = x₀:= N' •SR liftRep -Term ρ • liftSub -Term σ' in\<\\
\>    let σ₂∶Γ,A⇒Θ : σ₂ ∶ Γ ,T A ⇒C Θ\<\\
\>        σ₂∶Γ,A⇒Θ = compC (compSRC (botsubC N'∈EΘA) (liftRep-typed ρ∶Δ⇒Θ)) (liftSubC σ'∶Γ⇒CΔ) in --REFACTOR Common pattern\<\\
\>    let ρ' = liftRep -Path (liftRep -Term (liftRep -Term ρ)) in\<\\
\>    let step1 : x₀:= N • liftSub -Term (ρ •RS σ) ∼ σ₁\<\\
\>        step1 = sub-trans (comp-congr liftSub-compRS) \<\\
\>                  (assocRSSR \{ρ = x₀:= N\} \{σ = liftRep -Term ρ\} \{τ = liftSub -Term σ\}) in\<\\
\>    let step2 : x₀:= N' • liftSub -Term (ρ •RS σ') ∼ σ₂\<\\
\>        step2 = sub-trans (comp-congr liftSub-compRS) \<\\
\>                  (assocRSSR \{ρ = x₀:= N'\} \{σ = liftRep -Term ρ\} \{τ = liftSub -Term σ'\}) in\<\\
\>    let sub-rep-comp : ∀ (σ : Sub U V) (N : Term W) → M ⟦ x₀:= N •SR liftRep \_ ρ • liftSub \_ σ ⟧ ≡ M ⟦ liftSub \_ σ ⟧ 〈 liftRep \_ ρ 〉 ⟦ x₀:= N ⟧\<\\
\>        sub-rep-comp σ N = let open ≡-Reasoning in\<\\
\>             begin\<\\
\>               M ⟦ x₀:= N •SR liftRep -Term ρ • liftSub -Term σ ⟧\<\\
\>             ≡⟨ sub-comp M ⟩\<\\
\>               M ⟦ liftSub -Term σ ⟧ ⟦ x₀:= N •SR liftRep -Term ρ ⟧\<\\
\>             ≡⟨ sub-compSR (M ⟦ liftSub -Term σ ⟧) ⟩\<\\
\>               M ⟦ liftSub -Term σ ⟧ 〈 liftRep -Term ρ 〉 ⟦ x₀:= N ⟧\<\\
\>             ∎ in\<\\
\>    let ih : EE Θ (M ⟦ σ₁ ⟧ ≡〈 B 〉 M ⟦ σ₂ ⟧) \<\\
\>                  (M ⟦⟦ extendPS (ρ •RP τ) Q ∶ σ₁ ∼ σ₂ ⟧⟧)\<\\
\>        ih = (computable-path-substitution (U , -Term) W (extendPS (ρ •RP τ) Q) σ₁ σ₂ (Γ ,T A) Θ \_ \_ σ₁∶Γ,A⇒Θ σ₂∶Γ,A⇒Θ\<\\
\>             (change-ends \{σ = x₀:= N' • liftSub -Term (ρ •RS σ')\} \{σ' = σ₂\} (extendPS-typedC (compRP-typedC \{ρ = ρ\} \{τ\} \{σ\} \{σ'\} τ∶σ∼σ' ρ∶Δ⇒Θ) \<\\
\>               Q∈EΘN≡N')\<\\
\>                 step1 step2) Γ,A⊢M∶B validΘ) in\<\\
\>    let Δ,A⊢Mσ∶B : Δ ,T A ⊢ M ⟦ liftSub \_ σ ⟧ ∶ ty B\<\\
\>        Δ,A⊢Mσ∶B = substitution Γ,A⊢M∶B (ctxTR validΔ) (liftSub-typed σ∶Γ⇒Δ) in\<\\
\>    let Δ,A⊢Mσ'∶B : Δ ,T A ⊢ M ⟦ liftSub \_ σ' ⟧ ∶ ty B\<\\
\>        Δ,A⊢Mσ'∶B = substitution Γ,A⊢M∶B (ctxTR validΔ) (liftSub-typed σ'∶Γ⇒Δ) in\<\\
\>    let Θ,A⊢Mσ∶B : Θ ,T A ⊢ M ⟦ liftSub \_ σ ⟧ 〈 liftRep \_ ρ 〉 ∶ ty B\<\\
\>        Θ,A⊢Mσ∶B = weakening Δ,A⊢Mσ∶B (ctxTR validΘ) (liftRep-typed ρ∶Δ⇒Θ) in\<\\
\>    let Θ,A⊢Mσ'∶B : Θ ,T A ⊢ M ⟦ liftSub \_ σ' ⟧ 〈 liftRep \_ ρ 〉 ∶ ty B\<\\
\>        Θ,A⊢Mσ'∶B = weakening Δ,A⊢Mσ'∶B (ctxTR validΘ) (liftRep-typed ρ∶Δ⇒Θ) in --TODO Common pattern\<\\
\>    let Θ⊢N∶A : Θ ⊢ N ∶ ty A\<\\
\>        Θ⊢N∶A = E-typed N∈EΘA in\<\\
\>    let Θ⊢N'∶A : Θ ⊢ N' ∶ ty A\<\\
\>        Θ⊢N'∶A = E-typed N'∈EΘA in\<\\
\>        expand-EE (conv-EE \<\\
\>          (subst (EE Θ (M ⟦ σ₁ ⟧ ≡〈 B 〉 M ⟦ σSR ⟧)) (let open ≡-Reasoning in\<\\
\>          begin\<\\
\>            M ⟦⟦ extendPS (ρ •RP τ) Q ∶ σ₁ ∼\<\\
\>                 σSR ⟧⟧\<\\
\>          ≡⟨⟨ pathsub-cong M ∼∼-refl step1 step2 ⟩⟩\<\\
\>            M ⟦⟦ extendPS (ρ •RP τ) Q ∶ x₀:= N • liftSub -Term (ρ •RS σ) ∼\<\\
\>                 x₀:= N' • liftSub -Term (ρ •RS σ') ⟧⟧\<\\
\>          ≡⟨ pathsub-extendPS M ⟩\<\\
\>            M ⟦⟦ liftPathSub (ρ •RP τ) ∶ sub↖ (ρ •RS σ) ∼ sub↗ (ρ •RS σ') ⟧⟧ ⟦ x₀:= N • x₀:= (N' ⇑) • x₀:= (Q ⇑ ⇑) ⟧\<\\
\>          ≡⟨ sub-congl (pathsub-cong M liftPathSub-compRP sub↖-comp₁ sub↗-comp₁) ⟩\<\\
\>            M ⟦⟦ ρ' •RP liftPathSub τ ∶ ρ' •RS sub↖ σ ∼ ρ' •RS sub↗ σ' ⟧⟧ ⟦ x₀:= N • x₀:= (N' ⇑) • x₀:= (Q ⇑ ⇑) ⟧\<\\
\>          ≡⟨ sub-congl (pathsub-compRP M) ⟩\<\\
\>            (M ⟦⟦ liftPathSub τ ∶ sub↖ σ ∼ sub↗ σ' ⟧⟧) 〈 ρ' 〉 ⟦ x₀:= N • x₀:= (N' ⇑) • x₀:= (Q ⇑ ⇑) ⟧\<\\
\>          ∎) ih) \<\\
\>          (sym-conv (osr-conv (subst (λ a → appT ((ΛT A M) ⟦ σ ⟧ 〈 ρ 〉) N ⇒ a) (sym (sub-rep-comp σ N)) (redex ?)))) \<\\
\>          (sym-conv (osr-conv (subst (λ a → appT ((ΛT A M) ⟦ σ' ⟧ 〈 ρ 〉) N' ⇒ a) (sym (sub-rep-comp σ' N')) (redex ?)))) --REFACTOR Duplication\<\\
\>          (appR (ΛR Θ,A⊢Mσ∶B) Θ⊢N∶A) \<\\
\>          (appR (ΛR Θ,A⊢Mσ'∶B) (E-typed N'∈EΘA)))\<\\
\>        (let step3 : addpath Δ A ⊢\<\\
\>                         M ⟦⟦ liftPathSub τ ∶ sub↖ σ ∼ sub↗ σ' ⟧⟧\<\\
\>                         ∶ M ⟦ sub↖ σ ⟧ ≡〈 B 〉 M ⟦ sub↗ σ' ⟧\<\\
\>             step3 = Path-substitution Γ,A⊢M∶B (liftPathSub-typed (pathsubC-typed \{τ = τ\} \{σ\} \{σ'\} \{Γ\} \{Δ\} τ∶σ∼σ')) \<\\
\>                     sub↖σ-typed sub↗σ'-typed validΔAAE in\<\\
\>        let step4 : addpath Θ A ⊢\<\\
\>                    M ⟦⟦ liftPathSub τ ∶ sub↖ σ ∼ sub↗ σ' ⟧⟧ 〈 ρ' 〉\<\\
\>                  ∶ M ⟦ sub↖ σ ⟧ 〈 ρ' 〉 ≡〈 B 〉 M ⟦ sub↗ σ' ⟧ 〈 ρ' 〉\<\\
\>            step4 = weakening step3 \<\\
\>                    (ctxER (varR x₁ (ctxTR (ctxTR validΘ)))\<\\
\>                    (varR x₀ (ctxTR (ctxTR validΘ))))\<\\
\>                    (liftRep-typed (liftRep-typed (liftRep-typed ρ∶Δ⇒Θ))) in\<\\
\>        let step5 : ∀ σ x → σ ∶ Γ ⇒ Δ → typeof x (addpath Θ A) ≡ ty A → addpath Θ A ⊢\<\\
\>                    appT ((ΛT A M) ⟦ σ ⟧ 〈 ρ 〉 ⇑ ⇑ ⇑) (var x) ∶ ty B\<\\
\>            step5 σ x σ∶Γ⇒Θ x∶A∈ΘA = appR \<\\
\>                           (ΛR (subst (λ a → addpath Θ A ,T A ⊢ a ∶ ty B) \<\\
\>                           (trans (sub-compRS M) (trans (rep-comp (M ⟦ liftSub \_ σ ⟧))\<\\
\>                           (trans (rep-comp (M ⟦ liftSub \_ σ ⟧ 〈 liftRep \_ ρ 〉)) \<\\
\>                             (rep-comp (M ⟦ liftSub \_ σ ⟧ 〈 liftRep \_ ρ 〉 〈 liftRep \_ upRep 〉)))))\<\\
\>                         (substitution \{σ = liftRep \_ upRep •R liftRep \_ upRep •R liftRep \_ upRep •R liftRep \_ ρ •RS liftSub \_ σ\} Γ,A⊢M∶B \<\\
\>                         (ctxTR (ctxER (varR x₁ (ctxTR (ctxTR validΘ))) (varR x₀ (ctxTR (ctxTR validΘ)))))\<\\
\>                         (compRS-typed\<\\
\>                            \{ρ = liftRep \_ upRep •R liftRep \_ upRep •R liftRep \_ upRep •R liftRep \_ ρ\}\<\\
\>                            \{σ = liftSub \_ σ\} \<\\
\>                            (compR-typed \{ρ = liftRep \_ upRep •R liftRep \_ upRep •R liftRep \_ upRep\}\<\\
\>                              \{σ = liftRep \_ ρ\}\<\\
\>                              (compR-typed \{ρ = liftRep \_ upRep •R liftRep \_ upRep\} \{σ = liftRep \_ upRep\}\<\\
\>                                (compR-typed \{ρ = liftRep \_ upRep\} \{σ = liftRep \_ upRep\} (liftRep-typed upRep-typed) (liftRep-typed upRep-typed)) (liftRep-typed upRep-typed)) \<\\
\>                            (liftRep-typed ρ∶Δ⇒Θ))\<\\
\>                         (liftSub-typed σ∶Γ⇒Θ)))))\<\\
\>                         (change-type (varR x (ctxER (varR x₁ (ctxTR (ctxTR validΘ))) (varR x₀ (ctxTR (ctxTR validΘ))))) x∶A∈ΘA) in --TODO Extract last line as lemma\<\\
\>             let step6 : addpath Θ A ⊢\<\\
\>                         M ⟦⟦ liftPathSub τ ∶ sub↖ σ ∼ sub↗ σ' ⟧⟧ 〈 ρ' 〉\<\\
\>                         ∶ appT ((ΛT A M) ⟦ σ ⟧ 〈 ρ 〉 ⇑ ⇑ ⇑) (var x₂) ≡〈 B 〉 appT ((ΛT A M) ⟦ σ' ⟧ 〈 ρ 〉 ⇑ ⇑ ⇑) (var x₁)\<\\
\>                 step6 = convER step4 (step5 σ x₂ σ∶Γ⇒Δ refl) (step5 σ' x₁ σ'∶Γ⇒Δ refl)\<\\
\>                         (subst (λ a → a ≃ appT (((ΛT A M ⟦ σ ⟧) 〈 ρ 〉) ⇑ ⇑ ⇑) (var x₂)) \<\\
\>                         (let open ≡-Reasoning in\<\\
\>                           M ⟦ liftSub \_ σ ⟧ 〈 liftRep \_ ρ 〉 〈 liftRep \_ upRep 〉 〈 liftRep \_ upRep 〉 〈 liftRep \_ upRep 〉 ⟦ x₀:= (var x₂) ⟧\<\\
\>                         ≡⟨⟨ sub-compSR (M ⟦ liftSub \_ σ ⟧ 〈 liftRep \_ ρ 〉 〈 liftRep \_ upRep 〉 〈 liftRep \_ upRep 〉) ⟩⟩\<\\
\>                           M ⟦ liftSub \_ σ ⟧ 〈 liftRep \_ ρ 〉 〈 liftRep \_ upRep 〉 〈 liftRep \_ upRep 〉 ⟦ x₀:= (var x₂) •SR liftRep \_ upRep ⟧\<\\
\>                         ≡⟨⟨ sub-compSR (M ⟦ liftSub \_ σ ⟧ 〈 liftRep \_ ρ 〉 〈 liftRep \_ upRep 〉) ⟩⟩\<\\
\>                           M ⟦ liftSub \_ σ ⟧ 〈 liftRep \_ ρ 〉 〈 liftRep \_ upRep 〉 ⟦ x₀:= (var x₂) •SR liftRep \_ upRep •SR liftRep \_ upRep ⟧\<\\
\>                         ≡⟨⟨ sub-compSR (M ⟦ liftSub \_ σ ⟧ 〈 liftRep \_ ρ 〉) ⟩⟩\<\\
\>                           M ⟦ liftSub \_ σ ⟧ 〈 liftRep \_ ρ 〉 ⟦ x₀:= (var x₂) •SR liftRep \_ upRep •SR liftRep \_ upRep •SR liftRep \_ upRep ⟧\<\\
\>                         ≡⟨⟨ sub-compSR (M ⟦ liftSub \_ σ ⟧) ⟩⟩\<\\
\>                           M ⟦ liftSub \_ σ ⟧ ⟦ x₀:= (var x₂) •SR liftRep \_ upRep •SR liftRep \_ upRep •SR liftRep \_ upRep •SR liftRep \_ ρ ⟧\<\\
\>                         ≡⟨⟨ sub-comp M ⟩⟩\<\\
\>                           M ⟦ x₀:= (var x₂) •SR liftRep \_ upRep •SR liftRep \_ upRep •SR liftRep \_ upRep •SR liftRep \_ ρ • liftSub \_ σ ⟧\<\\
\>                         ≡⟨ sub-congr M aux ⟩\<\\
\>                           M ⟦ liftRep \_ (liftRep \_ (liftRep \_ ρ)) •RS sub↖ σ ⟧\<\\
\>                         ≡⟨ sub-compRS M ⟩ \<\\
\>                           M ⟦ sub↖ σ ⟧ 〈 liftRep \_ (liftRep \_ (liftRep \_ ρ)) 〉\<\\
\>                           ∎)\<\\
\>                           (sym-conv (osr-conv (redex ?)))) \<\\
\>                         (subst (λ a → a ≃ appT (((ΛT A M ⟦ σ' ⟧) 〈 ρ 〉) ⇑ ⇑ ⇑) (var x₁)) \<\\
\>                           (let open ≡-Reasoning in \<\\
\>                           M ⟦ liftSub \_ σ' ⟧ 〈 liftRep \_ ρ 〉 〈 liftRep \_ upRep 〉 〈 liftRep \_ upRep 〉 〈 liftRep \_ upRep 〉 ⟦ x₀:= (var x₁) ⟧\<\\
\>                         ≡⟨⟨ sub-compSR (M ⟦ liftSub \_ σ' ⟧ 〈 liftRep \_ ρ 〉 〈 liftRep \_ upRep 〉 〈 liftRep \_ upRep 〉) ⟩⟩\<\\
\>                           M ⟦ liftSub \_ σ' ⟧ 〈 liftRep \_ ρ 〉 〈 liftRep \_ upRep 〉 〈 liftRep \_ upRep 〉 ⟦ x₀:= (var x₁) •SR liftRep \_ upRep ⟧\<\\
\>                         ≡⟨⟨ sub-compSR (M ⟦ liftSub \_ σ' ⟧ 〈 liftRep \_ ρ 〉 〈 liftRep \_ upRep 〉) ⟩⟩\<\\
\>                           M ⟦ liftSub \_ σ' ⟧ 〈 liftRep \_ ρ 〉 〈 liftRep \_ upRep 〉 ⟦ x₀:= (var x₁) •SR liftRep \_ upRep •SR liftRep \_ upRep ⟧\<\\
\>                         ≡⟨⟨ sub-compSR (M ⟦ liftSub \_ σ' ⟧ 〈 liftRep \_ ρ 〉) ⟩⟩\<\\
\>                           M ⟦ liftSub \_ σ' ⟧ 〈 liftRep \_ ρ 〉 ⟦ x₀:= (var x₁) •SR liftRep \_ upRep •SR liftRep \_ upRep •SR liftRep \_ upRep ⟧\<\\
\>                         ≡⟨⟨ sub-compSR (M ⟦ liftSub \_ σ' ⟧) ⟩⟩\<\\
\>                           M ⟦ liftSub \_ σ' ⟧ ⟦ x₀:= (var x₁) •SR liftRep \_ upRep •SR liftRep \_ upRep •SR liftRep \_ upRep •SR liftRep \_ ρ ⟧\<\\
\>                         ≡⟨⟨ sub-comp M ⟩⟩\<\\
\>                           M ⟦ x₀:= (var x₁) •SR liftRep \_ upRep •SR liftRep \_ upRep •SR liftRep \_ upRep •SR liftRep \_ ρ • liftSub \_ σ' ⟧\<\\
\>                         ≡⟨ sub-congr M aux₂ ⟩\<\\
\>                           M ⟦ liftRep \_ (liftRep \_ (liftRep \_ ρ)) •RS sub↗ σ' ⟧\<\\
\>                         ≡⟨ sub-compRS M ⟩ \<\\
\>                           M ⟦ sub↗ σ' ⟧ 〈 liftRep \_ (liftRep \_ (liftRep \_ ρ)) 〉\<\\
\>                           ∎)\<\\
\>                           (sym-conv (osr-conv (redex ?)))) in\<\\
\>      app*R (E-typed N∈EΘA) (E-typed N'∈EΘA) \<\\
\>      (lllR step6) (EE-typed Q∈EΘN≡N'))\<\\
\>      ?) where\<\\
\>    aux : ∀ \{U\} \{V\} \{W\} \{ρ : Rep V W\} \{σ : Sub U V\} → \<\\
\>        x₀:= (var x₂) •SR liftRep \_ upRep •SR liftRep \_ upRep •SR liftRep \_ upRep •SR liftRep \_ ρ • liftSub \_ σ ∼ liftRep \_ (liftRep \_ (liftRep \_ ρ)) •RS sub↖ σ\<\\
\>    aux x₀ = refl\<\\
\>    aux \{ρ = ρ\} \{σ\} (↑ x) = let open ≡-Reasoning in \<\\
\>      begin\<\\
\>        σ \_ x ⇑ ⟦ x₀:= (var x₂) •SR liftRep -Term upRep •SR liftRep -Term upRep •SR\<\\
\>       liftRep -Term upRep\<\\
\>       •SR liftRep -Term ρ ⟧\<\\
\>      ≡⟨ sub-compSR (σ \_ x ⇑) ⟩\<\\
\>        σ \_ x ⇑ 〈 liftRep \_ ρ 〉 ⟦ x₀:= (var x₂) •SR liftRep -Term upRep •SR liftRep -Term upRep •SR liftRep -Term upRep ⟧\<\\
\>      ≡⟨ sub-congl (liftRep-upRep (σ \_ x)) ⟩\<\\
\>        σ \_ x 〈 ρ 〉 ⇑ ⟦ x₀:= (var x₂) •SR liftRep -Term upRep •SR liftRep -Term upRep •SR liftRep -Term upRep ⟧\<\\
\>      ≡⟨ sub-compSR (σ \_ x 〈 ρ 〉 ⇑) ⟩\<\\
\>        σ \_ x 〈 ρ 〉 ⇑ 〈 liftRep \_ upRep 〉 ⟦ x₀:= (var x₂) •SR liftRep -Term upRep •SR liftRep -Term upRep ⟧\<\\
\>      ≡⟨ sub-congl (liftRep-upRep (σ \_ x 〈 ρ 〉)) ⟩\<\\
\>        σ \_ x 〈 ρ 〉 ⇑ ⇑ ⟦ x₀:= (var x₂) •SR liftRep -Term upRep •SR liftRep -Term upRep ⟧\<\\
\>      ≡⟨ sub-compSR (σ \_ x 〈 ρ 〉 ⇑ ⇑) ⟩\<\\
\>        σ \_ x 〈 ρ 〉 ⇑ ⇑ 〈 liftRep \_ upRep 〉 ⟦ x₀:= (var x₂) •SR liftRep -Term upRep ⟧\<\\
\>      ≡⟨ sub-congl (liftRep-upRep (σ \_ x 〈 ρ 〉 ⇑)) ⟩\<\\
\>        σ \_ x 〈 ρ 〉 ⇑ ⇑ ⇑ ⟦ x₀:= (var x₂) •SR liftRep -Term upRep ⟧\<\\
\>      ≡⟨ sub-compSR (σ \_ x 〈 ρ 〉 ⇑ ⇑ ⇑) ⟩\<\\
\>        σ \_ x 〈 ρ 〉 ⇑ ⇑ ⇑ 〈 liftRep -Term upRep 〉 ⟦ x₀:= (var x₂) ⟧\<\\
\>      ≡⟨ sub-congl (liftRep-upRep (σ \_ x 〈 ρ 〉 ⇑ ⇑)) ⟩\<\\
\>        σ \_ x 〈 ρ 〉 ⇑ ⇑ ⇑ ⇑ ⟦ x₀:= (var x₂) ⟧\<\\
\>      ≡⟨ botsub-upRep (σ \_ x 〈 ρ 〉 ⇑ ⇑ ⇑) ⟩\<\\
\>        σ \_ x 〈 ρ 〉 ⇑ ⇑ ⇑\<\\
\>      ≡⟨⟨ liftRep-upRep₃ (σ \_ x) ⟩⟩\<\\
\>        σ \_ x ⇑ ⇑ ⇑ 〈 liftRep \_ (liftRep \_ (liftRep \_ ρ)) 〉\<\\
\>      ∎\<\\
\>    aux₂ : ∀ \{U\} \{V\} \{W\} \{ρ : Rep V W\} \{σ : Sub U V\} → \<\\
\>        x₀:= (var x₁) •SR liftRep \_ upRep •SR liftRep \_ upRep •SR liftRep \_ upRep •SR liftRep \_ ρ • liftSub \_ σ ∼ liftRep \_ (liftRep \_ (liftRep \_ ρ)) •RS sub↗ σ\<\\
\>    aux₂ x₀ = refl\<\\
\>    aux₂ \{ρ = ρ\} \{σ\} (↑ x) = let open ≡-Reasoning in \<\\
\>      begin\<\\
\>        σ \_ x ⇑ ⟦ x₀:= (var x₁) •SR liftRep -Term upRep •SR liftRep -Term upRep •SR\<\\
\>       liftRep -Term upRep\<\\
\>       •SR liftRep -Term ρ ⟧\<\\
\>      ≡⟨ sub-compSR (σ \_ x ⇑) ⟩\<\\
\>        σ \_ x ⇑ 〈 liftRep \_ ρ 〉 ⟦ x₀:= (var x₁) •SR liftRep -Term upRep •SR liftRep -Term upRep •SR liftRep -Term upRep ⟧\<\\
\>      ≡⟨ sub-congl (liftRep-upRep (σ \_ x)) ⟩\<\\
\>        σ \_ x 〈 ρ 〉 ⇑ ⟦ x₀:= (var x₁) •SR liftRep -Term upRep •SR liftRep -Term upRep •SR liftRep -Term upRep ⟧\<\\
\>      ≡⟨ sub-compSR (σ \_ x 〈 ρ 〉 ⇑) ⟩\<\\
\>        σ \_ x 〈 ρ 〉 ⇑ 〈 liftRep \_ upRep 〉 ⟦ x₀:= (var x₁) •SR liftRep -Term upRep •SR liftRep -Term upRep ⟧\<\\
\>      ≡⟨ sub-congl (liftRep-upRep (σ \_ x 〈 ρ 〉)) ⟩\<\\
\>        σ \_ x 〈 ρ 〉 ⇑ ⇑ ⟦ x₀:= (var x₁) •SR liftRep -Term upRep •SR liftRep -Term upRep ⟧\<\\
\>      ≡⟨ sub-compSR (σ \_ x 〈 ρ 〉 ⇑ ⇑) ⟩\<\\
\>        σ \_ x 〈 ρ 〉 ⇑ ⇑ 〈 liftRep \_ upRep 〉 ⟦ x₀:= (var x₁) •SR liftRep -Term upRep ⟧\<\\
\>      ≡⟨ sub-congl (liftRep-upRep (σ \_ x 〈 ρ 〉 ⇑)) ⟩\<\\
\>        σ \_ x 〈 ρ 〉 ⇑ ⇑ ⇑ ⟦ x₀:= (var x₁) •SR liftRep -Term upRep ⟧\<\\
\>      ≡⟨ sub-compSR (σ \_ x 〈 ρ 〉 ⇑ ⇑ ⇑) ⟩\<\\
\>        σ \_ x 〈 ρ 〉 ⇑ ⇑ ⇑ 〈 liftRep -Term upRep 〉 ⟦ x₀:= (var x₁) ⟧\<\\
\>      ≡⟨ sub-congl (liftRep-upRep (σ \_ x 〈 ρ 〉 ⇑ ⇑)) ⟩\<\\
\>        σ \_ x 〈 ρ 〉 ⇑ ⇑ ⇑ ⇑ ⟦ x₀:= (var x₁) ⟧\<\\
\>      ≡⟨ botsub-upRep (σ \_ x 〈 ρ 〉 ⇑ ⇑ ⇑) ⟩\<\\
\>        σ \_ x 〈 ρ 〉 ⇑ ⇑ ⇑\<\\
\>      ≡⟨⟨ liftRep-upRep₃ (σ \_ x) ⟩⟩\<\\
\>        σ \_ x ⇑ ⇑ ⇑ 〈 liftRep \_ (liftRep \_ (liftRep \_ ρ)) 〉\<\\
\>      ∎\<\\
\>    aux₃ : ∀ \{U\} \{V\} \{σ : Sub U V\} → \<\\
\>        x₀:= (var x₂) •SR liftRep \_ upRep •SR liftRep \_ upRep •SR liftRep \_ upRep • liftSub \_ σ ∼ sub↖ σ\<\\
\>    aux₃ x₀ = refl\<\\
\>    aux₃ \{σ = σ\} (↑ x) = let open ≡-Reasoning in \<\\
\>      begin\<\\
\>        σ \_ x ⇑ ⟦ x₀:= (var x₂) •SR liftRep -Term upRep •SR liftRep -Term upRep •SR liftRep -Term upRep ⟧\<\\
\>      ≡⟨ sub-compSR (σ \_ x ⇑) ⟩\<\\
\>        σ \_ x ⇑ 〈 liftRep \_ upRep 〉 ⟦ x₀:= (var x₂) •SR liftRep -Term upRep •SR liftRep -Term upRep ⟧\<\\
\>      ≡⟨ sub-congl (liftRep-upRep (σ \_ x)) ⟩\<\\
\>        σ \_ x ⇑ ⇑ ⟦ x₀:= (var x₂) •SR liftRep -Term upRep •SR liftRep -Term upRep ⟧\<\\
\>      ≡⟨ sub-compSR (σ \_ x  ⇑ ⇑) ⟩\<\\
\>        σ \_ x  ⇑ ⇑ 〈 liftRep \_ upRep 〉 ⟦ x₀:= (var x₂) •SR liftRep -Term upRep ⟧\<\\
\>      ≡⟨ sub-congl (liftRep-upRep (σ \_ x  ⇑)) ⟩\<\\
\>        σ \_ x  ⇑ ⇑ ⇑ ⟦ x₀:= (var x₂) •SR liftRep -Term upRep ⟧\<\\
\>      ≡⟨ sub-compSR (σ \_ x  ⇑ ⇑ ⇑) ⟩\<\\
\>        σ \_ x  ⇑ ⇑ ⇑ 〈 liftRep -Term upRep 〉 ⟦ x₀:= (var x₂) ⟧\<\\
\>      ≡⟨ sub-congl (liftRep-upRep (σ \_ x  ⇑ ⇑)) ⟩\<\\
\>        σ \_ x  ⇑ ⇑ ⇑ ⇑ ⟦ x₀:= (var x₂) ⟧\<\\
\>      ≡⟨ botsub-upRep (σ \_ x  ⇑ ⇑ ⇑) ⟩\<\\
\>        σ \_ x  ⇑ ⇑ ⇑\<\\
\>      ∎\<\\
\>    aux₄ : ∀ \{U\} \{V\} \{σ : Sub U V\} → \<\\
\>        x₀:= (var x₁) •SR liftRep \_ upRep •SR liftRep \_ upRep •SR liftRep \_ upRep • liftSub \_ σ ∼ sub↗ σ\<\\
\>    aux₄ x₀ = refl\<\\
\>    aux₄ \{σ = σ\} (↑ x) = let open ≡-Reasoning in \<\\
\>      begin\<\\
\>        σ \_ x ⇑ ⟦ x₀:= (var x₁) •SR liftRep -Term upRep •SR liftRep -Term upRep •SR liftRep -Term upRep ⟧\<\\
\>      ≡⟨ sub-compSR (σ \_ x  ⇑) ⟩\<\\
\>        σ \_ x  ⇑ 〈 liftRep \_ upRep 〉 ⟦ x₀:= (var x₁) •SR liftRep -Term upRep •SR liftRep -Term upRep ⟧\<\\
\>      ≡⟨ sub-congl (liftRep-upRep (σ \_ x )) ⟩\<\\
\>        σ \_ x  ⇑ ⇑ ⟦ x₀:= (var x₁) •SR liftRep -Term upRep •SR liftRep -Term upRep ⟧\<\\
\>      ≡⟨ sub-compSR (σ \_ x  ⇑ ⇑) ⟩\<\\
\>        σ \_ x  ⇑ ⇑ 〈 liftRep \_ upRep 〉 ⟦ x₀:= (var x₁) •SR liftRep -Term upRep ⟧\<\\
\>      ≡⟨ sub-congl (liftRep-upRep (σ \_ x  ⇑)) ⟩\<\\
\>        σ \_ x  ⇑ ⇑ ⇑ ⟦ x₀:= (var x₁) •SR liftRep -Term upRep ⟧\<\\
\>      ≡⟨ sub-compSR (σ \_ x  ⇑ ⇑ ⇑) ⟩\<\\
\>        σ \_ x  ⇑ ⇑ ⇑ 〈 liftRep -Term upRep 〉 ⟦ x₀:= (var x₁) ⟧\<\\
\>      ≡⟨ sub-congl (liftRep-upRep (σ \_ x  ⇑ ⇑)) ⟩\<\\
\>        σ \_ x  ⇑ ⇑ ⇑ ⇑ ⟦ x₀:= (var x₁) ⟧\<\\
\>      ≡⟨ botsub-upRep (σ \_ x  ⇑ ⇑ ⇑) ⟩\<\\
\>        σ \_ x  ⇑ ⇑ ⇑\<\\
\>      ∎ -\}}\<%
\end{code}
}

\begin{corollary}[Soundness]
If $\Gamma \vdash t : T$ then $t \in E_\Gamma(T)$.
\end{corollary}

\begin{proof}
We apply the theorem with $\sigma$ the identity substitution.  The identity substitution is computable
by Lemmas \ref{lm:varcompute1} and \ref{lm:varcompute2}.
\end{proof}

\begin{corollary}[Strong Normalization]
\label{cor:SN}
Every well-typed term, proof and path is strongly normalizing.
\end{corollary}

%<*Strong-Normalization>
\begin{code}%
\>\AgdaKeyword{postulate} \AgdaPostulate{all-Ectxt} \AgdaSymbol{:} \AgdaSymbol{∀} \AgdaSymbol{\{}\AgdaBound{V}\AgdaSymbol{\}} \AgdaSymbol{\{}\AgdaBound{Γ} \AgdaSymbol{:} \AgdaDatatype{Context} \AgdaBound{V}\AgdaSymbol{\}} \AgdaSymbol{→} \AgdaDatatype{valid} \AgdaBound{Γ} \AgdaSymbol{→} \AgdaFunction{Ectxt} \AgdaBound{Γ}\<%
\\
%
\\
\>\AgdaFunction{Strong-Normalization} \AgdaSymbol{:} \AgdaSymbol{∀} \AgdaBound{V} \AgdaBound{K} \AgdaSymbol{(}\AgdaBound{Γ} \AgdaSymbol{:} \AgdaDatatype{Context} \AgdaBound{V}\AgdaSymbol{)} \<[45]%
\>[45]\<%
\\
\>[0]\AgdaIndent{2}{}\<[2]%
\>[2]\AgdaSymbol{(}\AgdaBound{M} \AgdaSymbol{:} \AgdaFunction{Expression} \AgdaBound{V} \AgdaSymbol{(}\AgdaInductiveConstructor{varKind} \AgdaBound{K}\AgdaSymbol{))} \AgdaBound{A} \AgdaSymbol{→} \AgdaBound{Γ} \AgdaDatatype{⊢} \AgdaBound{M} \AgdaDatatype{∶} \AgdaBound{A} \AgdaSymbol{→} \AgdaDatatype{SN} \AgdaBound{M}\<%
\end{code}
%</Strong-Normalization>

\AgdaHide{
\begin{code}%
\>\AgdaFunction{Strong-Normalization} \AgdaBound{V} \AgdaBound{K} \AgdaBound{Γ} \AgdaBound{M} \AgdaBound{A} \AgdaBound{Γ⊢M∶A} \AgdaSymbol{=} \AgdaPostulate{E-SN}\<%
\\
\>[0]\AgdaIndent{2}{}\<[2]%
\>[2]\AgdaSymbol{(}\AgdaFunction{subst} \AgdaSymbol{(}\AgdaRecord{E} \AgdaBound{Γ} \AgdaSymbol{\_)} \AgdaFunction{sub-idSub}\<%
\\
\>[0]\AgdaIndent{2}{}\<[2]%
\>[2]\AgdaSymbol{(}\AgdaPostulate{Computable-Sub} \AgdaSymbol{(}\AgdaFunction{idSub} \AgdaBound{V}\AgdaSymbol{)} \AgdaSymbol{(}\AgdaFunction{idSubC} \AgdaSymbol{(}\AgdaFunction{context-validity} \AgdaBound{Γ⊢M∶A}\AgdaSymbol{)} \AgdaSymbol{(}\AgdaPostulate{all-Ectxt} \AgdaSymbol{(}\AgdaFunction{context-validity} \AgdaBound{Γ⊢M∶A}\AgdaSymbol{)))} \AgdaBound{Γ⊢M∶A} \AgdaSymbol{(}\AgdaFunction{context-validity} \AgdaBound{Γ⊢M∶A}\AgdaSymbol{)))}\<%
\end{code}
}

\begin{corollary}[Canonicity]
\label{cor:canon}
If $\vdash s : T$, then there is a unique canonical object $t$ of $T$ such that $s \twoheadrightarrow t$.
\end{corollary}

\begin{corollary}[Consistency]
There is no proof $\delta$ such that $\vdash \delta : \bot$.
\end{corollary}

\AgdaHide{
\begin{code}%
\>\AgdaKeyword{postulate} \AgdaPostulate{Consistency'} \AgdaSymbol{:} \AgdaSymbol{∀} \AgdaSymbol{\{}\AgdaBound{M} \AgdaSymbol{:} \AgdaFunction{Proof} \AgdaInductiveConstructor{∅}\AgdaSymbol{\}} \AgdaSymbol{→} \AgdaDatatype{SN} \AgdaBound{M} \AgdaSymbol{→} \AgdaInductiveConstructor{〈〉} \AgdaDatatype{⊢} \AgdaBound{M} \AgdaDatatype{∶} \AgdaFunction{⊥} \AgdaSymbol{→} \AgdaDatatype{Empty}\<%
\\
\>\AgdaComment{-- Consistency' (SNI M SNM) ⊢M∶⊥ = \{!!\}}\<%
\end{code}
}

%<*Consistency>
\begin{code}%
\>\AgdaKeyword{postulate} \AgdaPostulate{Consistency} \AgdaSymbol{:} \AgdaSymbol{∀} \AgdaSymbol{\{}\AgdaBound{M} \AgdaSymbol{:} \AgdaFunction{Proof} \AgdaInductiveConstructor{∅}\AgdaSymbol{\}} \AgdaSymbol{→} \AgdaInductiveConstructor{〈〉} \AgdaDatatype{⊢} \AgdaBound{M} \AgdaDatatype{∶} \AgdaFunction{⊥} \AgdaSymbol{→} \AgdaDatatype{Empty}\<%
\end{code}
%</Consistency>

\AgdaHide{
\begin{code}%
\>\AgdaComment{-- Consistency = \{!!\}}\<%
\end{code}
}

}

\section{Systems Two and Three}

\begin{frame}
\frametitle{Include equations in $\Omega$}

We can include the equations $M =_A N$ in our logic.

Now we can form (and prove) $sym_{MN} : M =_A N \supset N =_A M$ and $trans_{MNP} : M =_A N \supset N =_A P \supset M =_A P$ (by recursion on $A$).

\pause

\[ \infer{\Gamma \vdash M =_A N : \Omega}{\Gamma \vdash M : A \quad \Gamma \vdash N : A} \]
\[ \infer{\Gamma \vdash \delta =^* \epsilon : (M =_A N) =_\Omega (M' =_A N')}
{\Gamma \vdash \delta : M =_A M' \quad \Gamma \vdash \epsilon : N =_A N'} \]

\pause

\[ (\delta =^* \epsilon)^+ \rhd \lambda p. trans \, \epsilon \, (trans \, p \, (sym \delta)) \]
\[ (\delta =^* \epsilon)^- \rhd \lambda p. trans \, (sym \, \epsilon) \, (trans \, p \, \delta) \]
\end{frame}

\begin{frame}
\frametitle{Universal Quantification}

We can close $\Omega$ under universal quantification in any of the types $A$ (including $\Omega$!)

\[ \infer{\Gamma \vdash \forall I (x.\delta) : \forall x : A. \phi}{\Gamma, x : A vdash \delta : \phi} \]
\[ \infer{\Gamma \vdash \forall E(\delta, M) : \phi[x:=M]} {\Gamma \vdash \delta : \forall x :A.\phi \quad \Gamma \vdash M : A} \]
\[ \infer{\Gamma \vdash \forall* x:A.\delta : \forall x:A.\phi =_\Omega \forall x:A.\psi}{\Gamma, x : A \vdash \delta : \phi =_\Omega \psi} \]

\pause

\[ (\forall^* x:A.\delta)^+ \rhd \lambda p. \forall I(x.\delta^+ \forall E(p,x)) \]
\[ (\forall^* x:A.\delta)^- \rhd \lambda p.\forall I(x.\delta^- \forall E(p,x)) \]
\end{frame}

\section{Conclusion}

\begin{frame}
\frametitle{Conclusion}
\begin{itemize}[<+->]
\item
We have shown a system that has all these properties:
\begin{itemize}
\item
Univalence
\item
Strong Normalization
\item
Confluence of typed terms
\item
Canonicity
\end{itemize}
\item
So it is possible!
\item
The simplicity is due to the separation between terms and proofs.
\item
Let's see how far we can go.
\end{itemize}
\end{frame}
\appendix

\section{Replacement and Substitution}
\label{appendix:repsub}

\AgdaHide{
\begin{code}%
\>\AgdaKeyword{open} \AgdaKeyword{import} \AgdaModule{Grammar.Base}\<%
\\
%
\\
\>\AgdaKeyword{module} \AgdaModule{Grammar.OpFamily.PreOpFamily} \AgdaSymbol{(}\AgdaBound{G} \AgdaSymbol{:} \AgdaRecord{Grammar}\AgdaSymbol{)} \AgdaKeyword{where}\<%
\\
\>\AgdaKeyword{open} \AgdaKeyword{import} \AgdaModule{Prelims}\<%
\\
\>\AgdaKeyword{open} \AgdaModule{Grammar} \AgdaBound{G}\<%
\end{code}
}

\subsection{Families of Operations}

Our aim here is to define the operations of \emph{replacement} and \emph{substitution}.  In order to organise this work, we introduce the following definitions.

A \emph{family of operations} over a grammar $G$ consists of:
\begin{enumerate}
\item
for any alphabets $U$ and $V$, a set $F[U,V]$ of \emph{operations} $\sigma$ from $U$ to $V$, $\sigma : U \rightarrow V$;
\item
for any operation $\sigma : U \rightarrow V$ and variable $x \in U$ of kind $K$, an expression $\sigma(x)$ over $V$ of kind $K$;
\item
for any alphabet $V$ and variable kind $K$, an operation $\uparrow : V \rightarrow (V , K)$, the \emph{lifting} operation;
\item
for any alphabet $V$, an operation $\id{V} : V \rightarrow V$, the \emph{identity} operation;
\item
for any operation $\sigma : U \rightarrow V$ and variable kind $K$, an operation $(\sigma , K) : (U , K) \rightarrow (V , K)$, the result of \emph{lifting} $\sigma$;
\item
for any operations $\rho : U \rightarrow V$ and $\sigma : V \rightarrow W$, an operation
$\sigma \circ \rho : U \rightarrow W$, the \emph{composition} of $\sigma$ and $\rho$;
\end{enumerate}
such that:
\begin{itemize}
\item
$\uparrow (x) \equiv x$
\item
$\id{V}(x) \equiv x$
\item
If $\rho \sim \sigma$ then $(\rho , K) \sim (\sigma , K)$
\item
$(\rho , K)(x_0) \equiv x_0$
\item
Given $\sigma : U \rightarrow V$ and $x \in U$, we have $(\sigma , K)(x) \equiv x$
\item
$(\sigma \circ \rho , K) \sim (\sigma , K) \circ (\rho , K)$
\item
$(\sigma \circ \rho)(x) \equiv \rho(x) [ \sigma ]$
\end{itemize}
where for $\sigma, \rho : U \rightarrow V$ we write $\sigma \sim \rho$ iff $\sigma(x) \equiv \rho(x)$ for all $x \in U$; and, given $\sigma : U \rightarrow V$ and $E$ an expression over $U$, we define $E[\sigma]$, the result of \emph{applying} the operation $\sigma$ to $E$, as follows:

\begin{align*}
x[\sigma] & \eqdef \sigma(x) \\
\lefteqn{c([\vec{x_1}] E_1, \ldots, [\vec{x_n}] E_n) [\sigma]} \\
 & \eqdef
c([\vec{x_1}] E_1 [(\sigma , K_{11}, \ldots, K_{1r_1})], \ldots,
[\vec{x_n}] E_n [(\sigma, K_{n1}, \ldots, K_{nr_n})])
\end{align*}
for $c$ a constructor of type (\ref{eq:conkind}).

\subsubsection{Pre-Families}
We formalize this definition in stages.  First, we define a \emph{pre-family of operations} to be a family with items of data 1--4 above that satisfies the first two axioms:

\begin{code}%
\>\AgdaKeyword{record} \AgdaRecord{PreOpFamily} \AgdaSymbol{:} \AgdaPrimitiveType{Set₂} \AgdaKeyword{where}\<%
\\
\>[0]\AgdaIndent{2}{}\<[2]%
\>[2]\AgdaKeyword{field}\<%
\\
\>[2]\AgdaIndent{4}{}\<[4]%
\>[4]\AgdaField{Op} \AgdaSymbol{:} \AgdaDatatype{Alphabet} \AgdaSymbol{→} \AgdaDatatype{Alphabet} \AgdaSymbol{→} \AgdaPrimitiveType{Set}\<%
\\
\>[2]\AgdaIndent{4}{}\<[4]%
\>[4]\AgdaField{apV} \AgdaSymbol{:} \AgdaSymbol{∀} \AgdaSymbol{\{}\AgdaBound{U}\AgdaSymbol{\}} \AgdaSymbol{\{}\AgdaBound{V}\AgdaSymbol{\}} \AgdaSymbol{\{}\AgdaBound{K}\AgdaSymbol{\}} \AgdaSymbol{→} \AgdaField{Op} \AgdaBound{U} \AgdaBound{V} \AgdaSymbol{→} \AgdaDatatype{Var} \AgdaBound{U} \AgdaBound{K} \AgdaSymbol{→} \AgdaFunction{Expression} \AgdaBound{V} \AgdaSymbol{(}\AgdaInductiveConstructor{varKind} \AgdaBound{K}\AgdaSymbol{)}\<%
\\
\>[2]\AgdaIndent{4}{}\<[4]%
\>[4]\AgdaField{up} \AgdaSymbol{:} \AgdaSymbol{∀} \AgdaSymbol{\{}\AgdaBound{V}\AgdaSymbol{\}} \AgdaSymbol{\{}\AgdaBound{K}\AgdaSymbol{\}} \AgdaSymbol{→} \AgdaField{Op} \AgdaBound{V} \AgdaSymbol{(}\AgdaBound{V} \AgdaInductiveConstructor{,} \AgdaBound{K}\AgdaSymbol{)}\<%
\\
\>[2]\AgdaIndent{4}{}\<[4]%
\>[4]\AgdaField{apV-up} \AgdaSymbol{:} \AgdaSymbol{∀} \AgdaSymbol{\{}\AgdaBound{V}\AgdaSymbol{\}} \AgdaSymbol{\{}\AgdaBound{K}\AgdaSymbol{\}} \AgdaSymbol{\{}\AgdaBound{L}\AgdaSymbol{\}} \AgdaSymbol{\{}\AgdaBound{x} \AgdaSymbol{:} \AgdaDatatype{Var} \AgdaBound{V} \AgdaBound{K}\AgdaSymbol{\}} \AgdaSymbol{→} \AgdaField{apV} \AgdaSymbol{(}\AgdaField{up} \AgdaSymbol{\{}\AgdaArgument{K} \AgdaSymbol{=} \AgdaBound{L}\AgdaSymbol{\})} \AgdaBound{x} \AgdaDatatype{≡} \AgdaInductiveConstructor{var} \AgdaSymbol{(}\AgdaInductiveConstructor{↑} \AgdaBound{x}\AgdaSymbol{)}\<%
\\
\>[2]\AgdaIndent{4}{}\<[4]%
\>[4]\AgdaField{idOp} \AgdaSymbol{:} \AgdaSymbol{∀} \AgdaBound{V} \AgdaSymbol{→} \AgdaField{Op} \AgdaBound{V} \AgdaBound{V}\<%
\\
\>[2]\AgdaIndent{4}{}\<[4]%
\>[4]\AgdaField{apV-idOp} \AgdaSymbol{:} \AgdaSymbol{∀} \AgdaSymbol{\{}\AgdaBound{V}\AgdaSymbol{\}} \AgdaSymbol{\{}\AgdaBound{K}\AgdaSymbol{\}} \AgdaSymbol{(}\AgdaBound{x} \AgdaSymbol{:} \AgdaDatatype{Var} \AgdaBound{V} \AgdaBound{K}\AgdaSymbol{)} \AgdaSymbol{→} \AgdaField{apV} \AgdaSymbol{(}\AgdaField{idOp} \AgdaBound{V}\AgdaSymbol{)} \AgdaBound{x} \AgdaDatatype{≡} \AgdaInductiveConstructor{var} \AgdaBound{x}\<%
\end{code}

This allows us to define the relation $\sim$, and prove it is an equivalence relation:

\begin{code}%
\>[0]\AgdaIndent{2}{}\<[2]%
\>[2]\AgdaFunction{\_∼op\_} \AgdaSymbol{:} \AgdaSymbol{∀} \AgdaSymbol{\{}\AgdaBound{U}\AgdaSymbol{\}} \AgdaSymbol{\{}\AgdaBound{V}\AgdaSymbol{\}} \AgdaSymbol{→} \AgdaField{Op} \AgdaBound{U} \AgdaBound{V} \AgdaSymbol{→} \AgdaField{Op} \AgdaBound{U} \AgdaBound{V} \AgdaSymbol{→} \AgdaPrimitiveType{Set}\<%
\\
\>[0]\AgdaIndent{2}{}\<[2]%
\>[2]\AgdaFunction{\_∼op\_} \AgdaSymbol{\{}\AgdaBound{U}\AgdaSymbol{\}} \AgdaSymbol{\{}\AgdaBound{V}\AgdaSymbol{\}} \AgdaBound{ρ} \AgdaBound{σ} \AgdaSymbol{=} \AgdaSymbol{∀} \AgdaSymbol{\{}\AgdaBound{K}\AgdaSymbol{\}} \AgdaSymbol{(}\AgdaBound{x} \AgdaSymbol{:} \AgdaDatatype{Var} \AgdaBound{U} \AgdaBound{K}\AgdaSymbol{)} \AgdaSymbol{→} \AgdaField{apV} \AgdaBound{ρ} \AgdaBound{x} \AgdaDatatype{≡} \AgdaField{apV} \AgdaBound{σ} \AgdaBound{x}\<%
\\
\>[2]\AgdaIndent{4}{}\<[4]%
\>[4]\<%
\\
\>[0]\AgdaIndent{2}{}\<[2]%
\>[2]\AgdaFunction{∼-refl} \AgdaSymbol{:} \AgdaSymbol{∀} \AgdaSymbol{\{}\AgdaBound{U}\AgdaSymbol{\}} \AgdaSymbol{\{}\AgdaBound{V}\AgdaSymbol{\}} \AgdaSymbol{\{}\AgdaBound{σ} \AgdaSymbol{:} \AgdaField{Op} \AgdaBound{U} \AgdaBound{V}\AgdaSymbol{\}} \AgdaSymbol{→} \AgdaBound{σ} \AgdaFunction{∼op} \AgdaBound{σ}\<%
\\
\>[0]\AgdaIndent{2}{}\<[2]%
\>[2]\AgdaFunction{∼-refl} \AgdaSymbol{\_} \AgdaSymbol{=} \AgdaInductiveConstructor{refl}\<%
\\
\>[2]\AgdaIndent{4}{}\<[4]%
\>[4]\<%
\\
\>[0]\AgdaIndent{2}{}\<[2]%
\>[2]\AgdaFunction{∼-sym} \AgdaSymbol{:} \AgdaSymbol{∀} \AgdaSymbol{\{}\AgdaBound{U}\AgdaSymbol{\}} \AgdaSymbol{\{}\AgdaBound{V}\AgdaSymbol{\}} \AgdaSymbol{\{}\AgdaBound{σ} \AgdaBound{τ} \AgdaSymbol{:} \AgdaField{Op} \AgdaBound{U} \AgdaBound{V}\AgdaSymbol{\}} \AgdaSymbol{→} \AgdaBound{σ} \AgdaFunction{∼op} \AgdaBound{τ} \AgdaSymbol{→} \AgdaBound{τ} \AgdaFunction{∼op} \AgdaBound{σ}\<%
\\
\>[0]\AgdaIndent{2}{}\<[2]%
\>[2]\AgdaFunction{∼-sym} \AgdaBound{σ-is-τ} \AgdaBound{x} \AgdaSymbol{=} \AgdaFunction{sym} \AgdaSymbol{(}\AgdaBound{σ-is-τ} \AgdaBound{x}\AgdaSymbol{)}\<%
\\
%
\\
\>[0]\AgdaIndent{2}{}\<[2]%
\>[2]\AgdaFunction{∼-trans} \AgdaSymbol{:} \AgdaSymbol{∀} \AgdaSymbol{\{}\AgdaBound{U}\AgdaSymbol{\}} \AgdaSymbol{\{}\AgdaBound{V}\AgdaSymbol{\}} \AgdaSymbol{\{}\AgdaBound{ρ} \AgdaBound{σ} \AgdaBound{τ} \AgdaSymbol{:} \AgdaField{Op} \AgdaBound{U} \AgdaBound{V}\AgdaSymbol{\}} \AgdaSymbol{→} \AgdaBound{ρ} \AgdaFunction{∼op} \AgdaBound{σ} \AgdaSymbol{→} \AgdaBound{σ} \AgdaFunction{∼op} \AgdaBound{τ} \AgdaSymbol{→} \AgdaBound{ρ} \AgdaFunction{∼op} \AgdaBound{τ}\<%
\\
\>[0]\AgdaIndent{2}{}\<[2]%
\>[2]\AgdaFunction{∼-trans} \AgdaBound{ρ-is-σ} \AgdaBound{σ-is-τ} \AgdaBound{x} \AgdaSymbol{=} \AgdaFunction{trans} \AgdaSymbol{(}\AgdaBound{ρ-is-σ} \AgdaBound{x}\AgdaSymbol{)} \AgdaSymbol{(}\AgdaBound{σ-is-τ} \AgdaBound{x}\AgdaSymbol{)}\<%
\\
%
\\
\>[0]\AgdaIndent{2}{}\<[2]%
\>[2]\AgdaFunction{OP} \AgdaSymbol{:} \AgdaDatatype{Alphabet} \AgdaSymbol{→} \AgdaDatatype{Alphabet} \AgdaSymbol{→} \AgdaRecord{Setoid} \AgdaSymbol{\_} \AgdaSymbol{\_}\<%
\\
\>[0]\AgdaIndent{2}{}\<[2]%
\>[2]\AgdaFunction{OP} \AgdaBound{U} \AgdaBound{V} \AgdaSymbol{=} \AgdaKeyword{record} \AgdaSymbol{\{} \<[20]%
\>[20]\<%
\\
\>[2]\AgdaIndent{5}{}\<[5]%
\>[5]\AgdaField{Carrier} \AgdaSymbol{=} \AgdaField{Op} \AgdaBound{U} \AgdaBound{V} \AgdaSymbol{;} \<[24]%
\>[24]\<%
\\
\>[2]\AgdaIndent{5}{}\<[5]%
\>[5]\AgdaField{\_≈\_} \AgdaSymbol{=} \AgdaFunction{\_∼op\_} \AgdaSymbol{;} \<[19]%
\>[19]\<%
\\
\>[2]\AgdaIndent{5}{}\<[5]%
\>[5]\AgdaField{isEquivalence} \AgdaSymbol{=} \AgdaKeyword{record} \AgdaSymbol{\{} \<[30]%
\>[30]\<%
\\
\>[5]\AgdaIndent{7}{}\<[7]%
\>[7]\AgdaField{refl} \AgdaSymbol{=} \AgdaFunction{∼-refl} \AgdaSymbol{;} \<[23]%
\>[23]\<%
\\
\>[5]\AgdaIndent{7}{}\<[7]%
\>[7]\AgdaField{sym} \AgdaSymbol{=} \AgdaFunction{∼-sym} \AgdaSymbol{;} \<[21]%
\>[21]\<%
\\
\>[5]\AgdaIndent{7}{}\<[7]%
\>[7]\AgdaField{trans} \AgdaSymbol{=} \AgdaFunction{∼-trans} \AgdaSymbol{\}} \AgdaSymbol{\}}\<%
\end{code}

\AgdaHide{
\begin{code}%
\>\AgdaKeyword{open} \AgdaKeyword{import} \AgdaModule{Grammar.Base}\<%
\\
%
\\
\>\AgdaKeyword{module} \AgdaModule{Grammar.OpFamily.Lifting} \AgdaSymbol{(}\AgdaBound{G} \AgdaSymbol{:} \AgdaRecord{Grammar}\AgdaSymbol{)} \AgdaKeyword{where}\<%
\\
\>\AgdaKeyword{open} \AgdaKeyword{import} \AgdaModule{Data.List}\<%
\\
\>\AgdaKeyword{open} \AgdaKeyword{import} \AgdaModule{Prelims}\<%
\\
\>\AgdaKeyword{open} \AgdaModule{Grammar} \AgdaBound{G}\<%
\\
\>\AgdaKeyword{open} \AgdaKeyword{import} \AgdaModule{Grammar.OpFamily.PreOpFamily} \AgdaBound{G}\<%
\end{code}
}

\subsubsection{Liftings}

Define a \emph{lifting} on a pre-family to be an function $(- , K)$ that respects $\sim$:

\begin{code}%
\>\AgdaKeyword{record} \AgdaRecord{Lifting} \AgdaSymbol{(}\AgdaBound{F} \AgdaSymbol{:} \AgdaRecord{PreOpFamily}\AgdaSymbol{)} \AgdaSymbol{:} \AgdaPrimitiveType{Set₁} \AgdaKeyword{where}\<%
\\
\>[0]\AgdaIndent{2}{}\<[2]%
\>[2]\AgdaKeyword{open} \AgdaModule{PreOpFamily} \AgdaBound{F}\<%
\\
\>[0]\AgdaIndent{2}{}\<[2]%
\>[2]\AgdaKeyword{field}\<%
\\
\>[2]\AgdaIndent{4}{}\<[4]%
\>[4]\AgdaField{liftOp} \AgdaSymbol{:} \AgdaSymbol{∀} \AgdaSymbol{\{}\AgdaBound{U}\AgdaSymbol{\}} \AgdaSymbol{\{}\AgdaBound{V}\AgdaSymbol{\}} \AgdaBound{K} \AgdaSymbol{→} \AgdaFunction{Op} \AgdaBound{U} \AgdaBound{V} \AgdaSymbol{→} \AgdaFunction{Op} \AgdaSymbol{(}\AgdaBound{U} \AgdaInductiveConstructor{,} \AgdaBound{K}\AgdaSymbol{)} \AgdaSymbol{(}\AgdaBound{V} \AgdaInductiveConstructor{,} \AgdaBound{K}\AgdaSymbol{)}\<%
\\
\>[2]\AgdaIndent{4}{}\<[4]%
\>[4]\AgdaField{liftOp-cong} \AgdaSymbol{:} \AgdaSymbol{∀} \AgdaSymbol{\{}\AgdaBound{V}\AgdaSymbol{\}} \AgdaSymbol{\{}\AgdaBound{W}\AgdaSymbol{\}} \AgdaSymbol{\{}\AgdaBound{K}\AgdaSymbol{\}} \AgdaSymbol{\{}\AgdaBound{ρ} \AgdaBound{σ} \AgdaSymbol{:} \AgdaFunction{Op} \AgdaBound{V} \AgdaBound{W}\AgdaSymbol{\}} \AgdaSymbol{→} \<[49]%
\>[49]\<%
\\
\>[4]\AgdaIndent{6}{}\<[6]%
\>[6]\AgdaBound{ρ} \AgdaFunction{∼op} \AgdaBound{σ} \AgdaSymbol{→} \AgdaField{liftOp} \AgdaBound{K} \AgdaBound{ρ} \AgdaFunction{∼op} \AgdaField{liftOp} \AgdaBound{K} \AgdaBound{σ}\<%
\end{code}

Given an operation $\sigma : U \rightarrow V$ and a list of variable kinds $A \equiv (A_1 , \ldots , A_n)$, define
the \emph{repeated lifting} $\sigma^A$ to be $((\cdots(\sigma , A_1) , A_2) , \cdots ) , A_n)$.

\begin{code}%
\>\AgdaComment{\{-  liftOp' : ∀ \{U\} \{V\} A → Op U V → Op (extend U A) (extend V A)\<\\
\>  liftOp' [] σ = σ\<\\
\>  liftOp' (K ∷ A) σ = liftOp' A (liftOp K σ) -\}}\<%
\\
%
\\
\>[0]\AgdaIndent{2}{}\<[2]%
\>[2]\AgdaFunction{liftOp''} \AgdaSymbol{:} \AgdaSymbol{∀} \AgdaSymbol{\{}\AgdaBound{U}\AgdaSymbol{\}} \AgdaSymbol{\{}\AgdaBound{V}\AgdaSymbol{\}} \AgdaSymbol{\{}\AgdaBound{K}\AgdaSymbol{\}} \AgdaBound{A} \AgdaSymbol{→} \AgdaFunction{Op} \AgdaBound{U} \AgdaBound{V} \AgdaSymbol{→} \AgdaFunction{Op} \AgdaSymbol{(}\AgdaFunction{dom} \AgdaBound{U} \AgdaSymbol{\{}\AgdaBound{K}\AgdaSymbol{\}} \AgdaBound{A}\AgdaSymbol{)} \AgdaSymbol{(}\AgdaFunction{dom} \AgdaBound{V} \AgdaBound{A}\AgdaSymbol{)}\<%
\\
\>[0]\AgdaIndent{2}{}\<[2]%
\>[2]\AgdaFunction{liftOp''} \AgdaSymbol{(\_} \AgdaInductiveConstructor{✧}\AgdaSymbol{)} \AgdaBound{σ} \AgdaSymbol{=} \AgdaBound{σ}\<%
\\
\>[0]\AgdaIndent{2}{}\<[2]%
\>[2]\AgdaFunction{liftOp''} \AgdaSymbol{(}\AgdaBound{K} \AgdaInductiveConstructor{abs} \AgdaBound{A}\AgdaSymbol{)} \AgdaBound{σ} \AgdaSymbol{=} \AgdaFunction{liftOp''} \AgdaBound{A} \AgdaSymbol{(}\AgdaField{liftOp} \AgdaBound{K} \AgdaBound{σ}\AgdaSymbol{)}\<%
\\
%
\\
\>\AgdaComment{\{-  liftOp'-cong : ∀ \{U\} \{V\} A \{ρ σ : Op U V\} → \<\\
\>    ρ ∼op σ → liftOp' A ρ ∼op liftOp' A σ\<\\
\>}\<%
\end{code}

\AgdaHide{
\begin{code}%
\>\AgdaComment{\<\\
\>  liftOp'-cong [] ρ-is-σ = ρ-is-σ\<\\
\>  liftOp'-cong (\_ ∷ A) ρ-is-σ = liftOp'-cong A (liftOp-cong ρ-is-σ) -\}}\<%
\\
%
\\
\>[0]\AgdaIndent{2}{}\<[2]%
\>[2]\AgdaKeyword{postulate} \AgdaPostulate{liftOp''-cong} \AgdaSymbol{:} \AgdaSymbol{∀} \AgdaSymbol{\{}\AgdaBound{U}\AgdaSymbol{\}} \AgdaSymbol{\{}\AgdaBound{V}\AgdaSymbol{\}} \AgdaSymbol{\{}\AgdaBound{K}\AgdaSymbol{\}} \AgdaBound{A} \AgdaSymbol{\{}\AgdaBound{ρ} \AgdaBound{σ} \AgdaSymbol{:} \AgdaFunction{Op} \AgdaBound{U} \AgdaBound{V}\AgdaSymbol{\}} \AgdaSymbol{→} \<[61]%
\>[61]\<%
\\
\>[2]\AgdaIndent{26}{}\<[26]%
\>[26]\AgdaBound{ρ} \AgdaFunction{∼op} \AgdaBound{σ} \AgdaSymbol{→} \AgdaFunction{liftOp''} \AgdaSymbol{\{}\AgdaArgument{K} \AgdaSymbol{=} \AgdaBound{K}\AgdaSymbol{\}} \AgdaBound{A} \AgdaBound{ρ} \AgdaFunction{∼op} \AgdaFunction{liftOp''} \AgdaBound{A} \AgdaBound{σ}\<%
\end{code}
}

This allows us to define the action of \emph{application} $E[\sigma]$:

\begin{code}%
\>[0]\AgdaIndent{2}{}\<[2]%
\>[2]\AgdaFunction{ap} \AgdaSymbol{:} \AgdaSymbol{∀} \AgdaSymbol{\{}\AgdaBound{U}\AgdaSymbol{\}} \AgdaSymbol{\{}\AgdaBound{V}\AgdaSymbol{\}} \AgdaSymbol{\{}\AgdaBound{C}\AgdaSymbol{\}} \AgdaSymbol{\{}\AgdaBound{K}\AgdaSymbol{\}} \AgdaSymbol{→} \<[27]%
\>[27]\<%
\\
\>[2]\AgdaIndent{4}{}\<[4]%
\>[4]\AgdaFunction{Op} \AgdaBound{U} \AgdaBound{V} \AgdaSymbol{→} \AgdaDatatype{Subexpression} \AgdaBound{U} \AgdaBound{C} \AgdaBound{K} \AgdaSymbol{→} \AgdaDatatype{Subexpression} \AgdaBound{V} \AgdaBound{C} \AgdaBound{K}\<%
\\
\>[0]\AgdaIndent{2}{}\<[2]%
\>[2]\AgdaFunction{ap} \AgdaBound{ρ} \AgdaSymbol{(}\AgdaInductiveConstructor{var} \AgdaBound{x}\AgdaSymbol{)} \AgdaSymbol{=} \AgdaFunction{apV} \AgdaBound{ρ} \AgdaBound{x}\<%
\\
\>[0]\AgdaIndent{2}{}\<[2]%
\>[2]\AgdaFunction{ap} \AgdaBound{ρ} \AgdaSymbol{(}\AgdaInductiveConstructor{app} \AgdaBound{c} \AgdaBound{EE}\AgdaSymbol{)} \AgdaSymbol{=} \AgdaInductiveConstructor{app} \AgdaBound{c} \AgdaSymbol{(}\AgdaFunction{ap} \AgdaBound{ρ} \AgdaBound{EE}\AgdaSymbol{)}\<%
\\
\>[0]\AgdaIndent{2}{}\<[2]%
\>[2]\AgdaFunction{ap} \AgdaSymbol{\_} \AgdaInductiveConstructor{out} \AgdaSymbol{=} \AgdaInductiveConstructor{out}\<%
\\
\>[0]\AgdaIndent{2}{}\<[2]%
\>[2]\AgdaFunction{ap} \AgdaBound{ρ} \AgdaSymbol{(}\AgdaInductiveConstructor{\_,,\_} \AgdaSymbol{\{}\AgdaArgument{A} \AgdaSymbol{=} \AgdaBound{A}\AgdaSymbol{\}} \AgdaBound{E} \AgdaBound{EE}\AgdaSymbol{)} \AgdaSymbol{=} \AgdaFunction{ap} \AgdaSymbol{(}\AgdaFunction{liftOp''} \AgdaBound{A} \AgdaBound{ρ}\AgdaSymbol{)} \AgdaBound{E} \AgdaInductiveConstructor{,,} \AgdaFunction{ap} \AgdaBound{ρ} \AgdaBound{EE}\<%
\end{code}

We prove that application respects $\sim$.

\begin{code}%
\>[0]\AgdaIndent{2}{}\<[2]%
\>[2]\AgdaFunction{ap-congl} \AgdaSymbol{:} \AgdaSymbol{∀} \AgdaSymbol{\{}\AgdaBound{U}\AgdaSymbol{\}} \AgdaSymbol{\{}\AgdaBound{V}\AgdaSymbol{\}} \AgdaSymbol{\{}\AgdaBound{C}\AgdaSymbol{\}} \AgdaSymbol{\{}\AgdaBound{K}\AgdaSymbol{\}} \<[31]%
\>[31]\<%
\\
\>[2]\AgdaIndent{4}{}\<[4]%
\>[4]\AgdaSymbol{\{}\AgdaBound{ρ} \AgdaBound{σ} \AgdaSymbol{:} \AgdaFunction{Op} \AgdaBound{U} \AgdaBound{V}\AgdaSymbol{\}} \AgdaSymbol{(}\AgdaBound{E} \AgdaSymbol{:} \AgdaDatatype{Subexpression} \AgdaBound{U} \AgdaBound{C} \AgdaBound{K}\AgdaSymbol{)} \AgdaSymbol{→}\<%
\\
\>[2]\AgdaIndent{4}{}\<[4]%
\>[4]\AgdaBound{ρ} \AgdaFunction{∼op} \AgdaBound{σ} \AgdaSymbol{→} \AgdaFunction{ap} \AgdaBound{ρ} \AgdaBound{E} \AgdaDatatype{≡} \AgdaFunction{ap} \AgdaBound{σ} \AgdaBound{E}\<%
\end{code}

\AgdaHide{
\begin{code}%
\>[0]\AgdaIndent{2}{}\<[2]%
\>[2]\AgdaFunction{ap-congl} \AgdaSymbol{(}\AgdaInductiveConstructor{var} \AgdaBound{x}\AgdaSymbol{)} \AgdaBound{ρ-is-σ} \AgdaSymbol{=} \AgdaBound{ρ-is-σ} \AgdaBound{x}\<%
\\
\>[0]\AgdaIndent{2}{}\<[2]%
\>[2]\AgdaFunction{ap-congl} \AgdaSymbol{(}\AgdaInductiveConstructor{app} \AgdaBound{c} \AgdaBound{E}\AgdaSymbol{)} \AgdaBound{ρ-is-σ} \AgdaSymbol{=} \AgdaFunction{cong} \AgdaSymbol{(}\AgdaInductiveConstructor{app} \AgdaBound{c}\AgdaSymbol{)} \AgdaSymbol{(}\AgdaFunction{ap-congl} \AgdaBound{E} \AgdaBound{ρ-is-σ}\AgdaSymbol{)}\<%
\\
\>[0]\AgdaIndent{2}{}\<[2]%
\>[2]\AgdaFunction{ap-congl} \AgdaInductiveConstructor{out} \AgdaSymbol{\_} \AgdaSymbol{=} \AgdaInductiveConstructor{refl}\<%
\\
\>[0]\AgdaIndent{2}{}\<[2]%
\>[2]\AgdaFunction{ap-congl} \AgdaSymbol{(}\AgdaInductiveConstructor{\_,,\_} \AgdaSymbol{\{}\AgdaArgument{L} \AgdaSymbol{=} \AgdaBound{L}\AgdaSymbol{\}} \AgdaSymbol{\{}\AgdaArgument{A} \AgdaSymbol{=} \AgdaBound{A}\AgdaSymbol{\}} \AgdaBound{E} \AgdaBound{F}\AgdaSymbol{)} \AgdaBound{ρ-is-σ} \AgdaSymbol{=} \<[47]%
\>[47]\<%
\\
\>[2]\AgdaIndent{4}{}\<[4]%
\>[4]\AgdaFunction{cong₂} \AgdaInductiveConstructor{\_,,\_} \AgdaSymbol{(}\AgdaFunction{ap-congl} \AgdaBound{E} \AgdaSymbol{(}\AgdaPostulate{liftOp''-cong} \AgdaBound{A} \AgdaBound{ρ-is-σ}\AgdaSymbol{))} \AgdaSymbol{(}\AgdaFunction{ap-congl} \AgdaBound{F} \AgdaBound{ρ-is-σ}\AgdaSymbol{)}\<%
\\
%
\\
\>[0]\AgdaIndent{2}{}\<[2]%
\>[2]\AgdaFunction{ap-congr} \AgdaSymbol{:} \AgdaSymbol{∀} \AgdaSymbol{\{}\AgdaBound{U}\AgdaSymbol{\}} \AgdaSymbol{\{}\AgdaBound{V}\AgdaSymbol{\}} \AgdaSymbol{\{}\AgdaBound{C}\AgdaSymbol{\}} \AgdaSymbol{\{}\AgdaBound{K}\AgdaSymbol{\}}\<%
\\
\>[2]\AgdaIndent{4}{}\<[4]%
\>[4]\AgdaSymbol{\{}\AgdaBound{σ} \AgdaSymbol{:} \AgdaFunction{Op} \AgdaBound{U} \AgdaBound{V}\AgdaSymbol{\}} \AgdaSymbol{\{}\AgdaBound{E} \AgdaBound{F} \AgdaSymbol{:} \AgdaDatatype{Subexpression} \AgdaBound{U} \AgdaBound{C} \AgdaBound{K}\AgdaSymbol{\}} \AgdaSymbol{→}\<%
\\
\>[2]\AgdaIndent{4}{}\<[4]%
\>[4]\AgdaBound{E} \AgdaDatatype{≡} \AgdaBound{F} \AgdaSymbol{→} \AgdaFunction{ap} \AgdaBound{σ} \AgdaBound{E} \AgdaDatatype{≡} \AgdaFunction{ap} \AgdaBound{σ} \AgdaBound{F}\<%
\\
\>[0]\AgdaIndent{2}{}\<[2]%
\>[2]\AgdaFunction{ap-congr} \AgdaSymbol{\{}\AgdaArgument{σ} \AgdaSymbol{=} \AgdaBound{σ}\AgdaSymbol{\}} \AgdaSymbol{=} \AgdaFunction{cong} \AgdaSymbol{(}\AgdaFunction{ap} \AgdaBound{σ}\AgdaSymbol{)}\<%
\\
%
\\
\>[0]\AgdaIndent{2}{}\<[2]%
\>[2]\AgdaFunction{ap-cong} \AgdaSymbol{:} \AgdaSymbol{∀} \AgdaSymbol{\{}\AgdaBound{U}\AgdaSymbol{\}} \AgdaSymbol{\{}\AgdaBound{V}\AgdaSymbol{\}} \AgdaSymbol{\{}\AgdaBound{C}\AgdaSymbol{\}} \AgdaSymbol{\{}\AgdaBound{K}\AgdaSymbol{\}}\<%
\\
\>[2]\AgdaIndent{4}{}\<[4]%
\>[4]\AgdaSymbol{\{}\AgdaBound{ρ} \AgdaBound{σ} \AgdaSymbol{:} \AgdaFunction{Op} \AgdaBound{U} \AgdaBound{V}\AgdaSymbol{\}} \AgdaSymbol{\{}\AgdaBound{M} \AgdaBound{N} \AgdaSymbol{:} \AgdaDatatype{Subexpression} \AgdaBound{U} \AgdaBound{C} \AgdaBound{K}\AgdaSymbol{\}} \AgdaSymbol{→}\<%
\\
\>[2]\AgdaIndent{4}{}\<[4]%
\>[4]\AgdaBound{ρ} \AgdaFunction{∼op} \AgdaBound{σ} \AgdaSymbol{→} \AgdaBound{M} \AgdaDatatype{≡} \AgdaBound{N} \AgdaSymbol{→} \AgdaFunction{ap} \AgdaBound{ρ} \AgdaBound{M} \AgdaDatatype{≡} \AgdaFunction{ap} \AgdaBound{σ} \AgdaBound{N}\<%
\\
\>[0]\AgdaIndent{2}{}\<[2]%
\>[2]\AgdaFunction{ap-cong} \AgdaSymbol{\{}\AgdaArgument{ρ} \AgdaSymbol{=} \AgdaBound{ρ}\AgdaSymbol{\}} \AgdaSymbol{\{}\AgdaBound{σ}\AgdaSymbol{\}} \AgdaSymbol{\{}\AgdaBound{M}\AgdaSymbol{\}} \AgdaSymbol{\{}\AgdaBound{N}\AgdaSymbol{\}} \AgdaBound{ρ∼σ} \AgdaBound{M≡N} \AgdaSymbol{=} \AgdaKeyword{let} \AgdaKeyword{open} \AgdaModule{≡-Reasoning} \AgdaKeyword{in} \<[64]%
\>[64]\<%
\\
\>[2]\AgdaIndent{4}{}\<[4]%
\>[4]\AgdaFunction{begin}\<%
\\
\>[4]\AgdaIndent{6}{}\<[6]%
\>[6]\AgdaFunction{ap} \AgdaBound{ρ} \AgdaBound{M}\<%
\\
\>[0]\AgdaIndent{4}{}\<[4]%
\>[4]\AgdaFunction{≡⟨} \AgdaFunction{ap-congl} \AgdaBound{M} \AgdaBound{ρ∼σ} \AgdaFunction{⟩}\<%
\\
\>[4]\AgdaIndent{6}{}\<[6]%
\>[6]\AgdaFunction{ap} \AgdaBound{σ} \AgdaBound{M}\<%
\\
\>[0]\AgdaIndent{4}{}\<[4]%
\>[4]\AgdaFunction{≡⟨} \AgdaFunction{ap-congr} \AgdaBound{M≡N} \AgdaFunction{⟩}\<%
\\
\>[4]\AgdaIndent{6}{}\<[6]%
\>[6]\AgdaFunction{ap} \AgdaBound{σ} \AgdaBound{N}\<%
\\
\>[0]\AgdaIndent{4}{}\<[4]%
\>[4]\AgdaFunction{∎}\<%
\end{code}
}

\AgdaHide{
\begin{code}%
\>\AgdaKeyword{open} \AgdaKeyword{import} \AgdaModule{Grammar.Base}\<%
\\
%
\\
\>\AgdaKeyword{module} \AgdaModule{Grammar.Substitution.LiftFamily} \AgdaSymbol{(}\AgdaBound{G} \AgdaSymbol{:} \AgdaRecord{Grammar}\AgdaSymbol{)} \AgdaKeyword{where}\<%
\\
\>\AgdaKeyword{open} \AgdaKeyword{import} \AgdaModule{Prelims}\<%
\\
\>\AgdaKeyword{open} \AgdaModule{Grammar} \AgdaBound{G}\<%
\\
\>\AgdaKeyword{open} \AgdaKeyword{import} \AgdaModule{Grammar.OpFamily.LiftFamily} \AgdaBound{G}\<%
\\
\>\AgdaKeyword{open} \AgdaKeyword{import} \AgdaModule{Grammar.Substitution.PreOpFamily} \AgdaBound{G}\<%
\\
\>\AgdaKeyword{open} \AgdaKeyword{import} \AgdaModule{Grammar.Substitution.Lifting} \AgdaBound{G}\<%
\\
\>\AgdaKeyword{open} \AgdaKeyword{import} \AgdaModule{Grammar.Substitution.RepSub} \AgdaBound{G}\<%
\end{code}
}

It is now easy to show that substitution forms a pre-family with lifting.  If $\sigma : U \rightarrow V$ and $x \in U$ then $(\sigma , K)(\uparrow x) \equiv
\sigma(x) \langle \uparrow \rangle \equiv (\sigma , K)(x) [ \uparrow ]$.

\begin{code}%
\>\AgdaFunction{SubLF} \AgdaSymbol{:} \AgdaRecord{LiftFamily}\<%
\\
\>\AgdaFunction{SubLF} \AgdaSymbol{=} \AgdaKeyword{record} \AgdaSymbol{\{} \<[17]%
\>[17]\<%
\\
\>[0]\AgdaIndent{2}{}\<[2]%
\>[2]\AgdaField{preOpFamily} \AgdaSymbol{=} \AgdaFunction{pre-substitution} \AgdaSymbol{;} \<[35]%
\>[35]\<%
\\
\>[0]\AgdaIndent{2}{}\<[2]%
\>[2]\AgdaField{lifting} \AgdaSymbol{=} \AgdaFunction{SUB↑} \AgdaSymbol{;} \<[19]%
\>[19]\<%
\\
\>[0]\AgdaIndent{2}{}\<[2]%
\>[2]\AgdaField{isLiftFamily} \AgdaSymbol{=} \AgdaKeyword{record} \AgdaSymbol{\{} \<[26]%
\>[26]\<%
\\
\>[2]\AgdaIndent{4}{}\<[4]%
\>[4]\AgdaField{liftOp-x₀} \AgdaSymbol{=} \AgdaInductiveConstructor{refl} \AgdaSymbol{;} \<[23]%
\>[23]\<%
\\
\>[2]\AgdaIndent{4}{}\<[4]%
\>[4]\AgdaField{liftOp-↑} \AgdaSymbol{=} \AgdaSymbol{λ} \AgdaSymbol{\{}\AgdaBound{\_}\AgdaSymbol{\}} \AgdaSymbol{\{}\AgdaBound{\_}\AgdaSymbol{\}} \AgdaSymbol{\{}\AgdaBound{\_}\AgdaSymbol{\}} \AgdaSymbol{\{}\AgdaBound{\_}\AgdaSymbol{\}} \AgdaSymbol{\{}\AgdaBound{σ}\AgdaSymbol{\}} \AgdaBound{x} \AgdaSymbol{→} \AgdaFunction{rep-is-sub} \AgdaSymbol{(}\AgdaBound{σ} \AgdaSymbol{\_} \AgdaBound{x}\AgdaSymbol{)} \AgdaSymbol{\}\}}\<%
\end{code}

\AgdaHide{
\begin{code}%
\>\AgdaKeyword{open} \AgdaKeyword{import} \AgdaModule{Grammar.Base}\<%
\\
%
\\
\>\AgdaKeyword{module} \AgdaModule{Grammar.OpFamily.Composition} \AgdaSymbol{(}\AgdaBound{A} \AgdaSymbol{:} \AgdaRecord{Grammar}\AgdaSymbol{)} \AgdaKeyword{where}\<%
\\
\>\AgdaKeyword{open} \AgdaKeyword{import} \AgdaModule{Data.List}\<%
\\
\>\AgdaKeyword{open} \AgdaKeyword{import} \AgdaModule{Function.Equality} \AgdaKeyword{hiding} \AgdaSymbol{(}\AgdaField{cong}\AgdaSymbol{;}\AgdaFunction{\_∘\_}\AgdaSymbol{)}\<%
\\
\>\AgdaKeyword{open} \AgdaKeyword{import} \AgdaModule{Prelims}\<%
\\
\>\AgdaKeyword{open} \AgdaModule{Grammar} \AgdaBound{A} \AgdaKeyword{hiding} \AgdaSymbol{(}\_⟶\_\AgdaSymbol{)}\<%
\\
\>\AgdaKeyword{open} \AgdaKeyword{import} \AgdaModule{Grammar.OpFamily.LiftFamily} \AgdaBound{A}\<%
\\
%
\\
\>\AgdaKeyword{open} \AgdaModule{LiftFamily}\<%
\end{code}
}

\subsubsection{Compositions}

Let $F$, $G$ and $H$ be three pre-families with lifting.  A \emph{composition} $\circ : F;G \rightarrow H$ is a family of functions
\[ \circ_{UVW} : F[V,W] \times G[U,V] \rightarrow H[U,W] \]
for all alphabets $U$, $V$ and $W$ such that:
\begin{itemize}
\item
$(\sigma \circ \rho , K) \sim (\sigma , K) \circ (\rho , K)$
\item
$(\sigma \circ \rho)(x) \equiv \rho(x) [ \sigma ]$
\end{itemize}

\begin{code}%
\>\AgdaKeyword{record} \AgdaRecord{Composition} \AgdaSymbol{(}\AgdaBound{F} \AgdaBound{G} \AgdaBound{H} \AgdaSymbol{:} \AgdaRecord{LiftFamily}\AgdaSymbol{)} \AgdaSymbol{:} \AgdaPrimitiveType{Set} \AgdaKeyword{where}\<%
\\
\>[0]\AgdaIndent{2}{}\<[2]%
\>[2]\AgdaKeyword{infix} \AgdaNumber{25} \AgdaFixityOp{\_∘\_}\<%
\\
\>[0]\AgdaIndent{2}{}\<[2]%
\>[2]\AgdaKeyword{field}\<%
\\
\>[2]\AgdaIndent{4}{}\<[4]%
\>[4]\AgdaField{\_∘\_} \AgdaSymbol{:} \AgdaSymbol{∀} \AgdaSymbol{\{}\AgdaBound{U}\AgdaSymbol{\}} \AgdaSymbol{\{}\AgdaBound{V}\AgdaSymbol{\}} \AgdaSymbol{\{}\AgdaBound{W}\AgdaSymbol{\}} \AgdaSymbol{→} \AgdaFunction{Op} \AgdaBound{F} \AgdaBound{V} \AgdaBound{W} \AgdaSymbol{→} \AgdaFunction{Op} \AgdaBound{G} \AgdaBound{U} \AgdaBound{V} \AgdaSymbol{→} \AgdaFunction{Op} \AgdaBound{H} \AgdaBound{U} \AgdaBound{W}\<%
\\
\>[2]\AgdaIndent{4}{}\<[4]%
\>[4]\AgdaField{liftOp-comp} \AgdaSymbol{:} \AgdaSymbol{∀} \AgdaSymbol{\{}\AgdaBound{U} \AgdaBound{V} \AgdaBound{W} \AgdaBound{K} \AgdaBound{σ} \AgdaBound{ρ}\AgdaSymbol{\}} \AgdaSymbol{→} \<[36]%
\>[36]\<%
\\
\>[4]\AgdaIndent{6}{}\<[6]%
\>[6]\AgdaFunction{\_∼op\_} \AgdaBound{H} \AgdaSymbol{(}\AgdaFunction{liftOp} \AgdaBound{H} \AgdaBound{K} \AgdaSymbol{(}\AgdaField{\_∘\_} \AgdaSymbol{\{}\AgdaBound{U}\AgdaSymbol{\}} \AgdaSymbol{\{}\AgdaBound{V}\AgdaSymbol{\}} \AgdaSymbol{\{}\AgdaBound{W}\AgdaSymbol{\}} \AgdaBound{σ} \AgdaBound{ρ}\AgdaSymbol{))} \<[49]%
\>[49]\<%
\\
\>[6]\AgdaIndent{8}{}\<[8]%
\>[8]\AgdaSymbol{(}\AgdaFunction{liftOp} \AgdaBound{F} \AgdaBound{K} \AgdaBound{σ} \AgdaField{∘} \AgdaFunction{liftOp} \AgdaBound{G} \AgdaBound{K} \AgdaBound{ρ}\AgdaSymbol{)}\<%
\\
\>[0]\AgdaIndent{4}{}\<[4]%
\>[4]\AgdaField{apV-comp} \AgdaSymbol{:} \AgdaSymbol{∀} \AgdaSymbol{\{}\AgdaBound{U}\AgdaSymbol{\}} \AgdaSymbol{\{}\AgdaBound{V}\AgdaSymbol{\}} \AgdaSymbol{\{}\AgdaBound{W}\AgdaSymbol{\}} \AgdaSymbol{\{}\AgdaBound{K}\AgdaSymbol{\}} \AgdaSymbol{\{}\AgdaBound{σ}\AgdaSymbol{\}} \AgdaSymbol{\{}\AgdaBound{ρ}\AgdaSymbol{\}} \AgdaSymbol{\{}\AgdaBound{x} \AgdaSymbol{:} \AgdaDatatype{Var} \AgdaBound{U} \AgdaBound{K}\AgdaSymbol{\}} \AgdaSymbol{→} \<[57]%
\>[57]\<%
\\
\>[4]\AgdaIndent{6}{}\<[6]%
\>[6]\AgdaFunction{apV} \AgdaBound{H} \AgdaSymbol{(}\AgdaField{\_∘\_} \AgdaSymbol{\{}\AgdaBound{U}\AgdaSymbol{\}} \AgdaSymbol{\{}\AgdaBound{V}\AgdaSymbol{\}} \AgdaSymbol{\{}\AgdaBound{W}\AgdaSymbol{\}} \AgdaBound{σ} \AgdaBound{ρ}\AgdaSymbol{)} \AgdaBound{x} \AgdaDatatype{≡} \AgdaFunction{ap} \AgdaBound{F} \AgdaBound{σ} \AgdaSymbol{(}\AgdaFunction{apV} \AgdaBound{G} \AgdaBound{ρ} \AgdaBound{x}\AgdaSymbol{)}\<%
\end{code}

\begin{lemma}
For any composition $\circ$:
\begin{enumerate}
\item
If $\sigma \sim \sigma'$ and $\rho \sim \rho'$ then $\sigma \circ \rho \sim \sigma' \circ \rho'$
\item
$(\sigma \circ \rho)^A \sim \sigma^A \circ \rho^A$
\item
$E [ \sigma \circ \rho ] \equiv E [ \rho ] [ \sigma ]$
\end{enumerate}
\end{lemma}

\begin{code}%
\>[0]\AgdaIndent{2}{}\<[2]%
\>[2]\AgdaFunction{comp-cong} \AgdaSymbol{:} \AgdaSymbol{∀} \AgdaSymbol{\{}\AgdaBound{U} \AgdaBound{V} \AgdaBound{W}\AgdaSymbol{\}} \AgdaSymbol{\{}\AgdaBound{σ} \AgdaBound{σ'} \AgdaSymbol{:} \AgdaFunction{Op} \AgdaBound{F} \AgdaBound{V} \AgdaBound{W}\AgdaSymbol{\}} \AgdaSymbol{\{}\AgdaBound{ρ} \AgdaBound{ρ'} \AgdaSymbol{:} \AgdaFunction{Op} \AgdaBound{G} \AgdaBound{U} \AgdaBound{V}\AgdaSymbol{\}} \AgdaSymbol{→} \<[62]%
\>[62]\<%
\\
\>[2]\AgdaIndent{4}{}\<[4]%
\>[4]\AgdaFunction{\_∼op\_} \AgdaBound{F} \AgdaBound{σ} \AgdaBound{σ'} \AgdaSymbol{→} \AgdaFunction{\_∼op\_} \AgdaBound{G} \AgdaBound{ρ} \AgdaBound{ρ'} \AgdaSymbol{→} \AgdaFunction{\_∼op\_} \AgdaBound{H} \AgdaSymbol{(}\AgdaBound{σ} \AgdaField{∘} \AgdaBound{ρ}\AgdaSymbol{)} \AgdaSymbol{(}\AgdaBound{σ'} \AgdaField{∘} \AgdaBound{ρ'}\AgdaSymbol{)}\<%
\end{code}

\AgdaHide{
\begin{code}%
\>[0]\AgdaIndent{2}{}\<[2]%
\>[2]\AgdaFunction{comp-cong} \AgdaSymbol{\{}\AgdaBound{U}\AgdaSymbol{\}} \AgdaSymbol{\{}\AgdaBound{V}\AgdaSymbol{\}} \AgdaSymbol{\{}\AgdaBound{W}\AgdaSymbol{\}} \AgdaSymbol{\{}\AgdaBound{σ}\AgdaSymbol{\}} \AgdaSymbol{\{}\AgdaBound{σ'}\AgdaSymbol{\}} \AgdaSymbol{\{}\AgdaBound{ρ}\AgdaSymbol{\}} \AgdaSymbol{\{}\AgdaBound{ρ'}\AgdaSymbol{\}} \AgdaBound{σ∼σ'} \AgdaBound{ρ∼ρ'} \AgdaBound{x} \AgdaSymbol{=} \AgdaKeyword{let} \AgdaKeyword{open} \AgdaModule{≡-Reasoning} \AgdaKeyword{in} \<[80]%
\>[80]\<%
\\
\>[2]\AgdaIndent{4}{}\<[4]%
\>[4]\AgdaFunction{begin}\<%
\\
\>[4]\AgdaIndent{6}{}\<[6]%
\>[6]\AgdaFunction{apV} \AgdaBound{H} \AgdaSymbol{(}\AgdaBound{σ} \AgdaField{∘} \AgdaBound{ρ}\AgdaSymbol{)} \AgdaBound{x}\<%
\\
\>[0]\AgdaIndent{4}{}\<[4]%
\>[4]\AgdaFunction{≡⟨} \AgdaField{apV-comp} \AgdaFunction{⟩}\<%
\\
\>[4]\AgdaIndent{6}{}\<[6]%
\>[6]\AgdaFunction{ap} \AgdaBound{F} \AgdaBound{σ} \AgdaSymbol{(}\AgdaFunction{apV} \AgdaBound{G} \AgdaBound{ρ} \AgdaBound{x}\AgdaSymbol{)}\<%
\\
\>[0]\AgdaIndent{4}{}\<[4]%
\>[4]\AgdaFunction{≡⟨} \AgdaFunction{ap-cong} \AgdaBound{F} \AgdaBound{σ∼σ'} \AgdaSymbol{(}\AgdaBound{ρ∼ρ'} \AgdaBound{x}\AgdaSymbol{)} \AgdaFunction{⟩}\<%
\\
\>[4]\AgdaIndent{6}{}\<[6]%
\>[6]\AgdaFunction{ap} \AgdaBound{F} \AgdaBound{σ'} \AgdaSymbol{(}\AgdaFunction{apV} \AgdaBound{G} \AgdaBound{ρ'} \AgdaBound{x}\AgdaSymbol{)}\<%
\\
\>[0]\AgdaIndent{4}{}\<[4]%
\>[4]\AgdaFunction{≡⟨⟨} \AgdaField{apV-comp} \AgdaFunction{⟩⟩}\<%
\\
\>[4]\AgdaIndent{6}{}\<[6]%
\>[6]\AgdaFunction{apV} \AgdaBound{H} \AgdaSymbol{(}\AgdaBound{σ'} \AgdaField{∘} \AgdaBound{ρ'}\AgdaSymbol{)} \AgdaBound{x}\<%
\\
\>[0]\AgdaIndent{4}{}\<[4]%
\>[4]\AgdaFunction{∎}\<%
\\
%
\\
\>[0]\AgdaIndent{2}{}\<[2]%
\>[2]\AgdaFunction{comp-congl} \AgdaSymbol{:} \AgdaSymbol{∀} \AgdaSymbol{\{}\AgdaBound{U}\AgdaSymbol{\}} \AgdaSymbol{\{}\AgdaBound{V}\AgdaSymbol{\}} \AgdaSymbol{\{}\AgdaBound{W}\AgdaSymbol{\}} \AgdaSymbol{\{}\AgdaBound{σ} \AgdaBound{σ'} \AgdaSymbol{:} \AgdaFunction{Op} \AgdaBound{F} \AgdaBound{V} \AgdaBound{W}\AgdaSymbol{\}} \AgdaSymbol{\{}\AgdaBound{ρ} \AgdaSymbol{:} \AgdaFunction{Op} \AgdaBound{G} \AgdaBound{U} \AgdaBound{V}\AgdaSymbol{\}} \AgdaSymbol{→}\<%
\\
\>[2]\AgdaIndent{4}{}\<[4]%
\>[4]\AgdaFunction{\_∼op\_} \AgdaBound{F} \AgdaBound{σ} \AgdaBound{σ'} \AgdaSymbol{→} \AgdaFunction{\_∼op\_} \AgdaBound{H} \AgdaSymbol{(}\AgdaBound{σ} \AgdaField{∘} \AgdaBound{ρ}\AgdaSymbol{)} \AgdaSymbol{(}\AgdaBound{σ'} \AgdaField{∘} \AgdaBound{ρ}\AgdaSymbol{)}\<%
\\
\>[0]\AgdaIndent{2}{}\<[2]%
\>[2]\AgdaFunction{comp-congl} \AgdaSymbol{\{}\AgdaBound{U}\AgdaSymbol{\}} \AgdaSymbol{\{}\AgdaBound{V}\AgdaSymbol{\}} \AgdaSymbol{\{}\AgdaBound{W}\AgdaSymbol{\}} \AgdaSymbol{=} \AgdaFunction{Bifunction.congl} \AgdaSymbol{\{}\AgdaArgument{A} \AgdaSymbol{=} \AgdaFunction{OP} \AgdaBound{F} \AgdaBound{V} \AgdaBound{W}\AgdaSymbol{\}} \AgdaSymbol{\{}\AgdaArgument{B} \AgdaSymbol{=} \AgdaFunction{OP} \AgdaBound{G} \AgdaBound{U} \AgdaBound{V}\AgdaSymbol{\}} \AgdaSymbol{\{}\AgdaArgument{C} \AgdaSymbol{=} \AgdaFunction{OP} \AgdaBound{H} \AgdaBound{U} \AgdaBound{W}\AgdaSymbol{\}} \AgdaField{\_∘\_} \AgdaFunction{comp-cong}\<%
\\
%
\\
\>[0]\AgdaIndent{2}{}\<[2]%
\>[2]\AgdaFunction{comp-congr} \AgdaSymbol{:} \AgdaSymbol{∀} \AgdaSymbol{\{}\AgdaBound{U}\AgdaSymbol{\}} \AgdaSymbol{\{}\AgdaBound{V}\AgdaSymbol{\}} \AgdaSymbol{\{}\AgdaBound{W}\AgdaSymbol{\}} \AgdaSymbol{\{}\AgdaBound{σ} \AgdaSymbol{:} \AgdaFunction{Op} \AgdaBound{F} \AgdaBound{V} \AgdaBound{W}\AgdaSymbol{\}} \AgdaSymbol{\{}\AgdaBound{ρ} \AgdaBound{ρ'} \AgdaSymbol{:} \AgdaFunction{Op} \AgdaBound{G} \AgdaBound{U} \AgdaBound{V}\AgdaSymbol{\}} \AgdaSymbol{→}\<%
\\
\>[2]\AgdaIndent{4}{}\<[4]%
\>[4]\AgdaFunction{\_∼op\_} \AgdaBound{G} \AgdaBound{ρ} \AgdaBound{ρ'} \AgdaSymbol{→} \AgdaFunction{\_∼op\_} \AgdaBound{H} \AgdaSymbol{(}\AgdaBound{σ} \AgdaField{∘} \AgdaBound{ρ}\AgdaSymbol{)} \AgdaSymbol{(}\AgdaBound{σ} \AgdaField{∘} \AgdaBound{ρ'}\AgdaSymbol{)}\<%
\\
\>[0]\AgdaIndent{2}{}\<[2]%
\>[2]\AgdaFunction{comp-congr} \AgdaSymbol{\{}\AgdaBound{U}\AgdaSymbol{\}} \AgdaSymbol{\{}\AgdaBound{V}\AgdaSymbol{\}} \AgdaSymbol{\{}\AgdaBound{W}\AgdaSymbol{\}} \AgdaSymbol{=} \AgdaFunction{Bifunction.congr} \AgdaSymbol{\{}\AgdaArgument{A} \AgdaSymbol{=} \AgdaFunction{OP} \AgdaBound{F} \AgdaBound{V} \AgdaBound{W}\AgdaSymbol{\}} \AgdaSymbol{\{}\AgdaArgument{B} \AgdaSymbol{=} \AgdaFunction{OP} \AgdaBound{G} \AgdaBound{U} \AgdaBound{V}\AgdaSymbol{\}} \AgdaSymbol{\{}\AgdaArgument{C} \AgdaSymbol{=} \AgdaFunction{OP} \AgdaBound{H} \AgdaBound{U} \AgdaBound{W}\AgdaSymbol{\}} \AgdaField{\_∘\_} \AgdaFunction{comp-cong}\<%
\end{code}
}

\begin{code}%
\>[0]\AgdaIndent{2}{}\<[2]%
\>[2]\AgdaFunction{liftsOp-comp} \AgdaSymbol{:} \AgdaSymbol{∀} \AgdaSymbol{\{}\AgdaBound{U} \AgdaBound{V} \AgdaBound{W}\AgdaSymbol{\}} \AgdaBound{A} \AgdaSymbol{\{}\AgdaBound{σ} \AgdaBound{ρ}\AgdaSymbol{\}} \AgdaSymbol{→} \<[37]%
\>[37]\<%
\\
\>[2]\AgdaIndent{4}{}\<[4]%
\>[4]\AgdaFunction{\_∼op\_} \AgdaBound{H} \AgdaSymbol{(}\AgdaFunction{liftsOp} \AgdaBound{H} \AgdaBound{A} \AgdaSymbol{(}\AgdaField{\_∘\_} \AgdaSymbol{\{}\AgdaBound{U}\AgdaSymbol{\}} \AgdaSymbol{\{}\AgdaBound{V}\AgdaSymbol{\}} \AgdaSymbol{\{}\AgdaBound{W}\AgdaSymbol{\}} \AgdaBound{σ} \AgdaBound{ρ}\AgdaSymbol{))} \<[48]%
\>[48]\<%
\\
\>[4]\AgdaIndent{6}{}\<[6]%
\>[6]\AgdaSymbol{(}\AgdaFunction{liftsOp} \AgdaBound{F} \AgdaBound{A} \AgdaBound{σ} \AgdaField{∘} \AgdaFunction{liftsOp} \AgdaBound{G} \AgdaBound{A} \AgdaBound{ρ}\AgdaSymbol{)}\<%
\end{code}

\AgdaHide{
\begin{code}%
\>[0]\AgdaIndent{2}{}\<[2]%
\>[2]\AgdaFunction{liftsOp-comp} \AgdaInductiveConstructor{[]} \AgdaSymbol{=} \AgdaFunction{∼-refl} \AgdaBound{H}\<%
\\
\>[0]\AgdaIndent{2}{}\<[2]%
\>[2]\AgdaFunction{liftsOp-comp} \AgdaSymbol{\{}\AgdaBound{U}\AgdaSymbol{\}} \AgdaSymbol{\{}\AgdaBound{V}\AgdaSymbol{\}} \AgdaSymbol{\{}\AgdaBound{W}\AgdaSymbol{\}} \AgdaSymbol{(}\AgdaBound{K} \AgdaInductiveConstructor{∷} \AgdaBound{A}\AgdaSymbol{)} \AgdaSymbol{\{}\AgdaBound{σ}\AgdaSymbol{\}} \AgdaSymbol{\{}\AgdaBound{ρ}\AgdaSymbol{\}} \AgdaSymbol{=} \AgdaKeyword{let} \AgdaKeyword{open} \AgdaModule{EqReasoning} \AgdaSymbol{(}\AgdaFunction{OP} \AgdaBound{H} \AgdaSymbol{\_} \AgdaSymbol{\_)} \AgdaKeyword{in} \<[80]%
\>[80]\<%
\\
\>[2]\AgdaIndent{4}{}\<[4]%
\>[4]\AgdaFunction{begin}\<%
\\
\>[4]\AgdaIndent{6}{}\<[6]%
\>[6]\AgdaFunction{liftsOp} \AgdaBound{H} \AgdaBound{A} \AgdaSymbol{(}\AgdaFunction{liftOp} \AgdaBound{H} \AgdaBound{K} \AgdaSymbol{(}\AgdaBound{σ} \AgdaField{∘} \AgdaBound{ρ}\AgdaSymbol{))}\<%
\\
\>[0]\AgdaIndent{4}{}\<[4]%
\>[4]\AgdaFunction{≈⟨} \AgdaFunction{liftsOp-cong} \AgdaBound{H} \AgdaBound{A} \AgdaField{liftOp-comp} \AgdaFunction{⟩}\<%
\\
\>[4]\AgdaIndent{6}{}\<[6]%
\>[6]\AgdaFunction{liftsOp} \AgdaBound{H} \AgdaBound{A} \AgdaSymbol{(}\AgdaFunction{liftOp} \AgdaBound{F} \AgdaBound{K} \AgdaBound{σ} \AgdaField{∘} \AgdaFunction{liftOp} \AgdaBound{G} \AgdaBound{K} \AgdaBound{ρ}\AgdaSymbol{)}\<%
\\
\>[0]\AgdaIndent{4}{}\<[4]%
\>[4]\AgdaFunction{≈⟨} \AgdaFunction{liftsOp-comp} \AgdaBound{A} \AgdaFunction{⟩}\<%
\\
\>[4]\AgdaIndent{6}{}\<[6]%
\>[6]\AgdaFunction{liftsOp} \AgdaBound{F} \AgdaBound{A} \AgdaSymbol{(}\AgdaFunction{liftOp} \AgdaBound{F} \AgdaBound{K} \AgdaBound{σ}\AgdaSymbol{)} \AgdaField{∘} \AgdaFunction{liftsOp} \AgdaBound{G} \AgdaBound{A} \AgdaSymbol{(}\AgdaFunction{liftOp} \AgdaBound{G} \AgdaBound{K} \AgdaBound{ρ}\AgdaSymbol{)}\<%
\\
\>[0]\AgdaIndent{4}{}\<[4]%
\>[4]\AgdaFunction{∎}\<%
\end{code}
}

\begin{code}%
\>[0]\AgdaIndent{2}{}\<[2]%
\>[2]\AgdaFunction{ap-comp} \AgdaSymbol{:} \AgdaSymbol{∀} \AgdaSymbol{\{}\AgdaBound{U} \AgdaBound{V} \AgdaBound{W} \AgdaBound{C} \AgdaBound{K}\AgdaSymbol{\}} \AgdaSymbol{(}\AgdaBound{E} \AgdaSymbol{:} \AgdaDatatype{Subexpression} \AgdaBound{U} \AgdaBound{C} \AgdaBound{K}\AgdaSymbol{)} \AgdaSymbol{\{}\AgdaBound{σ} \AgdaBound{ρ}\AgdaSymbol{\}} \AgdaSymbol{→} \<[60]%
\>[60]\<%
\\
\>[2]\AgdaIndent{4}{}\<[4]%
\>[4]\AgdaFunction{ap} \AgdaBound{H} \AgdaSymbol{(}\AgdaField{\_∘\_} \AgdaSymbol{\{}\AgdaBound{U}\AgdaSymbol{\}} \AgdaSymbol{\{}\AgdaBound{V}\AgdaSymbol{\}} \AgdaSymbol{\{}\AgdaBound{W}\AgdaSymbol{\}} \AgdaBound{σ} \AgdaBound{ρ}\AgdaSymbol{)} \AgdaBound{E} \AgdaDatatype{≡} \AgdaFunction{ap} \AgdaBound{F} \AgdaBound{σ} \AgdaSymbol{(}\AgdaFunction{ap} \AgdaBound{G} \AgdaBound{ρ} \AgdaBound{E}\AgdaSymbol{)}\<%
\end{code}

\AgdaHide{
\begin{code}%
\>[0]\AgdaIndent{2}{}\<[2]%
\>[2]\AgdaFunction{ap-comp} \AgdaSymbol{(}\AgdaInductiveConstructor{var} \AgdaSymbol{\_)} \AgdaSymbol{=} \AgdaField{apV-comp}\<%
\\
\>[0]\AgdaIndent{2}{}\<[2]%
\>[2]\AgdaFunction{ap-comp} \AgdaSymbol{(}\AgdaInductiveConstructor{app} \AgdaBound{c} \AgdaBound{E}\AgdaSymbol{)} \AgdaSymbol{=} \AgdaFunction{cong} \AgdaSymbol{(}\AgdaInductiveConstructor{app} \AgdaBound{c}\AgdaSymbol{)} \AgdaSymbol{(}\AgdaFunction{ap-comp} \AgdaBound{E}\AgdaSymbol{)}\<%
\\
\>[0]\AgdaIndent{2}{}\<[2]%
\>[2]\AgdaFunction{ap-comp} \AgdaInductiveConstructor{[]} \AgdaSymbol{=} \AgdaInductiveConstructor{refl}\<%
\\
\>[0]\AgdaIndent{2}{}\<[2]%
\>[2]\AgdaFunction{ap-comp} \AgdaSymbol{(}\AgdaInductiveConstructor{\_∷\_} \AgdaSymbol{\{}\AgdaArgument{A} \AgdaSymbol{=} \AgdaInductiveConstructor{SK} \AgdaBound{A} \AgdaSymbol{\_\}} \AgdaBound{E} \AgdaBound{E'}\AgdaSymbol{)} \AgdaSymbol{\{}\AgdaBound{σ}\AgdaSymbol{\}} \AgdaSymbol{\{}\AgdaBound{ρ}\AgdaSymbol{\}} \AgdaSymbol{=} \AgdaFunction{cong₂} \AgdaInductiveConstructor{\_∷\_}\<%
\\
\>[2]\AgdaIndent{4}{}\<[4]%
\>[4]\AgdaSymbol{(}\AgdaKeyword{let} \AgdaKeyword{open} \AgdaModule{≡-Reasoning} \AgdaKeyword{in} \<[29]%
\>[29]\<%
\\
\>[2]\AgdaIndent{4}{}\<[4]%
\>[4]\AgdaFunction{begin}\<%
\\
\>[4]\AgdaIndent{6}{}\<[6]%
\>[6]\AgdaFunction{ap} \AgdaBound{H} \AgdaSymbol{(}\AgdaFunction{liftsOp} \AgdaBound{H} \AgdaBound{A} \AgdaSymbol{(}\AgdaBound{σ} \AgdaField{∘} \AgdaBound{ρ}\AgdaSymbol{))} \AgdaBound{E}\<%
\\
\>[0]\AgdaIndent{4}{}\<[4]%
\>[4]\AgdaFunction{≡⟨} \AgdaFunction{ap-congl} \AgdaBound{H} \AgdaSymbol{(}\AgdaFunction{liftsOp-comp} \AgdaBound{A}\AgdaSymbol{)} \AgdaBound{E} \AgdaFunction{⟩}\<%
\\
\>[4]\AgdaIndent{6}{}\<[6]%
\>[6]\AgdaFunction{ap} \AgdaBound{H} \AgdaSymbol{(}\AgdaFunction{liftsOp} \AgdaBound{F} \AgdaBound{A} \AgdaBound{σ} \AgdaField{∘} \AgdaFunction{liftsOp} \AgdaBound{G} \AgdaBound{A} \AgdaBound{ρ}\AgdaSymbol{)} \AgdaBound{E}\<%
\\
\>[0]\AgdaIndent{4}{}\<[4]%
\>[4]\AgdaFunction{≡⟨} \AgdaFunction{ap-comp} \AgdaBound{E} \AgdaFunction{⟩}\<%
\\
\>[4]\AgdaIndent{6}{}\<[6]%
\>[6]\AgdaFunction{ap} \AgdaBound{F} \AgdaSymbol{(}\AgdaFunction{liftsOp} \AgdaBound{F} \AgdaBound{A} \AgdaBound{σ}\AgdaSymbol{)} \AgdaSymbol{(}\AgdaFunction{ap} \AgdaBound{G} \AgdaSymbol{(}\AgdaFunction{liftsOp} \AgdaBound{G} \AgdaBound{A} \AgdaBound{ρ}\AgdaSymbol{)} \AgdaBound{E}\AgdaSymbol{)}\<%
\\
\>[0]\AgdaIndent{4}{}\<[4]%
\>[4]\AgdaFunction{∎}\AgdaSymbol{)} \<[7]%
\>[7]\<%
\\
\>[0]\AgdaIndent{4}{}\<[4]%
\>[4]\AgdaSymbol{(}\AgdaFunction{ap-comp} \AgdaBound{E'}\AgdaSymbol{)}\<%
\end{code}
}

\begin{lm}
Let $\circ_1 : F;G \rightarrow H$ and $\circ_2 : F';G' \rightarrow H$.  If
\[ \sigma \circ_1 \rho \sim \simga' \circ_2 \rho' \]
then $E [\rho] [\sigma] \equiv E [\rho'] [\sigma']$ for every expression $E$.
\end{lm}

\begin{code}%
\>\AgdaFunction{ap-comp-sim} \AgdaSymbol{:} \AgdaSymbol{∀} \AgdaSymbol{\{}\AgdaBound{F} \AgdaBound{F'} \AgdaBound{G} \AgdaBound{G'} \AgdaBound{H}\AgdaSymbol{\}} \AgdaSymbol{(}\AgdaBound{comp₁} \AgdaSymbol{:} \AgdaRecord{Composition} \AgdaBound{F} \AgdaBound{G} \AgdaBound{H}\AgdaSymbol{)} \AgdaSymbol{(}\AgdaBound{comp₂} \AgdaSymbol{:} \AgdaRecord{Composition} \AgdaBound{F'} \AgdaBound{G'} \AgdaBound{H}\AgdaSymbol{)} \AgdaSymbol{\{}\AgdaBound{U}\AgdaSymbol{\}} \AgdaSymbol{\{}\AgdaBound{V}\AgdaSymbol{\}} \AgdaSymbol{\{}\AgdaBound{V'}\AgdaSymbol{\}} \AgdaSymbol{\{}\AgdaBound{W}\AgdaSymbol{\}}\<%
\\
\>[0]\AgdaIndent{2}{}\<[2]%
\>[2]\AgdaSymbol{\{}\AgdaBound{σ} \AgdaSymbol{:} \AgdaFunction{Op} \AgdaBound{F} \AgdaBound{V} \AgdaBound{W}\AgdaSymbol{\}} \AgdaSymbol{\{}\AgdaBound{ρ} \AgdaSymbol{:} \AgdaFunction{Op} \AgdaBound{G} \AgdaBound{U} \AgdaBound{V}\AgdaSymbol{\}} \AgdaSymbol{\{}\AgdaBound{σ'} \AgdaSymbol{:} \AgdaFunction{Op} \AgdaBound{F'} \AgdaBound{V'} \AgdaBound{W}\AgdaSymbol{\}} \AgdaSymbol{\{}\AgdaBound{ρ'} \AgdaSymbol{:} \AgdaFunction{Op} \AgdaBound{G'} \AgdaBound{U} \AgdaBound{V'}\AgdaSymbol{\}} \AgdaSymbol{→}\<%
\\
\>[0]\AgdaIndent{2}{}\<[2]%
\>[2]\AgdaFunction{\_∼op\_} \AgdaBound{H} \AgdaSymbol{(}\AgdaField{Composition.\_∘\_} \AgdaBound{comp₁} \AgdaBound{σ} \AgdaBound{ρ}\AgdaSymbol{)} \AgdaSymbol{(}\AgdaField{Composition.\_∘\_} \AgdaBound{comp₂} \AgdaBound{σ'} \AgdaBound{ρ'}\AgdaSymbol{)} \AgdaSymbol{→}\<%
\\
\>[0]\AgdaIndent{2}{}\<[2]%
\>[2]\AgdaSymbol{∀} \AgdaSymbol{\{}\AgdaBound{C}\AgdaSymbol{\}} \AgdaSymbol{\{}\AgdaBound{K}\AgdaSymbol{\}} \AgdaSymbol{(}\AgdaBound{E} \AgdaSymbol{:} \AgdaDatatype{Subexpression} \AgdaBound{U} \AgdaBound{C} \AgdaBound{K}\AgdaSymbol{)} \AgdaSymbol{→}\<%
\\
\>[0]\AgdaIndent{2}{}\<[2]%
\>[2]\AgdaFunction{ap} \AgdaBound{F} \AgdaBound{σ} \AgdaSymbol{(}\AgdaFunction{ap} \AgdaBound{G} \AgdaBound{ρ} \AgdaBound{E}\AgdaSymbol{)} \AgdaDatatype{≡} \AgdaFunction{ap} \AgdaBound{F'} \AgdaBound{σ'} \AgdaSymbol{(}\AgdaFunction{ap} \AgdaBound{G'} \AgdaBound{ρ'} \AgdaBound{E}\AgdaSymbol{)}\<%
\end{code}

\AgdaHide{
\begin{code}%
\>\AgdaFunction{ap-comp-sim} \AgdaSymbol{\{}\AgdaBound{F}\AgdaSymbol{\}} \AgdaSymbol{\{}\AgdaBound{F'}\AgdaSymbol{\}} \AgdaSymbol{\{}\AgdaBound{G}\AgdaSymbol{\}} \AgdaSymbol{\{}\AgdaBound{G'}\AgdaSymbol{\}} \AgdaSymbol{\{}\AgdaBound{H}\AgdaSymbol{\}} \AgdaBound{comp₁} \AgdaBound{comp₂} \AgdaSymbol{\{}\AgdaBound{U}\AgdaSymbol{\}} \AgdaSymbol{\{}\AgdaBound{V}\AgdaSymbol{\}} \AgdaSymbol{\{}\AgdaBound{V'}\AgdaSymbol{\}} \AgdaSymbol{\{}\AgdaBound{W}\AgdaSymbol{\}} \AgdaSymbol{\{}\AgdaBound{σ}\AgdaSymbol{\}} \AgdaSymbol{\{}\AgdaBound{ρ}\AgdaSymbol{\}} \AgdaSymbol{\{}\AgdaBound{σ'}\AgdaSymbol{\}} \AgdaSymbol{\{}\AgdaBound{ρ'}\AgdaSymbol{\}} \AgdaBound{hyp} \AgdaSymbol{\{}\AgdaBound{C}\AgdaSymbol{\}} \AgdaSymbol{\{}\AgdaBound{K}\AgdaSymbol{\}} \AgdaBound{E} \AgdaSymbol{=}\<%
\\
\>[0]\AgdaIndent{2}{}\<[2]%
\>[2]\AgdaKeyword{let} \AgdaKeyword{open} \AgdaModule{≡-Reasoning} \AgdaKeyword{in} \<[26]%
\>[26]\<%
\\
\>[0]\AgdaIndent{2}{}\<[2]%
\>[2]\AgdaFunction{begin}\<%
\\
\>[2]\AgdaIndent{4}{}\<[4]%
\>[4]\AgdaFunction{ap} \AgdaBound{F} \AgdaBound{σ} \AgdaSymbol{(}\AgdaFunction{ap} \AgdaBound{G} \AgdaBound{ρ} \AgdaBound{E}\AgdaSymbol{)}\<%
\\
\>[0]\AgdaIndent{2}{}\<[2]%
\>[2]\AgdaFunction{≡⟨⟨} \AgdaFunction{Composition.ap-comp} \AgdaBound{comp₁} \AgdaBound{E} \AgdaSymbol{\{}\AgdaBound{σ}\AgdaSymbol{\}} \AgdaSymbol{\{}\AgdaBound{ρ}\AgdaSymbol{\}} \AgdaFunction{⟩⟩}\<%
\\
\>[2]\AgdaIndent{4}{}\<[4]%
\>[4]\AgdaFunction{ap} \AgdaBound{H} \AgdaSymbol{(}\AgdaField{Composition.\_∘\_} \AgdaBound{comp₁} \AgdaBound{σ} \AgdaBound{ρ}\AgdaSymbol{)} \AgdaBound{E}\<%
\\
\>[0]\AgdaIndent{2}{}\<[2]%
\>[2]\AgdaFunction{≡⟨} \AgdaFunction{ap-congl} \AgdaBound{H} \AgdaBound{hyp} \AgdaBound{E} \AgdaFunction{⟩}\<%
\\
\>[2]\AgdaIndent{4}{}\<[4]%
\>[4]\AgdaFunction{ap} \AgdaBound{H} \AgdaSymbol{(}\AgdaField{Composition.\_∘\_} \AgdaBound{comp₂} \AgdaBound{σ'} \AgdaBound{ρ'}\AgdaSymbol{)} \AgdaBound{E}\<%
\\
\>[0]\AgdaIndent{2}{}\<[2]%
\>[2]\AgdaFunction{≡⟨} \AgdaFunction{Composition.ap-comp} \AgdaBound{comp₂} \AgdaBound{E} \AgdaSymbol{\{}\AgdaBound{σ'}\AgdaSymbol{\}} \AgdaSymbol{\{}\AgdaBound{ρ'}\AgdaSymbol{\}} \AgdaFunction{⟩}\<%
\\
\>[2]\AgdaIndent{4}{}\<[4]%
\>[4]\AgdaFunction{ap} \AgdaBound{F'} \AgdaBound{σ'} \AgdaSymbol{(}\AgdaFunction{ap} \AgdaBound{G'} \AgdaBound{ρ'} \AgdaBound{E}\AgdaSymbol{)}\<%
\\
\>[0]\AgdaIndent{2}{}\<[2]%
\>[2]\AgdaFunction{∎}\<%
\end{code}
}

\begin{lm}
Suppose there exist compositions $F;G \rightarrow H$ and $F';F \rightarrow H$.
Let $\uparrow_F$, $\uparrow_{F'}$ and $\uparrow_G$ be the lifting operations of $F$, $F'$ and $G$.
Suppose $\up_F(E) \equiv \up_{F'}(E)$ for every subexpression $E$.  Then
$\uparrow_G(E)[F \uparrow] \equiv \uparrow_{F'}(\sigma(E))$ for every subexpression $E$.
\end{lm}

\begin{code}%
\>\AgdaFunction{liftOp-up-mixed} \AgdaSymbol{:} \AgdaSymbol{∀} \AgdaSymbol{\{}\AgdaBound{F}\AgdaSymbol{\}} \AgdaSymbol{\{}\AgdaBound{G}\AgdaSymbol{\}} \AgdaSymbol{\{}\AgdaBound{H}\AgdaSymbol{\}} \AgdaSymbol{\{}\AgdaBound{F'}\AgdaSymbol{\}} \AgdaSymbol{(}\AgdaBound{comp₁} \AgdaSymbol{:} \AgdaRecord{Composition} \AgdaBound{F} \AgdaBound{G} \AgdaBound{H}\AgdaSymbol{)} \AgdaSymbol{(}\AgdaBound{comp₂} \AgdaSymbol{:} \AgdaRecord{Composition} \AgdaBound{F'} \AgdaBound{F} \AgdaBound{H}\AgdaSymbol{)}\<%
\\
\>[0]\AgdaIndent{2}{}\<[2]%
\>[2]\AgdaSymbol{\{}\AgdaBound{U}\AgdaSymbol{\}} \AgdaSymbol{\{}\AgdaBound{V}\AgdaSymbol{\}} \AgdaSymbol{\{}\AgdaBound{C}\AgdaSymbol{\}} \AgdaSymbol{\{}\AgdaBound{K}\AgdaSymbol{\}} \AgdaSymbol{\{}\AgdaBound{L}\AgdaSymbol{\}} \AgdaSymbol{\{}\AgdaBound{σ} \AgdaSymbol{:} \AgdaFunction{Op} \AgdaBound{F} \AgdaBound{U} \AgdaBound{V}\AgdaSymbol{\}} \AgdaSymbol{→}\<%
\\
\>[0]\AgdaIndent{2}{}\<[2]%
\>[2]\AgdaSymbol{(∀} \AgdaSymbol{\{}\AgdaBound{V}\AgdaSymbol{\}} \AgdaSymbol{\{}\AgdaBound{C}\AgdaSymbol{\}} \AgdaSymbol{\{}\AgdaBound{K}\AgdaSymbol{\}} \AgdaSymbol{\{}\AgdaBound{L}\AgdaSymbol{\}} \AgdaSymbol{\{}\AgdaBound{E} \AgdaSymbol{:} \AgdaDatatype{Subexpression} \AgdaBound{V} \AgdaBound{C} \AgdaBound{K}\AgdaSymbol{\}} \AgdaSymbol{→} \AgdaFunction{ap} \AgdaBound{F} \AgdaSymbol{(}\AgdaFunction{up} \AgdaBound{F} \AgdaSymbol{\{}\AgdaBound{V}\AgdaSymbol{\}} \AgdaSymbol{\{}\AgdaBound{L}\AgdaSymbol{\})} \AgdaBound{E} \AgdaDatatype{≡} \AgdaFunction{ap} \AgdaBound{F'} \AgdaSymbol{(}\AgdaFunction{up} \AgdaBound{F'} \AgdaSymbol{\{}\AgdaBound{V}\AgdaSymbol{\}} \AgdaSymbol{\{}\AgdaBound{L}\AgdaSymbol{\})} \AgdaBound{E}\AgdaSymbol{)} \AgdaSymbol{→}\<%
\\
\>[0]\AgdaIndent{2}{}\<[2]%
\>[2]\AgdaSymbol{∀} \AgdaSymbol{\{}\AgdaBound{E} \AgdaSymbol{:} \AgdaDatatype{Subexpression} \AgdaBound{U} \AgdaBound{C} \AgdaBound{K}\AgdaSymbol{\}} \AgdaSymbol{→} \AgdaFunction{ap} \AgdaBound{F} \AgdaSymbol{(}\AgdaFunction{liftOp} \AgdaBound{F} \AgdaBound{L} \AgdaBound{σ}\AgdaSymbol{)} \AgdaSymbol{(}\AgdaFunction{ap} \AgdaBound{G} \AgdaSymbol{(}\AgdaFunction{up} \AgdaBound{G}\AgdaSymbol{)} \AgdaBound{E}\AgdaSymbol{)} \AgdaDatatype{≡} \AgdaFunction{ap} \AgdaBound{F'} \AgdaSymbol{(}\AgdaFunction{up} \AgdaBound{F'}\AgdaSymbol{)} \AgdaSymbol{(}\AgdaFunction{ap} \AgdaBound{F} \AgdaBound{σ} \AgdaBound{E}\AgdaSymbol{)}\<%
\\
\>\AgdaFunction{liftOp-up-mixed} \AgdaSymbol{\{}\AgdaBound{F}\AgdaSymbol{\}} \AgdaSymbol{\{}\AgdaBound{G}\AgdaSymbol{\}} \AgdaSymbol{\{}\AgdaBound{H}\AgdaSymbol{\}} \AgdaSymbol{\{}\AgdaBound{F'}\AgdaSymbol{\}} \AgdaBound{comp₁} \AgdaBound{comp₂} \AgdaSymbol{\{}\AgdaBound{U}\AgdaSymbol{\}} \AgdaSymbol{\{}\AgdaBound{V}\AgdaSymbol{\}} \AgdaSymbol{\{}\AgdaBound{C}\AgdaSymbol{\}} \AgdaSymbol{\{}\AgdaBound{K}\AgdaSymbol{\}} \AgdaSymbol{\{}\AgdaBound{L}\AgdaSymbol{\}} \AgdaSymbol{\{}\AgdaBound{σ}\AgdaSymbol{\}} \AgdaBound{hyp} \AgdaSymbol{\{}\AgdaArgument{E} \AgdaSymbol{=} \AgdaBound{E}\AgdaSymbol{\}} \AgdaSymbol{=} \AgdaFunction{ap-comp-sim} \AgdaBound{comp₁} \AgdaBound{comp₂} \<[107]%
\>[107]\<%
\\
\>[0]\AgdaIndent{2}{}\<[2]%
\>[2]\AgdaSymbol{(λ} \AgdaBound{x} \AgdaSymbol{→} \AgdaKeyword{let} \AgdaKeyword{open} \AgdaModule{≡-Reasoning} \AgdaKeyword{in} \<[33]%
\>[33]\<%
\\
\>[0]\AgdaIndent{2}{}\<[2]%
\>[2]\AgdaFunction{begin}\<%
\\
\>[2]\AgdaIndent{4}{}\<[4]%
\>[4]\AgdaFunction{apV} \AgdaBound{H} \AgdaSymbol{(}\AgdaField{Composition.\_∘\_} \AgdaBound{comp₁} \AgdaSymbol{(}\AgdaFunction{liftOp} \AgdaBound{F} \AgdaBound{L} \AgdaBound{σ}\AgdaSymbol{)} \AgdaSymbol{(}\AgdaFunction{up} \AgdaBound{G}\AgdaSymbol{))} \AgdaBound{x}\<%
\\
\>[0]\AgdaIndent{2}{}\<[2]%
\>[2]\AgdaFunction{≡⟨} \AgdaField{Composition.apV-comp} \AgdaBound{comp₁} \AgdaFunction{⟩}\<%
\\
\>[2]\AgdaIndent{4}{}\<[4]%
\>[4]\AgdaFunction{ap} \AgdaBound{F} \AgdaSymbol{(}\AgdaFunction{liftOp} \AgdaBound{F} \AgdaBound{L} \AgdaBound{σ}\AgdaSymbol{)} \AgdaSymbol{(}\AgdaFunction{apV} \AgdaBound{G} \AgdaSymbol{(}\AgdaFunction{up} \AgdaBound{G}\AgdaSymbol{)} \AgdaBound{x}\AgdaSymbol{)}\<%
\\
\>[0]\AgdaIndent{2}{}\<[2]%
\>[2]\AgdaFunction{≡⟨} \AgdaFunction{cong} \AgdaSymbol{(}\AgdaFunction{ap} \AgdaBound{F} \AgdaSymbol{(}\AgdaFunction{liftOp} \AgdaBound{F} \AgdaBound{L} \AgdaBound{σ}\AgdaSymbol{))} \AgdaSymbol{(}\AgdaFunction{apV-up} \AgdaBound{G}\AgdaSymbol{)} \AgdaFunction{⟩}\<%
\\
\>[2]\AgdaIndent{4}{}\<[4]%
\>[4]\AgdaFunction{apV} \AgdaBound{F} \AgdaSymbol{(}\AgdaFunction{liftOp} \AgdaBound{F} \AgdaBound{L} \AgdaBound{σ}\AgdaSymbol{)} \AgdaSymbol{(}\AgdaInductiveConstructor{↑} \AgdaBound{x}\AgdaSymbol{)}\<%
\\
\>[0]\AgdaIndent{2}{}\<[2]%
\>[2]\AgdaFunction{≡⟨} \AgdaFunction{liftOp-↑} \AgdaBound{F} \AgdaBound{x} \AgdaFunction{⟩}\<%
\\
\>[2]\AgdaIndent{4}{}\<[4]%
\>[4]\AgdaFunction{ap} \AgdaBound{F} \AgdaSymbol{(}\AgdaFunction{up} \AgdaBound{F}\AgdaSymbol{)} \AgdaSymbol{(}\AgdaFunction{apV} \AgdaBound{F} \AgdaBound{σ} \AgdaBound{x}\AgdaSymbol{)}\<%
\\
\>[0]\AgdaIndent{2}{}\<[2]%
\>[2]\AgdaFunction{≡⟨} \AgdaBound{hyp} \AgdaSymbol{\{}\AgdaArgument{E} \AgdaSymbol{=} \AgdaFunction{apV} \AgdaBound{F} \AgdaBound{σ} \AgdaBound{x}\AgdaSymbol{\}}\AgdaFunction{⟩}\<%
\\
\>[2]\AgdaIndent{4}{}\<[4]%
\>[4]\AgdaFunction{ap} \AgdaBound{F'} \AgdaSymbol{(}\AgdaFunction{up} \AgdaBound{F'}\AgdaSymbol{)} \AgdaSymbol{(}\AgdaFunction{apV} \AgdaBound{F} \AgdaBound{σ} \AgdaBound{x}\AgdaSymbol{)}\<%
\\
\>[0]\AgdaIndent{2}{}\<[2]%
\>[2]\AgdaFunction{≡⟨⟨} \AgdaField{Composition.apV-comp} \AgdaBound{comp₂} \AgdaFunction{⟩⟩}\<%
\\
\>[2]\AgdaIndent{4}{}\<[4]%
\>[4]\AgdaFunction{apV} \AgdaBound{H} \AgdaSymbol{(}\AgdaField{Composition.\_∘\_} \AgdaBound{comp₂} \AgdaSymbol{(}\AgdaFunction{up} \AgdaBound{F'}\AgdaSymbol{)} \AgdaBound{σ}\AgdaSymbol{)} \AgdaBound{x}\<%
\\
\>[0]\AgdaIndent{2}{}\<[2]%
\>[2]\AgdaFunction{∎}\AgdaSymbol{)} \<[5]%
\>[5]\<%
\\
\>[0]\AgdaIndent{2}{}\<[2]%
\>[2]\AgdaBound{E}\<%
\end{code}
}

\AgdaHide{
\begin{code}%
\>\AgdaKeyword{open} \AgdaKeyword{import} \AgdaModule{Grammar.Base}\<%
\\
%
\\
\>\AgdaKeyword{module} \AgdaModule{Grammar.OpFamily.OpFamily} \AgdaSymbol{(}\AgdaBound{G} \AgdaSymbol{:} \AgdaRecord{Grammar}\AgdaSymbol{)} \AgdaKeyword{where}\<%
\\
%
\\
\>\AgdaKeyword{open} \AgdaKeyword{import} \AgdaModule{Prelims}\<%
\\
\>\AgdaKeyword{open} \AgdaModule{Grammar} \AgdaBound{G}\<%
\\
\>\AgdaKeyword{open} \AgdaKeyword{import} \AgdaModule{Grammar.OpFamily.LiftFamily} \AgdaBound{G}\<%
\\
\>\AgdaKeyword{open} \AgdaKeyword{import} \AgdaModule{Grammar.OpFamily.Composition} \AgdaBound{G}\<%
\end{code}
}

\subsubsection{Family of Operations}

Finally. we can define: a \emph{family of operations} is a pre-family with lift $F$ together with a composition $\circ : F;F \rightarrow F$.

\begin{code}%
\>\AgdaKeyword{record} \AgdaRecord{IsOpFamily} \AgdaSymbol{(}\AgdaBound{F} \AgdaSymbol{:} \AgdaRecord{LiftFamily}\AgdaSymbol{)} \AgdaSymbol{:} \AgdaPrimitiveType{Set₂} \AgdaKeyword{where}\<%
\\
\>[0]\AgdaIndent{2}{}\<[2]%
\>[2]\AgdaKeyword{open} \AgdaModule{LiftFamily} \AgdaBound{F} \AgdaKeyword{public}\<%
\\
\>[0]\AgdaIndent{2}{}\<[2]%
\>[2]\AgdaKeyword{field}\<%
\\
\>[2]\AgdaIndent{4}{}\<[4]%
\>[4]\AgdaField{comp} \AgdaSymbol{:} \AgdaRecord{Composition} \AgdaBound{F} \AgdaBound{F} \AgdaBound{F}\<%
\\
%
\\
\>\AgdaComment{\{-  infix 50 \_∘\_\<\\
\>  field\<\\
\>    \_∘\_ : ∀ \{U\} \{V\} \{W\} → Op V W → Op U V → Op U W\<\\
\>    liftOp-comp : ∀ \{U\} \{V\} \{W\} \{K\} \{σ : Op V W\} \{ρ : Op U V\} →\<\\
\>      liftOp K (σ ∘ ρ) ∼op liftOp K σ ∘ liftOp K ρ\<\\
\>    apV-comp : ∀ \{U\} \{V\} \{W\} \{K\} \{σ : Op V W\} \{ρ : Op U V\} \{x : Var U K\} →\<\\
\>      apV (σ ∘ ρ) x ≡ ap σ (apV ρ x)\<\\
\>\<\\
\>  COMP : Composition F F F\<\\
\>  COMP = record \{ \<\\
\>    circ = \_∘\_ ; \<\\
\>    liftOp-circ = liftOp-comp ; \<\\
\>    apV-circ = apV-comp \} -\}}\<%
\\
%
\\
\>[0]\AgdaIndent{2}{}\<[2]%
\>[2]\AgdaKeyword{open} \AgdaModule{Composition} \AgdaField{comp} \AgdaKeyword{public}\<%
\\
%
\\
\>[0]\AgdaIndent{2}{}\<[2]%
\>[2]\AgdaFunction{comp-congl} \AgdaSymbol{:} \AgdaSymbol{∀} \AgdaSymbol{\{}\AgdaBound{U}\AgdaSymbol{\}} \AgdaSymbol{\{}\AgdaBound{V}\AgdaSymbol{\}} \AgdaSymbol{\{}\AgdaBound{W}\AgdaSymbol{\}} \AgdaSymbol{\{}\AgdaBound{σ} \AgdaBound{σ'} \AgdaSymbol{:} \AgdaFunction{Op} \AgdaBound{V} \AgdaBound{W}\AgdaSymbol{\}} \AgdaSymbol{\{}\AgdaBound{ρ} \AgdaSymbol{:} \AgdaFunction{Op} \AgdaBound{U} \AgdaBound{V}\AgdaSymbol{\}} \AgdaSymbol{→}\<%
\\
\>[2]\AgdaIndent{4}{}\<[4]%
\>[4]\AgdaBound{σ} \AgdaFunction{∼op} \AgdaBound{σ'} \AgdaSymbol{→} \AgdaBound{σ} \AgdaFunction{∘} \AgdaBound{ρ} \AgdaFunction{∼op} \AgdaBound{σ'} \AgdaFunction{∘} \AgdaBound{ρ}\<%
\\
\>[0]\AgdaIndent{2}{}\<[2]%
\>[2]\AgdaFunction{comp-congl} \AgdaSymbol{\{}\AgdaBound{U}\AgdaSymbol{\}} \AgdaSymbol{\{}\AgdaBound{V}\AgdaSymbol{\}} \AgdaSymbol{\{}\AgdaBound{W}\AgdaSymbol{\}} \AgdaSymbol{\{}\AgdaBound{σ}\AgdaSymbol{\}} \AgdaSymbol{\{}\AgdaBound{σ'}\AgdaSymbol{\}} \AgdaSymbol{\{}\AgdaBound{ρ}\AgdaSymbol{\}} \AgdaBound{σ∼σ'} \AgdaBound{x} \AgdaSymbol{=} \AgdaKeyword{let} \AgdaKeyword{open} \AgdaModule{≡-Reasoning} \AgdaKeyword{in} \<[71]%
\>[71]\<%
\\
\>[2]\AgdaIndent{4}{}\<[4]%
\>[4]\AgdaFunction{begin}\<%
\\
\>[4]\AgdaIndent{6}{}\<[6]%
\>[6]\AgdaFunction{apV} \AgdaSymbol{(}\AgdaBound{σ} \AgdaFunction{∘} \AgdaBound{ρ}\AgdaSymbol{)} \AgdaBound{x}\<%
\\
\>[0]\AgdaIndent{4}{}\<[4]%
\>[4]\AgdaFunction{≡⟨} \AgdaFunction{apV-comp} \AgdaFunction{⟩}\<%
\\
\>[4]\AgdaIndent{6}{}\<[6]%
\>[6]\AgdaFunction{ap} \AgdaBound{σ} \AgdaSymbol{(}\AgdaFunction{apV} \AgdaBound{ρ} \AgdaBound{x}\AgdaSymbol{)}\<%
\\
\>[0]\AgdaIndent{4}{}\<[4]%
\>[4]\AgdaFunction{≡⟨} \AgdaFunction{ap-congl} \AgdaBound{σ∼σ'} \AgdaSymbol{(}\AgdaFunction{apV} \AgdaBound{ρ} \AgdaBound{x}\AgdaSymbol{)} \AgdaFunction{⟩}\<%
\\
\>[4]\AgdaIndent{6}{}\<[6]%
\>[6]\AgdaFunction{ap} \AgdaBound{σ'} \AgdaSymbol{(}\AgdaFunction{apV} \AgdaBound{ρ} \AgdaBound{x}\AgdaSymbol{)}\<%
\\
\>[0]\AgdaIndent{4}{}\<[4]%
\>[4]\AgdaFunction{≡⟨⟨} \AgdaFunction{apV-comp} \AgdaFunction{⟩⟩}\<%
\\
\>[4]\AgdaIndent{6}{}\<[6]%
\>[6]\AgdaFunction{apV} \AgdaSymbol{(}\AgdaBound{σ'} \AgdaFunction{∘} \AgdaBound{ρ}\AgdaSymbol{)} \AgdaBound{x}\<%
\\
\>[0]\AgdaIndent{4}{}\<[4]%
\>[4]\AgdaFunction{∎}\<%
\\
\>[0]\AgdaIndent{2}{}\<[2]%
\>[2]\AgdaKeyword{postulate} \AgdaPostulate{comp-congr} \AgdaSymbol{:} \AgdaSymbol{∀} \AgdaSymbol{\{}\AgdaBound{U}\AgdaSymbol{\}} \AgdaSymbol{\{}\AgdaBound{V}\AgdaSymbol{\}} \AgdaSymbol{\{}\AgdaBound{W}\AgdaSymbol{\}} \AgdaSymbol{\{}\AgdaBound{σ} \AgdaSymbol{:} \AgdaFunction{Op} \AgdaBound{V} \AgdaBound{W}\AgdaSymbol{\}} \AgdaSymbol{\{}\AgdaBound{ρ} \AgdaBound{ρ'} \AgdaSymbol{:} \AgdaFunction{Op} \AgdaBound{U} \AgdaBound{V}\AgdaSymbol{\}} \AgdaSymbol{→}\<%
\\
\>[2]\AgdaIndent{23}{}\<[23]%
\>[23]\AgdaBound{ρ} \AgdaFunction{∼op} \AgdaBound{ρ'} \AgdaSymbol{→} \AgdaBound{σ} \AgdaFunction{∘} \AgdaBound{ρ} \AgdaFunction{∼op} \AgdaBound{σ} \AgdaFunction{∘} \AgdaBound{ρ'}\<%
\end{code}

The following results about operations are easy to prove.
\begin{lemma}$ $
  \begin{enumerate}
  \item $(\sigma , K) \circ \uparrow \sim \uparrow \circ \sigma$
  \item $(\id{V} , K) \sim \id{V,K}$
  \item $\id{V}[E] \equiv E$
  \item $(\sigma \circ \rho)[E] \equiv \sigma[\rho[E]]$
  \end{enumerate}
\end{lemma}

\begin{code}%
\>[0]\AgdaIndent{2}{}\<[2]%
\>[2]\AgdaFunction{liftOp-up'} \AgdaSymbol{:} \AgdaSymbol{∀} \AgdaSymbol{\{}\AgdaBound{U}\AgdaSymbol{\}} \AgdaSymbol{\{}\AgdaBound{V}\AgdaSymbol{\}} \AgdaSymbol{\{}\AgdaBound{C}\AgdaSymbol{\}} \AgdaSymbol{\{}\AgdaBound{K}\AgdaSymbol{\}} \AgdaSymbol{\{}\AgdaBound{L}\AgdaSymbol{\}}\<%
\\
\>[2]\AgdaIndent{4}{}\<[4]%
\>[4]\AgdaSymbol{\{}\AgdaBound{σ} \AgdaSymbol{:} \AgdaFunction{Op} \AgdaBound{U} \AgdaBound{V}\AgdaSymbol{\}} \AgdaSymbol{(}\AgdaBound{E} \AgdaSymbol{:} \AgdaDatatype{Subexpression} \AgdaBound{U} \AgdaBound{C} \AgdaBound{K}\AgdaSymbol{)} \AgdaSymbol{→}\<%
\\
\>[2]\AgdaIndent{4}{}\<[4]%
\>[4]\AgdaFunction{ap} \AgdaSymbol{(}\AgdaFunction{liftOp} \AgdaBound{L} \AgdaBound{σ}\AgdaSymbol{)} \AgdaSymbol{(}\AgdaFunction{ap} \AgdaFunction{up} \AgdaBound{E}\AgdaSymbol{)} \AgdaDatatype{≡} \AgdaFunction{ap} \AgdaFunction{up} \AgdaSymbol{(}\AgdaFunction{ap} \AgdaBound{σ} \AgdaBound{E}\AgdaSymbol{)}\<%
\end{code}

\AgdaHide{
\begin{code}%
\>[0]\AgdaIndent{2}{}\<[2]%
\>[2]\AgdaFunction{liftOp-up'} \AgdaBound{E} \AgdaSymbol{=} \AgdaFunction{liftOp-up-mixed} \AgdaField{comp} \AgdaField{comp} \AgdaInductiveConstructor{refl} \AgdaSymbol{\{}\AgdaArgument{E} \AgdaSymbol{=} \AgdaBound{E}\AgdaSymbol{\}}\<%
\end{code}
}

\newcommand{\Op}{\ensuremath{\mathbf{Op}}}

The alphabets and operations up to equivalence form
a category, which we denote $\Op$.
The action of application associates, with every operator family, a functor $\Op \rightarrow \Set$,
which maps an alphabet $U$ to the set of expressions over $U$, and every operation $\sigma$ to the function $\sigma[-]$.
This functor is faithful and injective on objects, and so $\Op$ can be seen as a subcategory of $\Set$.

\begin{code}%
\>[0]\AgdaIndent{2}{}\<[2]%
\>[2]\AgdaFunction{assoc} \AgdaSymbol{:} \AgdaSymbol{∀} \AgdaSymbol{\{}\AgdaBound{U}\AgdaSymbol{\}} \AgdaSymbol{\{}\AgdaBound{V}\AgdaSymbol{\}} \AgdaSymbol{\{}\AgdaBound{W}\AgdaSymbol{\}} \AgdaSymbol{\{}\AgdaBound{X}\AgdaSymbol{\}} \<[28]%
\>[28]\<%
\\
\>[2]\AgdaIndent{4}{}\<[4]%
\>[4]\AgdaSymbol{\{}\AgdaBound{τ} \AgdaSymbol{:} \AgdaFunction{Op} \AgdaBound{W} \AgdaBound{X}\AgdaSymbol{\}} \AgdaSymbol{\{}\AgdaBound{σ} \AgdaSymbol{:} \AgdaFunction{Op} \AgdaBound{V} \AgdaBound{W}\AgdaSymbol{\}} \AgdaSymbol{\{}\AgdaBound{ρ} \AgdaSymbol{:} \AgdaFunction{Op} \AgdaBound{U} \AgdaBound{V}\AgdaSymbol{\}} \AgdaSymbol{→} \<[45]%
\>[45]\<%
\\
\>[2]\AgdaIndent{4}{}\<[4]%
\>[4]\AgdaBound{τ} \AgdaFunction{∘} \AgdaSymbol{(}\AgdaBound{σ} \AgdaFunction{∘} \AgdaBound{ρ}\AgdaSymbol{)} \AgdaFunction{∼op} \AgdaSymbol{(}\AgdaBound{τ} \AgdaFunction{∘} \AgdaBound{σ}\AgdaSymbol{)} \AgdaFunction{∘} \AgdaBound{ρ}\<%
\end{code}

\AgdaHide{
\begin{code}%
\>[0]\AgdaIndent{2}{}\<[2]%
\>[2]\AgdaFunction{assoc} \AgdaSymbol{\{}\AgdaBound{U}\AgdaSymbol{\}} \AgdaSymbol{\{}\AgdaBound{V}\AgdaSymbol{\}} \AgdaSymbol{\{}\AgdaBound{W}\AgdaSymbol{\}} \AgdaSymbol{\{}\AgdaBound{X}\AgdaSymbol{\}} \AgdaSymbol{\{}\AgdaBound{τ}\AgdaSymbol{\}} \AgdaSymbol{\{}\AgdaBound{σ}\AgdaSymbol{\}} \AgdaSymbol{\{}\AgdaBound{ρ}\AgdaSymbol{\}} \AgdaSymbol{\{}\AgdaBound{K}\AgdaSymbol{\}} \AgdaBound{x} \AgdaSymbol{=} \AgdaKeyword{let} \AgdaKeyword{open} \AgdaModule{≡-Reasoning} \AgdaSymbol{\{}\AgdaArgument{A} \AgdaSymbol{=} \AgdaFunction{Expression} \AgdaBound{X} \AgdaSymbol{(}\AgdaInductiveConstructor{varKind} \AgdaBound{K}\AgdaSymbol{)\}} \AgdaKeyword{in} \<[99]%
\>[99]\<%
\\
\>[2]\AgdaIndent{4}{}\<[4]%
\>[4]\AgdaFunction{begin} \<[10]%
\>[10]\<%
\\
\>[4]\AgdaIndent{6}{}\<[6]%
\>[6]\AgdaFunction{apV} \AgdaSymbol{(}\AgdaBound{τ} \AgdaFunction{∘} \AgdaSymbol{(}\AgdaBound{σ} \AgdaFunction{∘} \AgdaBound{ρ}\AgdaSymbol{))} \AgdaBound{x}\<%
\\
\>[0]\AgdaIndent{4}{}\<[4]%
\>[4]\AgdaFunction{≡⟨} \AgdaFunction{apV-comp} \AgdaFunction{⟩}\<%
\\
\>[4]\AgdaIndent{6}{}\<[6]%
\>[6]\AgdaFunction{ap} \AgdaBound{τ} \AgdaSymbol{(}\AgdaFunction{apV} \AgdaSymbol{(}\AgdaBound{σ} \AgdaFunction{∘} \AgdaBound{ρ}\AgdaSymbol{)} \AgdaBound{x}\AgdaSymbol{)}\<%
\\
\>[0]\AgdaIndent{4}{}\<[4]%
\>[4]\AgdaFunction{≡⟨} \AgdaFunction{cong} \AgdaSymbol{(}\AgdaFunction{ap} \AgdaBound{τ}\AgdaSymbol{)} \AgdaFunction{apV-comp} \AgdaFunction{⟩}\<%
\\
\>[4]\AgdaIndent{6}{}\<[6]%
\>[6]\AgdaFunction{ap} \AgdaBound{τ} \AgdaSymbol{(}\AgdaFunction{ap} \AgdaBound{σ} \AgdaSymbol{(}\AgdaFunction{apV} \AgdaBound{ρ} \AgdaBound{x}\AgdaSymbol{))}\<%
\\
\>[0]\AgdaIndent{4}{}\<[4]%
\>[4]\AgdaFunction{≡⟨⟨} \AgdaFunction{ap-comp} \AgdaSymbol{(}\AgdaFunction{apV} \AgdaBound{ρ} \AgdaBound{x}\AgdaSymbol{)} \AgdaFunction{⟩⟩}\<%
\\
\>[4]\AgdaIndent{6}{}\<[6]%
\>[6]\AgdaFunction{ap} \AgdaSymbol{(}\AgdaBound{τ} \AgdaFunction{∘} \AgdaBound{σ}\AgdaSymbol{)} \AgdaSymbol{(}\AgdaFunction{apV} \AgdaBound{ρ} \AgdaBound{x}\AgdaSymbol{)}\<%
\\
\>[0]\AgdaIndent{4}{}\<[4]%
\>[4]\AgdaFunction{≡⟨⟨} \AgdaFunction{apV-comp} \AgdaFunction{⟩⟩}\<%
\\
\>[4]\AgdaIndent{6}{}\<[6]%
\>[6]\AgdaFunction{apV} \AgdaSymbol{((}\AgdaBound{τ} \AgdaFunction{∘} \AgdaBound{σ}\AgdaSymbol{)} \AgdaFunction{∘} \AgdaBound{ρ}\AgdaSymbol{)} \AgdaBound{x}\<%
\\
\>[0]\AgdaIndent{4}{}\<[4]%
\>[4]\AgdaFunction{∎}\<%
\end{code}
}

\begin{code}%
\>[0]\AgdaIndent{2}{}\<[2]%
\>[2]\AgdaFunction{unitl} \AgdaSymbol{:} \AgdaSymbol{∀} \AgdaSymbol{\{}\AgdaBound{U}\AgdaSymbol{\}} \AgdaSymbol{\{}\AgdaBound{V}\AgdaSymbol{\}} \AgdaSymbol{\{}\AgdaBound{σ} \AgdaSymbol{:} \AgdaFunction{Op} \AgdaBound{U} \AgdaBound{V}\AgdaSymbol{\}} \AgdaSymbol{→} \AgdaFunction{idOp} \AgdaBound{V} \AgdaFunction{∘} \AgdaBound{σ} \AgdaFunction{∼op} \AgdaBound{σ}\<%
\end{code}

\AgdaHide{
\begin{code}%
\>[0]\AgdaIndent{2}{}\<[2]%
\>[2]\AgdaFunction{unitl} \AgdaSymbol{\{}\AgdaBound{U}\AgdaSymbol{\}} \AgdaSymbol{\{}\AgdaBound{V}\AgdaSymbol{\}} \AgdaSymbol{\{}\AgdaBound{σ}\AgdaSymbol{\}} \AgdaSymbol{\{}\AgdaBound{K}\AgdaSymbol{\}} \AgdaBound{x} \AgdaSymbol{=} \AgdaKeyword{let} \AgdaKeyword{open} \AgdaModule{≡-Reasoning} \AgdaSymbol{\{}\AgdaArgument{A} \AgdaSymbol{=} \AgdaFunction{Expression} \AgdaBound{V} \AgdaSymbol{(}\AgdaInductiveConstructor{varKind} \AgdaBound{K}\AgdaSymbol{)\}} \AgdaKeyword{in} \<[83]%
\>[83]\<%
\\
\>[2]\AgdaIndent{4}{}\<[4]%
\>[4]\AgdaFunction{begin} \<[10]%
\>[10]\<%
\\
\>[4]\AgdaIndent{6}{}\<[6]%
\>[6]\AgdaFunction{apV} \AgdaSymbol{(}\AgdaFunction{idOp} \AgdaBound{V} \AgdaFunction{∘} \AgdaBound{σ}\AgdaSymbol{)} \AgdaBound{x}\<%
\\
\>[0]\AgdaIndent{4}{}\<[4]%
\>[4]\AgdaFunction{≡⟨} \AgdaFunction{apV-comp} \AgdaFunction{⟩}\<%
\\
\>[4]\AgdaIndent{6}{}\<[6]%
\>[6]\AgdaFunction{ap} \AgdaSymbol{(}\AgdaFunction{idOp} \AgdaBound{V}\AgdaSymbol{)} \AgdaSymbol{(}\AgdaFunction{apV} \AgdaBound{σ} \AgdaBound{x}\AgdaSymbol{)}\<%
\\
\>[0]\AgdaIndent{4}{}\<[4]%
\>[4]\AgdaFunction{≡⟨} \AgdaFunction{ap-idOp} \AgdaFunction{⟩}\<%
\\
\>[4]\AgdaIndent{6}{}\<[6]%
\>[6]\AgdaFunction{apV} \AgdaBound{σ} \AgdaBound{x}\<%
\\
\>[0]\AgdaIndent{4}{}\<[4]%
\>[4]\AgdaFunction{∎}\<%
\end{code}
}

\begin{code}%
\>[0]\AgdaIndent{2}{}\<[2]%
\>[2]\AgdaFunction{unitr} \AgdaSymbol{:} \AgdaSymbol{∀} \AgdaSymbol{\{}\AgdaBound{U}\AgdaSymbol{\}} \AgdaSymbol{\{}\AgdaBound{V}\AgdaSymbol{\}} \AgdaSymbol{\{}\AgdaBound{σ} \AgdaSymbol{:} \AgdaFunction{Op} \AgdaBound{U} \AgdaBound{V}\AgdaSymbol{\}} \AgdaSymbol{→} \AgdaBound{σ} \AgdaFunction{∘} \AgdaFunction{idOp} \AgdaBound{U} \AgdaFunction{∼op} \AgdaBound{σ}\<%
\end{code}

\AgdaHide{
\begin{code}%
\>[0]\AgdaIndent{2}{}\<[2]%
\>[2]\AgdaFunction{unitr} \AgdaSymbol{\{}\AgdaBound{U}\AgdaSymbol{\}} \AgdaSymbol{\{}\AgdaBound{V}\AgdaSymbol{\}} \AgdaSymbol{\{}\AgdaBound{σ}\AgdaSymbol{\}} \AgdaSymbol{\{}\AgdaBound{K}\AgdaSymbol{\}} \AgdaBound{x} \AgdaSymbol{=} \AgdaKeyword{let} \AgdaKeyword{open} \AgdaModule{≡-Reasoning} \AgdaSymbol{\{}\AgdaArgument{A} \AgdaSymbol{=} \AgdaFunction{Expression} \AgdaBound{V} \AgdaSymbol{(}\AgdaInductiveConstructor{varKind} \AgdaBound{K}\AgdaSymbol{)\}} \AgdaKeyword{in}\<%
\\
\>[2]\AgdaIndent{4}{}\<[4]%
\>[4]\AgdaFunction{begin} \<[10]%
\>[10]\<%
\\
\>[4]\AgdaIndent{6}{}\<[6]%
\>[6]\AgdaFunction{apV} \AgdaSymbol{(}\AgdaBound{σ} \AgdaFunction{∘} \AgdaFunction{idOp} \AgdaBound{U}\AgdaSymbol{)} \AgdaBound{x}\<%
\\
\>[0]\AgdaIndent{4}{}\<[4]%
\>[4]\AgdaFunction{≡⟨} \AgdaFunction{apV-comp} \AgdaFunction{⟩}\<%
\\
\>[4]\AgdaIndent{6}{}\<[6]%
\>[6]\AgdaFunction{ap} \AgdaBound{σ} \AgdaSymbol{(}\AgdaFunction{apV} \AgdaSymbol{(}\AgdaFunction{idOp} \AgdaBound{U}\AgdaSymbol{)} \AgdaBound{x}\AgdaSymbol{)}\<%
\\
\>[0]\AgdaIndent{4}{}\<[4]%
\>[4]\AgdaFunction{≡⟨} \AgdaFunction{cong} \AgdaSymbol{(}\AgdaFunction{ap} \AgdaBound{σ}\AgdaSymbol{)} \AgdaSymbol{(}\AgdaFunction{apV-idOp} \AgdaBound{x}\AgdaSymbol{)} \AgdaFunction{⟩}\<%
\\
\>[4]\AgdaIndent{6}{}\<[6]%
\>[6]\AgdaFunction{apV} \AgdaBound{σ} \AgdaBound{x}\<%
\\
\>[0]\AgdaIndent{4}{}\<[4]%
\>[4]\AgdaFunction{∎}\<%
\end{code}
}

\AgdaHide{
\begin{code}%
\>\AgdaKeyword{record} \AgdaRecord{OpFamily} \AgdaSymbol{:} \AgdaPrimitiveType{Set₂} \AgdaKeyword{where}\<%
\\
\>[0]\AgdaIndent{2}{}\<[2]%
\>[2]\AgdaKeyword{field}\<%
\\
\>[2]\AgdaIndent{4}{}\<[4]%
\>[4]\AgdaField{liftFamily} \AgdaSymbol{:} \AgdaRecord{LiftFamily}\<%
\\
\>[2]\AgdaIndent{4}{}\<[4]%
\>[4]\AgdaField{isOpFamily} \<[16]%
\>[16]\AgdaSymbol{:} \AgdaRecord{IsOpFamily} \AgdaField{liftFamily}\<%
\\
\>[0]\AgdaIndent{2}{}\<[2]%
\>[2]\AgdaKeyword{open} \AgdaModule{IsOpFamily} \AgdaField{isOpFamily} \AgdaKeyword{public}\<%
\end{code}
}


\AgdaHide{
\begin{code}%
\>\AgdaKeyword{open} \AgdaKeyword{import} \AgdaModule{Grammar.Base}\<%
\\
%
\\
\>\AgdaKeyword{module} \AgdaModule{Grammar.Replacement} \AgdaSymbol{(}\AgdaBound{G} \AgdaSymbol{:} \AgdaRecord{Grammar}\AgdaSymbol{)} \AgdaKeyword{where}\<%
\\
%
\\
\>\AgdaKeyword{open} \AgdaKeyword{import} \AgdaModule{Function}\<%
\\
\>\AgdaKeyword{open} \AgdaKeyword{import} \AgdaModule{Prelims}\<%
\\
\>\AgdaKeyword{open} \AgdaModule{Grammar} \AgdaBound{G}\<%
\\
\>\AgdaKeyword{open} \AgdaKeyword{import} \AgdaModule{Grammar.OpFamily.PreOpFamily} \AgdaBound{G}\<%
\\
\>\AgdaKeyword{open} \AgdaKeyword{import} \AgdaModule{Grammar.OpFamily.LiftFamily} \AgdaBound{G}\<%
\\
\>\AgdaKeyword{open} \AgdaKeyword{import} \AgdaModule{Grammar.OpFamily.OpFamily} \AgdaBound{G}\<%
\end{code}
}

\subsection{Replacement}

The operation family of \emph{replacement} is defined as follows.  A replacement $\rho : U \rightarrow V$ is a function
that maps every variable in $U$ to a variable in $V$ of the same kind.  Application, identity and composition are simply
function application, the identity function and function composition.  The successor is the canonical injection $V \rightarrow (V, K)$,
and $(\sigma , K)$ is the extension of $\sigma$ that maps $x_0$ to $x_0$.

\begin{code}%
\>\AgdaFunction{Rep} \AgdaSymbol{:} \AgdaDatatype{Alphabet} \AgdaSymbol{→} \AgdaDatatype{Alphabet} \AgdaSymbol{→} \AgdaPrimitiveType{Set}\<%
\\
\>\AgdaFunction{Rep} \AgdaBound{U} \AgdaBound{V} \AgdaSymbol{=} \AgdaSymbol{∀} \AgdaBound{K} \AgdaSymbol{→} \AgdaDatatype{Var} \AgdaBound{U} \AgdaBound{K} \AgdaSymbol{→} \AgdaDatatype{Var} \AgdaBound{V} \AgdaBound{K}\<%
\\
%
\\
\>\AgdaFunction{rep↑} \AgdaSymbol{:} \AgdaSymbol{∀} \AgdaSymbol{\{}\AgdaBound{U}\AgdaSymbol{\}} \AgdaSymbol{\{}\AgdaBound{V}\AgdaSymbol{\}} \AgdaBound{K} \AgdaSymbol{→} \AgdaFunction{Rep} \AgdaBound{U} \AgdaBound{V} \AgdaSymbol{→} \AgdaFunction{Rep} \AgdaSymbol{(}\AgdaBound{U} \AgdaInductiveConstructor{,} \AgdaBound{K}\AgdaSymbol{)} \AgdaSymbol{(}\AgdaBound{V} \AgdaInductiveConstructor{,} \AgdaBound{K}\AgdaSymbol{)}\<%
\\
\>\AgdaFunction{rep↑} \AgdaSymbol{\_} \AgdaSymbol{\_} \AgdaSymbol{\_} \AgdaInductiveConstructor{x₀} \AgdaSymbol{=} \AgdaInductiveConstructor{x₀}\<%
\\
\>\AgdaFunction{rep↑} \AgdaSymbol{\_} \AgdaBound{ρ} \AgdaBound{K} \AgdaSymbol{(}\AgdaInductiveConstructor{↑} \AgdaBound{x}\AgdaSymbol{)} \AgdaSymbol{=} \AgdaInductiveConstructor{↑} \AgdaSymbol{(}\AgdaBound{ρ} \AgdaBound{K} \AgdaBound{x}\AgdaSymbol{)}\<%
\\
%
\\
\>\AgdaFunction{upRep} \AgdaSymbol{:} \AgdaSymbol{∀} \AgdaSymbol{\{}\AgdaBound{V}\AgdaSymbol{\}} \AgdaSymbol{\{}\AgdaBound{K}\AgdaSymbol{\}} \AgdaSymbol{→} \AgdaFunction{Rep} \AgdaBound{V} \AgdaSymbol{(}\AgdaBound{V} \AgdaInductiveConstructor{,} \AgdaBound{K}\AgdaSymbol{)}\<%
\\
\>\AgdaFunction{upRep} \AgdaSymbol{\_} \AgdaSymbol{=} \AgdaInductiveConstructor{↑}\<%
\\
%
\\
\>\AgdaFunction{idRep} \AgdaSymbol{:} \AgdaSymbol{∀} \AgdaBound{V} \AgdaSymbol{→} \AgdaFunction{Rep} \AgdaBound{V} \AgdaBound{V}\<%
\\
\>\AgdaFunction{idRep} \AgdaSymbol{\_} \AgdaSymbol{\_} \AgdaBound{x} \AgdaSymbol{=} \AgdaBound{x}\<%
\\
%
\\
\>\AgdaFunction{pre-replacement} \AgdaSymbol{:} \AgdaRecord{PreOpFamily}\<%
\\
\>\AgdaFunction{pre-replacement} \AgdaSymbol{=} \AgdaKeyword{record} \AgdaSymbol{\{} \<[27]%
\>[27]\<%
\\
\>[0]\AgdaIndent{2}{}\<[2]%
\>[2]\AgdaField{Op} \AgdaSymbol{=} \AgdaFunction{Rep}\AgdaSymbol{;} \<[12]%
\>[12]\<%
\\
\>[0]\AgdaIndent{2}{}\<[2]%
\>[2]\AgdaField{apV} \AgdaSymbol{=} \AgdaSymbol{λ} \AgdaBound{ρ} \AgdaBound{x} \AgdaSymbol{→} \AgdaInductiveConstructor{var} \AgdaSymbol{(}\AgdaBound{ρ} \AgdaSymbol{\_} \AgdaBound{x}\AgdaSymbol{);} \<[29]%
\>[29]\<%
\\
\>[0]\AgdaIndent{2}{}\<[2]%
\>[2]\AgdaField{up} \AgdaSymbol{=} \AgdaFunction{upRep}\AgdaSymbol{;} \<[14]%
\>[14]\<%
\\
\>[0]\AgdaIndent{2}{}\<[2]%
\>[2]\AgdaField{apV-up} \AgdaSymbol{=} \AgdaInductiveConstructor{refl}\AgdaSymbol{;} \<[17]%
\>[17]\<%
\\
\>[0]\AgdaIndent{2}{}\<[2]%
\>[2]\AgdaField{idOp} \AgdaSymbol{=} \AgdaFunction{idRep}\AgdaSymbol{;} \<[16]%
\>[16]\<%
\\
\>[0]\AgdaIndent{2}{}\<[2]%
\>[2]\AgdaField{apV-idOp} \AgdaSymbol{=} \AgdaSymbol{λ} \AgdaBound{\_} \AgdaSymbol{→} \AgdaInductiveConstructor{refl} \AgdaSymbol{\}}\<%
\\
%
\\
\>\AgdaFunction{\_∼R\_} \AgdaSymbol{:} \AgdaSymbol{∀} \AgdaSymbol{\{}\AgdaBound{U}\AgdaSymbol{\}} \AgdaSymbol{\{}\AgdaBound{V}\AgdaSymbol{\}} \AgdaSymbol{→} \AgdaFunction{Rep} \AgdaBound{U} \AgdaBound{V} \AgdaSymbol{→} \AgdaFunction{Rep} \AgdaBound{U} \AgdaBound{V} \AgdaSymbol{→} \AgdaPrimitiveType{Set}\<%
\\
\>\AgdaFunction{\_∼R\_} \AgdaSymbol{=} \AgdaFunction{PreOpFamily.\_∼op\_} \AgdaFunction{pre-replacement}\<%
\\
%
\\
\>\AgdaFunction{rep↑-cong} \AgdaSymbol{:} \AgdaSymbol{∀} \AgdaSymbol{\{}\AgdaBound{U}\AgdaSymbol{\}} \AgdaSymbol{\{}\AgdaBound{V}\AgdaSymbol{\}} \AgdaSymbol{\{}\AgdaBound{K}\AgdaSymbol{\}} \AgdaSymbol{\{}\AgdaBound{ρ} \AgdaBound{ρ'} \AgdaSymbol{:} \AgdaFunction{Rep} \AgdaBound{U} \AgdaBound{V}\AgdaSymbol{\}} \AgdaSymbol{→} \<[45]%
\>[45]\<%
\\
\>[0]\AgdaIndent{2}{}\<[2]%
\>[2]\AgdaBound{ρ} \AgdaFunction{∼R} \AgdaBound{ρ'} \AgdaSymbol{→} \AgdaFunction{rep↑} \AgdaBound{K} \AgdaBound{ρ} \AgdaFunction{∼R} \AgdaFunction{rep↑} \AgdaBound{K} \AgdaBound{ρ'}\<%
\end{code}

\AgdaHide{
\begin{code}%
\>\AgdaFunction{rep↑-cong} \AgdaBound{ρ-is-ρ'} \AgdaInductiveConstructor{x₀} \AgdaSymbol{=} \AgdaInductiveConstructor{refl}\<%
\\
\>\AgdaFunction{rep↑-cong} \AgdaBound{ρ-is-ρ'} \AgdaSymbol{(}\AgdaInductiveConstructor{↑} \AgdaBound{x}\AgdaSymbol{)} \AgdaSymbol{=} \AgdaFunction{cong} \AgdaSymbol{(}\AgdaInductiveConstructor{var} \AgdaFunction{∘} \AgdaInductiveConstructor{↑}\AgdaSymbol{)} \AgdaSymbol{(}\AgdaFunction{var-inj} \AgdaSymbol{(}\AgdaBound{ρ-is-ρ'} \AgdaBound{x}\AgdaSymbol{))}\<%
\end{code}
}

\begin{code}%
\>\AgdaFunction{proto-replacement} \AgdaSymbol{:} \AgdaRecord{LiftFamily}\<%
\\
\>\AgdaFunction{proto-replacement} \AgdaSymbol{=} \AgdaKeyword{record} \AgdaSymbol{\{} \<[29]%
\>[29]\<%
\\
\>[0]\AgdaIndent{2}{}\<[2]%
\>[2]\AgdaField{preOpFamily} \AgdaSymbol{=} \AgdaFunction{pre-replacement} \AgdaSymbol{;} \<[34]%
\>[34]\<%
\\
\>[0]\AgdaIndent{2}{}\<[2]%
\>[2]\AgdaField{lifting} \AgdaSymbol{=} \AgdaKeyword{record} \AgdaSymbol{\{} \<[21]%
\>[21]\<%
\\
\>[2]\AgdaIndent{4}{}\<[4]%
\>[4]\AgdaField{liftOp} \AgdaSymbol{=} \AgdaFunction{rep↑} \AgdaSymbol{;} \<[20]%
\>[20]\<%
\\
\>[2]\AgdaIndent{4}{}\<[4]%
\>[4]\AgdaField{liftOp-cong} \AgdaSymbol{=} \AgdaFunction{rep↑-cong} \AgdaSymbol{\}} \AgdaSymbol{;} \<[32]%
\>[32]\<%
\\
\>[0]\AgdaIndent{2}{}\<[2]%
\>[2]\AgdaField{isLiftFamily} \AgdaSymbol{=} \AgdaKeyword{record} \AgdaSymbol{\{} \<[26]%
\>[26]\<%
\\
\>[2]\AgdaIndent{4}{}\<[4]%
\>[4]\AgdaField{liftOp-x₀} \AgdaSymbol{=} \AgdaInductiveConstructor{refl} \AgdaSymbol{;} \<[23]%
\>[23]\<%
\\
\>[2]\AgdaIndent{4}{}\<[4]%
\>[4]\AgdaField{liftOp-↑} \AgdaSymbol{=} \AgdaSymbol{λ} \AgdaBound{\_} \AgdaSymbol{→} \AgdaInductiveConstructor{refl} \AgdaSymbol{\}} \AgdaSymbol{\}}\<%
\\
%
\\
\>\AgdaKeyword{infix} \AgdaNumber{70} \AgdaFixityOp{\_〈\_〉}\<%
\\
\>\AgdaFunction{\_〈\_〉} \AgdaSymbol{:} \AgdaSymbol{∀} \AgdaSymbol{\{}\AgdaBound{U}\AgdaSymbol{\}} \AgdaSymbol{\{}\AgdaBound{V}\AgdaSymbol{\}} \AgdaSymbol{\{}\AgdaBound{C}\AgdaSymbol{\}} \AgdaSymbol{\{}\AgdaBound{K}\AgdaSymbol{\}} \AgdaSymbol{→} \<[27]%
\>[27]\<%
\\
\>[0]\AgdaIndent{2}{}\<[2]%
\>[2]\AgdaDatatype{Subexpression} \AgdaBound{U} \AgdaBound{C} \AgdaBound{K} \AgdaSymbol{→} \AgdaFunction{Rep} \AgdaBound{U} \AgdaBound{V} \AgdaSymbol{→} \AgdaDatatype{Subexpression} \AgdaBound{V} \AgdaBound{C} \AgdaBound{K}\<%
\\
\>\AgdaBound{E} \AgdaFunction{〈} \AgdaBound{ρ} \AgdaFunction{〉} \AgdaSymbol{=} \AgdaFunction{LiftFamily.ap} \AgdaFunction{proto-replacement} \AgdaBound{ρ} \AgdaBound{E}\<%
\\
%
\\
\>\AgdaKeyword{infixl} \AgdaNumber{75} \AgdaFixityOp{\_•R\_}\<%
\\
\>\AgdaFunction{\_•R\_} \AgdaSymbol{:} \AgdaSymbol{∀} \AgdaSymbol{\{}\AgdaBound{U}\AgdaSymbol{\}} \AgdaSymbol{\{}\AgdaBound{V}\AgdaSymbol{\}} \AgdaSymbol{\{}\AgdaBound{W}\AgdaSymbol{\}} \AgdaSymbol{→} \AgdaFunction{Rep} \AgdaBound{V} \AgdaBound{W} \AgdaSymbol{→} \AgdaFunction{Rep} \AgdaBound{U} \AgdaBound{V} \AgdaSymbol{→} \AgdaFunction{Rep} \AgdaBound{U} \AgdaBound{W}\<%
\\
\>\AgdaSymbol{(}\AgdaBound{ρ'} \AgdaFunction{•R} \AgdaBound{ρ}\AgdaSymbol{)} \AgdaBound{K} \AgdaBound{x} \AgdaSymbol{=} \AgdaBound{ρ'} \AgdaBound{K} \AgdaSymbol{(}\AgdaBound{ρ} \AgdaBound{K} \AgdaBound{x}\AgdaSymbol{)}\<%
\\
%
\\
\>\AgdaFunction{rep↑-comp} \AgdaSymbol{:} \AgdaSymbol{∀} \AgdaSymbol{\{}\AgdaBound{U}\AgdaSymbol{\}} \AgdaSymbol{\{}\AgdaBound{V}\AgdaSymbol{\}} \AgdaSymbol{\{}\AgdaBound{W}\AgdaSymbol{\}} \AgdaSymbol{\{}\AgdaBound{K}\AgdaSymbol{\}} \AgdaSymbol{\{}\AgdaBound{ρ'} \AgdaSymbol{:} \AgdaFunction{Rep} \AgdaBound{V} \AgdaBound{W}\AgdaSymbol{\}} \AgdaSymbol{\{}\AgdaBound{ρ} \AgdaSymbol{:} \AgdaFunction{Rep} \AgdaBound{U} \AgdaBound{V}\AgdaSymbol{\}} \AgdaSymbol{→} \<[61]%
\>[61]\<%
\\
\>[0]\AgdaIndent{2}{}\<[2]%
\>[2]\AgdaFunction{rep↑} \AgdaBound{K} \AgdaSymbol{(}\AgdaBound{ρ'} \AgdaFunction{•R} \AgdaBound{ρ}\AgdaSymbol{)} \AgdaFunction{∼R} \AgdaFunction{rep↑} \AgdaBound{K} \AgdaBound{ρ'} \AgdaFunction{•R} \AgdaFunction{rep↑} \AgdaBound{K} \AgdaBound{ρ}\<%
\end{code}

\AgdaHide{
\begin{code}%
\>\AgdaFunction{rep↑-comp} \AgdaInductiveConstructor{x₀} \AgdaSymbol{=} \AgdaInductiveConstructor{refl}\<%
\\
\>\AgdaFunction{rep↑-comp} \AgdaSymbol{(}\AgdaInductiveConstructor{↑} \AgdaSymbol{\_)} \AgdaSymbol{=} \AgdaInductiveConstructor{refl}\<%
\\
%
\\
\>\AgdaKeyword{postulate} \AgdaPostulate{rep↑-comp₄} \AgdaSymbol{:} \AgdaSymbol{∀} \AgdaSymbol{\{}\AgdaBound{U}\AgdaSymbol{\}} \AgdaSymbol{\{}\AgdaBound{V1}\AgdaSymbol{\}} \AgdaSymbol{\{}\AgdaBound{V2}\AgdaSymbol{\}} \AgdaSymbol{\{}\AgdaBound{V3}\AgdaSymbol{\}} \AgdaSymbol{\{}\AgdaBound{V4}\AgdaSymbol{\}} \AgdaSymbol{\{}\AgdaBound{K}\AgdaSymbol{\}} \AgdaSymbol{\{}\AgdaBound{ρ1} \AgdaSymbol{:} \AgdaFunction{Rep} \AgdaBound{U} \AgdaBound{V1}\AgdaSymbol{\}} \AgdaSymbol{\{}\AgdaBound{ρ2} \AgdaSymbol{:} \AgdaFunction{Rep} \AgdaBound{V1} \AgdaBound{V2}\AgdaSymbol{\}} \AgdaSymbol{\{}\AgdaBound{ρ3} \AgdaSymbol{:} \AgdaFunction{Rep} \AgdaBound{V2} \AgdaBound{V3}\AgdaSymbol{\}} \AgdaSymbol{\{}\AgdaBound{ρ4} \AgdaSymbol{:} \AgdaFunction{Rep} \AgdaBound{V3} \AgdaBound{V4}\AgdaSymbol{\}} \AgdaSymbol{→}\<%
\\
\>[2]\AgdaIndent{23}{}\<[23]%
\>[23]\AgdaFunction{rep↑} \AgdaBound{K} \AgdaSymbol{(}\AgdaBound{ρ4} \AgdaFunction{•R} \AgdaBound{ρ3} \AgdaFunction{•R} \AgdaBound{ρ2} \AgdaFunction{•R} \AgdaBound{ρ1}\AgdaSymbol{)} \AgdaFunction{∼R} \AgdaFunction{rep↑} \AgdaBound{K} \AgdaBound{ρ4} \AgdaFunction{•R} \AgdaFunction{rep↑} \AgdaBound{K} \AgdaBound{ρ3} \AgdaFunction{•R} \AgdaFunction{rep↑} \AgdaBound{K} \AgdaBound{ρ2} \AgdaFunction{•R} \AgdaFunction{rep↑} \AgdaBound{K} \AgdaBound{ρ1}\<%
\end{code}
}

\begin{code}%
\>\AgdaFunction{replacement} \AgdaSymbol{:} \AgdaRecord{OpFamily}\<%
\\
\>\AgdaFunction{replacement} \AgdaSymbol{=} \AgdaKeyword{record} \AgdaSymbol{\{} \<[23]%
\>[23]\<%
\\
\>[0]\AgdaIndent{2}{}\<[2]%
\>[2]\AgdaField{liftFamily} \AgdaSymbol{=} \AgdaFunction{proto-replacement} \AgdaSymbol{;} \<[35]%
\>[35]\<%
\\
\>[0]\AgdaIndent{2}{}\<[2]%
\>[2]\AgdaField{isOpFamily} \AgdaSymbol{=} \AgdaKeyword{record} \AgdaSymbol{\{} \<[24]%
\>[24]\<%
\\
\>[2]\AgdaIndent{4}{}\<[4]%
\>[4]\AgdaField{\_∘\_} \AgdaSymbol{=} \AgdaFunction{\_•R\_} \AgdaSymbol{;} \<[17]%
\>[17]\<%
\\
\>[2]\AgdaIndent{4}{}\<[4]%
\>[4]\AgdaField{apV-comp} \AgdaSymbol{=} \AgdaInductiveConstructor{refl} \AgdaSymbol{;} \<[22]%
\>[22]\<%
\\
\>[2]\AgdaIndent{4}{}\<[4]%
\>[4]\AgdaField{liftOp-comp} \AgdaSymbol{=} \AgdaFunction{rep↑-comp} \AgdaSymbol{\}} \AgdaSymbol{\}}\<%
\end{code}

\AgdaHide{
\begin{code}%
\>\AgdaKeyword{open} \AgdaModule{OpFamily} \AgdaFunction{replacement} \AgdaKeyword{public} \AgdaKeyword{using} \AgdaSymbol{()} \<[42]%
\>[42]\<%
\\
\>[0]\AgdaIndent{2}{}\<[2]%
\>[2]\AgdaKeyword{renaming} \AgdaSymbol{(}ap-congl \AgdaSymbol{to} rep-congr\AgdaSymbol{;}\<\\
\>           ap-congr \AgdaSymbol{to} rep-congl\AgdaSymbol{;}\<\\
\>           ap-idOp \AgdaSymbol{to} rep-idOp\AgdaSymbol{;}\<\\
\>           ap-circ \AgdaSymbol{to} rep-comp\AgdaSymbol{;}\<\\
\>           liftOp-idOp \AgdaSymbol{to} rep↑-idOp\AgdaSymbol{;}\<\\
\>           liftOp-up' \AgdaSymbol{to} rep↑-upRep\AgdaSymbol{)}\<%
\\
%
\\
\>\AgdaKeyword{postulate} \AgdaPostulate{rep-comp₄} \AgdaSymbol{:} \AgdaSymbol{∀} \AgdaSymbol{\{}\AgdaBound{U}\AgdaSymbol{\}} \AgdaSymbol{\{}\AgdaBound{V1}\AgdaSymbol{\}} \AgdaSymbol{\{}\AgdaBound{V2}\AgdaSymbol{\}} \AgdaSymbol{\{}\AgdaBound{V3}\AgdaSymbol{\}} \AgdaSymbol{\{}\AgdaBound{V4}\AgdaSymbol{\}} \<[48]%
\>[48]\<%
\\
\>[2]\AgdaIndent{22}{}\<[22]%
\>[22]\AgdaSymbol{\{}\AgdaBound{ρ1} \AgdaSymbol{:} \AgdaFunction{Rep} \AgdaBound{U} \AgdaBound{V1}\AgdaSymbol{\}} \AgdaSymbol{\{}\AgdaBound{ρ2} \AgdaSymbol{:} \AgdaFunction{Rep} \AgdaBound{V1} \AgdaBound{V2}\AgdaSymbol{\}} \AgdaSymbol{\{}\AgdaBound{ρ3} \AgdaSymbol{:} \AgdaFunction{Rep} \AgdaBound{V2} \AgdaBound{V3}\AgdaSymbol{\}} \AgdaSymbol{\{}\AgdaBound{ρ4} \AgdaSymbol{:} \AgdaFunction{Rep} \AgdaBound{V3} \AgdaBound{V4}\AgdaSymbol{\}} \<[89]%
\>[89]\<%
\\
\>[2]\AgdaIndent{22}{}\<[22]%
\>[22]\AgdaSymbol{\{}\AgdaBound{C}\AgdaSymbol{\}} \AgdaSymbol{\{}\AgdaBound{K}\AgdaSymbol{\}} \AgdaSymbol{(}\AgdaBound{E} \AgdaSymbol{:} \AgdaDatatype{Subexpression} \AgdaBound{U} \AgdaBound{C} \AgdaBound{K}\AgdaSymbol{)} \AgdaSymbol{→}\<%
\\
\>[2]\AgdaIndent{22}{}\<[22]%
\>[22]\AgdaBound{E} \AgdaFunction{〈} \AgdaBound{ρ4} \AgdaFunction{•R} \AgdaBound{ρ3} \AgdaFunction{•R} \AgdaBound{ρ2} \AgdaFunction{•R} \AgdaBound{ρ1} \AgdaFunction{〉} \AgdaDatatype{≡} \AgdaBound{E} \AgdaFunction{〈} \AgdaBound{ρ1} \AgdaFunction{〉} \AgdaFunction{〈} \AgdaBound{ρ2} \AgdaFunction{〉} \AgdaFunction{〈} \AgdaBound{ρ3} \AgdaFunction{〉} \AgdaFunction{〈} \AgdaBound{ρ4} \AgdaFunction{〉}\<%
\end{code}
}

We write $E \uparrow$ for $E \langle \uparrow \rangle$.

\begin{code}%
\>\AgdaKeyword{infixl} \AgdaNumber{70} \AgdaFixityOp{\_⇑}\<%
\\
\>\AgdaFunction{\_⇑} \AgdaSymbol{:} \AgdaSymbol{∀} \AgdaSymbol{\{}\AgdaBound{V}\AgdaSymbol{\}} \AgdaSymbol{\{}\AgdaBound{K}\AgdaSymbol{\}} \AgdaSymbol{\{}\AgdaBound{C}\AgdaSymbol{\}} \AgdaSymbol{\{}\AgdaBound{L}\AgdaSymbol{\}} \AgdaSymbol{→} \AgdaDatatype{Subexpression} \AgdaBound{V} \AgdaBound{C} \AgdaBound{L} \AgdaSymbol{→} \AgdaDatatype{Subexpression} \AgdaSymbol{(}\AgdaBound{V} \AgdaInductiveConstructor{,} \AgdaBound{K}\AgdaSymbol{)} \AgdaBound{C} \AgdaBound{L}\<%
\\
\>\AgdaBound{E} \AgdaFunction{⇑} \AgdaSymbol{=} \AgdaBound{E} \AgdaFunction{〈} \AgdaFunction{upRep} \AgdaFunction{〉}\<%
\end{code}

We define the unique replacement $\emptyset \rightarrow V$ for any V, and prove it unique:

\begin{code}%
\>\AgdaFunction{magic} \AgdaSymbol{:} \AgdaSymbol{∀} \AgdaSymbol{\{}\AgdaBound{V}\AgdaSymbol{\}} \AgdaSymbol{→} \AgdaFunction{Rep} \AgdaInductiveConstructor{∅} \AgdaBound{V}\<%
\\
\>\AgdaFunction{magic} \AgdaSymbol{\_} \AgdaSymbol{()}\<%
\\
%
\\
\>\AgdaFunction{magic-unique} \AgdaSymbol{:} \AgdaSymbol{∀} \AgdaSymbol{\{}\AgdaBound{V}\AgdaSymbol{\}} \AgdaSymbol{\{}\AgdaBound{ρ} \AgdaSymbol{:} \AgdaFunction{Rep} \AgdaInductiveConstructor{∅} \AgdaBound{V}\AgdaSymbol{\}} \AgdaSymbol{→} \AgdaBound{ρ} \AgdaFunction{∼R} \AgdaFunction{magic}\<%
\end{code}

\AgdaHide{
\begin{code}%
\>\AgdaFunction{magic-unique} \AgdaSymbol{\{}\AgdaBound{V}\AgdaSymbol{\}} \AgdaSymbol{\{}\AgdaBound{ρ}\AgdaSymbol{\}} \AgdaSymbol{()}\<%
\end{code}
}

\begin{code}%
\>\AgdaFunction{magic-unique'} \AgdaSymbol{:} \AgdaSymbol{∀} \AgdaSymbol{\{}\AgdaBound{U}\AgdaSymbol{\}} \AgdaSymbol{\{}\AgdaBound{V}\AgdaSymbol{\}} \AgdaSymbol{\{}\AgdaBound{C}\AgdaSymbol{\}} \AgdaSymbol{\{}\AgdaBound{K}\AgdaSymbol{\}}\<%
\\
\>[0]\AgdaIndent{2}{}\<[2]%
\>[2]\AgdaSymbol{(}\AgdaBound{E} \AgdaSymbol{:} \AgdaDatatype{Subexpression} \AgdaInductiveConstructor{∅} \AgdaBound{C} \AgdaBound{K}\AgdaSymbol{)} \AgdaSymbol{\{}\AgdaBound{ρ} \AgdaSymbol{:} \AgdaFunction{Rep} \AgdaBound{U} \AgdaBound{V}\AgdaSymbol{\}} \AgdaSymbol{→} \<[44]%
\>[44]\<%
\\
\>[0]\AgdaIndent{2}{}\<[2]%
\>[2]\AgdaBound{E} \AgdaFunction{〈} \AgdaFunction{magic} \AgdaFunction{〉} \AgdaFunction{〈} \AgdaBound{ρ} \AgdaFunction{〉} \AgdaDatatype{≡} \AgdaBound{E} \AgdaFunction{〈} \AgdaFunction{magic} \AgdaFunction{〉}\<%
\end{code}

\AgdaHide{
\begin{code}%
\>\AgdaFunction{magic-unique'} \AgdaBound{E} \AgdaSymbol{\{}\AgdaBound{ρ}\AgdaSymbol{\}} \AgdaSymbol{=} \AgdaKeyword{let} \AgdaKeyword{open} \AgdaModule{≡-Reasoning} \AgdaKeyword{in}\<%
\\
\>[0]\AgdaIndent{2}{}\<[2]%
\>[2]\AgdaFunction{begin}\<%
\\
\>[2]\AgdaIndent{4}{}\<[4]%
\>[4]\AgdaBound{E} \AgdaFunction{〈} \AgdaFunction{magic} \AgdaFunction{〉} \AgdaFunction{〈} \AgdaBound{ρ} \AgdaFunction{〉}\<%
\\
\>[0]\AgdaIndent{2}{}\<[2]%
\>[2]\AgdaFunction{≡⟨⟨} \AgdaFunction{rep-comp} \AgdaBound{E} \AgdaFunction{⟩⟩}\<%
\\
\>[2]\AgdaIndent{4}{}\<[4]%
\>[4]\AgdaBound{E} \AgdaFunction{〈} \AgdaBound{ρ} \AgdaFunction{•R} \AgdaFunction{magic} \AgdaFunction{〉}\<%
\\
\>[0]\AgdaIndent{2}{}\<[2]%
\>[2]\AgdaFunction{≡⟨} \AgdaFunction{rep-congr} \AgdaBound{E} \AgdaSymbol{(}\AgdaFunction{magic-unique} \AgdaSymbol{\{}\AgdaArgument{ρ} \AgdaSymbol{=} \AgdaBound{ρ} \AgdaFunction{•R} \AgdaFunction{magic}\AgdaSymbol{\})} \AgdaFunction{⟩}\<%
\\
\>[2]\AgdaIndent{4}{}\<[4]%
\>[4]\AgdaBound{E} \AgdaFunction{〈} \AgdaFunction{magic} \AgdaFunction{〉}\<%
\\
\>[0]\AgdaIndent{2}{}\<[2]%
\>[2]\AgdaFunction{∎}\<%
\\
%
\\
\>\AgdaFunction{rep↑-upRep₂} \AgdaSymbol{:} \AgdaSymbol{∀} \AgdaSymbol{\{}\AgdaBound{U}\AgdaSymbol{\}} \AgdaSymbol{\{}\AgdaBound{V}\AgdaSymbol{\}} \AgdaSymbol{\{}\AgdaBound{C}\AgdaSymbol{\}} \AgdaSymbol{\{}\AgdaBound{K}\AgdaSymbol{\}} \AgdaSymbol{\{}\AgdaBound{L}\AgdaSymbol{\}} \AgdaSymbol{\{}\AgdaBound{M}\AgdaSymbol{\}} \AgdaSymbol{(}\AgdaBound{E} \AgdaSymbol{:} \AgdaDatatype{Subexpression} \AgdaBound{U} \AgdaBound{C} \AgdaBound{M}\AgdaSymbol{)} \AgdaSymbol{\{}\AgdaBound{σ} \AgdaSymbol{:} \AgdaFunction{Rep} \AgdaBound{U} \AgdaBound{V}\AgdaSymbol{\}} \AgdaSymbol{→} \AgdaBound{E} \AgdaFunction{⇑} \AgdaFunction{⇑} \AgdaFunction{〈} \AgdaFunction{rep↑} \AgdaBound{K} \AgdaSymbol{(}\AgdaFunction{rep↑} \AgdaBound{L} \AgdaBound{σ}\AgdaSymbol{)} \AgdaFunction{〉} \AgdaDatatype{≡} \AgdaBound{E} \AgdaFunction{〈} \AgdaBound{σ} \AgdaFunction{〉} \AgdaFunction{⇑} \AgdaFunction{⇑}\<%
\\
\>\AgdaFunction{rep↑-upRep₂} \AgdaSymbol{\{}\AgdaBound{U}\AgdaSymbol{\}} \AgdaSymbol{\{}\AgdaBound{V}\AgdaSymbol{\}} \AgdaSymbol{\{}\AgdaBound{C}\AgdaSymbol{\}} \AgdaSymbol{\{}\AgdaBound{K}\AgdaSymbol{\}} \AgdaSymbol{\{}\AgdaBound{L}\AgdaSymbol{\}} \AgdaSymbol{\{}\AgdaBound{M}\AgdaSymbol{\}} \AgdaBound{E} \AgdaSymbol{\{}\AgdaBound{σ}\AgdaSymbol{\}} \AgdaSymbol{=} \AgdaKeyword{let} \AgdaKeyword{open} \AgdaModule{≡-Reasoning} \AgdaKeyword{in} \<[68]%
\>[68]\<%
\\
\>[0]\AgdaIndent{2}{}\<[2]%
\>[2]\AgdaFunction{begin}\<%
\\
\>[2]\AgdaIndent{4}{}\<[4]%
\>[4]\AgdaBound{E} \AgdaFunction{⇑} \AgdaFunction{⇑} \AgdaFunction{〈} \AgdaFunction{rep↑} \AgdaBound{K} \AgdaSymbol{(}\AgdaFunction{rep↑} \AgdaBound{L} \AgdaBound{σ}\AgdaSymbol{)} \AgdaFunction{〉}\<%
\\
\>[0]\AgdaIndent{2}{}\<[2]%
\>[2]\AgdaFunction{≡⟨} \AgdaFunction{rep↑-upRep} \AgdaSymbol{(}\AgdaBound{E} \AgdaFunction{⇑}\AgdaSymbol{)} \AgdaFunction{⟩}\<%
\\
\>[2]\AgdaIndent{4}{}\<[4]%
\>[4]\AgdaBound{E} \AgdaFunction{⇑} \AgdaFunction{〈} \AgdaFunction{rep↑} \AgdaBound{L} \AgdaBound{σ} \AgdaFunction{〉} \AgdaFunction{⇑}\<%
\\
\>[0]\AgdaIndent{2}{}\<[2]%
\>[2]\AgdaFunction{≡⟨} \AgdaFunction{rep-congl} \AgdaSymbol{(}\AgdaFunction{rep↑-upRep} \AgdaBound{E}\AgdaSymbol{)} \AgdaFunction{⟩}\<%
\\
\>[2]\AgdaIndent{4}{}\<[4]%
\>[4]\AgdaBound{E} \AgdaFunction{〈} \AgdaBound{σ} \AgdaFunction{〉} \AgdaFunction{⇑} \AgdaFunction{⇑}\<%
\\
\>[0]\AgdaIndent{2}{}\<[2]%
\>[2]\AgdaFunction{∎}\<%
\\
%
\\
\>\AgdaFunction{rep↑-upRep₃} \AgdaSymbol{:} \AgdaSymbol{∀} \AgdaSymbol{\{}\AgdaBound{U}\AgdaSymbol{\}} \AgdaSymbol{\{}\AgdaBound{V}\AgdaSymbol{\}} \AgdaSymbol{\{}\AgdaBound{C}\AgdaSymbol{\}} \AgdaSymbol{\{}\AgdaBound{K}\AgdaSymbol{\}} \AgdaSymbol{\{}\AgdaBound{L}\AgdaSymbol{\}} \AgdaSymbol{\{}\AgdaBound{M}\AgdaSymbol{\}} \AgdaSymbol{\{}\AgdaBound{N}\AgdaSymbol{\}} \AgdaSymbol{(}\AgdaBound{E} \AgdaSymbol{:} \AgdaDatatype{Subexpression} \AgdaBound{U} \AgdaBound{C} \AgdaBound{N}\AgdaSymbol{)} \AgdaSymbol{\{}\AgdaBound{σ} \AgdaSymbol{:} \AgdaFunction{Rep} \AgdaBound{U} \AgdaBound{V}\AgdaSymbol{\}} \AgdaSymbol{→} \<[86]%
\>[86]\<%
\\
\>[0]\AgdaIndent{2}{}\<[2]%
\>[2]\AgdaBound{E} \AgdaFunction{⇑} \AgdaFunction{⇑} \AgdaFunction{⇑} \AgdaFunction{〈} \AgdaFunction{rep↑} \AgdaBound{K} \AgdaSymbol{(}\AgdaFunction{rep↑} \AgdaBound{L} \AgdaSymbol{(}\AgdaFunction{rep↑} \AgdaBound{M} \AgdaBound{σ}\AgdaSymbol{))} \AgdaFunction{〉} \AgdaDatatype{≡} \AgdaBound{E} \AgdaFunction{〈} \AgdaBound{σ} \AgdaFunction{〉} \AgdaFunction{⇑} \AgdaFunction{⇑} \AgdaFunction{⇑}\<%
\\
\>\AgdaFunction{rep↑-upRep₃} \AgdaSymbol{\{}\AgdaBound{U}\AgdaSymbol{\}} \AgdaSymbol{\{}\AgdaBound{V}\AgdaSymbol{\}} \AgdaSymbol{\{}\AgdaBound{C}\AgdaSymbol{\}} \AgdaSymbol{\{}\AgdaBound{K}\AgdaSymbol{\}} \AgdaSymbol{\{}\AgdaBound{L}\AgdaSymbol{\}} \AgdaSymbol{\{}\AgdaBound{M}\AgdaSymbol{\}} \AgdaBound{E} \AgdaSymbol{\{}\AgdaBound{σ}\AgdaSymbol{\}} \AgdaSymbol{=} \AgdaKeyword{let} \AgdaKeyword{open} \AgdaModule{≡-Reasoning} \AgdaKeyword{in} \<[68]%
\>[68]\<%
\\
\>[0]\AgdaIndent{2}{}\<[2]%
\>[2]\AgdaFunction{begin}\<%
\\
\>[2]\AgdaIndent{4}{}\<[4]%
\>[4]\AgdaBound{E} \AgdaFunction{⇑} \AgdaFunction{⇑} \AgdaFunction{⇑} \AgdaFunction{〈} \AgdaFunction{rep↑} \AgdaBound{K} \AgdaSymbol{(}\AgdaFunction{rep↑} \AgdaBound{L} \AgdaSymbol{(}\AgdaFunction{rep↑} \AgdaBound{M} \AgdaBound{σ}\AgdaSymbol{))} \AgdaFunction{〉}\<%
\\
\>[0]\AgdaIndent{2}{}\<[2]%
\>[2]\AgdaFunction{≡⟨} \AgdaFunction{rep↑-upRep₂} \AgdaSymbol{(}\AgdaBound{E} \AgdaFunction{⇑}\AgdaSymbol{)} \AgdaFunction{⟩}\<%
\\
\>[2]\AgdaIndent{4}{}\<[4]%
\>[4]\AgdaBound{E} \AgdaFunction{⇑} \AgdaFunction{〈} \AgdaFunction{rep↑} \AgdaBound{M} \AgdaBound{σ} \AgdaFunction{〉} \AgdaFunction{⇑} \AgdaFunction{⇑}\<%
\\
\>[0]\AgdaIndent{2}{}\<[2]%
\>[2]\AgdaFunction{≡⟨} \AgdaFunction{rep-congl} \AgdaSymbol{(}\AgdaFunction{rep-congl} \AgdaSymbol{(}\AgdaFunction{rep↑-upRep} \AgdaBound{E}\AgdaSymbol{))} \AgdaFunction{⟩}\<%
\\
\>[2]\AgdaIndent{4}{}\<[4]%
\>[4]\AgdaBound{E} \AgdaFunction{〈} \AgdaBound{σ} \AgdaFunction{〉} \AgdaFunction{⇑} \AgdaFunction{⇑} \AgdaFunction{⇑}\<%
\\
\>[0]\AgdaIndent{2}{}\<[2]%
\>[2]\AgdaFunction{∎}\<%
\\
%
\\
\>\AgdaKeyword{postulate} \AgdaPostulate{rep↑-upRep₄'} \AgdaSymbol{:} \AgdaSymbol{∀} \AgdaSymbol{\{}\AgdaBound{U}\AgdaSymbol{\}} \AgdaSymbol{\{}\AgdaBound{V}\AgdaSymbol{\}} \AgdaSymbol{(}\AgdaBound{ρ} \AgdaSymbol{:} \AgdaFunction{Rep} \AgdaBound{U} \AgdaBound{V}\AgdaSymbol{)} \AgdaSymbol{\{}\AgdaBound{K1}\AgdaSymbol{\}} \AgdaSymbol{\{}\AgdaBound{K2}\AgdaSymbol{\}} \AgdaSymbol{\{}\AgdaBound{K3}\AgdaSymbol{\}} \AgdaSymbol{→} \AgdaFunction{upRep} \AgdaFunction{•R} \AgdaFunction{upRep} \AgdaFunction{•R} \AgdaFunction{upRep} \AgdaFunction{•R} \AgdaBound{ρ} \AgdaFunction{∼R} \AgdaFunction{rep↑} \AgdaBound{K1} \AgdaSymbol{(}\AgdaFunction{rep↑} \AgdaBound{K2} \AgdaSymbol{(}\AgdaFunction{rep↑} \AgdaBound{K3} \AgdaBound{ρ}\AgdaSymbol{))} \AgdaFunction{•R} \AgdaFunction{upRep} \AgdaFunction{•R} \AgdaFunction{upRep} \AgdaFunction{•R} \AgdaFunction{upRep}\<%
\end{code}
}

\AgdaHide{
\begin{code}%
\>\AgdaKeyword{open} \AgdaKeyword{import} \AgdaModule{Grammar.Base}\<%
\\
%
\\
\>\AgdaKeyword{module} \AgdaModule{Grammar.OpFamily.PreOpFamily} \AgdaSymbol{(}\AgdaBound{G} \AgdaSymbol{:} \AgdaRecord{Grammar}\AgdaSymbol{)} \AgdaKeyword{where}\<%
\\
\>\AgdaKeyword{open} \AgdaKeyword{import} \AgdaModule{Prelims}\<%
\\
\>\AgdaKeyword{open} \AgdaModule{Grammar} \AgdaBound{G}\<%
\end{code}
}

\subsection{Families of Operations}

Our aim here is to define the operations of \emph{replacement} and \emph{substitution}.  In order to organise this work, we introduce the following definitions.

A \emph{family of operations} over a grammar $G$ consists of:
\begin{enumerate}
\item
for any alphabets $U$ and $V$, a set $F[U,V]$ of \emph{operations} $\sigma$ from $U$ to $V$, $\sigma : U \rightarrow V$;
\item
for any operation $\sigma : U \rightarrow V$ and variable $x \in U$ of kind $K$, an expression $\sigma(x)$ over $V$ of kind $K$;
\item
for any alphabet $V$ and variable kind $K$, an operation $\uparrow : V \rightarrow (V , K)$, the \emph{lifting} operation;
\item
for any alphabet $V$, an operation $\id{V} : V \rightarrow V$, the \emph{identity} operation;
\item
for any operation $\sigma : U \rightarrow V$ and variable kind $K$, an operation $(\sigma , K) : (U , K) \rightarrow (V , K)$, the result of \emph{lifting} $\sigma$;
\item
for any operations $\rho : U \rightarrow V$ and $\sigma : V \rightarrow W$, an operation
$\sigma \circ \rho : U \rightarrow W$, the \emph{composition} of $\sigma$ and $\rho$;
\end{enumerate}
such that:
\begin{itemize}
\item
$\uparrow (x) \equiv x$
\item
$\id{V}(x) \equiv x$
\item
If $\rho \sim \sigma$ then $(\rho , K) \sim (\sigma , K)$
\item
$(\rho , K)(x_0) \equiv x_0$
\item
Given $\sigma : U \rightarrow V$ and $x \in U$, we have $(\sigma , K)(x) \equiv x$
\item
$(\sigma \circ \rho , K) \sim (\sigma , K) \circ (\rho , K)$
\item
$(\sigma \circ \rho)(x) \equiv \rho(x) [ \sigma ]$
\end{itemize}
where for $\sigma, \rho : U \rightarrow V$ we write $\sigma \sim \rho$ iff $\sigma(x) \equiv \rho(x)$ for all $x \in U$; and, given $\sigma : U \rightarrow V$ and $E$ an expression over $U$, we define $E[\sigma]$, the result of \emph{applying} the operation $\sigma$ to $E$, as follows:

\begin{align*}
x[\sigma] & \eqdef \sigma(x) \\
\lefteqn{c([\vec{x_1}] E_1, \ldots, [\vec{x_n}] E_n) [\sigma]} \\
 & \eqdef
c([\vec{x_1}] E_1 [(\sigma , K_{11}, \ldots, K_{1r_1})], \ldots,
[\vec{x_n}] E_n [(\sigma, K_{n1}, \ldots, K_{nr_n})])
\end{align*}
for $c$ a constructor of type (\ref{eq:conkind}).

\subsubsection{Pre-Families}
We formalize this definition in stages.  First, we define a \emph{pre-family of operations} to be a family with items of data 1--4 above that satisfies the first two axioms:

\begin{code}%
\>\AgdaKeyword{record} \AgdaRecord{PreOpFamily} \AgdaSymbol{:} \AgdaPrimitiveType{Set₂} \AgdaKeyword{where}\<%
\\
\>[0]\AgdaIndent{2}{}\<[2]%
\>[2]\AgdaKeyword{field}\<%
\\
\>[2]\AgdaIndent{4}{}\<[4]%
\>[4]\AgdaField{Op} \AgdaSymbol{:} \AgdaDatatype{Alphabet} \AgdaSymbol{→} \AgdaDatatype{Alphabet} \AgdaSymbol{→} \AgdaPrimitiveType{Set}\<%
\\
\>[2]\AgdaIndent{4}{}\<[4]%
\>[4]\AgdaField{apV} \AgdaSymbol{:} \AgdaSymbol{∀} \AgdaSymbol{\{}\AgdaBound{U}\AgdaSymbol{\}} \AgdaSymbol{\{}\AgdaBound{V}\AgdaSymbol{\}} \AgdaSymbol{\{}\AgdaBound{K}\AgdaSymbol{\}} \AgdaSymbol{→} \AgdaField{Op} \AgdaBound{U} \AgdaBound{V} \AgdaSymbol{→} \AgdaDatatype{Var} \AgdaBound{U} \AgdaBound{K} \AgdaSymbol{→} \AgdaFunction{Expression} \AgdaBound{V} \AgdaSymbol{(}\AgdaInductiveConstructor{varKind} \AgdaBound{K}\AgdaSymbol{)}\<%
\\
\>[2]\AgdaIndent{4}{}\<[4]%
\>[4]\AgdaField{up} \AgdaSymbol{:} \AgdaSymbol{∀} \AgdaSymbol{\{}\AgdaBound{V}\AgdaSymbol{\}} \AgdaSymbol{\{}\AgdaBound{K}\AgdaSymbol{\}} \AgdaSymbol{→} \AgdaField{Op} \AgdaBound{V} \AgdaSymbol{(}\AgdaBound{V} \AgdaInductiveConstructor{,} \AgdaBound{K}\AgdaSymbol{)}\<%
\\
\>[2]\AgdaIndent{4}{}\<[4]%
\>[4]\AgdaField{apV-up} \AgdaSymbol{:} \AgdaSymbol{∀} \AgdaSymbol{\{}\AgdaBound{V}\AgdaSymbol{\}} \AgdaSymbol{\{}\AgdaBound{K}\AgdaSymbol{\}} \AgdaSymbol{\{}\AgdaBound{L}\AgdaSymbol{\}} \AgdaSymbol{\{}\AgdaBound{x} \AgdaSymbol{:} \AgdaDatatype{Var} \AgdaBound{V} \AgdaBound{K}\AgdaSymbol{\}} \AgdaSymbol{→} \AgdaField{apV} \AgdaSymbol{(}\AgdaField{up} \AgdaSymbol{\{}\AgdaArgument{K} \AgdaSymbol{=} \AgdaBound{L}\AgdaSymbol{\})} \AgdaBound{x} \AgdaDatatype{≡} \AgdaInductiveConstructor{var} \AgdaSymbol{(}\AgdaInductiveConstructor{↑} \AgdaBound{x}\AgdaSymbol{)}\<%
\\
\>[2]\AgdaIndent{4}{}\<[4]%
\>[4]\AgdaField{idOp} \AgdaSymbol{:} \AgdaSymbol{∀} \AgdaBound{V} \AgdaSymbol{→} \AgdaField{Op} \AgdaBound{V} \AgdaBound{V}\<%
\\
\>[2]\AgdaIndent{4}{}\<[4]%
\>[4]\AgdaField{apV-idOp} \AgdaSymbol{:} \AgdaSymbol{∀} \AgdaSymbol{\{}\AgdaBound{V}\AgdaSymbol{\}} \AgdaSymbol{\{}\AgdaBound{K}\AgdaSymbol{\}} \AgdaSymbol{(}\AgdaBound{x} \AgdaSymbol{:} \AgdaDatatype{Var} \AgdaBound{V} \AgdaBound{K}\AgdaSymbol{)} \AgdaSymbol{→} \AgdaField{apV} \AgdaSymbol{(}\AgdaField{idOp} \AgdaBound{V}\AgdaSymbol{)} \AgdaBound{x} \AgdaDatatype{≡} \AgdaInductiveConstructor{var} \AgdaBound{x}\<%
\end{code}

This allows us to define the relation $\sim$, and prove it is an equivalence relation:

\begin{code}%
\>[0]\AgdaIndent{2}{}\<[2]%
\>[2]\AgdaFunction{\_∼op\_} \AgdaSymbol{:} \AgdaSymbol{∀} \AgdaSymbol{\{}\AgdaBound{U}\AgdaSymbol{\}} \AgdaSymbol{\{}\AgdaBound{V}\AgdaSymbol{\}} \AgdaSymbol{→} \AgdaField{Op} \AgdaBound{U} \AgdaBound{V} \AgdaSymbol{→} \AgdaField{Op} \AgdaBound{U} \AgdaBound{V} \AgdaSymbol{→} \AgdaPrimitiveType{Set}\<%
\\
\>[0]\AgdaIndent{2}{}\<[2]%
\>[2]\AgdaFunction{\_∼op\_} \AgdaSymbol{\{}\AgdaBound{U}\AgdaSymbol{\}} \AgdaSymbol{\{}\AgdaBound{V}\AgdaSymbol{\}} \AgdaBound{ρ} \AgdaBound{σ} \AgdaSymbol{=} \AgdaSymbol{∀} \AgdaSymbol{\{}\AgdaBound{K}\AgdaSymbol{\}} \AgdaSymbol{(}\AgdaBound{x} \AgdaSymbol{:} \AgdaDatatype{Var} \AgdaBound{U} \AgdaBound{K}\AgdaSymbol{)} \AgdaSymbol{→} \AgdaField{apV} \AgdaBound{ρ} \AgdaBound{x} \AgdaDatatype{≡} \AgdaField{apV} \AgdaBound{σ} \AgdaBound{x}\<%
\\
\>[2]\AgdaIndent{4}{}\<[4]%
\>[4]\<%
\\
\>[0]\AgdaIndent{2}{}\<[2]%
\>[2]\AgdaFunction{∼-refl} \AgdaSymbol{:} \AgdaSymbol{∀} \AgdaSymbol{\{}\AgdaBound{U}\AgdaSymbol{\}} \AgdaSymbol{\{}\AgdaBound{V}\AgdaSymbol{\}} \AgdaSymbol{\{}\AgdaBound{σ} \AgdaSymbol{:} \AgdaField{Op} \AgdaBound{U} \AgdaBound{V}\AgdaSymbol{\}} \AgdaSymbol{→} \AgdaBound{σ} \AgdaFunction{∼op} \AgdaBound{σ}\<%
\\
\>[0]\AgdaIndent{2}{}\<[2]%
\>[2]\AgdaFunction{∼-refl} \AgdaSymbol{\_} \AgdaSymbol{=} \AgdaInductiveConstructor{refl}\<%
\\
\>[2]\AgdaIndent{4}{}\<[4]%
\>[4]\<%
\\
\>[0]\AgdaIndent{2}{}\<[2]%
\>[2]\AgdaFunction{∼-sym} \AgdaSymbol{:} \AgdaSymbol{∀} \AgdaSymbol{\{}\AgdaBound{U}\AgdaSymbol{\}} \AgdaSymbol{\{}\AgdaBound{V}\AgdaSymbol{\}} \AgdaSymbol{\{}\AgdaBound{σ} \AgdaBound{τ} \AgdaSymbol{:} \AgdaField{Op} \AgdaBound{U} \AgdaBound{V}\AgdaSymbol{\}} \AgdaSymbol{→} \AgdaBound{σ} \AgdaFunction{∼op} \AgdaBound{τ} \AgdaSymbol{→} \AgdaBound{τ} \AgdaFunction{∼op} \AgdaBound{σ}\<%
\\
\>[0]\AgdaIndent{2}{}\<[2]%
\>[2]\AgdaFunction{∼-sym} \AgdaBound{σ-is-τ} \AgdaBound{x} \AgdaSymbol{=} \AgdaFunction{sym} \AgdaSymbol{(}\AgdaBound{σ-is-τ} \AgdaBound{x}\AgdaSymbol{)}\<%
\\
%
\\
\>[0]\AgdaIndent{2}{}\<[2]%
\>[2]\AgdaFunction{∼-trans} \AgdaSymbol{:} \AgdaSymbol{∀} \AgdaSymbol{\{}\AgdaBound{U}\AgdaSymbol{\}} \AgdaSymbol{\{}\AgdaBound{V}\AgdaSymbol{\}} \AgdaSymbol{\{}\AgdaBound{ρ} \AgdaBound{σ} \AgdaBound{τ} \AgdaSymbol{:} \AgdaField{Op} \AgdaBound{U} \AgdaBound{V}\AgdaSymbol{\}} \AgdaSymbol{→} \AgdaBound{ρ} \AgdaFunction{∼op} \AgdaBound{σ} \AgdaSymbol{→} \AgdaBound{σ} \AgdaFunction{∼op} \AgdaBound{τ} \AgdaSymbol{→} \AgdaBound{ρ} \AgdaFunction{∼op} \AgdaBound{τ}\<%
\\
\>[0]\AgdaIndent{2}{}\<[2]%
\>[2]\AgdaFunction{∼-trans} \AgdaBound{ρ-is-σ} \AgdaBound{σ-is-τ} \AgdaBound{x} \AgdaSymbol{=} \AgdaFunction{trans} \AgdaSymbol{(}\AgdaBound{ρ-is-σ} \AgdaBound{x}\AgdaSymbol{)} \AgdaSymbol{(}\AgdaBound{σ-is-τ} \AgdaBound{x}\AgdaSymbol{)}\<%
\\
%
\\
\>[0]\AgdaIndent{2}{}\<[2]%
\>[2]\AgdaFunction{OP} \AgdaSymbol{:} \AgdaDatatype{Alphabet} \AgdaSymbol{→} \AgdaDatatype{Alphabet} \AgdaSymbol{→} \AgdaRecord{Setoid} \AgdaSymbol{\_} \AgdaSymbol{\_}\<%
\\
\>[0]\AgdaIndent{2}{}\<[2]%
\>[2]\AgdaFunction{OP} \AgdaBound{U} \AgdaBound{V} \AgdaSymbol{=} \AgdaKeyword{record} \AgdaSymbol{\{} \<[20]%
\>[20]\<%
\\
\>[2]\AgdaIndent{5}{}\<[5]%
\>[5]\AgdaField{Carrier} \AgdaSymbol{=} \AgdaField{Op} \AgdaBound{U} \AgdaBound{V} \AgdaSymbol{;} \<[24]%
\>[24]\<%
\\
\>[2]\AgdaIndent{5}{}\<[5]%
\>[5]\AgdaField{\_≈\_} \AgdaSymbol{=} \AgdaFunction{\_∼op\_} \AgdaSymbol{;} \<[19]%
\>[19]\<%
\\
\>[2]\AgdaIndent{5}{}\<[5]%
\>[5]\AgdaField{isEquivalence} \AgdaSymbol{=} \AgdaKeyword{record} \AgdaSymbol{\{} \<[30]%
\>[30]\<%
\\
\>[5]\AgdaIndent{7}{}\<[7]%
\>[7]\AgdaField{refl} \AgdaSymbol{=} \AgdaFunction{∼-refl} \AgdaSymbol{;} \<[23]%
\>[23]\<%
\\
\>[5]\AgdaIndent{7}{}\<[7]%
\>[7]\AgdaField{sym} \AgdaSymbol{=} \AgdaFunction{∼-sym} \AgdaSymbol{;} \<[21]%
\>[21]\<%
\\
\>[5]\AgdaIndent{7}{}\<[7]%
\>[7]\AgdaField{trans} \AgdaSymbol{=} \AgdaFunction{∼-trans} \AgdaSymbol{\}} \AgdaSymbol{\}}\<%
\end{code}

\AgdaHide{
\begin{code}%
\>\AgdaKeyword{open} \AgdaKeyword{import} \AgdaModule{Grammar.Base}\<%
\\
%
\\
\>\AgdaKeyword{module} \AgdaModule{Grammar.OpFamily.Lifting} \AgdaSymbol{(}\AgdaBound{G} \AgdaSymbol{:} \AgdaRecord{Grammar}\AgdaSymbol{)} \AgdaKeyword{where}\<%
\\
\>\AgdaKeyword{open} \AgdaKeyword{import} \AgdaModule{Data.List}\<%
\\
\>\AgdaKeyword{open} \AgdaKeyword{import} \AgdaModule{Prelims}\<%
\\
\>\AgdaKeyword{open} \AgdaModule{Grammar} \AgdaBound{G}\<%
\\
\>\AgdaKeyword{open} \AgdaKeyword{import} \AgdaModule{Grammar.OpFamily.PreOpFamily} \AgdaBound{G}\<%
\end{code}
}

\subsubsection{Liftings}

Define a \emph{lifting} on a pre-family to be an function $(- , K)$ that respects $\sim$:

\begin{code}%
\>\AgdaKeyword{record} \AgdaRecord{Lifting} \AgdaSymbol{(}\AgdaBound{F} \AgdaSymbol{:} \AgdaRecord{PreOpFamily}\AgdaSymbol{)} \AgdaSymbol{:} \AgdaPrimitiveType{Set₁} \AgdaKeyword{where}\<%
\\
\>[0]\AgdaIndent{2}{}\<[2]%
\>[2]\AgdaKeyword{open} \AgdaModule{PreOpFamily} \AgdaBound{F}\<%
\\
\>[0]\AgdaIndent{2}{}\<[2]%
\>[2]\AgdaKeyword{field}\<%
\\
\>[2]\AgdaIndent{4}{}\<[4]%
\>[4]\AgdaField{liftOp} \AgdaSymbol{:} \AgdaSymbol{∀} \AgdaSymbol{\{}\AgdaBound{U}\AgdaSymbol{\}} \AgdaSymbol{\{}\AgdaBound{V}\AgdaSymbol{\}} \AgdaBound{K} \AgdaSymbol{→} \AgdaFunction{Op} \AgdaBound{U} \AgdaBound{V} \AgdaSymbol{→} \AgdaFunction{Op} \AgdaSymbol{(}\AgdaBound{U} \AgdaInductiveConstructor{,} \AgdaBound{K}\AgdaSymbol{)} \AgdaSymbol{(}\AgdaBound{V} \AgdaInductiveConstructor{,} \AgdaBound{K}\AgdaSymbol{)}\<%
\\
\>[2]\AgdaIndent{4}{}\<[4]%
\>[4]\AgdaField{liftOp-cong} \AgdaSymbol{:} \AgdaSymbol{∀} \AgdaSymbol{\{}\AgdaBound{V}\AgdaSymbol{\}} \AgdaSymbol{\{}\AgdaBound{W}\AgdaSymbol{\}} \AgdaSymbol{\{}\AgdaBound{K}\AgdaSymbol{\}} \AgdaSymbol{\{}\AgdaBound{ρ} \AgdaBound{σ} \AgdaSymbol{:} \AgdaFunction{Op} \AgdaBound{V} \AgdaBound{W}\AgdaSymbol{\}} \AgdaSymbol{→} \<[49]%
\>[49]\<%
\\
\>[4]\AgdaIndent{6}{}\<[6]%
\>[6]\AgdaBound{ρ} \AgdaFunction{∼op} \AgdaBound{σ} \AgdaSymbol{→} \AgdaField{liftOp} \AgdaBound{K} \AgdaBound{ρ} \AgdaFunction{∼op} \AgdaField{liftOp} \AgdaBound{K} \AgdaBound{σ}\<%
\end{code}

Given an operation $\sigma : U \rightarrow V$ and a list of variable kinds $A \equiv (A_1 , \ldots , A_n)$, define
the \emph{repeated lifting} $\sigma^A$ to be $((\cdots(\sigma , A_1) , A_2) , \cdots ) , A_n)$.

\begin{code}%
\>\AgdaComment{\{-  liftOp' : ∀ \{U\} \{V\} A → Op U V → Op (extend U A) (extend V A)\<\\
\>  liftOp' [] σ = σ\<\\
\>  liftOp' (K ∷ A) σ = liftOp' A (liftOp K σ) -\}}\<%
\\
%
\\
\>[0]\AgdaIndent{2}{}\<[2]%
\>[2]\AgdaFunction{liftOp''} \AgdaSymbol{:} \AgdaSymbol{∀} \AgdaSymbol{\{}\AgdaBound{U}\AgdaSymbol{\}} \AgdaSymbol{\{}\AgdaBound{V}\AgdaSymbol{\}} \AgdaSymbol{\{}\AgdaBound{K}\AgdaSymbol{\}} \AgdaBound{A} \AgdaSymbol{→} \AgdaFunction{Op} \AgdaBound{U} \AgdaBound{V} \AgdaSymbol{→} \AgdaFunction{Op} \AgdaSymbol{(}\AgdaFunction{dom} \AgdaBound{U} \AgdaSymbol{\{}\AgdaBound{K}\AgdaSymbol{\}} \AgdaBound{A}\AgdaSymbol{)} \AgdaSymbol{(}\AgdaFunction{dom} \AgdaBound{V} \AgdaBound{A}\AgdaSymbol{)}\<%
\\
\>[0]\AgdaIndent{2}{}\<[2]%
\>[2]\AgdaFunction{liftOp''} \AgdaSymbol{(\_} \AgdaInductiveConstructor{✧}\AgdaSymbol{)} \AgdaBound{σ} \AgdaSymbol{=} \AgdaBound{σ}\<%
\\
\>[0]\AgdaIndent{2}{}\<[2]%
\>[2]\AgdaFunction{liftOp''} \AgdaSymbol{(}\AgdaBound{K} \AgdaInductiveConstructor{abs} \AgdaBound{A}\AgdaSymbol{)} \AgdaBound{σ} \AgdaSymbol{=} \AgdaFunction{liftOp''} \AgdaBound{A} \AgdaSymbol{(}\AgdaField{liftOp} \AgdaBound{K} \AgdaBound{σ}\AgdaSymbol{)}\<%
\\
%
\\
\>\AgdaComment{\{-  liftOp'-cong : ∀ \{U\} \{V\} A \{ρ σ : Op U V\} → \<\\
\>    ρ ∼op σ → liftOp' A ρ ∼op liftOp' A σ\<\\
\>}\<%
\end{code}

\AgdaHide{
\begin{code}%
\>\AgdaComment{\<\\
\>  liftOp'-cong [] ρ-is-σ = ρ-is-σ\<\\
\>  liftOp'-cong (\_ ∷ A) ρ-is-σ = liftOp'-cong A (liftOp-cong ρ-is-σ) -\}}\<%
\\
%
\\
\>[0]\AgdaIndent{2}{}\<[2]%
\>[2]\AgdaKeyword{postulate} \AgdaPostulate{liftOp''-cong} \AgdaSymbol{:} \AgdaSymbol{∀} \AgdaSymbol{\{}\AgdaBound{U}\AgdaSymbol{\}} \AgdaSymbol{\{}\AgdaBound{V}\AgdaSymbol{\}} \AgdaSymbol{\{}\AgdaBound{K}\AgdaSymbol{\}} \AgdaBound{A} \AgdaSymbol{\{}\AgdaBound{ρ} \AgdaBound{σ} \AgdaSymbol{:} \AgdaFunction{Op} \AgdaBound{U} \AgdaBound{V}\AgdaSymbol{\}} \AgdaSymbol{→} \<[61]%
\>[61]\<%
\\
\>[2]\AgdaIndent{26}{}\<[26]%
\>[26]\AgdaBound{ρ} \AgdaFunction{∼op} \AgdaBound{σ} \AgdaSymbol{→} \AgdaFunction{liftOp''} \AgdaSymbol{\{}\AgdaArgument{K} \AgdaSymbol{=} \AgdaBound{K}\AgdaSymbol{\}} \AgdaBound{A} \AgdaBound{ρ} \AgdaFunction{∼op} \AgdaFunction{liftOp''} \AgdaBound{A} \AgdaBound{σ}\<%
\end{code}
}

This allows us to define the action of \emph{application} $E[\sigma]$:

\begin{code}%
\>[0]\AgdaIndent{2}{}\<[2]%
\>[2]\AgdaFunction{ap} \AgdaSymbol{:} \AgdaSymbol{∀} \AgdaSymbol{\{}\AgdaBound{U}\AgdaSymbol{\}} \AgdaSymbol{\{}\AgdaBound{V}\AgdaSymbol{\}} \AgdaSymbol{\{}\AgdaBound{C}\AgdaSymbol{\}} \AgdaSymbol{\{}\AgdaBound{K}\AgdaSymbol{\}} \AgdaSymbol{→} \<[27]%
\>[27]\<%
\\
\>[2]\AgdaIndent{4}{}\<[4]%
\>[4]\AgdaFunction{Op} \AgdaBound{U} \AgdaBound{V} \AgdaSymbol{→} \AgdaDatatype{Subexpression} \AgdaBound{U} \AgdaBound{C} \AgdaBound{K} \AgdaSymbol{→} \AgdaDatatype{Subexpression} \AgdaBound{V} \AgdaBound{C} \AgdaBound{K}\<%
\\
\>[0]\AgdaIndent{2}{}\<[2]%
\>[2]\AgdaFunction{ap} \AgdaBound{ρ} \AgdaSymbol{(}\AgdaInductiveConstructor{var} \AgdaBound{x}\AgdaSymbol{)} \AgdaSymbol{=} \AgdaFunction{apV} \AgdaBound{ρ} \AgdaBound{x}\<%
\\
\>[0]\AgdaIndent{2}{}\<[2]%
\>[2]\AgdaFunction{ap} \AgdaBound{ρ} \AgdaSymbol{(}\AgdaInductiveConstructor{app} \AgdaBound{c} \AgdaBound{EE}\AgdaSymbol{)} \AgdaSymbol{=} \AgdaInductiveConstructor{app} \AgdaBound{c} \AgdaSymbol{(}\AgdaFunction{ap} \AgdaBound{ρ} \AgdaBound{EE}\AgdaSymbol{)}\<%
\\
\>[0]\AgdaIndent{2}{}\<[2]%
\>[2]\AgdaFunction{ap} \AgdaSymbol{\_} \AgdaInductiveConstructor{out} \AgdaSymbol{=} \AgdaInductiveConstructor{out}\<%
\\
\>[0]\AgdaIndent{2}{}\<[2]%
\>[2]\AgdaFunction{ap} \AgdaBound{ρ} \AgdaSymbol{(}\AgdaInductiveConstructor{\_,,\_} \AgdaSymbol{\{}\AgdaArgument{A} \AgdaSymbol{=} \AgdaBound{A}\AgdaSymbol{\}} \AgdaBound{E} \AgdaBound{EE}\AgdaSymbol{)} \AgdaSymbol{=} \AgdaFunction{ap} \AgdaSymbol{(}\AgdaFunction{liftOp''} \AgdaBound{A} \AgdaBound{ρ}\AgdaSymbol{)} \AgdaBound{E} \AgdaInductiveConstructor{,,} \AgdaFunction{ap} \AgdaBound{ρ} \AgdaBound{EE}\<%
\end{code}

We prove that application respects $\sim$.

\begin{code}%
\>[0]\AgdaIndent{2}{}\<[2]%
\>[2]\AgdaFunction{ap-congl} \AgdaSymbol{:} \AgdaSymbol{∀} \AgdaSymbol{\{}\AgdaBound{U}\AgdaSymbol{\}} \AgdaSymbol{\{}\AgdaBound{V}\AgdaSymbol{\}} \AgdaSymbol{\{}\AgdaBound{C}\AgdaSymbol{\}} \AgdaSymbol{\{}\AgdaBound{K}\AgdaSymbol{\}} \<[31]%
\>[31]\<%
\\
\>[2]\AgdaIndent{4}{}\<[4]%
\>[4]\AgdaSymbol{\{}\AgdaBound{ρ} \AgdaBound{σ} \AgdaSymbol{:} \AgdaFunction{Op} \AgdaBound{U} \AgdaBound{V}\AgdaSymbol{\}} \AgdaSymbol{(}\AgdaBound{E} \AgdaSymbol{:} \AgdaDatatype{Subexpression} \AgdaBound{U} \AgdaBound{C} \AgdaBound{K}\AgdaSymbol{)} \AgdaSymbol{→}\<%
\\
\>[2]\AgdaIndent{4}{}\<[4]%
\>[4]\AgdaBound{ρ} \AgdaFunction{∼op} \AgdaBound{σ} \AgdaSymbol{→} \AgdaFunction{ap} \AgdaBound{ρ} \AgdaBound{E} \AgdaDatatype{≡} \AgdaFunction{ap} \AgdaBound{σ} \AgdaBound{E}\<%
\end{code}

\AgdaHide{
\begin{code}%
\>[0]\AgdaIndent{2}{}\<[2]%
\>[2]\AgdaFunction{ap-congl} \AgdaSymbol{(}\AgdaInductiveConstructor{var} \AgdaBound{x}\AgdaSymbol{)} \AgdaBound{ρ-is-σ} \AgdaSymbol{=} \AgdaBound{ρ-is-σ} \AgdaBound{x}\<%
\\
\>[0]\AgdaIndent{2}{}\<[2]%
\>[2]\AgdaFunction{ap-congl} \AgdaSymbol{(}\AgdaInductiveConstructor{app} \AgdaBound{c} \AgdaBound{E}\AgdaSymbol{)} \AgdaBound{ρ-is-σ} \AgdaSymbol{=} \AgdaFunction{cong} \AgdaSymbol{(}\AgdaInductiveConstructor{app} \AgdaBound{c}\AgdaSymbol{)} \AgdaSymbol{(}\AgdaFunction{ap-congl} \AgdaBound{E} \AgdaBound{ρ-is-σ}\AgdaSymbol{)}\<%
\\
\>[0]\AgdaIndent{2}{}\<[2]%
\>[2]\AgdaFunction{ap-congl} \AgdaInductiveConstructor{out} \AgdaSymbol{\_} \AgdaSymbol{=} \AgdaInductiveConstructor{refl}\<%
\\
\>[0]\AgdaIndent{2}{}\<[2]%
\>[2]\AgdaFunction{ap-congl} \AgdaSymbol{(}\AgdaInductiveConstructor{\_,,\_} \AgdaSymbol{\{}\AgdaArgument{L} \AgdaSymbol{=} \AgdaBound{L}\AgdaSymbol{\}} \AgdaSymbol{\{}\AgdaArgument{A} \AgdaSymbol{=} \AgdaBound{A}\AgdaSymbol{\}} \AgdaBound{E} \AgdaBound{F}\AgdaSymbol{)} \AgdaBound{ρ-is-σ} \AgdaSymbol{=} \<[47]%
\>[47]\<%
\\
\>[2]\AgdaIndent{4}{}\<[4]%
\>[4]\AgdaFunction{cong₂} \AgdaInductiveConstructor{\_,,\_} \AgdaSymbol{(}\AgdaFunction{ap-congl} \AgdaBound{E} \AgdaSymbol{(}\AgdaPostulate{liftOp''-cong} \AgdaBound{A} \AgdaBound{ρ-is-σ}\AgdaSymbol{))} \AgdaSymbol{(}\AgdaFunction{ap-congl} \AgdaBound{F} \AgdaBound{ρ-is-σ}\AgdaSymbol{)}\<%
\\
%
\\
\>[0]\AgdaIndent{2}{}\<[2]%
\>[2]\AgdaFunction{ap-congr} \AgdaSymbol{:} \AgdaSymbol{∀} \AgdaSymbol{\{}\AgdaBound{U}\AgdaSymbol{\}} \AgdaSymbol{\{}\AgdaBound{V}\AgdaSymbol{\}} \AgdaSymbol{\{}\AgdaBound{C}\AgdaSymbol{\}} \AgdaSymbol{\{}\AgdaBound{K}\AgdaSymbol{\}}\<%
\\
\>[2]\AgdaIndent{4}{}\<[4]%
\>[4]\AgdaSymbol{\{}\AgdaBound{σ} \AgdaSymbol{:} \AgdaFunction{Op} \AgdaBound{U} \AgdaBound{V}\AgdaSymbol{\}} \AgdaSymbol{\{}\AgdaBound{E} \AgdaBound{F} \AgdaSymbol{:} \AgdaDatatype{Subexpression} \AgdaBound{U} \AgdaBound{C} \AgdaBound{K}\AgdaSymbol{\}} \AgdaSymbol{→}\<%
\\
\>[2]\AgdaIndent{4}{}\<[4]%
\>[4]\AgdaBound{E} \AgdaDatatype{≡} \AgdaBound{F} \AgdaSymbol{→} \AgdaFunction{ap} \AgdaBound{σ} \AgdaBound{E} \AgdaDatatype{≡} \AgdaFunction{ap} \AgdaBound{σ} \AgdaBound{F}\<%
\\
\>[0]\AgdaIndent{2}{}\<[2]%
\>[2]\AgdaFunction{ap-congr} \AgdaSymbol{\{}\AgdaArgument{σ} \AgdaSymbol{=} \AgdaBound{σ}\AgdaSymbol{\}} \AgdaSymbol{=} \AgdaFunction{cong} \AgdaSymbol{(}\AgdaFunction{ap} \AgdaBound{σ}\AgdaSymbol{)}\<%
\\
%
\\
\>[0]\AgdaIndent{2}{}\<[2]%
\>[2]\AgdaFunction{ap-cong} \AgdaSymbol{:} \AgdaSymbol{∀} \AgdaSymbol{\{}\AgdaBound{U}\AgdaSymbol{\}} \AgdaSymbol{\{}\AgdaBound{V}\AgdaSymbol{\}} \AgdaSymbol{\{}\AgdaBound{C}\AgdaSymbol{\}} \AgdaSymbol{\{}\AgdaBound{K}\AgdaSymbol{\}}\<%
\\
\>[2]\AgdaIndent{4}{}\<[4]%
\>[4]\AgdaSymbol{\{}\AgdaBound{ρ} \AgdaBound{σ} \AgdaSymbol{:} \AgdaFunction{Op} \AgdaBound{U} \AgdaBound{V}\AgdaSymbol{\}} \AgdaSymbol{\{}\AgdaBound{M} \AgdaBound{N} \AgdaSymbol{:} \AgdaDatatype{Subexpression} \AgdaBound{U} \AgdaBound{C} \AgdaBound{K}\AgdaSymbol{\}} \AgdaSymbol{→}\<%
\\
\>[2]\AgdaIndent{4}{}\<[4]%
\>[4]\AgdaBound{ρ} \AgdaFunction{∼op} \AgdaBound{σ} \AgdaSymbol{→} \AgdaBound{M} \AgdaDatatype{≡} \AgdaBound{N} \AgdaSymbol{→} \AgdaFunction{ap} \AgdaBound{ρ} \AgdaBound{M} \AgdaDatatype{≡} \AgdaFunction{ap} \AgdaBound{σ} \AgdaBound{N}\<%
\\
\>[0]\AgdaIndent{2}{}\<[2]%
\>[2]\AgdaFunction{ap-cong} \AgdaSymbol{\{}\AgdaArgument{ρ} \AgdaSymbol{=} \AgdaBound{ρ}\AgdaSymbol{\}} \AgdaSymbol{\{}\AgdaBound{σ}\AgdaSymbol{\}} \AgdaSymbol{\{}\AgdaBound{M}\AgdaSymbol{\}} \AgdaSymbol{\{}\AgdaBound{N}\AgdaSymbol{\}} \AgdaBound{ρ∼σ} \AgdaBound{M≡N} \AgdaSymbol{=} \AgdaKeyword{let} \AgdaKeyword{open} \AgdaModule{≡-Reasoning} \AgdaKeyword{in} \<[64]%
\>[64]\<%
\\
\>[2]\AgdaIndent{4}{}\<[4]%
\>[4]\AgdaFunction{begin}\<%
\\
\>[4]\AgdaIndent{6}{}\<[6]%
\>[6]\AgdaFunction{ap} \AgdaBound{ρ} \AgdaBound{M}\<%
\\
\>[0]\AgdaIndent{4}{}\<[4]%
\>[4]\AgdaFunction{≡⟨} \AgdaFunction{ap-congl} \AgdaBound{M} \AgdaBound{ρ∼σ} \AgdaFunction{⟩}\<%
\\
\>[4]\AgdaIndent{6}{}\<[6]%
\>[6]\AgdaFunction{ap} \AgdaBound{σ} \AgdaBound{M}\<%
\\
\>[0]\AgdaIndent{4}{}\<[4]%
\>[4]\AgdaFunction{≡⟨} \AgdaFunction{ap-congr} \AgdaBound{M≡N} \AgdaFunction{⟩}\<%
\\
\>[4]\AgdaIndent{6}{}\<[6]%
\>[6]\AgdaFunction{ap} \AgdaBound{σ} \AgdaBound{N}\<%
\\
\>[0]\AgdaIndent{4}{}\<[4]%
\>[4]\AgdaFunction{∎}\<%
\end{code}
}

\begin{code}%
\>\AgdaKeyword{open} \AgdaKeyword{import} \AgdaModule{Grammar.Base}\<%
\\
%
\\
\>\AgdaKeyword{module} \AgdaModule{Grammar.Substitution.RepSub} \AgdaSymbol{(}\AgdaBound{G} \AgdaSymbol{:} \AgdaRecord{Grammar}\AgdaSymbol{)} \AgdaKeyword{where}\<%
\\
\>\AgdaKeyword{open} \AgdaKeyword{import} \AgdaModule{Data.List}\<%
\\
\>\AgdaKeyword{open} \AgdaKeyword{import} \AgdaModule{Prelims}\<%
\\
\>\AgdaKeyword{open} \AgdaModule{Grammar} \AgdaBound{G}\<%
\\
\>\AgdaKeyword{open} \AgdaKeyword{import} \AgdaModule{Grammar.OpFamily} \AgdaBound{G}\<%
\\
\>\AgdaKeyword{open} \AgdaKeyword{import} \AgdaModule{Grammar.Replacement} \AgdaBound{G}\<%
\\
\>\AgdaKeyword{open} \AgdaKeyword{import} \AgdaModule{Grammar.Substitution.PreOpFamily} \AgdaBound{G}\<%
\\
\>\AgdaKeyword{open} \AgdaKeyword{import} \AgdaModule{Grammar.Substitution.Lifting} \AgdaBound{G}\<%
\\
%
\\
\>\AgdaKeyword{open} \AgdaModule{OpFamily} \AgdaFunction{replacement} \AgdaKeyword{using} \AgdaSymbol{()} \AgdaKeyword{renaming} \AgdaSymbol{(}liftOp' \AgdaSymbol{to} liftOp'R\AgdaSymbol{)}\<%
\\
\>\AgdaKeyword{open} \AgdaModule{PreOpFamily} \AgdaFunction{pre-substitution}\<%
\\
\>\AgdaKeyword{open} \AgdaModule{Lifting} \AgdaFunction{SUB↑}\<%
\end{code}

We can consider replacement to be a special case of substitution.  That is,
we can identify every replacement $\rho : U \rightarrow V$ with the substitution
that maps $x$ to $\rho(x)$.  
\begin{lemma}
Let $\rho$ be a replacement $U \rightarrow V$.
\begin{enumerate}
\item
The replacement $(\rho , K)$ and the substitution $(\rho , K)$ are equal.
\item
The replacement $\uparrow$ and the substitution $\uparrow$ are equal.
\item
The replacement $\rho^A$ and the substitution $\rho^A$ are equal.
\item
$ E \langle \rho \rangle \equiv E [ \rho ] $
\item
Hence $ E \langle \uparrow \rangle \equiv E [ \uparrow ]$.
\item
Substitution is a pre-family with lifting.
\end{enumerate}
\end{lemma}

\begin{code}%
\>\AgdaFunction{rep2sub} \AgdaSymbol{:} \AgdaSymbol{∀} \AgdaSymbol{\{}\AgdaBound{U}\AgdaSymbol{\}} \AgdaSymbol{\{}\AgdaBound{V}\AgdaSymbol{\}} \AgdaSymbol{→} \AgdaFunction{Rep} \AgdaBound{U} \AgdaBound{V} \AgdaSymbol{→} \AgdaFunction{Sub} \AgdaBound{U} \AgdaBound{V}\<%
\\
\>\AgdaFunction{rep2sub} \AgdaBound{ρ} \AgdaBound{K} \AgdaBound{x} \AgdaSymbol{=} \AgdaInductiveConstructor{var} \AgdaSymbol{(}\AgdaBound{ρ} \AgdaBound{K} \AgdaBound{x}\AgdaSymbol{)}\<%
\\
%
\\
\>\AgdaFunction{rep↑-is-sub↑} \AgdaSymbol{:} \AgdaSymbol{∀} \AgdaSymbol{\{}\AgdaBound{U}\AgdaSymbol{\}} \AgdaSymbol{\{}\AgdaBound{V}\AgdaSymbol{\}} \AgdaSymbol{\{}\AgdaBound{ρ} \AgdaSymbol{:} \AgdaFunction{Rep} \AgdaBound{U} \AgdaBound{V}\AgdaSymbol{\}} \AgdaSymbol{\{}\AgdaBound{K}\AgdaSymbol{\}} \AgdaSymbol{→} \<[45]%
\>[45]\<%
\\
\>[0]\AgdaIndent{2}{}\<[2]%
\>[2]\AgdaFunction{rep2sub} \AgdaSymbol{(}\AgdaFunction{rep↑} \AgdaBound{K} \AgdaBound{ρ}\AgdaSymbol{)} \AgdaFunction{∼} \AgdaFunction{sub↑} \AgdaBound{K} \AgdaSymbol{(}\AgdaFunction{rep2sub} \AgdaBound{ρ}\AgdaSymbol{)}\<%
\end{code}

\AgdaHide{
\begin{code}%
\>\AgdaFunction{rep↑-is-sub↑} \AgdaInductiveConstructor{x₀} \AgdaSymbol{=} \AgdaInductiveConstructor{refl}\<%
\\
\>\AgdaFunction{rep↑-is-sub↑} \AgdaSymbol{(}\AgdaInductiveConstructor{↑} \AgdaSymbol{\_)} \AgdaSymbol{=} \AgdaInductiveConstructor{refl}\<%
\end{code}
}

\begin{code}%
\>\AgdaFunction{up-is-up} \AgdaSymbol{:} \AgdaSymbol{∀} \AgdaSymbol{\{}\AgdaBound{V}\AgdaSymbol{\}} \AgdaSymbol{\{}\AgdaBound{K}\AgdaSymbol{\}} \AgdaSymbol{→} \AgdaFunction{rep2sub} \AgdaSymbol{(}\AgdaFunction{upRep} \AgdaSymbol{\{}\AgdaBound{V}\AgdaSymbol{\}} \AgdaSymbol{\{}\AgdaBound{K}\AgdaSymbol{\})} \AgdaFunction{∼} \AgdaFunction{upSub}\<%
\end{code}

\AgdaHide{
\begin{code}%
\>\AgdaFunction{up-is-up} \AgdaSymbol{\_} \AgdaSymbol{=} \AgdaInductiveConstructor{refl}\<%
\end{code}
}

\begin{code}%
\>\AgdaFunction{liftOp'-is-liftOp'} \AgdaSymbol{:} \AgdaSymbol{∀} \AgdaSymbol{\{}\AgdaBound{U}\AgdaSymbol{\}} \AgdaSymbol{\{}\AgdaBound{V}\AgdaSymbol{\}} \AgdaSymbol{\{}\AgdaBound{ρ} \AgdaSymbol{:} \AgdaFunction{Rep} \AgdaBound{U} \AgdaBound{V}\AgdaSymbol{\}} \AgdaSymbol{\{}\AgdaBound{A}\AgdaSymbol{\}} \AgdaSymbol{→} \<[51]%
\>[51]\<%
\\
\>[0]\AgdaIndent{2}{}\<[2]%
\>[2]\AgdaFunction{rep2sub} \AgdaSymbol{(}\AgdaFunction{liftOp'R} \<[21]%
\>[21]\AgdaBound{A} \AgdaBound{ρ}\AgdaSymbol{)} \AgdaFunction{∼} \AgdaFunction{liftOp'} \AgdaBound{A} \AgdaSymbol{(}\AgdaFunction{rep2sub} \AgdaBound{ρ}\AgdaSymbol{)}\<%
\end{code}

\AgdaHide{
\begin{code}%
\>\AgdaFunction{liftOp'-is-liftOp'} \AgdaSymbol{\{}\AgdaArgument{ρ} \AgdaSymbol{=} \AgdaBound{ρ}\AgdaSymbol{\}} \AgdaSymbol{\{}\AgdaArgument{A} \AgdaSymbol{=} \AgdaInductiveConstructor{[]}\AgdaSymbol{\}} \AgdaSymbol{=} \AgdaFunction{∼-refl} \AgdaSymbol{\{}\AgdaArgument{σ} \AgdaSymbol{=} \AgdaFunction{rep2sub} \AgdaBound{ρ}\AgdaSymbol{\}}\<%
\\
\>\AgdaFunction{liftOp'-is-liftOp'} \AgdaSymbol{\{}\AgdaBound{U}\AgdaSymbol{\}} \AgdaSymbol{\{}\AgdaBound{V}\AgdaSymbol{\}} \AgdaSymbol{\{}\AgdaBound{ρ}\AgdaSymbol{\}} \AgdaSymbol{\{}\AgdaBound{K} \AgdaInductiveConstructor{∷} \AgdaBound{A}\AgdaSymbol{\}} \AgdaSymbol{=} \AgdaKeyword{let} \AgdaKeyword{open} \AgdaModule{EqReasoning} \AgdaSymbol{(}\AgdaFunction{OP} \AgdaSymbol{\_} \AgdaSymbol{\_)} \AgdaKeyword{in} \<[74]%
\>[74]\<%
\\
\>[0]\AgdaIndent{2}{}\<[2]%
\>[2]\AgdaFunction{begin}\<%
\\
\>[2]\AgdaIndent{4}{}\<[4]%
\>[4]\AgdaFunction{rep2sub} \AgdaSymbol{(}\AgdaFunction{liftOp'R} \AgdaBound{A} \AgdaSymbol{(}\AgdaFunction{rep↑} \AgdaBound{K} \AgdaBound{ρ}\AgdaSymbol{))}\<%
\\
\>[0]\AgdaIndent{2}{}\<[2]%
\>[2]\AgdaFunction{≈⟨} \AgdaFunction{liftOp'-is-liftOp'} \AgdaSymbol{\{}\AgdaArgument{A} \AgdaSymbol{=} \AgdaBound{A}\AgdaSymbol{\}} \AgdaFunction{⟩}\<%
\\
\>[2]\AgdaIndent{4}{}\<[4]%
\>[4]\AgdaFunction{liftOp'} \AgdaBound{A} \AgdaSymbol{(}\AgdaFunction{rep2sub} \AgdaSymbol{(}\AgdaFunction{rep↑} \AgdaBound{K} \AgdaBound{ρ}\AgdaSymbol{))}\<%
\\
\>[0]\AgdaIndent{2}{}\<[2]%
\>[2]\AgdaFunction{≈⟨} \AgdaFunction{liftOp'-cong} \AgdaBound{A} \AgdaFunction{rep↑-is-sub↑} \AgdaFunction{⟩}\<%
\\
\>[2]\AgdaIndent{4}{}\<[4]%
\>[4]\AgdaFunction{liftOp'} \AgdaBound{A} \AgdaSymbol{(}\AgdaFunction{sub↑} \AgdaBound{K} \AgdaSymbol{(}\AgdaFunction{rep2sub} \AgdaBound{ρ}\AgdaSymbol{))}\<%
\\
\>[0]\AgdaIndent{2}{}\<[2]%
\>[2]\AgdaFunction{∎}\<%
\end{code}
}

\begin{code}%
\>\AgdaFunction{rep-is-sub} \AgdaSymbol{:} \AgdaSymbol{∀} \AgdaSymbol{\{}\AgdaBound{U}\AgdaSymbol{\}} \AgdaSymbol{\{}\AgdaBound{V}\AgdaSymbol{\}} \AgdaSymbol{\{}\AgdaBound{K}\AgdaSymbol{\}} \AgdaSymbol{\{}\AgdaBound{C}\AgdaSymbol{\}} \AgdaSymbol{(}\AgdaBound{E} \AgdaSymbol{:} \AgdaDatatype{Subexpression} \AgdaBound{U} \AgdaBound{K} \AgdaBound{C}\AgdaSymbol{)} \AgdaSymbol{\{}\AgdaBound{ρ} \AgdaSymbol{:} \AgdaFunction{Rep} \AgdaBound{U} \AgdaBound{V}\AgdaSymbol{\}} \AgdaSymbol{→} \<[73]%
\>[73]\<%
\\
\>[0]\AgdaIndent{2}{}\<[2]%
\>[2]\AgdaBound{E} \AgdaFunction{〈} \AgdaBound{ρ} \AgdaFunction{〉} \AgdaDatatype{≡} \AgdaBound{E} \AgdaFunction{⟦} \AgdaFunction{rep2sub} \AgdaBound{ρ} \AgdaFunction{⟧}\<%
\end{code}

\AgdaHide{
\begin{code}%
\>\AgdaFunction{rep-is-sub} \AgdaSymbol{(}\AgdaInductiveConstructor{var} \AgdaSymbol{\_)} \AgdaSymbol{=} \AgdaInductiveConstructor{refl}\<%
\\
\>\AgdaFunction{rep-is-sub} \AgdaSymbol{(}\AgdaInductiveConstructor{app} \AgdaBound{c} \AgdaBound{E}\AgdaSymbol{)} \AgdaSymbol{=} \AgdaFunction{cong} \AgdaSymbol{(}\AgdaInductiveConstructor{app} \AgdaBound{c}\AgdaSymbol{)} \AgdaSymbol{(}\AgdaFunction{rep-is-sub} \AgdaBound{E}\AgdaSymbol{)}\<%
\\
\>\AgdaFunction{rep-is-sub} \AgdaInductiveConstructor{out} \AgdaSymbol{=} \AgdaInductiveConstructor{refl}\<%
\\
\>\AgdaFunction{rep-is-sub} \AgdaSymbol{\{}\AgdaBound{U}\AgdaSymbol{\}} \AgdaSymbol{\{}\AgdaBound{V}\AgdaSymbol{\}} \AgdaSymbol{(}\AgdaInductiveConstructor{\_,,\_} \AgdaSymbol{\{}\AgdaArgument{A} \AgdaSymbol{=} \AgdaBound{A}\AgdaSymbol{\}} \AgdaSymbol{\{}\AgdaArgument{L} \AgdaSymbol{=} \AgdaBound{L}\AgdaSymbol{\}} \AgdaBound{E} \AgdaBound{F}\AgdaSymbol{)} \AgdaSymbol{\{}\AgdaBound{ρ}\AgdaSymbol{\}} \AgdaSymbol{=} \AgdaFunction{cong₂} \AgdaInductiveConstructor{\_,,\_} \<[63]%
\>[63]\<%
\\
\>[0]\AgdaIndent{2}{}\<[2]%
\>[2]\AgdaSymbol{(}\AgdaKeyword{let} \AgdaKeyword{open} \AgdaModule{≡-Reasoning} \AgdaSymbol{\{}\AgdaArgument{A} \AgdaSymbol{=} \AgdaFunction{Expression} \AgdaSymbol{(}\AgdaFunction{extend} \AgdaBound{V} \AgdaBound{A}\AgdaSymbol{)} \AgdaBound{L}\AgdaSymbol{\}} \AgdaKeyword{in}\<%
\\
\>[0]\AgdaIndent{2}{}\<[2]%
\>[2]\AgdaFunction{begin} \<[8]%
\>[8]\<%
\\
\>[2]\AgdaIndent{4}{}\<[4]%
\>[4]\AgdaBound{E} \AgdaFunction{〈} \AgdaFunction{liftOp'R} \AgdaBound{A} \AgdaBound{ρ} \AgdaFunction{〉}\<%
\\
\>[0]\AgdaIndent{2}{}\<[2]%
\>[2]\AgdaFunction{≡⟨} \AgdaFunction{rep-is-sub} \AgdaBound{E} \AgdaFunction{⟩}\<%
\\
\>[2]\AgdaIndent{4}{}\<[4]%
\>[4]\AgdaBound{E} \AgdaFunction{⟦} \AgdaSymbol{(λ} \AgdaBound{K} \AgdaBound{x} \AgdaSymbol{→} \AgdaInductiveConstructor{var} \AgdaSymbol{(}\AgdaFunction{liftOp'R} \AgdaBound{A} \AgdaBound{ρ} \AgdaBound{K} \AgdaBound{x}\AgdaSymbol{))} \AgdaFunction{⟧} \<[43]%
\>[43]\<%
\\
\>[0]\AgdaIndent{2}{}\<[2]%
\>[2]\AgdaFunction{≡⟨} \AgdaFunction{ap-congl} \AgdaBound{E} \AgdaSymbol{(}\AgdaFunction{liftOp'-is-liftOp'} \AgdaSymbol{\{}\AgdaArgument{A} \AgdaSymbol{=} \AgdaBound{A}\AgdaSymbol{\})} \AgdaFunction{⟩}\<%
\\
\>[2]\AgdaIndent{4}{}\<[4]%
\>[4]\AgdaBound{E} \AgdaFunction{⟦} \AgdaFunction{liftOp'} \AgdaBound{A} \AgdaSymbol{(λ} \AgdaBound{K} \AgdaBound{x} \AgdaSymbol{→} \AgdaInductiveConstructor{var} \AgdaSymbol{(}\AgdaBound{ρ} \AgdaBound{K} \AgdaBound{x}\AgdaSymbol{))} \AgdaFunction{⟧}\<%
\\
\>[0]\AgdaIndent{2}{}\<[2]%
\>[2]\AgdaFunction{∎}\AgdaSymbol{)}\<%
\\
\>[0]\AgdaIndent{2}{}\<[2]%
\>[2]\AgdaSymbol{(}\AgdaFunction{rep-is-sub} \AgdaBound{F}\AgdaSymbol{)}\<%
\end{code}
}

\begin{code}%
\>\AgdaFunction{up-is-up'} \AgdaSymbol{:} \AgdaSymbol{∀} \AgdaSymbol{\{}\AgdaBound{V}\AgdaSymbol{\}} \AgdaSymbol{\{}\AgdaBound{C}\AgdaSymbol{\}} \AgdaSymbol{\{}\AgdaBound{K}\AgdaSymbol{\}} \AgdaSymbol{\{}\AgdaBound{L}\AgdaSymbol{\}} \AgdaSymbol{\{}\AgdaBound{E} \AgdaSymbol{:} \AgdaDatatype{Subexpression} \AgdaBound{V} \AgdaBound{C} \AgdaBound{K}\AgdaSymbol{\}} \AgdaSymbol{→} \<[58]%
\>[58]\<%
\\
\>[0]\AgdaIndent{2}{}\<[2]%
\>[2]\AgdaBound{E} \AgdaFunction{〈} \AgdaFunction{upRep} \AgdaSymbol{\{}\AgdaArgument{K} \AgdaSymbol{=} \AgdaBound{L}\AgdaSymbol{\}} \AgdaFunction{〉} \AgdaDatatype{≡} \AgdaBound{E} \AgdaFunction{⟦} \AgdaFunction{upSub} \AgdaFunction{⟧}\<%
\end{code}

\AgdaHide{
\begin{code}%
\>\AgdaFunction{up-is-up'} \AgdaSymbol{\{}\AgdaArgument{E} \AgdaSymbol{=} \AgdaBound{E}\AgdaSymbol{\}} \AgdaSymbol{=} \AgdaFunction{rep-is-sub} \AgdaBound{E}\<%
\end{code}
}

\AgdaHide{
\begin{code}%
\>\AgdaKeyword{open} \AgdaKeyword{import} \AgdaModule{Grammar.Base}\<%
\\
%
\\
\>\AgdaKeyword{module} \AgdaModule{Grammar.Substitution.LiftFamily} \AgdaSymbol{(}\AgdaBound{G} \AgdaSymbol{:} \AgdaRecord{Grammar}\AgdaSymbol{)} \AgdaKeyword{where}\<%
\\
\>\AgdaKeyword{open} \AgdaKeyword{import} \AgdaModule{Prelims}\<%
\\
\>\AgdaKeyword{open} \AgdaModule{Grammar} \AgdaBound{G}\<%
\\
\>\AgdaKeyword{open} \AgdaKeyword{import} \AgdaModule{Grammar.OpFamily.LiftFamily} \AgdaBound{G}\<%
\\
\>\AgdaKeyword{open} \AgdaKeyword{import} \AgdaModule{Grammar.Substitution.PreOpFamily} \AgdaBound{G}\<%
\\
\>\AgdaKeyword{open} \AgdaKeyword{import} \AgdaModule{Grammar.Substitution.Lifting} \AgdaBound{G}\<%
\\
\>\AgdaKeyword{open} \AgdaKeyword{import} \AgdaModule{Grammar.Substitution.RepSub} \AgdaBound{G}\<%
\end{code}
}

It is now easy to show that substitution forms a pre-family with lifting.  If $\sigma : U \rightarrow V$ and $x \in U$ then $(\sigma , K)(\uparrow x) \equiv
\sigma(x) \langle \uparrow \rangle \equiv (\sigma , K)(x) [ \uparrow ]$.

\begin{code}%
\>\AgdaFunction{SubLF} \AgdaSymbol{:} \AgdaRecord{LiftFamily}\<%
\\
\>\AgdaFunction{SubLF} \AgdaSymbol{=} \AgdaKeyword{record} \AgdaSymbol{\{} \<[17]%
\>[17]\<%
\\
\>[0]\AgdaIndent{2}{}\<[2]%
\>[2]\AgdaField{preOpFamily} \AgdaSymbol{=} \AgdaFunction{pre-substitution} \AgdaSymbol{;} \<[35]%
\>[35]\<%
\\
\>[0]\AgdaIndent{2}{}\<[2]%
\>[2]\AgdaField{lifting} \AgdaSymbol{=} \AgdaFunction{SUB↑} \AgdaSymbol{;} \<[19]%
\>[19]\<%
\\
\>[0]\AgdaIndent{2}{}\<[2]%
\>[2]\AgdaField{isLiftFamily} \AgdaSymbol{=} \AgdaKeyword{record} \AgdaSymbol{\{} \<[26]%
\>[26]\<%
\\
\>[2]\AgdaIndent{4}{}\<[4]%
\>[4]\AgdaField{liftOp-x₀} \AgdaSymbol{=} \AgdaInductiveConstructor{refl} \AgdaSymbol{;} \<[23]%
\>[23]\<%
\\
\>[2]\AgdaIndent{4}{}\<[4]%
\>[4]\AgdaField{liftOp-↑} \AgdaSymbol{=} \AgdaSymbol{λ} \AgdaSymbol{\{}\AgdaBound{\_}\AgdaSymbol{\}} \AgdaSymbol{\{}\AgdaBound{\_}\AgdaSymbol{\}} \AgdaSymbol{\{}\AgdaBound{\_}\AgdaSymbol{\}} \AgdaSymbol{\{}\AgdaBound{\_}\AgdaSymbol{\}} \AgdaSymbol{\{}\AgdaBound{σ}\AgdaSymbol{\}} \AgdaBound{x} \AgdaSymbol{→} \AgdaFunction{rep-is-sub} \AgdaSymbol{(}\AgdaBound{σ} \AgdaSymbol{\_} \AgdaBound{x}\AgdaSymbol{)} \AgdaSymbol{\}\}}\<%
\end{code}

\AgdaHide{
\begin{code}%
\>\AgdaKeyword{open} \AgdaKeyword{import} \AgdaModule{Grammar.Base}\<%
\\
%
\\
\>\AgdaKeyword{module} \AgdaModule{Grammar.OpFamily.OpFamily} \AgdaSymbol{(}\AgdaBound{G} \AgdaSymbol{:} \AgdaRecord{Grammar}\AgdaSymbol{)} \AgdaKeyword{where}\<%
\\
%
\\
\>\AgdaKeyword{open} \AgdaKeyword{import} \AgdaModule{Prelims}\<%
\\
\>\AgdaKeyword{open} \AgdaModule{Grammar} \AgdaBound{G}\<%
\\
\>\AgdaKeyword{open} \AgdaKeyword{import} \AgdaModule{Grammar.OpFamily.LiftFamily} \AgdaBound{G}\<%
\\
\>\AgdaKeyword{open} \AgdaKeyword{import} \AgdaModule{Grammar.OpFamily.Composition} \AgdaBound{G}\<%
\end{code}
}

\subsubsection{Family of Operations}

Finally. we can define: a \emph{family of operations} is a pre-family with lift $F$ together with a composition $\circ : F;F \rightarrow F$.

\begin{code}%
\>\AgdaKeyword{record} \AgdaRecord{IsOpFamily} \AgdaSymbol{(}\AgdaBound{F} \AgdaSymbol{:} \AgdaRecord{LiftFamily}\AgdaSymbol{)} \AgdaSymbol{:} \AgdaPrimitiveType{Set₂} \AgdaKeyword{where}\<%
\\
\>[0]\AgdaIndent{2}{}\<[2]%
\>[2]\AgdaKeyword{open} \AgdaModule{LiftFamily} \AgdaBound{F} \AgdaKeyword{public}\<%
\\
\>[0]\AgdaIndent{2}{}\<[2]%
\>[2]\AgdaKeyword{field}\<%
\\
\>[2]\AgdaIndent{4}{}\<[4]%
\>[4]\AgdaField{comp} \AgdaSymbol{:} \AgdaRecord{Composition} \AgdaBound{F} \AgdaBound{F} \AgdaBound{F}\<%
\\
%
\\
\>\AgdaComment{\{-  infix 50 \_∘\_\<\\
\>  field\<\\
\>    \_∘\_ : ∀ \{U\} \{V\} \{W\} → Op V W → Op U V → Op U W\<\\
\>    liftOp-comp : ∀ \{U\} \{V\} \{W\} \{K\} \{σ : Op V W\} \{ρ : Op U V\} →\<\\
\>      liftOp K (σ ∘ ρ) ∼op liftOp K σ ∘ liftOp K ρ\<\\
\>    apV-comp : ∀ \{U\} \{V\} \{W\} \{K\} \{σ : Op V W\} \{ρ : Op U V\} \{x : Var U K\} →\<\\
\>      apV (σ ∘ ρ) x ≡ ap σ (apV ρ x)\<\\
\>\<\\
\>  COMP : Composition F F F\<\\
\>  COMP = record \{ \<\\
\>    circ = \_∘\_ ; \<\\
\>    liftOp-circ = liftOp-comp ; \<\\
\>    apV-circ = apV-comp \} -\}}\<%
\\
%
\\
\>[0]\AgdaIndent{2}{}\<[2]%
\>[2]\AgdaKeyword{open} \AgdaModule{Composition} \AgdaField{comp} \AgdaKeyword{public}\<%
\\
%
\\
\>[0]\AgdaIndent{2}{}\<[2]%
\>[2]\AgdaFunction{comp-congl} \AgdaSymbol{:} \AgdaSymbol{∀} \AgdaSymbol{\{}\AgdaBound{U}\AgdaSymbol{\}} \AgdaSymbol{\{}\AgdaBound{V}\AgdaSymbol{\}} \AgdaSymbol{\{}\AgdaBound{W}\AgdaSymbol{\}} \AgdaSymbol{\{}\AgdaBound{σ} \AgdaBound{σ'} \AgdaSymbol{:} \AgdaFunction{Op} \AgdaBound{V} \AgdaBound{W}\AgdaSymbol{\}} \AgdaSymbol{\{}\AgdaBound{ρ} \AgdaSymbol{:} \AgdaFunction{Op} \AgdaBound{U} \AgdaBound{V}\AgdaSymbol{\}} \AgdaSymbol{→}\<%
\\
\>[2]\AgdaIndent{4}{}\<[4]%
\>[4]\AgdaBound{σ} \AgdaFunction{∼op} \AgdaBound{σ'} \AgdaSymbol{→} \AgdaBound{σ} \AgdaFunction{∘} \AgdaBound{ρ} \AgdaFunction{∼op} \AgdaBound{σ'} \AgdaFunction{∘} \AgdaBound{ρ}\<%
\\
\>[0]\AgdaIndent{2}{}\<[2]%
\>[2]\AgdaFunction{comp-congl} \AgdaSymbol{\{}\AgdaBound{U}\AgdaSymbol{\}} \AgdaSymbol{\{}\AgdaBound{V}\AgdaSymbol{\}} \AgdaSymbol{\{}\AgdaBound{W}\AgdaSymbol{\}} \AgdaSymbol{\{}\AgdaBound{σ}\AgdaSymbol{\}} \AgdaSymbol{\{}\AgdaBound{σ'}\AgdaSymbol{\}} \AgdaSymbol{\{}\AgdaBound{ρ}\AgdaSymbol{\}} \AgdaBound{σ∼σ'} \AgdaBound{x} \AgdaSymbol{=} \AgdaKeyword{let} \AgdaKeyword{open} \AgdaModule{≡-Reasoning} \AgdaKeyword{in} \<[71]%
\>[71]\<%
\\
\>[2]\AgdaIndent{4}{}\<[4]%
\>[4]\AgdaFunction{begin}\<%
\\
\>[4]\AgdaIndent{6}{}\<[6]%
\>[6]\AgdaFunction{apV} \AgdaSymbol{(}\AgdaBound{σ} \AgdaFunction{∘} \AgdaBound{ρ}\AgdaSymbol{)} \AgdaBound{x}\<%
\\
\>[0]\AgdaIndent{4}{}\<[4]%
\>[4]\AgdaFunction{≡⟨} \AgdaFunction{apV-comp} \AgdaFunction{⟩}\<%
\\
\>[4]\AgdaIndent{6}{}\<[6]%
\>[6]\AgdaFunction{ap} \AgdaBound{σ} \AgdaSymbol{(}\AgdaFunction{apV} \AgdaBound{ρ} \AgdaBound{x}\AgdaSymbol{)}\<%
\\
\>[0]\AgdaIndent{4}{}\<[4]%
\>[4]\AgdaFunction{≡⟨} \AgdaFunction{ap-congl} \AgdaBound{σ∼σ'} \AgdaSymbol{(}\AgdaFunction{apV} \AgdaBound{ρ} \AgdaBound{x}\AgdaSymbol{)} \AgdaFunction{⟩}\<%
\\
\>[4]\AgdaIndent{6}{}\<[6]%
\>[6]\AgdaFunction{ap} \AgdaBound{σ'} \AgdaSymbol{(}\AgdaFunction{apV} \AgdaBound{ρ} \AgdaBound{x}\AgdaSymbol{)}\<%
\\
\>[0]\AgdaIndent{4}{}\<[4]%
\>[4]\AgdaFunction{≡⟨⟨} \AgdaFunction{apV-comp} \AgdaFunction{⟩⟩}\<%
\\
\>[4]\AgdaIndent{6}{}\<[6]%
\>[6]\AgdaFunction{apV} \AgdaSymbol{(}\AgdaBound{σ'} \AgdaFunction{∘} \AgdaBound{ρ}\AgdaSymbol{)} \AgdaBound{x}\<%
\\
\>[0]\AgdaIndent{4}{}\<[4]%
\>[4]\AgdaFunction{∎}\<%
\\
\>[0]\AgdaIndent{2}{}\<[2]%
\>[2]\AgdaKeyword{postulate} \AgdaPostulate{comp-congr} \AgdaSymbol{:} \AgdaSymbol{∀} \AgdaSymbol{\{}\AgdaBound{U}\AgdaSymbol{\}} \AgdaSymbol{\{}\AgdaBound{V}\AgdaSymbol{\}} \AgdaSymbol{\{}\AgdaBound{W}\AgdaSymbol{\}} \AgdaSymbol{\{}\AgdaBound{σ} \AgdaSymbol{:} \AgdaFunction{Op} \AgdaBound{V} \AgdaBound{W}\AgdaSymbol{\}} \AgdaSymbol{\{}\AgdaBound{ρ} \AgdaBound{ρ'} \AgdaSymbol{:} \AgdaFunction{Op} \AgdaBound{U} \AgdaBound{V}\AgdaSymbol{\}} \AgdaSymbol{→}\<%
\\
\>[2]\AgdaIndent{23}{}\<[23]%
\>[23]\AgdaBound{ρ} \AgdaFunction{∼op} \AgdaBound{ρ'} \AgdaSymbol{→} \AgdaBound{σ} \AgdaFunction{∘} \AgdaBound{ρ} \AgdaFunction{∼op} \AgdaBound{σ} \AgdaFunction{∘} \AgdaBound{ρ'}\<%
\end{code}

The following results about operations are easy to prove.
\begin{lemma}$ $
  \begin{enumerate}
  \item $(\sigma , K) \circ \uparrow \sim \uparrow \circ \sigma$
  \item $(\id{V} , K) \sim \id{V,K}$
  \item $\id{V}[E] \equiv E$
  \item $(\sigma \circ \rho)[E] \equiv \sigma[\rho[E]]$
  \end{enumerate}
\end{lemma}

\begin{code}%
\>[0]\AgdaIndent{2}{}\<[2]%
\>[2]\AgdaFunction{liftOp-up'} \AgdaSymbol{:} \AgdaSymbol{∀} \AgdaSymbol{\{}\AgdaBound{U}\AgdaSymbol{\}} \AgdaSymbol{\{}\AgdaBound{V}\AgdaSymbol{\}} \AgdaSymbol{\{}\AgdaBound{C}\AgdaSymbol{\}} \AgdaSymbol{\{}\AgdaBound{K}\AgdaSymbol{\}} \AgdaSymbol{\{}\AgdaBound{L}\AgdaSymbol{\}}\<%
\\
\>[2]\AgdaIndent{4}{}\<[4]%
\>[4]\AgdaSymbol{\{}\AgdaBound{σ} \AgdaSymbol{:} \AgdaFunction{Op} \AgdaBound{U} \AgdaBound{V}\AgdaSymbol{\}} \AgdaSymbol{(}\AgdaBound{E} \AgdaSymbol{:} \AgdaDatatype{Subexpression} \AgdaBound{U} \AgdaBound{C} \AgdaBound{K}\AgdaSymbol{)} \AgdaSymbol{→}\<%
\\
\>[2]\AgdaIndent{4}{}\<[4]%
\>[4]\AgdaFunction{ap} \AgdaSymbol{(}\AgdaFunction{liftOp} \AgdaBound{L} \AgdaBound{σ}\AgdaSymbol{)} \AgdaSymbol{(}\AgdaFunction{ap} \AgdaFunction{up} \AgdaBound{E}\AgdaSymbol{)} \AgdaDatatype{≡} \AgdaFunction{ap} \AgdaFunction{up} \AgdaSymbol{(}\AgdaFunction{ap} \AgdaBound{σ} \AgdaBound{E}\AgdaSymbol{)}\<%
\end{code}

\AgdaHide{
\begin{code}%
\>[0]\AgdaIndent{2}{}\<[2]%
\>[2]\AgdaFunction{liftOp-up'} \AgdaBound{E} \AgdaSymbol{=} \AgdaFunction{liftOp-up-mixed} \AgdaField{comp} \AgdaField{comp} \AgdaInductiveConstructor{refl} \AgdaSymbol{\{}\AgdaArgument{E} \AgdaSymbol{=} \AgdaBound{E}\AgdaSymbol{\}}\<%
\end{code}
}

\newcommand{\Op}{\ensuremath{\mathbf{Op}}}

The alphabets and operations up to equivalence form
a category, which we denote $\Op$.
The action of application associates, with every operator family, a functor $\Op \rightarrow \Set$,
which maps an alphabet $U$ to the set of expressions over $U$, and every operation $\sigma$ to the function $\sigma[-]$.
This functor is faithful and injective on objects, and so $\Op$ can be seen as a subcategory of $\Set$.

\begin{code}%
\>[0]\AgdaIndent{2}{}\<[2]%
\>[2]\AgdaFunction{assoc} \AgdaSymbol{:} \AgdaSymbol{∀} \AgdaSymbol{\{}\AgdaBound{U}\AgdaSymbol{\}} \AgdaSymbol{\{}\AgdaBound{V}\AgdaSymbol{\}} \AgdaSymbol{\{}\AgdaBound{W}\AgdaSymbol{\}} \AgdaSymbol{\{}\AgdaBound{X}\AgdaSymbol{\}} \<[28]%
\>[28]\<%
\\
\>[2]\AgdaIndent{4}{}\<[4]%
\>[4]\AgdaSymbol{\{}\AgdaBound{τ} \AgdaSymbol{:} \AgdaFunction{Op} \AgdaBound{W} \AgdaBound{X}\AgdaSymbol{\}} \AgdaSymbol{\{}\AgdaBound{σ} \AgdaSymbol{:} \AgdaFunction{Op} \AgdaBound{V} \AgdaBound{W}\AgdaSymbol{\}} \AgdaSymbol{\{}\AgdaBound{ρ} \AgdaSymbol{:} \AgdaFunction{Op} \AgdaBound{U} \AgdaBound{V}\AgdaSymbol{\}} \AgdaSymbol{→} \<[45]%
\>[45]\<%
\\
\>[2]\AgdaIndent{4}{}\<[4]%
\>[4]\AgdaBound{τ} \AgdaFunction{∘} \AgdaSymbol{(}\AgdaBound{σ} \AgdaFunction{∘} \AgdaBound{ρ}\AgdaSymbol{)} \AgdaFunction{∼op} \AgdaSymbol{(}\AgdaBound{τ} \AgdaFunction{∘} \AgdaBound{σ}\AgdaSymbol{)} \AgdaFunction{∘} \AgdaBound{ρ}\<%
\end{code}

\AgdaHide{
\begin{code}%
\>[0]\AgdaIndent{2}{}\<[2]%
\>[2]\AgdaFunction{assoc} \AgdaSymbol{\{}\AgdaBound{U}\AgdaSymbol{\}} \AgdaSymbol{\{}\AgdaBound{V}\AgdaSymbol{\}} \AgdaSymbol{\{}\AgdaBound{W}\AgdaSymbol{\}} \AgdaSymbol{\{}\AgdaBound{X}\AgdaSymbol{\}} \AgdaSymbol{\{}\AgdaBound{τ}\AgdaSymbol{\}} \AgdaSymbol{\{}\AgdaBound{σ}\AgdaSymbol{\}} \AgdaSymbol{\{}\AgdaBound{ρ}\AgdaSymbol{\}} \AgdaSymbol{\{}\AgdaBound{K}\AgdaSymbol{\}} \AgdaBound{x} \AgdaSymbol{=} \AgdaKeyword{let} \AgdaKeyword{open} \AgdaModule{≡-Reasoning} \AgdaSymbol{\{}\AgdaArgument{A} \AgdaSymbol{=} \AgdaFunction{Expression} \AgdaBound{X} \AgdaSymbol{(}\AgdaInductiveConstructor{varKind} \AgdaBound{K}\AgdaSymbol{)\}} \AgdaKeyword{in} \<[99]%
\>[99]\<%
\\
\>[2]\AgdaIndent{4}{}\<[4]%
\>[4]\AgdaFunction{begin} \<[10]%
\>[10]\<%
\\
\>[4]\AgdaIndent{6}{}\<[6]%
\>[6]\AgdaFunction{apV} \AgdaSymbol{(}\AgdaBound{τ} \AgdaFunction{∘} \AgdaSymbol{(}\AgdaBound{σ} \AgdaFunction{∘} \AgdaBound{ρ}\AgdaSymbol{))} \AgdaBound{x}\<%
\\
\>[0]\AgdaIndent{4}{}\<[4]%
\>[4]\AgdaFunction{≡⟨} \AgdaFunction{apV-comp} \AgdaFunction{⟩}\<%
\\
\>[4]\AgdaIndent{6}{}\<[6]%
\>[6]\AgdaFunction{ap} \AgdaBound{τ} \AgdaSymbol{(}\AgdaFunction{apV} \AgdaSymbol{(}\AgdaBound{σ} \AgdaFunction{∘} \AgdaBound{ρ}\AgdaSymbol{)} \AgdaBound{x}\AgdaSymbol{)}\<%
\\
\>[0]\AgdaIndent{4}{}\<[4]%
\>[4]\AgdaFunction{≡⟨} \AgdaFunction{cong} \AgdaSymbol{(}\AgdaFunction{ap} \AgdaBound{τ}\AgdaSymbol{)} \AgdaFunction{apV-comp} \AgdaFunction{⟩}\<%
\\
\>[4]\AgdaIndent{6}{}\<[6]%
\>[6]\AgdaFunction{ap} \AgdaBound{τ} \AgdaSymbol{(}\AgdaFunction{ap} \AgdaBound{σ} \AgdaSymbol{(}\AgdaFunction{apV} \AgdaBound{ρ} \AgdaBound{x}\AgdaSymbol{))}\<%
\\
\>[0]\AgdaIndent{4}{}\<[4]%
\>[4]\AgdaFunction{≡⟨⟨} \AgdaFunction{ap-comp} \AgdaSymbol{(}\AgdaFunction{apV} \AgdaBound{ρ} \AgdaBound{x}\AgdaSymbol{)} \AgdaFunction{⟩⟩}\<%
\\
\>[4]\AgdaIndent{6}{}\<[6]%
\>[6]\AgdaFunction{ap} \AgdaSymbol{(}\AgdaBound{τ} \AgdaFunction{∘} \AgdaBound{σ}\AgdaSymbol{)} \AgdaSymbol{(}\AgdaFunction{apV} \AgdaBound{ρ} \AgdaBound{x}\AgdaSymbol{)}\<%
\\
\>[0]\AgdaIndent{4}{}\<[4]%
\>[4]\AgdaFunction{≡⟨⟨} \AgdaFunction{apV-comp} \AgdaFunction{⟩⟩}\<%
\\
\>[4]\AgdaIndent{6}{}\<[6]%
\>[6]\AgdaFunction{apV} \AgdaSymbol{((}\AgdaBound{τ} \AgdaFunction{∘} \AgdaBound{σ}\AgdaSymbol{)} \AgdaFunction{∘} \AgdaBound{ρ}\AgdaSymbol{)} \AgdaBound{x}\<%
\\
\>[0]\AgdaIndent{4}{}\<[4]%
\>[4]\AgdaFunction{∎}\<%
\end{code}
}

\begin{code}%
\>[0]\AgdaIndent{2}{}\<[2]%
\>[2]\AgdaFunction{unitl} \AgdaSymbol{:} \AgdaSymbol{∀} \AgdaSymbol{\{}\AgdaBound{U}\AgdaSymbol{\}} \AgdaSymbol{\{}\AgdaBound{V}\AgdaSymbol{\}} \AgdaSymbol{\{}\AgdaBound{σ} \AgdaSymbol{:} \AgdaFunction{Op} \AgdaBound{U} \AgdaBound{V}\AgdaSymbol{\}} \AgdaSymbol{→} \AgdaFunction{idOp} \AgdaBound{V} \AgdaFunction{∘} \AgdaBound{σ} \AgdaFunction{∼op} \AgdaBound{σ}\<%
\end{code}

\AgdaHide{
\begin{code}%
\>[0]\AgdaIndent{2}{}\<[2]%
\>[2]\AgdaFunction{unitl} \AgdaSymbol{\{}\AgdaBound{U}\AgdaSymbol{\}} \AgdaSymbol{\{}\AgdaBound{V}\AgdaSymbol{\}} \AgdaSymbol{\{}\AgdaBound{σ}\AgdaSymbol{\}} \AgdaSymbol{\{}\AgdaBound{K}\AgdaSymbol{\}} \AgdaBound{x} \AgdaSymbol{=} \AgdaKeyword{let} \AgdaKeyword{open} \AgdaModule{≡-Reasoning} \AgdaSymbol{\{}\AgdaArgument{A} \AgdaSymbol{=} \AgdaFunction{Expression} \AgdaBound{V} \AgdaSymbol{(}\AgdaInductiveConstructor{varKind} \AgdaBound{K}\AgdaSymbol{)\}} \AgdaKeyword{in} \<[83]%
\>[83]\<%
\\
\>[2]\AgdaIndent{4}{}\<[4]%
\>[4]\AgdaFunction{begin} \<[10]%
\>[10]\<%
\\
\>[4]\AgdaIndent{6}{}\<[6]%
\>[6]\AgdaFunction{apV} \AgdaSymbol{(}\AgdaFunction{idOp} \AgdaBound{V} \AgdaFunction{∘} \AgdaBound{σ}\AgdaSymbol{)} \AgdaBound{x}\<%
\\
\>[0]\AgdaIndent{4}{}\<[4]%
\>[4]\AgdaFunction{≡⟨} \AgdaFunction{apV-comp} \AgdaFunction{⟩}\<%
\\
\>[4]\AgdaIndent{6}{}\<[6]%
\>[6]\AgdaFunction{ap} \AgdaSymbol{(}\AgdaFunction{idOp} \AgdaBound{V}\AgdaSymbol{)} \AgdaSymbol{(}\AgdaFunction{apV} \AgdaBound{σ} \AgdaBound{x}\AgdaSymbol{)}\<%
\\
\>[0]\AgdaIndent{4}{}\<[4]%
\>[4]\AgdaFunction{≡⟨} \AgdaFunction{ap-idOp} \AgdaFunction{⟩}\<%
\\
\>[4]\AgdaIndent{6}{}\<[6]%
\>[6]\AgdaFunction{apV} \AgdaBound{σ} \AgdaBound{x}\<%
\\
\>[0]\AgdaIndent{4}{}\<[4]%
\>[4]\AgdaFunction{∎}\<%
\end{code}
}

\begin{code}%
\>[0]\AgdaIndent{2}{}\<[2]%
\>[2]\AgdaFunction{unitr} \AgdaSymbol{:} \AgdaSymbol{∀} \AgdaSymbol{\{}\AgdaBound{U}\AgdaSymbol{\}} \AgdaSymbol{\{}\AgdaBound{V}\AgdaSymbol{\}} \AgdaSymbol{\{}\AgdaBound{σ} \AgdaSymbol{:} \AgdaFunction{Op} \AgdaBound{U} \AgdaBound{V}\AgdaSymbol{\}} \AgdaSymbol{→} \AgdaBound{σ} \AgdaFunction{∘} \AgdaFunction{idOp} \AgdaBound{U} \AgdaFunction{∼op} \AgdaBound{σ}\<%
\end{code}

\AgdaHide{
\begin{code}%
\>[0]\AgdaIndent{2}{}\<[2]%
\>[2]\AgdaFunction{unitr} \AgdaSymbol{\{}\AgdaBound{U}\AgdaSymbol{\}} \AgdaSymbol{\{}\AgdaBound{V}\AgdaSymbol{\}} \AgdaSymbol{\{}\AgdaBound{σ}\AgdaSymbol{\}} \AgdaSymbol{\{}\AgdaBound{K}\AgdaSymbol{\}} \AgdaBound{x} \AgdaSymbol{=} \AgdaKeyword{let} \AgdaKeyword{open} \AgdaModule{≡-Reasoning} \AgdaSymbol{\{}\AgdaArgument{A} \AgdaSymbol{=} \AgdaFunction{Expression} \AgdaBound{V} \AgdaSymbol{(}\AgdaInductiveConstructor{varKind} \AgdaBound{K}\AgdaSymbol{)\}} \AgdaKeyword{in}\<%
\\
\>[2]\AgdaIndent{4}{}\<[4]%
\>[4]\AgdaFunction{begin} \<[10]%
\>[10]\<%
\\
\>[4]\AgdaIndent{6}{}\<[6]%
\>[6]\AgdaFunction{apV} \AgdaSymbol{(}\AgdaBound{σ} \AgdaFunction{∘} \AgdaFunction{idOp} \AgdaBound{U}\AgdaSymbol{)} \AgdaBound{x}\<%
\\
\>[0]\AgdaIndent{4}{}\<[4]%
\>[4]\AgdaFunction{≡⟨} \AgdaFunction{apV-comp} \AgdaFunction{⟩}\<%
\\
\>[4]\AgdaIndent{6}{}\<[6]%
\>[6]\AgdaFunction{ap} \AgdaBound{σ} \AgdaSymbol{(}\AgdaFunction{apV} \AgdaSymbol{(}\AgdaFunction{idOp} \AgdaBound{U}\AgdaSymbol{)} \AgdaBound{x}\AgdaSymbol{)}\<%
\\
\>[0]\AgdaIndent{4}{}\<[4]%
\>[4]\AgdaFunction{≡⟨} \AgdaFunction{cong} \AgdaSymbol{(}\AgdaFunction{ap} \AgdaBound{σ}\AgdaSymbol{)} \AgdaSymbol{(}\AgdaFunction{apV-idOp} \AgdaBound{x}\AgdaSymbol{)} \AgdaFunction{⟩}\<%
\\
\>[4]\AgdaIndent{6}{}\<[6]%
\>[6]\AgdaFunction{apV} \AgdaBound{σ} \AgdaBound{x}\<%
\\
\>[0]\AgdaIndent{4}{}\<[4]%
\>[4]\AgdaFunction{∎}\<%
\end{code}
}

\AgdaHide{
\begin{code}%
\>\AgdaKeyword{record} \AgdaRecord{OpFamily} \AgdaSymbol{:} \AgdaPrimitiveType{Set₂} \AgdaKeyword{where}\<%
\\
\>[0]\AgdaIndent{2}{}\<[2]%
\>[2]\AgdaKeyword{field}\<%
\\
\>[2]\AgdaIndent{4}{}\<[4]%
\>[4]\AgdaField{liftFamily} \AgdaSymbol{:} \AgdaRecord{LiftFamily}\<%
\\
\>[2]\AgdaIndent{4}{}\<[4]%
\>[4]\AgdaField{isOpFamily} \<[16]%
\>[16]\AgdaSymbol{:} \AgdaRecord{IsOpFamily} \AgdaField{liftFamily}\<%
\\
\>[0]\AgdaIndent{2}{}\<[2]%
\>[2]\AgdaKeyword{open} \AgdaModule{IsOpFamily} \AgdaField{isOpFamily} \AgdaKeyword{public}\<%
\end{code}
}


\AgdaHide{
\begin{code}%
\>\AgdaKeyword{open} \AgdaKeyword{import} \AgdaModule{Grammar.Base}\<%
\\
\>[0]\AgdaIndent{2}{}\<[2]%
\>[2]\<%
\\
\>\AgdaKeyword{module} \AgdaModule{Grammar.Substitution.Botsub} \AgdaSymbol{(}\AgdaBound{G} \AgdaSymbol{:} \AgdaRecord{Grammar}\AgdaSymbol{)} \AgdaKeyword{where}\<%
\\
\>\AgdaKeyword{open} \AgdaKeyword{import} \AgdaModule{Prelims}\<%
\\
\>\AgdaKeyword{open} \AgdaModule{Grammar} \AgdaBound{G}\<%
\\
\>\AgdaKeyword{open} \AgdaKeyword{import} \AgdaModule{Grammar.OpFamily} \AgdaBound{G}\<%
\\
\>\AgdaKeyword{open} \AgdaKeyword{import} \AgdaModule{Grammar.Replacement} \AgdaBound{G}\<%
\\
\>\AgdaKeyword{open} \AgdaKeyword{import} \AgdaModule{Grammar.Substitution.PreOpFamily} \AgdaBound{G}\<%
\\
\>\AgdaKeyword{open} \AgdaKeyword{import} \AgdaModule{Grammar.Substitution.Lifting} \AgdaBound{G}\<%
\\
\>\AgdaKeyword{open} \AgdaKeyword{import} \AgdaModule{Grammar.Substitution.LiftFamily} \AgdaBound{G}\<%
\\
\>\AgdaKeyword{open} \AgdaKeyword{import} \AgdaModule{Grammar.Substitution.OpFamily} \AgdaBound{G}\<%
\end{code}
}

\subsubsection{Substitution for an Individual Variable}

Let $E$ be an expression of kind $K$ over $V$.  Then we write $[x_0 := E]$ for the following substitution
$(V , K) \Rightarrow V$:

\AgdaHide{
\begin{code}%
\>\AgdaFunction{botsub} \AgdaSymbol{:} \AgdaSymbol{∀} \AgdaSymbol{\{}\AgdaBound{V}\AgdaSymbol{\}} \AgdaSymbol{\{}\AgdaBound{A}\AgdaSymbol{\}} \AgdaSymbol{→} \AgdaDatatype{ExpList} \AgdaBound{V} \AgdaBound{A} \AgdaSymbol{→} \AgdaFunction{Sub} \AgdaSymbol{(}\AgdaFunction{snoc-extend} \AgdaBound{V} \AgdaBound{A}\AgdaSymbol{)} \AgdaBound{V}\<%
\\
\>\AgdaFunction{botsub} \AgdaSymbol{\{}\AgdaArgument{A} \AgdaSymbol{=} \AgdaInductiveConstructor{[]}\AgdaSymbol{\}} \AgdaSymbol{\_} \AgdaSymbol{\_} \AgdaBound{x} \AgdaSymbol{=} \AgdaInductiveConstructor{var} \AgdaBound{x}\<%
\\
\>\AgdaFunction{botsub} \AgdaSymbol{\{}\AgdaArgument{A} \AgdaSymbol{=} \AgdaSymbol{\_} \AgdaInductiveConstructor{snoc} \AgdaSymbol{\_\}} \AgdaSymbol{(\_} \AgdaInductiveConstructor{snoc} \AgdaBound{E}\AgdaSymbol{)} \AgdaSymbol{\_} \AgdaInductiveConstructor{x₀} \AgdaSymbol{=} \AgdaBound{E}\<%
\\
\>\AgdaFunction{botsub} \AgdaSymbol{\{}\AgdaArgument{A} \AgdaSymbol{=} \AgdaSymbol{\_} \AgdaInductiveConstructor{snoc} \AgdaSymbol{\_\}} \AgdaSymbol{(}\AgdaBound{EE} \AgdaInductiveConstructor{snoc} \AgdaSymbol{\_)} \AgdaBound{L} \AgdaSymbol{(}\AgdaInductiveConstructor{↑} \AgdaBound{x}\AgdaSymbol{)} \AgdaSymbol{=} \AgdaFunction{botsub} \AgdaBound{EE} \AgdaBound{L} \AgdaBound{x}\<%
\end{code}
}

\begin{code}%
\>\AgdaKeyword{infix} \AgdaNumber{65} \AgdaFixityOp{x₀:=\_}\<%
\\
\>\AgdaFunction{x₀:=\_} \AgdaSymbol{:} \AgdaSymbol{∀} \AgdaSymbol{\{}\AgdaBound{V}\AgdaSymbol{\}} \AgdaSymbol{\{}\AgdaBound{K}\AgdaSymbol{\}} \AgdaSymbol{→} \AgdaFunction{Expression} \AgdaBound{V} \AgdaSymbol{(}\AgdaInductiveConstructor{varKind} \AgdaBound{K}\AgdaSymbol{)} \AgdaSymbol{→} \AgdaFunction{Sub} \AgdaSymbol{(}\AgdaBound{V} \AgdaInductiveConstructor{,} \AgdaBound{K}\AgdaSymbol{)} \AgdaBound{V}\<%
\\
\>\AgdaFunction{x₀:=} \AgdaBound{E} \AgdaSymbol{=} \AgdaFunction{botsub} \AgdaSymbol{(}\AgdaInductiveConstructor{[]} \AgdaInductiveConstructor{snoc} \AgdaBound{E}\AgdaSymbol{)}\<%
\end{code}

\begin{lemma}$ $
\begin{enumerate}
\item
$ \rho \bullet_1 [x_0 := E] \sim [x_0 := E \langle \rho \rangle] \bullet_2 (\rho , K) $
\item
$ \sigma \bullet [x_0 := E] \sim [x_0 := E[\sigma]] \bullet (\sigma , K) $
\item
$ E [ \uparrow ] [ x_0 := F ] \equiv E$
\end{enumerate}
\end{lemma}

\begin{code}%
\>\AgdaKeyword{open} \AgdaModule{LiftFamily}\<%
\\
%
\\
\>\AgdaFunction{botsub-up'} \AgdaSymbol{:} \AgdaSymbol{∀} \AgdaSymbol{\{}\AgdaBound{F}\AgdaSymbol{\}} \AgdaSymbol{\{}\AgdaBound{V}\AgdaSymbol{\}} \AgdaSymbol{\{}\AgdaBound{K}\AgdaSymbol{\}} \AgdaSymbol{\{}\AgdaBound{E} \AgdaSymbol{:} \AgdaFunction{Expression} \AgdaBound{V} \AgdaSymbol{(}\AgdaInductiveConstructor{varKind} \AgdaBound{K}\AgdaSymbol{)\}} \AgdaSymbol{(}\AgdaBound{circ} \AgdaSymbol{:} \AgdaRecord{Composition} \AgdaFunction{SubLF} \AgdaBound{F} \AgdaFunction{SubLF}\AgdaSymbol{)} \AgdaSymbol{→}\<%
\\
\>[0]\AgdaIndent{2}{}\<[2]%
\>[2]\AgdaField{Composition.circ} \AgdaBound{circ} \AgdaSymbol{(}\AgdaFunction{x₀:=} \AgdaBound{E}\AgdaSymbol{)} \AgdaSymbol{(}\AgdaFunction{up} \AgdaBound{F}\AgdaSymbol{)} \AgdaFunction{∼} \AgdaFunction{idSub} \AgdaBound{V}\<%
\\
\>\AgdaFunction{botsub-up'} \AgdaSymbol{\{}\AgdaBound{F}\AgdaSymbol{\}} \AgdaSymbol{\{}\AgdaBound{V}\AgdaSymbol{\}} \AgdaSymbol{\{}\AgdaBound{K}\AgdaSymbol{\}} \AgdaSymbol{\{}\AgdaBound{E}\AgdaSymbol{\}} \AgdaBound{circ} \AgdaBound{x} \AgdaSymbol{=} \AgdaKeyword{let} \AgdaKeyword{open} \AgdaModule{≡-Reasoning} \AgdaKeyword{in} \<[60]%
\>[60]\<%
\\
\>[0]\AgdaIndent{2}{}\<[2]%
\>[2]\AgdaFunction{begin}\<%
\\
\>[2]\AgdaIndent{4}{}\<[4]%
\>[4]\AgdaSymbol{(}\AgdaField{Composition.circ} \AgdaBound{circ} \AgdaSymbol{(}\AgdaFunction{x₀:=} \AgdaBound{E}\AgdaSymbol{)} \AgdaSymbol{(}\AgdaFunction{up} \AgdaBound{F}\AgdaSymbol{))} \AgdaSymbol{\_} \AgdaBound{x}\<%
\\
\>[0]\AgdaIndent{2}{}\<[2]%
\>[2]\AgdaFunction{≡⟨} \AgdaField{Composition.apV-circ} \AgdaBound{circ} \AgdaFunction{⟩}\<%
\\
\>[2]\AgdaIndent{4}{}\<[4]%
\>[4]\AgdaFunction{apV} \AgdaBound{F} \AgdaSymbol{(}\AgdaFunction{up} \AgdaBound{F}\AgdaSymbol{)} \AgdaBound{x} \AgdaFunction{⟦} \AgdaFunction{x₀:=} \AgdaBound{E} \AgdaFunction{⟧}\<%
\\
\>[0]\AgdaIndent{2}{}\<[2]%
\>[2]\AgdaFunction{≡⟨} \AgdaFunction{sub-congl} \AgdaSymbol{(}\AgdaFunction{apV-up} \AgdaBound{F}\AgdaSymbol{)} \AgdaFunction{⟩}\<%
\\
\>[2]\AgdaIndent{4}{}\<[4]%
\>[4]\AgdaInductiveConstructor{var} \AgdaBound{x}\<%
\\
\>[0]\AgdaIndent{2}{}\<[2]%
\>[2]\AgdaFunction{∎}\<%
\\
%
\\
\>\AgdaFunction{botsub-up} \AgdaSymbol{:} \AgdaSymbol{∀} \AgdaSymbol{\{}\AgdaBound{F}\AgdaSymbol{\}} \AgdaSymbol{\{}\AgdaBound{V}\AgdaSymbol{\}} \AgdaSymbol{\{}\AgdaBound{K}\AgdaSymbol{\}} \AgdaSymbol{\{}\AgdaBound{C}\AgdaSymbol{\}} \AgdaSymbol{\{}\AgdaBound{L}\AgdaSymbol{\}} \AgdaSymbol{\{}\AgdaBound{E} \AgdaSymbol{:} \AgdaFunction{Expression} \AgdaBound{V} \AgdaSymbol{(}\AgdaInductiveConstructor{varKind} \AgdaBound{K}\AgdaSymbol{)\}} \AgdaSymbol{(}\AgdaBound{circ} \AgdaSymbol{:} \AgdaRecord{Composition} \AgdaFunction{SubLF} \AgdaBound{F} \AgdaFunction{SubLF}\AgdaSymbol{)} \AgdaSymbol{\{}\AgdaBound{E'} \AgdaSymbol{:} \AgdaDatatype{Subexpression} \AgdaBound{V} \AgdaBound{C} \AgdaBound{L}\AgdaSymbol{\}} \AgdaSymbol{→}\<%
\\
\>[0]\AgdaIndent{2}{}\<[2]%
\>[2]\AgdaFunction{ap} \AgdaBound{F} \AgdaSymbol{(}\AgdaFunction{up} \AgdaBound{F}\AgdaSymbol{)} \AgdaBound{E'} \AgdaFunction{⟦} \AgdaFunction{x₀:=} \AgdaBound{E} \AgdaFunction{⟧} \AgdaDatatype{≡} \AgdaBound{E'}\<%
\\
\>\AgdaFunction{botsub-up} \AgdaSymbol{\{}\AgdaBound{F}\AgdaSymbol{\}} \AgdaSymbol{\{}\AgdaBound{V}\AgdaSymbol{\}} \AgdaSymbol{\{}\AgdaBound{K}\AgdaSymbol{\}} \AgdaSymbol{\{}\AgdaBound{C}\AgdaSymbol{\}} \AgdaSymbol{\{}\AgdaBound{L}\AgdaSymbol{\}} \AgdaSymbol{\{}\AgdaBound{E}\AgdaSymbol{\}} \AgdaBound{circ} \AgdaSymbol{\{}\AgdaBound{E'}\AgdaSymbol{\}} \AgdaSymbol{=} \AgdaKeyword{let} \AgdaKeyword{open} \AgdaModule{≡-Reasoning} \AgdaKeyword{in}\<%
\\
\>[0]\AgdaIndent{2}{}\<[2]%
\>[2]\AgdaFunction{begin}\<%
\\
\>[2]\AgdaIndent{4}{}\<[4]%
\>[4]\AgdaFunction{ap} \AgdaBound{F} \AgdaSymbol{(}\AgdaFunction{up} \AgdaBound{F}\AgdaSymbol{)} \AgdaBound{E'} \AgdaFunction{⟦} \AgdaFunction{x₀:=} \AgdaBound{E} \AgdaFunction{⟧}\<%
\\
\>[0]\AgdaIndent{2}{}\<[2]%
\>[2]\AgdaFunction{≡⟨⟨} \AgdaFunction{Composition.ap-circ} \AgdaBound{circ} \AgdaBound{E'} \AgdaFunction{⟩⟩}\<%
\\
\>[2]\AgdaIndent{4}{}\<[4]%
\>[4]\AgdaBound{E'} \AgdaFunction{⟦} \AgdaField{Composition.circ} \AgdaBound{circ} \AgdaSymbol{(}\AgdaFunction{x₀:=} \AgdaBound{E}\AgdaSymbol{)} \AgdaSymbol{(}\AgdaFunction{up} \AgdaBound{F}\AgdaSymbol{)} \AgdaFunction{⟧}\<%
\\
\>[0]\AgdaIndent{2}{}\<[2]%
\>[2]\AgdaFunction{≡⟨} \AgdaFunction{sub-congr} \AgdaBound{E'} \AgdaSymbol{(}\AgdaFunction{botsub-up'} \AgdaBound{circ}\AgdaSymbol{)} \AgdaFunction{⟩}\<%
\\
\>[2]\AgdaIndent{4}{}\<[4]%
\>[4]\AgdaBound{E'} \AgdaFunction{⟦} \AgdaFunction{idSub} \AgdaBound{V} \AgdaFunction{⟧}\<%
\\
\>[0]\AgdaIndent{2}{}\<[2]%
\>[2]\AgdaFunction{≡⟨} \AgdaFunction{sub-idOp} \AgdaFunction{⟩}\<%
\\
\>[2]\AgdaIndent{4}{}\<[4]%
\>[4]\AgdaBound{E'}\<%
\\
\>[0]\AgdaIndent{2}{}\<[2]%
\>[2]\AgdaFunction{∎}\<%
\\
%
\\
\>\AgdaFunction{circ-botsub'} \AgdaSymbol{:} \AgdaSymbol{∀} \AgdaSymbol{\{}\AgdaBound{F}\AgdaSymbol{\}} \AgdaSymbol{\{}\AgdaBound{U}\AgdaSymbol{\}} \AgdaSymbol{\{}\AgdaBound{V}\AgdaSymbol{\}} \AgdaSymbol{\{}\AgdaBound{K}\AgdaSymbol{\}} \AgdaSymbol{\{}\AgdaBound{E} \AgdaSymbol{:} \AgdaFunction{Expression} \AgdaBound{U} \AgdaSymbol{(}\AgdaInductiveConstructor{varKind} \AgdaBound{K}\AgdaSymbol{)\}} \<[64]%
\>[64]\<%
\\
\>[0]\AgdaIndent{2}{}\<[2]%
\>[2]\AgdaSymbol{(}\AgdaBound{circ₁} \AgdaSymbol{:} \AgdaRecord{Composition} \AgdaBound{F} \AgdaFunction{SubLF} \AgdaFunction{SubLF}\AgdaSymbol{)} \<[38]%
\>[38]\<%
\\
\>[0]\AgdaIndent{2}{}\<[2]%
\>[2]\AgdaSymbol{(}\AgdaBound{circ₂} \AgdaSymbol{:} \AgdaRecord{Composition} \AgdaFunction{SubLF} \AgdaBound{F} \AgdaFunction{SubLF}\AgdaSymbol{)}\<%
\\
\>[0]\AgdaIndent{2}{}\<[2]%
\>[2]\AgdaSymbol{\{}\AgdaBound{σ} \AgdaSymbol{:} \AgdaFunction{Op} \AgdaBound{F} \AgdaBound{U} \AgdaBound{V}\AgdaSymbol{\}} \AgdaSymbol{→}\<%
\\
\>[0]\AgdaIndent{2}{}\<[2]%
\>[2]\AgdaField{Composition.circ} \AgdaBound{circ₁} \AgdaBound{σ} \AgdaSymbol{(}\AgdaFunction{x₀:=} \AgdaBound{E}\AgdaSymbol{)} \AgdaFunction{∼} \AgdaField{Composition.circ} \AgdaBound{circ₂} \AgdaSymbol{(}\AgdaFunction{x₀:=} \AgdaSymbol{(}\AgdaFunction{ap} \AgdaBound{F} \AgdaBound{σ} \AgdaBound{E}\AgdaSymbol{))} \AgdaSymbol{(}\AgdaFunction{liftOp} \AgdaBound{F} \AgdaBound{K} \AgdaBound{σ}\AgdaSymbol{)}\<%
\\
\>\AgdaFunction{circ-botsub'} \AgdaSymbol{\{}\AgdaBound{F}\AgdaSymbol{\}} \AgdaSymbol{\{}\AgdaBound{U}\AgdaSymbol{\}} \AgdaSymbol{\{}\AgdaBound{V}\AgdaSymbol{\}} \AgdaSymbol{\{}\AgdaBound{K}\AgdaSymbol{\}} \AgdaSymbol{\{}\AgdaBound{E}\AgdaSymbol{\}} \AgdaBound{circ₁} \AgdaBound{circ₂} \AgdaSymbol{\{}\AgdaBound{σ}\AgdaSymbol{\}} \AgdaInductiveConstructor{x₀} \AgdaSymbol{=} \AgdaKeyword{let} \AgdaKeyword{open} \AgdaModule{≡-Reasoning} \AgdaKeyword{in} \<[78]%
\>[78]\<%
\\
\>[0]\AgdaIndent{2}{}\<[2]%
\>[2]\AgdaFunction{begin}\<%
\\
\>[2]\AgdaIndent{4}{}\<[4]%
\>[4]\AgdaSymbol{(}\AgdaField{Composition.circ} \AgdaBound{circ₁} \AgdaBound{σ} \AgdaSymbol{(}\AgdaFunction{x₀:=} \AgdaBound{E}\AgdaSymbol{))} \AgdaSymbol{\_} \AgdaInductiveConstructor{x₀}\<%
\\
\>[0]\AgdaIndent{2}{}\<[2]%
\>[2]\AgdaFunction{≡⟨} \AgdaField{Composition.apV-circ} \AgdaBound{circ₁} \AgdaFunction{⟩}\<%
\\
\>[2]\AgdaIndent{4}{}\<[4]%
\>[4]\AgdaFunction{ap} \AgdaBound{F} \AgdaBound{σ} \AgdaBound{E}\<%
\\
\>[0]\AgdaIndent{2}{}\<[2]%
\>[2]\AgdaFunction{≡⟨⟨} \AgdaFunction{sub-congl} \AgdaSymbol{(}\AgdaFunction{liftOp-x₀} \AgdaBound{F}\AgdaSymbol{)} \AgdaFunction{⟩⟩}\<%
\\
\>[2]\AgdaIndent{4}{}\<[4]%
\>[4]\AgdaSymbol{(}\AgdaFunction{apV} \AgdaBound{F} \AgdaSymbol{(}\AgdaFunction{liftOp} \AgdaBound{F} \AgdaBound{K} \AgdaBound{σ}\AgdaSymbol{)} \AgdaInductiveConstructor{x₀}\AgdaSymbol{)} \AgdaFunction{⟦} \AgdaFunction{x₀:=} \AgdaSymbol{(}\AgdaFunction{ap} \AgdaBound{F} \AgdaBound{σ} \AgdaBound{E}\AgdaSymbol{)} \AgdaFunction{⟧}\<%
\\
\>[0]\AgdaIndent{2}{}\<[2]%
\>[2]\AgdaFunction{≡⟨⟨} \AgdaField{Composition.apV-circ} \AgdaBound{circ₂} \AgdaFunction{⟩⟩}\<%
\\
\>[2]\AgdaIndent{4}{}\<[4]%
\>[4]\AgdaSymbol{(}\AgdaField{Composition.circ} \AgdaBound{circ₂} \AgdaSymbol{(}\AgdaFunction{x₀:=} \AgdaSymbol{(}\AgdaFunction{ap} \AgdaBound{F} \AgdaBound{σ} \AgdaBound{E}\AgdaSymbol{))} \AgdaSymbol{(}\AgdaFunction{liftOp} \AgdaBound{F} \AgdaBound{K} \AgdaBound{σ}\AgdaSymbol{))} \AgdaSymbol{\_} \AgdaInductiveConstructor{x₀}\<%
\\
\>[0]\AgdaIndent{2}{}\<[2]%
\>[2]\AgdaFunction{∎}\<%
\\
\>\AgdaFunction{circ-botsub'} \AgdaSymbol{\{}\AgdaBound{F}\AgdaSymbol{\}} \AgdaSymbol{\{}\AgdaBound{U}\AgdaSymbol{\}} \AgdaSymbol{\{}\AgdaBound{V}\AgdaSymbol{\}} \AgdaSymbol{\{}\AgdaBound{K}\AgdaSymbol{\}} \AgdaSymbol{\{}\AgdaBound{E}\AgdaSymbol{\}} \AgdaBound{circ₁} \AgdaBound{circ₂} \AgdaSymbol{\{}\AgdaBound{σ}\AgdaSymbol{\}} \AgdaSymbol{(}\AgdaInductiveConstructor{↑} \AgdaBound{x}\AgdaSymbol{)} \AgdaSymbol{=} \AgdaKeyword{let} \AgdaKeyword{open} \AgdaModule{≡-Reasoning} \AgdaKeyword{in} \<[81]%
\>[81]\<%
\\
\>[0]\AgdaIndent{2}{}\<[2]%
\>[2]\AgdaFunction{begin}\<%
\\
\>[2]\AgdaIndent{4}{}\<[4]%
\>[4]\AgdaSymbol{(}\AgdaField{Composition.circ} \AgdaBound{circ₁} \AgdaBound{σ} \AgdaSymbol{(}\AgdaFunction{x₀:=} \AgdaBound{E}\AgdaSymbol{))} \AgdaSymbol{\_} \AgdaSymbol{(}\AgdaInductiveConstructor{↑} \AgdaBound{x}\AgdaSymbol{)}\<%
\\
\>[0]\AgdaIndent{2}{}\<[2]%
\>[2]\AgdaFunction{≡⟨} \AgdaField{Composition.apV-circ} \AgdaBound{circ₁} \AgdaFunction{⟩}\<%
\\
\>[2]\AgdaIndent{4}{}\<[4]%
\>[4]\AgdaFunction{apV} \AgdaBound{F} \AgdaBound{σ} \AgdaBound{x}\<%
\\
\>[0]\AgdaIndent{2}{}\<[2]%
\>[2]\AgdaFunction{≡⟨⟨} \AgdaFunction{sub-idOp} \AgdaFunction{⟩⟩}\<%
\\
\>[2]\AgdaIndent{4}{}\<[4]%
\>[4]\AgdaFunction{apV} \AgdaBound{F} \AgdaBound{σ} \AgdaBound{x} \AgdaFunction{⟦} \AgdaFunction{idSub} \AgdaBound{V} \AgdaFunction{⟧}\<%
\\
\>[0]\AgdaIndent{2}{}\<[2]%
\>[2]\AgdaFunction{≡⟨⟨} \AgdaFunction{sub-congr} \AgdaSymbol{(}\AgdaFunction{apV} \AgdaBound{F} \AgdaBound{σ} \AgdaBound{x}\AgdaSymbol{)} \AgdaSymbol{(}\AgdaFunction{botsub-up'} \AgdaBound{circ₂}\AgdaSymbol{)} \AgdaFunction{⟩⟩}\<%
\\
\>[2]\AgdaIndent{4}{}\<[4]%
\>[4]\AgdaFunction{apV} \AgdaBound{F} \AgdaBound{σ} \AgdaBound{x} \AgdaFunction{⟦} \AgdaField{Composition.circ} \AgdaBound{circ₂} \AgdaSymbol{(}\AgdaFunction{x₀:=} \AgdaSymbol{(}\AgdaFunction{ap} \AgdaBound{F} \AgdaBound{σ} \AgdaBound{E}\AgdaSymbol{))} \AgdaSymbol{(}\AgdaFunction{up} \AgdaBound{F}\AgdaSymbol{)} \AgdaFunction{⟧}\<%
\\
\>[0]\AgdaIndent{2}{}\<[2]%
\>[2]\AgdaFunction{≡⟨} \AgdaFunction{Composition.ap-circ} \AgdaBound{circ₂} \AgdaSymbol{(}\AgdaFunction{apV} \AgdaBound{F} \AgdaBound{σ} \AgdaBound{x}\AgdaSymbol{)} \AgdaFunction{⟩}\<%
\\
\>[2]\AgdaIndent{4}{}\<[4]%
\>[4]\AgdaFunction{ap} \AgdaBound{F} \AgdaSymbol{(}\AgdaFunction{up} \AgdaBound{F}\AgdaSymbol{)} \AgdaSymbol{(}\AgdaFunction{apV} \AgdaBound{F} \AgdaBound{σ} \AgdaBound{x}\AgdaSymbol{)} \AgdaFunction{⟦} \AgdaFunction{x₀:=} \AgdaSymbol{(}\AgdaFunction{ap} \AgdaBound{F} \AgdaBound{σ} \AgdaBound{E}\AgdaSymbol{)} \AgdaFunction{⟧}\<%
\\
\>[0]\AgdaIndent{2}{}\<[2]%
\>[2]\AgdaFunction{≡⟨⟨} \AgdaFunction{sub-congl} \AgdaSymbol{(}\AgdaFunction{liftOp-↑} \AgdaBound{F} \AgdaBound{x}\AgdaSymbol{)} \AgdaFunction{⟩⟩}\<%
\\
\>[2]\AgdaIndent{4}{}\<[4]%
\>[4]\AgdaSymbol{(}\AgdaFunction{apV} \AgdaBound{F} \AgdaSymbol{(}\AgdaFunction{liftOp} \AgdaBound{F} \AgdaBound{K} \AgdaBound{σ}\AgdaSymbol{)} \AgdaSymbol{(}\AgdaInductiveConstructor{↑} \AgdaBound{x}\AgdaSymbol{))} \AgdaFunction{⟦} \AgdaFunction{x₀:=} \AgdaSymbol{(}\AgdaFunction{ap} \AgdaBound{F} \AgdaBound{σ} \AgdaBound{E}\AgdaSymbol{)} \AgdaFunction{⟧}\<%
\\
\>[0]\AgdaIndent{2}{}\<[2]%
\>[2]\AgdaFunction{≡⟨⟨} \AgdaField{Composition.apV-circ} \AgdaBound{circ₂} \AgdaFunction{⟩⟩}\<%
\\
\>[2]\AgdaIndent{4}{}\<[4]%
\>[4]\AgdaSymbol{(}\AgdaField{Composition.circ} \AgdaBound{circ₂} \AgdaSymbol{(}\AgdaFunction{x₀:=} \AgdaSymbol{(}\AgdaFunction{ap} \AgdaBound{F} \AgdaBound{σ} \AgdaBound{E}\AgdaSymbol{))} \AgdaSymbol{(}\AgdaFunction{liftOp} \AgdaBound{F} \AgdaBound{K} \AgdaBound{σ}\AgdaSymbol{))} \AgdaSymbol{\_} \AgdaSymbol{(}\AgdaInductiveConstructor{↑} \AgdaBound{x}\AgdaSymbol{)}\<%
\\
\>[0]\AgdaIndent{2}{}\<[2]%
\>[2]\AgdaFunction{∎}\<%
\\
%
\\
\>\AgdaFunction{circ-botsub} \AgdaSymbol{:} \AgdaSymbol{∀} \AgdaSymbol{\{}\AgdaBound{F}\AgdaSymbol{\}} \AgdaSymbol{\{}\AgdaBound{U}\AgdaSymbol{\}} \AgdaSymbol{\{}\AgdaBound{V}\AgdaSymbol{\}} \AgdaSymbol{\{}\AgdaBound{K}\AgdaSymbol{\}} \AgdaSymbol{\{}\AgdaBound{C}\AgdaSymbol{\}} \AgdaSymbol{\{}\AgdaBound{L}\AgdaSymbol{\}} \<[40]%
\>[40]\<%
\\
\>[0]\AgdaIndent{2}{}\<[2]%
\>[2]\AgdaSymbol{\{}\AgdaBound{E} \AgdaSymbol{:} \AgdaFunction{Expression} \AgdaBound{U} \AgdaSymbol{(}\AgdaInductiveConstructor{varKind} \AgdaBound{K}\AgdaSymbol{)\}} \AgdaSymbol{\{}\AgdaBound{E'} \AgdaSymbol{:} \AgdaDatatype{Subexpression} \AgdaSymbol{(}\AgdaBound{U} \AgdaInductiveConstructor{,} \AgdaBound{K}\AgdaSymbol{)} \AgdaBound{C} \AgdaBound{L}\AgdaSymbol{\}} \AgdaSymbol{\{}\AgdaBound{σ} \AgdaSymbol{:} \AgdaFunction{Op} \AgdaBound{F} \AgdaBound{U} \AgdaBound{V}\AgdaSymbol{\}} \AgdaSymbol{→}\<%
\\
\>[0]\AgdaIndent{2}{}\<[2]%
\>[2]\AgdaRecord{Composition} \AgdaBound{F} \AgdaFunction{SubLF} \AgdaFunction{SubLF} \AgdaSymbol{→}\<%
\\
\>[0]\AgdaIndent{2}{}\<[2]%
\>[2]\AgdaRecord{Composition} \AgdaFunction{SubLF} \AgdaBound{F} \AgdaFunction{SubLF} \AgdaSymbol{→}\<%
\\
\>[0]\AgdaIndent{2}{}\<[2]%
\>[2]\AgdaFunction{ap} \AgdaBound{F} \AgdaBound{σ} \AgdaSymbol{(}\AgdaBound{E'} \AgdaFunction{⟦} \AgdaFunction{x₀:=} \AgdaBound{E} \AgdaFunction{⟧}\AgdaSymbol{)} \AgdaDatatype{≡} \AgdaSymbol{(}\AgdaFunction{ap} \AgdaBound{F} \AgdaSymbol{(}\AgdaFunction{liftOp} \AgdaBound{F} \AgdaBound{K} \AgdaBound{σ}\AgdaSymbol{)} \AgdaBound{E'}\AgdaSymbol{)} \AgdaFunction{⟦} \AgdaFunction{x₀:=} \AgdaSymbol{(}\AgdaFunction{ap} \AgdaBound{F} \AgdaBound{σ} \AgdaBound{E}\AgdaSymbol{)} \AgdaFunction{⟧}\<%
\\
\>\AgdaFunction{circ-botsub} \AgdaSymbol{\{}\AgdaArgument{E'} \AgdaSymbol{=} \AgdaBound{E'}\AgdaSymbol{\}} \AgdaBound{circ₁} \AgdaBound{circ₂} \AgdaSymbol{=} \AgdaFunction{ap-circ-sim} \AgdaBound{circ₁} \AgdaBound{circ₂} \AgdaSymbol{(}\AgdaFunction{circ-botsub'} \AgdaBound{circ₁} \AgdaBound{circ₂}\AgdaSymbol{)} \AgdaBound{E'}\<%
\\
%
\\
\>\AgdaFunction{compRS-botsub} \AgdaSymbol{:} \AgdaSymbol{∀} \AgdaSymbol{\{}\AgdaBound{U}\AgdaSymbol{\}} \AgdaSymbol{\{}\AgdaBound{V}\AgdaSymbol{\}} \AgdaSymbol{\{}\AgdaBound{C}\AgdaSymbol{\}} \AgdaSymbol{\{}\AgdaBound{K}\AgdaSymbol{\}} \AgdaSymbol{\{}\AgdaBound{L}\AgdaSymbol{\}} \AgdaSymbol{(}\AgdaBound{E} \AgdaSymbol{:} \AgdaDatatype{Subexpression} \AgdaSymbol{(}\AgdaBound{U} \AgdaInductiveConstructor{,} \AgdaBound{K}\AgdaSymbol{)} \AgdaBound{C} \AgdaBound{L}\AgdaSymbol{)} \AgdaSymbol{\{}\AgdaBound{F} \AgdaSymbol{:} \AgdaFunction{Expression} \AgdaBound{U} \AgdaSymbol{(}\AgdaInductiveConstructor{varKind} \AgdaBound{K}\AgdaSymbol{)\}} \AgdaSymbol{\{}\AgdaBound{ρ} \AgdaSymbol{:} \AgdaFunction{Rep} \AgdaBound{U} \AgdaBound{V}\AgdaSymbol{\}} \AgdaSymbol{→}\<%
\\
\>[0]\AgdaIndent{2}{}\<[2]%
\>[2]\AgdaBound{E} \AgdaFunction{⟦} \AgdaFunction{x₀:=} \AgdaBound{F} \AgdaFunction{⟧} \AgdaFunction{〈} \AgdaBound{ρ} \AgdaFunction{〉} \AgdaDatatype{≡} \AgdaBound{E} \AgdaFunction{〈} \AgdaFunction{rep↑} \AgdaBound{K} \AgdaBound{ρ} \AgdaFunction{〉} \AgdaFunction{⟦} \AgdaFunction{x₀:=} \AgdaSymbol{(}\AgdaBound{F} \AgdaFunction{〈} \AgdaBound{ρ} \AgdaFunction{〉}\AgdaSymbol{)} \AgdaFunction{⟧}\<%
\\
\>\AgdaComment{--TODO Common pattern with rep↑-botsub₃}\<%
\end{code}

\AgdaHide{
\begin{code}%
\>\AgdaFunction{compRS-botsub} \AgdaBound{E} \AgdaSymbol{=} \AgdaFunction{circ-botsub} \AgdaSymbol{\{}\AgdaArgument{E'} \AgdaSymbol{=} \AgdaBound{E}\AgdaSymbol{\}} \AgdaFunction{COMPRS} \AgdaFunction{COMPSR}\<%
\end{code}
}

\begin{code}%
\>\AgdaFunction{comp-botsub} \AgdaSymbol{:} \AgdaSymbol{∀} \AgdaSymbol{\{}\AgdaBound{U}\AgdaSymbol{\}} \AgdaSymbol{\{}\AgdaBound{V}\AgdaSymbol{\}} \AgdaSymbol{\{}\AgdaBound{C}\AgdaSymbol{\}} \AgdaSymbol{\{}\AgdaBound{K}\AgdaSymbol{\}} \AgdaSymbol{\{}\AgdaBound{L}\AgdaSymbol{\}} \<[36]%
\>[36]\<%
\\
\>[0]\AgdaIndent{2}{}\<[2]%
\>[2]\AgdaSymbol{\{}\AgdaBound{E} \AgdaSymbol{:} \AgdaFunction{Expression} \AgdaBound{U} \AgdaSymbol{(}\AgdaInductiveConstructor{varKind} \AgdaBound{K}\AgdaSymbol{)\}} \AgdaSymbol{\{}\AgdaBound{σ} \AgdaSymbol{:} \AgdaFunction{Sub} \AgdaBound{U} \AgdaBound{V}\AgdaSymbol{\}} \AgdaSymbol{(}\AgdaBound{F} \AgdaSymbol{:} \AgdaDatatype{Subexpression} \AgdaSymbol{(}\AgdaBound{U} \AgdaInductiveConstructor{,} \AgdaBound{K}\AgdaSymbol{)} \AgdaBound{C} \AgdaBound{L}\AgdaSymbol{)} \AgdaSymbol{→}\<%
\\
\>[2]\AgdaIndent{3}{}\<[3]%
\>[3]\AgdaBound{F} \AgdaFunction{⟦} \AgdaFunction{x₀:=} \AgdaBound{E} \AgdaFunction{⟧} \AgdaFunction{⟦} \AgdaBound{σ} \AgdaFunction{⟧} \AgdaDatatype{≡} \AgdaBound{F} \AgdaFunction{⟦} \AgdaFunction{sub↑} \AgdaBound{K} \AgdaBound{σ} \AgdaFunction{⟧} \AgdaFunction{⟦} \AgdaFunction{x₀:=} \AgdaSymbol{(}\AgdaBound{E} \AgdaFunction{⟦} \AgdaBound{σ} \AgdaFunction{⟧}\AgdaSymbol{)} \AgdaFunction{⟧}\<%
\end{code}

\AgdaHide{
\begin{code}%
\>\AgdaFunction{comp-botsub} \AgdaBound{F} \AgdaSymbol{=} \AgdaKeyword{let} \AgdaBound{COMP} \AgdaSymbol{=} \AgdaFunction{OpFamily.COMP} \AgdaFunction{SUB} \AgdaKeyword{in} \AgdaFunction{circ-botsub} \AgdaSymbol{\{}\AgdaArgument{E'} \AgdaSymbol{=} \AgdaBound{F}\AgdaSymbol{\}} \AgdaBound{COMP} \AgdaBound{COMP}\<%
\end{code}
}

\begin{code}%
\>\AgdaFunction{botsub-upRep} \AgdaSymbol{:} \AgdaSymbol{∀} \AgdaSymbol{\{}\AgdaBound{U}\AgdaSymbol{\}} \AgdaSymbol{\{}\AgdaBound{C}\AgdaSymbol{\}} \AgdaSymbol{\{}\AgdaBound{K}\AgdaSymbol{\}} \AgdaSymbol{\{}\AgdaBound{L}\AgdaSymbol{\}}\<%
\\
\>[0]\AgdaIndent{2}{}\<[2]%
\>[2]\AgdaSymbol{(}\AgdaBound{E} \AgdaSymbol{:} \AgdaDatatype{Subexpression} \AgdaBound{U} \AgdaBound{C} \AgdaBound{K}\AgdaSymbol{)} \AgdaSymbol{\{}\AgdaBound{F} \AgdaSymbol{:} \AgdaFunction{Expression} \AgdaBound{U} \AgdaSymbol{(}\AgdaInductiveConstructor{varKind} \AgdaBound{L}\AgdaSymbol{)\}} \AgdaSymbol{→} \<[61]%
\>[61]\<%
\\
\>[0]\AgdaIndent{2}{}\<[2]%
\>[2]\AgdaBound{E} \AgdaFunction{〈} \AgdaFunction{upRep} \AgdaFunction{〉} \AgdaFunction{⟦} \AgdaFunction{x₀:=} \AgdaBound{F} \AgdaFunction{⟧} \AgdaDatatype{≡} \AgdaBound{E}\<%
\end{code}

\AgdaHide{
\begin{code}%
\>\AgdaFunction{botsub-upRep} \AgdaSymbol{\_} \AgdaSymbol{=} \AgdaFunction{botsub-up} \AgdaFunction{COMPSR}\<%
\\
%
\\
\>\AgdaFunction{botsub-botsub'} \AgdaSymbol{:} \AgdaSymbol{∀} \AgdaSymbol{\{}\AgdaBound{V}\AgdaSymbol{\}} \AgdaSymbol{\{}\AgdaBound{K}\AgdaSymbol{\}} \AgdaSymbol{\{}\AgdaBound{L}\AgdaSymbol{\}} \AgdaSymbol{(}\AgdaBound{N} \AgdaSymbol{:} \AgdaFunction{Expression} \AgdaBound{V} \AgdaSymbol{(}\AgdaInductiveConstructor{varKind} \AgdaBound{K}\AgdaSymbol{))} \AgdaSymbol{(}\AgdaBound{N'} \AgdaSymbol{:} \AgdaFunction{Expression} \AgdaBound{V} \AgdaSymbol{(}\AgdaInductiveConstructor{varKind} \AgdaBound{L}\AgdaSymbol{))} \AgdaSymbol{→} \AgdaFunction{x₀:=} \AgdaBound{N'} \AgdaFunction{•} \AgdaFunction{sub↑} \AgdaBound{L} \AgdaSymbol{(}\AgdaFunction{x₀:=} \AgdaBound{N}\AgdaSymbol{)} \AgdaFunction{∼} \AgdaFunction{x₀:=} \AgdaBound{N} \AgdaFunction{•} \AgdaFunction{x₀:=} \AgdaSymbol{(}\AgdaBound{N'} \AgdaFunction{⇑}\AgdaSymbol{)}\<%
\\
\>\AgdaFunction{botsub-botsub'} \AgdaBound{N} \AgdaBound{N'} \AgdaInductiveConstructor{x₀} \AgdaSymbol{=} \AgdaFunction{sym} \AgdaSymbol{(}\AgdaFunction{botsub-upRep} \AgdaBound{N'}\AgdaSymbol{)}\<%
\\
\>\AgdaFunction{botsub-botsub'} \AgdaBound{N} \AgdaBound{N'} \AgdaSymbol{(}\AgdaInductiveConstructor{↑} \AgdaInductiveConstructor{x₀}\AgdaSymbol{)} \AgdaSymbol{=} \AgdaFunction{botsub-upRep} \AgdaBound{N}\<%
\\
\>\AgdaFunction{botsub-botsub'} \AgdaBound{N} \AgdaBound{N'} \AgdaSymbol{(}\AgdaInductiveConstructor{↑} \AgdaSymbol{(}\AgdaInductiveConstructor{↑} \AgdaBound{x}\AgdaSymbol{))} \AgdaSymbol{=} \AgdaInductiveConstructor{refl}\<%
\\
%
\\
\>\AgdaFunction{botsub-botsub} \AgdaSymbol{:} \AgdaSymbol{∀} \AgdaSymbol{\{}\AgdaBound{V}\AgdaSymbol{\}} \AgdaSymbol{\{}\AgdaBound{K}\AgdaSymbol{\}} \AgdaSymbol{\{}\AgdaBound{L}\AgdaSymbol{\}} \AgdaSymbol{\{}\AgdaBound{M}\AgdaSymbol{\}} \AgdaSymbol{(}\AgdaBound{E} \AgdaSymbol{:} \AgdaFunction{Expression} \AgdaSymbol{(}\AgdaBound{V} \AgdaInductiveConstructor{,} \AgdaBound{K} \AgdaInductiveConstructor{,} \AgdaBound{L}\AgdaSymbol{)} \AgdaBound{M}\AgdaSymbol{)} \AgdaBound{F} \AgdaBound{G} \AgdaSymbol{→} \AgdaBound{E} \AgdaFunction{⟦} \AgdaFunction{sub↑} \AgdaBound{L} \AgdaSymbol{(}\AgdaFunction{x₀:=} \AgdaBound{F}\AgdaSymbol{)} \AgdaFunction{⟧} \AgdaFunction{⟦} \AgdaFunction{x₀:=} \AgdaBound{G} \AgdaFunction{⟧} \AgdaDatatype{≡} \AgdaBound{E} \AgdaFunction{⟦} \AgdaFunction{x₀:=} \AgdaSymbol{(}\AgdaBound{G} \AgdaFunction{⇑}\AgdaSymbol{)} \AgdaFunction{⟧} \AgdaFunction{⟦} \AgdaFunction{x₀:=} \AgdaBound{F} \AgdaFunction{⟧}\<%
\\
\>\AgdaFunction{botsub-botsub} \AgdaSymbol{\{}\AgdaBound{V}\AgdaSymbol{\}} \AgdaSymbol{\{}\AgdaBound{K}\AgdaSymbol{\}} \AgdaSymbol{\{}\AgdaBound{L}\AgdaSymbol{\}} \AgdaSymbol{\{}\AgdaBound{M}\AgdaSymbol{\}} \AgdaBound{E} \AgdaBound{F} \AgdaBound{G} \AgdaSymbol{=} \AgdaKeyword{let} \AgdaBound{COMP} \AgdaSymbol{=} \AgdaFunction{OpFamily.COMP} \AgdaFunction{SUB} \AgdaKeyword{in} \AgdaFunction{ap-circ-sim} \AgdaBound{COMP} \AgdaBound{COMP} \AgdaSymbol{(}\AgdaFunction{botsub-botsub'} \AgdaBound{F} \AgdaBound{G}\AgdaSymbol{)} \AgdaBound{E}\<%
\\
%
\\
\>\AgdaFunction{x₂:=\_,x₁:=\_,x₀:=\_} \AgdaSymbol{:} \AgdaSymbol{∀} \AgdaSymbol{\{}\AgdaBound{V}\AgdaSymbol{\}} \AgdaSymbol{\{}\AgdaBound{K1}\AgdaSymbol{\}} \AgdaSymbol{\{}\AgdaBound{K2}\AgdaSymbol{\}} \AgdaSymbol{\{}\AgdaBound{K3}\AgdaSymbol{\}} \AgdaSymbol{→} \AgdaFunction{Expression} \AgdaBound{V} \AgdaSymbol{(}\AgdaInductiveConstructor{varKind} \AgdaBound{K1}\AgdaSymbol{)} \AgdaSymbol{→} \AgdaFunction{Expression} \AgdaBound{V} \AgdaSymbol{(}\AgdaInductiveConstructor{varKind} \AgdaBound{K2}\AgdaSymbol{)} \AgdaSymbol{→} \AgdaFunction{Expression} \AgdaBound{V} \AgdaSymbol{(}\AgdaInductiveConstructor{varKind} \AgdaBound{K3}\AgdaSymbol{)} \AgdaSymbol{→} \AgdaFunction{Sub} \AgdaSymbol{(}\AgdaBound{V} \AgdaInductiveConstructor{,} \AgdaBound{K1} \AgdaInductiveConstructor{,} \AgdaBound{K2} \AgdaInductiveConstructor{,} \AgdaBound{K3}\AgdaSymbol{)} \AgdaBound{V}\<%
\\
\>\AgdaFunction{x₂:=\_,x₁:=\_,x₀:=\_} \AgdaBound{M1} \AgdaBound{M2} \AgdaBound{M3} \AgdaSymbol{=} \AgdaFunction{botsub} \AgdaSymbol{(}\AgdaInductiveConstructor{[]} \AgdaInductiveConstructor{snoc} \AgdaBound{M1} \AgdaInductiveConstructor{snoc} \AgdaBound{M2} \AgdaInductiveConstructor{snoc} \AgdaBound{M3}\AgdaSymbol{)}\<%
\\
%
\\
\>\AgdaKeyword{postulate} \AgdaPostulate{botsub-upRep₃} \AgdaSymbol{:} \AgdaSymbol{∀} \AgdaSymbol{\{}\AgdaBound{V}\AgdaSymbol{\}} \AgdaSymbol{\{}\AgdaBound{K1}\AgdaSymbol{\}} \AgdaSymbol{\{}\AgdaBound{K2}\AgdaSymbol{\}} \AgdaSymbol{\{}\AgdaBound{K3}\AgdaSymbol{\}} \AgdaSymbol{\{}\AgdaBound{L}\AgdaSymbol{\}} \AgdaSymbol{\{}\AgdaBound{M} \AgdaSymbol{:} \AgdaFunction{Expression} \AgdaBound{V} \AgdaBound{L}\AgdaSymbol{\}} \<[72]%
\>[72]\<%
\\
\>[2]\AgdaIndent{26}{}\<[26]%
\>[26]\AgdaSymbol{\{}\AgdaBound{N1} \AgdaSymbol{:} \AgdaFunction{Expression} \AgdaBound{V} \AgdaSymbol{(}\AgdaInductiveConstructor{varKind} \AgdaBound{K1}\AgdaSymbol{)\}} \AgdaSymbol{\{}\AgdaBound{N2} \AgdaSymbol{:} \AgdaFunction{Expression} \AgdaBound{V} \AgdaSymbol{(}\AgdaInductiveConstructor{varKind} \AgdaBound{K2}\AgdaSymbol{)\}} \AgdaSymbol{\{}\AgdaBound{N3} \AgdaSymbol{:} \AgdaFunction{Expression} \AgdaBound{V} \AgdaSymbol{(}\AgdaInductiveConstructor{varKind} \AgdaBound{K3}\AgdaSymbol{)\}} \AgdaSymbol{→}\<%
\\
\>[2]\AgdaIndent{26}{}\<[26]%
\>[26]\AgdaBound{M} \AgdaFunction{⇑} \AgdaFunction{⇑} \AgdaFunction{⇑} \AgdaFunction{⟦} \AgdaFunction{x₂:=} \AgdaBound{N1} \AgdaFunction{,x₁:=} \AgdaBound{N2} \AgdaFunction{,x₀:=} \AgdaBound{N3} \AgdaFunction{⟧} \AgdaDatatype{≡} \AgdaBound{M}\<%
\\
%
\\
\>\AgdaComment{--TODO Definition for Expression varKind}\<%
\\
\>\AgdaFunction{botsub₃-rep↑₃'} \AgdaSymbol{:} \AgdaSymbol{∀} \AgdaSymbol{\{}\AgdaBound{U}\AgdaSymbol{\}} \AgdaSymbol{\{}\AgdaBound{V}\AgdaSymbol{\}} \AgdaSymbol{\{}\AgdaBound{K2}\AgdaSymbol{\}} \AgdaSymbol{\{}\AgdaBound{K1}\AgdaSymbol{\}} \AgdaSymbol{\{}\AgdaBound{K0}\AgdaSymbol{\}}\<%
\\
\>[0]\AgdaIndent{2}{}\<[2]%
\>[2]\AgdaSymbol{\{}\AgdaBound{M2} \AgdaSymbol{:} \AgdaFunction{Expression} \AgdaBound{U} \AgdaSymbol{(}\AgdaInductiveConstructor{varKind} \AgdaBound{K1}\AgdaSymbol{)\}} \AgdaSymbol{\{}\AgdaBound{M1} \AgdaSymbol{:} \AgdaFunction{Expression} \AgdaBound{U} \AgdaSymbol{(}\AgdaInductiveConstructor{varKind} \AgdaBound{K2}\AgdaSymbol{)\}} \AgdaSymbol{\{}\AgdaBound{M0} \AgdaSymbol{:} \AgdaFunction{Expression} \AgdaBound{U} \AgdaSymbol{(}\AgdaInductiveConstructor{varKind} \AgdaBound{K0}\AgdaSymbol{)\}} \AgdaSymbol{\{}\AgdaBound{ρ} \AgdaSymbol{:} \AgdaFunction{Rep} \AgdaBound{U} \AgdaBound{V}\AgdaSymbol{\}} \AgdaSymbol{→}\<%
\\
\>[0]\AgdaIndent{2}{}\<[2]%
\>[2]\AgdaSymbol{(}\AgdaFunction{x₂:=} \AgdaBound{M2} \AgdaFunction{〈} \AgdaBound{ρ} \AgdaFunction{〉} \AgdaFunction{,x₁:=} \AgdaBound{M1} \AgdaFunction{〈} \AgdaBound{ρ} \AgdaFunction{〉} \AgdaFunction{,x₀:=} \AgdaBound{M0} \AgdaFunction{〈} \AgdaBound{ρ} \AgdaFunction{〉}\AgdaSymbol{)} \AgdaFunction{•SR} \AgdaFunction{rep↑} \AgdaSymbol{\_} \AgdaSymbol{(}\AgdaFunction{rep↑} \AgdaSymbol{\_} \AgdaSymbol{(}\AgdaFunction{rep↑} \AgdaSymbol{\_} \AgdaBound{ρ}\AgdaSymbol{))}\<%
\\
\>[0]\AgdaIndent{2}{}\<[2]%
\>[2]\AgdaFunction{∼} \AgdaBound{ρ} \AgdaFunction{•RS} \AgdaSymbol{(}\AgdaFunction{x₂:=} \AgdaBound{M2} \AgdaFunction{,x₁:=} \AgdaBound{M1} \AgdaFunction{,x₀:=} \AgdaBound{M0}\AgdaSymbol{)}\<%
\\
\>\AgdaFunction{botsub₃-rep↑₃'} \AgdaInductiveConstructor{x₀} \AgdaSymbol{=} \AgdaInductiveConstructor{refl}\<%
\\
\>\AgdaFunction{botsub₃-rep↑₃'} \AgdaSymbol{(}\AgdaInductiveConstructor{↑} \AgdaInductiveConstructor{x₀}\AgdaSymbol{)} \AgdaSymbol{=} \AgdaInductiveConstructor{refl}\<%
\\
\>\AgdaFunction{botsub₃-rep↑₃'} \AgdaSymbol{(}\AgdaInductiveConstructor{↑} \AgdaSymbol{(}\AgdaInductiveConstructor{↑} \AgdaInductiveConstructor{x₀}\AgdaSymbol{))} \AgdaSymbol{=} \AgdaInductiveConstructor{refl} \<[33]%
\>[33]\<%
\\
\>\AgdaFunction{botsub₃-rep↑₃'} \AgdaSymbol{(}\AgdaInductiveConstructor{↑} \AgdaSymbol{(}\AgdaInductiveConstructor{↑} \AgdaSymbol{(}\AgdaInductiveConstructor{↑} \AgdaBound{x}\AgdaSymbol{)))} \AgdaSymbol{=} \AgdaInductiveConstructor{refl}\<%
\\
%
\\
\>\AgdaFunction{botsub₃-rep↑₃} \AgdaSymbol{:} \AgdaSymbol{∀} \AgdaSymbol{\{}\AgdaBound{U}\AgdaSymbol{\}} \AgdaSymbol{\{}\AgdaBound{V}\AgdaSymbol{\}} \AgdaSymbol{\{}\AgdaBound{K2}\AgdaSymbol{\}} \AgdaSymbol{\{}\AgdaBound{K1}\AgdaSymbol{\}} \AgdaSymbol{\{}\AgdaBound{K0}\AgdaSymbol{\}} \AgdaSymbol{\{}\AgdaBound{L}\AgdaSymbol{\}}\<%
\\
\>[0]\AgdaIndent{2}{}\<[2]%
\>[2]\AgdaSymbol{\{}\AgdaBound{M2} \AgdaSymbol{:} \AgdaFunction{Expression} \AgdaBound{U} \AgdaSymbol{(}\AgdaInductiveConstructor{varKind} \AgdaBound{K2}\AgdaSymbol{)\}} \AgdaSymbol{\{}\AgdaBound{M1} \AgdaSymbol{:} \AgdaFunction{Expression} \AgdaBound{U} \AgdaSymbol{(}\AgdaInductiveConstructor{varKind} \AgdaBound{K1}\AgdaSymbol{)\}} \AgdaSymbol{\{}\AgdaBound{M0} \AgdaSymbol{:} \AgdaFunction{Expression} \AgdaBound{U} \AgdaSymbol{(}\AgdaInductiveConstructor{varKind} \AgdaBound{K0}\AgdaSymbol{)\}} \AgdaSymbol{\{}\AgdaBound{ρ} \AgdaSymbol{:} \AgdaFunction{Rep} \AgdaBound{U} \AgdaBound{V}\AgdaSymbol{\}} \AgdaSymbol{(}\AgdaBound{N} \AgdaSymbol{:} \AgdaFunction{Expression} \AgdaSymbol{(}\AgdaBound{U} \AgdaInductiveConstructor{,} \AgdaBound{K2} \AgdaInductiveConstructor{,} \AgdaBound{K1} \AgdaInductiveConstructor{,} \AgdaBound{K0}\AgdaSymbol{)} \AgdaBound{L}\AgdaSymbol{)} \AgdaSymbol{→}\<%
\\
\>[0]\AgdaIndent{2}{}\<[2]%
\>[2]\AgdaBound{N} \AgdaFunction{〈} \AgdaFunction{rep↑} \AgdaSymbol{\_} \AgdaSymbol{(}\AgdaFunction{rep↑} \AgdaSymbol{\_} \AgdaSymbol{(}\AgdaFunction{rep↑} \AgdaSymbol{\_} \AgdaBound{ρ}\AgdaSymbol{))} \AgdaFunction{〉} \AgdaFunction{⟦} \AgdaFunction{x₂:=} \AgdaBound{M2} \AgdaFunction{〈} \AgdaBound{ρ} \AgdaFunction{〉} \AgdaFunction{,x₁:=} \AgdaBound{M1} \AgdaFunction{〈} \AgdaBound{ρ} \AgdaFunction{〉} \AgdaFunction{,x₀:=} \AgdaBound{M0} \AgdaFunction{〈} \AgdaBound{ρ} \AgdaFunction{〉} \AgdaFunction{⟧}\<%
\\
\>[0]\AgdaIndent{2}{}\<[2]%
\>[2]\AgdaDatatype{≡} \AgdaBound{N} \AgdaFunction{⟦} \AgdaFunction{x₂:=} \AgdaBound{M2} \AgdaFunction{,x₁:=} \AgdaBound{M1} \AgdaFunction{,x₀:=} \AgdaBound{M0} \AgdaFunction{⟧} \AgdaFunction{〈} \AgdaBound{ρ} \AgdaFunction{〉}\<%
\\
\>\AgdaFunction{botsub₃-rep↑₃} \AgdaSymbol{\{}\AgdaArgument{M2} \AgdaSymbol{=} \AgdaBound{M2}\AgdaSymbol{\}} \AgdaSymbol{\{}\AgdaBound{M1}\AgdaSymbol{\}} \AgdaSymbol{\{}\AgdaBound{M0}\AgdaSymbol{\}} \AgdaSymbol{\{}\AgdaBound{ρ}\AgdaSymbol{\}} \AgdaBound{N} \AgdaSymbol{=} \AgdaKeyword{let} \AgdaKeyword{open} \AgdaModule{≡-Reasoning} \AgdaKeyword{in}\<%
\\
\>[2]\AgdaIndent{14}{}\<[14]%
\>[14]\AgdaFunction{begin}\<%
\\
\>[14]\AgdaIndent{16}{}\<[16]%
\>[16]\AgdaBound{N} \AgdaFunction{〈} \AgdaFunction{rep↑} \AgdaSymbol{\_} \AgdaSymbol{(}\AgdaFunction{rep↑} \AgdaSymbol{\_} \AgdaSymbol{(}\AgdaFunction{rep↑} \AgdaSymbol{\_} \AgdaBound{ρ}\AgdaSymbol{))} \AgdaFunction{〉} \AgdaFunction{⟦} \AgdaFunction{x₂:=} \AgdaBound{M2} \AgdaFunction{〈} \AgdaBound{ρ} \AgdaFunction{〉} \AgdaFunction{,x₁:=} \AgdaBound{M1} \AgdaFunction{〈} \AgdaBound{ρ} \AgdaFunction{〉} \AgdaFunction{,x₀:=} \AgdaBound{M0} \AgdaFunction{〈} \AgdaBound{ρ} \AgdaFunction{〉} \AgdaFunction{⟧}\<%
\\
\>[0]\AgdaIndent{14}{}\<[14]%
\>[14]\AgdaFunction{≡⟨⟨} \AgdaFunction{sub-compSR} \AgdaBound{N} \AgdaFunction{⟩⟩}\<%
\\
\>[14]\AgdaIndent{16}{}\<[16]%
\>[16]\AgdaBound{N} \AgdaFunction{⟦} \AgdaSymbol{(}\AgdaFunction{x₂:=} \AgdaBound{M2} \AgdaFunction{〈} \AgdaBound{ρ} \AgdaFunction{〉} \AgdaFunction{,x₁:=} \AgdaBound{M1} \AgdaFunction{〈} \AgdaBound{ρ} \AgdaFunction{〉} \AgdaFunction{,x₀:=} \AgdaBound{M0} \AgdaFunction{〈} \AgdaBound{ρ} \AgdaFunction{〉}\AgdaSymbol{)} \AgdaFunction{•SR} \AgdaFunction{rep↑} \AgdaSymbol{\_} \AgdaSymbol{(}\AgdaFunction{rep↑} \AgdaSymbol{\_} \AgdaSymbol{(}\AgdaFunction{rep↑} \AgdaSymbol{\_} \AgdaBound{ρ}\AgdaSymbol{))} \AgdaFunction{⟧}\<%
\\
\>[0]\AgdaIndent{14}{}\<[14]%
\>[14]\AgdaFunction{≡⟨} \AgdaFunction{sub-congr} \AgdaBound{N} \AgdaFunction{botsub₃-rep↑₃'} \AgdaFunction{⟩}\<%
\\
\>[14]\AgdaIndent{16}{}\<[16]%
\>[16]\AgdaBound{N} \AgdaFunction{⟦} \AgdaBound{ρ} \AgdaFunction{•RS} \AgdaSymbol{(}\AgdaFunction{x₂:=} \AgdaBound{M2} \AgdaFunction{,x₁:=} \AgdaBound{M1} \AgdaFunction{,x₀:=} \AgdaBound{M0}\AgdaSymbol{)} \AgdaFunction{⟧}\<%
\\
\>[0]\AgdaIndent{14}{}\<[14]%
\>[14]\AgdaFunction{≡⟨} \AgdaFunction{sub-compRS} \AgdaBound{N} \AgdaFunction{⟩}\<%
\\
\>[14]\AgdaIndent{16}{}\<[16]%
\>[16]\AgdaBound{N} \AgdaFunction{⟦} \AgdaFunction{x₂:=} \AgdaBound{M2} \AgdaFunction{,x₁:=} \AgdaBound{M1} \AgdaFunction{,x₀:=} \AgdaBound{M0} \AgdaFunction{⟧} \AgdaFunction{〈} \AgdaBound{ρ} \AgdaFunction{〉}\<%
\\
\>[0]\AgdaIndent{14}{}\<[14]%
\>[14]\AgdaFunction{∎}\<%
\\
\>\AgdaComment{--TODO General lemma for this}\<%
\\
\>\AgdaComment{--TODO Deletable?}\<%
\end{code}
}


\end{document}
