\documentclass{beamer}

\title{An Agda Library for Metatheory of Syntax with Binding}
\author{Robin Adams}
\date{17 November 2016}

\usepackage{../agda}
\usepackage{catchfilebetweentags}
\usepackage{amsmath}
\usepackage{amssymb}
\usepackage{autofe}
\usepackage{bbm}
\usepackage[greek,english]{babel}
\usepackage{etex}
\usepackage{framed}
\usepackage[utf8x]{inputenc}
\usepackage{proof}
\usepackage{stmaryrd}
\usepackage{suffix}
\usepackage{textalpha}
\usepackage{ucs}
\DeclareUnicodeCharacter{8598}{\ensuremath{\nwarrow}}
\DeclareUnicodeCharacter{8599}{\ensuremath{\nearrow}}
\DeclareUnicodeCharacter{8608}{\ensuremath{\twoheadrightarrow}}
\DeclareUnicodeCharacter{8657}{\ensuremath{\Uparrow}}
\DeclareUnicodeCharacter{8667}{\ensuremath{\Rrightarrow}}
\DeclareUnicodeCharacter{8718}{\ensuremath{\qed}}
\DeclareUnicodeCharacter{8759}{\ensuremath{::}}
\DeclareUnicodeCharacter{8988}{\ensuremath{\ulcorner}}
\DeclareUnicodeCharacter{8989}{\ensuremath{\urcorner}}
\DeclareUnicodeCharacter{8803}{\ensuremath{\overline{\equiv}}}
\DeclareUnicodeCharacter{9001}{\ensuremath{\langle}}
\DeclareUnicodeCharacter{9002}{\ensuremath{\rangle}}
\DeclareUnicodeCharacter{9655}{\ensuremath{\rhd}}
\DeclareUnicodeCharacter{9679}{\ensuremath{\bullet}}
\DeclareUnicodeCharacter{10214}{\ensuremath{[}}
\DeclareUnicodeCharacter{10215}{\ensuremath{]}}
\DeclareUnicodeCharacter{10219}{\ensuremath{\rangle\rangle}}
\DeclareUnicodeCharacter{10230}{\ensuremath{\longrightarrow}}

\newtheorem{prop}[theorem]{Proposition}
\newtheorem{lm}[theorem]{Lemma}
\newtheorem{cor}{Corollary}[theorem]

\newcommand*{\Set}{\mathbf{Set}}
\newcommand*{\eqdef}{\mathrel{\smash{\stackrel{\text{def}}{=}}}}
\newcommand*{\isotoid}{\ensuremath{isotoid}}
\newcommand*{\vald}{\ensuremath{\ \mathrm{valid}}}
\newcommand*{\reff}[1]{\ensuremath{\mathsf{ref} \left( {#1} \right)}}
\newcommand*{\univ}[4]{\ensuremath{\mathsf{univ}_{{#1} , {#2}} \left( {#3} , {#4} \right)}}
\newcommand*{\triplelambda}{\ensuremath{\lambda \!\! \lambda \!\! \lambda}}
\newcommand*{\SN}{\ensuremath{\mathbf{SN}}}
\newcommand*{\dom}{\ensuremath{\operatorname{dom}}}
\newcommand*{\sym}[4]{\ensuremath{\mathsf{sym}_{{#1},{#2},{#3}} \left( {#4} \right)}}
\WithSuffix\newcommand\sym*[1]{\ensuremath{\mathsf{sym} \left( {#1} \right)}}
\newcommand*{\trans}[6]{\ensuremath{\mathsf{trans}_{{#1},{#2},{#3},{#4}} \left( {#5} , {#6} \right)}}
\WithSuffix\newcommand\trans*[2]{\ensuremath{\mathsf{trans} \left( {#1} , {#2} \right)}}
\newcommand*{\kr}{\mathop{\rhd \!\!\! \rhd}}



\begin{document}

\begin{frame}
\maketitle
\end{frame}

\begin{frame}
\frametitle{How Do We Formalize Metatheory?}
We give the terms as an inductive datatype / family:

\ExecuteMetaData[example.tex]{Term}

then prove statements by induction:

\ExecuteMetaData[example.tex]{Sub}

\begin{itemize}
\item
Code cannot be reused when we start a new formalization with a different object theory.
\item
A change to the object theory means a lot of work must be redone.
\item
One similar case for each binding constructor --- repeated work.
\end{itemize}
\end{frame}

\begin{frame}
\frametitle{Principles for Formalization}
\begin{itemize}[<+->]
\item
DRY --- Don't Repeat Yourself
\item
YAGNI --- You Ain't Gonna Need It
\item
Refactor early, refactor often
\item
Work in type theory, not in Haskell
\end{itemize}
\end{frame}

\begin{frame}
\frametitle{Running Example}
The syntax of the $\overline{\lambda} \mu \tilde{\mu}$-calculus:
\[ \begin{array}{lrcl}
\text{Producer} & M & ::= & x \mid \lambda x M \mid \mu \alpha C \\
\text{Consumer} & E & ::= & \alpha \mid \tilde{\mu} x C \mid M E \\
\text{Command} & C & ::= & \langle M | E \rangle
\end{array} \]
\end{frame}

\begin{frame}
\frametitle{Taxonomy}
A \emph{taxonomy} consists of a set of \emph{expression kinds}, divided into \emph{variable kinds} and \emph{non-variable kinds}.

\ExecuteMetaData[../Grammar/Taxonomy.tex]{Taxonomy}
\end{frame}

\begin{frame}
\frametitle{Taxonomy --- Example}
\ExecuteMetaData[example.tex]{Taxonomy}
\end{frame}

\begin{frame}
\frametitle{Alphabets}
An \emph{alphabet} $A$ is a finite set of \emph{variables}, each with
a variable kind $K$.

We write $\mathsf{Var}\ A\ K$ for the set of all variables in $A$ of kind $K$.

\ExecuteMetaData[../Grammar/Taxonomy.tex]{Alphabet}

\ExecuteMetaData[../Grammar/Taxonomy.tex]{Var}
\end{frame}

\begin{frame}
\[ \begin{array}{lrcl}
\text{Producer} & M & ::= & x \mid \lambda x M \mid \mu \alpha C \\
\text{Consumer} & E & ::= & \alpha \mid \tilde{\mu} x C \mid M E \\
\text{Command} & C & ::= & \langle M | E \rangle
\end{array} \]

A grammar is given by a set of \emph{constructors}:
\newcommand{\prd}{\mathrm{Prod}}
\newcommand{\cons}{\mathrm{Cons}}
\newcommand{\comm}{\mathrm{Comm}}
\begin{align*}
\lambda & : (\prd \rightarrow \prd) \rightarrow \prd \\
\mu & : (\cons \rightarrow \comm) \rightarrow \prd \\
\tilde{\mu} & : (\prd \rightarrow \comm) \rightarrow \cons \\
\mathrm{app} & : \prd \rightarrow \cons \rightarrow \cons \\
\langle | \rangle & : \prd \rightarrow \cons \rightarrow \comm
\end{align*}
\end{frame}

\begin{frame}
\frametitle{Constructor Kinds}
\newcommand{\prd}{\mathrm{Prod}}
\newcommand{\cons}{\mathrm{Cons}}
\newcommand{\comm}{\mathrm{Comm}}

An expression like $(\prd \rightarrow \prd)$ is an \emph{abstraction kind}.

An expression like $(\prd \rightarrow \prd) \rightarrow \prd$ is a \emph{constructor kind}.

\ExecuteMetaData[../Grammar/Taxonomy.tex]{SimpleKind}

\ExecuteMetaData[../Grammar/Taxonomy.tex]{ConKind}
\end{frame}

\begin{frame}
\frametitle{Grammar}
A \emph{grammar} is a set of \emph{constructors}, each with a constructor kind.

\ExecuteMetaData[../Grammar/Base.tex]{Grammar}

Example --- $\overline{\lambda} \mu \tilde{\mu}$:

\ExecuteMetaData[example.tex]{Grammar}
\end{frame}

\begin{frame}
\frametitle{Expressions}
Define simultaneously:
\begin{itemize}
\item
If $x$ is a variable of kind $K$ in $V$ \\
then $x$ is an expression of kind $K$ over $V$
\item
If $c$ is a constructor of kind $\vec{A} \longrightarrow K$ over $V$, \\
and $\vec{E}$ an abstraction list of kind $\vec{A}$ over $V$, \\
then $c \vec{E}$ is an expression of kind $K$ over $V$.
\item
An abstraction of kind $\vec{A} \rightarrow K$ over $V$ \\
is an expression of kind $K$ over $(V , \vec{A})$.
\item
An abstraction list of kind $A_1, \ldots, A_n$ \\
is an abstraction of kind $A_1$, \ldots, an abstraction of kind $A_n$.
\end{itemize}
\end{frame}

\begin{frame}
\frametitle{Expressions}
\ExecuteMetaData[../Grammar/Base.tex]{Subexp}
\end{frame}

\begin{frame}
\frametitle{Replacement and Substitution}
A \emph{replacement} is a mapping from variables to variables:

\ExecuteMetaData[../Grammar/Replacement.tex]{Rep}

A \emph{substitution} is a mapping from variables to expressions:

\ExecuteMetaData[../Grammar/Substitution/PreOpFamily.tex]{Sub}

With these definitions, we can prove general results such as:

\ExecuteMetaData[../Grammar/Substitution/OpFamily.tex]{sub-comp}
\end{frame}

\begin{frame}
\frametitle{Reduction Relations}
A \emph{reduction relation} is a relation $R$ between expressions of the same kind such that,
if $E R F$, then $E$ is not a variable.
\ExecuteMetaData[../Grammar/Base.tex]{Red}
We can define $\rightarrow_R$, $\twoheadrightarrow_R$, $\simeq_R$ and prove results like:
\ExecuteMetaData[../PHOPL/Red/Meta.tex]{osr-subl}
\end{frame}

\begin{frame}
\frametitle{Future Work}
\begin{itemize}
\item
Translation between two grammars.
\item
General 
\item
Reduction relation and rules as instantiations of \emph{patterns} with second-order variables.
\item
The POPLMark challenge.
\end{itemize}
\end{frame}

\end{document}