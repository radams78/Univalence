\newcommand{\id}[1]{\mathsf{id}_{#1}}

\section{Grammars}

\subsection{Taxonomy}

\AgdaHide{
\begin{code}%
\>\AgdaKeyword{module} \AgdaModule{Grammar.Taxonomy} \AgdaKeyword{where}\<%
\\
\>\AgdaKeyword{open} \AgdaKeyword{import} \AgdaModule{Data.List} \AgdaKeyword{public}\<%
\\
\>\AgdaKeyword{open} \AgdaKeyword{import} \AgdaModule{Prelims}\<%
\end{code}
}

Before we begin investigating the several theories we wish to consider, we present a general theory of syntax and
capture-avoiding substitution.

A \emph{taxononmy} consists of:
\begin{itemize}
\item a set of \emph{expression kinds};
\item a subset of expression kinds, called the \emph{variable kinds}.  We refer to the other expession kinds as \emph{non-variable kinds}.
\end{itemize}

%<*Taxonomy>
\begin{code}%
\>\AgdaKeyword{record} \AgdaRecord{Taxonomy} \AgdaSymbol{:} \AgdaPrimitiveType{Set₁} \AgdaKeyword{where}\<%
\\
\>[0]\AgdaIndent{2}{}\<[2]%
\>[2]\AgdaKeyword{field}\<%
\\
\>[2]\AgdaIndent{4}{}\<[4]%
\>[4]\AgdaField{VarKind} \AgdaSymbol{:} \AgdaPrimitiveType{Set}\<%
\\
\>[2]\AgdaIndent{4}{}\<[4]%
\>[4]\AgdaField{NonVarKind} \AgdaSymbol{:} \AgdaPrimitiveType{Set}\<%
\\
%
\\
\>[0]\AgdaIndent{2}{}\<[2]%
\>[2]\AgdaKeyword{data} \AgdaDatatype{ExpressionKind} \AgdaSymbol{:} \AgdaPrimitiveType{Set} \AgdaKeyword{where}\<%
\\
\>[2]\AgdaIndent{4}{}\<[4]%
\>[4]\AgdaInductiveConstructor{varKind} \AgdaSymbol{:} \AgdaField{VarKind} \AgdaSymbol{→} \AgdaDatatype{ExpressionKind}\<%
\\
\>[2]\AgdaIndent{4}{}\<[4]%
\>[4]\AgdaInductiveConstructor{nonVarKind} \AgdaSymbol{:} \AgdaField{NonVarKind} \AgdaSymbol{→} \AgdaDatatype{ExpressionKind}\<%
\\
%
\\
\>[0]\AgdaIndent{2}{}\<[2]%
\>[2]\AgdaFunction{AbstractionKind} \AgdaSymbol{:} \AgdaPrimitiveType{Set}\<%
\\
\>[0]\AgdaIndent{2}{}\<[2]%
\>[2]\AgdaFunction{AbstractionKind} \AgdaSymbol{=} \AgdaRecord{PiExp} \AgdaField{VarKind} \AgdaDatatype{ExpressionKind}\<%
\\
\>\AgdaComment{--TODO: Maybe get rid of NonVarKind?}\<%
\end{code}
%</Taxonomy>

A constructor $c$ of kind (\ref{eq:conkind}) is a constructor that takes $m$ arguments of kind $B_1$, \ldots, $B_m$, and binds $r_i$ variables in its $i$th argument of kind $A_{ij}$,
producing an expression of kind $C$.  We write this expression as

\begin{equation}
\label{eq:expression}
c([x_{11}, \ldots, x_{1r_1}]E_1, \ldots, [x_{m1}, \ldots, x_{mr_m}]E_m) \enspace .
\end{equation}

The subexpressions of the form $[x_{i1}, \ldots, x_{ir_i}]E_i$ shall be called \emph{abstractions}.

When giving a specific grammar, we shall feel free to use BNF notation.  

We formalise this as follows.  First, we construct the sets of expression kinds and constructor kinds over a taxonomy:

\begin{frame}[fragile]
There are two \emph{classes} of kinds: expression kinds and constructor kinds.

\begin{code}%
\>[0]\AgdaIndent{2}{}\<[2]%
\>[2]\AgdaKeyword{data} \AgdaDatatype{KindClass} \AgdaSymbol{:} \AgdaPrimitiveType{Set} \AgdaKeyword{where}\<%
\\
\>[2]\AgdaIndent{4}{}\<[4]%
\>[4]\AgdaInductiveConstructor{-Expression} \AgdaSymbol{:} \AgdaDatatype{KindClass}\<%
\\
\>[2]\AgdaIndent{4}{}\<[4]%
\>[4]\AgdaInductiveConstructor{-Constructor} \AgdaSymbol{:} \AgdaDatatype{KindClass}\<%
\\
%
\\
\>[0]\AgdaIndent{2}{}\<[2]%
\>[2]\AgdaFunction{Kind} \AgdaSymbol{:} \AgdaDatatype{KindClass} \AgdaSymbol{→} \AgdaPrimitiveType{Set}\<%
\\
\>[0]\AgdaIndent{2}{}\<[2]%
\>[2]\AgdaFunction{Kind} \AgdaInductiveConstructor{-Expression} \AgdaSymbol{=} \AgdaDatatype{ExpressionKind}\<%
\\
\>[0]\AgdaIndent{2}{}\<[2]%
\>[2]\AgdaFunction{Kind} \AgdaInductiveConstructor{-Constructor} \AgdaSymbol{=} \AgdaDatatype{List} \AgdaSymbol{(}\AgdaRecord{PiExp} \AgdaField{VarKind} \AgdaDatatype{ExpressionKind}\AgdaSymbol{)}\<%
\end{code}
\end{frame}
%TODO Colours in Agda code?
