\AgdaHide{
\begin{code}%
\>\AgdaKeyword{module} \AgdaModule{PHOPL.KeyRedex2} \AgdaKeyword{where}\<%
\end{code}
}

Let $\SN$ be the set of all strongly normalizing expressions.

\begin{prop}
\label{prop:SN}
$ $
% \begin{enumerate}
% \item
\begin{enumerate}
\item
\label{prop:SNT}
If $M[x:=N]L_1 \cdots L_n \in \SN$ and $N \in \SN$ then $(\lambda x:A.M)NL_1 \cdots L_n \in \SN$.
% \item
% \label{lm:SN2}
% If $(\reff{M[x:=N]N_1 \cdots N_m}_{L_1 L_1'} (P_1)_{L_2 L_2'} \cdots_{L_n L_n'} P_n)^+ \delta_1 \cdots \delta_m \in \SN$ and $N \in \SN$ then
% $(\reff{(\lambda x:A.M)NN_1 \cdots N_m}_{L_1 L_1'} (P_1)_{L_2 L_2'} \cdots_{L_n L_n'} P_n)^+ \delta_1 \cdots \delta_m \in \SN$.
% \item
% \label{lm:SN3}
% Suppose that:
% \begin{enumerate}
% \item
% \label{hypi}
% $(M\{x:=P:N\sim N'\}_{L_1 L_1'} P_1 \cdots_{L_n L_n'} P_n)^+ \delta_1 \cdots \delta_m \in \SN$
% \item
% $P, N, N' \in \SN$
% \item
% \label{hypiii}
% if $\nf{P} \equiv \reff{L}$ then $(\reff{M[x:=L]}_{L_1 L_1'} P_1 \cdots_{L_n L_n'} P_n)^+ \delta_1 \cdots \delta_m \in \SN$.
% \end{enumerate}
% Then $(\reff{\lambda x:A.M}_{N N'} P_{L_1 L_1'} P_1 \cdots_{L_n L_n'} P_n)^+ \delta_1 \cdots \delta_m \in \SN$.
% \item
% Suppose that:
% \begin{enumerate}
% \item
% \label{hypi}
% $(M\{x:=P:N\sim N'\}_{L_1 L_1'} P_1 \cdots_{L_n L_n'} P_n)^- \delta_1 \cdots \delta_m \in \SN$
% \item
% $P, N, N' \in \SN$
% \item
% \label{hypiii}
% if $\nf{P} \equiv \reff{L}$ then $(\reff{M[x:=L]}_{L_1 L_1'} P_1 \cdots_{L_n L_n'} P_n)^- \delta_1 \cdots \delta_m \in \SN$.
% \end{enumerate}
% Then $(\reff{\lambda x:A.M}_{N N'} P_{L_1 L_1'} P_1 \cdots_{L_n L_n'} P_n)^- \delta_1 \cdots \delta_m \in \SN$.
% \item
% \label{lm:SN4}
% If $(\reff{M[x:=N]}_{L_1 L_1'} (P_1)_{L_2 L_2'} \cdots_{L_n L_n'} P_n)^+ \delta_1 \cdots \delta_m \in \SN$ and $N_1, N_2, N \in \SN$ then \\
% $(\reff{\lambda x:A.M}_{N_1 N_2}\reff{N}_{L_1 L_1'} (P_1)_{L_2 L_2'} \cdots_{L_n L_n'} P_n)^+ \delta_1 \cdots \delta_m \in \SN$.
% \item
% If $(\reff{M[x:=N]}_{L_1 L_1'} (P_1)_{L_2 L_2'} \cdots_{L_n L_n'} P_n)^- \delta_1 \cdots \delta_m \in \SN$ and $N_1, N_2, N \in \SN$ then \\
% $(\reff{\lambda x:A.M}_{N_1 N_2}\reff{N}_{L_1 L_1'} (P_1)_{L_2 L_2'} \cdots_{L_n L_n'} P_n)^- \delta_1 \cdots \delta_m \in \SN$.
% \item
% \label{lm:SN5}
% If $\delta, \phi \in \SN$ then $\reff{\phi}^+ \delta, \reff{\phi}^- \delta \in \SN$.
% \end{enumerate}
\item
\label{prop:SNP}
If $\delta[p:=\epsilon], \phi, \epsilon \in \SN$ then $(\lambda p:\phi.\delta)\epsilon \in \SN$.
\item
\label{prop:SNE}
If $(P[x:=L, y:=L', e:=Q]_{M_1 N_1} Q_1 \cdots_{M_n N_n} Q_n)^+ \delta_1 \cdots \delta_m \in \SN$ and $L, L', Q \in \SN$ then $((\triplelambda e:x =_A y.P)_{L L'} Q_{M_1 N_1} Q_1 \cdots_{M_n N_n} Q_n)^+ \delta_1 \cdots \delta_m \in \SN$.
\item
If $(P[x:=L, y:=L', e:=Q]_{M_1 N_1} Q_1 \cdots_{M_n N_n} Q_n)^- \delta_1 \cdots \delta_m \in \SN$ and $L, L', Q \in \SN$ then $((\triplelambda e:x =_A y.P)_{L L'} Q_{M_1 N_1} Q_1 \cdots_{M_n N_n} Q_n)^- \delta_1 \cdots \delta_m \in \SN$.
\end{enumerate}
\end{prop}

\begin{proof}
We prove part \ref{prop:SNT}; the proofs of the other parts are similar.

The proof is by a double induction on the hypotheses.  Consider all possible one-step reductions from $(\lambda x:A.M) N \vec{L}$.  The possibilities are:
\begin{description}
\item[$(\lambda x:A.M) N \vec{L} \rightarrow (\lambda x:A.M')N \vec{L}$, where $M \rightarrow M'$]
$ $

In this case, we have $M[x:=N] \vec{L} \rightarrow M'[x:=N] \vec{L}$, and the result follows by the induction hypothesis.  Similarly for the case
where we reduce $N$ or one of the $L_i$.
\item[$(\lambda x:A.M)N \vec{L} \rightarrow M{[x:=N]} \vec{L}$]
$ $

In this case, the result follows immediately from the hypothesis.
\end{description}
\end{proof}

% \begin{lemma}
% \label{lm:SNrefapp}
% If $\reff{M}_{N_1 N_2} \reff{N}_{K_1 K_1'} P_1 \cdots_{K_n K_n'} P_n \in \SN$ then \\
% $\reff{MN}_{K_1 K_1'} P_1 \cdots_{K_n K_n'} P_n \in \SN$.
% \end{lemma}

% \begin{proof}
% The proof is by induction on the hypothesis.  These are the possible one-step reductions from $\reff{MN} \vec{P}$:

% \begin{description}
% \item[$\reff{MN} \vec{P} \rightarrow \reff{M'N} \vec{P}$ where $M \rightarrow M'$]
% Then we have $\reff{M}_{N_1 N_2} \reff{N} \vec{P} \rightarrow \reff{M'}_{N_1 N_2} \reff{N} \vec{P} \in \SN$, and the result follows by the
% induction hypothesis.  Similarly if we reduce $N$ or one of the $P_i$, $K_i$ or $K_i'$.

% \item[$MN$ is a redex]
% Then $M$ and $N$ are closed normal forms, and so \\
% $\reff{M}_{N_1 N_2} \reff{N} \vec{P} \rightarrow \reff{MN} \vec{P}$, hence
% $\reff{MN} \vec{P} \in \SN$.
% \end{description}
% \end{proof}

% \begin{lemma}
% \label{lm:wte_loi1}
% If $(\delta \chi_{L_1 L_1'} \theta_1 \cdots_{L_m L_m'} \theta_m)^+ \alpha \beta_1 \cdots \beta_n \in \SN$ and $\phi, \psi, \epsilon \in \SN$, then
% $(\univ{\phi}{\psi}{\delta}{\epsilon}^+ \chi_{L_1 L_1'} \theta_1 \cdots_{L_m L_m'} \theta_m)^+ \alpha \beta_1 \cdots \beta_n \in \SN$.
% \end{lemma}

% \begin{proof}
% Similar.
% \end{proof}

% \begin{lemma}
% \label{lm:wte_loi2}
% If $(\delta[p:=\epsilon]_{L_1 L_1'} \theta_1 \cdots_{L_m L_m'} \theta_m)^+ \alpha \beta_1 \cdots \beta_n \in \SN$ and $\phi, \epsilon \in \SN$, then
% $((\lambda p:\phi.\delta) \epsilon_{L_1 L_1'} \theta_1 \cdots_{L_m L_m'} \theta_m)^+ \alpha \beta_1 \cdots \beta_n \in \SN$.
% \end{lemma}

% \begin{proof}
% Similar.
% \end{proof}

% \begin{lemma}
% \label{lm:wte_loi3}
% If $(\delta[x:=M, y:=N, p:=\epsilon]_{L_1 L_1'} \theta_1 \cdots_{L_m L_m'} \theta_m)^+ \alpha \beta_1 \cdots \beta_n \in \SN$ and $M, N, \epsilon \in \SN$, then
% $((\triplelambda p:x =_A y.\delta)_{MN} \epsilon_{L_1 L_1'} \theta_1 \cdots_{L_m L_m'} \theta_m)^+ \alpha \beta_1 \cdots \beta_n \in \SN$.
% \end{lemma}

% \begin{proof}
% Similar.
% \end{proof}

% \begin{lemma}
% \label{lm:SNred1}
% If $(\reff{M}_{NN} \reff{N}_{L_1 L_1'} P_1 \cdots_{L_n L_n'} P_n)^+ \delta_1 \cdots \delta_m \in \SN$ then \\
% $(\reff{MN}_{L_1 L_1'} P_1 \cdots_{L_n L_n'} P_n)^+ \delta_1 \cdots \delta_m \in \SN$.
% \end{lemma}

% \begin{proof}
% The proof is by induction on the hypothesis.  Consider all possible one-step reductions from $(\reff{MN} \vec{P})^+ \vec{\delta}$.

% If $(\reff{MN} \vec{P})^+ \vec{\delta} \rightarrow (\reff{MN'} \vec{P})^+ \vec{\delta}$ where $M \rightarrow M'$, then we have $(\reff{M}_{N' N'} \reff{N'} \vec{P})^+ \vec{\delta} \in \SN$, and so
% $(\reff{MN'} \vec{P})^+ \vec{\delta} \in \SN$ by the induction hypothesis.  Similarly for the case where we reduce $M$ or one of the $L_i$, $L_i'$, $P_i$ or $\delta_j$.

% The only other case is is $MN$ is a redex.  In this case, $M$ and $N$ are closed normal forms, and so we have $(\reff{M}_{NN} \reff{N} \vec{P})^+ \vec{\delta} \rightarrow (\reff{MN} \vec{P})^+ \vec{\delta} \in \SN$.
% \end{proof}

% \begin{lemma}
% If $(\reff{M}_{NN} \reff{N}_{L_1 L_1'} P_1 \cdots_{L_n L_n'} P_n)^- \delta_1 \cdots \delta_m \in \SN$ then \\
% $(\reff{MN}_{L_1 L_1'} P_1 \cdots_{L_n L_n'} P_n)^- \delta_1 \cdots \delta_m \in \SN$.
% \end{lemma}

% \begin{proof}
% Similar.
% \end{proof}