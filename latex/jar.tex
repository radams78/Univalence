%%%%%%%%%%%%%%%%%%%%%%% file template.tex %%%%%%%%%%%%%%%%%%%%%%%%%
%
% This is a general template file for the LaTeX package SVJour3
% for Springer journals.          Springer Heidelberg 2010/09/16
%
% Copy it to a new file with a new name and use it as the basis
% for your article. Delete % signs as needed.
%
% This template includes a few options for different layouts and
% content for various journals. Please consult a previous issue of
% your journal as needed.
%
%%%%%%%%%%%%%%%%%%%%%%%%%%%%%%%%%%%%%%%%%%%%%%%%%%%%%%%%%%%%%%%%%%%
%
% First comes an example EPS file -- just ignore it and
% proceed on the \documentclass line
% your LaTeX will extract the file if required
\begin{filecontents*}{example.eps}
%!PS-Adobe-3.0 EPSF-3.0
%%BoundingBox: 19 19 221 221
%%CreationDate: Mon Sep 29 1997
%%Creator: programmed by hand (JK)
%%EndComments
gsave
newpath
  20 20 moveto
  20 220 lineto
  220 220 lineto
  220 20 lineto
closepath
2 setlinewidth
gsave
  .4 setgray fill
grestore
stroke
grestore
\end{filecontents*}
%
\RequirePackage{fix-cm}
%
%\documentclass{svjour3}                     % onecolumn (standard format)
\documentclass[smallcondensed,draft]{svjour3}     % onecolumn (ditto)
%TODO Remove 'draft' before submitting
%\documentclass[smallextended]{svjour3}       % onecolumn (second format)
%\documentclass[twocolumn]{svjour3}          % twocolumn
%
\smartqed  % flush right qed marks, e.g. at end of proof
%
\usepackage{graphicx}
%
\usepackage{mathptmx}      % use Times fonts if available on your TeX system
%
% insert here the call for the packages your document requires
\usepackage{agda}
\usepackage{amsmath}
\usepackage{amssymb}
\usepackage{autofe}
\usepackage{bbm}
\usepackage[greek,english]{babel}
\usepackage{etex}
\usepackage{framed}
\usepackage[utf8x]{inputenc}
\usepackage{proof}
\usepackage{stmaryrd}
\usepackage{suffix}
\usepackage{textalpha}
\usepackage{todo}
\usepackage{ucs}
\usepackage{comment}
% etc.
%
% please place your own definitions here and don't use \def but
% \newcommand{}{}
\DeclareUnicodeCharacter{8598}{\ensuremath{\nwarrow}}
\DeclareUnicodeCharacter{8599}{\ensuremath{\nearrow}}
\DeclareUnicodeCharacter{8608}{\ensuremath{\twoheadrightarrow}}
\DeclareUnicodeCharacter{8657}{\ensuremath{\Uparrow}}
\DeclareUnicodeCharacter{8667}{\ensuremath{\Rrightarrow}}
\DeclareUnicodeCharacter{8718}{\ensuremath{\qed}}
\DeclareUnicodeCharacter{8759}{\ensuremath{::}}
\DeclareUnicodeCharacter{8988}{\ensuremath{\ulcorner}}
\DeclareUnicodeCharacter{8989}{\ensuremath{\urcorner}}
\DeclareUnicodeCharacter{8803}{\ensuremath{\overline{\equiv}}}
\DeclareUnicodeCharacter{9001}{\ensuremath{\langle}}
\DeclareUnicodeCharacter{9002}{\ensuremath{\rangle}}
\DeclareUnicodeCharacter{9655}{\ensuremath{\rhd}}
\DeclareUnicodeCharacter{9679}{\ensuremath{\bullet}}
\DeclareUnicodeCharacter{10214}{\ensuremath{[}}
\DeclareUnicodeCharacter{10215}{\ensuremath{]}}
\DeclareUnicodeCharacter{10219}{\ensuremath{\rangle\rangle}}
\DeclareUnicodeCharacter{10230}{\ensuremath{\longrightarrow}}

\newtheorem{prop}[theorem]{Proposition}
\newtheorem{lm}[theorem]{Lemma}
\newtheorem{cor}{Corollary}[theorem]

\newcommand*{\Set}{\mathbf{Set}}
\newcommand*{\eqdef}{\mathrel{\smash{\stackrel{\text{def}}{=}}}}
\newcommand*{\isotoid}{\ensuremath{isotoid}}
\newcommand*{\vald}{\ensuremath{\ \mathrm{valid}}}
\newcommand*{\reff}[1]{\ensuremath{\mathsf{ref} \left( {#1} \right)}}
\newcommand*{\univ}[4]{\ensuremath{\mathsf{univ}_{{#1} , {#2}} \left( {#3} , {#4} \right)}}
\newcommand*{\triplelambda}{\ensuremath{\lambda \!\! \lambda \!\! \lambda}}
\newcommand*{\SN}{\ensuremath{\mathbf{SN}}}
\newcommand*{\dom}{\ensuremath{\operatorname{dom}}}
\newcommand*{\sym}[4]{\ensuremath{\mathsf{sym}_{{#1},{#2},{#3}} \left( {#4} \right)}}
\WithSuffix\newcommand\sym*[1]{\ensuremath{\mathsf{sym} \left( {#1} \right)}}
\newcommand*{\trans}[6]{\ensuremath{\mathsf{trans}_{{#1},{#2},{#3},{#4}} \left( {#5} , {#6} \right)}}
\WithSuffix\newcommand\trans*[2]{\ensuremath{\mathsf{trans} \left( {#1} , {#2} \right)}}
\newcommand*{\kr}{\mathop{\rhd \!\!\! \rhd}}


\newcommand{\frametitle}[1]{}

% \newif\ifagda
% \agdatrue
% % Set to true to produce literate Agda document

% \ifagda
% \usepackage{fancyvrb}
% \DefineVerbatimEnvironment{code}{Verbatim}{fontsize=\footnotesize}
% \else
% \excludecomment{code}
% \fi
%
% Insert the name of "your journal" with
% \journalname{myjournal}
\journalname{Journal of Automated Reasoning}
%
\begin{document}

\title{A Formalised Proof of Strong Normalisation for Higher-Order Minimal Logic with Univalence}
%TODO Insert details of CAU grant

%\subtitle{Do you have a subtitle?\\ If so, write it here}

\titlerunning{Higher-Order Minimal Logic with Univalence}        % if too long for running head

\author{Robin Adams \and Marc Bezem \and Thierry Coquand}

%\authorrunning{Short form of author list} % if too long for running head

\institute{Robin Adams \at
              Universitetet i Bergen \\
              Institutt for Informatikk \\
              Postboks 7800 \\
              N-5020 BERGEN \\
              Norway \\
              \email{robin.adams@uib.no}           %  \\
%             \emph{Present address:} of F. Author  %  if needed
           \and
           Marc Bezem \at
              Universitetet i Bergen \\
              Institutt for Informatikk \\
              Postboks 7800 \\
              N-5020 BERGEN \\
              Norway \\
              \email{bezem@ii.uib.no}           %  \\
           \and
           Thierry Coquand \at
              Chalmers tekniska h\"{o}gskola \\
              Data- och informationsteknik \\
              412 96 G\"{o}teborg \\
              Sweden \\
              \email{coquand@chalmers.se}
}

\date{Received: date / Accepted: date}
% The correct dates will be entered by the editor


\maketitle

\begin{abstract}
    Homotopy type theory offers the promise of a formal system for the univalent foundations of mathematics. However, if we simply add the univalence axiom to type theory, then we lose the property of canonicity --- that every term computes to a normal form. A computation becomes `stuck' when it reaches the point that it needs to evaluate a proof term that is an application of the univalence axiom. So we wish to find a way to compute with the univalence axiom.

    As a first step, we present here a system of higher-order propositional logic, with a universe Omega of propositions closed under implication and quantification over any simple type over Omega. We add a type $M =_A N$ for any terms $M$, $N$ of type $A$, and two ways to prove an equality: reflexivity, and the univalence axiom. We present reduction relations for this system, and prove the reduction confluent and strongly normalizing on the well-typed terms. 

This proof has been formalised in Agda.  The formalisation provides a general notion of grammar for a syntax with binding and reduction relation.
\end{abstract}

\section{Introduction}
\label{intro}
The rules of deduction of a type theory are traditionally justified by a \emph{meaning explanation} \cite{ML:ITT}, in which to know that a given term has a given type is to know that it computes to a \emph{canonical object} of that type.  For such a meaning explanation to be possible, the type theory should have the following properties:
\begin{itemize}
\item \textbf{Confluence} --- The reduction relation should be confluent.
\item \textbf{Normalization} --- Every well-typed term should reduce to a normal form.
\item Every closed normal form of type $A$ is a canonical object of type $A$.
\end{itemize}
From these three properties, we have:
\begin{itemize}
\item \textbf{Canonicity} --- Every term of type $A$ reduces to a unique canonical object of type $A$.
\end{itemize}

It is desirable to have, in addition, \emph{strong normalization}, so that we know that we are free to choose whatever reduction strategy we please.

The \emph{univalence axiom} of Homotopy Type theory (HoTT) \cite{hottbook} breaks the property of canonicity.  It postulates a
constant
\[ \isotoid : A \simeq B \rightarrow A = B \]
that is an inverse to the canonical function $A = B \rightarrow A \simeq B$.  When a computation reaches a point
where we eliminate a path (proof of equality) formed by $\isotoid$, it gets 'stuck'.

As possible solutions to this problem, we may try to do with a weaker property than canonicity, such as \emph{propositional canonicity}.
We may attempt to prove that every closed term of type $\mathbb{N}$ is \emph{propositionally} equal to a numeral, as conjectured by Voevodsky.  Or we may attempt to change the definition of equality to make $\isotoid$ definable\cite{Polonsky14a}, or
shift to an entirely different type theory such as Cubical Type Theory\cite{cchm:cubical}.

We could also try a more conservative approach, and simply attempt to find a reduction relation for a type theory involving $\isotoid$ that satisfies
all three of the properties above.  There seems to be no reason \emph{a priori} to believe this is not possible, but it is difficult to do because
the full Homotopy Type Theory is a complex and interdependent system.  We can tackle the problem by adding univalence to a much simpler system, finding
a well-behaved reduction relation, then doing the same for more and more complex systems, gradually approaching the full strength of HoTT.

In this paper, we present a system we call $\lambda o e$, or predicative higher-order minimal logic.  It is a type theory with two universes: the universe $\Omega$
of \emph{propositions}, and the universe of \emph{types}.  The propositions are closed under $\supset$ (implication) and include $\bot$ (falsehood), and an equality proposition $M =_A N$ for
any type $A$ and terms $M : A$ and $N : A$.  The types include $\Omega$ itself and are closed under $\rightarrow$ (non-dependent function type).

There are two canonical forms for proofs of $M =_\Omega N$.  For any term $M : \Omega$, we have $\reff{M} : M =_\Omega M$.  We also add univalence for this system, in this form:
if $\delta : \phi \supset \psi$ and $\epsilon : \psi \supset\phi$, then $\univ{\phi}{\psi}{\delta}{\epsilon} : \phi =_\Omega \psi$.  

We present a reduction relation for this system, and prove it satisfies confluence (Corollary \ref{cor:confluence}), strong normalization (Corollary \ref{cor:SN}) and canonicity (Corollary \ref{cor:canon}).

For the future, we wish to expand the system with universal quantification, and expand it to a 2-dimensional system (with equations between proofs).

\newcommand{\id}[1]{\mathsf{id}_{#1}}

\section{Grammars}

\subsection{Taxonomy}

\AgdaHide{
\begin{code}%
\>\AgdaKeyword{module} \AgdaModule{Grammar.Taxonomy} \AgdaKeyword{where}\<%
\\
\>\AgdaKeyword{open} \AgdaKeyword{import} \AgdaModule{Data.List} \AgdaKeyword{public}\<%
\\
\>\AgdaKeyword{open} \AgdaKeyword{import} \AgdaModule{Prelims}\<%
\end{code}
}

Before we begin investigating the several theories we wish to consider, we present a general theory of syntax and
capture-avoiding substitution.

A \emph{taxononmy} consists of:
\begin{itemize}
\item a set of \emph{expression kinds};
\item a subset of expression kinds, called the \emph{variable kinds}.  We refer to the other expession kinds as \emph{non-variable kinds}.
\end{itemize}

%<*Taxonomy>
\begin{code}%
\>\AgdaKeyword{record} \AgdaRecord{Taxonomy} \AgdaSymbol{:} \AgdaPrimitiveType{Set₁} \AgdaKeyword{where}\<%
\\
\>[0]\AgdaIndent{2}{}\<[2]%
\>[2]\AgdaKeyword{field}\<%
\\
\>[2]\AgdaIndent{4}{}\<[4]%
\>[4]\AgdaField{VarKind} \AgdaSymbol{:} \AgdaPrimitiveType{Set}\<%
\\
\>[2]\AgdaIndent{4}{}\<[4]%
\>[4]\AgdaField{NonVarKind} \AgdaSymbol{:} \AgdaPrimitiveType{Set}\<%
\\
%
\\
\>[0]\AgdaIndent{2}{}\<[2]%
\>[2]\AgdaKeyword{data} \AgdaDatatype{ExpressionKind} \AgdaSymbol{:} \AgdaPrimitiveType{Set} \AgdaKeyword{where}\<%
\\
\>[2]\AgdaIndent{4}{}\<[4]%
\>[4]\AgdaInductiveConstructor{varKind} \AgdaSymbol{:} \AgdaField{VarKind} \AgdaSymbol{→} \AgdaDatatype{ExpressionKind}\<%
\\
\>[2]\AgdaIndent{4}{}\<[4]%
\>[4]\AgdaInductiveConstructor{nonVarKind} \AgdaSymbol{:} \AgdaField{NonVarKind} \AgdaSymbol{→} \AgdaDatatype{ExpressionKind}\<%
\\
\>\AgdaComment{--TODO: Maybe get rid of NonVarKind, and just have parent : VarKind -> Set?}\<%
\end{code}
%</Taxonomy>

\begin{frame}[fragile]
\frametitle{Alphabets}
An \emph{alphabet} $A$ consists of a finite set of \emph{variables},
\mode<article>{to each of which is assigned a variable kind $K$.
Let $\emptyset$ be the empty alphabet, and $(A , K)$ be the result of extending the alphabet $A$ with one
fresh variable $x₀$ of kind $K$.  We write $\mathsf{Var}\ A\ K$ for the set of all variables in $A$ of kind $K$.}
\mode<beamer>{each with a variable kind.}

\begin{code}%
\>[0]\AgdaIndent{2}{}\<[2]%
\>[2]\AgdaKeyword{infixl} \AgdaNumber{55} \AgdaFixityOp{\_,\_}\<%
\\
\>[0]\AgdaIndent{2}{}\<[2]%
\>[2]\AgdaKeyword{data} \AgdaDatatype{Alphabet} \AgdaSymbol{:} \AgdaPrimitiveType{Set} \AgdaKeyword{where}\<%
\\
\>[2]\AgdaIndent{4}{}\<[4]%
\>[4]\AgdaInductiveConstructor{∅} \AgdaSymbol{:} \AgdaDatatype{Alphabet}\<%
\\
\>[2]\AgdaIndent{4}{}\<[4]%
\>[4]\AgdaInductiveConstructor{\_,\_} \AgdaSymbol{:} \AgdaDatatype{Alphabet} \AgdaSymbol{→} \AgdaField{VarKind} \AgdaSymbol{→} \AgdaDatatype{Alphabet}\<%
\end{code}

\AgdaHide{
\begin{code}%
\>[0]\AgdaIndent{2}{}\<[2]%
\>[2]\AgdaFunction{extend} \AgdaSymbol{:} \AgdaDatatype{Alphabet} \AgdaSymbol{→} \AgdaDatatype{List} \AgdaField{VarKind} \AgdaSymbol{→} \AgdaDatatype{Alphabet}\<%
\\
\>[0]\AgdaIndent{2}{}\<[2]%
\>[2]\AgdaFunction{extend} \AgdaBound{A} \AgdaInductiveConstructor{[]} \AgdaSymbol{=} \AgdaBound{A}\<%
\\
\>[0]\AgdaIndent{2}{}\<[2]%
\>[2]\AgdaFunction{extend} \AgdaBound{A} \AgdaSymbol{(}\AgdaBound{K} \AgdaInductiveConstructor{∷} \AgdaBound{KK}\AgdaSymbol{)} \AgdaSymbol{=} \AgdaFunction{extend} \AgdaSymbol{(}\AgdaBound{A} \AgdaInductiveConstructor{,} \AgdaBound{K}\AgdaSymbol{)} \AgdaBound{KK}\<%
\\
%
\\
\>[0]\AgdaIndent{2}{}\<[2]%
\>[2]\AgdaFunction{snoc-extend} \AgdaSymbol{:} \AgdaDatatype{Alphabet} \AgdaSymbol{→} \AgdaDatatype{snocList} \AgdaField{VarKind} \AgdaSymbol{→} \AgdaDatatype{Alphabet}\<%
\\
\>[0]\AgdaIndent{2}{}\<[2]%
\>[2]\AgdaFunction{snoc-extend} \AgdaBound{A} \AgdaInductiveConstructor{[]} \AgdaSymbol{=} \AgdaBound{A}\<%
\\
\>[0]\AgdaIndent{2}{}\<[2]%
\>[2]\AgdaFunction{snoc-extend} \AgdaBound{A} \AgdaSymbol{(}\AgdaBound{KK} \AgdaInductiveConstructor{snoc} \AgdaBound{K}\AgdaSymbol{)} \AgdaSymbol{=} \AgdaFunction{snoc-extend} \AgdaBound{A} \AgdaBound{KK} \AgdaInductiveConstructor{,} \AgdaBound{K}\<%
\end{code}
}

\begin{code}%
\>[0]\AgdaIndent{2}{}\<[2]%
\>[2]\AgdaKeyword{data} \AgdaDatatype{Var} \AgdaSymbol{:} \AgdaDatatype{Alphabet} \AgdaSymbol{→} \AgdaField{VarKind} \AgdaSymbol{→} \AgdaPrimitiveType{Set} \AgdaKeyword{where}\<%
\\
\>[2]\AgdaIndent{4}{}\<[4]%
\>[4]\AgdaInductiveConstructor{x₀} \AgdaSymbol{:} \AgdaSymbol{∀} \AgdaSymbol{\{}\AgdaBound{V}\AgdaSymbol{\}} \AgdaSymbol{\{}\AgdaBound{K}\AgdaSymbol{\}} \AgdaSymbol{→} \AgdaDatatype{Var} \AgdaSymbol{(}\AgdaBound{V} \AgdaInductiveConstructor{,} \AgdaBound{K}\AgdaSymbol{)} \AgdaBound{K}\<%
\\
\>[2]\AgdaIndent{4}{}\<[4]%
\>[4]\AgdaInductiveConstructor{↑} \AgdaSymbol{:} \AgdaSymbol{∀} \AgdaSymbol{\{}\AgdaBound{V}\AgdaSymbol{\}} \AgdaSymbol{\{}\AgdaBound{K}\AgdaSymbol{\}} \AgdaSymbol{\{}\AgdaBound{L}\AgdaSymbol{\}} \AgdaSymbol{→} \AgdaDatatype{Var} \AgdaBound{V} \AgdaBound{L} \AgdaSymbol{→} \AgdaDatatype{Var} \AgdaSymbol{(}\AgdaBound{V} \AgdaInductiveConstructor{,} \AgdaBound{K}\AgdaSymbol{)} \AgdaBound{L}\<%
\end{code}

\AgdaHide{
\begin{code}%
\>[0]\AgdaIndent{2}{}\<[2]%
\>[2]\AgdaFunction{x₁} \AgdaSymbol{:} \AgdaSymbol{∀} \AgdaSymbol{\{}\AgdaBound{V}\AgdaSymbol{\}} \AgdaSymbol{\{}\AgdaBound{K}\AgdaSymbol{\}} \AgdaSymbol{\{}\AgdaBound{L}\AgdaSymbol{\}} \AgdaSymbol{→} \AgdaDatatype{Var} \AgdaSymbol{(}\AgdaBound{V} \AgdaInductiveConstructor{,} \AgdaBound{K} \AgdaInductiveConstructor{,} \AgdaBound{L}\AgdaSymbol{)} \AgdaBound{K}\<%
\end{code}
}

\begin{code}%
\>[0]\AgdaIndent{2}{}\<[2]%
\>[2]\AgdaFunction{x₁} \AgdaSymbol{=} \AgdaInductiveConstructor{↑} \AgdaInductiveConstructor{x₀}\<%
\end{code}

\AgdaHide{
\begin{code}%
\>[0]\AgdaIndent{2}{}\<[2]%
\>[2]\AgdaFunction{x₂} \AgdaSymbol{:} \AgdaSymbol{∀} \AgdaSymbol{\{}\AgdaBound{V}\AgdaSymbol{\}} \AgdaSymbol{\{}\AgdaBound{K}\AgdaSymbol{\}} \AgdaSymbol{\{}\AgdaBound{L}\AgdaSymbol{\}} \AgdaSymbol{\{}\AgdaBound{L'}\AgdaSymbol{\}} \AgdaSymbol{→} \AgdaDatatype{Var} \AgdaSymbol{(}\AgdaBound{V} \AgdaInductiveConstructor{,} \AgdaBound{K} \AgdaInductiveConstructor{,} \AgdaBound{L} \AgdaInductiveConstructor{,} \AgdaBound{L'}\AgdaSymbol{)} \AgdaBound{K}\<%
\end{code}
}

\begin{code}%
\>[0]\AgdaIndent{2}{}\<[2]%
\>[2]\AgdaFunction{x₂} \AgdaSymbol{=} \AgdaInductiveConstructor{↑} \AgdaFunction{x₁}\<%
\end{code}
\end{frame}

A constructor $c$ of kind (\ref{eq:conkind}) is a constructor that takes $m$ arguments of kind $B_1$, \ldots, $B_m$, and binds $r_i$ variables in its $i$th argument of kind $A_{ij}$,
producing an expression of kind $C$.  We write this expression as

\begin{equation}
\label{eq:expression}
c([x_{11}, \ldots, x_{1r_1}]E_1, \ldots, [x_{m1}, \ldots, x_{mr_m}]E_m) \enspace .
\end{equation}

The subexpressions of the form $[x_{i1}, \ldots, x_{ir_i}]E_i$ shall be called \emph{abstractions}.

When giving a specific grammar, we shall feel free to use BNF notation.  

We formalise this as follows.  First, we construct the sets of expression kinds and constructor kinds over a taxonomy:

\begin{frame}[fragile]
There are two \emph{classes} of kinds: expression kinds and constructor kinds.

\begin{code}%
\>[0]\AgdaIndent{2}{}\<[2]%
\>[2]\AgdaKeyword{infix} \AgdaNumber{22} \AgdaFixityOp{\_✧}\<%
\\
\>[0]\AgdaIndent{2}{}\<[2]%
\>[2]\AgdaKeyword{infixr} \AgdaNumber{21} \AgdaFixityOp{\_abs\_}\<%
\\
\>[0]\AgdaIndent{2}{}\<[2]%
\>[2]\AgdaKeyword{data} \AgdaDatatype{AbstractionKind} \AgdaSymbol{:} \AgdaDatatype{ExpressionKind} \AgdaSymbol{→} \AgdaPrimitiveType{Set} \AgdaKeyword{where}\<%
\\
\>[2]\AgdaIndent{4}{}\<[4]%
\>[4]\AgdaInductiveConstructor{\_✧} \AgdaSymbol{:} \AgdaSymbol{∀} \AgdaBound{K} \AgdaSymbol{→} \AgdaDatatype{AbstractionKind} \AgdaBound{K}\<%
\\
\>[2]\AgdaIndent{4}{}\<[4]%
\>[4]\AgdaInductiveConstructor{\_abs\_} \AgdaSymbol{:} \AgdaSymbol{∀} \AgdaSymbol{\{}\AgdaBound{K}\AgdaSymbol{\}} \AgdaSymbol{→} \AgdaField{VarKind} \AgdaSymbol{→} \AgdaDatatype{AbstractionKind} \AgdaBound{K} \AgdaSymbol{→} \AgdaDatatype{AbstractionKind} \AgdaBound{K}\<%
\\
%
\\
\>[0]\AgdaIndent{2}{}\<[2]%
\>[2]\AgdaKeyword{infix} \AgdaNumber{22} \AgdaFixityOp{\_●}\<%
\\
\>[0]\AgdaIndent{2}{}\<[2]%
\>[2]\AgdaKeyword{infixr} \AgdaNumber{20} \AgdaFixityOp{\_⟶\_}\<%
\\
\>[0]\AgdaIndent{2}{}\<[2]%
\>[2]\AgdaKeyword{data} \AgdaDatatype{ConstructorKind} \AgdaSymbol{:} \AgdaDatatype{ExpressionKind} \AgdaSymbol{→} \AgdaPrimitiveType{Set} \AgdaKeyword{where}\<%
\\
\>[2]\AgdaIndent{4}{}\<[4]%
\>[4]\AgdaInductiveConstructor{\_●} \AgdaSymbol{:} \AgdaSymbol{∀} \AgdaBound{K} \AgdaSymbol{→} \AgdaDatatype{ConstructorKind} \AgdaBound{K}\<%
\\
\>[2]\AgdaIndent{4}{}\<[4]%
\>[4]\AgdaInductiveConstructor{\_⟶\_} \AgdaSymbol{:} \AgdaSymbol{∀} \AgdaSymbol{\{}\AgdaBound{L}\AgdaSymbol{\}} \AgdaSymbol{\{}\AgdaBound{K}\AgdaSymbol{\}} \AgdaSymbol{→} \AgdaDatatype{AbstractionKind} \AgdaBound{L} \AgdaSymbol{→} \AgdaDatatype{ConstructorKind} \AgdaBound{K} \AgdaSymbol{→} \AgdaDatatype{ConstructorKind} \AgdaBound{K}\<%
\\
%
\\
\>[0]\AgdaIndent{2}{}\<[2]%
\>[2]\AgdaKeyword{data} \AgdaDatatype{KindClass} \AgdaSymbol{:} \AgdaPrimitiveType{Set} \AgdaKeyword{where}\<%
\\
\>[2]\AgdaIndent{4}{}\<[4]%
\>[4]\AgdaInductiveConstructor{-Expression} \AgdaSymbol{:} \AgdaDatatype{KindClass}\<%
\\
\>[2]\AgdaIndent{4}{}\<[4]%
\>[4]\AgdaInductiveConstructor{-Constructor} \AgdaSymbol{:} \AgdaDatatype{ExpressionKind} \AgdaSymbol{→} \AgdaDatatype{KindClass}\<%
\\
%
\\
\>[0]\AgdaIndent{2}{}\<[2]%
\>[2]\AgdaFunction{Kind} \AgdaSymbol{:} \AgdaDatatype{KindClass} \AgdaSymbol{→} \AgdaPrimitiveType{Set}\<%
\\
\>[0]\AgdaIndent{2}{}\<[2]%
\>[2]\AgdaFunction{Kind} \AgdaInductiveConstructor{-Expression} \AgdaSymbol{=} \AgdaDatatype{ExpressionKind}\<%
\\
\>[0]\AgdaIndent{2}{}\<[2]%
\>[2]\AgdaFunction{Kind} \AgdaSymbol{(}\AgdaInductiveConstructor{-Constructor} \AgdaBound{K}\AgdaSymbol{)} \AgdaSymbol{=} \AgdaDatatype{ConstructorKind} \AgdaBound{K}\<%
\end{code}
\end{frame}
%TODO Colours in Agda code?

\AgdaHide{
\begin{code}%
\>\AgdaComment{\{- Metavariable conventions:\<\\
\>  A, B    range over abstraction kinds\<\\
\>  C       range over kind classes\<\\
\>  AA, BB  range over lists of abstraction kinds\<\\
\>  E, F, G range over subexpressions\<\\
\>  K, L    range over expression kinds including variable kinds\<\\
\>  M, N, P range over expressions\<\\
\>  U, V, W range over alphabets -\}}\<%
\\
\>\AgdaKeyword{open} \AgdaKeyword{import} \AgdaModule{Function}\<%
\\
\>\AgdaKeyword{open} \AgdaKeyword{import} \AgdaModule{Data.List}\<%
\\
\>\AgdaKeyword{open} \AgdaKeyword{import} \AgdaModule{Prelims}\<%
\\
\>\AgdaKeyword{open} \AgdaKeyword{import} \AgdaModule{Grammar.Taxonomy}\<%
\\
%
\\
\>\AgdaKeyword{module} \AgdaModule{Grammar.Base} \AgdaKeyword{where}\<%
\\
%
\\
\>\AgdaKeyword{record} \AgdaRecord{IsGrammar} \AgdaSymbol{(}\AgdaBound{T} \AgdaSymbol{:} \AgdaRecord{Taxonomy}\AgdaSymbol{)} \AgdaSymbol{:} \AgdaPrimitiveType{Set₁} \AgdaKeyword{where}\<%
\\
\>[0]\AgdaIndent{2}{}\<[2]%
\>[2]\AgdaKeyword{open} \AgdaModule{Taxonomy} \AgdaBound{T}\<%
\\
\>[0]\AgdaIndent{2}{}\<[2]%
\>[2]\AgdaKeyword{field}\<%
\\
\>[2]\AgdaIndent{4}{}\<[4]%
\>[4]\AgdaField{Constructor} \<[19]%
\>[19]\AgdaSymbol{:} \AgdaFunction{ConKind} \AgdaSymbol{→} \AgdaPrimitiveType{Set}\<%
\\
\>[2]\AgdaIndent{4}{}\<[4]%
\>[4]\AgdaField{parent} \<[19]%
\>[19]\AgdaSymbol{:} \AgdaFunction{VarKind} \AgdaSymbol{→} \AgdaDatatype{ExpKind}\<%
\\
%
\\
\>\AgdaKeyword{record} \AgdaRecord{Grammar} \AgdaSymbol{:} \AgdaPrimitiveType{Set₁} \AgdaKeyword{where}\<%
\\
\>[0]\AgdaIndent{2}{}\<[2]%
\>[2]\AgdaKeyword{field}\<%
\\
\>[2]\AgdaIndent{4}{}\<[4]%
\>[4]\AgdaField{taxonomy} \AgdaSymbol{:} \AgdaRecord{Taxonomy}\<%
\\
\>[2]\AgdaIndent{4}{}\<[4]%
\>[4]\AgdaField{isGrammar} \AgdaSymbol{:} \AgdaRecord{IsGrammar} \AgdaField{taxonomy}\<%
\\
\>[0]\AgdaIndent{2}{}\<[2]%
\>[2]\AgdaKeyword{open} \AgdaModule{Taxonomy} \AgdaField{taxonomy} \AgdaKeyword{public}\<%
\\
\>[0]\AgdaIndent{2}{}\<[2]%
\>[2]\AgdaKeyword{open} \AgdaModule{IsGrammar} \AgdaField{isGrammar} \AgdaKeyword{public}\<%
\end{code}
}

%<*Expression>
\begin{code}%
\>[0]\AgdaIndent{2}{}\<[2]%
\>[2]\AgdaKeyword{data} \AgdaDatatype{Subexpression} \AgdaSymbol{(}\AgdaBound{V} \AgdaSymbol{:} \AgdaDatatype{Alphabet}\AgdaSymbol{)} \AgdaSymbol{:} \AgdaSymbol{∀} \AgdaBound{C} \AgdaSymbol{→} \AgdaFunction{Kind} \AgdaBound{C} \AgdaSymbol{→} \AgdaPrimitiveType{Set}\<%
\\
\>[0]\AgdaIndent{2}{}\<[2]%
\>[2]\AgdaFunction{Expression} \AgdaSymbol{:} \AgdaDatatype{Alphabet} \AgdaSymbol{→} \AgdaDatatype{ExpKind} \AgdaSymbol{→} \AgdaPrimitiveType{Set}\<%
\\
\>[0]\AgdaIndent{2}{}\<[2]%
\>[2]\AgdaFunction{VExpression} \AgdaSymbol{:} \AgdaDatatype{Alphabet} \AgdaSymbol{→} \AgdaFunction{VarKind} \AgdaSymbol{→} \AgdaPrimitiveType{Set}\<%
\\
\>[0]\AgdaIndent{2}{}\<[2]%
\>[2]\AgdaFunction{Abstraction} \AgdaSymbol{:} \AgdaDatatype{Alphabet} \AgdaSymbol{→} \AgdaFunction{AbsKind} \AgdaSymbol{→} \AgdaPrimitiveType{Set}\<%
\\
\>[0]\AgdaIndent{2}{}\<[2]%
\>[2]\AgdaFunction{ListAbs} \AgdaSymbol{:} \AgdaDatatype{Alphabet} \AgdaSymbol{→} \AgdaDatatype{List} \AgdaFunction{AbsKind} \AgdaSymbol{→} \AgdaPrimitiveType{Set}\<%
\\
%
\\
\>[0]\AgdaIndent{2}{}\<[2]%
\>[2]\AgdaFunction{Expression} \AgdaBound{V} \AgdaBound{K} \AgdaSymbol{=} \AgdaDatatype{Subexpression} \AgdaBound{V} \AgdaInductiveConstructor{-Expression} \AgdaBound{K}\<%
\\
\>[0]\AgdaIndent{2}{}\<[2]%
\>[2]\AgdaFunction{VExpression} \AgdaBound{V} \AgdaBound{K} \AgdaSymbol{=} \AgdaFunction{Expression} \AgdaBound{V} \AgdaSymbol{(}\AgdaInductiveConstructor{varKind} \AgdaBound{K}\AgdaSymbol{)}\<%
\\
\>[0]\AgdaIndent{2}{}\<[2]%
\>[2]\AgdaFunction{Abstraction} \AgdaBound{V} \AgdaSymbol{(}\AgdaInductiveConstructor{SK} \AgdaBound{AA} \AgdaBound{K}\AgdaSymbol{)} \AgdaSymbol{=} \AgdaFunction{Expression} \AgdaSymbol{(}\AgdaFunction{extend} \AgdaBound{V} \AgdaBound{AA}\AgdaSymbol{)} \AgdaBound{K}\<%
\\
\>[0]\AgdaIndent{2}{}\<[2]%
\>[2]\AgdaFunction{ListAbs} \AgdaBound{V} \AgdaBound{AA} \AgdaSymbol{=} \AgdaDatatype{Subexpression} \AgdaBound{V} \AgdaInductiveConstructor{-ListAbs} \AgdaBound{AA}\<%
\\
%
\\
\>[0]\AgdaIndent{2}{}\<[2]%
\>[2]\AgdaKeyword{infixr} \AgdaNumber{5} \AgdaFixityOp{\_∷\_}\<%
\\
\>[0]\AgdaIndent{2}{}\<[2]%
\>[2]\AgdaKeyword{data} \AgdaDatatype{Subexpression} \AgdaBound{V} \AgdaKeyword{where}\<%
\\
\>[2]\AgdaIndent{4}{}\<[4]%
\>[4]\AgdaInductiveConstructor{var} \AgdaSymbol{:} \AgdaSymbol{∀} \AgdaSymbol{\{}\AgdaBound{K}\AgdaSymbol{\}} \AgdaSymbol{→} \AgdaDatatype{Var} \AgdaBound{V} \AgdaBound{K} \AgdaSymbol{→} \AgdaFunction{VExpression} \AgdaBound{V} \AgdaBound{K}\<%
\\
\>[2]\AgdaIndent{4}{}\<[4]%
\>[4]\AgdaInductiveConstructor{app} \AgdaSymbol{:} \AgdaSymbol{∀} \AgdaSymbol{\{}\AgdaBound{AA}\AgdaSymbol{\}} \AgdaSymbol{\{}\AgdaBound{K}\AgdaSymbol{\}} \AgdaSymbol{→} \AgdaFunction{Constructor} \AgdaSymbol{(}\AgdaInductiveConstructor{SK} \AgdaBound{AA} \AgdaBound{K}\AgdaSymbol{)} \AgdaSymbol{→} \AgdaFunction{ListAbs} \AgdaBound{V} \AgdaBound{AA} \AgdaSymbol{→} \AgdaFunction{Expression} \AgdaBound{V} \AgdaBound{K}\<%
\\
\>[2]\AgdaIndent{4}{}\<[4]%
\>[4]\AgdaInductiveConstructor{[]} \AgdaSymbol{:} \AgdaFunction{ListAbs} \AgdaBound{V} \AgdaInductiveConstructor{[]}\<%
\\
\>[2]\AgdaIndent{4}{}\<[4]%
\>[4]\AgdaInductiveConstructor{\_∷\_} \AgdaSymbol{:} \AgdaSymbol{∀} \AgdaSymbol{\{}\AgdaBound{A}\AgdaSymbol{\}} \AgdaSymbol{\{}\AgdaBound{AA}\AgdaSymbol{\}} \AgdaSymbol{→} \AgdaFunction{Abstraction} \AgdaBound{V} \AgdaBound{A} \AgdaSymbol{→} \AgdaFunction{ListAbs} \AgdaBound{V} \AgdaBound{AA} \AgdaSymbol{→} \AgdaFunction{ListAbs} \AgdaBound{V} \AgdaSymbol{(}\AgdaBound{A} \AgdaInductiveConstructor{∷} \AgdaBound{AA}\AgdaSymbol{)}\<%
\end{code}
%</Expression>

We prove that the constructor \AgdaRef{var} is injective.

\begin{code}%
\>[0]\AgdaIndent{2}{}\<[2]%
\>[2]\AgdaFunction{var-inj} \AgdaSymbol{:} \AgdaSymbol{∀} \AgdaSymbol{\{}\AgdaBound{V}\AgdaSymbol{\}} \AgdaSymbol{\{}\AgdaBound{K}\AgdaSymbol{\}} \AgdaSymbol{\{}\AgdaBound{x} \AgdaBound{y} \AgdaSymbol{:} \AgdaDatatype{Var} \AgdaBound{V} \AgdaBound{K}\AgdaSymbol{\}} \AgdaSymbol{→} \AgdaInductiveConstructor{var} \AgdaBound{x} \AgdaDatatype{≡} \AgdaInductiveConstructor{var} \AgdaBound{y} \AgdaSymbol{→} \AgdaBound{x} \AgdaDatatype{≡} \AgdaBound{y}\<%
\\
\>[0]\AgdaIndent{2}{}\<[2]%
\>[2]\AgdaFunction{var-inj} \AgdaInductiveConstructor{refl} \AgdaSymbol{=} \AgdaInductiveConstructor{refl}\<%
\end{code}

For the future, we also define the type of all snoc-lists of expressions $(M_1, \ldots, M_n)$
such that $M_i$ is of type $K_i$, given a snoc-list of variable kinds $(K_1, \ldots, K_n)$.

\begin{code}%
\>[0]\AgdaIndent{2}{}\<[2]%
\>[2]\AgdaKeyword{infixl} \AgdaNumber{20} \AgdaFixityOp{\_snoc\_}\<%
\\
\>[0]\AgdaIndent{2}{}\<[2]%
\>[2]\AgdaKeyword{data} \AgdaDatatype{snocListExp} \AgdaBound{V} \AgdaSymbol{:} \AgdaDatatype{snocList} \AgdaFunction{VarKind} \AgdaSymbol{→} \AgdaPrimitiveType{Set} \AgdaKeyword{where}\<%
\\
\>[2]\AgdaIndent{4}{}\<[4]%
\>[4]\AgdaInductiveConstructor{[]} \AgdaSymbol{:} \AgdaDatatype{snocListExp} \AgdaBound{V} \AgdaInductiveConstructor{[]}\<%
\\
\>[2]\AgdaIndent{4}{}\<[4]%
\>[4]\AgdaInductiveConstructor{\_snoc\_} \AgdaSymbol{:} \AgdaSymbol{∀} \AgdaSymbol{\{}\AgdaBound{A}\AgdaSymbol{\}} \AgdaSymbol{\{}\AgdaBound{K}\AgdaSymbol{\}} \AgdaSymbol{→} \AgdaDatatype{snocListExp} \AgdaBound{V} \AgdaBound{A} \AgdaSymbol{→} \AgdaFunction{Expression} \AgdaBound{V} \AgdaSymbol{(}\AgdaInductiveConstructor{varKind} \AgdaBound{K}\AgdaSymbol{)} \AgdaSymbol{→} \AgdaDatatype{snocListExp} \AgdaBound{V} \AgdaSymbol{(}\AgdaBound{A} \AgdaInductiveConstructor{snoc} \AgdaBound{K}\AgdaSymbol{)}\<%
\end{code}

A \emph{reduction} is a relation $\rhd$ between expressions such that, if $E \rhd F$,
then $E$ is not a variable.  It is given by a term $R : \AgdaRef{Reduction}$ such that
$R\, c\, MM\, N$ iff $c[MM] \rhd N$.

\begin{code}%
\>[0]\AgdaIndent{2}{}\<[2]%
\>[2]\AgdaFunction{Reduction} \AgdaSymbol{:} \AgdaPrimitiveType{Set₁}\<%
\\
\>[0]\AgdaIndent{2}{}\<[2]%
\>[2]\AgdaFunction{Reduction} \AgdaSymbol{=} \AgdaSymbol{∀} \AgdaSymbol{\{}\AgdaBound{V}\AgdaSymbol{\}} \AgdaSymbol{\{}\AgdaBound{AA}\AgdaSymbol{\}} \AgdaSymbol{\{}\AgdaBound{K}\AgdaSymbol{\}} \AgdaSymbol{→} \AgdaFunction{Constructor} \AgdaSymbol{(}\AgdaInductiveConstructor{SK} \AgdaBound{AA} \AgdaBound{K}\AgdaSymbol{)} \AgdaSymbol{→} \AgdaFunction{ListAbs} \AgdaBound{V} \AgdaBound{AA} \AgdaSymbol{→} \AgdaFunction{Expression} \AgdaBound{V} \AgdaBound{K} \AgdaSymbol{→} \AgdaPrimitiveType{Set}\<%
\end{code}
}

\AgdaHide{
\begin{code}%
\>\AgdaKeyword{open} \AgdaKeyword{import} \AgdaModule{Grammar.Base}\<%
\\
%
\\
\>\AgdaKeyword{module} \AgdaModule{Grammar.OpFamily.PreOpFamily} \AgdaSymbol{(}\AgdaBound{G} \AgdaSymbol{:} \AgdaRecord{Grammar}\AgdaSymbol{)} \AgdaKeyword{where}\<%
\\
\>\AgdaKeyword{open} \AgdaKeyword{import} \AgdaModule{Prelims}\<%
\\
\>\AgdaKeyword{open} \AgdaModule{Grammar} \AgdaBound{G}\<%
\end{code}
}

\subsection{Families of Operations}

Our aim here is to define the operations of \emph{replacement} and \emph{substitution}.  In order to organise this work, we introduce the following definitions.

A \emph{family of operations} over a grammar $G$ consists of:
\begin{enumerate}
\item
for any alphabets $U$ and $V$, a set $F[U,V]$ of \emph{operations} $\sigma$ from $U$ to $V$, $\sigma : U \rightarrow V$;
\item
for any operation $\sigma : U \rightarrow V$ and variable $x \in U$ of kind $K$, an expression $\sigma(x)$ over $V$ of kind $K$;
\item
for any alphabet $V$ and variable kind $K$, an operation $\uparrow : V \rightarrow (V , K)$, the \emph{lifting} operation;
\item
for any alphabet $V$, an operation $\id{V} : V \rightarrow V$, the \emph{identity} operation;
\item
for any operation $\sigma : U \rightarrow V$ and variable kind $K$, an operation $(\sigma , K) : (U , K) \rightarrow (V , K)$, the result of \emph{lifting} $\sigma$;
\item
for any operations $\rho : U \rightarrow V$ and $\sigma : V \rightarrow W$, an operation
$\sigma \circ \rho : U \rightarrow W$, the \emph{composition} of $\sigma$ and $\rho$;
\end{enumerate}
such that:
\begin{itemize}
\item
$\uparrow (x) \equiv x$
\item
$\id{V}(x) \equiv x$
\item
If $\rho \sim \sigma$ then $(\rho , K) \sim (\sigma , K)$
\item
$(\rho , K)(x_0) \equiv x_0$
\item
Given $\sigma : U \rightarrow V$ and $x \in U$, we have $(\sigma , K)(x) \equiv x$
\item
$(\sigma \circ \rho , K) \sim (\sigma , K) \circ (\rho , K)$
\item
$(\sigma \circ \rho)(x) \equiv \rho(x) [ \sigma ]$
\end{itemize}
where for $\sigma, \rho : U \rightarrow V$ we write $\sigma \sim \rho$ iff $\sigma(x) \equiv \rho(x)$ for all $x \in U$; and, given $\sigma : U \rightarrow V$ and $E$ an expression over $U$, we define $E[\sigma]$, the result of \emph{applying} the operation $\sigma$ to $E$, as follows:

\begin{align*}
x[\sigma] & \eqdef \sigma(x) \\
\lefteqn{c([\vec{x_1}] E_1, \ldots, [\vec{x_n}] E_n) [\sigma]} \\
 & \eqdef
c([\vec{x_1}] E_1 [(\sigma , K_{11}, \ldots, K_{1r_1})], \ldots,
[\vec{x_n}] E_n [(\sigma, K_{n1}, \ldots, K_{nr_n})])
\end{align*}
for $c$ a constructor of type (\ref{eq:conkind}).

\subsubsection{Pre-Families}
We formalize this definition in stages.  First, we define a \emph{pre-family of operations} to be a family with items of data 1--4 above that satisfies the first two axioms:

\begin{code}%
\>\AgdaKeyword{record} \AgdaRecord{PreOpFamily} \AgdaSymbol{:} \AgdaPrimitiveType{Set₂} \AgdaKeyword{where}\<%
\\
\>[0]\AgdaIndent{2}{}\<[2]%
\>[2]\AgdaKeyword{field}\<%
\\
\>[2]\AgdaIndent{4}{}\<[4]%
\>[4]\AgdaField{Op} \AgdaSymbol{:} \AgdaDatatype{Alphabet} \AgdaSymbol{→} \AgdaDatatype{Alphabet} \AgdaSymbol{→} \AgdaPrimitiveType{Set}\<%
\\
\>[2]\AgdaIndent{4}{}\<[4]%
\>[4]\AgdaField{apV} \AgdaSymbol{:} \AgdaSymbol{∀} \AgdaSymbol{\{}\AgdaBound{U}\AgdaSymbol{\}} \AgdaSymbol{\{}\AgdaBound{V}\AgdaSymbol{\}} \AgdaSymbol{\{}\AgdaBound{K}\AgdaSymbol{\}} \AgdaSymbol{→} \AgdaField{Op} \AgdaBound{U} \AgdaBound{V} \AgdaSymbol{→} \AgdaDatatype{Var} \AgdaBound{U} \AgdaBound{K} \AgdaSymbol{→} \AgdaFunction{Expression} \AgdaBound{V} \AgdaSymbol{(}\AgdaInductiveConstructor{varKind} \AgdaBound{K}\AgdaSymbol{)}\<%
\\
\>[2]\AgdaIndent{4}{}\<[4]%
\>[4]\AgdaField{up} \AgdaSymbol{:} \AgdaSymbol{∀} \AgdaSymbol{\{}\AgdaBound{V}\AgdaSymbol{\}} \AgdaSymbol{\{}\AgdaBound{K}\AgdaSymbol{\}} \AgdaSymbol{→} \AgdaField{Op} \AgdaBound{V} \AgdaSymbol{(}\AgdaBound{V} \AgdaInductiveConstructor{,} \AgdaBound{K}\AgdaSymbol{)}\<%
\\
\>[2]\AgdaIndent{4}{}\<[4]%
\>[4]\AgdaField{apV-up} \AgdaSymbol{:} \AgdaSymbol{∀} \AgdaSymbol{\{}\AgdaBound{V}\AgdaSymbol{\}} \AgdaSymbol{\{}\AgdaBound{K}\AgdaSymbol{\}} \AgdaSymbol{\{}\AgdaBound{L}\AgdaSymbol{\}} \AgdaSymbol{\{}\AgdaBound{x} \AgdaSymbol{:} \AgdaDatatype{Var} \AgdaBound{V} \AgdaBound{K}\AgdaSymbol{\}} \AgdaSymbol{→} \AgdaField{apV} \AgdaSymbol{(}\AgdaField{up} \AgdaSymbol{\{}\AgdaArgument{K} \AgdaSymbol{=} \AgdaBound{L}\AgdaSymbol{\})} \AgdaBound{x} \AgdaDatatype{≡} \AgdaInductiveConstructor{var} \AgdaSymbol{(}\AgdaInductiveConstructor{↑} \AgdaBound{x}\AgdaSymbol{)}\<%
\\
\>[2]\AgdaIndent{4}{}\<[4]%
\>[4]\AgdaField{idOp} \AgdaSymbol{:} \AgdaSymbol{∀} \AgdaBound{V} \AgdaSymbol{→} \AgdaField{Op} \AgdaBound{V} \AgdaBound{V}\<%
\\
\>[2]\AgdaIndent{4}{}\<[4]%
\>[4]\AgdaField{apV-idOp} \AgdaSymbol{:} \AgdaSymbol{∀} \AgdaSymbol{\{}\AgdaBound{V}\AgdaSymbol{\}} \AgdaSymbol{\{}\AgdaBound{K}\AgdaSymbol{\}} \AgdaSymbol{(}\AgdaBound{x} \AgdaSymbol{:} \AgdaDatatype{Var} \AgdaBound{V} \AgdaBound{K}\AgdaSymbol{)} \AgdaSymbol{→} \AgdaField{apV} \AgdaSymbol{(}\AgdaField{idOp} \AgdaBound{V}\AgdaSymbol{)} \AgdaBound{x} \AgdaDatatype{≡} \AgdaInductiveConstructor{var} \AgdaBound{x}\<%
\end{code}

This allows us to define the relation $\sim$, and prove it is an equivalence relation:

\begin{code}%
\>[0]\AgdaIndent{2}{}\<[2]%
\>[2]\AgdaFunction{\_∼op\_} \AgdaSymbol{:} \AgdaSymbol{∀} \AgdaSymbol{\{}\AgdaBound{U}\AgdaSymbol{\}} \AgdaSymbol{\{}\AgdaBound{V}\AgdaSymbol{\}} \AgdaSymbol{→} \AgdaField{Op} \AgdaBound{U} \AgdaBound{V} \AgdaSymbol{→} \AgdaField{Op} \AgdaBound{U} \AgdaBound{V} \AgdaSymbol{→} \AgdaPrimitiveType{Set}\<%
\\
\>[0]\AgdaIndent{2}{}\<[2]%
\>[2]\AgdaFunction{\_∼op\_} \AgdaSymbol{\{}\AgdaBound{U}\AgdaSymbol{\}} \AgdaSymbol{\{}\AgdaBound{V}\AgdaSymbol{\}} \AgdaBound{ρ} \AgdaBound{σ} \AgdaSymbol{=} \AgdaSymbol{∀} \AgdaSymbol{\{}\AgdaBound{K}\AgdaSymbol{\}} \AgdaSymbol{(}\AgdaBound{x} \AgdaSymbol{:} \AgdaDatatype{Var} \AgdaBound{U} \AgdaBound{K}\AgdaSymbol{)} \AgdaSymbol{→} \AgdaField{apV} \AgdaBound{ρ} \AgdaBound{x} \AgdaDatatype{≡} \AgdaField{apV} \AgdaBound{σ} \AgdaBound{x}\<%
\\
\>[2]\AgdaIndent{4}{}\<[4]%
\>[4]\<%
\\
\>[0]\AgdaIndent{2}{}\<[2]%
\>[2]\AgdaFunction{∼-refl} \AgdaSymbol{:} \AgdaSymbol{∀} \AgdaSymbol{\{}\AgdaBound{U}\AgdaSymbol{\}} \AgdaSymbol{\{}\AgdaBound{V}\AgdaSymbol{\}} \AgdaSymbol{\{}\AgdaBound{σ} \AgdaSymbol{:} \AgdaField{Op} \AgdaBound{U} \AgdaBound{V}\AgdaSymbol{\}} \AgdaSymbol{→} \AgdaBound{σ} \AgdaFunction{∼op} \AgdaBound{σ}\<%
\\
\>[0]\AgdaIndent{2}{}\<[2]%
\>[2]\AgdaFunction{∼-refl} \AgdaSymbol{\_} \AgdaSymbol{=} \AgdaInductiveConstructor{refl}\<%
\\
\>[2]\AgdaIndent{4}{}\<[4]%
\>[4]\<%
\\
\>[0]\AgdaIndent{2}{}\<[2]%
\>[2]\AgdaFunction{∼-sym} \AgdaSymbol{:} \AgdaSymbol{∀} \AgdaSymbol{\{}\AgdaBound{U}\AgdaSymbol{\}} \AgdaSymbol{\{}\AgdaBound{V}\AgdaSymbol{\}} \AgdaSymbol{\{}\AgdaBound{σ} \AgdaBound{τ} \AgdaSymbol{:} \AgdaField{Op} \AgdaBound{U} \AgdaBound{V}\AgdaSymbol{\}} \AgdaSymbol{→} \AgdaBound{σ} \AgdaFunction{∼op} \AgdaBound{τ} \AgdaSymbol{→} \AgdaBound{τ} \AgdaFunction{∼op} \AgdaBound{σ}\<%
\\
\>[0]\AgdaIndent{2}{}\<[2]%
\>[2]\AgdaFunction{∼-sym} \AgdaBound{σ-is-τ} \AgdaBound{x} \AgdaSymbol{=} \AgdaFunction{sym} \AgdaSymbol{(}\AgdaBound{σ-is-τ} \AgdaBound{x}\AgdaSymbol{)}\<%
\\
%
\\
\>[0]\AgdaIndent{2}{}\<[2]%
\>[2]\AgdaFunction{∼-trans} \AgdaSymbol{:} \AgdaSymbol{∀} \AgdaSymbol{\{}\AgdaBound{U}\AgdaSymbol{\}} \AgdaSymbol{\{}\AgdaBound{V}\AgdaSymbol{\}} \AgdaSymbol{\{}\AgdaBound{ρ} \AgdaBound{σ} \AgdaBound{τ} \AgdaSymbol{:} \AgdaField{Op} \AgdaBound{U} \AgdaBound{V}\AgdaSymbol{\}} \AgdaSymbol{→} \AgdaBound{ρ} \AgdaFunction{∼op} \AgdaBound{σ} \AgdaSymbol{→} \AgdaBound{σ} \AgdaFunction{∼op} \AgdaBound{τ} \AgdaSymbol{→} \AgdaBound{ρ} \AgdaFunction{∼op} \AgdaBound{τ}\<%
\\
\>[0]\AgdaIndent{2}{}\<[2]%
\>[2]\AgdaFunction{∼-trans} \AgdaBound{ρ-is-σ} \AgdaBound{σ-is-τ} \AgdaBound{x} \AgdaSymbol{=} \AgdaFunction{trans} \AgdaSymbol{(}\AgdaBound{ρ-is-σ} \AgdaBound{x}\AgdaSymbol{)} \AgdaSymbol{(}\AgdaBound{σ-is-τ} \AgdaBound{x}\AgdaSymbol{)}\<%
\\
%
\\
\>[0]\AgdaIndent{2}{}\<[2]%
\>[2]\AgdaFunction{OP} \AgdaSymbol{:} \AgdaDatatype{Alphabet} \AgdaSymbol{→} \AgdaDatatype{Alphabet} \AgdaSymbol{→} \AgdaRecord{Setoid} \AgdaSymbol{\_} \AgdaSymbol{\_}\<%
\\
\>[0]\AgdaIndent{2}{}\<[2]%
\>[2]\AgdaFunction{OP} \AgdaBound{U} \AgdaBound{V} \AgdaSymbol{=} \AgdaKeyword{record} \AgdaSymbol{\{} \<[20]%
\>[20]\<%
\\
\>[2]\AgdaIndent{5}{}\<[5]%
\>[5]\AgdaField{Carrier} \AgdaSymbol{=} \AgdaField{Op} \AgdaBound{U} \AgdaBound{V} \AgdaSymbol{;} \<[24]%
\>[24]\<%
\\
\>[2]\AgdaIndent{5}{}\<[5]%
\>[5]\AgdaField{\_≈\_} \AgdaSymbol{=} \AgdaFunction{\_∼op\_} \AgdaSymbol{;} \<[19]%
\>[19]\<%
\\
\>[2]\AgdaIndent{5}{}\<[5]%
\>[5]\AgdaField{isEquivalence} \AgdaSymbol{=} \AgdaKeyword{record} \AgdaSymbol{\{} \<[30]%
\>[30]\<%
\\
\>[5]\AgdaIndent{7}{}\<[7]%
\>[7]\AgdaField{refl} \AgdaSymbol{=} \AgdaFunction{∼-refl} \AgdaSymbol{;} \<[23]%
\>[23]\<%
\\
\>[5]\AgdaIndent{7}{}\<[7]%
\>[7]\AgdaField{sym} \AgdaSymbol{=} \AgdaFunction{∼-sym} \AgdaSymbol{;} \<[21]%
\>[21]\<%
\\
\>[5]\AgdaIndent{7}{}\<[7]%
\>[7]\AgdaField{trans} \AgdaSymbol{=} \AgdaFunction{∼-trans} \AgdaSymbol{\}} \AgdaSymbol{\}}\<%
\end{code}

\AgdaHide{
\begin{code}%
\>\AgdaKeyword{open} \AgdaKeyword{import} \AgdaModule{Grammar.Base}\<%
\\
%
\\
\>\AgdaKeyword{module} \AgdaModule{Grammar.OpFamily.Lifting} \AgdaSymbol{(}\AgdaBound{G} \AgdaSymbol{:} \AgdaRecord{Grammar}\AgdaSymbol{)} \AgdaKeyword{where}\<%
\\
\>\AgdaKeyword{open} \AgdaKeyword{import} \AgdaModule{Data.List}\<%
\\
\>\AgdaKeyword{open} \AgdaKeyword{import} \AgdaModule{Prelims}\<%
\\
\>\AgdaKeyword{open} \AgdaModule{Grammar} \AgdaBound{G}\<%
\\
\>\AgdaKeyword{open} \AgdaKeyword{import} \AgdaModule{Grammar.OpFamily.PreOpFamily} \AgdaBound{G}\<%
\end{code}
}

\subsubsection{Liftings}

Define a \emph{lifting} on a pre-family to be an function $(- , K)$ that respects $\sim$:

\begin{code}%
\>\AgdaKeyword{record} \AgdaRecord{Lifting} \AgdaSymbol{(}\AgdaBound{F} \AgdaSymbol{:} \AgdaRecord{PreOpFamily}\AgdaSymbol{)} \AgdaSymbol{:} \AgdaPrimitiveType{Set₁} \AgdaKeyword{where}\<%
\\
\>[0]\AgdaIndent{2}{}\<[2]%
\>[2]\AgdaKeyword{open} \AgdaModule{PreOpFamily} \AgdaBound{F}\<%
\\
\>[0]\AgdaIndent{2}{}\<[2]%
\>[2]\AgdaKeyword{field}\<%
\\
\>[2]\AgdaIndent{4}{}\<[4]%
\>[4]\AgdaField{liftOp} \AgdaSymbol{:} \AgdaSymbol{∀} \AgdaSymbol{\{}\AgdaBound{U}\AgdaSymbol{\}} \AgdaSymbol{\{}\AgdaBound{V}\AgdaSymbol{\}} \AgdaBound{K} \AgdaSymbol{→} \AgdaFunction{Op} \AgdaBound{U} \AgdaBound{V} \AgdaSymbol{→} \AgdaFunction{Op} \AgdaSymbol{(}\AgdaBound{U} \AgdaInductiveConstructor{,} \AgdaBound{K}\AgdaSymbol{)} \AgdaSymbol{(}\AgdaBound{V} \AgdaInductiveConstructor{,} \AgdaBound{K}\AgdaSymbol{)}\<%
\\
\>[2]\AgdaIndent{4}{}\<[4]%
\>[4]\AgdaField{liftOp-cong} \AgdaSymbol{:} \AgdaSymbol{∀} \AgdaSymbol{\{}\AgdaBound{V}\AgdaSymbol{\}} \AgdaSymbol{\{}\AgdaBound{W}\AgdaSymbol{\}} \AgdaSymbol{\{}\AgdaBound{K}\AgdaSymbol{\}} \AgdaSymbol{\{}\AgdaBound{ρ} \AgdaBound{σ} \AgdaSymbol{:} \AgdaFunction{Op} \AgdaBound{V} \AgdaBound{W}\AgdaSymbol{\}} \AgdaSymbol{→} \<[49]%
\>[49]\<%
\\
\>[4]\AgdaIndent{6}{}\<[6]%
\>[6]\AgdaBound{ρ} \AgdaFunction{∼op} \AgdaBound{σ} \AgdaSymbol{→} \AgdaField{liftOp} \AgdaBound{K} \AgdaBound{ρ} \AgdaFunction{∼op} \AgdaField{liftOp} \AgdaBound{K} \AgdaBound{σ}\<%
\end{code}

Given an operation $\sigma : U \rightarrow V$ and a list of variable kinds $A \equiv (A_1 , \ldots , A_n)$, define
the \emph{repeated lifting} $\sigma^A$ to be $((\cdots(\sigma , A_1) , A_2) , \cdots ) , A_n)$.

\begin{code}%
\>\AgdaComment{\{-  liftOp' : ∀ \{U\} \{V\} A → Op U V → Op (extend U A) (extend V A)\<\\
\>  liftOp' [] σ = σ\<\\
\>  liftOp' (K ∷ A) σ = liftOp' A (liftOp K σ) -\}}\<%
\\
%
\\
\>[0]\AgdaIndent{2}{}\<[2]%
\>[2]\AgdaFunction{liftOp''} \AgdaSymbol{:} \AgdaSymbol{∀} \AgdaSymbol{\{}\AgdaBound{U}\AgdaSymbol{\}} \AgdaSymbol{\{}\AgdaBound{V}\AgdaSymbol{\}} \AgdaSymbol{\{}\AgdaBound{K}\AgdaSymbol{\}} \AgdaBound{A} \AgdaSymbol{→} \AgdaFunction{Op} \AgdaBound{U} \AgdaBound{V} \AgdaSymbol{→} \AgdaFunction{Op} \AgdaSymbol{(}\AgdaFunction{dom} \AgdaBound{U} \AgdaSymbol{\{}\AgdaBound{K}\AgdaSymbol{\}} \AgdaBound{A}\AgdaSymbol{)} \AgdaSymbol{(}\AgdaFunction{dom} \AgdaBound{V} \AgdaBound{A}\AgdaSymbol{)}\<%
\\
\>[0]\AgdaIndent{2}{}\<[2]%
\>[2]\AgdaFunction{liftOp''} \AgdaSymbol{(\_} \AgdaInductiveConstructor{✧}\AgdaSymbol{)} \AgdaBound{σ} \AgdaSymbol{=} \AgdaBound{σ}\<%
\\
\>[0]\AgdaIndent{2}{}\<[2]%
\>[2]\AgdaFunction{liftOp''} \AgdaSymbol{(}\AgdaBound{K} \AgdaInductiveConstructor{abs} \AgdaBound{A}\AgdaSymbol{)} \AgdaBound{σ} \AgdaSymbol{=} \AgdaFunction{liftOp''} \AgdaBound{A} \AgdaSymbol{(}\AgdaField{liftOp} \AgdaBound{K} \AgdaBound{σ}\AgdaSymbol{)}\<%
\\
%
\\
\>\AgdaComment{\{-  liftOp'-cong : ∀ \{U\} \{V\} A \{ρ σ : Op U V\} → \<\\
\>    ρ ∼op σ → liftOp' A ρ ∼op liftOp' A σ\<\\
\>}\<%
\end{code}

\AgdaHide{
\begin{code}%
\>\AgdaComment{\<\\
\>  liftOp'-cong [] ρ-is-σ = ρ-is-σ\<\\
\>  liftOp'-cong (\_ ∷ A) ρ-is-σ = liftOp'-cong A (liftOp-cong ρ-is-σ) -\}}\<%
\\
%
\\
\>[0]\AgdaIndent{2}{}\<[2]%
\>[2]\AgdaKeyword{postulate} \AgdaPostulate{liftOp''-cong} \AgdaSymbol{:} \AgdaSymbol{∀} \AgdaSymbol{\{}\AgdaBound{U}\AgdaSymbol{\}} \AgdaSymbol{\{}\AgdaBound{V}\AgdaSymbol{\}} \AgdaSymbol{\{}\AgdaBound{K}\AgdaSymbol{\}} \AgdaBound{A} \AgdaSymbol{\{}\AgdaBound{ρ} \AgdaBound{σ} \AgdaSymbol{:} \AgdaFunction{Op} \AgdaBound{U} \AgdaBound{V}\AgdaSymbol{\}} \AgdaSymbol{→} \<[61]%
\>[61]\<%
\\
\>[2]\AgdaIndent{26}{}\<[26]%
\>[26]\AgdaBound{ρ} \AgdaFunction{∼op} \AgdaBound{σ} \AgdaSymbol{→} \AgdaFunction{liftOp''} \AgdaSymbol{\{}\AgdaArgument{K} \AgdaSymbol{=} \AgdaBound{K}\AgdaSymbol{\}} \AgdaBound{A} \AgdaBound{ρ} \AgdaFunction{∼op} \AgdaFunction{liftOp''} \AgdaBound{A} \AgdaBound{σ}\<%
\end{code}
}

This allows us to define the action of \emph{application} $E[\sigma]$:

\begin{code}%
\>[0]\AgdaIndent{2}{}\<[2]%
\>[2]\AgdaFunction{ap} \AgdaSymbol{:} \AgdaSymbol{∀} \AgdaSymbol{\{}\AgdaBound{U}\AgdaSymbol{\}} \AgdaSymbol{\{}\AgdaBound{V}\AgdaSymbol{\}} \AgdaSymbol{\{}\AgdaBound{C}\AgdaSymbol{\}} \AgdaSymbol{\{}\AgdaBound{K}\AgdaSymbol{\}} \AgdaSymbol{→} \<[27]%
\>[27]\<%
\\
\>[2]\AgdaIndent{4}{}\<[4]%
\>[4]\AgdaFunction{Op} \AgdaBound{U} \AgdaBound{V} \AgdaSymbol{→} \AgdaDatatype{Subexpression} \AgdaBound{U} \AgdaBound{C} \AgdaBound{K} \AgdaSymbol{→} \AgdaDatatype{Subexpression} \AgdaBound{V} \AgdaBound{C} \AgdaBound{K}\<%
\\
\>[0]\AgdaIndent{2}{}\<[2]%
\>[2]\AgdaFunction{ap} \AgdaBound{ρ} \AgdaSymbol{(}\AgdaInductiveConstructor{var} \AgdaBound{x}\AgdaSymbol{)} \AgdaSymbol{=} \AgdaFunction{apV} \AgdaBound{ρ} \AgdaBound{x}\<%
\\
\>[0]\AgdaIndent{2}{}\<[2]%
\>[2]\AgdaFunction{ap} \AgdaBound{ρ} \AgdaSymbol{(}\AgdaInductiveConstructor{app} \AgdaBound{c} \AgdaBound{EE}\AgdaSymbol{)} \AgdaSymbol{=} \AgdaInductiveConstructor{app} \AgdaBound{c} \AgdaSymbol{(}\AgdaFunction{ap} \AgdaBound{ρ} \AgdaBound{EE}\AgdaSymbol{)}\<%
\\
\>[0]\AgdaIndent{2}{}\<[2]%
\>[2]\AgdaFunction{ap} \AgdaSymbol{\_} \AgdaInductiveConstructor{out} \AgdaSymbol{=} \AgdaInductiveConstructor{out}\<%
\\
\>[0]\AgdaIndent{2}{}\<[2]%
\>[2]\AgdaFunction{ap} \AgdaBound{ρ} \AgdaSymbol{(}\AgdaInductiveConstructor{\_,,\_} \AgdaSymbol{\{}\AgdaArgument{A} \AgdaSymbol{=} \AgdaBound{A}\AgdaSymbol{\}} \AgdaBound{E} \AgdaBound{EE}\AgdaSymbol{)} \AgdaSymbol{=} \AgdaFunction{ap} \AgdaSymbol{(}\AgdaFunction{liftOp''} \AgdaBound{A} \AgdaBound{ρ}\AgdaSymbol{)} \AgdaBound{E} \AgdaInductiveConstructor{,,} \AgdaFunction{ap} \AgdaBound{ρ} \AgdaBound{EE}\<%
\end{code}

We prove that application respects $\sim$.

\begin{code}%
\>[0]\AgdaIndent{2}{}\<[2]%
\>[2]\AgdaFunction{ap-congl} \AgdaSymbol{:} \AgdaSymbol{∀} \AgdaSymbol{\{}\AgdaBound{U}\AgdaSymbol{\}} \AgdaSymbol{\{}\AgdaBound{V}\AgdaSymbol{\}} \AgdaSymbol{\{}\AgdaBound{C}\AgdaSymbol{\}} \AgdaSymbol{\{}\AgdaBound{K}\AgdaSymbol{\}} \<[31]%
\>[31]\<%
\\
\>[2]\AgdaIndent{4}{}\<[4]%
\>[4]\AgdaSymbol{\{}\AgdaBound{ρ} \AgdaBound{σ} \AgdaSymbol{:} \AgdaFunction{Op} \AgdaBound{U} \AgdaBound{V}\AgdaSymbol{\}} \AgdaSymbol{(}\AgdaBound{E} \AgdaSymbol{:} \AgdaDatatype{Subexpression} \AgdaBound{U} \AgdaBound{C} \AgdaBound{K}\AgdaSymbol{)} \AgdaSymbol{→}\<%
\\
\>[2]\AgdaIndent{4}{}\<[4]%
\>[4]\AgdaBound{ρ} \AgdaFunction{∼op} \AgdaBound{σ} \AgdaSymbol{→} \AgdaFunction{ap} \AgdaBound{ρ} \AgdaBound{E} \AgdaDatatype{≡} \AgdaFunction{ap} \AgdaBound{σ} \AgdaBound{E}\<%
\end{code}

\AgdaHide{
\begin{code}%
\>[0]\AgdaIndent{2}{}\<[2]%
\>[2]\AgdaFunction{ap-congl} \AgdaSymbol{(}\AgdaInductiveConstructor{var} \AgdaBound{x}\AgdaSymbol{)} \AgdaBound{ρ-is-σ} \AgdaSymbol{=} \AgdaBound{ρ-is-σ} \AgdaBound{x}\<%
\\
\>[0]\AgdaIndent{2}{}\<[2]%
\>[2]\AgdaFunction{ap-congl} \AgdaSymbol{(}\AgdaInductiveConstructor{app} \AgdaBound{c} \AgdaBound{E}\AgdaSymbol{)} \AgdaBound{ρ-is-σ} \AgdaSymbol{=} \AgdaFunction{cong} \AgdaSymbol{(}\AgdaInductiveConstructor{app} \AgdaBound{c}\AgdaSymbol{)} \AgdaSymbol{(}\AgdaFunction{ap-congl} \AgdaBound{E} \AgdaBound{ρ-is-σ}\AgdaSymbol{)}\<%
\\
\>[0]\AgdaIndent{2}{}\<[2]%
\>[2]\AgdaFunction{ap-congl} \AgdaInductiveConstructor{out} \AgdaSymbol{\_} \AgdaSymbol{=} \AgdaInductiveConstructor{refl}\<%
\\
\>[0]\AgdaIndent{2}{}\<[2]%
\>[2]\AgdaFunction{ap-congl} \AgdaSymbol{(}\AgdaInductiveConstructor{\_,,\_} \AgdaSymbol{\{}\AgdaArgument{L} \AgdaSymbol{=} \AgdaBound{L}\AgdaSymbol{\}} \AgdaSymbol{\{}\AgdaArgument{A} \AgdaSymbol{=} \AgdaBound{A}\AgdaSymbol{\}} \AgdaBound{E} \AgdaBound{F}\AgdaSymbol{)} \AgdaBound{ρ-is-σ} \AgdaSymbol{=} \<[47]%
\>[47]\<%
\\
\>[2]\AgdaIndent{4}{}\<[4]%
\>[4]\AgdaFunction{cong₂} \AgdaInductiveConstructor{\_,,\_} \AgdaSymbol{(}\AgdaFunction{ap-congl} \AgdaBound{E} \AgdaSymbol{(}\AgdaPostulate{liftOp''-cong} \AgdaBound{A} \AgdaBound{ρ-is-σ}\AgdaSymbol{))} \AgdaSymbol{(}\AgdaFunction{ap-congl} \AgdaBound{F} \AgdaBound{ρ-is-σ}\AgdaSymbol{)}\<%
\\
%
\\
\>[0]\AgdaIndent{2}{}\<[2]%
\>[2]\AgdaFunction{ap-congr} \AgdaSymbol{:} \AgdaSymbol{∀} \AgdaSymbol{\{}\AgdaBound{U}\AgdaSymbol{\}} \AgdaSymbol{\{}\AgdaBound{V}\AgdaSymbol{\}} \AgdaSymbol{\{}\AgdaBound{C}\AgdaSymbol{\}} \AgdaSymbol{\{}\AgdaBound{K}\AgdaSymbol{\}}\<%
\\
\>[2]\AgdaIndent{4}{}\<[4]%
\>[4]\AgdaSymbol{\{}\AgdaBound{σ} \AgdaSymbol{:} \AgdaFunction{Op} \AgdaBound{U} \AgdaBound{V}\AgdaSymbol{\}} \AgdaSymbol{\{}\AgdaBound{E} \AgdaBound{F} \AgdaSymbol{:} \AgdaDatatype{Subexpression} \AgdaBound{U} \AgdaBound{C} \AgdaBound{K}\AgdaSymbol{\}} \AgdaSymbol{→}\<%
\\
\>[2]\AgdaIndent{4}{}\<[4]%
\>[4]\AgdaBound{E} \AgdaDatatype{≡} \AgdaBound{F} \AgdaSymbol{→} \AgdaFunction{ap} \AgdaBound{σ} \AgdaBound{E} \AgdaDatatype{≡} \AgdaFunction{ap} \AgdaBound{σ} \AgdaBound{F}\<%
\\
\>[0]\AgdaIndent{2}{}\<[2]%
\>[2]\AgdaFunction{ap-congr} \AgdaSymbol{\{}\AgdaArgument{σ} \AgdaSymbol{=} \AgdaBound{σ}\AgdaSymbol{\}} \AgdaSymbol{=} \AgdaFunction{cong} \AgdaSymbol{(}\AgdaFunction{ap} \AgdaBound{σ}\AgdaSymbol{)}\<%
\\
%
\\
\>[0]\AgdaIndent{2}{}\<[2]%
\>[2]\AgdaFunction{ap-cong} \AgdaSymbol{:} \AgdaSymbol{∀} \AgdaSymbol{\{}\AgdaBound{U}\AgdaSymbol{\}} \AgdaSymbol{\{}\AgdaBound{V}\AgdaSymbol{\}} \AgdaSymbol{\{}\AgdaBound{C}\AgdaSymbol{\}} \AgdaSymbol{\{}\AgdaBound{K}\AgdaSymbol{\}}\<%
\\
\>[2]\AgdaIndent{4}{}\<[4]%
\>[4]\AgdaSymbol{\{}\AgdaBound{ρ} \AgdaBound{σ} \AgdaSymbol{:} \AgdaFunction{Op} \AgdaBound{U} \AgdaBound{V}\AgdaSymbol{\}} \AgdaSymbol{\{}\AgdaBound{M} \AgdaBound{N} \AgdaSymbol{:} \AgdaDatatype{Subexpression} \AgdaBound{U} \AgdaBound{C} \AgdaBound{K}\AgdaSymbol{\}} \AgdaSymbol{→}\<%
\\
\>[2]\AgdaIndent{4}{}\<[4]%
\>[4]\AgdaBound{ρ} \AgdaFunction{∼op} \AgdaBound{σ} \AgdaSymbol{→} \AgdaBound{M} \AgdaDatatype{≡} \AgdaBound{N} \AgdaSymbol{→} \AgdaFunction{ap} \AgdaBound{ρ} \AgdaBound{M} \AgdaDatatype{≡} \AgdaFunction{ap} \AgdaBound{σ} \AgdaBound{N}\<%
\\
\>[0]\AgdaIndent{2}{}\<[2]%
\>[2]\AgdaFunction{ap-cong} \AgdaSymbol{\{}\AgdaArgument{ρ} \AgdaSymbol{=} \AgdaBound{ρ}\AgdaSymbol{\}} \AgdaSymbol{\{}\AgdaBound{σ}\AgdaSymbol{\}} \AgdaSymbol{\{}\AgdaBound{M}\AgdaSymbol{\}} \AgdaSymbol{\{}\AgdaBound{N}\AgdaSymbol{\}} \AgdaBound{ρ∼σ} \AgdaBound{M≡N} \AgdaSymbol{=} \AgdaKeyword{let} \AgdaKeyword{open} \AgdaModule{≡-Reasoning} \AgdaKeyword{in} \<[64]%
\>[64]\<%
\\
\>[2]\AgdaIndent{4}{}\<[4]%
\>[4]\AgdaFunction{begin}\<%
\\
\>[4]\AgdaIndent{6}{}\<[6]%
\>[6]\AgdaFunction{ap} \AgdaBound{ρ} \AgdaBound{M}\<%
\\
\>[0]\AgdaIndent{4}{}\<[4]%
\>[4]\AgdaFunction{≡⟨} \AgdaFunction{ap-congl} \AgdaBound{M} \AgdaBound{ρ∼σ} \AgdaFunction{⟩}\<%
\\
\>[4]\AgdaIndent{6}{}\<[6]%
\>[6]\AgdaFunction{ap} \AgdaBound{σ} \AgdaBound{M}\<%
\\
\>[0]\AgdaIndent{4}{}\<[4]%
\>[4]\AgdaFunction{≡⟨} \AgdaFunction{ap-congr} \AgdaBound{M≡N} \AgdaFunction{⟩}\<%
\\
\>[4]\AgdaIndent{6}{}\<[6]%
\>[6]\AgdaFunction{ap} \AgdaBound{σ} \AgdaBound{N}\<%
\\
\>[0]\AgdaIndent{4}{}\<[4]%
\>[4]\AgdaFunction{∎}\<%
\end{code}
}

\AgdaHide{
\begin{code}%
\>\AgdaKeyword{open} \AgdaKeyword{import} \AgdaModule{Grammar.Base}\<%
\\
%
\\
\>\AgdaKeyword{module} \AgdaModule{Grammar.Substitution.LiftFamily} \AgdaSymbol{(}\AgdaBound{G} \AgdaSymbol{:} \AgdaRecord{Grammar}\AgdaSymbol{)} \AgdaKeyword{where}\<%
\\
\>\AgdaKeyword{open} \AgdaKeyword{import} \AgdaModule{Prelims}\<%
\\
\>\AgdaKeyword{open} \AgdaModule{Grammar} \AgdaBound{G}\<%
\\
\>\AgdaKeyword{open} \AgdaKeyword{import} \AgdaModule{Grammar.OpFamily.LiftFamily} \AgdaBound{G}\<%
\\
\>\AgdaKeyword{open} \AgdaKeyword{import} \AgdaModule{Grammar.Substitution.PreOpFamily} \AgdaBound{G}\<%
\\
\>\AgdaKeyword{open} \AgdaKeyword{import} \AgdaModule{Grammar.Substitution.Lifting} \AgdaBound{G}\<%
\\
\>\AgdaKeyword{open} \AgdaKeyword{import} \AgdaModule{Grammar.Substitution.RepSub} \AgdaBound{G}\<%
\end{code}
}

It is now easy to show that substitution forms a pre-family with lifting.  If $\sigma : U \rightarrow V$ and $x \in U$ then $(\sigma , K)(\uparrow x) \equiv
\sigma(x) \langle \uparrow \rangle \equiv (\sigma , K)(x) [ \uparrow ]$.

\begin{code}%
\>\AgdaFunction{SubLF} \AgdaSymbol{:} \AgdaRecord{LiftFamily}\<%
\\
\>\AgdaFunction{SubLF} \AgdaSymbol{=} \AgdaKeyword{record} \AgdaSymbol{\{} \<[17]%
\>[17]\<%
\\
\>[0]\AgdaIndent{2}{}\<[2]%
\>[2]\AgdaField{preOpFamily} \AgdaSymbol{=} \AgdaFunction{pre-substitution} \AgdaSymbol{;} \<[35]%
\>[35]\<%
\\
\>[0]\AgdaIndent{2}{}\<[2]%
\>[2]\AgdaField{lifting} \AgdaSymbol{=} \AgdaFunction{SUB↑} \AgdaSymbol{;} \<[19]%
\>[19]\<%
\\
\>[0]\AgdaIndent{2}{}\<[2]%
\>[2]\AgdaField{isLiftFamily} \AgdaSymbol{=} \AgdaKeyword{record} \AgdaSymbol{\{} \<[26]%
\>[26]\<%
\\
\>[2]\AgdaIndent{4}{}\<[4]%
\>[4]\AgdaField{liftOp-x₀} \AgdaSymbol{=} \AgdaInductiveConstructor{refl} \AgdaSymbol{;} \<[23]%
\>[23]\<%
\\
\>[2]\AgdaIndent{4}{}\<[4]%
\>[4]\AgdaField{liftOp-↑} \AgdaSymbol{=} \AgdaSymbol{λ} \AgdaSymbol{\{}\AgdaBound{\_}\AgdaSymbol{\}} \AgdaSymbol{\{}\AgdaBound{\_}\AgdaSymbol{\}} \AgdaSymbol{\{}\AgdaBound{\_}\AgdaSymbol{\}} \AgdaSymbol{\{}\AgdaBound{\_}\AgdaSymbol{\}} \AgdaSymbol{\{}\AgdaBound{σ}\AgdaSymbol{\}} \AgdaBound{x} \AgdaSymbol{→} \AgdaFunction{rep-is-sub} \AgdaSymbol{(}\AgdaBound{σ} \AgdaSymbol{\_} \AgdaBound{x}\AgdaSymbol{)} \AgdaSymbol{\}\}}\<%
\end{code}

\AgdaHide{
\begin{code}%
\>\AgdaKeyword{open} \AgdaKeyword{import} \AgdaModule{Grammar.Base}\<%
\\
%
\\
\>\AgdaKeyword{module} \AgdaModule{Grammar.OpFamily.Composition} \AgdaSymbol{(}\AgdaBound{A} \AgdaSymbol{:} \AgdaRecord{Grammar}\AgdaSymbol{)} \AgdaKeyword{where}\<%
\\
\>\AgdaKeyword{open} \AgdaKeyword{import} \AgdaModule{Data.List}\<%
\\
\>\AgdaKeyword{open} \AgdaKeyword{import} \AgdaModule{Function.Equality} \AgdaKeyword{hiding} \AgdaSymbol{(}\AgdaField{cong}\AgdaSymbol{;}\AgdaFunction{\_∘\_}\AgdaSymbol{)}\<%
\\
\>\AgdaKeyword{open} \AgdaKeyword{import} \AgdaModule{Prelims}\<%
\\
\>\AgdaKeyword{open} \AgdaModule{Grammar} \AgdaBound{A} \AgdaKeyword{hiding} \AgdaSymbol{(}\_⟶\_\AgdaSymbol{)}\<%
\\
\>\AgdaKeyword{open} \AgdaKeyword{import} \AgdaModule{Grammar.OpFamily.LiftFamily} \AgdaBound{A}\<%
\\
%
\\
\>\AgdaKeyword{open} \AgdaModule{LiftFamily}\<%
\end{code}
}

\subsubsection{Compositions}

Let $F$, $G$ and $H$ be three pre-families with lifting.  A \emph{composition} $\circ : F;G \rightarrow H$ is a family of functions
\[ \circ_{UVW} : F[V,W] \times G[U,V] \rightarrow H[U,W] \]
for all alphabets $U$, $V$ and $W$ such that:
\begin{itemize}
\item
$(\sigma \circ \rho , K) \sim (\sigma , K) \circ (\rho , K)$
\item
$(\sigma \circ \rho)(x) \equiv \rho(x) [ \sigma ]$
\end{itemize}

\begin{code}%
\>\AgdaKeyword{record} \AgdaRecord{Composition} \AgdaSymbol{(}\AgdaBound{F} \AgdaBound{G} \AgdaBound{H} \AgdaSymbol{:} \AgdaRecord{LiftFamily}\AgdaSymbol{)} \AgdaSymbol{:} \AgdaPrimitiveType{Set} \AgdaKeyword{where}\<%
\\
\>[0]\AgdaIndent{2}{}\<[2]%
\>[2]\AgdaKeyword{infix} \AgdaNumber{25} \AgdaFixityOp{\_∘\_}\<%
\\
\>[0]\AgdaIndent{2}{}\<[2]%
\>[2]\AgdaKeyword{field}\<%
\\
\>[2]\AgdaIndent{4}{}\<[4]%
\>[4]\AgdaField{\_∘\_} \AgdaSymbol{:} \AgdaSymbol{∀} \AgdaSymbol{\{}\AgdaBound{U}\AgdaSymbol{\}} \AgdaSymbol{\{}\AgdaBound{V}\AgdaSymbol{\}} \AgdaSymbol{\{}\AgdaBound{W}\AgdaSymbol{\}} \AgdaSymbol{→} \AgdaFunction{Op} \AgdaBound{F} \AgdaBound{V} \AgdaBound{W} \AgdaSymbol{→} \AgdaFunction{Op} \AgdaBound{G} \AgdaBound{U} \AgdaBound{V} \AgdaSymbol{→} \AgdaFunction{Op} \AgdaBound{H} \AgdaBound{U} \AgdaBound{W}\<%
\\
\>[2]\AgdaIndent{4}{}\<[4]%
\>[4]\AgdaField{liftOp-comp} \AgdaSymbol{:} \AgdaSymbol{∀} \AgdaSymbol{\{}\AgdaBound{U} \AgdaBound{V} \AgdaBound{W} \AgdaBound{K} \AgdaBound{σ} \AgdaBound{ρ}\AgdaSymbol{\}} \AgdaSymbol{→} \<[36]%
\>[36]\<%
\\
\>[4]\AgdaIndent{6}{}\<[6]%
\>[6]\AgdaFunction{\_∼op\_} \AgdaBound{H} \AgdaSymbol{(}\AgdaFunction{liftOp} \AgdaBound{H} \AgdaBound{K} \AgdaSymbol{(}\AgdaField{\_∘\_} \AgdaSymbol{\{}\AgdaBound{U}\AgdaSymbol{\}} \AgdaSymbol{\{}\AgdaBound{V}\AgdaSymbol{\}} \AgdaSymbol{\{}\AgdaBound{W}\AgdaSymbol{\}} \AgdaBound{σ} \AgdaBound{ρ}\AgdaSymbol{))} \<[49]%
\>[49]\<%
\\
\>[6]\AgdaIndent{8}{}\<[8]%
\>[8]\AgdaSymbol{(}\AgdaFunction{liftOp} \AgdaBound{F} \AgdaBound{K} \AgdaBound{σ} \AgdaField{∘} \AgdaFunction{liftOp} \AgdaBound{G} \AgdaBound{K} \AgdaBound{ρ}\AgdaSymbol{)}\<%
\\
\>[0]\AgdaIndent{4}{}\<[4]%
\>[4]\AgdaField{apV-comp} \AgdaSymbol{:} \AgdaSymbol{∀} \AgdaSymbol{\{}\AgdaBound{U}\AgdaSymbol{\}} \AgdaSymbol{\{}\AgdaBound{V}\AgdaSymbol{\}} \AgdaSymbol{\{}\AgdaBound{W}\AgdaSymbol{\}} \AgdaSymbol{\{}\AgdaBound{K}\AgdaSymbol{\}} \AgdaSymbol{\{}\AgdaBound{σ}\AgdaSymbol{\}} \AgdaSymbol{\{}\AgdaBound{ρ}\AgdaSymbol{\}} \AgdaSymbol{\{}\AgdaBound{x} \AgdaSymbol{:} \AgdaDatatype{Var} \AgdaBound{U} \AgdaBound{K}\AgdaSymbol{\}} \AgdaSymbol{→} \<[57]%
\>[57]\<%
\\
\>[4]\AgdaIndent{6}{}\<[6]%
\>[6]\AgdaFunction{apV} \AgdaBound{H} \AgdaSymbol{(}\AgdaField{\_∘\_} \AgdaSymbol{\{}\AgdaBound{U}\AgdaSymbol{\}} \AgdaSymbol{\{}\AgdaBound{V}\AgdaSymbol{\}} \AgdaSymbol{\{}\AgdaBound{W}\AgdaSymbol{\}} \AgdaBound{σ} \AgdaBound{ρ}\AgdaSymbol{)} \AgdaBound{x} \AgdaDatatype{≡} \AgdaFunction{ap} \AgdaBound{F} \AgdaBound{σ} \AgdaSymbol{(}\AgdaFunction{apV} \AgdaBound{G} \AgdaBound{ρ} \AgdaBound{x}\AgdaSymbol{)}\<%
\end{code}

\begin{lemma}
For any composition $\circ$:
\begin{enumerate}
\item
If $\sigma \sim \sigma'$ and $\rho \sim \rho'$ then $\sigma \circ \rho \sim \sigma' \circ \rho'$
\item
$(\sigma \circ \rho)^A \sim \sigma^A \circ \rho^A$
\item
$E [ \sigma \circ \rho ] \equiv E [ \rho ] [ \sigma ]$
\end{enumerate}
\end{lemma}

\begin{code}%
\>[0]\AgdaIndent{2}{}\<[2]%
\>[2]\AgdaFunction{comp-cong} \AgdaSymbol{:} \AgdaSymbol{∀} \AgdaSymbol{\{}\AgdaBound{U} \AgdaBound{V} \AgdaBound{W}\AgdaSymbol{\}} \AgdaSymbol{\{}\AgdaBound{σ} \AgdaBound{σ'} \AgdaSymbol{:} \AgdaFunction{Op} \AgdaBound{F} \AgdaBound{V} \AgdaBound{W}\AgdaSymbol{\}} \AgdaSymbol{\{}\AgdaBound{ρ} \AgdaBound{ρ'} \AgdaSymbol{:} \AgdaFunction{Op} \AgdaBound{G} \AgdaBound{U} \AgdaBound{V}\AgdaSymbol{\}} \AgdaSymbol{→} \<[62]%
\>[62]\<%
\\
\>[2]\AgdaIndent{4}{}\<[4]%
\>[4]\AgdaFunction{\_∼op\_} \AgdaBound{F} \AgdaBound{σ} \AgdaBound{σ'} \AgdaSymbol{→} \AgdaFunction{\_∼op\_} \AgdaBound{G} \AgdaBound{ρ} \AgdaBound{ρ'} \AgdaSymbol{→} \AgdaFunction{\_∼op\_} \AgdaBound{H} \AgdaSymbol{(}\AgdaBound{σ} \AgdaField{∘} \AgdaBound{ρ}\AgdaSymbol{)} \AgdaSymbol{(}\AgdaBound{σ'} \AgdaField{∘} \AgdaBound{ρ'}\AgdaSymbol{)}\<%
\end{code}

\AgdaHide{
\begin{code}%
\>[0]\AgdaIndent{2}{}\<[2]%
\>[2]\AgdaFunction{comp-cong} \AgdaSymbol{\{}\AgdaBound{U}\AgdaSymbol{\}} \AgdaSymbol{\{}\AgdaBound{V}\AgdaSymbol{\}} \AgdaSymbol{\{}\AgdaBound{W}\AgdaSymbol{\}} \AgdaSymbol{\{}\AgdaBound{σ}\AgdaSymbol{\}} \AgdaSymbol{\{}\AgdaBound{σ'}\AgdaSymbol{\}} \AgdaSymbol{\{}\AgdaBound{ρ}\AgdaSymbol{\}} \AgdaSymbol{\{}\AgdaBound{ρ'}\AgdaSymbol{\}} \AgdaBound{σ∼σ'} \AgdaBound{ρ∼ρ'} \AgdaBound{x} \AgdaSymbol{=} \AgdaKeyword{let} \AgdaKeyword{open} \AgdaModule{≡-Reasoning} \AgdaKeyword{in} \<[80]%
\>[80]\<%
\\
\>[2]\AgdaIndent{4}{}\<[4]%
\>[4]\AgdaFunction{begin}\<%
\\
\>[4]\AgdaIndent{6}{}\<[6]%
\>[6]\AgdaFunction{apV} \AgdaBound{H} \AgdaSymbol{(}\AgdaBound{σ} \AgdaField{∘} \AgdaBound{ρ}\AgdaSymbol{)} \AgdaBound{x}\<%
\\
\>[0]\AgdaIndent{4}{}\<[4]%
\>[4]\AgdaFunction{≡⟨} \AgdaField{apV-comp} \AgdaFunction{⟩}\<%
\\
\>[4]\AgdaIndent{6}{}\<[6]%
\>[6]\AgdaFunction{ap} \AgdaBound{F} \AgdaBound{σ} \AgdaSymbol{(}\AgdaFunction{apV} \AgdaBound{G} \AgdaBound{ρ} \AgdaBound{x}\AgdaSymbol{)}\<%
\\
\>[0]\AgdaIndent{4}{}\<[4]%
\>[4]\AgdaFunction{≡⟨} \AgdaFunction{ap-cong} \AgdaBound{F} \AgdaBound{σ∼σ'} \AgdaSymbol{(}\AgdaBound{ρ∼ρ'} \AgdaBound{x}\AgdaSymbol{)} \AgdaFunction{⟩}\<%
\\
\>[4]\AgdaIndent{6}{}\<[6]%
\>[6]\AgdaFunction{ap} \AgdaBound{F} \AgdaBound{σ'} \AgdaSymbol{(}\AgdaFunction{apV} \AgdaBound{G} \AgdaBound{ρ'} \AgdaBound{x}\AgdaSymbol{)}\<%
\\
\>[0]\AgdaIndent{4}{}\<[4]%
\>[4]\AgdaFunction{≡⟨⟨} \AgdaField{apV-comp} \AgdaFunction{⟩⟩}\<%
\\
\>[4]\AgdaIndent{6}{}\<[6]%
\>[6]\AgdaFunction{apV} \AgdaBound{H} \AgdaSymbol{(}\AgdaBound{σ'} \AgdaField{∘} \AgdaBound{ρ'}\AgdaSymbol{)} \AgdaBound{x}\<%
\\
\>[0]\AgdaIndent{4}{}\<[4]%
\>[4]\AgdaFunction{∎}\<%
\\
%
\\
\>[0]\AgdaIndent{2}{}\<[2]%
\>[2]\AgdaFunction{comp-congl} \AgdaSymbol{:} \AgdaSymbol{∀} \AgdaSymbol{\{}\AgdaBound{U}\AgdaSymbol{\}} \AgdaSymbol{\{}\AgdaBound{V}\AgdaSymbol{\}} \AgdaSymbol{\{}\AgdaBound{W}\AgdaSymbol{\}} \AgdaSymbol{\{}\AgdaBound{σ} \AgdaBound{σ'} \AgdaSymbol{:} \AgdaFunction{Op} \AgdaBound{F} \AgdaBound{V} \AgdaBound{W}\AgdaSymbol{\}} \AgdaSymbol{\{}\AgdaBound{ρ} \AgdaSymbol{:} \AgdaFunction{Op} \AgdaBound{G} \AgdaBound{U} \AgdaBound{V}\AgdaSymbol{\}} \AgdaSymbol{→}\<%
\\
\>[2]\AgdaIndent{4}{}\<[4]%
\>[4]\AgdaFunction{\_∼op\_} \AgdaBound{F} \AgdaBound{σ} \AgdaBound{σ'} \AgdaSymbol{→} \AgdaFunction{\_∼op\_} \AgdaBound{H} \AgdaSymbol{(}\AgdaBound{σ} \AgdaField{∘} \AgdaBound{ρ}\AgdaSymbol{)} \AgdaSymbol{(}\AgdaBound{σ'} \AgdaField{∘} \AgdaBound{ρ}\AgdaSymbol{)}\<%
\\
\>[0]\AgdaIndent{2}{}\<[2]%
\>[2]\AgdaFunction{comp-congl} \AgdaSymbol{\{}\AgdaBound{U}\AgdaSymbol{\}} \AgdaSymbol{\{}\AgdaBound{V}\AgdaSymbol{\}} \AgdaSymbol{\{}\AgdaBound{W}\AgdaSymbol{\}} \AgdaSymbol{=} \AgdaFunction{Bifunction.congl} \AgdaSymbol{\{}\AgdaArgument{A} \AgdaSymbol{=} \AgdaFunction{OP} \AgdaBound{F} \AgdaBound{V} \AgdaBound{W}\AgdaSymbol{\}} \AgdaSymbol{\{}\AgdaArgument{B} \AgdaSymbol{=} \AgdaFunction{OP} \AgdaBound{G} \AgdaBound{U} \AgdaBound{V}\AgdaSymbol{\}} \AgdaSymbol{\{}\AgdaArgument{C} \AgdaSymbol{=} \AgdaFunction{OP} \AgdaBound{H} \AgdaBound{U} \AgdaBound{W}\AgdaSymbol{\}} \AgdaField{\_∘\_} \AgdaFunction{comp-cong}\<%
\\
%
\\
\>[0]\AgdaIndent{2}{}\<[2]%
\>[2]\AgdaFunction{comp-congr} \AgdaSymbol{:} \AgdaSymbol{∀} \AgdaSymbol{\{}\AgdaBound{U}\AgdaSymbol{\}} \AgdaSymbol{\{}\AgdaBound{V}\AgdaSymbol{\}} \AgdaSymbol{\{}\AgdaBound{W}\AgdaSymbol{\}} \AgdaSymbol{\{}\AgdaBound{σ} \AgdaSymbol{:} \AgdaFunction{Op} \AgdaBound{F} \AgdaBound{V} \AgdaBound{W}\AgdaSymbol{\}} \AgdaSymbol{\{}\AgdaBound{ρ} \AgdaBound{ρ'} \AgdaSymbol{:} \AgdaFunction{Op} \AgdaBound{G} \AgdaBound{U} \AgdaBound{V}\AgdaSymbol{\}} \AgdaSymbol{→}\<%
\\
\>[2]\AgdaIndent{4}{}\<[4]%
\>[4]\AgdaFunction{\_∼op\_} \AgdaBound{G} \AgdaBound{ρ} \AgdaBound{ρ'} \AgdaSymbol{→} \AgdaFunction{\_∼op\_} \AgdaBound{H} \AgdaSymbol{(}\AgdaBound{σ} \AgdaField{∘} \AgdaBound{ρ}\AgdaSymbol{)} \AgdaSymbol{(}\AgdaBound{σ} \AgdaField{∘} \AgdaBound{ρ'}\AgdaSymbol{)}\<%
\\
\>[0]\AgdaIndent{2}{}\<[2]%
\>[2]\AgdaFunction{comp-congr} \AgdaSymbol{\{}\AgdaBound{U}\AgdaSymbol{\}} \AgdaSymbol{\{}\AgdaBound{V}\AgdaSymbol{\}} \AgdaSymbol{\{}\AgdaBound{W}\AgdaSymbol{\}} \AgdaSymbol{=} \AgdaFunction{Bifunction.congr} \AgdaSymbol{\{}\AgdaArgument{A} \AgdaSymbol{=} \AgdaFunction{OP} \AgdaBound{F} \AgdaBound{V} \AgdaBound{W}\AgdaSymbol{\}} \AgdaSymbol{\{}\AgdaArgument{B} \AgdaSymbol{=} \AgdaFunction{OP} \AgdaBound{G} \AgdaBound{U} \AgdaBound{V}\AgdaSymbol{\}} \AgdaSymbol{\{}\AgdaArgument{C} \AgdaSymbol{=} \AgdaFunction{OP} \AgdaBound{H} \AgdaBound{U} \AgdaBound{W}\AgdaSymbol{\}} \AgdaField{\_∘\_} \AgdaFunction{comp-cong}\<%
\end{code}
}

\begin{code}%
\>[0]\AgdaIndent{2}{}\<[2]%
\>[2]\AgdaFunction{liftsOp-comp} \AgdaSymbol{:} \AgdaSymbol{∀} \AgdaSymbol{\{}\AgdaBound{U} \AgdaBound{V} \AgdaBound{W}\AgdaSymbol{\}} \AgdaBound{A} \AgdaSymbol{\{}\AgdaBound{σ} \AgdaBound{ρ}\AgdaSymbol{\}} \AgdaSymbol{→} \<[37]%
\>[37]\<%
\\
\>[2]\AgdaIndent{4}{}\<[4]%
\>[4]\AgdaFunction{\_∼op\_} \AgdaBound{H} \AgdaSymbol{(}\AgdaFunction{liftsOp} \AgdaBound{H} \AgdaBound{A} \AgdaSymbol{(}\AgdaField{\_∘\_} \AgdaSymbol{\{}\AgdaBound{U}\AgdaSymbol{\}} \AgdaSymbol{\{}\AgdaBound{V}\AgdaSymbol{\}} \AgdaSymbol{\{}\AgdaBound{W}\AgdaSymbol{\}} \AgdaBound{σ} \AgdaBound{ρ}\AgdaSymbol{))} \<[48]%
\>[48]\<%
\\
\>[4]\AgdaIndent{6}{}\<[6]%
\>[6]\AgdaSymbol{(}\AgdaFunction{liftsOp} \AgdaBound{F} \AgdaBound{A} \AgdaBound{σ} \AgdaField{∘} \AgdaFunction{liftsOp} \AgdaBound{G} \AgdaBound{A} \AgdaBound{ρ}\AgdaSymbol{)}\<%
\end{code}

\AgdaHide{
\begin{code}%
\>[0]\AgdaIndent{2}{}\<[2]%
\>[2]\AgdaFunction{liftsOp-comp} \AgdaInductiveConstructor{[]} \AgdaSymbol{=} \AgdaFunction{∼-refl} \AgdaBound{H}\<%
\\
\>[0]\AgdaIndent{2}{}\<[2]%
\>[2]\AgdaFunction{liftsOp-comp} \AgdaSymbol{\{}\AgdaBound{U}\AgdaSymbol{\}} \AgdaSymbol{\{}\AgdaBound{V}\AgdaSymbol{\}} \AgdaSymbol{\{}\AgdaBound{W}\AgdaSymbol{\}} \AgdaSymbol{(}\AgdaBound{K} \AgdaInductiveConstructor{∷} \AgdaBound{A}\AgdaSymbol{)} \AgdaSymbol{\{}\AgdaBound{σ}\AgdaSymbol{\}} \AgdaSymbol{\{}\AgdaBound{ρ}\AgdaSymbol{\}} \AgdaSymbol{=} \AgdaKeyword{let} \AgdaKeyword{open} \AgdaModule{EqReasoning} \AgdaSymbol{(}\AgdaFunction{OP} \AgdaBound{H} \AgdaSymbol{\_} \AgdaSymbol{\_)} \AgdaKeyword{in} \<[80]%
\>[80]\<%
\\
\>[2]\AgdaIndent{4}{}\<[4]%
\>[4]\AgdaFunction{begin}\<%
\\
\>[4]\AgdaIndent{6}{}\<[6]%
\>[6]\AgdaFunction{liftsOp} \AgdaBound{H} \AgdaBound{A} \AgdaSymbol{(}\AgdaFunction{liftOp} \AgdaBound{H} \AgdaBound{K} \AgdaSymbol{(}\AgdaBound{σ} \AgdaField{∘} \AgdaBound{ρ}\AgdaSymbol{))}\<%
\\
\>[0]\AgdaIndent{4}{}\<[4]%
\>[4]\AgdaFunction{≈⟨} \AgdaFunction{liftsOp-cong} \AgdaBound{H} \AgdaBound{A} \AgdaField{liftOp-comp} \AgdaFunction{⟩}\<%
\\
\>[4]\AgdaIndent{6}{}\<[6]%
\>[6]\AgdaFunction{liftsOp} \AgdaBound{H} \AgdaBound{A} \AgdaSymbol{(}\AgdaFunction{liftOp} \AgdaBound{F} \AgdaBound{K} \AgdaBound{σ} \AgdaField{∘} \AgdaFunction{liftOp} \AgdaBound{G} \AgdaBound{K} \AgdaBound{ρ}\AgdaSymbol{)}\<%
\\
\>[0]\AgdaIndent{4}{}\<[4]%
\>[4]\AgdaFunction{≈⟨} \AgdaFunction{liftsOp-comp} \AgdaBound{A} \AgdaFunction{⟩}\<%
\\
\>[4]\AgdaIndent{6}{}\<[6]%
\>[6]\AgdaFunction{liftsOp} \AgdaBound{F} \AgdaBound{A} \AgdaSymbol{(}\AgdaFunction{liftOp} \AgdaBound{F} \AgdaBound{K} \AgdaBound{σ}\AgdaSymbol{)} \AgdaField{∘} \AgdaFunction{liftsOp} \AgdaBound{G} \AgdaBound{A} \AgdaSymbol{(}\AgdaFunction{liftOp} \AgdaBound{G} \AgdaBound{K} \AgdaBound{ρ}\AgdaSymbol{)}\<%
\\
\>[0]\AgdaIndent{4}{}\<[4]%
\>[4]\AgdaFunction{∎}\<%
\end{code}
}

\begin{code}%
\>[0]\AgdaIndent{2}{}\<[2]%
\>[2]\AgdaFunction{ap-comp} \AgdaSymbol{:} \AgdaSymbol{∀} \AgdaSymbol{\{}\AgdaBound{U} \AgdaBound{V} \AgdaBound{W} \AgdaBound{C} \AgdaBound{K}\AgdaSymbol{\}} \AgdaSymbol{(}\AgdaBound{E} \AgdaSymbol{:} \AgdaDatatype{Subexpression} \AgdaBound{U} \AgdaBound{C} \AgdaBound{K}\AgdaSymbol{)} \AgdaSymbol{\{}\AgdaBound{σ} \AgdaBound{ρ}\AgdaSymbol{\}} \AgdaSymbol{→} \<[60]%
\>[60]\<%
\\
\>[2]\AgdaIndent{4}{}\<[4]%
\>[4]\AgdaFunction{ap} \AgdaBound{H} \AgdaSymbol{(}\AgdaField{\_∘\_} \AgdaSymbol{\{}\AgdaBound{U}\AgdaSymbol{\}} \AgdaSymbol{\{}\AgdaBound{V}\AgdaSymbol{\}} \AgdaSymbol{\{}\AgdaBound{W}\AgdaSymbol{\}} \AgdaBound{σ} \AgdaBound{ρ}\AgdaSymbol{)} \AgdaBound{E} \AgdaDatatype{≡} \AgdaFunction{ap} \AgdaBound{F} \AgdaBound{σ} \AgdaSymbol{(}\AgdaFunction{ap} \AgdaBound{G} \AgdaBound{ρ} \AgdaBound{E}\AgdaSymbol{)}\<%
\end{code}

\AgdaHide{
\begin{code}%
\>[0]\AgdaIndent{2}{}\<[2]%
\>[2]\AgdaFunction{ap-comp} \AgdaSymbol{(}\AgdaInductiveConstructor{var} \AgdaSymbol{\_)} \AgdaSymbol{=} \AgdaField{apV-comp}\<%
\\
\>[0]\AgdaIndent{2}{}\<[2]%
\>[2]\AgdaFunction{ap-comp} \AgdaSymbol{(}\AgdaInductiveConstructor{app} \AgdaBound{c} \AgdaBound{E}\AgdaSymbol{)} \AgdaSymbol{=} \AgdaFunction{cong} \AgdaSymbol{(}\AgdaInductiveConstructor{app} \AgdaBound{c}\AgdaSymbol{)} \AgdaSymbol{(}\AgdaFunction{ap-comp} \AgdaBound{E}\AgdaSymbol{)}\<%
\\
\>[0]\AgdaIndent{2}{}\<[2]%
\>[2]\AgdaFunction{ap-comp} \AgdaInductiveConstructor{[]} \AgdaSymbol{=} \AgdaInductiveConstructor{refl}\<%
\\
\>[0]\AgdaIndent{2}{}\<[2]%
\>[2]\AgdaFunction{ap-comp} \AgdaSymbol{(}\AgdaInductiveConstructor{\_∷\_} \AgdaSymbol{\{}\AgdaArgument{A} \AgdaSymbol{=} \AgdaInductiveConstructor{SK} \AgdaBound{A} \AgdaSymbol{\_\}} \AgdaBound{E} \AgdaBound{E'}\AgdaSymbol{)} \AgdaSymbol{\{}\AgdaBound{σ}\AgdaSymbol{\}} \AgdaSymbol{\{}\AgdaBound{ρ}\AgdaSymbol{\}} \AgdaSymbol{=} \AgdaFunction{cong₂} \AgdaInductiveConstructor{\_∷\_}\<%
\\
\>[2]\AgdaIndent{4}{}\<[4]%
\>[4]\AgdaSymbol{(}\AgdaKeyword{let} \AgdaKeyword{open} \AgdaModule{≡-Reasoning} \AgdaKeyword{in} \<[29]%
\>[29]\<%
\\
\>[2]\AgdaIndent{4}{}\<[4]%
\>[4]\AgdaFunction{begin}\<%
\\
\>[4]\AgdaIndent{6}{}\<[6]%
\>[6]\AgdaFunction{ap} \AgdaBound{H} \AgdaSymbol{(}\AgdaFunction{liftsOp} \AgdaBound{H} \AgdaBound{A} \AgdaSymbol{(}\AgdaBound{σ} \AgdaField{∘} \AgdaBound{ρ}\AgdaSymbol{))} \AgdaBound{E}\<%
\\
\>[0]\AgdaIndent{4}{}\<[4]%
\>[4]\AgdaFunction{≡⟨} \AgdaFunction{ap-congl} \AgdaBound{H} \AgdaSymbol{(}\AgdaFunction{liftsOp-comp} \AgdaBound{A}\AgdaSymbol{)} \AgdaBound{E} \AgdaFunction{⟩}\<%
\\
\>[4]\AgdaIndent{6}{}\<[6]%
\>[6]\AgdaFunction{ap} \AgdaBound{H} \AgdaSymbol{(}\AgdaFunction{liftsOp} \AgdaBound{F} \AgdaBound{A} \AgdaBound{σ} \AgdaField{∘} \AgdaFunction{liftsOp} \AgdaBound{G} \AgdaBound{A} \AgdaBound{ρ}\AgdaSymbol{)} \AgdaBound{E}\<%
\\
\>[0]\AgdaIndent{4}{}\<[4]%
\>[4]\AgdaFunction{≡⟨} \AgdaFunction{ap-comp} \AgdaBound{E} \AgdaFunction{⟩}\<%
\\
\>[4]\AgdaIndent{6}{}\<[6]%
\>[6]\AgdaFunction{ap} \AgdaBound{F} \AgdaSymbol{(}\AgdaFunction{liftsOp} \AgdaBound{F} \AgdaBound{A} \AgdaBound{σ}\AgdaSymbol{)} \AgdaSymbol{(}\AgdaFunction{ap} \AgdaBound{G} \AgdaSymbol{(}\AgdaFunction{liftsOp} \AgdaBound{G} \AgdaBound{A} \AgdaBound{ρ}\AgdaSymbol{)} \AgdaBound{E}\AgdaSymbol{)}\<%
\\
\>[0]\AgdaIndent{4}{}\<[4]%
\>[4]\AgdaFunction{∎}\AgdaSymbol{)} \<[7]%
\>[7]\<%
\\
\>[0]\AgdaIndent{4}{}\<[4]%
\>[4]\AgdaSymbol{(}\AgdaFunction{ap-comp} \AgdaBound{E'}\AgdaSymbol{)}\<%
\end{code}
}

\begin{lm}
Let $\circ_1 : F;G \rightarrow H$ and $\circ_2 : F';G' \rightarrow H$.  If
\[ \sigma \circ_1 \rho \sim \simga' \circ_2 \rho' \]
then $E [\rho] [\sigma] \equiv E [\rho'] [\sigma']$ for every expression $E$.
\end{lm}

\begin{code}%
\>\AgdaFunction{ap-comp-sim} \AgdaSymbol{:} \AgdaSymbol{∀} \AgdaSymbol{\{}\AgdaBound{F} \AgdaBound{F'} \AgdaBound{G} \AgdaBound{G'} \AgdaBound{H}\AgdaSymbol{\}} \AgdaSymbol{(}\AgdaBound{comp₁} \AgdaSymbol{:} \AgdaRecord{Composition} \AgdaBound{F} \AgdaBound{G} \AgdaBound{H}\AgdaSymbol{)} \AgdaSymbol{(}\AgdaBound{comp₂} \AgdaSymbol{:} \AgdaRecord{Composition} \AgdaBound{F'} \AgdaBound{G'} \AgdaBound{H}\AgdaSymbol{)} \AgdaSymbol{\{}\AgdaBound{U}\AgdaSymbol{\}} \AgdaSymbol{\{}\AgdaBound{V}\AgdaSymbol{\}} \AgdaSymbol{\{}\AgdaBound{V'}\AgdaSymbol{\}} \AgdaSymbol{\{}\AgdaBound{W}\AgdaSymbol{\}}\<%
\\
\>[0]\AgdaIndent{2}{}\<[2]%
\>[2]\AgdaSymbol{\{}\AgdaBound{σ} \AgdaSymbol{:} \AgdaFunction{Op} \AgdaBound{F} \AgdaBound{V} \AgdaBound{W}\AgdaSymbol{\}} \AgdaSymbol{\{}\AgdaBound{ρ} \AgdaSymbol{:} \AgdaFunction{Op} \AgdaBound{G} \AgdaBound{U} \AgdaBound{V}\AgdaSymbol{\}} \AgdaSymbol{\{}\AgdaBound{σ'} \AgdaSymbol{:} \AgdaFunction{Op} \AgdaBound{F'} \AgdaBound{V'} \AgdaBound{W}\AgdaSymbol{\}} \AgdaSymbol{\{}\AgdaBound{ρ'} \AgdaSymbol{:} \AgdaFunction{Op} \AgdaBound{G'} \AgdaBound{U} \AgdaBound{V'}\AgdaSymbol{\}} \AgdaSymbol{→}\<%
\\
\>[0]\AgdaIndent{2}{}\<[2]%
\>[2]\AgdaFunction{\_∼op\_} \AgdaBound{H} \AgdaSymbol{(}\AgdaField{Composition.\_∘\_} \AgdaBound{comp₁} \AgdaBound{σ} \AgdaBound{ρ}\AgdaSymbol{)} \AgdaSymbol{(}\AgdaField{Composition.\_∘\_} \AgdaBound{comp₂} \AgdaBound{σ'} \AgdaBound{ρ'}\AgdaSymbol{)} \AgdaSymbol{→}\<%
\\
\>[0]\AgdaIndent{2}{}\<[2]%
\>[2]\AgdaSymbol{∀} \AgdaSymbol{\{}\AgdaBound{C}\AgdaSymbol{\}} \AgdaSymbol{\{}\AgdaBound{K}\AgdaSymbol{\}} \AgdaSymbol{(}\AgdaBound{E} \AgdaSymbol{:} \AgdaDatatype{Subexpression} \AgdaBound{U} \AgdaBound{C} \AgdaBound{K}\AgdaSymbol{)} \AgdaSymbol{→}\<%
\\
\>[0]\AgdaIndent{2}{}\<[2]%
\>[2]\AgdaFunction{ap} \AgdaBound{F} \AgdaBound{σ} \AgdaSymbol{(}\AgdaFunction{ap} \AgdaBound{G} \AgdaBound{ρ} \AgdaBound{E}\AgdaSymbol{)} \AgdaDatatype{≡} \AgdaFunction{ap} \AgdaBound{F'} \AgdaBound{σ'} \AgdaSymbol{(}\AgdaFunction{ap} \AgdaBound{G'} \AgdaBound{ρ'} \AgdaBound{E}\AgdaSymbol{)}\<%
\end{code}

\AgdaHide{
\begin{code}%
\>\AgdaFunction{ap-comp-sim} \AgdaSymbol{\{}\AgdaBound{F}\AgdaSymbol{\}} \AgdaSymbol{\{}\AgdaBound{F'}\AgdaSymbol{\}} \AgdaSymbol{\{}\AgdaBound{G}\AgdaSymbol{\}} \AgdaSymbol{\{}\AgdaBound{G'}\AgdaSymbol{\}} \AgdaSymbol{\{}\AgdaBound{H}\AgdaSymbol{\}} \AgdaBound{comp₁} \AgdaBound{comp₂} \AgdaSymbol{\{}\AgdaBound{U}\AgdaSymbol{\}} \AgdaSymbol{\{}\AgdaBound{V}\AgdaSymbol{\}} \AgdaSymbol{\{}\AgdaBound{V'}\AgdaSymbol{\}} \AgdaSymbol{\{}\AgdaBound{W}\AgdaSymbol{\}} \AgdaSymbol{\{}\AgdaBound{σ}\AgdaSymbol{\}} \AgdaSymbol{\{}\AgdaBound{ρ}\AgdaSymbol{\}} \AgdaSymbol{\{}\AgdaBound{σ'}\AgdaSymbol{\}} \AgdaSymbol{\{}\AgdaBound{ρ'}\AgdaSymbol{\}} \AgdaBound{hyp} \AgdaSymbol{\{}\AgdaBound{C}\AgdaSymbol{\}} \AgdaSymbol{\{}\AgdaBound{K}\AgdaSymbol{\}} \AgdaBound{E} \AgdaSymbol{=}\<%
\\
\>[0]\AgdaIndent{2}{}\<[2]%
\>[2]\AgdaKeyword{let} \AgdaKeyword{open} \AgdaModule{≡-Reasoning} \AgdaKeyword{in} \<[26]%
\>[26]\<%
\\
\>[0]\AgdaIndent{2}{}\<[2]%
\>[2]\AgdaFunction{begin}\<%
\\
\>[2]\AgdaIndent{4}{}\<[4]%
\>[4]\AgdaFunction{ap} \AgdaBound{F} \AgdaBound{σ} \AgdaSymbol{(}\AgdaFunction{ap} \AgdaBound{G} \AgdaBound{ρ} \AgdaBound{E}\AgdaSymbol{)}\<%
\\
\>[0]\AgdaIndent{2}{}\<[2]%
\>[2]\AgdaFunction{≡⟨⟨} \AgdaFunction{Composition.ap-comp} \AgdaBound{comp₁} \AgdaBound{E} \AgdaSymbol{\{}\AgdaBound{σ}\AgdaSymbol{\}} \AgdaSymbol{\{}\AgdaBound{ρ}\AgdaSymbol{\}} \AgdaFunction{⟩⟩}\<%
\\
\>[2]\AgdaIndent{4}{}\<[4]%
\>[4]\AgdaFunction{ap} \AgdaBound{H} \AgdaSymbol{(}\AgdaField{Composition.\_∘\_} \AgdaBound{comp₁} \AgdaBound{σ} \AgdaBound{ρ}\AgdaSymbol{)} \AgdaBound{E}\<%
\\
\>[0]\AgdaIndent{2}{}\<[2]%
\>[2]\AgdaFunction{≡⟨} \AgdaFunction{ap-congl} \AgdaBound{H} \AgdaBound{hyp} \AgdaBound{E} \AgdaFunction{⟩}\<%
\\
\>[2]\AgdaIndent{4}{}\<[4]%
\>[4]\AgdaFunction{ap} \AgdaBound{H} \AgdaSymbol{(}\AgdaField{Composition.\_∘\_} \AgdaBound{comp₂} \AgdaBound{σ'} \AgdaBound{ρ'}\AgdaSymbol{)} \AgdaBound{E}\<%
\\
\>[0]\AgdaIndent{2}{}\<[2]%
\>[2]\AgdaFunction{≡⟨} \AgdaFunction{Composition.ap-comp} \AgdaBound{comp₂} \AgdaBound{E} \AgdaSymbol{\{}\AgdaBound{σ'}\AgdaSymbol{\}} \AgdaSymbol{\{}\AgdaBound{ρ'}\AgdaSymbol{\}} \AgdaFunction{⟩}\<%
\\
\>[2]\AgdaIndent{4}{}\<[4]%
\>[4]\AgdaFunction{ap} \AgdaBound{F'} \AgdaBound{σ'} \AgdaSymbol{(}\AgdaFunction{ap} \AgdaBound{G'} \AgdaBound{ρ'} \AgdaBound{E}\AgdaSymbol{)}\<%
\\
\>[0]\AgdaIndent{2}{}\<[2]%
\>[2]\AgdaFunction{∎}\<%
\end{code}
}

\begin{lm}
Suppose there exist compositions $F;G \rightarrow H$ and $F';F \rightarrow H$.
Let $\uparrow_F$, $\uparrow_{F'}$ and $\uparrow_G$ be the lifting operations of $F$, $F'$ and $G$.
Suppose $\up_F(E) \equiv \up_{F'}(E)$ for every subexpression $E$.  Then
$\uparrow_G(E)[F \uparrow] \equiv \uparrow_{F'}(\sigma(E))$ for every subexpression $E$.
\end{lm}

\begin{code}%
\>\AgdaFunction{liftOp-up-mixed} \AgdaSymbol{:} \AgdaSymbol{∀} \AgdaSymbol{\{}\AgdaBound{F}\AgdaSymbol{\}} \AgdaSymbol{\{}\AgdaBound{G}\AgdaSymbol{\}} \AgdaSymbol{\{}\AgdaBound{H}\AgdaSymbol{\}} \AgdaSymbol{\{}\AgdaBound{F'}\AgdaSymbol{\}} \AgdaSymbol{(}\AgdaBound{comp₁} \AgdaSymbol{:} \AgdaRecord{Composition} \AgdaBound{F} \AgdaBound{G} \AgdaBound{H}\AgdaSymbol{)} \AgdaSymbol{(}\AgdaBound{comp₂} \AgdaSymbol{:} \AgdaRecord{Composition} \AgdaBound{F'} \AgdaBound{F} \AgdaBound{H}\AgdaSymbol{)}\<%
\\
\>[0]\AgdaIndent{2}{}\<[2]%
\>[2]\AgdaSymbol{\{}\AgdaBound{U}\AgdaSymbol{\}} \AgdaSymbol{\{}\AgdaBound{V}\AgdaSymbol{\}} \AgdaSymbol{\{}\AgdaBound{C}\AgdaSymbol{\}} \AgdaSymbol{\{}\AgdaBound{K}\AgdaSymbol{\}} \AgdaSymbol{\{}\AgdaBound{L}\AgdaSymbol{\}} \AgdaSymbol{\{}\AgdaBound{σ} \AgdaSymbol{:} \AgdaFunction{Op} \AgdaBound{F} \AgdaBound{U} \AgdaBound{V}\AgdaSymbol{\}} \AgdaSymbol{→}\<%
\\
\>[0]\AgdaIndent{2}{}\<[2]%
\>[2]\AgdaSymbol{(∀} \AgdaSymbol{\{}\AgdaBound{V}\AgdaSymbol{\}} \AgdaSymbol{\{}\AgdaBound{C}\AgdaSymbol{\}} \AgdaSymbol{\{}\AgdaBound{K}\AgdaSymbol{\}} \AgdaSymbol{\{}\AgdaBound{L}\AgdaSymbol{\}} \AgdaSymbol{\{}\AgdaBound{E} \AgdaSymbol{:} \AgdaDatatype{Subexpression} \AgdaBound{V} \AgdaBound{C} \AgdaBound{K}\AgdaSymbol{\}} \AgdaSymbol{→} \AgdaFunction{ap} \AgdaBound{F} \AgdaSymbol{(}\AgdaFunction{up} \AgdaBound{F} \AgdaSymbol{\{}\AgdaBound{V}\AgdaSymbol{\}} \AgdaSymbol{\{}\AgdaBound{L}\AgdaSymbol{\})} \AgdaBound{E} \AgdaDatatype{≡} \AgdaFunction{ap} \AgdaBound{F'} \AgdaSymbol{(}\AgdaFunction{up} \AgdaBound{F'} \AgdaSymbol{\{}\AgdaBound{V}\AgdaSymbol{\}} \AgdaSymbol{\{}\AgdaBound{L}\AgdaSymbol{\})} \AgdaBound{E}\AgdaSymbol{)} \AgdaSymbol{→}\<%
\\
\>[0]\AgdaIndent{2}{}\<[2]%
\>[2]\AgdaSymbol{∀} \AgdaSymbol{\{}\AgdaBound{E} \AgdaSymbol{:} \AgdaDatatype{Subexpression} \AgdaBound{U} \AgdaBound{C} \AgdaBound{K}\AgdaSymbol{\}} \AgdaSymbol{→} \AgdaFunction{ap} \AgdaBound{F} \AgdaSymbol{(}\AgdaFunction{liftOp} \AgdaBound{F} \AgdaBound{L} \AgdaBound{σ}\AgdaSymbol{)} \AgdaSymbol{(}\AgdaFunction{ap} \AgdaBound{G} \AgdaSymbol{(}\AgdaFunction{up} \AgdaBound{G}\AgdaSymbol{)} \AgdaBound{E}\AgdaSymbol{)} \AgdaDatatype{≡} \AgdaFunction{ap} \AgdaBound{F'} \AgdaSymbol{(}\AgdaFunction{up} \AgdaBound{F'}\AgdaSymbol{)} \AgdaSymbol{(}\AgdaFunction{ap} \AgdaBound{F} \AgdaBound{σ} \AgdaBound{E}\AgdaSymbol{)}\<%
\\
\>\AgdaFunction{liftOp-up-mixed} \AgdaSymbol{\{}\AgdaBound{F}\AgdaSymbol{\}} \AgdaSymbol{\{}\AgdaBound{G}\AgdaSymbol{\}} \AgdaSymbol{\{}\AgdaBound{H}\AgdaSymbol{\}} \AgdaSymbol{\{}\AgdaBound{F'}\AgdaSymbol{\}} \AgdaBound{comp₁} \AgdaBound{comp₂} \AgdaSymbol{\{}\AgdaBound{U}\AgdaSymbol{\}} \AgdaSymbol{\{}\AgdaBound{V}\AgdaSymbol{\}} \AgdaSymbol{\{}\AgdaBound{C}\AgdaSymbol{\}} \AgdaSymbol{\{}\AgdaBound{K}\AgdaSymbol{\}} \AgdaSymbol{\{}\AgdaBound{L}\AgdaSymbol{\}} \AgdaSymbol{\{}\AgdaBound{σ}\AgdaSymbol{\}} \AgdaBound{hyp} \AgdaSymbol{\{}\AgdaArgument{E} \AgdaSymbol{=} \AgdaBound{E}\AgdaSymbol{\}} \AgdaSymbol{=} \AgdaFunction{ap-comp-sim} \AgdaBound{comp₁} \AgdaBound{comp₂} \<[107]%
\>[107]\<%
\\
\>[0]\AgdaIndent{2}{}\<[2]%
\>[2]\AgdaSymbol{(λ} \AgdaBound{x} \AgdaSymbol{→} \AgdaKeyword{let} \AgdaKeyword{open} \AgdaModule{≡-Reasoning} \AgdaKeyword{in} \<[33]%
\>[33]\<%
\\
\>[0]\AgdaIndent{2}{}\<[2]%
\>[2]\AgdaFunction{begin}\<%
\\
\>[2]\AgdaIndent{4}{}\<[4]%
\>[4]\AgdaFunction{apV} \AgdaBound{H} \AgdaSymbol{(}\AgdaField{Composition.\_∘\_} \AgdaBound{comp₁} \AgdaSymbol{(}\AgdaFunction{liftOp} \AgdaBound{F} \AgdaBound{L} \AgdaBound{σ}\AgdaSymbol{)} \AgdaSymbol{(}\AgdaFunction{up} \AgdaBound{G}\AgdaSymbol{))} \AgdaBound{x}\<%
\\
\>[0]\AgdaIndent{2}{}\<[2]%
\>[2]\AgdaFunction{≡⟨} \AgdaField{Composition.apV-comp} \AgdaBound{comp₁} \AgdaFunction{⟩}\<%
\\
\>[2]\AgdaIndent{4}{}\<[4]%
\>[4]\AgdaFunction{ap} \AgdaBound{F} \AgdaSymbol{(}\AgdaFunction{liftOp} \AgdaBound{F} \AgdaBound{L} \AgdaBound{σ}\AgdaSymbol{)} \AgdaSymbol{(}\AgdaFunction{apV} \AgdaBound{G} \AgdaSymbol{(}\AgdaFunction{up} \AgdaBound{G}\AgdaSymbol{)} \AgdaBound{x}\AgdaSymbol{)}\<%
\\
\>[0]\AgdaIndent{2}{}\<[2]%
\>[2]\AgdaFunction{≡⟨} \AgdaFunction{cong} \AgdaSymbol{(}\AgdaFunction{ap} \AgdaBound{F} \AgdaSymbol{(}\AgdaFunction{liftOp} \AgdaBound{F} \AgdaBound{L} \AgdaBound{σ}\AgdaSymbol{))} \AgdaSymbol{(}\AgdaFunction{apV-up} \AgdaBound{G}\AgdaSymbol{)} \AgdaFunction{⟩}\<%
\\
\>[2]\AgdaIndent{4}{}\<[4]%
\>[4]\AgdaFunction{apV} \AgdaBound{F} \AgdaSymbol{(}\AgdaFunction{liftOp} \AgdaBound{F} \AgdaBound{L} \AgdaBound{σ}\AgdaSymbol{)} \AgdaSymbol{(}\AgdaInductiveConstructor{↑} \AgdaBound{x}\AgdaSymbol{)}\<%
\\
\>[0]\AgdaIndent{2}{}\<[2]%
\>[2]\AgdaFunction{≡⟨} \AgdaFunction{liftOp-↑} \AgdaBound{F} \AgdaBound{x} \AgdaFunction{⟩}\<%
\\
\>[2]\AgdaIndent{4}{}\<[4]%
\>[4]\AgdaFunction{ap} \AgdaBound{F} \AgdaSymbol{(}\AgdaFunction{up} \AgdaBound{F}\AgdaSymbol{)} \AgdaSymbol{(}\AgdaFunction{apV} \AgdaBound{F} \AgdaBound{σ} \AgdaBound{x}\AgdaSymbol{)}\<%
\\
\>[0]\AgdaIndent{2}{}\<[2]%
\>[2]\AgdaFunction{≡⟨} \AgdaBound{hyp} \AgdaSymbol{\{}\AgdaArgument{E} \AgdaSymbol{=} \AgdaFunction{apV} \AgdaBound{F} \AgdaBound{σ} \AgdaBound{x}\AgdaSymbol{\}}\AgdaFunction{⟩}\<%
\\
\>[2]\AgdaIndent{4}{}\<[4]%
\>[4]\AgdaFunction{ap} \AgdaBound{F'} \AgdaSymbol{(}\AgdaFunction{up} \AgdaBound{F'}\AgdaSymbol{)} \AgdaSymbol{(}\AgdaFunction{apV} \AgdaBound{F} \AgdaBound{σ} \AgdaBound{x}\AgdaSymbol{)}\<%
\\
\>[0]\AgdaIndent{2}{}\<[2]%
\>[2]\AgdaFunction{≡⟨⟨} \AgdaField{Composition.apV-comp} \AgdaBound{comp₂} \AgdaFunction{⟩⟩}\<%
\\
\>[2]\AgdaIndent{4}{}\<[4]%
\>[4]\AgdaFunction{apV} \AgdaBound{H} \AgdaSymbol{(}\AgdaField{Composition.\_∘\_} \AgdaBound{comp₂} \AgdaSymbol{(}\AgdaFunction{up} \AgdaBound{F'}\AgdaSymbol{)} \AgdaBound{σ}\AgdaSymbol{)} \AgdaBound{x}\<%
\\
\>[0]\AgdaIndent{2}{}\<[2]%
\>[2]\AgdaFunction{∎}\AgdaSymbol{)} \<[5]%
\>[5]\<%
\\
\>[0]\AgdaIndent{2}{}\<[2]%
\>[2]\AgdaBound{E}\<%
\end{code}
}

\AgdaHide{
\begin{code}%
\>\AgdaKeyword{open} \AgdaKeyword{import} \AgdaModule{Grammar.Base}\<%
\\
%
\\
\>\AgdaKeyword{module} \AgdaModule{Grammar.OpFamily.OpFamily} \AgdaSymbol{(}\AgdaBound{G} \AgdaSymbol{:} \AgdaRecord{Grammar}\AgdaSymbol{)} \AgdaKeyword{where}\<%
\\
%
\\
\>\AgdaKeyword{open} \AgdaKeyword{import} \AgdaModule{Prelims}\<%
\\
\>\AgdaKeyword{open} \AgdaModule{Grammar} \AgdaBound{G}\<%
\\
\>\AgdaKeyword{open} \AgdaKeyword{import} \AgdaModule{Grammar.OpFamily.LiftFamily} \AgdaBound{G}\<%
\\
\>\AgdaKeyword{open} \AgdaKeyword{import} \AgdaModule{Grammar.OpFamily.Composition} \AgdaBound{G}\<%
\end{code}
}

\subsubsection{Family of Operations}

Finally. we can define: a \emph{family of operations} is a pre-family with lift $F$ together with a composition $\circ : F;F \rightarrow F$.

\begin{code}%
\>\AgdaKeyword{record} \AgdaRecord{IsOpFamily} \AgdaSymbol{(}\AgdaBound{F} \AgdaSymbol{:} \AgdaRecord{LiftFamily}\AgdaSymbol{)} \AgdaSymbol{:} \AgdaPrimitiveType{Set₂} \AgdaKeyword{where}\<%
\\
\>[0]\AgdaIndent{2}{}\<[2]%
\>[2]\AgdaKeyword{open} \AgdaModule{LiftFamily} \AgdaBound{F} \AgdaKeyword{public}\<%
\\
\>[0]\AgdaIndent{2}{}\<[2]%
\>[2]\AgdaKeyword{field}\<%
\\
\>[2]\AgdaIndent{4}{}\<[4]%
\>[4]\AgdaField{comp} \AgdaSymbol{:} \AgdaRecord{Composition} \AgdaBound{F} \AgdaBound{F} \AgdaBound{F}\<%
\\
%
\\
\>\AgdaComment{\{-  infix 50 \_∘\_\<\\
\>  field\<\\
\>    \_∘\_ : ∀ \{U\} \{V\} \{W\} → Op V W → Op U V → Op U W\<\\
\>    liftOp-comp : ∀ \{U\} \{V\} \{W\} \{K\} \{σ : Op V W\} \{ρ : Op U V\} →\<\\
\>      liftOp K (σ ∘ ρ) ∼op liftOp K σ ∘ liftOp K ρ\<\\
\>    apV-comp : ∀ \{U\} \{V\} \{W\} \{K\} \{σ : Op V W\} \{ρ : Op U V\} \{x : Var U K\} →\<\\
\>      apV (σ ∘ ρ) x ≡ ap σ (apV ρ x)\<\\
\>\<\\
\>  COMP : Composition F F F\<\\
\>  COMP = record \{ \<\\
\>    circ = \_∘\_ ; \<\\
\>    liftOp-circ = liftOp-comp ; \<\\
\>    apV-circ = apV-comp \} -\}}\<%
\\
%
\\
\>[0]\AgdaIndent{2}{}\<[2]%
\>[2]\AgdaKeyword{open} \AgdaModule{Composition} \AgdaField{comp} \AgdaKeyword{public}\<%
\\
%
\\
\>[0]\AgdaIndent{2}{}\<[2]%
\>[2]\AgdaFunction{comp-congl} \AgdaSymbol{:} \AgdaSymbol{∀} \AgdaSymbol{\{}\AgdaBound{U}\AgdaSymbol{\}} \AgdaSymbol{\{}\AgdaBound{V}\AgdaSymbol{\}} \AgdaSymbol{\{}\AgdaBound{W}\AgdaSymbol{\}} \AgdaSymbol{\{}\AgdaBound{σ} \AgdaBound{σ'} \AgdaSymbol{:} \AgdaFunction{Op} \AgdaBound{V} \AgdaBound{W}\AgdaSymbol{\}} \AgdaSymbol{\{}\AgdaBound{ρ} \AgdaSymbol{:} \AgdaFunction{Op} \AgdaBound{U} \AgdaBound{V}\AgdaSymbol{\}} \AgdaSymbol{→}\<%
\\
\>[2]\AgdaIndent{4}{}\<[4]%
\>[4]\AgdaBound{σ} \AgdaFunction{∼op} \AgdaBound{σ'} \AgdaSymbol{→} \AgdaBound{σ} \AgdaFunction{∘} \AgdaBound{ρ} \AgdaFunction{∼op} \AgdaBound{σ'} \AgdaFunction{∘} \AgdaBound{ρ}\<%
\\
\>[0]\AgdaIndent{2}{}\<[2]%
\>[2]\AgdaFunction{comp-congl} \AgdaSymbol{\{}\AgdaBound{U}\AgdaSymbol{\}} \AgdaSymbol{\{}\AgdaBound{V}\AgdaSymbol{\}} \AgdaSymbol{\{}\AgdaBound{W}\AgdaSymbol{\}} \AgdaSymbol{\{}\AgdaBound{σ}\AgdaSymbol{\}} \AgdaSymbol{\{}\AgdaBound{σ'}\AgdaSymbol{\}} \AgdaSymbol{\{}\AgdaBound{ρ}\AgdaSymbol{\}} \AgdaBound{σ∼σ'} \AgdaBound{x} \AgdaSymbol{=} \AgdaKeyword{let} \AgdaKeyword{open} \AgdaModule{≡-Reasoning} \AgdaKeyword{in} \<[71]%
\>[71]\<%
\\
\>[2]\AgdaIndent{4}{}\<[4]%
\>[4]\AgdaFunction{begin}\<%
\\
\>[4]\AgdaIndent{6}{}\<[6]%
\>[6]\AgdaFunction{apV} \AgdaSymbol{(}\AgdaBound{σ} \AgdaFunction{∘} \AgdaBound{ρ}\AgdaSymbol{)} \AgdaBound{x}\<%
\\
\>[0]\AgdaIndent{4}{}\<[4]%
\>[4]\AgdaFunction{≡⟨} \AgdaFunction{apV-comp} \AgdaFunction{⟩}\<%
\\
\>[4]\AgdaIndent{6}{}\<[6]%
\>[6]\AgdaFunction{ap} \AgdaBound{σ} \AgdaSymbol{(}\AgdaFunction{apV} \AgdaBound{ρ} \AgdaBound{x}\AgdaSymbol{)}\<%
\\
\>[0]\AgdaIndent{4}{}\<[4]%
\>[4]\AgdaFunction{≡⟨} \AgdaFunction{ap-congl} \AgdaBound{σ∼σ'} \AgdaSymbol{(}\AgdaFunction{apV} \AgdaBound{ρ} \AgdaBound{x}\AgdaSymbol{)} \AgdaFunction{⟩}\<%
\\
\>[4]\AgdaIndent{6}{}\<[6]%
\>[6]\AgdaFunction{ap} \AgdaBound{σ'} \AgdaSymbol{(}\AgdaFunction{apV} \AgdaBound{ρ} \AgdaBound{x}\AgdaSymbol{)}\<%
\\
\>[0]\AgdaIndent{4}{}\<[4]%
\>[4]\AgdaFunction{≡⟨⟨} \AgdaFunction{apV-comp} \AgdaFunction{⟩⟩}\<%
\\
\>[4]\AgdaIndent{6}{}\<[6]%
\>[6]\AgdaFunction{apV} \AgdaSymbol{(}\AgdaBound{σ'} \AgdaFunction{∘} \AgdaBound{ρ}\AgdaSymbol{)} \AgdaBound{x}\<%
\\
\>[0]\AgdaIndent{4}{}\<[4]%
\>[4]\AgdaFunction{∎}\<%
\\
\>[0]\AgdaIndent{2}{}\<[2]%
\>[2]\AgdaKeyword{postulate} \AgdaPostulate{comp-congr} \AgdaSymbol{:} \AgdaSymbol{∀} \AgdaSymbol{\{}\AgdaBound{U}\AgdaSymbol{\}} \AgdaSymbol{\{}\AgdaBound{V}\AgdaSymbol{\}} \AgdaSymbol{\{}\AgdaBound{W}\AgdaSymbol{\}} \AgdaSymbol{\{}\AgdaBound{σ} \AgdaSymbol{:} \AgdaFunction{Op} \AgdaBound{V} \AgdaBound{W}\AgdaSymbol{\}} \AgdaSymbol{\{}\AgdaBound{ρ} \AgdaBound{ρ'} \AgdaSymbol{:} \AgdaFunction{Op} \AgdaBound{U} \AgdaBound{V}\AgdaSymbol{\}} \AgdaSymbol{→}\<%
\\
\>[2]\AgdaIndent{23}{}\<[23]%
\>[23]\AgdaBound{ρ} \AgdaFunction{∼op} \AgdaBound{ρ'} \AgdaSymbol{→} \AgdaBound{σ} \AgdaFunction{∘} \AgdaBound{ρ} \AgdaFunction{∼op} \AgdaBound{σ} \AgdaFunction{∘} \AgdaBound{ρ'}\<%
\end{code}

The following results about operations are easy to prove.
\begin{lemma}$ $
  \begin{enumerate}
  \item $(\sigma , K) \circ \uparrow \sim \uparrow \circ \sigma$
  \item $(\id{V} , K) \sim \id{V,K}$
  \item $\id{V}[E] \equiv E$
  \item $(\sigma \circ \rho)[E] \equiv \sigma[\rho[E]]$
  \end{enumerate}
\end{lemma}

\begin{code}%
\>[0]\AgdaIndent{2}{}\<[2]%
\>[2]\AgdaFunction{liftOp-up'} \AgdaSymbol{:} \AgdaSymbol{∀} \AgdaSymbol{\{}\AgdaBound{U}\AgdaSymbol{\}} \AgdaSymbol{\{}\AgdaBound{V}\AgdaSymbol{\}} \AgdaSymbol{\{}\AgdaBound{C}\AgdaSymbol{\}} \AgdaSymbol{\{}\AgdaBound{K}\AgdaSymbol{\}} \AgdaSymbol{\{}\AgdaBound{L}\AgdaSymbol{\}}\<%
\\
\>[2]\AgdaIndent{4}{}\<[4]%
\>[4]\AgdaSymbol{\{}\AgdaBound{σ} \AgdaSymbol{:} \AgdaFunction{Op} \AgdaBound{U} \AgdaBound{V}\AgdaSymbol{\}} \AgdaSymbol{(}\AgdaBound{E} \AgdaSymbol{:} \AgdaDatatype{Subexpression} \AgdaBound{U} \AgdaBound{C} \AgdaBound{K}\AgdaSymbol{)} \AgdaSymbol{→}\<%
\\
\>[2]\AgdaIndent{4}{}\<[4]%
\>[4]\AgdaFunction{ap} \AgdaSymbol{(}\AgdaFunction{liftOp} \AgdaBound{L} \AgdaBound{σ}\AgdaSymbol{)} \AgdaSymbol{(}\AgdaFunction{ap} \AgdaFunction{up} \AgdaBound{E}\AgdaSymbol{)} \AgdaDatatype{≡} \AgdaFunction{ap} \AgdaFunction{up} \AgdaSymbol{(}\AgdaFunction{ap} \AgdaBound{σ} \AgdaBound{E}\AgdaSymbol{)}\<%
\end{code}

\AgdaHide{
\begin{code}%
\>[0]\AgdaIndent{2}{}\<[2]%
\>[2]\AgdaFunction{liftOp-up'} \AgdaBound{E} \AgdaSymbol{=} \AgdaFunction{liftOp-up-mixed} \AgdaField{comp} \AgdaField{comp} \AgdaInductiveConstructor{refl} \AgdaSymbol{\{}\AgdaArgument{E} \AgdaSymbol{=} \AgdaBound{E}\AgdaSymbol{\}}\<%
\end{code}
}

\newcommand{\Op}{\ensuremath{\mathbf{Op}}}

The alphabets and operations up to equivalence form
a category, which we denote $\Op$.
The action of application associates, with every operator family, a functor $\Op \rightarrow \Set$,
which maps an alphabet $U$ to the set of expressions over $U$, and every operation $\sigma$ to the function $\sigma[-]$.
This functor is faithful and injective on objects, and so $\Op$ can be seen as a subcategory of $\Set$.

\begin{code}%
\>[0]\AgdaIndent{2}{}\<[2]%
\>[2]\AgdaFunction{assoc} \AgdaSymbol{:} \AgdaSymbol{∀} \AgdaSymbol{\{}\AgdaBound{U}\AgdaSymbol{\}} \AgdaSymbol{\{}\AgdaBound{V}\AgdaSymbol{\}} \AgdaSymbol{\{}\AgdaBound{W}\AgdaSymbol{\}} \AgdaSymbol{\{}\AgdaBound{X}\AgdaSymbol{\}} \<[28]%
\>[28]\<%
\\
\>[2]\AgdaIndent{4}{}\<[4]%
\>[4]\AgdaSymbol{\{}\AgdaBound{τ} \AgdaSymbol{:} \AgdaFunction{Op} \AgdaBound{W} \AgdaBound{X}\AgdaSymbol{\}} \AgdaSymbol{\{}\AgdaBound{σ} \AgdaSymbol{:} \AgdaFunction{Op} \AgdaBound{V} \AgdaBound{W}\AgdaSymbol{\}} \AgdaSymbol{\{}\AgdaBound{ρ} \AgdaSymbol{:} \AgdaFunction{Op} \AgdaBound{U} \AgdaBound{V}\AgdaSymbol{\}} \AgdaSymbol{→} \<[45]%
\>[45]\<%
\\
\>[2]\AgdaIndent{4}{}\<[4]%
\>[4]\AgdaBound{τ} \AgdaFunction{∘} \AgdaSymbol{(}\AgdaBound{σ} \AgdaFunction{∘} \AgdaBound{ρ}\AgdaSymbol{)} \AgdaFunction{∼op} \AgdaSymbol{(}\AgdaBound{τ} \AgdaFunction{∘} \AgdaBound{σ}\AgdaSymbol{)} \AgdaFunction{∘} \AgdaBound{ρ}\<%
\end{code}

\AgdaHide{
\begin{code}%
\>[0]\AgdaIndent{2}{}\<[2]%
\>[2]\AgdaFunction{assoc} \AgdaSymbol{\{}\AgdaBound{U}\AgdaSymbol{\}} \AgdaSymbol{\{}\AgdaBound{V}\AgdaSymbol{\}} \AgdaSymbol{\{}\AgdaBound{W}\AgdaSymbol{\}} \AgdaSymbol{\{}\AgdaBound{X}\AgdaSymbol{\}} \AgdaSymbol{\{}\AgdaBound{τ}\AgdaSymbol{\}} \AgdaSymbol{\{}\AgdaBound{σ}\AgdaSymbol{\}} \AgdaSymbol{\{}\AgdaBound{ρ}\AgdaSymbol{\}} \AgdaSymbol{\{}\AgdaBound{K}\AgdaSymbol{\}} \AgdaBound{x} \AgdaSymbol{=} \AgdaKeyword{let} \AgdaKeyword{open} \AgdaModule{≡-Reasoning} \AgdaSymbol{\{}\AgdaArgument{A} \AgdaSymbol{=} \AgdaFunction{Expression} \AgdaBound{X} \AgdaSymbol{(}\AgdaInductiveConstructor{varKind} \AgdaBound{K}\AgdaSymbol{)\}} \AgdaKeyword{in} \<[99]%
\>[99]\<%
\\
\>[2]\AgdaIndent{4}{}\<[4]%
\>[4]\AgdaFunction{begin} \<[10]%
\>[10]\<%
\\
\>[4]\AgdaIndent{6}{}\<[6]%
\>[6]\AgdaFunction{apV} \AgdaSymbol{(}\AgdaBound{τ} \AgdaFunction{∘} \AgdaSymbol{(}\AgdaBound{σ} \AgdaFunction{∘} \AgdaBound{ρ}\AgdaSymbol{))} \AgdaBound{x}\<%
\\
\>[0]\AgdaIndent{4}{}\<[4]%
\>[4]\AgdaFunction{≡⟨} \AgdaFunction{apV-comp} \AgdaFunction{⟩}\<%
\\
\>[4]\AgdaIndent{6}{}\<[6]%
\>[6]\AgdaFunction{ap} \AgdaBound{τ} \AgdaSymbol{(}\AgdaFunction{apV} \AgdaSymbol{(}\AgdaBound{σ} \AgdaFunction{∘} \AgdaBound{ρ}\AgdaSymbol{)} \AgdaBound{x}\AgdaSymbol{)}\<%
\\
\>[0]\AgdaIndent{4}{}\<[4]%
\>[4]\AgdaFunction{≡⟨} \AgdaFunction{cong} \AgdaSymbol{(}\AgdaFunction{ap} \AgdaBound{τ}\AgdaSymbol{)} \AgdaFunction{apV-comp} \AgdaFunction{⟩}\<%
\\
\>[4]\AgdaIndent{6}{}\<[6]%
\>[6]\AgdaFunction{ap} \AgdaBound{τ} \AgdaSymbol{(}\AgdaFunction{ap} \AgdaBound{σ} \AgdaSymbol{(}\AgdaFunction{apV} \AgdaBound{ρ} \AgdaBound{x}\AgdaSymbol{))}\<%
\\
\>[0]\AgdaIndent{4}{}\<[4]%
\>[4]\AgdaFunction{≡⟨⟨} \AgdaFunction{ap-comp} \AgdaSymbol{(}\AgdaFunction{apV} \AgdaBound{ρ} \AgdaBound{x}\AgdaSymbol{)} \AgdaFunction{⟩⟩}\<%
\\
\>[4]\AgdaIndent{6}{}\<[6]%
\>[6]\AgdaFunction{ap} \AgdaSymbol{(}\AgdaBound{τ} \AgdaFunction{∘} \AgdaBound{σ}\AgdaSymbol{)} \AgdaSymbol{(}\AgdaFunction{apV} \AgdaBound{ρ} \AgdaBound{x}\AgdaSymbol{)}\<%
\\
\>[0]\AgdaIndent{4}{}\<[4]%
\>[4]\AgdaFunction{≡⟨⟨} \AgdaFunction{apV-comp} \AgdaFunction{⟩⟩}\<%
\\
\>[4]\AgdaIndent{6}{}\<[6]%
\>[6]\AgdaFunction{apV} \AgdaSymbol{((}\AgdaBound{τ} \AgdaFunction{∘} \AgdaBound{σ}\AgdaSymbol{)} \AgdaFunction{∘} \AgdaBound{ρ}\AgdaSymbol{)} \AgdaBound{x}\<%
\\
\>[0]\AgdaIndent{4}{}\<[4]%
\>[4]\AgdaFunction{∎}\<%
\end{code}
}

\begin{code}%
\>[0]\AgdaIndent{2}{}\<[2]%
\>[2]\AgdaFunction{unitl} \AgdaSymbol{:} \AgdaSymbol{∀} \AgdaSymbol{\{}\AgdaBound{U}\AgdaSymbol{\}} \AgdaSymbol{\{}\AgdaBound{V}\AgdaSymbol{\}} \AgdaSymbol{\{}\AgdaBound{σ} \AgdaSymbol{:} \AgdaFunction{Op} \AgdaBound{U} \AgdaBound{V}\AgdaSymbol{\}} \AgdaSymbol{→} \AgdaFunction{idOp} \AgdaBound{V} \AgdaFunction{∘} \AgdaBound{σ} \AgdaFunction{∼op} \AgdaBound{σ}\<%
\end{code}

\AgdaHide{
\begin{code}%
\>[0]\AgdaIndent{2}{}\<[2]%
\>[2]\AgdaFunction{unitl} \AgdaSymbol{\{}\AgdaBound{U}\AgdaSymbol{\}} \AgdaSymbol{\{}\AgdaBound{V}\AgdaSymbol{\}} \AgdaSymbol{\{}\AgdaBound{σ}\AgdaSymbol{\}} \AgdaSymbol{\{}\AgdaBound{K}\AgdaSymbol{\}} \AgdaBound{x} \AgdaSymbol{=} \AgdaKeyword{let} \AgdaKeyword{open} \AgdaModule{≡-Reasoning} \AgdaSymbol{\{}\AgdaArgument{A} \AgdaSymbol{=} \AgdaFunction{Expression} \AgdaBound{V} \AgdaSymbol{(}\AgdaInductiveConstructor{varKind} \AgdaBound{K}\AgdaSymbol{)\}} \AgdaKeyword{in} \<[83]%
\>[83]\<%
\\
\>[2]\AgdaIndent{4}{}\<[4]%
\>[4]\AgdaFunction{begin} \<[10]%
\>[10]\<%
\\
\>[4]\AgdaIndent{6}{}\<[6]%
\>[6]\AgdaFunction{apV} \AgdaSymbol{(}\AgdaFunction{idOp} \AgdaBound{V} \AgdaFunction{∘} \AgdaBound{σ}\AgdaSymbol{)} \AgdaBound{x}\<%
\\
\>[0]\AgdaIndent{4}{}\<[4]%
\>[4]\AgdaFunction{≡⟨} \AgdaFunction{apV-comp} \AgdaFunction{⟩}\<%
\\
\>[4]\AgdaIndent{6}{}\<[6]%
\>[6]\AgdaFunction{ap} \AgdaSymbol{(}\AgdaFunction{idOp} \AgdaBound{V}\AgdaSymbol{)} \AgdaSymbol{(}\AgdaFunction{apV} \AgdaBound{σ} \AgdaBound{x}\AgdaSymbol{)}\<%
\\
\>[0]\AgdaIndent{4}{}\<[4]%
\>[4]\AgdaFunction{≡⟨} \AgdaFunction{ap-idOp} \AgdaFunction{⟩}\<%
\\
\>[4]\AgdaIndent{6}{}\<[6]%
\>[6]\AgdaFunction{apV} \AgdaBound{σ} \AgdaBound{x}\<%
\\
\>[0]\AgdaIndent{4}{}\<[4]%
\>[4]\AgdaFunction{∎}\<%
\end{code}
}

\begin{code}%
\>[0]\AgdaIndent{2}{}\<[2]%
\>[2]\AgdaFunction{unitr} \AgdaSymbol{:} \AgdaSymbol{∀} \AgdaSymbol{\{}\AgdaBound{U}\AgdaSymbol{\}} \AgdaSymbol{\{}\AgdaBound{V}\AgdaSymbol{\}} \AgdaSymbol{\{}\AgdaBound{σ} \AgdaSymbol{:} \AgdaFunction{Op} \AgdaBound{U} \AgdaBound{V}\AgdaSymbol{\}} \AgdaSymbol{→} \AgdaBound{σ} \AgdaFunction{∘} \AgdaFunction{idOp} \AgdaBound{U} \AgdaFunction{∼op} \AgdaBound{σ}\<%
\end{code}

\AgdaHide{
\begin{code}%
\>[0]\AgdaIndent{2}{}\<[2]%
\>[2]\AgdaFunction{unitr} \AgdaSymbol{\{}\AgdaBound{U}\AgdaSymbol{\}} \AgdaSymbol{\{}\AgdaBound{V}\AgdaSymbol{\}} \AgdaSymbol{\{}\AgdaBound{σ}\AgdaSymbol{\}} \AgdaSymbol{\{}\AgdaBound{K}\AgdaSymbol{\}} \AgdaBound{x} \AgdaSymbol{=} \AgdaKeyword{let} \AgdaKeyword{open} \AgdaModule{≡-Reasoning} \AgdaSymbol{\{}\AgdaArgument{A} \AgdaSymbol{=} \AgdaFunction{Expression} \AgdaBound{V} \AgdaSymbol{(}\AgdaInductiveConstructor{varKind} \AgdaBound{K}\AgdaSymbol{)\}} \AgdaKeyword{in}\<%
\\
\>[2]\AgdaIndent{4}{}\<[4]%
\>[4]\AgdaFunction{begin} \<[10]%
\>[10]\<%
\\
\>[4]\AgdaIndent{6}{}\<[6]%
\>[6]\AgdaFunction{apV} \AgdaSymbol{(}\AgdaBound{σ} \AgdaFunction{∘} \AgdaFunction{idOp} \AgdaBound{U}\AgdaSymbol{)} \AgdaBound{x}\<%
\\
\>[0]\AgdaIndent{4}{}\<[4]%
\>[4]\AgdaFunction{≡⟨} \AgdaFunction{apV-comp} \AgdaFunction{⟩}\<%
\\
\>[4]\AgdaIndent{6}{}\<[6]%
\>[6]\AgdaFunction{ap} \AgdaBound{σ} \AgdaSymbol{(}\AgdaFunction{apV} \AgdaSymbol{(}\AgdaFunction{idOp} \AgdaBound{U}\AgdaSymbol{)} \AgdaBound{x}\AgdaSymbol{)}\<%
\\
\>[0]\AgdaIndent{4}{}\<[4]%
\>[4]\AgdaFunction{≡⟨} \AgdaFunction{cong} \AgdaSymbol{(}\AgdaFunction{ap} \AgdaBound{σ}\AgdaSymbol{)} \AgdaSymbol{(}\AgdaFunction{apV-idOp} \AgdaBound{x}\AgdaSymbol{)} \AgdaFunction{⟩}\<%
\\
\>[4]\AgdaIndent{6}{}\<[6]%
\>[6]\AgdaFunction{apV} \AgdaBound{σ} \AgdaBound{x}\<%
\\
\>[0]\AgdaIndent{4}{}\<[4]%
\>[4]\AgdaFunction{∎}\<%
\end{code}
}

\AgdaHide{
\begin{code}%
\>\AgdaKeyword{record} \AgdaRecord{OpFamily} \AgdaSymbol{:} \AgdaPrimitiveType{Set₂} \AgdaKeyword{where}\<%
\\
\>[0]\AgdaIndent{2}{}\<[2]%
\>[2]\AgdaKeyword{field}\<%
\\
\>[2]\AgdaIndent{4}{}\<[4]%
\>[4]\AgdaField{liftFamily} \AgdaSymbol{:} \AgdaRecord{LiftFamily}\<%
\\
\>[2]\AgdaIndent{4}{}\<[4]%
\>[4]\AgdaField{isOpFamily} \<[16]%
\>[16]\AgdaSymbol{:} \AgdaRecord{IsOpFamily} \AgdaField{liftFamily}\<%
\\
\>[0]\AgdaIndent{2}{}\<[2]%
\>[2]\AgdaKeyword{open} \AgdaModule{IsOpFamily} \AgdaField{isOpFamily} \AgdaKeyword{public}\<%
\end{code}
}


\AgdaHide{
\begin{code}%
\>\AgdaKeyword{open} \AgdaKeyword{import} \AgdaModule{Grammar.Base}\<%
\\
%
\\
\>\AgdaKeyword{module} \AgdaModule{Grammar.Replacement} \AgdaSymbol{(}\AgdaBound{G} \AgdaSymbol{:} \AgdaRecord{Grammar}\AgdaSymbol{)} \AgdaKeyword{where}\<%
\\
%
\\
\>\AgdaKeyword{open} \AgdaKeyword{import} \AgdaModule{Function}\<%
\\
\>\AgdaKeyword{open} \AgdaKeyword{import} \AgdaModule{Prelims}\<%
\\
\>\AgdaKeyword{open} \AgdaModule{Grammar} \AgdaBound{G}\<%
\\
\>\AgdaKeyword{open} \AgdaKeyword{import} \AgdaModule{Grammar.OpFamily.PreOpFamily} \AgdaBound{G}\<%
\\
\>\AgdaKeyword{open} \AgdaKeyword{import} \AgdaModule{Grammar.OpFamily.LiftFamily} \AgdaBound{G}\<%
\\
\>\AgdaKeyword{open} \AgdaKeyword{import} \AgdaModule{Grammar.OpFamily.OpFamily} \AgdaBound{G}\<%
\end{code}
}

\subsection{Replacement}

The operation family of \emph{replacement} is defined as follows.  A replacement $\rho : U \rightarrow V$ is a function
that maps every variable in $U$ to a variable in $V$ of the same kind.  Application, identity and composition are simply
function application, the identity function and function composition.  The successor is the canonical injection $V \rightarrow (V, K)$,
and $(\sigma , K)$ is the extension of $\sigma$ that maps $x_0$ to $x_0$.

\begin{code}%
\>\AgdaFunction{Rep} \AgdaSymbol{:} \AgdaDatatype{Alphabet} \AgdaSymbol{→} \AgdaDatatype{Alphabet} \AgdaSymbol{→} \AgdaPrimitiveType{Set}\<%
\\
\>\AgdaFunction{Rep} \AgdaBound{U} \AgdaBound{V} \AgdaSymbol{=} \AgdaSymbol{∀} \AgdaBound{K} \AgdaSymbol{→} \AgdaDatatype{Var} \AgdaBound{U} \AgdaBound{K} \AgdaSymbol{→} \AgdaDatatype{Var} \AgdaBound{V} \AgdaBound{K}\<%
\\
%
\\
\>\AgdaFunction{rep↑} \AgdaSymbol{:} \AgdaSymbol{∀} \AgdaSymbol{\{}\AgdaBound{U}\AgdaSymbol{\}} \AgdaSymbol{\{}\AgdaBound{V}\AgdaSymbol{\}} \AgdaBound{K} \AgdaSymbol{→} \AgdaFunction{Rep} \AgdaBound{U} \AgdaBound{V} \AgdaSymbol{→} \AgdaFunction{Rep} \AgdaSymbol{(}\AgdaBound{U} \AgdaInductiveConstructor{,} \AgdaBound{K}\AgdaSymbol{)} \AgdaSymbol{(}\AgdaBound{V} \AgdaInductiveConstructor{,} \AgdaBound{K}\AgdaSymbol{)}\<%
\\
\>\AgdaFunction{rep↑} \AgdaSymbol{\_} \AgdaSymbol{\_} \AgdaSymbol{\_} \AgdaInductiveConstructor{x₀} \AgdaSymbol{=} \AgdaInductiveConstructor{x₀}\<%
\\
\>\AgdaFunction{rep↑} \AgdaSymbol{\_} \AgdaBound{ρ} \AgdaBound{K} \AgdaSymbol{(}\AgdaInductiveConstructor{↑} \AgdaBound{x}\AgdaSymbol{)} \AgdaSymbol{=} \AgdaInductiveConstructor{↑} \AgdaSymbol{(}\AgdaBound{ρ} \AgdaBound{K} \AgdaBound{x}\AgdaSymbol{)}\<%
\\
%
\\
\>\AgdaFunction{upRep} \AgdaSymbol{:} \AgdaSymbol{∀} \AgdaSymbol{\{}\AgdaBound{V}\AgdaSymbol{\}} \AgdaSymbol{\{}\AgdaBound{K}\AgdaSymbol{\}} \AgdaSymbol{→} \AgdaFunction{Rep} \AgdaBound{V} \AgdaSymbol{(}\AgdaBound{V} \AgdaInductiveConstructor{,} \AgdaBound{K}\AgdaSymbol{)}\<%
\\
\>\AgdaFunction{upRep} \AgdaSymbol{\_} \AgdaSymbol{=} \AgdaInductiveConstructor{↑}\<%
\\
%
\\
\>\AgdaFunction{idRep} \AgdaSymbol{:} \AgdaSymbol{∀} \AgdaBound{V} \AgdaSymbol{→} \AgdaFunction{Rep} \AgdaBound{V} \AgdaBound{V}\<%
\\
\>\AgdaFunction{idRep} \AgdaSymbol{\_} \AgdaSymbol{\_} \AgdaBound{x} \AgdaSymbol{=} \AgdaBound{x}\<%
\\
%
\\
\>\AgdaFunction{pre-replacement} \AgdaSymbol{:} \AgdaRecord{PreOpFamily}\<%
\\
\>\AgdaFunction{pre-replacement} \AgdaSymbol{=} \AgdaKeyword{record} \AgdaSymbol{\{} \<[27]%
\>[27]\<%
\\
\>[0]\AgdaIndent{2}{}\<[2]%
\>[2]\AgdaField{Op} \AgdaSymbol{=} \AgdaFunction{Rep}\AgdaSymbol{;} \<[12]%
\>[12]\<%
\\
\>[0]\AgdaIndent{2}{}\<[2]%
\>[2]\AgdaField{apV} \AgdaSymbol{=} \AgdaSymbol{λ} \AgdaBound{ρ} \AgdaBound{x} \AgdaSymbol{→} \AgdaInductiveConstructor{var} \AgdaSymbol{(}\AgdaBound{ρ} \AgdaSymbol{\_} \AgdaBound{x}\AgdaSymbol{);} \<[29]%
\>[29]\<%
\\
\>[0]\AgdaIndent{2}{}\<[2]%
\>[2]\AgdaField{up} \AgdaSymbol{=} \AgdaFunction{upRep}\AgdaSymbol{;} \<[14]%
\>[14]\<%
\\
\>[0]\AgdaIndent{2}{}\<[2]%
\>[2]\AgdaField{apV-up} \AgdaSymbol{=} \AgdaInductiveConstructor{refl}\AgdaSymbol{;} \<[17]%
\>[17]\<%
\\
\>[0]\AgdaIndent{2}{}\<[2]%
\>[2]\AgdaField{idOp} \AgdaSymbol{=} \AgdaFunction{idRep}\AgdaSymbol{;} \<[16]%
\>[16]\<%
\\
\>[0]\AgdaIndent{2}{}\<[2]%
\>[2]\AgdaField{apV-idOp} \AgdaSymbol{=} \AgdaSymbol{λ} \AgdaBound{\_} \AgdaSymbol{→} \AgdaInductiveConstructor{refl} \AgdaSymbol{\}}\<%
\\
%
\\
\>\AgdaFunction{\_∼R\_} \AgdaSymbol{:} \AgdaSymbol{∀} \AgdaSymbol{\{}\AgdaBound{U}\AgdaSymbol{\}} \AgdaSymbol{\{}\AgdaBound{V}\AgdaSymbol{\}} \AgdaSymbol{→} \AgdaFunction{Rep} \AgdaBound{U} \AgdaBound{V} \AgdaSymbol{→} \AgdaFunction{Rep} \AgdaBound{U} \AgdaBound{V} \AgdaSymbol{→} \AgdaPrimitiveType{Set}\<%
\\
\>\AgdaFunction{\_∼R\_} \AgdaSymbol{=} \AgdaFunction{PreOpFamily.\_∼op\_} \AgdaFunction{pre-replacement}\<%
\\
%
\\
\>\AgdaFunction{rep↑-cong} \AgdaSymbol{:} \AgdaSymbol{∀} \AgdaSymbol{\{}\AgdaBound{U}\AgdaSymbol{\}} \AgdaSymbol{\{}\AgdaBound{V}\AgdaSymbol{\}} \AgdaSymbol{\{}\AgdaBound{K}\AgdaSymbol{\}} \AgdaSymbol{\{}\AgdaBound{ρ} \AgdaBound{ρ'} \AgdaSymbol{:} \AgdaFunction{Rep} \AgdaBound{U} \AgdaBound{V}\AgdaSymbol{\}} \AgdaSymbol{→} \<[45]%
\>[45]\<%
\\
\>[0]\AgdaIndent{2}{}\<[2]%
\>[2]\AgdaBound{ρ} \AgdaFunction{∼R} \AgdaBound{ρ'} \AgdaSymbol{→} \AgdaFunction{rep↑} \AgdaBound{K} \AgdaBound{ρ} \AgdaFunction{∼R} \AgdaFunction{rep↑} \AgdaBound{K} \AgdaBound{ρ'}\<%
\end{code}

\AgdaHide{
\begin{code}%
\>\AgdaFunction{rep↑-cong} \AgdaBound{ρ-is-ρ'} \AgdaInductiveConstructor{x₀} \AgdaSymbol{=} \AgdaInductiveConstructor{refl}\<%
\\
\>\AgdaFunction{rep↑-cong} \AgdaBound{ρ-is-ρ'} \AgdaSymbol{(}\AgdaInductiveConstructor{↑} \AgdaBound{x}\AgdaSymbol{)} \AgdaSymbol{=} \AgdaFunction{cong} \AgdaSymbol{(}\AgdaInductiveConstructor{var} \AgdaFunction{∘} \AgdaInductiveConstructor{↑}\AgdaSymbol{)} \AgdaSymbol{(}\AgdaFunction{var-inj} \AgdaSymbol{(}\AgdaBound{ρ-is-ρ'} \AgdaBound{x}\AgdaSymbol{))}\<%
\end{code}
}

\begin{code}%
\>\AgdaFunction{proto-replacement} \AgdaSymbol{:} \AgdaRecord{LiftFamily}\<%
\\
\>\AgdaFunction{proto-replacement} \AgdaSymbol{=} \AgdaKeyword{record} \AgdaSymbol{\{} \<[29]%
\>[29]\<%
\\
\>[0]\AgdaIndent{2}{}\<[2]%
\>[2]\AgdaField{preOpFamily} \AgdaSymbol{=} \AgdaFunction{pre-replacement} \AgdaSymbol{;} \<[34]%
\>[34]\<%
\\
\>[0]\AgdaIndent{2}{}\<[2]%
\>[2]\AgdaField{lifting} \AgdaSymbol{=} \AgdaKeyword{record} \AgdaSymbol{\{} \<[21]%
\>[21]\<%
\\
\>[2]\AgdaIndent{4}{}\<[4]%
\>[4]\AgdaField{liftOp} \AgdaSymbol{=} \AgdaFunction{rep↑} \AgdaSymbol{;} \<[20]%
\>[20]\<%
\\
\>[2]\AgdaIndent{4}{}\<[4]%
\>[4]\AgdaField{liftOp-cong} \AgdaSymbol{=} \AgdaFunction{rep↑-cong} \AgdaSymbol{\}} \AgdaSymbol{;} \<[32]%
\>[32]\<%
\\
\>[0]\AgdaIndent{2}{}\<[2]%
\>[2]\AgdaField{isLiftFamily} \AgdaSymbol{=} \AgdaKeyword{record} \AgdaSymbol{\{} \<[26]%
\>[26]\<%
\\
\>[2]\AgdaIndent{4}{}\<[4]%
\>[4]\AgdaField{liftOp-x₀} \AgdaSymbol{=} \AgdaInductiveConstructor{refl} \AgdaSymbol{;} \<[23]%
\>[23]\<%
\\
\>[2]\AgdaIndent{4}{}\<[4]%
\>[4]\AgdaField{liftOp-↑} \AgdaSymbol{=} \AgdaSymbol{λ} \AgdaBound{\_} \AgdaSymbol{→} \AgdaInductiveConstructor{refl} \AgdaSymbol{\}} \AgdaSymbol{\}}\<%
\\
%
\\
\>\AgdaKeyword{infix} \AgdaNumber{70} \AgdaFixityOp{\_〈\_〉}\<%
\\
\>\AgdaFunction{\_〈\_〉} \AgdaSymbol{:} \AgdaSymbol{∀} \AgdaSymbol{\{}\AgdaBound{U}\AgdaSymbol{\}} \AgdaSymbol{\{}\AgdaBound{V}\AgdaSymbol{\}} \AgdaSymbol{\{}\AgdaBound{C}\AgdaSymbol{\}} \AgdaSymbol{\{}\AgdaBound{K}\AgdaSymbol{\}} \AgdaSymbol{→} \<[27]%
\>[27]\<%
\\
\>[0]\AgdaIndent{2}{}\<[2]%
\>[2]\AgdaDatatype{Subexpression} \AgdaBound{U} \AgdaBound{C} \AgdaBound{K} \AgdaSymbol{→} \AgdaFunction{Rep} \AgdaBound{U} \AgdaBound{V} \AgdaSymbol{→} \AgdaDatatype{Subexpression} \AgdaBound{V} \AgdaBound{C} \AgdaBound{K}\<%
\\
\>\AgdaBound{E} \AgdaFunction{〈} \AgdaBound{ρ} \AgdaFunction{〉} \AgdaSymbol{=} \AgdaFunction{LiftFamily.ap} \AgdaFunction{proto-replacement} \AgdaBound{ρ} \AgdaBound{E}\<%
\\
%
\\
\>\AgdaKeyword{infixl} \AgdaNumber{75} \AgdaFixityOp{\_•R\_}\<%
\\
\>\AgdaFunction{\_•R\_} \AgdaSymbol{:} \AgdaSymbol{∀} \AgdaSymbol{\{}\AgdaBound{U}\AgdaSymbol{\}} \AgdaSymbol{\{}\AgdaBound{V}\AgdaSymbol{\}} \AgdaSymbol{\{}\AgdaBound{W}\AgdaSymbol{\}} \AgdaSymbol{→} \AgdaFunction{Rep} \AgdaBound{V} \AgdaBound{W} \AgdaSymbol{→} \AgdaFunction{Rep} \AgdaBound{U} \AgdaBound{V} \AgdaSymbol{→} \AgdaFunction{Rep} \AgdaBound{U} \AgdaBound{W}\<%
\\
\>\AgdaSymbol{(}\AgdaBound{ρ'} \AgdaFunction{•R} \AgdaBound{ρ}\AgdaSymbol{)} \AgdaBound{K} \AgdaBound{x} \AgdaSymbol{=} \AgdaBound{ρ'} \AgdaBound{K} \AgdaSymbol{(}\AgdaBound{ρ} \AgdaBound{K} \AgdaBound{x}\AgdaSymbol{)}\<%
\\
%
\\
\>\AgdaFunction{rep↑-comp} \AgdaSymbol{:} \AgdaSymbol{∀} \AgdaSymbol{\{}\AgdaBound{U}\AgdaSymbol{\}} \AgdaSymbol{\{}\AgdaBound{V}\AgdaSymbol{\}} \AgdaSymbol{\{}\AgdaBound{W}\AgdaSymbol{\}} \AgdaSymbol{\{}\AgdaBound{K}\AgdaSymbol{\}} \AgdaSymbol{\{}\AgdaBound{ρ'} \AgdaSymbol{:} \AgdaFunction{Rep} \AgdaBound{V} \AgdaBound{W}\AgdaSymbol{\}} \AgdaSymbol{\{}\AgdaBound{ρ} \AgdaSymbol{:} \AgdaFunction{Rep} \AgdaBound{U} \AgdaBound{V}\AgdaSymbol{\}} \AgdaSymbol{→} \<[61]%
\>[61]\<%
\\
\>[0]\AgdaIndent{2}{}\<[2]%
\>[2]\AgdaFunction{rep↑} \AgdaBound{K} \AgdaSymbol{(}\AgdaBound{ρ'} \AgdaFunction{•R} \AgdaBound{ρ}\AgdaSymbol{)} \AgdaFunction{∼R} \AgdaFunction{rep↑} \AgdaBound{K} \AgdaBound{ρ'} \AgdaFunction{•R} \AgdaFunction{rep↑} \AgdaBound{K} \AgdaBound{ρ}\<%
\end{code}

\AgdaHide{
\begin{code}%
\>\AgdaFunction{rep↑-comp} \AgdaInductiveConstructor{x₀} \AgdaSymbol{=} \AgdaInductiveConstructor{refl}\<%
\\
\>\AgdaFunction{rep↑-comp} \AgdaSymbol{(}\AgdaInductiveConstructor{↑} \AgdaSymbol{\_)} \AgdaSymbol{=} \AgdaInductiveConstructor{refl}\<%
\\
%
\\
\>\AgdaKeyword{postulate} \AgdaPostulate{rep↑-comp₄} \AgdaSymbol{:} \AgdaSymbol{∀} \AgdaSymbol{\{}\AgdaBound{U}\AgdaSymbol{\}} \AgdaSymbol{\{}\AgdaBound{V1}\AgdaSymbol{\}} \AgdaSymbol{\{}\AgdaBound{V2}\AgdaSymbol{\}} \AgdaSymbol{\{}\AgdaBound{V3}\AgdaSymbol{\}} \AgdaSymbol{\{}\AgdaBound{V4}\AgdaSymbol{\}} \AgdaSymbol{\{}\AgdaBound{K}\AgdaSymbol{\}} \AgdaSymbol{\{}\AgdaBound{ρ1} \AgdaSymbol{:} \AgdaFunction{Rep} \AgdaBound{U} \AgdaBound{V1}\AgdaSymbol{\}} \AgdaSymbol{\{}\AgdaBound{ρ2} \AgdaSymbol{:} \AgdaFunction{Rep} \AgdaBound{V1} \AgdaBound{V2}\AgdaSymbol{\}} \AgdaSymbol{\{}\AgdaBound{ρ3} \AgdaSymbol{:} \AgdaFunction{Rep} \AgdaBound{V2} \AgdaBound{V3}\AgdaSymbol{\}} \AgdaSymbol{\{}\AgdaBound{ρ4} \AgdaSymbol{:} \AgdaFunction{Rep} \AgdaBound{V3} \AgdaBound{V4}\AgdaSymbol{\}} \AgdaSymbol{→}\<%
\\
\>[2]\AgdaIndent{23}{}\<[23]%
\>[23]\AgdaFunction{rep↑} \AgdaBound{K} \AgdaSymbol{(}\AgdaBound{ρ4} \AgdaFunction{•R} \AgdaBound{ρ3} \AgdaFunction{•R} \AgdaBound{ρ2} \AgdaFunction{•R} \AgdaBound{ρ1}\AgdaSymbol{)} \AgdaFunction{∼R} \AgdaFunction{rep↑} \AgdaBound{K} \AgdaBound{ρ4} \AgdaFunction{•R} \AgdaFunction{rep↑} \AgdaBound{K} \AgdaBound{ρ3} \AgdaFunction{•R} \AgdaFunction{rep↑} \AgdaBound{K} \AgdaBound{ρ2} \AgdaFunction{•R} \AgdaFunction{rep↑} \AgdaBound{K} \AgdaBound{ρ1}\<%
\end{code}
}

\begin{code}%
\>\AgdaFunction{replacement} \AgdaSymbol{:} \AgdaRecord{OpFamily}\<%
\\
\>\AgdaFunction{replacement} \AgdaSymbol{=} \AgdaKeyword{record} \AgdaSymbol{\{} \<[23]%
\>[23]\<%
\\
\>[0]\AgdaIndent{2}{}\<[2]%
\>[2]\AgdaField{liftFamily} \AgdaSymbol{=} \AgdaFunction{proto-replacement} \AgdaSymbol{;} \<[35]%
\>[35]\<%
\\
\>[0]\AgdaIndent{2}{}\<[2]%
\>[2]\AgdaField{isOpFamily} \AgdaSymbol{=} \AgdaKeyword{record} \AgdaSymbol{\{} \<[24]%
\>[24]\<%
\\
\>[2]\AgdaIndent{4}{}\<[4]%
\>[4]\AgdaField{\_∘\_} \AgdaSymbol{=} \AgdaFunction{\_•R\_} \AgdaSymbol{;} \<[17]%
\>[17]\<%
\\
\>[2]\AgdaIndent{4}{}\<[4]%
\>[4]\AgdaField{apV-comp} \AgdaSymbol{=} \AgdaInductiveConstructor{refl} \AgdaSymbol{;} \<[22]%
\>[22]\<%
\\
\>[2]\AgdaIndent{4}{}\<[4]%
\>[4]\AgdaField{liftOp-comp} \AgdaSymbol{=} \AgdaFunction{rep↑-comp} \AgdaSymbol{\}} \AgdaSymbol{\}}\<%
\end{code}

\AgdaHide{
\begin{code}%
\>\AgdaKeyword{open} \AgdaModule{OpFamily} \AgdaFunction{replacement} \AgdaKeyword{public} \AgdaKeyword{using} \AgdaSymbol{()} \<[42]%
\>[42]\<%
\\
\>[0]\AgdaIndent{2}{}\<[2]%
\>[2]\AgdaKeyword{renaming} \AgdaSymbol{(}ap-congl \AgdaSymbol{to} rep-congr\AgdaSymbol{;}\<\\
\>           ap-congr \AgdaSymbol{to} rep-congl\AgdaSymbol{;}\<\\
\>           ap-idOp \AgdaSymbol{to} rep-idOp\AgdaSymbol{;}\<\\
\>           ap-circ \AgdaSymbol{to} rep-comp\AgdaSymbol{;}\<\\
\>           liftOp-idOp \AgdaSymbol{to} rep↑-idOp\AgdaSymbol{;}\<\\
\>           liftOp-up' \AgdaSymbol{to} rep↑-upRep\AgdaSymbol{)}\<%
\\
%
\\
\>\AgdaKeyword{postulate} \AgdaPostulate{rep-comp₄} \AgdaSymbol{:} \AgdaSymbol{∀} \AgdaSymbol{\{}\AgdaBound{U}\AgdaSymbol{\}} \AgdaSymbol{\{}\AgdaBound{V1}\AgdaSymbol{\}} \AgdaSymbol{\{}\AgdaBound{V2}\AgdaSymbol{\}} \AgdaSymbol{\{}\AgdaBound{V3}\AgdaSymbol{\}} \AgdaSymbol{\{}\AgdaBound{V4}\AgdaSymbol{\}} \<[48]%
\>[48]\<%
\\
\>[2]\AgdaIndent{22}{}\<[22]%
\>[22]\AgdaSymbol{\{}\AgdaBound{ρ1} \AgdaSymbol{:} \AgdaFunction{Rep} \AgdaBound{U} \AgdaBound{V1}\AgdaSymbol{\}} \AgdaSymbol{\{}\AgdaBound{ρ2} \AgdaSymbol{:} \AgdaFunction{Rep} \AgdaBound{V1} \AgdaBound{V2}\AgdaSymbol{\}} \AgdaSymbol{\{}\AgdaBound{ρ3} \AgdaSymbol{:} \AgdaFunction{Rep} \AgdaBound{V2} \AgdaBound{V3}\AgdaSymbol{\}} \AgdaSymbol{\{}\AgdaBound{ρ4} \AgdaSymbol{:} \AgdaFunction{Rep} \AgdaBound{V3} \AgdaBound{V4}\AgdaSymbol{\}} \<[89]%
\>[89]\<%
\\
\>[2]\AgdaIndent{22}{}\<[22]%
\>[22]\AgdaSymbol{\{}\AgdaBound{C}\AgdaSymbol{\}} \AgdaSymbol{\{}\AgdaBound{K}\AgdaSymbol{\}} \AgdaSymbol{(}\AgdaBound{E} \AgdaSymbol{:} \AgdaDatatype{Subexpression} \AgdaBound{U} \AgdaBound{C} \AgdaBound{K}\AgdaSymbol{)} \AgdaSymbol{→}\<%
\\
\>[2]\AgdaIndent{22}{}\<[22]%
\>[22]\AgdaBound{E} \AgdaFunction{〈} \AgdaBound{ρ4} \AgdaFunction{•R} \AgdaBound{ρ3} \AgdaFunction{•R} \AgdaBound{ρ2} \AgdaFunction{•R} \AgdaBound{ρ1} \AgdaFunction{〉} \AgdaDatatype{≡} \AgdaBound{E} \AgdaFunction{〈} \AgdaBound{ρ1} \AgdaFunction{〉} \AgdaFunction{〈} \AgdaBound{ρ2} \AgdaFunction{〉} \AgdaFunction{〈} \AgdaBound{ρ3} \AgdaFunction{〉} \AgdaFunction{〈} \AgdaBound{ρ4} \AgdaFunction{〉}\<%
\end{code}
}

We write $E \uparrow$ for $E \langle \uparrow \rangle$.

\begin{code}%
\>\AgdaKeyword{infixl} \AgdaNumber{70} \AgdaFixityOp{\_⇑}\<%
\\
\>\AgdaFunction{\_⇑} \AgdaSymbol{:} \AgdaSymbol{∀} \AgdaSymbol{\{}\AgdaBound{V}\AgdaSymbol{\}} \AgdaSymbol{\{}\AgdaBound{K}\AgdaSymbol{\}} \AgdaSymbol{\{}\AgdaBound{C}\AgdaSymbol{\}} \AgdaSymbol{\{}\AgdaBound{L}\AgdaSymbol{\}} \AgdaSymbol{→} \AgdaDatatype{Subexpression} \AgdaBound{V} \AgdaBound{C} \AgdaBound{L} \AgdaSymbol{→} \AgdaDatatype{Subexpression} \AgdaSymbol{(}\AgdaBound{V} \AgdaInductiveConstructor{,} \AgdaBound{K}\AgdaSymbol{)} \AgdaBound{C} \AgdaBound{L}\<%
\\
\>\AgdaBound{E} \AgdaFunction{⇑} \AgdaSymbol{=} \AgdaBound{E} \AgdaFunction{〈} \AgdaFunction{upRep} \AgdaFunction{〉}\<%
\end{code}

We define the unique replacement $\emptyset \rightarrow V$ for any V, and prove it unique:

\begin{code}%
\>\AgdaFunction{magic} \AgdaSymbol{:} \AgdaSymbol{∀} \AgdaSymbol{\{}\AgdaBound{V}\AgdaSymbol{\}} \AgdaSymbol{→} \AgdaFunction{Rep} \AgdaInductiveConstructor{∅} \AgdaBound{V}\<%
\\
\>\AgdaFunction{magic} \AgdaSymbol{\_} \AgdaSymbol{()}\<%
\\
%
\\
\>\AgdaFunction{magic-unique} \AgdaSymbol{:} \AgdaSymbol{∀} \AgdaSymbol{\{}\AgdaBound{V}\AgdaSymbol{\}} \AgdaSymbol{\{}\AgdaBound{ρ} \AgdaSymbol{:} \AgdaFunction{Rep} \AgdaInductiveConstructor{∅} \AgdaBound{V}\AgdaSymbol{\}} \AgdaSymbol{→} \AgdaBound{ρ} \AgdaFunction{∼R} \AgdaFunction{magic}\<%
\end{code}

\AgdaHide{
\begin{code}%
\>\AgdaFunction{magic-unique} \AgdaSymbol{\{}\AgdaBound{V}\AgdaSymbol{\}} \AgdaSymbol{\{}\AgdaBound{ρ}\AgdaSymbol{\}} \AgdaSymbol{()}\<%
\end{code}
}

\begin{code}%
\>\AgdaFunction{magic-unique'} \AgdaSymbol{:} \AgdaSymbol{∀} \AgdaSymbol{\{}\AgdaBound{U}\AgdaSymbol{\}} \AgdaSymbol{\{}\AgdaBound{V}\AgdaSymbol{\}} \AgdaSymbol{\{}\AgdaBound{C}\AgdaSymbol{\}} \AgdaSymbol{\{}\AgdaBound{K}\AgdaSymbol{\}}\<%
\\
\>[0]\AgdaIndent{2}{}\<[2]%
\>[2]\AgdaSymbol{(}\AgdaBound{E} \AgdaSymbol{:} \AgdaDatatype{Subexpression} \AgdaInductiveConstructor{∅} \AgdaBound{C} \AgdaBound{K}\AgdaSymbol{)} \AgdaSymbol{\{}\AgdaBound{ρ} \AgdaSymbol{:} \AgdaFunction{Rep} \AgdaBound{U} \AgdaBound{V}\AgdaSymbol{\}} \AgdaSymbol{→} \<[44]%
\>[44]\<%
\\
\>[0]\AgdaIndent{2}{}\<[2]%
\>[2]\AgdaBound{E} \AgdaFunction{〈} \AgdaFunction{magic} \AgdaFunction{〉} \AgdaFunction{〈} \AgdaBound{ρ} \AgdaFunction{〉} \AgdaDatatype{≡} \AgdaBound{E} \AgdaFunction{〈} \AgdaFunction{magic} \AgdaFunction{〉}\<%
\end{code}

\AgdaHide{
\begin{code}%
\>\AgdaFunction{magic-unique'} \AgdaBound{E} \AgdaSymbol{\{}\AgdaBound{ρ}\AgdaSymbol{\}} \AgdaSymbol{=} \AgdaKeyword{let} \AgdaKeyword{open} \AgdaModule{≡-Reasoning} \AgdaKeyword{in}\<%
\\
\>[0]\AgdaIndent{2}{}\<[2]%
\>[2]\AgdaFunction{begin}\<%
\\
\>[2]\AgdaIndent{4}{}\<[4]%
\>[4]\AgdaBound{E} \AgdaFunction{〈} \AgdaFunction{magic} \AgdaFunction{〉} \AgdaFunction{〈} \AgdaBound{ρ} \AgdaFunction{〉}\<%
\\
\>[0]\AgdaIndent{2}{}\<[2]%
\>[2]\AgdaFunction{≡⟨⟨} \AgdaFunction{rep-comp} \AgdaBound{E} \AgdaFunction{⟩⟩}\<%
\\
\>[2]\AgdaIndent{4}{}\<[4]%
\>[4]\AgdaBound{E} \AgdaFunction{〈} \AgdaBound{ρ} \AgdaFunction{•R} \AgdaFunction{magic} \AgdaFunction{〉}\<%
\\
\>[0]\AgdaIndent{2}{}\<[2]%
\>[2]\AgdaFunction{≡⟨} \AgdaFunction{rep-congr} \AgdaBound{E} \AgdaSymbol{(}\AgdaFunction{magic-unique} \AgdaSymbol{\{}\AgdaArgument{ρ} \AgdaSymbol{=} \AgdaBound{ρ} \AgdaFunction{•R} \AgdaFunction{magic}\AgdaSymbol{\})} \AgdaFunction{⟩}\<%
\\
\>[2]\AgdaIndent{4}{}\<[4]%
\>[4]\AgdaBound{E} \AgdaFunction{〈} \AgdaFunction{magic} \AgdaFunction{〉}\<%
\\
\>[0]\AgdaIndent{2}{}\<[2]%
\>[2]\AgdaFunction{∎}\<%
\\
%
\\
\>\AgdaFunction{rep↑-upRep₂} \AgdaSymbol{:} \AgdaSymbol{∀} \AgdaSymbol{\{}\AgdaBound{U}\AgdaSymbol{\}} \AgdaSymbol{\{}\AgdaBound{V}\AgdaSymbol{\}} \AgdaSymbol{\{}\AgdaBound{C}\AgdaSymbol{\}} \AgdaSymbol{\{}\AgdaBound{K}\AgdaSymbol{\}} \AgdaSymbol{\{}\AgdaBound{L}\AgdaSymbol{\}} \AgdaSymbol{\{}\AgdaBound{M}\AgdaSymbol{\}} \AgdaSymbol{(}\AgdaBound{E} \AgdaSymbol{:} \AgdaDatatype{Subexpression} \AgdaBound{U} \AgdaBound{C} \AgdaBound{M}\AgdaSymbol{)} \AgdaSymbol{\{}\AgdaBound{σ} \AgdaSymbol{:} \AgdaFunction{Rep} \AgdaBound{U} \AgdaBound{V}\AgdaSymbol{\}} \AgdaSymbol{→} \AgdaBound{E} \AgdaFunction{⇑} \AgdaFunction{⇑} \AgdaFunction{〈} \AgdaFunction{rep↑} \AgdaBound{K} \AgdaSymbol{(}\AgdaFunction{rep↑} \AgdaBound{L} \AgdaBound{σ}\AgdaSymbol{)} \AgdaFunction{〉} \AgdaDatatype{≡} \AgdaBound{E} \AgdaFunction{〈} \AgdaBound{σ} \AgdaFunction{〉} \AgdaFunction{⇑} \AgdaFunction{⇑}\<%
\\
\>\AgdaFunction{rep↑-upRep₂} \AgdaSymbol{\{}\AgdaBound{U}\AgdaSymbol{\}} \AgdaSymbol{\{}\AgdaBound{V}\AgdaSymbol{\}} \AgdaSymbol{\{}\AgdaBound{C}\AgdaSymbol{\}} \AgdaSymbol{\{}\AgdaBound{K}\AgdaSymbol{\}} \AgdaSymbol{\{}\AgdaBound{L}\AgdaSymbol{\}} \AgdaSymbol{\{}\AgdaBound{M}\AgdaSymbol{\}} \AgdaBound{E} \AgdaSymbol{\{}\AgdaBound{σ}\AgdaSymbol{\}} \AgdaSymbol{=} \AgdaKeyword{let} \AgdaKeyword{open} \AgdaModule{≡-Reasoning} \AgdaKeyword{in} \<[68]%
\>[68]\<%
\\
\>[0]\AgdaIndent{2}{}\<[2]%
\>[2]\AgdaFunction{begin}\<%
\\
\>[2]\AgdaIndent{4}{}\<[4]%
\>[4]\AgdaBound{E} \AgdaFunction{⇑} \AgdaFunction{⇑} \AgdaFunction{〈} \AgdaFunction{rep↑} \AgdaBound{K} \AgdaSymbol{(}\AgdaFunction{rep↑} \AgdaBound{L} \AgdaBound{σ}\AgdaSymbol{)} \AgdaFunction{〉}\<%
\\
\>[0]\AgdaIndent{2}{}\<[2]%
\>[2]\AgdaFunction{≡⟨} \AgdaFunction{rep↑-upRep} \AgdaSymbol{(}\AgdaBound{E} \AgdaFunction{⇑}\AgdaSymbol{)} \AgdaFunction{⟩}\<%
\\
\>[2]\AgdaIndent{4}{}\<[4]%
\>[4]\AgdaBound{E} \AgdaFunction{⇑} \AgdaFunction{〈} \AgdaFunction{rep↑} \AgdaBound{L} \AgdaBound{σ} \AgdaFunction{〉} \AgdaFunction{⇑}\<%
\\
\>[0]\AgdaIndent{2}{}\<[2]%
\>[2]\AgdaFunction{≡⟨} \AgdaFunction{rep-congl} \AgdaSymbol{(}\AgdaFunction{rep↑-upRep} \AgdaBound{E}\AgdaSymbol{)} \AgdaFunction{⟩}\<%
\\
\>[2]\AgdaIndent{4}{}\<[4]%
\>[4]\AgdaBound{E} \AgdaFunction{〈} \AgdaBound{σ} \AgdaFunction{〉} \AgdaFunction{⇑} \AgdaFunction{⇑}\<%
\\
\>[0]\AgdaIndent{2}{}\<[2]%
\>[2]\AgdaFunction{∎}\<%
\\
%
\\
\>\AgdaFunction{rep↑-upRep₃} \AgdaSymbol{:} \AgdaSymbol{∀} \AgdaSymbol{\{}\AgdaBound{U}\AgdaSymbol{\}} \AgdaSymbol{\{}\AgdaBound{V}\AgdaSymbol{\}} \AgdaSymbol{\{}\AgdaBound{C}\AgdaSymbol{\}} \AgdaSymbol{\{}\AgdaBound{K}\AgdaSymbol{\}} \AgdaSymbol{\{}\AgdaBound{L}\AgdaSymbol{\}} \AgdaSymbol{\{}\AgdaBound{M}\AgdaSymbol{\}} \AgdaSymbol{\{}\AgdaBound{N}\AgdaSymbol{\}} \AgdaSymbol{(}\AgdaBound{E} \AgdaSymbol{:} \AgdaDatatype{Subexpression} \AgdaBound{U} \AgdaBound{C} \AgdaBound{N}\AgdaSymbol{)} \AgdaSymbol{\{}\AgdaBound{σ} \AgdaSymbol{:} \AgdaFunction{Rep} \AgdaBound{U} \AgdaBound{V}\AgdaSymbol{\}} \AgdaSymbol{→} \<[86]%
\>[86]\<%
\\
\>[0]\AgdaIndent{2}{}\<[2]%
\>[2]\AgdaBound{E} \AgdaFunction{⇑} \AgdaFunction{⇑} \AgdaFunction{⇑} \AgdaFunction{〈} \AgdaFunction{rep↑} \AgdaBound{K} \AgdaSymbol{(}\AgdaFunction{rep↑} \AgdaBound{L} \AgdaSymbol{(}\AgdaFunction{rep↑} \AgdaBound{M} \AgdaBound{σ}\AgdaSymbol{))} \AgdaFunction{〉} \AgdaDatatype{≡} \AgdaBound{E} \AgdaFunction{〈} \AgdaBound{σ} \AgdaFunction{〉} \AgdaFunction{⇑} \AgdaFunction{⇑} \AgdaFunction{⇑}\<%
\\
\>\AgdaFunction{rep↑-upRep₃} \AgdaSymbol{\{}\AgdaBound{U}\AgdaSymbol{\}} \AgdaSymbol{\{}\AgdaBound{V}\AgdaSymbol{\}} \AgdaSymbol{\{}\AgdaBound{C}\AgdaSymbol{\}} \AgdaSymbol{\{}\AgdaBound{K}\AgdaSymbol{\}} \AgdaSymbol{\{}\AgdaBound{L}\AgdaSymbol{\}} \AgdaSymbol{\{}\AgdaBound{M}\AgdaSymbol{\}} \AgdaBound{E} \AgdaSymbol{\{}\AgdaBound{σ}\AgdaSymbol{\}} \AgdaSymbol{=} \AgdaKeyword{let} \AgdaKeyword{open} \AgdaModule{≡-Reasoning} \AgdaKeyword{in} \<[68]%
\>[68]\<%
\\
\>[0]\AgdaIndent{2}{}\<[2]%
\>[2]\AgdaFunction{begin}\<%
\\
\>[2]\AgdaIndent{4}{}\<[4]%
\>[4]\AgdaBound{E} \AgdaFunction{⇑} \AgdaFunction{⇑} \AgdaFunction{⇑} \AgdaFunction{〈} \AgdaFunction{rep↑} \AgdaBound{K} \AgdaSymbol{(}\AgdaFunction{rep↑} \AgdaBound{L} \AgdaSymbol{(}\AgdaFunction{rep↑} \AgdaBound{M} \AgdaBound{σ}\AgdaSymbol{))} \AgdaFunction{〉}\<%
\\
\>[0]\AgdaIndent{2}{}\<[2]%
\>[2]\AgdaFunction{≡⟨} \AgdaFunction{rep↑-upRep₂} \AgdaSymbol{(}\AgdaBound{E} \AgdaFunction{⇑}\AgdaSymbol{)} \AgdaFunction{⟩}\<%
\\
\>[2]\AgdaIndent{4}{}\<[4]%
\>[4]\AgdaBound{E} \AgdaFunction{⇑} \AgdaFunction{〈} \AgdaFunction{rep↑} \AgdaBound{M} \AgdaBound{σ} \AgdaFunction{〉} \AgdaFunction{⇑} \AgdaFunction{⇑}\<%
\\
\>[0]\AgdaIndent{2}{}\<[2]%
\>[2]\AgdaFunction{≡⟨} \AgdaFunction{rep-congl} \AgdaSymbol{(}\AgdaFunction{rep-congl} \AgdaSymbol{(}\AgdaFunction{rep↑-upRep} \AgdaBound{E}\AgdaSymbol{))} \AgdaFunction{⟩}\<%
\\
\>[2]\AgdaIndent{4}{}\<[4]%
\>[4]\AgdaBound{E} \AgdaFunction{〈} \AgdaBound{σ} \AgdaFunction{〉} \AgdaFunction{⇑} \AgdaFunction{⇑} \AgdaFunction{⇑}\<%
\\
\>[0]\AgdaIndent{2}{}\<[2]%
\>[2]\AgdaFunction{∎}\<%
\\
%
\\
\>\AgdaKeyword{postulate} \AgdaPostulate{rep↑-upRep₄'} \AgdaSymbol{:} \AgdaSymbol{∀} \AgdaSymbol{\{}\AgdaBound{U}\AgdaSymbol{\}} \AgdaSymbol{\{}\AgdaBound{V}\AgdaSymbol{\}} \AgdaSymbol{(}\AgdaBound{ρ} \AgdaSymbol{:} \AgdaFunction{Rep} \AgdaBound{U} \AgdaBound{V}\AgdaSymbol{)} \AgdaSymbol{\{}\AgdaBound{K1}\AgdaSymbol{\}} \AgdaSymbol{\{}\AgdaBound{K2}\AgdaSymbol{\}} \AgdaSymbol{\{}\AgdaBound{K3}\AgdaSymbol{\}} \AgdaSymbol{→} \AgdaFunction{upRep} \AgdaFunction{•R} \AgdaFunction{upRep} \AgdaFunction{•R} \AgdaFunction{upRep} \AgdaFunction{•R} \AgdaBound{ρ} \AgdaFunction{∼R} \AgdaFunction{rep↑} \AgdaBound{K1} \AgdaSymbol{(}\AgdaFunction{rep↑} \AgdaBound{K2} \AgdaSymbol{(}\AgdaFunction{rep↑} \AgdaBound{K3} \AgdaBound{ρ}\AgdaSymbol{))} \AgdaFunction{•R} \AgdaFunction{upRep} \AgdaFunction{•R} \AgdaFunction{upRep} \AgdaFunction{•R} \AgdaFunction{upRep}\<%
\end{code}
}


%\begin{acknowledgements}
%If you'd like to thank anyone, place your comments here
%and remove the percent signs.
%\end{acknowledgements}

% BibTeX users please use one of
%\bibliographystyle{spbasic}      % basic style, author-year citations
\bibliographystyle{spmpsci}      % mathematics and physical sciences
%\bibliographystyle{spphys}       % APS-like style for physics
\bibliography{type}   % name your BibTeX data base

\end{document}

