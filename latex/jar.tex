%%%%%%%%%%%%%%%%%%%%%%% file template.tex %%%%%%%%%%%%%%%%%%%%%%%%%
%
% This is a general template file for the LaTeX package SVJour3
% for Springer journals.          Springer Heidelberg 2010/09/16
%
% Copy it to a new file with a new name and use it as the basis
% for your article. Delete % signs as needed.
%
% This template includes a few options for different layouts and
% content for various journals. Please consult a previous issue of
% your journal as needed.
%
%%%%%%%%%%%%%%%%%%%%%%%%%%%%%%%%%%%%%%%%%%%%%%%%%%%%%%%%%%%%%%%%%%%
%
% First comes an example EPS file -- just ignore it and
% proceed on the \documentclass line
% your LaTeX will extract the file if required
\begin{filecontents*}{example.eps}
%!PS-Adobe-3.0 EPSF-3.0
%%BoundingBox: 19 19 221 221
%%CreationDate: Mon Sep 29 1997
%%Creator: programmed by hand (JK)
%%EndComments
gsave
newpath
  20 20 moveto
  20 220 lineto
  220 220 lineto
  220 20 lineto
closepath
2 setlinewidth
gsave
  .4 setgray fill
grestore
stroke
grestore
\end{filecontents*}
%
\RequirePackage{fix-cm}
%
%\documentclass{svjour3}                     % onecolumn (standard format)
\documentclass[smallcondensed,draft]{svjour3}     % onecolumn (ditto)
%TODO Remove 'draft' before submitting
%\documentclass[smallextended]{svjour3}       % onecolumn (second format)
%\documentclass[twocolumn]{svjour3}          % twocolumn
%
\smartqed  % flush right qed marks, e.g. at end of proof
%
\usepackage{graphicx}
%
 \usepackage{mathptmx}      % use Times fonts if available on your TeX system
%
% insert here the call for the packages your document requires
%\usepackage{latexsym}
% etc.
%
% please place your own definitions here and don't use \def but
% \newcommand{}{}
%
% Insert the name of "your journal" with
% \journalname{myjournal}
\journalname{Journal of Automated Reasoning}
%
\begin{document}
\DeclareUnicodeCharacter{8598}{\ensuremath{\nwarrow}}
\DeclareUnicodeCharacter{8599}{\ensuremath{\nearrow}}
\DeclareUnicodeCharacter{8608}{\ensuremath{\twoheadrightarrow}}
\DeclareUnicodeCharacter{8657}{\ensuremath{\Uparrow}}
\DeclareUnicodeCharacter{8667}{\ensuremath{\Rrightarrow}}
\DeclareUnicodeCharacter{8718}{\ensuremath{\qed}}
\DeclareUnicodeCharacter{8759}{\ensuremath{::}}
\DeclareUnicodeCharacter{8988}{\ensuremath{\ulcorner}}
\DeclareUnicodeCharacter{8989}{\ensuremath{\urcorner}}
\DeclareUnicodeCharacter{8803}{\ensuremath{\overline{\equiv}}}
\DeclareUnicodeCharacter{9001}{\ensuremath{\langle}}
\DeclareUnicodeCharacter{9002}{\ensuremath{\rangle}}
\DeclareUnicodeCharacter{9655}{\ensuremath{\rhd}}
\DeclareUnicodeCharacter{9679}{\ensuremath{\bullet}}
\DeclareUnicodeCharacter{10214}{\ensuremath{[}}
\DeclareUnicodeCharacter{10215}{\ensuremath{]}}
\DeclareUnicodeCharacter{10219}{\ensuremath{\rangle\rangle}}
\DeclareUnicodeCharacter{10230}{\ensuremath{\longrightarrow}}

\newtheorem{prop}[theorem]{Proposition}
\newtheorem{lm}[theorem]{Lemma}
\newtheorem{cor}{Corollary}[theorem]

\newcommand*{\Set}{\mathbf{Set}}
\newcommand*{\eqdef}{\mathrel{\smash{\stackrel{\text{def}}{=}}}}
\newcommand*{\isotoid}{\ensuremath{isotoid}}
\newcommand*{\vald}{\ensuremath{\ \mathrm{valid}}}
\newcommand*{\reff}[1]{\ensuremath{\mathsf{ref} \left( {#1} \right)}}
\newcommand*{\univ}[4]{\ensuremath{\mathsf{univ}_{{#1} , {#2}} \left( {#3} , {#4} \right)}}
\newcommand*{\triplelambda}{\ensuremath{\lambda \!\! \lambda \!\! \lambda}}
\newcommand*{\SN}{\ensuremath{\mathbf{SN}}}
\newcommand*{\dom}{\ensuremath{\operatorname{dom}}}
\newcommand*{\sym}[4]{\ensuremath{\mathsf{sym}_{{#1},{#2},{#3}} \left( {#4} \right)}}
\WithSuffix\newcommand\sym*[1]{\ensuremath{\mathsf{sym} \left( {#1} \right)}}
\newcommand*{\trans}[6]{\ensuremath{\mathsf{trans}_{{#1},{#2},{#3},{#4}} \left( {#5} , {#6} \right)}}
\WithSuffix\newcommand\trans*[2]{\ensuremath{\mathsf{trans} \left( {#1} , {#2} \right)}}
\newcommand*{\kr}{\mathop{\rhd \!\!\! \rhd}}



\title{A Formalised Proof of Strong Normalisation for Higher-Order Minimal Logic with Univalence}
%TODO Insert details of CAU grant

%\subtitle{Do you have a subtitle?\\ If so, write it here}

\titlerunning{Higher-Order Minimal Logic with Univalence}        % if too long for running head

\author{Robin Adams \and Marc Bezem \and Thierry Coquand}

%\authorrunning{Short form of author list} % if too long for running head

\institute{Robin Adams \at
              Universitetet i Bergen \\
              Institutt for Informatikk \\
              Postboks 7800 \\
              N-5020 BERGEN \\
              Norway \\
              \email{robin.adams@uib.no}           %  \\
%             \emph{Present address:} of F. Author  %  if needed
           \and
           Marc Bezem \at
              Universitetet i Bergen \\
              Institutt for Informatikk \\
              Postboks 7800 \\
              N-5020 BERGEN \\
              Norway \\
              \email{bezem@ii.uib.no}           %  \\
           \and
           Thierry Coquand \at
              Chalmers tekniska h\"{o}gskola \\
              Data- och informationsteknik \\
              412 96 G\"{o}teborg \\
              Sweden \\
              \email{coquand@chalmers.se}
}

\date{Received: date / Accepted: date}
% The correct dates will be entered by the editor


\maketitle

\begin{abstract}
    Homotopy type theory offers the promise of a formal system for the univalent foundations of mathematics. However, if we simply add the univalence axiom to type theory, then we lose the property of canonicity --- that every term computes to a normal form. A computation becomes `stuck' when it reaches the point that it needs to evaluate a proof term that is an application of the univalence axiom. So we wish to find a way to compute with the univalence axiom.

    As a first step, we present here a system of higher-order propositional logic, with a universe Omega of propositions closed under implication and quantification over any simple type over Omega. We add a type $M =_A N$ for any terms $M$, $N$ of type $A$, and two ways to prove an equality: reflexivity, and the univalence axiom. We present reduction relations for this system, and prove the reduction confluent and strongly normalizing on the well-typed terms. 

This proof has been formalised in Agda.  The formalisation provides a general notion of grammar for a syntax with binding and reduction relation.
\end{abstract}

\section{Introduction}
\label{intro}
The rules of deduction of a type theory are traditionally justified by a \emph{meaning explanation} \cite{ML:ITT}, in which to know that a given term has a given type is to know that it computes to a \emph{canonical object} of that type.  For such a meaning explanation to be possible, the type theory should have the following properties:
\begin{itemize}
\item \textbf{Confluence} --- The reduction relation should be confluent.
\item \textbf{Normalization} --- Every well-typed term should reduce to a normal form.
\item Every closed normal form of type $A$ is a canonical object of type $A$.
\end{itemize}
From these three properties, we have:
\begin{itemize}
\item \textbf{Canonicity} --- Every term of type $A$ reduces to a unique canonical object of type $A$.
\end{itemize}

It is desirable to have, in addition, \emph{strong normalization}, so that we know that we are free to choose whatever reduction strategy we please.

The \emph{univalence axiom} of Homotopy Type theory (HoTT) \cite{hottbook} breaks the property of canonicity.  It postulates a
constant
\[ \isotoid : A \simeq B \rightarrow A = B \]
that is an inverse to the canonical function $A = B \rightarrow A \simeq B$.  When a computation reaches a point
where we eliminate a path (proof of equality) formed by $\isotoid$, it gets 'stuck'.

As possible solutions to this problem, we may try to do with a weaker property than canonicity, such as \emph{propositional canonicity}.
We may attempt to prove that every closed term of type $\mathbb{N}$ is \emph{propositionally} equal to a numeral, as conjectured by Voevodsky.  Or we may attempt to change the definition of equality to make $\isotoid$ definable\cite{Polonsky14a}, or
shift to an entirely different type theory such as Cubical Type Theory\cite{cchm:cubical}.

We could also try a more conservative approach, and simply attempt to find a reduction relation for a type theory involving $\isotoid$ that satisfies
all three of the properties above.  There seems to be no reason \emph{a priori} to believe this is not possible, but it is difficult to do because
the full Homotopy Type Theory is a complex and interdependent system.  We can tackle the problem by adding univalence to a much simpler system, finding
a well-behaved reduction relation, then doing the same for more and more complex systems, gradually approaching the full strength of HoTT.

In this paper, we present a system we call $\lambda o e$, or predicative higher-order minimal logic.  It is a type theory with two universes: the universe $\Omega$
of \emph{propositions}, and the universe of \emph{types}.  The propositions are closed under $\supset$ (implication) and include $\bot$ (falsehood), and an equality proposition $M =_A N$ for
any type $A$ and terms $M : A$ and $N : A$.  The types include $\Omega$ itself and are closed under $\rightarrow$ (non-dependent function type).

There are two canonical forms for proofs of $M =_\Omega N$.  For any term $M : \Omega$, we have $\reff{M} : M =_\Omega M$.  We also add univalence for this system, in this form:
if $\delta : \phi \supset \psi$ and $\epsilon : \psi \supset\phi$, then $\univ{\phi}{\psi}{\delta}{\epsilon} : \phi =_\Omega \psi$.  

We present a reduction relation for this system, and prove it satisfies confluence (Corollary \ref{cor:confluence}), strong normalization (Corollary \ref{cor:SN}) and canonicity (Corollary \ref{cor:canon}).

For the future, we wish to expand the system with universal quantification, and expand it to a 2-dimensional system (with equations between proofs).

%\begin{acknowledgements}
%If you'd like to thank anyone, place your comments here
%and remove the percent signs.
%\end{acknowledgements}

% BibTeX users please use one of
%\bibliographystyle{spbasic}      % basic style, author-year citations
\bibliographystyle{spmpsci}      % mathematics and physical sciences
%\bibliographystyle{spphys}       % APS-like style for physics
\bibliography{type}   % name your BibTeX data base

\end{document}

